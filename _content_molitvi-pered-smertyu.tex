\newcommand{\pominalnayaslava}[1][]{

\myparsep

\slavainynen 

Аллил\'{у}иа, аллил\'{у}иа, аллил\'{у}иа, сл\'{а}ва Теб\'{е}, Б\'{о}же. \myemph{(Трижды)}

Г\'{о}споди, пом\'{и}луй. \myemph{(Трижды)}

\slavan

Помян\'{и}, Г\'{о}споди, Б\'{о}же наш, в в\'{е}ре и над\'{е}жди живот\'{а} в\'{е}чнаго прест\'{а}вльшагося#1 раб\'{а} Твоег\'{о}, бр\'{а}та н\'{а}шего \myemph{(имярек)}, и \'{я}ко Благ и Человекол\'{ю}бец отпущ\'{а}яй грех\'{и} и потребл\'{я}яй непр\'{а}вды, осл\'{а}би, ост\'{а}ви и прост\'{и} вс\'{я} в\'{о}льная ег\'{о} согреш\'{е}ния и нев\'{о}льная, изб\'{а}ви ег\'{о} в\'{е}чная м\'{у}ки и огн\'{я} ге\'{е}нскаго и д\'{а}руй ем\'{у} прич\'{а}стие и наслажд\'{е}ние в\'{е}чных Тво\'{и}х благ\'{и}х, угот\'{о}ванных л\'{ю}бящим Тя: \'{а}ще бо и согреш\'{и}, но не отступ\'{и} от Теб\'{е}, и несумн\'{е}нно во Отц\'{а} и С\'{ы}на и Свят\'{а}го Д\'{у}ха, Б\'{о}га Тя в Тр\'{о}ице сл\'{а}вимаго в\'{е}рова, и Ед\'{и}ницу в Тр\'{о}ице и Тр\'{о}ицу во Ед\'{и}нстве, правосл\'{а}вно д\'{а}же до посл\'{е}дняго своег\'{о} издых\'{а}ния испов\'{е}да. Т\'{е}мже м\'{и}лостив том\'{у} б\'{у}ди, и в\'{е}ру, \'{я}же в Тя вм\'{е}сто дел вмен\'{и}, и со свят\'{ы}ми Тво\'{и}ми, \'{я}ко Щедр, упок\'{о}й: несть бо челов\'{е}ка, \'{и}же пожив\'{е}т и не согреш\'{и}т. Но Ты Ед\'{и}н ес\'{и} кром\'{е} вс\'{я}каго грех\'{а}, и пр\'{а}вда Тво\'{я}, пр\'{а}вда во в\'{е}ки, и Ты ес\'{и} Ед\'{и}н Бог м\'{и}лостей и щедр\'{о}т, и человекол\'{ю}бия, и Теб\'{е} сл\'{а}ву возсыл\'{а}ем, Отц\'{у} и С\'{ы}ну и Свят\'{о}му Д\'{у}ху, н\'{ы}не и пр\'{и}сно и во в\'{е}ки век\'{о}в. Ам\'{и}нь. 

\inynen

\myparsep} 

\mypart{НАПУТСТВИЕ ХРИСТИАНИНА ПЕРЕД СМЕРТЬЮ И ЗАУПОКОЙНЫЕ МОЛИТВЫ}\label{_content_molitvi-pered-smertyu}
%http://www.molitvoslov.org/content/molitvi-pered-smertyu



\mychapter{Кончина человека}\begin{mymulticols}
%http://www.molitvoslov.org/text182.htm 


Если положение больного безнадежно, то при явных признаках приближающейся смерти священник читает отходную молитву "--- «Канон молебный при разлучении души от тела» или более полно он называется «Канон молебный ко Господу нашему Иисусу Христу и Пречистой Богородице Матери Господни при разлучении души от тела всякаго правовернаго». Родственники сами могут прочитать этот канон, если невозможно пригласить священника, кроме чтения «молитвы, от иерея глаголемой на исход души», которая находится в конце канона. Этот канон читается «от лица человека с душею разлучающагося и не могущаго глаголати» и имеется в православных молитвословах. Чтение канона мирскими людьми начинается возгласом: «Молитвами святых отец наших Господи Иисусе Христе Боже наш, помилуй нас», затем следуют предначинательные молитвы: «Трисвятое», «Пресвятая Троице», «Отче наш» и далее по молитвослову. 

При чтении канона возжигается свеча и лампадка перед домашней святой иконой. Если дома иконы нет, то нужно обязательно приобрести в храме иконы Спасителя и Божией Матери. 

Для умирающих младенцев (детей до семи лет) из-за отсутствия грехов, перечисляемых в каноне, которые несвойственны им по малолетству, канон не читается. 

Кроме канона при разлучении души от тела еще существует «Чин, бываемый на разлучение души от тела, когда человек долго страждет». Этот чин читается над человеком, который испытывает тяжкие предсмертные мучения и никак не может умереть (как правило, читается священником). 

После смерти человека над ним немедленно читается «Последование по исходе души от тела». 

\end{mymulticols}

\mychapterending

\mychapter{Приготовление усопшего к погребению}\begin{mymulticols}
%http://www.molitvoslov.org/text237.htm 


Умерших мы именуем усопшими, то есть уснувшими. Называем их мы так по нашей христианской вере, что души после смерти не уничтожаются, не исчезают в небытии, а отлучаются от тела и переходят из этой жизни в другую "--- загробную. Там пребывают они после частного суда по делам земным в приличествующем им месте до Страшного суда Божия, когда по слову Господа души всех умерших людей воссоединяться с телами и воскреснут. И тогда окончательно определится участь каждого: праведные наследуют Царство Небесное, блаженную вечность с Богом, а грешные "--- вечное наказание. 

Историческое обоснование погребения усопших дано в образе погребения Иисуса Христа. По примеру благочестивой древности и в настоящее время погребение предваряется совершением различных многозначительных символических действий. 

Тело усопшего омывают теплой водой, чтобы предстал он перед Богом по воскресении в чистоте и непорочности. При омовении читают «Трисвятое»: «Святый Боже, Святый Крепкий, Святый Бессмертный, помилуй нас» или «Господи, помилуй». Возжигается лампадка или свеча и горит до тех пор, пока в доме находится покойник. После омовения тело христианина одевают в чистые и, по возможности, новые одежды "--- по званию и служению его, на усопшем обязательно должен быть нательный крестик. Омовение обычно совершают люди старшего возраста, а если таковых нет, то омыть тело усопшего может любой из родственников, за исключением женщин, находящихся в данный момент в естественной нечистоте. Обычай указывает, что в омовении тела женщины участвуют только женщины. Если известно, что почивший был монахом (монахиней) или священнослужителем, то надо о его кончине сообщить в храм. Тело покойного полагают на стол, покрывают белым покрывалом "--- саваном. Затем умершего покрывают особым освященным покровом (погребальным покрывалом), на котором изображены крест, лики святых и молитвенные надписи. Все это означает, что умерший оставался верным Богу и теперь остается под покровом Божиим. Глаза должны быть закрыты, уста сомкнуты, руки сложены крестообразно, правая поверх левой. Руки и ноги усопшего связывают, чтобы развязать перед последним прощанием. В руки покойного полагают погребальный крест, на грудь кладут святую икону, для мужчин "--- образ Спасителя, для женщин "--- образ Божией Матери. На лоб умершего возлагают венчик "--- полоску бумаги с изображением Спасителя, Божией Матери и Иоанна Предтечи. Изображения эти обрамлены надписью «Трисвятого». Венчик, символизирующий соблюдение веры усопшим христианином и совершение им христианского жизненного подвига, возлагается в надежде, что почивший в вере получит по воскресении от Бога небесную награду и нетленный венец. Как правило, венчик отпечатан на одном листке с разрешительной молитвой. После приобретения молитвы-венчика в храме венчик отрезается ножницами (после отпевания листок с молитвой будет вложен в руку почившего). Перед положением усопшего во гроб тело его и гроб окропляют святой водой, причем гроб окропляют извне и изнутри. В гроб покойного полагают лицом вверх, под голову кладут подушку, набитую соломой или опилками. Гроб обыкновенно ставят посреди комнаты перед домашними иконами, головой к образам. Вокруг гроба возжигают четыре свечи: у головы, в ногах и по обеим сторонам на уровне скрещенных рук. Возженные свечи вместе изображают крест и символизируют переход умершего в Царство Истинного Света. 

К православным традициям не относятся бытующие во многих семьях различные суеверия, связанные с покойником, такие, как занавешивание зеркал, отказ пользоваться вилками во время поминальной трапезы, обычай оставлять часть блюд или стакан с водой (а еще хуже с водкой) перед портретом усопшего и т. п. 

Все эти суеверия никакого отношения к православию не имеют. Занавешивание зеркал в доме, где лежит тело покойного, оправдано лишь в том случае, когда мы, думая об усопшем, уходим от внешней суеты и отдаём свой последний молитвенный вздох об упокоении преставльшейся души.

\end{mymulticols}

\mychapterending

\mychapter{17-я кафизма (поминальная), чтомая в дни особого поминовения усопших}
%http://www.molitvoslov.org/text227.htm 

{\centering \myemph{(Читается ежедневно в течение 40 дней по смерти)}\par}

\begin{mymulticols}

\section{Молитвы перед началом чтения Псалтири}

\MolitvamiSviatyhOtecNashih 

Сл\'{а}ва Теб\'{е}, Б\'{о}же наш, сл\'{а}ва Теб\'{е}.

\TsariuNebesnyj

\TrisviatoePoOtcheNash

\mysubtitle{Тропарь:}

Пом\'{и}луй нас, Г\'{о}споди, пом\'{и}луй нас; вс\'{я}каго бо отв\'{е}та недоум\'{е}юще, си\'{ю} Ти мол\'{и}тву \'{я}ко Влад\'{ы}це гр\'{е}шнии прин\'{о}сим: пом\'{и}луй нас. 

\slavan

Честн\'{о}е прор\'{о}ка Твоег\'{о}, Г\'{о}споди, торжеств\'{о}, н\'{е}бо Ц\'{е}рковь показ\'{а}, с челов\'{е}ки лик\'{у}ют \'{а}нгели: тог\'{о} мол\'{и}твами, Христ\'{е} Б\'{о}же, в м\'{и}ре упр\'{а}ви жив\'{о}т наш, да по\'{е}м Ти: Аллил\'{у}иа. 

\inynen

Мн\'{о}гая мн\'{о}жества мо\'{и}х, Бо\-го\-р\'{о}\-ди\-це, прегреш\'{е}ний, к Теб\'{е} прибег\'{о}х, Ч\'{и}стая, спас\'{е}ния тр\'{е}буя: посет\'{и} немощств\'{у}ющую мо\'{ю} д\'{у}шу, и мол\'{и} С\'{ы}на Твоег\'{о} и Б\'{о}га н\'{а}шего дать ми оставл\'{е}ние, \'{я}же сод\'{е}ях л\'{ю}тых, Ед\'{и}на Благослов\'{е}нная. 

Г\'{о}споди, пом\'{и}луй \myemph{(40 раз). И поклоны по силе.}

\mysubtitle{Молитва Святей Живоначальней Троице}

Всесвят\'{а}я Тр\'{о}ице, Б\'{о}же и Сод\'{е}телю всег\'{о} м\'{и}ра, поспеш\'{и} и напр\'{а}ви с\'{е}рдце мо\'{е}, нач\'{а}ти с р\'{а}зумом и конч\'{а}ти д\'{е}лы благ\'{и}ми богодухнов\'{е}нныя си\'{я} кн\'{и}ги, \'{я}же Свят\'{ы}й Дух уст\'{ы} Дав\'{и}довы отр\'{ы}гну, \'{и}хже н\'{ы}не хощ\'{у} глаг\'{о}лати аз, недост\'{о}йный, разум\'{е}я же сво\'{е} нев\'{е}жество, прип\'{а}дая мол\'{ю}ся Ти, и \'{е}же от Теб\'{е} п\'{о}мощи прос\'{я}: Г\'{о}споди, упр\'{а}ви ум мой и утверд\'{и} с\'{е}рдце мо\'{е}, не о глаг\'{о}лании уст\'{е}н стуж\'{а}ти си, но о р\'{а}зуме глаг\'{о}лемых весел\'{и}тися, и пригот\'{о}витися на твор\'{е}ние д\'{о}брых дел, \'{я}же уч\'{у}ся, и глаг\'{о}лю: да д\'{о}брыми д\'{е}лы просвещ\'{е}н, на суд\'{и}щи десн\'{ы}я Ти стран\'{ы} прич\'{а}стник б\'{у}ду со вс\'{е}ми избр\'{а}нными Тво\'{и}ми. И н\'{ы}не, Влад\'{ы}ко, благослов\'{и}, да, воздохн\'{у}в от с\'{е}рдца, и яз\'{ы}ком воспо\'{ю}, глаг\'{о}ля с\'{и}це:

\priiditepoklonimsia

\section{Псалом 118}

Блаж\'{е}ни непор\'{о}чнии в путь, ход\'{я}щии в зак\'{о}не Госп\'{о}дни. Блаж\'{е}ни испыт\'{а}ющии свид\'{е}ния Ег\'{о}, всем с\'{е}рдцем вз\'{ы}щут Ег\'{о}, не д\'{е}лающии бо беззак\'{о}ния, в пут\'{е}х Ег\'{о} ход\'{и}ша. Ты запов\'{е}дал ес\'{и} з\'{а}поведи Тво\'{я} сохран\'{и}ти зел\'{о}. Даб\'{ы} испр\'{а}вилися пути\'{е} мо\'{и}, сохран\'{и}ти оправд\'{а}ния Тво\'{я}. Тогд\'{а} не постыж\'{у}ся, внегд\'{а} призр\'{е}ти ми на вся з\'{а}поведи Тво\'{я}. Испов\'{е}мся Теб\'{е} в пр\'{а}вости с\'{е}рдца, внегд\'{а} науч\'{и}ти ми ся судьб\'{а}м пр\'{а}вды Твое\'{я}. Оправд\'{а}ния Тво\'{я} сохран\'{ю}, не ост\'{а}ви мен\'{е} до зел\'{а}. В чес\'{о}м испр\'{а}вит юн\'{е}йший путь свой; внегд\'{а} сохран\'{и}ти словес\'{а} Тво\'{я}. Всем с\'{е}рдцем мо\'{и}м взыск\'{а}х Теб\'{е}, не отр\'{и}ни мен\'{е} от з\'{а}поведей Тво\'{и}х. В с\'{е}рдце мо\'{е}м скрых словес\'{а} Тво\'{я}, \'{я}ко да не согреш\'{у} Теб\'{е}. Благослов\'{е}н ес\'{и}, Г\'{о}споди: науч\'{и} мя оправд\'{а}нием Тво\'{и}м. Устн\'{а}ма мо\'{и}ма возвест\'{и}х вся судьб\'{ы} уст Тво\'{и}х. На пут\'{и} свид\'{е}ний Тво\'{и}х наслад\'{и}хся, \'{я}ко о вс\'{я}ком бог\'{а}тстве. В з\'{а}поведех Тво\'{и}х поглумл\'{ю}ся, и уразум\'{е}ю пут\'{и} Тво\'{я}. Во оправд\'{а}ниих Тво\'{и}х поуч\'{у}ся, не заб\'{у}ду слов\'{е}с Тво\'{и}х. Возд\'{а}ждь раб\'{у} Твоем\'{у}: жив\'{и} мя, и сохран\'{ю} словес\'{а} Тво\'{я}. Откр\'{ы}й \'{о}чи мо\'{и}, и уразум\'{е}ю чудес\'{а} от зак\'{о}на Твоег\'{о}. Пришл\'{е}ц аз есмь на земл\'{и}: не скрый от мен\'{е} з\'{а}поведи Тво\'{я}. Возлюб\'{и} душ\'{а} мо\'{я} возжел\'{а}ти судьб\'{ы} Тво\'{я} на вс\'{я}кое вр\'{е}мя. Запрет\'{и}л ес\'{и} г\'{о}рдым: пр\'{о}кляти уклон\'{я}ющиися от з\'{а}поведей Тво\'{и}х. Отьим\'{и} от мен\'{е} пон\'{о}с и уничиж\'{е}ние, \'{я}ко свид\'{е}ний Тво\'{и}х взыск\'{а}х. \'{И}бо сед\'{о}ша кн\'{я}зи, и на мя клевет\'{а}ху, раб же Твой глумл\'{я}шеся во оправд\'{а}ниих Тво\'{и}х: \'{И}бо свид\'{е}ния Тво\'{я} поуч\'{е}ние мо\'{е} есть, и сов\'{е}ти мо\'{и} оправд\'{а}ния Тво\'{я}. Прильп\'{е} земл\'{и} душ\'{а} мо\'{я}: жив\'{и} мя по словес\'{и} Твоем\'{у}. Пут\'{и} мо\'{я} возвест\'{и}х, и усл\'{ы}шал мя ес\'{и}: науч\'{и} мя оправд\'{а}нием Тво\'{и}м: Путь оправд\'{а}ний Тво\'{и}х вразум\'{и} ми, и поглумл\'{ю}ся в чудес\'{е}х Тво\'{и}х. Воздрем\'{а} душ\'{а} мо\'{я} от ун\'{ы}ния: утверд\'{и} мя в словес\'{е}х Тво\'{и}х. Путь непр\'{а}вды отст\'{а}ви от мен\'{е}, и зак\'{о}ном Тво\'{и}м пом\'{и}луй мя. Путь \'{и}стины изв\'{о}лих, и судьб\'{ы} Тво\'{я} не заб\'{ы}х. Прилеп\'{и}хся свид\'{е}нием Тво\'{и}м, Г\'{о}споди, не посрам\'{и} мен\'{е}. Путь з\'{а}поведей Тво\'{и}х тек\'{о}х, егд\'{а} расшир\'{и}л ес\'{и} с\'{е}рдце мо\'{е}. Законополож\'{и} мне, Г\'{о}споди, путь оправд\'{а}ний Тво\'{и}х, и взыщ\'{у} \'{и} в\'{ы}ну: Вразум\'{и} мя, и испыт\'{а}ю зак\'{о}н Твой, и сохран\'{ю} \'{и} всем с\'{е}рдцем мо\'{и}м. Наст\'{а}ви мя на стез\'{ю} з\'{а}поведей Тво\'{и}х, \'{я}ко т\'{у}ю восхот\'{е}х. Приклон\'{и} с\'{е}рдце мо\'{е} во свид\'{е}ния Тво\'{я}, а не в лихо\'{и}мство. Отврат\'{и} \'{о}чи мо\'{и}, \'{е}же не в\'{и}дети сует\'{ы}, в пут\'{и} Тво\'{е}м жив\'{и} мя. Пост\'{а}ви раб\'{у} Твоем\'{у} сл\'{о}во Тво\'{е} в страх Твой. Отьим\'{и} понош\'{е}ние мо\'{е}, \'{е}же непщев\'{а}х: \'{я}ко судьб\'{ы} Тво\'{я} бл\'{а}ги. Се возжел\'{а}х з\'{а}поведи Тво\'{я}, в пр\'{а}вде Тво\'{е}й жив\'{и} мя. И да при\'{и}дет на мя м\'{и}лость Тво\'{я}, Г\'{о}споди, спас\'{е}ние Тво\'{е} по словес\'{и} Твоем\'{у}. И отвещ\'{а}ю понош\'{а}ющым ми сл\'{о}во: \'{я}ко упов\'{а}х на словес\'{а} Тво\'{я}. И не отьим\'{и} от уст мо\'{и}х словес\'{е} \'{и}стинна до зел\'{а}, \'{я}ко на судьб\'{ы} Тво\'{я} упов\'{а}х. И сохран\'{ю} зак\'{о}н Твой в\'{ы}ну, в век и в век в\'{е}ка. И хожд\'{а}х в широт\'{е}, \'{я}ко з\'{а}поведи Тво\'{я} взыск\'{а}х. И глаг\'{о}лах о свид\'{е}ниих Тво\'{и}х пред цар\'{и}, и не стыд\'{я}хся: И поуч\'{а}хся в з\'{а}поведех Тво\'{и}х, \'{я}же возлюб\'{и}х зел\'{о}: И воздвиг\'{о}х р\'{у}це мо\'{и} к з\'{а}поведем Тво\'{и}м, \'{я}же возлюб\'{и}х, и глумл\'{я}хся во оправд\'{а}ниих Тво\'{и}х. Помян\'{и} словес\'{а} Тво\'{я} раб\'{у} Твоем\'{у}, \'{и}хже упов\'{а}ние дал ми ес\'{и}. То мя ут\'{е}ши во смир\'{е}нии мо\'{е}м, \'{я}ко сл\'{о}во Тво\'{е} жив\'{и} мя. Г\'{о}рдии законопреступов\'{а}ху до зел\'{а}: от зак\'{о}на же Твоег\'{о} не уклон\'{и}хся. Помян\'{у}х судьб\'{ы} Тво\'{я} от в\'{е}ка, Г\'{о}споди, и ут\'{е}шихся. Печ\'{а}ль при\'{я}т мя от гр\'{е}шник, оставл\'{я}ющих зак\'{о}н Твой. П\'{е}та б\'{я}ху мне оправд\'{а}ния Тво\'{я}, на м\'{е}сте приш\'{е}льствия моег\'{о}. Помян\'{у}х в нощ\'{и} \'{И}мя Тво\'{е}, Г\'{о}споди, и сохран\'{и}х зак\'{о}н Твой. Сей бысть мне, \'{я}ко оправд\'{а}ний Тво\'{и}х взыск\'{а}х. Часть мо\'{я} ес\'{и}, Г\'{о}споди, рех сохран\'{и}ти зак\'{о}н Твой. Помол\'{и}хся лиц\'{у} Твоем\'{у} всем с\'{е}рдцем мо\'{и}м: пом\'{и}луй мя по словес\'{и} Твоем\'{у}. Пом\'{ы}слих пут\'{и} Тво\'{я}, и возврат\'{и}х н\'{о}зе мо\'{и} во свид\'{е}ния Тво\'{я}. Угот\'{о}вихся и не смут\'{и}хся сохран\'{и}ти з\'{а}поведи Тво\'{я}. \'{У}жя гр\'{е}шник обяз\'{а}шася мне, и зак\'{о}на Твоег\'{о} не заб\'{ы}х. Пол\'{у}нощи вост\'{а}х испов\'{е}датися Теб\'{е} о судьб\'{а}х пр\'{а}вды Твое\'{я}. Прич\'{а}стник аз есмь всем бо\'{я}щымся Теб\'{е}, и хран\'{я}щым з\'{а}поведи Тво\'{я}. М\'{и}лости Твое\'{я}, Г\'{о}споди, исп\'{о}лнь земл\'{я}: оправд\'{а}нием Тво\'{и}м науч\'{и} мя. Бл\'{а}гость сотвор\'{и}л ес\'{и} с раб\'{о}м Тво\'{и}м, Г\'{о}споди, по словес\'{и} Твоем\'{у}. Бл\'{а}гости, и наказ\'{а}нию и р\'{а}зуму науч\'{и} мя, \'{я}ко з\'{а}поведем Тво\'{и}м в\'{е}ровах. Пр\'{е}жде д\'{а}же не смир\'{и}ти ми ся, аз прегреш\'{и}х: сег\'{о} р\'{а}ди сл\'{о}во Тво\'{е} сохран\'{и}х. Благ ес\'{и} Ты, Г\'{о}споди, и бл\'{а}гостию Тво\'{е}ю науч\'{и} мя оправд\'{а}нием Тво\'{и}м. Умн\'{о}жися на мя непр\'{а}вда г\'{о}рдых, аз же всем с\'{е}рдцем мо\'{и}м испыт\'{а}ю з\'{а}поведи Тво\'{я}. Усыр\'{и}ся \'{я}ко млек\'{о} с\'{е}рдце их, аз же зак\'{о}ну Твоем\'{у} поуч\'{и}хся. Бл\'{а}го мне, \'{я}ко смир\'{и}л мя ес\'{и}, \'{я}ко да науч\'{у}ся оправд\'{а}нием Тво\'{и}м. Благ мне зак\'{о}н уст Тво\'{и}х, п\'{а}че т\'{ы}сящ зл\'{а}та и сребр\'{а}.

\pominalnayaslava[\footnote{До 40-го дня после смерти полагается читать «новопрест\'{а}вленного», в дальнейшем "--- «прест\'{а}вльшагося».}]

Р\'{у}це Тво\'{и} сотвор\'{и}сте мя, и созд\'{а}сте мя: вразум\'{и} мя, и науч\'{у}ся з\'{а}поведем Тво\'{и}м. Бо\'{я}щиися Теб\'{е} \'{у}зрят мя и возвесел\'{я}тся, \'{я}ко на словес\'{а} Тво\'{я} упов\'{а}х. Разум\'{е}х Г\'{о}споди, \'{я}ко пр\'{а}вда судьб\'{ы} Тво\'{я}, и во\'{и}стинну смир\'{и}л мя ес\'{и}. Б\'{у}ди же м\'{и}лость Тво\'{я}, да ут\'{е}шит мя по словес\'{и} Твоем\'{у} раб\'{у} Твоем\'{у}. Да при\'{и}дут мне щедр\'{о}ты Тво\'{я}, и жив б\'{у}ду, \'{я}ко зак\'{о}н Твой поуч\'{е}ние мо\'{е} есть. Да постыд\'{я}тся г\'{о}рдии, \'{я}ко непр\'{а}ведно беззак\'{о}нноваша на мя, аз же поглумл\'{ю}ся в з\'{а}поведех Тво\'{и}х. Да обрат\'{я}т мя бо\'{я}щиися Теб\'{е}, и в\'{е}дящии свид\'{е}ния Тво\'{я}. Б\'{у}ди с\'{е}рдце мо\'{е} непор\'{о}чно во оправд\'{а}ниих Тво\'{и}х, \'{я}ко да не постыж\'{у}ся. Исчез\'{а}ет во спас\'{е}ние Тво\'{е} душ\'{а} мо\'{я}, на словес\'{а} Тво\'{я} упов\'{а}х. Исчез\'{о}ша \'{о}чи мо\'{и} в сл\'{о}во Тво\'{е}, глаг\'{о}люще: когд\'{а} ут\'{е}шиши мя; Зан\'{е} бых \'{я}ко мех на сл\'{а}не: оправд\'{а}ний Тво\'{и}х не заб\'{ы}х. Кол\'{и}ко есть дней раб\'{а} Твоег\'{о}; когд\'{а} сотвор\'{и}ши ми от гон\'{я}щих мя суд; Пов\'{е}даша мне законопрест\'{у}пницы глумл\'{е}ния, но не \'{я}ко зак\'{о}н Твой, Г\'{о}споди. Вся з\'{а}поведи Тво\'{я} \'{и}стина: непр\'{а}ведно погн\'{а}ша мя, помоз\'{и} ми. Вм\'{а}ле не сконч\'{а}ша мен\'{е} на земл\'{и}: аз же не ост\'{а}вих з\'{а}поведей Тво\'{и}х. По м\'{и}лости Тво\'{е}й жив\'{и} мя, и сохран\'{ю} свид\'{е}ния уст Тво\'{и}х. Во век, Г\'{о}споди, сл\'{о}во Тво\'{е} пребыв\'{а}ет на Небес\'{и}. В род и род \'{и}стина Тво\'{я}. Основ\'{а}л ес\'{и} з\'{е}млю, и пребыв\'{а}ет. Учин\'{е}нием Тво\'{и}м пребыв\'{а}ет день, \'{я}ко вс\'{я}ческая раб\'{о}тна Теб\'{е}. \'{Я}ко \'{а}ще бы не зак\'{о}н Твой поуч\'{е}ние мо\'{е} был, тогд\'{а} \'{у}бо пог\'{и}бл бых во смир\'{е}нии мо\'{е}м. Во век не заб\'{у}ду оправд\'{а}ний Тво\'{и}х, \'{я}ко в них ожив\'{и}л мя ес\'{и}. Твой есмь аз, спас\'{и} мя, \'{я}ко оправд\'{а}ний Тво\'{и}х взыск\'{а}х. Мен\'{е} жд\'{а}ша гр\'{е}шницы погуб\'{и}ти мя, свид\'{е}ния Тво\'{я} разум\'{е}х. Вс\'{я}кия конч\'{и}ны в\'{и}дех кон\'{е}ц, широк\'{а} з\'{а}поведь Тво\'{я} зел\'{о}. Коль возлюб\'{и}х зак\'{о}н Твой, Г\'{о}споди, весь день поуч\'{е}ние мо\'{е} есть. П\'{а}че враг мо\'{и}х умудр\'{и}л мя ес\'{и} з\'{а}поведию Тво\'{е}ю, \'{я}ко в век мо\'{я} есть. П\'{а}че всех уч\'{а}щих мя разум\'{е}х, \'{я}ко свид\'{е}ния Тво\'{я} поуч\'{е}ние мо\'{е} есть. П\'{а}че ст\'{а}рец разум\'{е}х, \'{я}ко з\'{а}поведи Тво\'{я} взыск\'{а}х. От вс\'{я}каго пут\'{и} лук\'{а}ва возбран\'{и}х ног\'{а}м мо\'{и}м, \'{я}ко да сохран\'{ю} словес\'{а} Тво\'{я}. От суд\'{е}б Тво\'{и}х не уклон\'{и}хся, \'{я}ко Ты законополож\'{и}л ми ес\'{и}. Коль сладк\'{а} горт\'{а}ни моем\'{у} словес\'{а} Тво\'{я}, п\'{а}че м\'{е}да уст\'{о}м мо\'{и}м. От з\'{а}поведей Тво\'{и}х разум\'{е}х, сег\'{о} р\'{а}ди возненав\'{и}дех всяк путь непр\'{а}вды. Свет\'{и}льник ног\'{а}ма мо\'{и}ма зак\'{о}н Твой, и свет стез\'{я}м мо\'{и}м. Кл\'{я}хся и пост\'{а}вих сохран\'{и}ти судьб\'{ы} пр\'{а}вды Твое\'{я}. Смир\'{и}хся до зел\'{а}, Г\'{о}споди, жив\'{и} мя по словес\'{и} Твоем\'{у}. В\'{о}льная уст мо\'{и}х благовол\'{и} же, Г\'{о}споди, и судьб\'{а}м Тво\'{и}м науч\'{и} мя. Душ\'{а} мо\'{я} в рук\'{у} Тво\'{е}ю в\'{ы}ну, и зак\'{о}на Твоег\'{о} не заб\'{ы}х. Полож\'{и}ша гр\'{е}шницы сеть мне, и от з\'{а}поведей Тво\'{и}х не заблуд\'{и}х. Насл\'{е}довах свид\'{е}ния Тво\'{я} во век, \'{я}ко р\'{а}дование с\'{е}рдца моег\'{о} суть. Приклон\'{и}х с\'{е}рдце мо\'{е} сотвор\'{и}ти оправд\'{а}ния Тво\'{я} в век за возда\'{я}ние. Законопрест\'{у}пныя возненав\'{и}дех, зак\'{о}н же Твой возлюб\'{и}х. Пом\'{о}щник мой и Заст\'{у}пник мой ес\'{и} Ты, на словес\'{а} Тво\'{я} упов\'{а}х. Уклон\'{и}теся от мен\'{е} лук\'{а}внующии, и испыт\'{а}ю з\'{а}поведи Б\'{о}га моег\'{о}. Заступ\'{и} мя по словес\'{и} Твоем\'{у}, и жив б\'{у}ду, и не посрам\'{и} мен\'{е} от ч\'{а}яния моег\'{о}. Помоз\'{и} ми, и спас\'{у}ся, и поуч\'{у}ся во оправд\'{а}ниих Тво\'{и}х в\'{ы}ну. Уничиж\'{и}л ес\'{и} вся отступ\'{а}ющия от оправд\'{а}ний Тво\'{и}х, \'{я}ко непр\'{а}ведно помышл\'{е}ние их. Преступ\'{а}ющия непщев\'{а}х вся гр\'{е}шныя земл\'{и}, сег\'{о} р\'{а}ди возлюб\'{и}х свид\'{е}ния Тво\'{я}. Пригвозд\'{и} стр\'{а}ху Твоем\'{у} пл\'{о}ти мо\'{я}, от суд\'{е}б бо Тво\'{и}х убо\'{я}хся. Сотвор\'{и}х суд и пр\'{а}вду, не пред\'{а}ждь мен\'{е} об\'{и}дящым мя. Восприим\'{и} раб\'{а} Твоег\'{о} во бл\'{а}го, да не оклевет\'{а}ют мен\'{е} г\'{о}рдии. \'{О}чи мо\'{и} исчез\'{о}сте во спас\'{е}ние Тво\'{е}, и в сл\'{о}во пр\'{а}вды Твое\'{я}. Сотвор\'{и} с раб\'{о}м Тво\'{и}м по м\'{и}лости Тво\'{е}й, и оправд\'{а}нием Тво\'{и}м науч\'{и} мя. Раб Твой есмь аз: вразум\'{и} мя, и ув\'{е}м свид\'{е}ния Тво\'{я}. Вр\'{е}мя сотвор\'{и}ти Г\'{о}сподеви: разор\'{и}ша зак\'{о}н Твой. Сег\'{о} р\'{а}ди возлюб\'{и}х з\'{а}поведи Тво\'{я} п\'{а}че зл\'{а}та и топ\'{а}зия. Сег\'{о} р\'{а}ди ко всем з\'{а}поведем Тво\'{и}м направл\'{я}хся, всяк путь непр\'{а}вды возненав\'{и}дех. Д\'{и}вна свид\'{е}ния Тво\'{я}, сег\'{о} р\'{а}ди испыт\'{а} \'{я} душ\'{а} мо\'{я}. Явл\'{е}ние слов\'{е}с Тво\'{и}х просвещ\'{а}ет и вразумл\'{я}ет млад\'{е}нцы. Уст\'{а} мо\'{я} отверз\'{о}х, и привлек\'{о}х дух, \'{я}ко з\'{а}поведей Тво\'{и}х жел\'{а}х.

\pominalnayaslava

Пр\'{и}зри на мя и пом\'{и}луй мя, по суд\'{у} л\'{ю}бящих \'{и}мя Тво\'{е}. Стоп\'{ы} мо\'{я} напр\'{а}ви по словес\'{и} Твоем\'{у}, и да не облад\'{а}ет мн\'{о}ю вс\'{я}кое беззак\'{о}ние. Изб\'{а}ви мя от клевет\'{ы} челов\'{е}ческия, и сохран\'{ю} з\'{а}поведи Тво\'{я}. Лиц\'{е} Тво\'{е} просвет\'{и} на раб\'{а} Твоег\'{о}, и науч\'{и} мя оправд\'{а}нием Тво\'{и}м. Исх\'{о}дища водн\'{а}я извед\'{о}сте \'{о}чи мо\'{и}, пон\'{е}же не сохран\'{и}х зак\'{о}на Твоег\'{о}. Пр\'{а}веден ес\'{и}, Г\'{о}споди, и пр\'{а}ви суд\'{и} Тво\'{и}. Запов\'{е}дал ес\'{и} пр\'{а}вду свид\'{е}ния Тво\'{я}, и \'{и}стину зел\'{о}. Ист\'{а}яла мя есть р\'{е}вность Тво\'{я}, \'{я}ко заб\'{ы}ша словес\'{а} Тво\'{я} враз\'{и} мо\'{и}. Разжж\'{е}но сл\'{о}во Тво\'{е} зел\'{о}, и раб Твой возлюб\'{и} \'{е}. Юн\'{е}йший аз есмь и уничиж\'{е}н, оправд\'{а}ний Тво\'{и}х не заб\'{ы}х. Пр\'{а}вда Тво\'{я} пр\'{а}вда во век, и зак\'{о}н Твой \'{и}стина. Ск\'{о}рби и н\'{у}жды обрет\'{о}ша мя, з\'{а}поведи Тво\'{я} поуч\'{е}ние мо\'{е}. Пр\'{а}вда свид\'{е}ния Тво\'{я} в век, вразум\'{и} мя, и жив б\'{у}ду. Воззв\'{а}х всем с\'{е}рдцем мо\'{и}м, усл\'{ы}ши мя, Г\'{о}споди, оправд\'{а}ния Тво\'{я} взыщ\'{у}. Воззв\'{а}х Ти, спас\'{и} мя, и сохран\'{ю} свид\'{е}ния Тво\'{я}. Предвар\'{и}х в безг\'{о}дии и воззв\'{а}х, на словес\'{а} Тво\'{я} упов\'{а}х. Предвар\'{и}сте \'{о}чи мо\'{и} ко \'{у}тру, поуч\'{и}тися словес\'{е}м Тво\'{и}м. Глас мой усл\'{ы}ши, Г\'{о}споди, по м\'{и}лости Тво\'{е}й: по судьб\'{е} Тво\'{е}й жив\'{и} мя. Прибл\'{и}жишася гон\'{я}щии мя беззак\'{о}нием, от зак\'{о}на же Твоег\'{о} удал\'{и}шася. Близ ес\'{и} Ты, Г\'{о}споди, и вси пути\'{е} Тво\'{и} \'{и}стина. Исп\'{е}рва позн\'{а}х от свид\'{е}ний Тво\'{и}х, \'{я}ко в век основ\'{а}л \'{я} ес\'{и}. Виждь смир\'{е}ние мо\'{е} и изм\'{и} мя, \'{я}ко зак\'{о}на Твоег\'{о} не заб\'{ы}х. Суд\'{и} суд мой и изб\'{а}ви мя, словес\'{е} р\'{а}ди Твоег\'{о} жив\'{и} мя. Дал\'{е}че от гр\'{е}шник спас\'{е}ние, \'{я}ко оправд\'{а}ний Тво\'{и}х не взыск\'{а}ша. Щедр\'{о}ты Тво\'{я} мн\'{о}ги, Г\'{о}споди, по судьб\'{е} Тво\'{е}й жив\'{и} мя. Мн\'{о}зи изгон\'{я}щии мя и стуж\'{а}ющии ми, от свид\'{е}ний Тво\'{и}х не уклон\'{и}хся. В\'{и}дех неразумев\'{а}ющия и ист\'{а}ях, \'{я}ко слов\'{е}с Тво\'{и}х не сохран\'{и}ша. Виждь, \'{я}ко з\'{а}поведи Тво\'{я} возлюб\'{и}х, Г\'{о}споди, по м\'{и}лости Тво\'{е}й жив\'{и} мя. Нач\'{а}ло слов\'{е}с Тво\'{и}х \'{и}стина, и во век вся судьб\'{ы} пр\'{а}вды Твое\'{я}. Кн\'{я}зи погн\'{а}ша мя т\'{у}не, и от слов\'{е}с Тво\'{и}х убо\'{я}ся с\'{е}рдце мо\'{е}. Возр\'{а}дуюся аз о словес\'{е}х Тво\'{и}х, \'{я}ко обрет\'{а}яй кор\'{ы}сть мн\'{о}гу. Непр\'{а}вду возненав\'{и}дех и омерз\'{и}х, зак\'{о}н же Твой возлюб\'{и}х. Седмер\'{и}цею днем хвал\'{и}х Тя о судьб\'{а}х пр\'{а}вды Твое\'{я}. Мир мног л\'{ю}бящым зак\'{о}н Твой, и несть им собл\'{а}зна. Ч\'{а}ях спас\'{е}ния Твоег\'{о}, Г\'{о}споди, и з\'{а}поведи Тво\'{я} возлюб\'{и}х. Сохран\'{и} душ\'{а} мо\'{я} свид\'{е}ния Тво\'{я} и возлюб\'{и} \'{я} зел\'{о}. Сохран\'{и}х з\'{а}поведи Тво\'{я} и свид\'{е}ния Тво\'{я}, \'{я}ко вси пути\'{е} мо\'{и} пред Тоб\'{о}ю, Г\'{о}споди. Да прибл\'{и}жится мол\'{е}ние мо\'{е} пред Тя, Г\'{о}споди, по словес\'{и} Твоем\'{у} вразум\'{и} мя. Да вн\'{и}дет прош\'{е}ние мо\'{е} пред Тя, Г\'{о}споди, по словес\'{и} Твоем\'{у} изб\'{а}ви мя. Отр\'{ы}гнут устн\'{е} мо\'{и} п\'{е}ние, егд\'{а} науч\'{и}ши мя оправд\'{а}нием Тво\'{и}м. Провещ\'{а}ет яз\'{ы}к мой словес\'{а} Тво\'{я}, \'{я}ко вся з\'{а}поведи Тво\'{я} пр\'{а}вда. Да б\'{у}дет рук\'{а} Тво\'{я} \'{е}же спаст\'{и} мя, \'{я}ко з\'{а}поведи Тво\'{я} изв\'{о}лих. Возжел\'{а}х спас\'{е}ние Тво\'{е}, Г\'{о}споди, и зак\'{о}н Твой поуч\'{е}ние мо\'{е} есть. Жив\'{а} б\'{у}дет душ\'{а} мо\'{я} и восхв\'{а}лит Тя, и судьб\'{ы} Тво\'{я} пом\'{о}гут мне. Заблуд\'{и}х, \'{я}ко овч\'{а} пог\'{и}бшее, взыщ\'{и} раб\'{а} Твоег\'{о}, \'{я}ко з\'{а}поведей Тво\'{и}х не заб\'{ы}х.

\pominalnayaslava

\mysubtitle{По 17-й кафизме}

\TrisviatoePoOtcheNash

\mysubtitle{И тропари, глас 2-й}

Согреш\'{и}х к Теб\'{е}, Сп\'{а}се, \'{я}ко бл\'{у}дный сын: приим\'{и} мя, \'{О}тче, к\'{а}ющагося, и пом\'{и}луй мя, Б\'{о}же. 

\slavan

Зов\'{у} к Теб\'{е}, Христ\'{е} Сп\'{а}се, мытар\'{е}вым гл\'{а}сом: оч\'{и}сти мя \'{я}коже \'{о}наго, и пом\'{и}луй мя, Б\'{о}же. 

\inynen

Богор\'{о}дице, не пр\'{е}зри мя тр\'{е}бующа заступл\'{е}ния Твоег\'{о}: на Тя бо упов\'{а} душ\'{а} м\'{о}я, и пом\'{и}луй мя. 

Г\'{о}споди, пом\'{и}луй. \myemph{(40 раз)}

\myemph{И поклоны по силе.}

\mysubtitle{Молитва}

Влад\'{ы}ко Г\'{о}споди Вседерж\'{и}телю и Тв\'{о}рче всех, щедр\'{о}т От\'{е}ц, и м\'{и}лости Бог, от земл\'{и} созд\'{а}вый челов\'{е}ка, и показ\'{а}вый eг\'{о} по \'{о}бразу Твоем\'{у} и по под\'{о}бию, да и тем просл\'{а}вится великол\'{е}пое имя Тво\'{е} на земл\'{и}, и ист\'{о}ргнена \'{у}бо преступл\'{е}нием Тво\'{и}х з\'{а}поведей, п\'{а}ки на л\'{у}чшее возсозд\'{а}вый eг\'{о} во Христ\'{е} Тво\'{е}м, и возвед\'{ы}й на Небес\'{а}: благодар\'{ю} Тя, \'{я}ко умн\'{о}жил ес\'{и} на мне вел\'{и}чие Тво\'{е}, и не пр\'{е}дал мя ес\'{и} враг\'{о}м мо\'{и}м в кон\'{е}ц, ист\'{о}ргнути мя \'{и}щущым в пр\'{о}пасть \'{а}дову, ниж\'{е} ост\'{а}вил мя ес\'{и} пог\'{и}бнути со беззак\'{о}нии мо\'{и}ми. Н\'{ы}не \'{у}бо, Многом\'{и}лостиве и Любобл\'{а}же Г\'{о}споди, не хот\'{я}й см\'{е}рти гр\'{е}шнаго, но обращ\'{е}ния ожид\'{а}яй, и при\'{е}мляй: \'{и}же низв\'{е}рженныя исправл\'{я}яй, сокруш\'{е}нныя исцел\'{я}яй, обрат\'{и} и мен\'{е} к пока\'{я}нию, и низв\'{е}рженнаго испр\'{а}ви, и сокруш\'{е}ннаго исцел\'{и}: помян\'{и} Тво\'{я} щедр\'{о}ты, и \'{я}же от в\'{е}ка Тво\'{ю} непостиж\'{и}мую бл\'{а}гость и мо\'{я} безм\'{е}рная заб\'{у}ди беззак\'{о}ния, \'{я}же д\'{е}лом и сл\'{о}вом, и м\'{ы}слию соверш\'{и}х: разреш\'{и} ослепл\'{е}ние с\'{е}рдца моег\'{о}, и даждь ми сл\'{е}зы умил\'{е}ния на очищ\'{е}ние скв\'{е}рны м\'{ы}сли мое\'{я}. Усл\'{ы}ши, Г\'{о}споди, вонм\'{и}, Человекол\'{ю}бче, оч\'{и}сти, Благоутр\'{о}бне, и от муч\'{и}тельства во мне ц\'{а}рствувщих страст\'{е}й ока\'{я}нную мо\'{ю} д\'{у}шу свобод\'{и}. И не ктом\'{у} да содерж\'{и}т мя грех, ниж\'{е} да возм\'{о}жет на мя бор\'{и}тель д\'{е}мон, ниж\'{е} к своем\'{у} хот\'{е}нию да вед\'{е}т мя, но держ\'{а}вною Тво\'{е}ю рук\'{о}ю, eг\'{о} влад\'{ы}чества исх\'{и}тивый мя, Ты ц\'{а}рствуй во мне, Благ\'{и}й и Человеколюб\'{и}вый Г\'{о}споди, и всег\'{о} Твоег\'{о} б\'{ы}ти, и ж\'{и}ти мне пр\'{о}чее по Тво\'{е}й благовол\'{и} в\'{о}ли. И под\'{а}ждь ми неизреч\'{е}нною бл\'{а}гостию с\'{е}рдца очищ\'{е}ние, уст хран\'{е}ние, правот\'{у} де\'{я}ний, мудров\'{а}ние смир\'{е}нное, мир помысл\'{о}в, тишин\'{у} душ\'{е}вных мо\'{и}х сил, р\'{а}дость дух\'{о}вную, люб\'{о}вь \'{и}стинную, долготерп\'{е}ние, бл\'{а}гость, кр\'{о}тость, в\'{е}ру нелицем\'{е}рну, воздерж\'{а}ние обдерж\'{а}тельное, и всех мя благ\'{и}х плод\'{о}в исп\'{о}лни, даров\'{а}нием Свят\'{а}го Твоег\'{о} Д\'{у}ха. И не возвед\'{и} мен\'{е} в преполов\'{е}ние дний мо\'{и}х, ниж\'{е} неиспр\'{а}влену и негот\'{о}ву д\'{у}шу мо\'{ю} восх\'{и}тиши, но соверш\'{и} мя Тво\'{и}м соверш\'{е}нством, и т\'{а}ко мя насто\'{я}щаго жит\'{и}я извед\'{и}, \'{я}ко да невозбр\'{а}нно прош\'{е}д нач\'{а}ла и вл\'{а}сти тьм\'{ы}, Тво\'{е}ю благод\'{а}тию узр\'{ю} и аз неприст\'{у}пныя Твое\'{я} сл\'{а}вы добр\'{о}ту неизреч\'{е}нную, со вс\'{е}ми свят\'{ы}ми Тво\'{и}ми, в н\'{и}хже освят\'{и}ся, и просл\'{а}вися всечестн\'{о}е и великол\'{е}пое \'{и}мя Тво\'{е}, Отц\'{а} и С\'{ы}на и Свят\'{а}го Д\'{у}ха, н\'{ы}не и пр\'{и}сно, и во в\'{е}ки век\'{о}в, ам\'{и}нь.

\mysubtitle{Окончание молитв}

\Chestneyshuyu

\slavainynen

Г\'{о}споди, пом\'{и}луй. \myemph{(Трижды)}

\MolitvamiSviatyhOtecNashih

Г\'{о}споди Иис\'{у}се Христ\'{е}, С\'{ы}не Б\'{о}жий, мол\'{и}тв р\'{а}ди Преч\'{и}стыя Твое\'{я} М\'{а}тери, с\'{и}лою Честн\'{а}го и Животвор\'{я}щаго Крест\'{а}, и свят\'{ы}х неб\'{е}сных сил безпл\'{о}тных, и препод\'{о}бных и богон\'{о}сных от\'{е}ц н\'{а}ших, и свят\'{а}го прор\'{о}ка Дав\'{и}да, и всех свят\'{ы}х, пом\'{и}луй и спас\'{и} мя гр\'{е}шнаго, \'{я}ко Благ и Человекол\'{ю}бец, ам\'{и}нь.

\end{mymulticols}

\mychapterending

\mychapter{Значение 17-й кафизмы}\begin{mymulticols}
%http://www.molitvoslov.org/text229.htm 


В течение всех сорока дней по смерти человека его родные и близкие должны читать Псалтирь. Сколько кафизм в день "--- зависит от времени и сил читающих, но непременно чтение должно быть ежедневным. Когда прочитана вся Псалтирь, она читается сначала. Только не следует забывать, что после каждой «Славы…» надо читать молитвенное прошение о поминовении усопшего (из «Последования по исходе души от тела»). Многие родные и близкие умершего, ссылаясь на то, что нет времени или не имеют Псалтири, или не умеют читать по-церковнославянски, доверяют это чтение другим (чтецам) за плату или иное вознаграждение. Но молитва будет сильнее, искреннее, чище, если родной или близкий умершему человек будет сам просить Бога о помиловании усопшего. 

В третий, девятый, сороковой дни следует читать по усопшему 17-ю кафизму. 

В этой кафизме изображается блаженство ходивших в законе Господнем, т.~е. блаженство праведных людей, старавшихся жить по заповедям Божиим. 

Смысл и значение 118 псалма раскрываются в 19-м стихе: «Пришлец (странник) аз есмь на земли: не скрый от мене заповеди Твоя». Толковая Библия под ред. А.П. Лопухина дает этому стиху следующее объяснение: «Жизнь на земле есть странствование, путешествие, совершаемое человеком для достижения своего отечества и постоянного, вечного местопребывания. Очевидно, последнее "--- не на земле, а за гробом. Если же так, то земная жизнь должна быть подготовленном к загробной и к ней может привести только безошибочно избранный на земле путь. Как и где найти последний? Этот путь указан в заповедях Закона. Кто не следует им, тот заблуждается и не достигнет загробной обители, т.~е. загробного упокоения, как награды за понесенные труды к его достижению. Здесь "--- довольно ясное учение о цели земного существования, бессмертии человеческой души и загробном мздовоздаянии». 

\end{mymulticols}

\mychapterending

\mychapter{Последование панихиды}\begin{mymulticols}
%http://www.molitvoslov.org/text231.htm 


\myemph{После обычного начала и 90-го псалма:} «Живый в помощи Вышняго…» \myemph{Великая ектения} «Миром Господу помолимся» \myemph{с заупокойными прошениями, затем} «Аллилуиа» \myemph{со стихами, потом:}

\mysubtitle{Тропарь, глас 8-й:}

Глубиною мудрости человеколюбно вся строяй, и полезное всем подаваяй, Едине Содетелю, упокой, Господи, души раб Твоих: на Тя бо упование возложиша, Творца и Зиждителя и Бога нашего. 

Слава:, \inynen

Тебе и стену и пристанище имамы, и Молитвенницу благоприятну к Богу, Егоже родила еси, Богородице Безневестная, верных спасение.

\mysubtitle{Тропари за упокой, глас 5-й}

\pripev[Запев:]{Благословен еси, Господи, научи мя оправданием Твоим.} 

Святых лик обрете источник жизни и дверь райскую, да обрящу и аз путь покаянием, погибшее овча аз есмь, воззови мя, Спасе, и спаси мя. 

\pripev[Запев:]{Благословен еси, Господи, научи мя оправданием Твоим.} 

Агнца Божия проповедавше и заклани бывше якоже агнцы, и к жизни нестареемей, святии; и присносущней преставльшеся: Того прилежно, мученицы, молите долгов разрешение нам даровати. 

\pripev[Запев:]{Благословен еси, Господи, научи мя оправданием Твоим.} 

В путь узкий хождшии прискорбный, вси в житии крест яко ярем вземшии и Мне последовавший верою, приидите насладитеся, ихже уготовах вам почестей и венцев небесных. 

\pripev[Запев:]{Благословен еси, Господи, научи мя оправданием Твоим.} 

Образ есмь неизреченныя Твоея славы, аще и язвы ношу прегрешений, ущедри Твое создание, Владыко, и очисти Твоим благоутробием, и возжеленное Отечество подаждь ми, рая паки жителя мя сотворяя. 

\pripev[Запев:]{Благословен еси, Господи, научи мя оправданием Твоим.} 

Древле убо от не сущих создавый мя и образом Твоим Божественным почтый, преступлением же заповеди, паки мя возвративый в землю, от неяже взят бых, но еже по подобию возведи древнею добротою возобразитися. 

\pripev[Запев:]{Благословен еси, Господи, научи мя оправданием Твоим.} 

Упокой, Боже, рабы Твоя и учини я в раи, идеже лицы святых, Господи, и праведницы сияют яко светила; усопшия рабы Твоя упокой, презирая их вся согрешения. 

\slava 

Трисиятельное Единаго Божества, благочестно поем вопиюще: Свят еси, Отче Безначальный, Собезначальный Сыне, и Божественный Душе; просвети нас, верою Тебе служащих, и вечнаго огня исхити. 

\inyne 

Радуйся, Чистая, Бога плотию рождшая во спасение всех, Еюже род человеческий обрете спасение, Тобою да обрящем рай, Богородице Чистая, Благословенная! 

Аллилуиа, аллилуиа, аллилуиа, слава Тебе, Боже. \myemph{(Трижды)} 

\myemph{Посем ектения:} Паки и паки… 

Еще молимся о упокоении душ усопших рабов Божиих \myemph{(имярек)}, и о еже простится им всякому прегрешению, вольному же и невольному. 

\myemph{Хор:} Яко да Господь Бог учинит души их, идеже праведнии упокояются. 

\myemph{Хор:} Господи, помилуй. 

Милости Божия, Царства Небеснаго, и оставления грехов их, у Христа Безсмертнаго Царя и Бога нашего просим. 

\myemph{Хор:} Подай, Господи.

\mysubtitle{Молитва}

Боже духов и всякия плоти, смерть поправый и диавола упразднивый, и живот миру Твоему даровавый! Сам Господи, упокой души усопших рабов Твоих \myemph{(имярек)}, в месте светле, в месте злачне, в месте покойне, отнюдуже отбеже болезнь, печаль и воздыхание. Всякое согрешение, содеянное ими, словом, или делом, или помышлением, яко Благий Человеколюбец Бог прости, яко несть человек, иже жив будет и не согрешит, Ты бо един, кроме греха, правда Твоя правда во веки, и слово Твое истина. 

\myemph{Возглас:} Яко Ты еси Воскресение, и Живот, и покой усопших рабов Твоих…

\mysubtitle{Седален, глас 5-й:}

Покой, Спасе наш, с праведными рабы Твоя, и сия всели во дворы Твоя, якоже есть писано, презирая яко Благ, прегрешения их, вольная и невольная, и вся яже в ведении и не в ведении, Человеколюбче. 

\slavainynen

От Девы возсиявый миру, Христе Боже, сыны света Тою показавый, помилуй нас.

\mysubtitle{Ирмосы канона, глас 8:}

\mysubtitle{ПЕСНЬ 1:} Яко по суху пешешествовав Израиль, по бездне стопами, гонителя фараона видя потопляема, Богу победную песнь поим вопияше. 

\myemph{Катавасия:} Воду прошед яко сушу, и египетскаго зла избежав, израильтянин вопияше: 

Избавителю и Богу нашему поим. 

Запевы: Покой, Господи, души усопших раб Твоих. 

\slava 

\inyne

\mysubtitle{ПЕСНЬ 3:} Несть свят, якоже Ты, Господи Боже мой, вознесый рог верных Твоих, Блаже, и утвердивый нас на камени исповедания Твоего. 

\myemph{Катавасия:} Небеснаго круга Верхотворче, Господи, и Церкве Зиждителю, Ты мене утверди в любви Твоей, желаний краю, верных утверждение, Едине Человеколюбче. 

\mysubtitle{ПЕСНЬ 6:} Житейское море, воздвизаемое зря напастей бурею, к тихому пристанищу Твоему притек, вопию Ти: возведи от тли живот мой, Многомилостиве. 

\myemph{Катавасия:} Молитву пролию ко Господу, и Тому возвещу печали моя, яко зол душа моя исполнися, и живот мой аду приближися, и молюся яко Иона: от тли, Боже, возведи мя. 

\myemph{Кондак, глас 8-й:} Со святыми упокой, Христе, душы раб Твоих, идеже несть болезнь, ни печаль, ни воздыхание, но жизнь безконечная. 

\myemph{Икос:} Сам Един еси Безсмертный, сотворивый и создавый человека: земнии убо от земли создахомся, и в землю туюжде пойдем, якоже повелел еси, Создавый мя и рекий ми: яко земля еси и в землю отыдеши, аможе вси человецы пойдем, надгробное рыдание творяще песнь: аллилуиа, аллилуиа, аллилуиа. 

\myemph{Иерей:} Богородицу и Матерь Света в песнех возвеличим. 

\myemph{Хор:} Дуси и души праведных восхвалят Тя, Господи. 

\mysubtitle{ПЕСНЬ 9:} Бога человеком невозможно видети, на Него же не смеют чини ангельстии взирати: Тобою бо, Всечистая, явися человеком Слово воплощенно, Егоже величающе, с небесными вой Тя ублажаем. 

\myemph{Катавасия:} Ужасеся о сем небо, и земли удивишася концы, яко Бог явися человеком плотски, и чрево Твое бысть пространнейшее небес. Тем Тя, Богородицу, ангелов и человек чиноначалия величают. 

\TrisviatoePoOtcheNash

\myemph{Тропари, глас 4-й:} Со духи праведных скончавшихся души раб Твоих, Спасе, упокой, сохраняя их во блаженней жизни, яже у Тебе, Человеколюбче. 

В покоищи Твоем, Господи, идеже вси святии Твои упокоеваются, упокой и души раб Твоих, яко Един еси Человеколюбец. 

\slavan

Ты еси Бог, сошедый во ад и узы окованных разрешивый, Сам и души раб Твоих упокой. 

\inynen

Едина Чистая и Непорочная Дево, Бога без семене рождшая, моли спастися душам их. 

\myemph{Посем ектения:} Помилуй нас, Боже… 

\myemph{Отпуст и возглашение иерея или диакона:}

Во блаженном успении вечный покой подаждь, Господи, усопшим рабом Твоим \myemph{(имярек)} и сотвори им вечную память. 

\myemph{Хор:} Вечная память. \myemph{(Трижды)}

\end{mymulticols}

\mychapterending

\mychapter{Чин литии, совершаемой мирянином дома и на кладбище}\begin{mymulticols}
%http://www.molitvoslov.org/text233.htm 


\MolitvamiSviatyhOtecNashih 

Сл\'{а}ва Теб\'{е}, Б\'{о}же наш, сл\'{а}ва Теб\'{е}. 

\TsariuNebesnyj

\TrisviatoePoOtcheNash

Г\'{о}споди, пом\'{и}луй. \myemph{(12 раз)}

\slavainynen

\priiditepoklonimsia

\mysubtitle{Псалом 90}

\PsalmNinety

\slavainyne 

Аллил\'{у}иа, аллил\'{у}иа, аллил\'{у}иа, сл\'{а}ва Теб\'{е}, Б\'{о}же. \myemph{(Трижды)}

\mysubtitle{Тропарь, глас 4-й:}

Со духи праведных скончавшихся, душу раба Твоего, Спасе, упокой, сохраняя ю во блаженной жизни, яже у Тебе, Человеколюбче. 

В покоищи Твоем, Господи, идеже вси святии Твои упокоеваются, упокой и душу раба Твоего, яко Един еси Человеколюбец. 

\slavan

Ты еси Бог, сошедый во ад, и узы окованных разрешимый, Сам и душу раба Твоего упокой. 

\inynen

Едина Чистая и непорочная Дево, Бога без семени рождшая, моли спастися душе его.

\mysubtitle{Седален, глас 5-й:}

Покой, Спасе наш, с праведными раба Твоего, и сего всели во дворы Твоя, якоже есть писано, презирая, яко Благ, прегрешения его вольная и невольная, и вся яже в ведении и не в ведении, Человеколюбче. 

\mysubtitle{Кондак, глас 8-й:}

Со святыми упокой, Христе, душу раба Твоего, идеже несть болезнь, ни печаль, ни воздыхание, но жизнь безконечная.

\mysubtitle{Икос:}

Сам Един еси Безсмертный, сотворивый и создавый человека, земнии убо от земли создахомся, и в землю туюжде пойдем, якоже повелел еси, Создавый мя и рекий ми: яко земля еси, и в землю отъидеши, аможе вси человецы пойдем, надгробное рыдание творяще песнь: аллилуиа, аллилуиа, аллилуиа. 

\Chestneyshuyu 

\slavainyne 

Господи, помилуй \myemph{(Трижды)}, благослови. 

\MolitvamiSviatyhOtecNashih

Во блаженном успении вечный покой подаждь, Господи, усопшему рабу Твоему \myemph{(имя)}, и сотвори ему вечную память. 

Вечная память. \myemph{(Трижды)}

Душа его во благих водворится, и память его в род и род. 

\end{mymulticols}

\mychapterending

\mychapter{Акафист за единоумершего}
%http://www.molitvoslov.org/text230.htm 


{\centering \myemph{Читается ежедневно в течение 40 дней по смерти и столько же перед годовщиной смерти.}\par} 

\begin{mymulticols}

\newcommand{\akafistZaEdinKondakOne}{%
\mysubtitle{Кондак 1}

Избранный Ходатаю и Первосвященниче, о спасении мира грешнаго душу Свою положивый, давый нам власть чадами Божиими быти и обитати в невечернем дни Царствия Твоего! Даруй прощение и вечную радость усопшему, о немже с мольбою взываем Ти: Иисусе, Судие Всемилостивый, рая сладости сподоби раба Твоего.
}

\newcommand{\akafistZaEdinIkosOne}{%
\mysubtitle{Икос 1}

Господом данный Ангеле хранителю святый, прииди помолиться о рабе твоем, егоже на всех путех жизни сопровождал, хранил и наставлял еcи, воззови с нами ко Спасу Всещедрому. 

Иисусе, истреби рукописание грехов раба Твоего \myemph{(имярек)}. 

Иисусе, уврачуй его язвы душевные. 

Иисусе, да не будет на земли горьких воспоминаний о нем. 

Иисусе, сего ради помилуй огорчавших его и обиженных им. 

Иисусе покрый его несовершенства светоносною ризою Твоего искупления. 

Иисусе, возвесели его милосердием Твоим. 

Иисусе неизреченный, великий и чудный, явися ему Сам. 

Иисусе, Судие Всемилостивый, рая сладости сподоби раба Твоего.
}

\akafistZaEdinKondakOne

\akafistZaEdinIkosOne

\mysubtitle{Кондак 2}

Яко безутешная горлица носится душа над юдолию земною, созерцая с высоты божественного разумения грехи и соблазны минувшаго пути, горько скорбя о каждом невозвратном дне, ушедшем без пользы, но помилуй раба Твоего, Владыко, да внидет он в покой Твой, взывая: Аллилуиа.

\mysubtitle{Икос 2}

Если Ты страдал о всем мире, если Ты проливал слезы и кровавый пот о живых и мертвых, то кто нас удержит от молитвы за усопшаго. Подражая Тебе, сошедшему даже до ада, молимся о спасении раба Твоего. 

Иисусе, жизни Подателю, озари его светом Твоим. 

Иисусе, да будет он едино с Тобою и Отцем. 

Иисусе, всех призываяй в виноградник Твой, не забуди озарити его светом Твоим. 

Иисусе, щедрый Раздаятелю вечных наград, яви его светом чертога Твоего. 

Иисусе, верни душе его благодатныя силы первозданныя чистоты. 

Иисусе, да умножатся во имя его добрыя дела. 

Иисусе, согрей осиротевших Твоею таинственною отрадою. 

Иисусе, Судие Всемилостивый, рая сладости сподоби раба Твоего.

\mysubtitle{Кондак 3}

Связанный узами плоти раб Твой падал греховно, но дух его томился по Твоей вечной правде и святости, ныне же, когда немощь плотская скована могильным тлением, да вознесется душа его превыше солнца к Тебе Всесвятому и воспоет песнь избавления: Аллилуиа.

\mysubtitle{Икос З}

Верховный апостол Твой в холодную нощь у костра трижды отрекся от Тебе и Ты спасл еси его. Единый ведый человеческого естества немощь, прости и рабу Твоему \myemph{(имярек)} многовидныя отпадения от воли Твоея. 

Иисусе, всели его там, где нет заблуждений. 

Иисусе, избави его от тягостных терзаний совести. 

Иисусе, да сгинет навеки память грехов. 

Иисусе, соблазнов юности его не помяни. 

Иисусе, от тайных беззаконий очисти его. 

Иисусе, осени его тихим светом спасения. 

Иисусе, Судие Всемилостивый, рая сладости сподоби раба Твоего.

\mysubtitle{Кондак 4}

Бури жизни миновали, страдания земные окончены, безсильны враги с их злобою, но сильна любовь, избавляющая от вечного мрака и спасающая всех, о ком возносится Тебе дерзновенная песнь: Аллилуиа.

\mysubtitle{Икос 4}

Ты без числа милосерд к нам. Ты единый Избавитель, что мы прибавим к подвигу спасающей любви Твоей, но Симон Киринейский помогал нести Крест Тебе Всесильному, так и ныне благости Твоей угодно спасение близких совершать с участием нашим. 

Иисусе, Ты заповедал друг друга тяготы носити. 

Иисусе, союз любви положивый между мертвыми и живыми. 

Иисусе, да послужат подвиги любящих во спасение рабу Твоему \myemph{(имярек)}. 

Иисусе, услыши его вопль сердечный, возносимый нашими устами. 

Иисусе, в наших слезах приими его покаяние. 

Иисусе, Судие Всемилостивый, рая сладости сподоби раба Твоего.

\mysubtitle{Кондак 5}

Боже, да будет принят Тобою его предсмертный вздох сокрушения, якоже мольба благоразумнаго разбойника. Он угас на жизненном кресте, дай наследовать ему Твоя обетования, якоже тому: «Аминь, глаголю тебе, со Мною будеши в раю», где сонмы раскаявшихся грешников радостно поют: Аллилуиа.

\mysubtitle{Икос 5}

За нас Распятый, за нас измученный, простри руку с Твоего Креста, каплями крови Твоея безследно смой излитыя согрешения его, благообразною Твоею наготою согрей обнаженную осиротевшую душу. 

Иисусе, Ты знал его жизнь до рождения и возлюбил его. 

Иисусе, Ты видел его из далече с высоты Креста Твоего. 

Иисусе, Ты простирал ему в даль грядущему изъязвленные объятия Твои. 

Иисусе, Ты Взывал о прощении его на кровавой Голгофе. 

Иисусе, Ты кротко умирал за него в тяжких муках. 

Иисусе, претерпевый во гробе положение, освяти его могильный покой. 

Иисусе воскресший, вознеси ко Отцу озлобленную миром и Тобою спасенную душу. 

Иисусе, Судие Всемилостивый, рая сладости сподоби раба Твоего.

\mysubtitle{Кондак 6}

Спит он вечным сном могилы, но душа его не дремлет, чает Тебя Господи, жаждет Тебя вечного Жениха. Да исполнятся на умершем слова Твои: «Ядый Мою Плоть и пияй Мою Кровь, имать живот вечный». Дай ему ясти от манны сокровенныя и пети у Престола Твоего: Аллилуиа.

\mysubtitle{Икос 6}

Смерть разлучила со всеми ближними, стала дальше душа, знаемии сокрушаются, и токмо Ты Един остался близок. Разрушились преграды плоти и Ты открылся в неприступном величии Божества с ожиданием ответа. 

Иисусе, Любовь превыше всякого разумения, помилуй раба Твоего. 

Иисусе, удаляясь от Тебя, он тяжко страдал. 

Иисусе, прости неверность его сердца. Иисусе, обманутые надежды рождали тоску по Тебе. 

Иисусе, вспомни те часы, когда душа его трепетала восторгом Твоим. 

Иисусе, дай скончавшемуся неземную радость и покой. 

Иисусе, единый верный, неизменный, приими его. 

Иисусе, Судие Всемилостивый, рая сладости сподоби раба Твоего.

\mysubtitle{Кондак 7}

Веруем, что недолгой будет разлука наша. Мы хороним тебя, как на ниве зерно, ты произрастешь в иной стране. Да погибнут в могиле плевелы грехов твоих, а дела добрые там просияют, где семена добра приносят нетленные плоды, где души святые поют: Аллилуиа.

\mysubtitle{Икос 7}

Когда уделом умершего станет забвение, когда образ его поблекнет в сердцах и время изгладит место с могилою и ревность молитвы о нем, тогда Ты не остави его, дай отраду одинокой душе. 

Иисусе, Твоя любовь не охладевает. 

Иисусе, неистощимо Твое благоволение. 

Иисусе, в неумолкаемых мольбах Церкви да омоются грехи его приношением Жертвы Безкровной. 

Иисусе, предстательством всех святых даруй ему благодать молитвы о живых. 

Иисусе, во дни испытаний наших приими его ходатайство о нас. 

Иисусе, Судие Всемилостивый, рая сладости сподоби раба Твоего.

\mysubtitle{Кондак 8}

Будем молиться со слезами, когда память об усопшем мучительно свежа, будем поминать имя его в нощи и во дни, раздавая милостыню, питая гладных, из глубины души взывая: Аллилуиа.

\mysubtitle{Икос 8}

Тайнозритель Иоанн Богослов созерцал у Престола Агнца Божия великое множество людей, облеченных в белые ризы: это те, кто пришел от великой скорби. Они радостно служат Богу день и ночь, и Бог обитает с ними, и не коснется их мука. 

Иисусе, причти к ним и раба Твоего \myemph{(имярек)}. 

Иисусе, он много страдал и томился. 

Иисусе, известны Тебе все горькие часы и тягостные минуты его. 

Иисусе, на земли он имел печали и скорби, дай на небе отраду. 

Иисусе, услади его от источников живыя воды. 

Иисусе, отыми всяку слезу от очей его. 

Иисусе, всели его, где не опаляет, но живит солнце правды Твоея. 

Иисусе, Судие Всемилостивый, рая сладости сподоби раба Твоего.

\mysubtitle{Кондак 9}

Кончено странствование земное, какой благодатный переход в мир Духа, какое созерцание новых неведомых миру земному вещей и небесных красот, душа возвращается в отечество свое, где светлое солнце, правда Божия просвещает поющих: Аллилуиа.

\mysubtitle{Икос 9}

Если отблеск и след Твой полагает сияние на лице смертных, то каков же Ты Сам. Если плоды Твоих рук так прекрасны, и земля, отражающая только тень Твою, полна невыразимым величием, то каков же невидимый лик Твой. Открой славу Твою усопшему рабу Твоему \myemph{(имярек)}. 

Иисусе, обостри его слух к восприятию Твоего Божества. 

Иисусе, обостри его слух к разумению небесных. 

Иисусе, да будет его радость преисполненная. 

Иисусе, подкрепи его надеждою встречи в обителях блаженных. 

Иисусе, дай нам почувствовать благодатную силу заупокойной молитвы. 

Иисусе, Судие Всемилостивый, рая сладости сподоби раба Твоего.

\mysubtitle{Кондак 10}

Отче наш, скончавшагося в Царствие Твое приими, где нет греха и зла, где нерушима Святая воля, где в сонмах чистейших душ и непорочных ангелов святится Твое благодатное имя и благоухает хвала: Аллилуиа.

\mysubtitle{Икос 10}

В тот день ангелы поставят Престол Твой, Судие, и Ты возсияеши во славе Отца Твоего, неся воздаяние всякому человеку. О, воззри тогда милостиво на смиреннаго раба Твоего \myemph{(имярек)}, рцы ему: «Прииди одесную Мене». 

Иисусе, яко Бог власть имаши оставляти грехи. 

Иисусе, прости его согрешения забытыя или стыдом утаенныя. 

Иисусе, отпусти беззакония немощи и неведения. 

Иисусе, избави его от несвятимых глубин адскаго отчаяния. 

Иисусе, да унаследует он Твои животворящия обетования. 

Иисусе, сопричти его благословенным Отца Твоего. 

Иисусе, дай ему во веки нескончаемое блаженство. 

Иисусе, Судие Всемилостивый, рая сладости сподоби раба Твоего.

\mysubtitle{Кондак 11}

Владыко Всеблагий, да отверзятся усопшему солнцевидные врата райские, да встретят его с ликованием соборы праведных и святых, сонмы близких и любящих его, да возрадуются о нем светоносные ангелы Твои, да узрит он и Присноблаженную Матерь Твою там, где победно звучит: Аллилуиа.

\mysubtitle{Икос 11}

Под дыханием Твоим оживают цветы, воскресает природа, пробуждаются сонмы мельчайших тварей, Твой взор светлее весенних небес, Твоя любовь, Иисусе, теплее лучей солнечных. Ты из праха земного воскресил бренную плоть человеческую к расцвету вечной нетленной жизни весны, тогда озари и раба Твоего \myemph{(имярек)} светом милостей Твоих. 

Иисусе, в Твоей деснице благоволение и жизнь. 

Иисусе, во взоре Твоем свет и любовь. 

Иисусе, избави усопшего от вечныя смерти духовныя. 

Иисусе, он уснул с надеждою, подобно Нилу-реке пред холодной зимой. 

Иисусе, пробуди его, когда терния земли облекутся цветом вечности. 

Иисусе, да не омрачит ничто земное его последнего сна. 

Иисусе, Счастье Неизменное и цель нашего бытия. 

Иисусе, Судие Всемилостивый, рая сладости сподоби раба Твоего.

\mysubtitle{Кондак 12}

Христе! Ты "--- Царство Небесное, Ты "--- земля кротких, Ты "--- обитель многих, Ты "--- питие совершенно новое, Ты "--- одеяние и венец преподобных, Ты "--- ложе упокоения святых, Ты "--- Сладчайший Иисусе! Тебе подобает хвала: Аллилуиа.

\mysubtitle{Икос 12}

Под образом тихих садов неземной красоты, и светлых как солнце обителей и в великолепии небесных песнопений Ты открыл нам блаженство любящих Тя. 

Иисусе, да внидет раб Твой в радость Твою. 

Иисусе, облеки его сиянием славы Отца. 

Иисусе, просвяти его озарением Духа Святаго. 

Иисусе, да услышит он неизреченную песнь Херувимов. 

Иисусе, да будет он восходить от славы в славу. 

Иисусе, да узрит он Тебя лицем к лицу. 

Иисусе, Судие Всемилостивый, рая сладости сподоби раба Твоего.

\mysubtitle{Кондак 13}

О, Женише Безсмертный, в полночь греха и неверия грядый с небес со ангелами судити миру всему. Отверзи двери славного чертога Твоего рабу Твоему \myemph{(имярек)}, да в безчисленных сонмах святых во веки поет: Аллилуиа, Аллилуиа, Аллилуиа. 

\myemph{Этот кондак читается трижды, затем 1-й икос «Господом данный Ангеле…» и 1-й кондак «Избранный Ходатаю».}

\akafistZaEdinIkosOne

\akafistZaEdinKondakOne

\end{mymulticols}

\mychapterending

\mychapter{Молитва родителей за детей}\begin{mymulticols}
%http://www.molitvoslov.org/text225.htm 


Господи Иисусе Христе, Боже наш, Владыко живота и смерти. Утешителю скорбящих! С сокрушенным и умиленным сердцем прибегаю к Тебе и молюся Ти: помяни, Господи, во Царствии Твоем усопшего раба Твоего (-ую рабу Твою), чадо мое \myemph{(имярек)}, и сотвори ему (ей) вечную память. Ты, Владыко живота и смерти, даровал еси мне чадо сие. Твоей же благой и премудрой воле изволися и отьяти е у мене. Буди благословенно Имя Твое, Господи. Молю Тя, Судие неба и земли, безконечною любовию Твоею к нам, грешным, прости усопшему чаду моему вся согрешения его, вольная и невольная, яже словом, яже делом, яже ведением и неведением. Прости, Милостиве, и наша родительская согрешения, да не пребудут они на чадах наших: вем, яко множицею согрешихом пред Тобою, множицею не соблюдохом, не сотворихом, якоже заповедал еси нам. Аще же усопшее чадо наше, нашея или своея ради вины, бяше в житии сем, работая миру и плоти своея и не паче Тебе, Господу и Богу своему: аще возлюби прелести мира сего, а не паче Слово Твое и Заповеди Твоя, аще предавшеся сластем житейским, а не паче сокрушению о гресех своих, и в невоздержании бдение, пост и молитву забвению предавше,"--- молю Тя усердно, прости преблагий Отче, чаду моему вся таковыя прегрешения его, прости и ослаби, аще и ино злое сотвори в житии сем, Христе Иисусе! Ты воскресил еси дщерь Иаира по вере и молитве отца ея. Ты исцелил еси дщерь жены-хананеянки по вере и прошению матери ея: услыши убо и молитву мою, не презри и моления моего о чаде моем. Прости, Господи, прости, вся согрешения его и, простив и очистив душу его, изми муки вечныя и всели ю со всеми святыми Твоими, от века благоугодившими Тебе, идеже несть болезнь, ни печаль, ни воздыхание, но жизнь безконечная: яко несть человек, иже жив будет и не согрешит. Ты Един кроме греха: да егда имаши судити мирови, услышит чадо мое превожделенный глас Твой: приидите, благословении Отца Моего, и наследуйте уготованное вам Царствие от сложения мира. Яко Ты еси Отец милостей и щедрот, Ты живот и воскрешение наше, и Тебе славу возсылаем со Отцем и Святым Духом, ныне и присно и во веки веков. Аминь. 

\end{mymulticols}

\mychapterending

\mychapter{Молитва вдовца за супругу}\begin{mymulticols}
%http://www.molitvoslov.org/text223.htm 


Христе Иисусе, Господи и Вседержителю! В сокрушении и умилении сердца моего молюся Тебе: упокой, Господи, душу усопшия рабы Твоея \myemph{(имярек)}, в Небесном Царствии Твоем. Владыко Вседержителю! Ты благоволил еси супружеский союз мужа и жены, егда рекл еси: не добро быти человеку единому, сотворим ему помощника по нему. Ты освятил еси союз сей во образ духовнаго союза Христа с Церковью. Верую, Господи, и исповедую, яко Ты благословил еси сочетати и мене сим святым союзом с единою из рабынь Твоих. Твоей же благой и премудрой воле изволися отъяти у мене сию рабу Твою, юже дал еси мне, яко помощницу и сопутницу жизни моея. Преклоняюся пред сею Твоею волею, и молюся Ти от всего сердца моего, приими моление мое сие о рабе Твоей \myemph{(имярек)}, и прости ей, аще согреши словом, делом, помышлением, ведением и неведением; аще земное возлюби паче небеснаго; аще о одежде и украшении тела своего печеся паче, неже о просвещении одеяния души своея; или аще небреже о чадех своих; аще преогорчи кого словом или делом; аще проропта в сердце своем на ближняго своего или осуди кого или ино что от таковых злых содела. Вся сия прости ей, яко благий и человеколюбивый: яко несть человек, иже жив будет и не согрешит. Не вниди в суд с рабою Твоею, яко созданием Твоим, не осуди ю по грехом ея на вечныя муки, но пощади и помилуй по велицей милости Твоей. Молю и прошу Тя, Господи Сил, даруй ми по вся дни жизни моея не преставати молитися о усопшей рабе Твоей, и даже до кончины живота моего просити ей у Тебе, Судии всего мира, оставления согрешений ея. Да якоже Ты, Боже, положил еси на главу ея венец от камене честна, венчая ю зде на земли; тако увенчай ю вечною Твоею славою в Небесном Царствии Твоем, со всеми святыми, тамо ликующими, да вкупе с ними вечно воспевает Всесвятое Имя Твое с Отцем и Святым Духом. Аминь. 

\end{mymulticols}

\mychapterending

\mychapter{Молитва вдовицы за супруга}\begin{mymulticols}
%http://www.molitvoslov.org/text226.htm 


Христе Иисусе, Господи и Вседержителю! Ты "--- плачущих утешение, сирых и вдовиц заступление. Ты рекл еси: призови Мя в день скорби твоея, и изму тя. Во дни скорби своея прибегаю к Тебе аз и молюся Ти: не отврати лица Твоего от мене и услыши моление мое, приносимое Тебе со слезами. Ты, Господи Владыко всяческих, благословил еси сочетати мя со единем из рабов Твоих, во еже быти нам едино тело и един дух; Ты дал еси мне сего раба, яко сожителя и защитника. Твоей же благой и премудрой воле изволися отьяти от мене сего раба Твоего и оставити мя едину. Преклоняйся пред сею Твоею волею и к Тебе прибегаю во дни скорби моея: утоли печаль мою о разлучении с рабом Твоим, другом моим. Аще отъял еси его от меня, не отыми от меня Твоея милости. Якоже некогда приял вдовицы две лепты, тако приими и сие моление мое. Помяни, Господи, душу усопшаго раба Твоего \myemph{(имярек)}, прости ему вся согрешения его, вольная и невольная, аще словом, аще делом, аще ведением и неведением, не погуби его со беззаконьми его и не предаждь вечной муки, но по велицей милости Твоей и по множеству щедрот Твоих ослаби и прости вся согрешения его и вчини его со святыми Твоими, идеже несть болезнь, ни печаль, ни воздыхание, но жизнь безконечная. Молю и прошу Тя, Господи, даруй ми во вся дни жизни моея не преставати молитися о усопшем рабе Твоем, и даже до исхода моего просити у Тебе, Судии всего мира, оставления всех согрешений его и вселения его в небесныя обители, яже еси уготовал любящим Тя. Яко аще бо и согреши, но не отступи от Тебе, и несумненно Отца и Сына и Святаго Духа православно даже до последняго своего издыхания исповеда; тем же веру его, яже в Тя, вместо дел ему вмени: яко несть человек, иже жив будет и не согрешит, Ты Един кроме греха, и правда Твоя "--- правда во веки. Верую Господи, и исповедую, яко Ты услышиши моление мое и не отвратиши лица Твоего от мене. Видя вдовицу, зельне плачущу, умилосердився, сына ея, на погребение несома, воскресил еси: тако, умилосердився, утиши и скорбь мою. Яко же отверзл еси рабу Твоему Феофилу, отшедшему к Тебе, двери милосердия Твоего и простил еси ему прегрешения его по молитвам святыя Церкви Твоея внемля молитвам и милостыням супруги его: сице и аз молю Тя, приими и мое моление о рабе Твоем, и введи его в жизнь вечную. Яко Ты еси упование наше, Ты еси Бог, еже миловати и спасати, и Тебе славу возсылаем со Отцем и Святым Духом, ныне и присно и во веки веков. Аминь!

\end{mymulticols}

\mychapterending{}

\mychapter{О мертворожденных младенцах}
%http://www.molitvoslov.org/text222.htm 


Господи, помилуй чад моих, умерших во утробе моей. За веру и слезы мои, ради милосердия Твоего, Господи, не лиши их света Твоего Божественного! 

\mychapterending

\mychapter{Молитва детей за родителей}\begin{mymulticols}
%http://www.molitvoslov.org/text224.htm 


Господи, Иисусе Христе, Боже наш! Ты сирых хранитель, скорбящих прибежище и плачущих утешитель. Прибегаю к Тебе аз, сирый, стеня и плача, и молюся Тебе: услыши моление мое и не отврати лица Твоего от воздыханий сердца моего и от слез очей моих. Молюся Тебе, милосердный Господи, утоли скорбь мою о разлучении с родителем (материю) моим (-ею) \myemph{(имярек)}, душу же его (ея), яко отшедшею (-ия) к Тебе с истинною верою в Тя и твердою надеждою на Твое человеколюбие и милость, приими в Царство Твое Небесное. Преклоняюсь пред Твоею святою волею, еюже отъят (-а) бысть у мене, и прошу Тя, не отыми точию от него (нея или них) милости и благосердия Твоего. Вем, Господи, яко Ты, Судия мира сего, грехи и нечестия отцев наказуеши в детях, внуках и правнуках даже до третьяго и четвертаго рода: но и милуеши отцев за молитвы и добродетели чад их, внуков и правнуков. С сокрушением и умилением сердца молю Тя, милостивый Судие, не наказуй вечным наказанием усопшаго (-ую) незабвеннаго (-ую) для мене раба (-у) Твоего (ю), родителя (матерь) моего (-ю) (имярек), но отпусти ему (ей) вся согрешения его (ея) вольная и невольная, словом и делом, ведением и неведением сотворенная им (ею) в житии его (ея) зде на земли, и по милосердию и человеколюбию Твоему, молитв ради Пречистая Богородицы и всех святых, помилуй его (ю) и вечныя муки избави. Ты, милосердный Отче отцев и чад! Даруй мне, во вся дни жизни моея, до последняго издыхания моего, не преставати памятовати о усопшем родителе (матери) моем (ей) в молитвах своих, и умоляти Тя, праведнаго Судию, да вчиниши его (ю) в месте светле, в месте прохладно и в месте покойне, со всеми святыми, отнюдуже отбеже всяка болезнь, печаль и воздыхание. Милостиве Господи! Приими днесь о рабе Твоем (Твоей) (имярек) теплую молитву мою сию и воздай ему (ей) воздаянием Твоим за труды и попечения воспитания моего в вере и христианском благочестии, яко научившему (-ей) мя первее всего ведети Тя, своего Господа, в благоговении молитися Тебе, на Тебе Единого уповати в бедах, скорбех и болезнех и хранити Заповеди Твоя; за благопопечение его-(ея) о моем духовном преуспеянии, за тепле приносимыя им (ею) о мне моления пред Тобою и за все дары, им (ею) испрошенные мне от Тебе, воздай ему (ей) Своею милостию, Своими небесными благами и радостями в вечном Царствии Твоем. Ты бо еси Бог милостей и щедрот и человеколюбия. Ты покой и радость верных рабов Твоих, и Тебе славу возсылаем со Отцем и Святым Духом, и ныне и присно и во веки веков. Аминь. 

\end{mymulticols}

\mychapterending{}

\mychapter{Молитва за усопших внезапною (скоропостижною) смертию}\begin{mymulticols}
%http://www.molitvoslov.org/text221.htm 


Неисповедимы судьбы Твои, Господи! Неизследимы пути Твои! Даяй дыхание всякой твари и вся от не сущих в бытие приведый, Ты овому посылаеши Ангела смерти в День, егоже не весть, и в час, егоже не чает; оваго же исхищаеши из руки смерти, даруеши живот при последнем издыхании; овому долготерпиши и даеши время на покаяние; оваго же, яко злак, сечеши мечем смерти во едином часе, во мгновении ока; оваго поражаеши громом и молниею, оваго сожигаеши пламенем, оваго же предаеши в снедь зверем дубравниим; овых повелеваеши поглотити волнам и безднам морским и пропастям земным; овых похищаеши язвою губительною, идеже смерть яко жнец пожинает и разлучает отца или матерь от чад их, брата от брата, супруга от супруги, младенца отторгает от лона матерняго, бездыханными повергает сильных земли, богатых и убогих. Что убо сие есть? Дивное и недоумеваемое нами смотрение Твое, Боже! Но Господи, Господи! Ты токмо Един, ведый вся, веси, чесо ради бывает сия и чесо ради быть, яко раб Твоей (раба Твоя) \myemph{\myemph{(имя)}} во едино мгновение ока пожерт(-та) бысть зиянием смерти. Аще наказуеши за многая, тяжкия прегрешения его(ея), молим Тя, Многомилостиве и Всемилостиве Господи, да не яростию Твоею обличиши его (ю) и накажеши вконец, но, по благости Твоей и по безприкладному Твоему милосердию, яви ему (ей) велию милость Твою в прощении и оставлении грехов. Аще ли же усопший (-шая) раб Твой (раба Твоя), в житии сем помышляя День Судный, позна свое окаянство и возжела принести Тебе плоды достойные покаяния, но не достиже сего, зане позван(-на) бысть Тобою в день, егоже не веда, и в час, егоже не жда, того убо ради паче молим Тя, Преблагий и Премилосердый Господи, несоделанное покаяние, еже видесте очи Твои, и неоконченное дело спасения его(ея) исправи, устрой, соверши Твоею неизреченною благостию и человеколюбием; едину бо надежду имамы на Твое безконечное милосердие: у Тебе бо суд и наказание, у Тебе правда и неистощимая милость; Ты наказуеши, вкупе же и милуеши; биеши, вкупе же и приемлеши; прилежно молим Тя, Господи Боже наш, не накажи внезапу позванного(-ую) к Тебе Страшным Судом Твоим, но пощади, пощади его (ю) и не отрини от лица Твоего. О, страшно внезапу впасти в руце Твои, Господи, и предстати на суд Твой нелицеприятный! Страшно отъити к Тебе без благодатнаго напутствия, без Покаяния и причащения Святых Твоих страшных и животворящих Тайн, Господи! Аще внезапу усопший(-шая) приснопоминаемый(-мая) нами раб Твой (раба Твоя) толико многогрешен(-на), толико повинен(-на) есть осуждению на Твоем праведном суде, молим Тя, умилосердися над ним (нею), не осуди его (ю) на вечное мучение, на вечную смерть; потерпи нам еще, даждь нам долготу дний наших, еже молитися Ти по вся дни о усопшем(-шей) рабе Твоем(ей), дондеже услышиши нас и приимеши милостию Твоею внезапу отшедшаго(-ую) к Тебе; и даждь нам, Владыко, омыти грехи его (ея) слезами сокрушения и воздыханиями нашими пред Тобою, да не низведен(-на) будет по грехом своим на место мучения раб Твой (раба Твоя) \myemph{(имя)}, но да вселится в место упокоения. Ты Сам, Господи, повелеваеши ударяти в двери милосердия Твоего, молим Тя убо, Прещедрый Царю, и не престанем умоляти Твое милосердие и взывати с кающимся Давидом: помилуй, помилуй раба Твоего (рабу Твою), Боже, по велицей Твоей милости. Аще же недовольно Ти словес наших, сего малаго моления нашего, умоляем Тя, Господи, верою в спасительныя заслуги Твои, упованием на искупительную и чудодейственную силу Твоея жертвы, принесенныя Тобою за грехи всего мира; молим Тя, о Сладчайший Иисусе! Ты еси Агнец Божий, вземляй грехи мира, распныйся ради нашего спасения! Молим Тя, яко Спасителя и Искупителя нашего, спаси и помилуй и вечныя муки избави душу приснопоминаемаго(-ыя) нами внезапу усопшаго(-шия) раба Твоего (рабы Твоея) \myemph{(имя)}, не остави его (ю) погибнути во веки, но сподоби достигнути тихаго пристанища Твоего и упокой тамо, идеже вси святии Твои упокоеваются. Вкупе же молим Тя, Господи Иисусе Христе Боже наш, приими милостию Твоею и всех внезапу преставльшихся к Тебе рабов Твоих \myemph{(имена)}, ихже вода покры, трус объят, убийцы убиша, огнь попали, град, снег, мраз, голоть и дух бурен умертвиша, гром и молния попали, губительная язва порази, или иною коею виною умреша, по Твоему изволению и попущению, молим Тя, приими их под Твое благоутробие и воскреси их в жизнь вечную, святую и блаженную. Аминь.

\end{mymulticols}

\mychapterending

\mychapter{Об ослаблении вечных мук умерших некрещеными}
%http://www.molitvoslov.org/text236.htm 


\section{Мученику Уару}\begin{mymulticols}

\mysubtitle{Тропарь, глас 4-й:}

Воинством святых страстотерпец страждущие законне, зря онех, показал еси мужески крепость свою. И устремився на страсть волею, и умрети вожделе за Христа, Иже приял еси почесть победы твоего страдания, Уаре, моли спастися душам нашим.

\mysubtitle{Кондак, глас 4:}

Христу последуя, мучениче Уаре, Того испив чашу, и венцем мучения увязеся, и со Ангелы ликовствуеши, моли непрестанно за души наша

\mysubtitle{Молитва:}

О, святый мучениче Уаре досточудный, ревностию по Владыце Христу разжигаемь, Небеснаго Царя пред мучителем исповедал еси, и о Нем усердно пострадал еси, и ныне предстоиши Ему со ангелы, и в вышних ликуеши, и зриши ясно Святую Троицу, и светом Безначальнаго Сияния наслаждаешися, воспомяни и наших сродников томление, умерших в нечестии, приими наше прошение, и якоже Клеопатрин род неверный молитвами твоими от вечных мук свободил еси, тако воспомяни елицы противобожне погребенныя, умершия некрещенными, потщися испросити оным от вечныя тьмы избавление, да вси едиными усты и единем сердцем восхвалим Премилосерднаго Творца во веки веков. Аминь

\end{mymulticols}

\mychapterending

\mychapter{Об ослаблении вечных мук умерших без покаяния}
%http://www.molitvoslov.org/text235.htm 


\section{Преподобному Паисию Великому}\begin{mymulticols}

\mysubtitle{Тропарь, глас 2-й}

Божественною любовию от юности распалаемь, преподобне, вся красная, яже в мире, возненавиде, Христа единаго возлюбил еси, сего ради в пустыню вселися, идеже сподобися Божественнаго посещения, Егоже неудобь зрети и ангельскима очима, паде, поклонися. Великий же Дародатель, яко Человеколюбец рече к тебе: не ужасайся, возлюбленный Мой, дела твоя угодна Мне. Се даю тебе дар: о коем-либо грешнице помолишися, отпустятся ему греси. Ты же в чистоте сердца твоего возгореся, прием воду и прикоснуся Неприкосновенному, умы нозе Его и, воду пив, даром чудес обогатися, болящия исцеляти, бесы от человек отгоняти и грешники от муки молитвою своею избавляти. О преподобне отче Паисие, молю тя, да умолиши и о мне, якоже тебе Бог обетова, ибо от сих грешник первый аз есмь, да даст ми Господь время покаяния и простит мое согрешение, яко Благий и Человеколюбец, да со всеми и аз воспою Ему: аллилуия.

\mysubtitle{Кондак, глас 2-й:}

Житейских молв оставль, безмолвное житие возлюбил еси, Крестителю подобяся всеми образы, с нимже тя почитаем, отче отцев Паисие.

\mysubtitle{Молитва:}

Страстей победителя, душам помощника, о всех молебника, всем спасения ходатая и наставника, из глубины сердца воздыхая, усердно и пламенно молим тя, Паисие Преподобне! Внемли и помози нам, не отринь и не презри нас, но абие услыши в смирении сердца притекающих к тебе. Ты, преподобне, к спасению ближних прилежно стремился и многих грешников к свету спасения привел еси. Подвиги чрезмерные успокоением считал по себе, пречудне, и, любовью ко Господу всегда горя, явления Христа Спаса сподобился еси, и Ему, за людей Умершему, любовию подражая, и об отрекшихся от Христа молился еси. Услыши нас, Паисие прехвальне, ибо недостойны есмы молитися о даровании нам великой милости Господней, понеже грешны есмы, и уста оскверненныя и сердца отягощенныя имеем, и под бременем прегрешений страждем, и не достигает молитва наша до Господа. Сего ради помолися за нас мольбою твоею крепкою и богоприятною, святый Паисие, да избавлены будут скончавшиеся без покаяния сродники, ближние и знаемые наши от муки вечныя, и молитву твою во благоволении приимет Спас наш и милосердие Свое вместо добрых дел их даст им, свободит их, веруем, от страданий и вселит в селениях праведных, и нас в покаянии скончатися удостоит, да прославим вкупе Всесвятое и великолепое имя Отца и Сына и Святаго Духа, во веки веков. Аминь.

\end{mymulticols}

\mychapterending

\mychapter{КАНОН о самовольне живот свой скончавших}
%http://www.molitvoslov.org/text916.htm 


{\centering \myemph{(для келейного чтения)\symbolfootnote[1]{Печатается по изданию Свято-Успенского Псково-Печерского монастыря, 1993.}}

}

\begin{mymulticols}

В 1925 году на Сергиевском Подворье в Париже "--- Рю де Крим, 93 "--- покончил с собой выстрелом один солдат. Он, несомненно, был нездоров умом…

У нас его отпевали…

Но, смущенный смертью его на нашей земле, я обратился через одно лицо к преподобному о. Нектарию Оптинскому с вопросом: что нам делать? Он ответил, чтобы я нашел еще двух человек, которые бы согласились в течение 40 дней читать за него заупокойный канон, а затем предать его в волю Божию.

Разумеется, канон читать нужно было дома, а не за богослужением в храме. Так мы трое тогда и сделали…

После ко мне обращались и другие по этому вопросу; а недавно просили даже отпевать самоубийцу женщину. Я отказывался,"--- согласно канонам "--- правило Тимофея Александрийского "--- и по внутреннему противлению души моей; но помня наставление о.Нектария, а также и указание митрополита Новгородского и Петроградского Григория, что можно молиться о самоубийцах в домашней молитве, но и тогда "--- с предварительною молитвою о помиловании нас молящихся, да не прогневаем Господа,"--- я обратился в каноник "--- «можно и во всякую субботу в Oктоихе найти заупокойные каноны»; но написанный там канон совершенно непригоден был для молитвы о самоубийцах, ибо там говорилось о скончавшихся «благочестно», «в вере», «в упокоении» и прочее,"--- все это не подходило к вольным, «не в сумасшествии», самоубийцам. Поэтому нужно было целиком почти переделать все тропари.

Может, кому-либо пригодится…

А меня да помилует Господь за дерзновение такое…

Многогрешный митрополит Вениамин (Федченков)

\mysubtitle{КАНОН, ГЛАС 2-й} {\centering \myemph{(читается келейно и с благословения духовника)}

}

\mysubtitle{ПЕСНЬ 1}

\irmos{Во глубине постла иногда фараонитское всевоинство преоруженная сила; воплощшееся же Слово всезлобный грех потребило есть: препрославленный Господь, славно бо прославися.}

\pripev{Упокой, Господи, душу усопшаго раба Твоего.}

Всемилостивый Господи, помилуй мя по милости Твоей, пети Тебе начинающа и молитися о убившемся дерзко, да не в суд ми сие будет, ни в осуждение, Судие Праведнейший.

\pripev{Упокой, Господи, душу усопшаго раба Твоего.}

Да не будет молитва моя во грех, яко кроткий Давид глаголет: ниже во отягощение души самовольней смерти предавшемуся; но мученик мольбами оставление прегрешений наших даруй нам.

\slava

Богатую милость присно всем источая, Всеблагий Христе Боже, и ныне рабу Твоему \myemph{(имя)}, нами Тебе молящемуся, помилование даруй и с преставльшимися во благочестии причти, аще и нечестиво скончался в неразумии.

\inyne

Утвердивши колеблемый ум мой, Мати Божия, укрепи мя Божественными повеленьми Рождшагося из Твоего освященного чрева и упразднившего, Владычице, адово мрачное царство, со дерзновением неосужденно молитися об усопшем \myemph{(имя)}. 

\mysubtitle{ПЕСНЬ 3}

\irmos{На камени мя веры утвердив, разширил еси уста моя на враги моя: возвесели бо ся дух мой, внегда пети: несть свят, якоже Бог наш, и несть праведен паче Тебе, Господи.}

\pripev{Упокой, Господи, душу усопшаго раба Твоего.}

Твоея красоты сиянием озаритися усопшему благоволи, в вере с почившими, богатый в милости; Ты бо еси Бог наш Единый безгрешный и многомилостивый.

\pripev{Упокой, Господи, душу усопшаго раба Твоего.}

На месте прохлаждения в вечнем покоище сподоби прияти и раба Твоего, малодушие жизнь свою пресекшаго: Ты бо еси Един Сильный и немощи наша на Ся восприявый, Господи Боже наш.

\slava

В чертозе небеснем водворитися, Владыко, приими раба Твоего \myemph{(имя рек)}, да не услышит он праведнаго Твоего гласа: «не вем тя», аще и воистину повинен есть сему.

\inyne

Умерщвленных нас воздвигла еси, Богородице, жизни нетленней Жизнодавца рождши; и раба Твоего \myemph{(имя)} возведи из ада преисподнейшаго, Сына Твоего о милости умоляющи. 

\mysubtitle{ПЕСНЬ 4}

\irmos{Пою Тя, слухом бо. Господа услышах и ужасохся: до мене бо идеши, мене ища заблуждшаго. Тем многое Твое снизхождение, еже на мя, прославляю, Многомилостиве.}

\pripev{Упокой, Господи, душу усопшаго раба Твоего.}

Во уповании на Тя, Боже, изнемогшего, и насилием вражиим одоленнаго, раба Твоего пощади, Спасе презирая, яко Благ, прегрешения его вольная и невольная.

\pripev{Упокой, Господи, душу усопшаго раба Твоего.}

Милосердия имый неизследимую бездну, Господи, бездну греховную раба Твоего покрый, Человеколюбче,"--- омыв Кровию Своею грех неведения и маловерия его.

\slava

Живыми господствуя и мертвыми владычествуя, помилуй и раба Твоего \myemph{(имя)}, власть бо имаши всякую на небеси и на земли, Христе Боже, Отцу соседый и Духом вся очищаяй.

\inyne

Исцелила еси, Богородительнице, Евино поползновение окаянное; Зиждителя бо родила еси, исправити могущаго тварь погибающую: Того моли и сего раба \myemph{(имя)} неключимаго, заповедь Божию преступившего, спасти по единой милости. 

\mysubtitle{ПЕСНЬ 5}

\irmos{Света подателю и веков Творче, Господи, во свете Твоих повелений настави нас: разве бо Тебе иного Бога не знаем.}

\pripev{Упокой, Господи, душу усопшаго раба Твоего.}

Умертвишася и во дно адово поползнувшагося возведи, Боже, преступление того заглаждая; на Кресте бо грехи наша пришел еси омыти честною Кровию Своею, Христе, Искупителю всяческих.

\pripev{Упокой, Господи, душу усопшаго раба Твоего.}

Спасти нас пришедый, Христе Боже, приими ныне и наша недостойная моления о рабе Твоем \myemph{(имя)}, да муки вечныя избавиши, помилован будет с нами, Спасителю Преблагий.

\slava

Молим Тя, Слове Божий, аще и дерзновенно сие, прогнева бо Тя раб Твой \myemph{(имя)}, обаче благодатию Твоею умоляем, не возгнушайся молитв наших и помилуй нас, вкупе вси бо повинны есмы пред Тобою, Всесвятый Владыко.

\inyne

Под кровом Твоим, Мати Божия, спасаются вси надеющийся на Тя: Твоим заступлением защити и нас, и сподоби со дерзновением неосужденно молити Сына Твоего о живот свой скончавших безнадежно \myemph{(имя)}. 

\mysubtitle{ПЕСНЬ 6}

\irmos{В бездне греховней валяяся, неизследную милосердия Твоего призываю бездну: от тли, Боже, мя возведи.}

\pripev{Упокой, Господи, душу усопшаго раба Твоего.}

Из земли создавши человека, Боже, и в землю ныне возвратишася, пощади, Господи, в неразумии сокрушившаго скудельный сосуд свой: можеши бо, Владыко Творче, и паки возставити его от истления.

\pripev{Упокой, Господи, душу усопшаго раба Твоего.}

Неизреченный и непостижимый, Господи, щедротами несказаннаго Человеколюбия Твоего помилуй живот свой скончавшаго, презирая согрешения его, яже в ведении и неведении содеянная.

\slava

Озаритися милостию Твоею сподоби, Владыко, вольне оставльшаго \myemph{(имя)} житие свое и к свету Твоему возведи, из тьмы несветимыя изымая, милосердия ради.

\inyne

Избавительнице, усердно призывающим Тя, Пречистая Владычице, Сына Твоего умоли скончавшагося \myemph{(имя)} ущедрити: живота и смерти Владыку родила еси. 

\mysubtitle{ПЕСНЬ 7}

\irmos{Богопротивное веление беззаконнующаго мучителя высок пламень вознесло есть; Христос же простре благочестивым отроком росу духовную: Сый благословен и препрославлен.}

\pripev{Упокой, Господи, душу усопшаго раба Твоего.}

Во ад с душею, яко Бог, сошедый, и в рай разбойника с Собою введый, помяни и ныне раба Твоего \myemph{(имя)}, преступлением многим Тя прогневавша, ныне же Тебе нами молящася бедне.

\pripev{Упокой, Господи, душу усопшаго раба Твоего.}

Един безгрешный, в мертвых волею вменился еси, да умерщвленных силою человекоубийцы воскресивши с Собою, от державы того исхити и раба Твоего, Всесильне.

\slava

Радость присносущную даровати нам Воскресением Своим изволивый, от мрачнаго горькаго царства изми раба Твоего \myemph{(имя)}, да от муки вечныя свободившися, узрит себе одесную Тебе, Судие Многомилостивый.

\inyne

Сияние славы Отчей, плотию от Тебе явившегося умоли, Пресвятая, из адовы мглы свободити непотребнаго раба Твоего \myemph{(имя)}, той бо яко Матерь, Тебе нами просит. 

\vspace{-\baselineskip}\mysubtitle{ПЕСНЬ 8}

\irmos{Пещь иногда огненная в Вавилоне действа разделяше, Божиим велением халдеи опаляющая, верныя же орошающая, поющия: Благословите, вся дела Господня, Господа.}

\pripev{Упокой, Господи, душу усопшаго раба Твоего.}

Смертное царство Своею смертию разоривый, Божественною же силою вход в вечный живот отверзый, не затвори благоутробия Твоего и рабу Твоему, аще и прегрешившему, обаче из ада ныне горце взывающа.

\pripev{Упокой, Господи, душу усопшаго раба Твоего.}

Красоты Твоея насладитися сподоби, студ греха безобразнаго очищая умершаго смертию поносной: Един бо еси, Владыко, греха свободен, крестною смертию за нас умертвился еси.

\slava

Исправити хотя падших в персть смертную, Сам смерти причастен явился еси, Безсмертный, от смертных удолий силою Твоею возведи раба Твоего \myemph{(имя)}, лютою смертию поползнувшагося.

\inyne

Велия воистину есть Рождества Твоего тайна, Богомати: Бога воплотившагося родила еси; Его же моли, яко Сына Твоего, спасти бедне скончавшагося \myemph{(имя)}: имаши бо дерзновение к Рождшемуся от Тебя Благоутробному Спасу мира. 

\vspace{-\baselineskip}\mysubtitle{ПЕСНЬ 9}

\irmos{Яже прежде солнца Светильника Бога возсиявшаго, плотски к нам пришедшаго, из боку девичу неизреченно воплотившая, Благословенная, Всечистая, Тя, Богородице, величаем.}

\pripev{Упокой, Господи, душу усопшаго раба Твоего.}

Имеяй власть живых и мертвых, пощади, Спасе, раба Твоего \myemph{(имя)}, смертию горькою скончавшагося: Тебе бо от Отца Твоего дадеся живити и миловати, ихже хощеши, Сыне Божий.

\pripev{Упокой, Господи, душу усопшаго раба Твоего.}

Преславное множество святых Твоих, Христе, непрестанно молит Тя за грешныя рабы Твоя, да помилуеши ихже волиши милосердием Твоим и геенны вечныя избавиши.
%\end{mymulticols}

%\newpage

%\begin{mymulticols}

\slava

Благости воистину пучина неизследимая, благодатию Твоею помилуй прогневавшаго Тя раба Твоего \myemph{(имя)}, и нас молящихся.

\inyne

Достойно Тя, Богородице, ублажающе, к Тебе, яко Матери нашей, притекаем надежно; сами себе и брата нашего \myemph{(имя)}, малодушие скончавшагося, Тебе Заступнице и с Тобою Христу Богу предаем смиренно.

\end{mymulticols}

\mychapterending

\mychapter{Молитва за благодетелей, особенно руководивших к добродетели}\begin{mymulticols}
%http://www.molitvoslov.org/text197.htm 


Господи Иисусе Христе Боже наш! Ты заповедал еси нам воздавати всем должное, егда рекл еси: «Якоже хощете да творят вам человецы, и вы творите им такожде». Что убо сотворю, Господи, что воздам ныне, во мнозе благотворившему(-шей) мне, но уже отшедшему (-шей) от земли живых рабу Твоему (рабе Твоей)! У Тебе, Господи, у Тебе, богатый в милостех Боже наш, всякое воздаяние есть. Сего ради молю Тя, Подателю благих, приими от мене сие малое моление мое, вкупе же и благодарение за все благодеяния Твои, дарованныя мне в приснопоминаемом(-мой) усопшем(-шей) рабе Твоем (Твоей). Молю Тя, Боже, Спасителю мой призри с высоты Святаго жилища Твоего на молитву мою и помяни милостию Твоею отшедшаго(-ую) к Тебе раба Твоего (рабу Твою) (имя), толико благодеявшаго(-ую) ми; яви ему (ей) силу благости Твоей и неизреченное Твое милосердие в прощении и оставлении грехов и покрый его (ю) оправданием Твоим; подаждь ему, Господи, Царство Небесное и вечный покой, яко помогавшему(-шей) ми и ближним моим в нуждех и лишениях, в скорбех и болезнех и подававшему (-шей) ми благая от избытков даров Твоих. Молюся убо Тебе, Благость вечная, воздай ему (ей) за благотворения его (ея) сторицею "--- за земныя блага небесными, за временныя вечными, и подаждь ему (ей) вечный покой; и яко научавшему(-шей) мя делам милосердия, кротости, смиренномудрию, терпению, воздержанию, вселявшему (-шей) в мя дух мира и любви христианской, подававшему (-шей) ми благие советы и благие образы христианскаго жития, даруй, Премилосердый Господи, достойно узрети невечерний свет Твоего Царствия и насыти его (ю) от небесныя трапезы Твоея, яко во имя Твое препитавшаго(-ую) жаждущую и алчущую мою душу. Помяни, Господи, и всех отшедших к Тебе, в вере и добродетельном житии подвизавшихся во славу Пресвятаго Имени Твоего, добро творивших сирым, нищим и убогим; упокой в лоне Авраама, Исаака, Иакова, посещавших недугующих, страждущих и в темнице заключенных; приими в чертог Твой принимавших в дому своем странна и пришельца. Помяни, Господи, во Царствии Твоем усопших рабов Твоих, выну предстоявших и молившихся Тебе теплою молитвою о нашем душевном спасении и благоденствии, о сохранении и благостоянии святых Твоих церквей, о мире и спасении всего мира; упокой со святыми Твоими рабов Твоих, молившихся о обращении к Тебе заблудших и отпадших от веры и о спасении погибающих во грехах. Молю Тя, Христе Боже наш, молю Тя, яко Судию, имущаго судити мирови, воздай им по человеколюбию Твоему, и сподоби их услышати вожделенный оный глас: «Приидите, благословеннии Отца Моего, и наследуйте уготованное вам Царствие от сложения мира», и да внидут в онь, хваля и прославляя Твое великолепое имя в безконечные веки. Аминь. 

\end{mymulticols}

\mychapterending

\mychapter{Молитва за скончавшихся вне своего Отечества, за безродных и убогих}\begin{mymulticols}
%http://www.molitvoslov.org/text217.htm 


Господи Иисусе Христе Боже наш, сирых Защитниче, странных Наставниче, плавающих Кормчий, обуреваемых пристанище, с путешествующими путешествующий, утешение плачущих, нищих кормителю! Молим Тя, Многомилостиве Господи, помяни во Царствии Твоем души усопших раб Твоих \myemph{(имена)}, в вере преставльшихся к Тебе, но отшедших в земли чуждей, без матерняго Святыя Твоея Церкви напутствия и благословения. Приими, Владыко, под Твое благоутробие рабов преставльшихся тамо в скорби и печали о своих присных и ближних, имже никтоже бе утешаяй в страшный час смерти, идеже никтоже бе возносяй Тебе теплое моление о благотишном прохождении их от сея привременныя жизни в мир горний. Темже убо молим Тя, Преблагий, молим Тя, вездесущий и присносущный Господи, воззри на сих рабов Твоих благосердием Твоим, прости им всякое согрешение вольное и невольное, словом и делом содеянное и вся, яже в ведении и неведении. Умоляем Тя молитвами и предстательством Матере Твоея, Пресвятая Владычицы нашея Богородицы, и всех святых, не погуби отшедших к Тебе со беззаконием их, но сотвори им велию милость Твою, яви им Твое человеколюбие и щедроты Твоя, приведи их к тихому Твоему пристанищу и водвори их в Дому Твоем. 

Еще молимся о упокоении душ усопших раб Твоих (души усопшаго (-шия) раба Твоего (рабы Твоея) \myemph{(имена)}, отшедших к Тебе безродными и никогоже от присных своих о себе молитвенников имущими, разве Святыя Твоея Церкви; сего убо ради молим Тя, Господи, приими и о сих наше моление: прости им вся согрешения и помилуй их по велицей Твоей милости; всели их во Святом жилище Твоем, да обрящут в Тебе покой, яко в Спасителе и Искупителе рода человескаго. 

Господи! Господи! Аще усопший раби Твои, иже быша в нищете и убожестве, за чашу студеныя воды, за укрух хлеба, за едину лепту, поданныя им во имя Твое, приносили быша Тебе моления о нас, прошения и благодарения; аше сирые благословляли быша благотворящую им руку, моляся о ниспослании нам благ Твоих, того убо ради молим Тя ныне, Христе Боже наш, и мы о них: помяни убо во Царствии Твоем сих, яко меньших, по реченному Тобою, братий Твоих; прости им всякое прегрешение вольное и невольное, покрый их Твоим благоутробием и всели их во Святый чертог Твой; и яко получавших зде на земли пропитание чрез призывание святаго Твоего имени, препитай тамо на небеси от тука дома Твоего и да утешатся, и да насладятся нескончаемою радостию вси истинно плакавшие к Тебе; водвори их со святыми Твоими, да вси купно у Тебе, Судие живых и мертвых, составят сонмы избранных Твоих, и тако да славят, хвалят и превозносят безпредельное Твое милосердие в безпредельные веки. Аминь. 

\end{mymulticols}

\mychapterending

\mychapter{Молитва за всякого усопшего}\begin{mymulticols}
%http://www.molitvoslov.org/text196.htm 

Помяни, Господи Боже наш, в вере и надежди живота вечнаго преставльшагося раба Твоего, брата нашего \myemph{(имя)}, яко Благ и Человеколюбец, отпущаяй грехи и потребляяй неправды, ослаби, остави и прости вся вольная его согрешения и невольная, избави его вечныя муки и огня геенскаго, и даруй ему причастие и наслаждение вечных Твоих благих, уготованных любящым Тя: аще бо и согреши, но не отступи от Тебе, и несумненно во Отца и Сына и Святаго Духа, Бога Тя в Троице славимаго, верова, и Единицу в Троице и Троицу в Единстве православно даже до последняго своего издыхания исповеда. Темже милостив тому буди, и веру яже в Тя вместо дел вмени, и со святыми Твоими яко Щедр упокой: несть бо человека, иже поживет и не согрешит. Но Ты Един еси кроме всякаго греха, и правда Твоя правда во веки, и Ты еси Един Бог милостей и щедрот, и человеколюбия, и Тебе славу возсылаем, Отцу и Сыну и Святому Духу, ныне и присно и во веки веков. Аминь. 

\end{mymulticols}

\mychapterending

\mychapterz{Молитва за обидевших и ненавидевших нас}{\LargeМолитва за обидевших и ненавидевших нас}
%http://www.molitvoslov.org/text200.htm 

\vspace{-\baselineskip}

\begin{mymulticols}

Владыко Человеколюбче Господи, Иисусе Христе, Сыне Божий! Ты, по неизреченной любви Своей к нам, грешным и недостойным рабам Твоим, сияеши солнце Свое на злыя и благая, дождиши на праведныя и неправедныя; Ты, Преблагий, заповедуеши нам любити врагов наших, добро творити ненавидевшим и обидевшим нас, благославляти клянущих нас и молитися за творящих нам напасть и изгонящих нас. Ты, Спасителю наш, вися на крестном древе, и Сам прощал еси врагов Своих, хульно ругавшихся Тебе, и молился за мучителей Твоих; Ты дал еси образ нам, да последуем стопам Твоим. Ты, о дражайший Искупителю наш, научивый нас прощати врагов, повелел еси вкупе и молитися за них; молюся убо Тебе, Иисусе Прещедрый, Сыне и Агнче Божий, вземляй грехи мира, прости отшедшаго (-ую) к Тебе раба твоего (рабу Твою) \myemph{(имя)} и приими его (ю) не яко врага моего, соделавшаго ми злая, но яко согрешившаго (-ую) пред Тобою, Молю Тя, безпредельный в милостех, Господи Боже наш, приими с миром, пресельшагося (-уюся) к Тебе от мира сего без примирения со мною; спаси и помилуй его (ю), Боже, Твоею великою и богатою милостию. Господи, Господи! Да не яростию Твоею, ниже гневом Твоим накажеши раба Твоего (рабу Твою), творившаго ми напасть, оскорбление, поношение и злохуление; молю Тя, не помяни сих прегрешений его (ея), но отпусти и прости ему (ей) вся сия по человеколюбию Твоему, и помилуй по велицей Твоей милости. Молю Тя, о Преблагий и Прещедрый Иисусе, яко уз адовых Решителя, смерти Победителя, грешных Спасителя, разреши рабу Твоему (рабе Твоей) сии согрешения, имиже, яко пленицами ада, связася усопший (-шая). Ты бо, Господи, рекл еси: «аще не отпущаете человеком согрешения их, ни Отец ваш Небесный отпустит вам согрешений ваших»; о, да не будет сего! Со умилением и сокрушением сердца умоляю Тя, Спасителю Премилостивый, разреши ему (ей) сии узы злых наваждений и козней диавольских, не погуби усопшаго (-ую) гневом Твоим, но отверзи ему (ей), Жизнодавче, двери милосердия Твоего, да внидет во святый Твой град, хваля всесвятое и великолепое Имя Твое и воспевая неизреченную любовь Духа Твоего Святаго к погибающим грешникам. И якоже Ты, Благость вечная, помянул еси на кресте благоразумнаго разбойника, с Тобою распята, путесотворив ему вход в рай, сице, молю Тя, Всещедрый, помяни во Царствии Твоем и отшедшаго (-ую) к Тебе раба Твоего (рабу Твою) \myemph{(имя)} не затвори, но отверзи и ему (ей) двери милосердия Твоего, Твое бо есть, еже миловати и спасти ны, Боже наш, и Тебе славу возсылаем со Безначальным Твоим Отцем, Пресвятым и Благим и Животворящим Твоим Духом, ныне и присно и во веки веков. Аминь. 

\end{mymulticols}

\mychapterending

\mychapter{Молитва за умирающего}\begin{mymulticols}
%http://www.molitvoslov.org/text195.htm 


Господи, Иисусе Христе Сыне Божий, заступи, спаси, помилуй и сохрани Боже, Твоею благодатию душу раба Твоего \myemph{(имя)}, и грехи юности и неведения его не помяни, и даруй ему кончину христианску, непостыдну и мирну, и да не узрит душа его мрачного взора лукавых демонов, да приимут его Ангели Твои светлии и пресветлии, и на Страшном Суде Твоем милостив ему буди, ибо Твое есть единого Господа, еже миловати и спасати нас. 

\end{mymulticols}

\mychapterending

\mychapter{Моление о упокоении православных воинов, за веру и Отечество на брани убиенных}\begin{mymulticols}
%http://www.molitvoslov.org/text220.htm 


Непобедимый, непостижимый и крепкий во бранех Господи Боже наш! Ты, по неисповедимым судьбам Твоим, овому посылаеши Ангела смерти под кровом его, овому на селе, овому на мори, овомуже на поле брани от оружий бранных, изрыгающих страшныя и смертоносныя силы, разрушающия телеса, расторгающия члены и сокрушающия кости ратующих; веруем, яко по Твоему, Господи, премудрому смотрению, такову приемлют смерть защитники веры и Отечества. 

Молим Тя, Преблагий Господи, помяни во Царствии Твоем православных воинов, на брани убиенных, и приими их в небесный чертог Твой, яко мучеников изъязвленных, обагренных своею кровию, яко пострадавших за Святую Церковь Твою и за Отечество, еже благословил еси, яко достояние Твое. Молим Тя, приими убо отшедших к Тебе воинов в сонмы воев Небесных Сил, приими их милостию Твоею, яко павших во брани за независимость земли Русския от ига неверных, яко защищавших от врагов веру православную, защищавших Отечество в тяжкие годины от иноплеменных полчищ; помяни, Господи, и всех, добрым подвигом подвизавшихся за древнехранимое Апостольское Православие, за освященную и в язык свят избранную Тобою землю Русскую, в нюже враги Креста и Православия приношаху и огнь, и меч. Приими с миром души раб Твоих \myemph{(имена)}, воинствовавших за благоденствие наше, за мир и покой наш, и подаждь им вечное упокоение, яко спасавшим грады и веси и ограждавшим собою Отечество, и помилуй павших на брани православных воинов Твоим милосердием, прости им вся согрешения, в житии сем содеянная словом, делом, ведением и неведением. Призри благосердием Твоим, о Премилосердый Господи, на раны их, мучения, стенания и страдания, и вмени им вся сия в подвиг добрый и Тебе благоугодный; приими их милостию Твоею, зане лютыя скорби и тяготу зде прияша, в нуждех, тесноте, в трудех и бдениих быша, глад и жажду, изнурение и изнеможение претерпеша, вменяеми быша яко овцы заколения. Молим Тя, Господи, да будут раны их врачеством и елеем, возлиянным на греховныя язвы их. Призри с небесе, Боже, и виждь слезы сирых, лишившихся отцев своих, и приими умиленныя о них мольбы сынов и дщерей их; услыши молитвенныя воздыхания отцев и матерей, лишившихся чад своих; услыши, благоутробне Господи, неутешных вдовиц, лишившихся супругов своих; братий и сестер, плачущих о своих присных,"--- и помяни мужей, убиенных в крепости сил и во цвете лет, старцев, в силе духа и мужества; воззри на сердечныя скорби наша, виждь сетование наше и умилосердися, Преблагий, к молящимся Тебе, Господи! Ты отъял еси от нас присных наших, но не лиши нас Твоея милости: услыши молитву нашу и приими милостивно отшедших к Тебе приснопоминаемых нами рабов Твоих \myemph{(имена)}; воззови их в чертог Твой, яко доблих воинов, положивших живот свой за веру и Отечество на полях сражений; приими их в сонмы избранных Твоих, яко послуживших Тебе верою и правдою, и упокой их во Царствии Твоем, яко мучеников, отшедших к Тебе израненными, изъязвленными и в страшных мучениях предававшими дух свой; всели во святый Твой град всех приснопоминаемых нами рабов Твоих \myemph{(имена)}, яко воинов доблих, мужественно подвизавшихся в страшных приснопамятных нам бранех; облецы их тамо в виссон светел и чист, яко зде убеливших ризы своя в крови своей, и венцев мученических сподоби; сотвори их купно участниками в торжестве и славе победителей, ратоборствовавших под знаменем Креста Твоего с миром, плотию и диаволом; водвори их в сонме славных страстотерпцев, добропобедных мучеников, праведных и всех святых Твоих. Аминь. 

\end{mymulticols}

\mychapterending

\mychapter{Молитва за наставников и воспитателей}\begin{mymulticols}
%http://www.molitvoslov.org/text199.htm 


Боже разума, Боже чудес, живый, приснотекущий Источниче премудрости! Молю Тя, помяни во Царствии Твоем души усопших раб Твоих, наставников моих \myemph{(имена)}, яко просвещавших мой ум и вселявших в сердце мое дух премудрости и разума, дух совета и крепости, дух ведения и правды, истины и добродетели. Молю Тя, Владыко, милостив буди о гресех их; ослаби, прости и отпусти им вся согрешения, елика сотворили в житии сем, словом, делом, помышлением, ведением и неведением. Воздаждь им, света Подателю, милостию Твоею, щедротами Твоими за вся попечения их о нашем духовном преуспеянии; яви им милость благоутробия Твоего за благая их поучения, за благопотребные их советы и вразумления, яже посеяваху и насаждаху во умах и сердцах наших, яко благия семена; воздаждь им за вся благая небесными и вечными Твоими благами, да узрят в Тебе, Боже, лицем к лицу вечную Истину и да озаряются и наслаждаются трисиятельным Светом Ея в безконечные веки во Царствии Твоем. Аминь. 

\end{mymulticols}

\mychapterending

\mychapter{Молитва четвертая за всех в вере скончавшихся православных христиан}\begin{mymulticols}
%http://www.molitvoslov.org/text194.htm 


Господи, Господи! коль неизбежен, коль страшен нелицеприятный и неизменимо-вечный Суд Твой! В великом и притрепетном смущении дух наш мятется, сердце содрогается и истаевает, яко воск, от единаго возвещения Твоего непреложнаго глагола о последнем Суде; немотствует язык от единаго ожидания гласа последней трубы Архангела, имущей возбудити от мертвых и воззвати от живых на Страшный Суд Твой от востока и запада, от севера и юга. О, воистину страшен день той, в оньже приидеши, Боже, на землю во славе со Ангелами и всеми святыми! Горнии и дольнии все предстанут тогда со страхом и трепетом и полчища преисподних вострепещут пред Тобою, Судие мира сего; небо и землю призовеши на Суд, судити вселенней в правде и истине! Страшен день той, в оньже престоли поставятся, книги разгнутся, и явная и тайная наша деяния, словеса и помышления обличатся пред Тобою, пред Ангелами и человеками, вкупе же и осудятся: ничтоже бо, Господи, у Тебе сокровенно, еже тогда не открыется, и тайно, еже не уведано будет. Кто возможет, Судие мира сего, устояти тогда пред лицем Твоим и целым миром горним и дольним! Аще праведник едва спасется, нечестивый и грешный како постоит? И кто убо заступит за ны, аще не Твое милосердие, Господи? Камо убежим от праведнаго Суда Твоего? О, пощади, пощади тогда нас и отшедщаго (-ую) к Тебе приснопоминаемаго (-ую) нами раба Твоего (рабу Твою) \myemph{(имя)} и не осуди его (ю) тогда на вечное мучение по грехом его (ея). Благоволи убо, Премилосердый Господи, прияти в небесный и мысленный Твой жертвенник наши жертвы за него (ню) "--- молитвы наши и милостыни, яко кадило благовонное, вкупе с молитвами и безкровным священнодействием пастырей Святая Твоея Церкви, да не постыдится пред Тобою, пред Ангелами и целым миром усопший (-шая) раб Твой (раба Твоя). Услыши ны, Боже наш, и не отрини до конца; виждь коленопреклоненных, повергшихся долу пред Тобою, Господи; услыши просящих Твоея милости и Царства Небеснаго усопшему (-шей) рабу Твоему (рабе Твоей). Молим Тя, о Премилосердый и Прещедрый Иисусе, егда приидеши со Ангелы Святыми Твоими судити мирови, пощади, пощади тогда. Спасе, создание Твое: образ бо есть неизреченныя Твоея славы. С сокрушенным сердцем молим, просим Тя, Спасителю и Искупителю наш, не осуди раба Твоего (рабу Твою) \myemph{(имя)} праведным судом Твоим на вечное мучение, аще достоин (-на) есть всякаго осуждения и муки; не отлучи его (ю) от стада избранных Твоих, но, по неизреченному милосердию Твоему, долготерпению и любви к нам грешным и недостойным рабам Твоим, ради Твоих дражайших заслуг сподоби его (ю), Владыко, Царствия Твоего, еже уготовал еси любящим Тя от сложения мира, и да внидет в место упокоения, хваля всесвятое и великолепое Имя Твое, славя неизреченное милосердие Твое, величая человеколюбие и щедроты Твоя, яко Ты еси Бог милости и щедрот, и Тебе славу возсылаем, со безначальным Твоим Отцем, Пресвятым и Благим и Животворящим Твоим Духом, ныне и присно и во веки веков. Аминь. 

\end{mymulticols}

\mychapterending

\mychapter{Молитва о упокоении скончавшихся после тяжкой и продолжительной болезни}\begin{mymulticols}
%http://www.molitvoslov.org/text219.htm 


Господи, Господи! Праведен еси Ты, и суд Твой праведен: Ты, в предвечной Своей Премудрости, положил еси предел жизни нашея, егоже никтоже прейдет. Премудры Твои законы, неизследимы путие Твои! Ты повелеваеши ангелу смерти изъяти от тела душу у младенца и старца, у мужа и юноши, у здраваго и больнаго по несказанным и недоведомым нам судьбам Твоим; но веруем, яко на сие есть Твоя святая воля, зане, по суду правды Твоея, Ты, Преблагий Господи, яко премудрый и всемогущий и всеведущий Врач душ и телес наших, болезни и недуги, беды и злоключения посылаеши человеку, яко врачевство духовное. Ты поражаеши его и исцеляеши, умерщвляеши в нем мертвенное и оживотворяеши безсмертное, и, яко чадолюбивый Отец, наказуеши, егоже приемлеши: молим Тя, Человеколюбче Господи, приими преставльшагося к Тебе раба Твоего (рабу Твою) \myemph{(имя)}, егоже (юже) взыскал еси человеколюбием Своим, наказавый тяжкою болезнию телесною, во еже спасти душу от болезни смертныя; и аще вся сия приял(-ла) есть от Тебе со смиренномудрием, терпением и любовию к Тебе, яко вседетельному Врачу душ и телес наших, яви убо ему (ей) днесь богатую милость Твою, яко претерпевшему(-ей) вся сия грех своих ради. Вмени ему (ей), Господи, сию привременную тяжкую болезнь в некое наказание за прегрешения, содеянныя в сей юдоли плача, и исцели душу его (ея) от недугов греховных. Помилуй, Господи, помилуй взысканнаго(-ую) Тобою, и, наказаннаго(-ую) временно, молю Тя, не накажи лишением вечных Твоих небесных благ, но сподоби его (ю) наслаждатися ими во Царствии Твоем. Аще же усопший(-шая) раб Твой (раба Твоя), не разсуждая в себе, чесо ради бысть сие прикосновение целительныя и промыслительныя Десницы Твоея, стропотно что рече в себе, или, по неразумию своему, возропта в сердце своем, яко бремя сие возмни себе неудобоносимым, или, по немощи естества своего, стужаше долгою болезнию и огорчися злоключением, молим Тя, Долготерпеливе и Многомилостиве Господи, прости ему (ей) сия согрешения по безпредельной Твоей милости и безприкладному Твоему милосердию к нам грешным и недостойным рабам Твоим, прости ради любви Твоей к роду человеческому; аще ли же беззакония его (ея) превзыдоша главу его (ея), болезнь же и недуги не подвигоша его (ю) к полному и искреннему покаянию, умоляем Тя, Начальниче жизни нашея, умоляем Твоими искупительными заслугами, помилуй и спаси, Спасе, раба Твоего (рабу Твою) от вечныя смерти. Господи Боже, Спасителю наш! Ты, по вере в Тя, даровал еси прощение и отпущение грехов, даровав помилование и исцеление тридесятиосмилетнему разслабленному, егда рекл еси: «Отпущаются тебе греси твои»; с сею же верою и упованием на Твою благость, прибегаем к Твоему, о Иисусе Прещедрый, неизреченному милосердию и во умилении сердца нашего молим Тя, Господи: рцы убо и днесь тако слово помилования, слово отпущения грехов усопшему(-шей), приснопоминаемому(-мой) нами рабу Твоему (рабе Твоей) \myemph{(имя)}, да исцелеет духовне, и да вселится в месте светле, в месте покойне, идеже несть болезнь, ни печаль, ни воздыхание, и да пременятся тамо болезни и недуги его (ея), слезы страданий и скорбей в источник радости о Дусе Святе. Аминь. 

\end{mymulticols}

\mychapterending

\mychapterz{Молитва за пастырей Православной Церкви}{\LargeМолитва за пастырей Православной Церкви}
%http://www.molitvoslov.org/text198.htm 

\vspace{-\baselineskip}

\begin{mymulticols}

Господи Иисусе Христе Боже наш, давый нам овы апостолы, овы же пророки, овы же благовестники, овы же пастыри и учители к совершению Святых, в дело служения, в созидение Тела Твоего! Молим Тя, Владыко Многомилостивый, услыши молитвенный глас наш и помяни во Царствии Твоем преставльшихся из дольняго мира сего в мир горний рабов Твоих: православных Патриархов, митрополитов \myemph{(имена)}, архиепископов и епископов \myemph{(имена)}, архимандритов, игуменов, иеромонахов, иереев \myemph{(имена)}, иеродиаконов и диаконов \myemph{(имена)}; водвори их, Господи, в небесные кровы вечнаго покоя. Яко послуживших ко славе Пресвятаго имене Твоего и во славу Святая Соборныя и Апостольския Церкви Твоея, к назиданию и спасению чад Ея; помяни и всех в монашеском чине и причте церковном Тебе послуживших. Молим Тя, Господи Иисусе Христе Боже наш, молим яко Пастыреначальника, приими милостию Твоею души усопших раб Твоих, яко избранных слуг и строителей Тайн Твоих, предстоявших страшному престолу Твоему, совершавших безкровное священнодействие и подававших нам Святые и Животворящие Дары Твои "--- Честное Твое Тело и Честную Кровь Твою, во оставление грехов и в жизнь вечную; того ради молим Тя, Боже наш, соделай и их причастниками благодатных даров Духа Твоего Святаго во Царствии Твоем. Господи! Ты дал еси им власть во имя Твое прощати и отпущати нам грехи, вязати и решити, прилежно молим Тя: прости, отпусти и им согрешения их и не помяни грехопадений их; спаси и помилуй их по велицей Твоей милости; призри на них благосердием Твоим и приими их, Христе, под Твое благоутробие, яко учивших ны блюсти вся, елика заповедал еси нам; со святыми упокой в дому Твоем, яко домостроителей, смиренно подвизавшихся в насаждении и укреплении Православныя Веры и благочестия в сердцах наших; сподоби их неосужденно предстати пред Тобою на судищи Твоем, яко пред Пастыреначальником, и всели их в небесный Твой вертоград, яко насаждения масличная. Приведи их, Светодавче Господи, к невечернему свету Царствия Твоего, яко просвещавших ны светом Евангельским, всели их во Святый Твой град, яко вселявших в нас дух премудрости и разума, дух совета и крепости, дух ведения и благочестия, дух страха Твоего,"--- и даждь им наслаждатися от тука дома Твоего, яко питавших и наслаждавших души наша от божественныя Трапезы Твоея. Молим Тя, Царю, молитвами апостолов и пророков, святителей и учителей и всех святых, прослави их, яко служителей словесе во Святой Церкви Твоей, воинствующей на земли, прослави в Церкви, торжествующей на Небеси, и сопричти их к лику святых Твоих апостолов, святителей и учителей Церкви, послуживших и благоугодивших Тебе словом и делом. Аминь. 

\end{mymulticols}

\mychapterending

\mychapter{Молитва о скончавшихся в состоянии душевного заболевания}
%http://www.molitvoslov.org/text218.htm 

\begin{mymulticols}

Дивна дела Твоя, Господи, и величию Твоего разума несть конца! Коль многообразными и неизследимыми стезями преводиши Ты люди Своя в веце сем от дольняго мира к горнему! Разоряеши бо премудрость премудрых, отвергаеши разум разумных, высокия и гордыя расточаеши и смиряеши, сильныя и доброродныя низлагаеши и посрамляеши; буия же и смиренныя, уничиженныя, немощныя и худородныя избираеши, возвышаеши, укрепляеши и прославляеши, открывая премудрые и благие советы всепромыслительного смотрения Твоего младенцам и утаевая их от тех, иже о себе мнятся быти совершени умы своими. Кто доволен есть уразумети вся судьбы Твоя, Боже, имиже, по богатству благости Своея и долготерпения, вразумляеши, научаеши и спасаеши нас грешных и недостойных рабов Твоих? Овому прилагаеши смышление и даруеши здравое разсуждение, оваго же обуяеши, у оваго исторгаеши сердце каменно и влагаеши плотяно; овому ниспосылаеши крепость и бодрость телесну, вкупе же и мужество душевно; овому же болети душею и телом попускаеши; оваго радуеши, овому же печаловатися хощеши; оваго живиши, оваго же мертвиши. О, великий и неисповедимый в тайнах Своего смотрения о нас, Господи Боже наш! Вем и от всего сердца исповедую пред Тобою, яко и со усопшим (-шею) приснопоминаемым (-мою) мною рабом Твоим (рабою Твоею) (имя), не без премудраго и благаго изволения и смотрения Твоего бысть сокрушительный недуг душевный, егоже истинная вина и вся сила Тебе Единому точию ведома суть: темже благоговейно повергаяся пред Твоими, Господи, неизреченными, но выну спасительными судьбами, со умилением вопию Ти: приклони ухо Твое к молению моему и человеколюбие вонми ми, и скоро услыши молитву мою о сем рабе Твоем (Твоей). Недоумеваю бо и зело скорблю о необычайных судьбах его (ея), зане, не якоже инии человецы смысленнии, самосознательни и тверди в памяти, отьиде к Тебе от нас, но яко лишен (-на) сый ума, извращен в смысле, болезненно опечалован (-на) сердцем, не имеяй самообладательныя силы своего изволения, неистово и свирепо являющися в деяниях своих. Господи! Господи! Что сие есть неудоборазумеваемое некое в душевней жизни человечестей явление? Како созданный по образу и по подобию Твоему впадает в крайнее неразумие, безумствует, неистовствует и содевает неподобная? Что убо сие есть с дыханием жизни Твоея, Боже Вседержителю? Воистинну, по Твоим, Господи, неизследимым путям и неизреченным советам вся сия премудро и благим намерением Тобою устрояются: наводиши бо, яко привременное наказание, и попускаеши толикая нестроения и превращения в чудном составе существа нашего, да сокровенными у Тебе тако путями устроиши, возставиши и исправиши вся стропотная и развращенная, яже неразсудным, злым и лукавым обычаем учинена суть, и тако видимая злая у человек, у Тебе, Всеблагий Владыко, приснотекущий Источниче Премудрости и разума, у Тебе обращаются во благая. Того ради смиренно припадая к неистощимым щедротам Твоим, Человеколюбче Спасе наш, молю Тя, Подателю благих, приими усопшаго (-ую) раба Твоего (рабу Твою) (имя), яко тако люте болезновавшаго (-ую) и страдавшаго (-ую) в житии сем, приими милостию Твоею, да не пробавиши, Блаже, и тамо, во стране отшедших от нас братий наших, стужения и страдания его (ея), яже суть безмерно лютейша неже зде земная; но отврати от него (нея) прещение гнева Твоего и воззри на него (на ню) пресветлым оком милосердия Твоего; но не предаждь люте погибнути во веки, зане и Сам, Премилосердый Спасе, не хощеши смерти грешника, не хощеши, да кто погибнет, Сладчайший Иисусе Христе, Сыне Божий, исцеляяй неисцельно-недугующих телом, прокаженных очищаяй и бесных творяй смыслящими! Молю Тя, Господи, исцели душу усопшаго (-шия) раба Твоего (рабы Твоея) \myemph{(имя)}, очисти его (ю) от всех вольных и невольных, ведомых и неведомых прегрешений и грехопадений его (ея), Сам, Спасе, очисти, якоже не возмогшаго разумно и в здравой памяти исповедатися Тебе в них, того ради умилосердися о нем (о ней), Господи, и возьми от него (от нея) все тяжкое бремя грехов. Вземляй грехи мира! Покрый и оправдай его (ю) Своею благодатию, и, имиже веси судьбами. Спаси раба (-бу) Твоего (-ю) от избранных Твоих, но сподоби общником (-цею) быти и небесных ликостояний, яко убогаго (-ую) раба Твоего (рабу Твою), иже (яже) зде на земли отлученна (-у) и отчужденна (-у) себе имяше от сродников своих, другов и ближних, и не лиши его Небеснаго Твоего Царствия, яко лишенна (-ну) зде на земли видети благая Твоя и оставлнна (-ну) влачити дни свои исполнении горькоплачевных деяний: того ради, молю Тя, Господи, упокой его (ю) со святыми Твоими в Дому Отца Твоего, с Нимже Тебе, Избавителю и Жизнодавцу нашему, купно со Пресвятым Твоим Духом славу возсылаем всегда, ныне и присно и во веки веков. Аминь

\end{mymulticols}

\mychapterending

\mychapter{Молитва по исходе души из тела}\begin{mymulticols}
%http://www.molitvoslov.org/text185.htm 


Боже Духов и всякия плоти! Ты твориши ангелы Своя духи, и слуги Своя пламень огненный. Пред Тобою трепещут Херувимы и Серафимы и тьмы тем и тысяща тысящ со страхом и трепетом выну предстоят Престолу Твоему. Ты за хотящих улучити спасение посылаеши на служение святых Твоих Ангелов; Ты и нам грешным коемуждо даеши святаго Ангела Твоего, яко пестуна, еже хранили ны во всех путех наших от всякаго зла и таинственно наставляти и вразумляти ны даже до последняго издыхания нашего. Господи! Ты повеле еси изъяти душу от приснопоминаемаго(-мой) нами раба Твоего (рабы Твоея) \myemph{(имя)}, воля Твоя "--- воля святая; молим Тя, Жизнодавче Господи, не отними точию ныне от души его (ея) сего пестуна и хранителя ея, не остави ю едину, яко в путь шествующую; повели ему, яко хранителю, не удалятися помощию в сем страшном прохождении ея в мир горний невидимый; молим Тя, да будет убо ей заступником и защитником от злаго сопротивника в прохождении мытарств, дондеже преведет ю к Тебе, яко к Судии неба и земли. О, страшно прохождение сие для души, грядущей на суд Твой нелицеприятный, и имущей в прохождении сем истязатися духами злобы поднебесными! Темже убо молим Тя, Преблагий Господи, благоволи и еще послали святых Твоих Ангелов душе преставльшагося(-шейся) к Тебе раба Твоего (рабы Твоея) \myemph{(имя)}, да защитят, оградят и сохранят ю от нападения и истязания страшных и злых оных духов, яко истязателей и мытарей воздушных, служителей князя тьмы; молим Тя, свободи ю сего злаго обстояния, да не срящет ю злых демонов полчище; сподоби ю безбоязненно, благотишно и невозбранно преяти от земли страшный сей путь со Ангелы Твоими, да вознесут ю поклонитися Престолу Твоему и да приведут ю к свету Твоего милосердия.

\end{mymulticols}

\mychapterending

\mychapter{Моление о том, чтобы Бог даровал нам усердие к молитве за усопших и принял бы её}\begin{mymulticols}
%http://www.molitvoslov.org/text184.htm 


Многомилостиве и Всемилостиве Господи, Владыко и Судие живых и мертвых, дивный и неисповедимый в судьбах Своих Боже наш! Ты зриши печали, скорби и болезни, обдержащия ны зело, и воздыхания наша слышиши, и гласу нашему внемлеши. Како убо во многих скорбех наших обрящем утешение и покой душам нашим, аще не Твоим благопоспешеством, Всесильне Спасе наш? Ты бо еси Бог наш, и разве Тебе иного не знаем. Приими, Человеколюбче Господи, молитву нашу, юже приносим Тебе от всего сердца нашего и от всея души нашея. Веси, Господи, яко не довольны есмы достойную принести Тебе молитву; но Сам, Всеблагий, научи нас, како подобает молитися Ти. Согрей сердца наша теплотою Духа Твоего Святаго, да не тща будет молитва наша за души от нас преставльшихся раб Твоих, и даждь убо нам со дерзновением и неосужденно смети призывати Тебе, Небеснаго Бога Отца, яко Ты еси упование наше и прибежище; припадаем пред Тобою, в Троице покланяемый и прославляемый Боже наш, и Тебе единому приносим молитву нашу в скорбех и болезнех наших, в радости и печали, яко ты еси утешение наше, радование наше, сокровище наше и жизни нашея Податель: от Тебе бо исходит всяко деяние благо и всяк дар совершен. Сего ради дерзаем взывала к Тебе нашими бренными устнами: услыши нас, Боже, услыши нас грешных, молящихся Тебе; приклони ухо Твое к молению нашему и вопль наш к Тебе да приидет. Приими, Преблагий Господи, молитву нашу за души усопших раб Твоих, яко благовонное кадило. Тебе приносимое. Отжени от нас весь помысл лукавый; созижди в нас, Боже, сердце чисто и дух сокрушен; всели в нас, Боже, корень благих "--- страх Твой в сердца наши; виждь слезы наши, виждь сетование наше; зри скорби наша, зри горько плачущих о преставльшемся (-шейся) к Тебе рабе Твоем (Твоей) \myemph{(имя)} и упокой его (ю) во Царствии Твоем, яко Ты еси Живот, Воскресение и Покой усопших раб Твоих, и Тебе славу возсылаем, яко Безначальному Отцу со Единородным Твоим Сыном и Всесвятым и Благим и Животворящим Твоим Духом всегда, ныне и присно и во веки веков. Аминь. 

\end{mymulticols}

\mychapterending

\mychapter{Молитва за всех в вере скончавшихся православных христиан}\begin{mymulticols}
%http://www.molitvoslov.org/text189.htm 


Владыко Многомилостиве и Всемилостиве Господи Иисусе Христе Боже наш! Милосерд ты к нам грешным и милосердию Твоему несть конца: не хощеши бо нашей вечной погибели, зане рекл еси: «Не хощу смерти грешника, но еже обратитися и живу быти ему». Слава милосердию Твоему! Слава смотрению Твоему! Слава долготерпению Твоему, Господи! От всего сердца нашего, от всея души нашея благодарим Тя, Боже наш, яко сподобил еси усопшаго(ую) раба Твоего (рабу Твою) \myemph{(имя)} на одре болезни исповедатися Тебе во гресех своих. Ты же, Господи, Един ведый вся, веси, коль тяжки греси его (ея), но обаче молим Тя, приими раба Твоего (рабу Твою), отшедшаго(ую) к Тебе аще и грешником(-цею), но не нераскаянным(-ою), и воздаждь ему (ей) по раскаянному сердцу его (ея). И якоже Ты приемлеши и единое слово, и единый вздох, и единую каплю слезную от исповедающегося Тебе с чистым и сокрушенным сердцем, темже убо молим Тя, Сердцеведче Святый, приими ныне от нас нашу молитву за усопшаго(-ую), вонми молений наших гласу, и услыши ны; со умиленною душею и сокрушенным сердцем молим Тя, Господи, воззри всемилостивым оком Своим на покаявшагося (-шуюся) Тебе усопшаго(-ую) раба Твоего (рабу Твою) (имя) и попали невещественным огнем Духа Твоего Святаго вся терния грехов его (ея), неисповеданных пред лицем Твоим. Аще же усопший(-шая) раб Твой (раба Твоя) при покаянии забвению что предаде по немощи естества, или тяжкой болезни, или страха ради часа смертнаго не возможе в сокрушении и умилении сердца покаятися пред Тобою, Господи, ниже возможе принести Тебе плодов достойных покаяния, со умилением просим, молим Тя, Спасителю наш, восполни его (ея) покаяние молитвами Святыя Твоея Церкви и всех святых Твоих, от века благоугодивших Тебе, наипаче же восполни сию духовную скудость Твоею благостию, Твоими дражайшими искупительными и спасительными заслугами; и якоже мытаря оправдал еси, и разбойника, на кресте молившагося Ти, помиловал еси, сице молим Тя: помилуй преставльшагося(-уюся) к Тебе с верою и упованием и прости ему (ей) всякое согрешение вольное и невольное, словом, делом, ведением и неведением содеянное, и такода неосужденно явится пред лицем Твоим в день великаго и Страшнаго Суда Твоего. Молим Тя, дражайший Искупителю наш, сподоби усопшаго(ую) раба Твоего (рабу Твою) \myemph{(имя)} внити в Царствие Твое, яко сподобившагося(-уюся), по Твоей благости, причаститися святых Твоих страшных и животворящих Тайн во оставление грехов и в жизнь вечную. Ты бо, Господи, пречистыми устами рекл еси: «Ядый Мою Плоть и пияй Мою Кровь имать живот вечный». Ты, Господи, Сам сподобил еси усопшаго(-ую) раба Твоего (рабу Твою) причаститися пречистаго Твоего Тела и пречистыя Твоея Крове, яко залога вечнаго блаженства; сего ради паки молим Тя: да будут ему (ей) святыя и животворящия Тайны Твоя, яко угль, попаляющ вся согрешения его (ея), вся беззакония и неправды его (ея), да будут во очищение, во освящение, оправдание и оставление грехов, и в залог живота и блаженства вечнаго. Аминь 

\end{mymulticols}

\mychapterending

\mychapter{Молитва вторая за всех в вере скончавшихся православных христиан}\begin{mymulticols}
%http://www.molitvoslov.org/text193.htm 


Боже щедрот и всякия благодати! Призри с высоты Святаго жилища Твоего на нас грешных и недостойных рабов Твоих, молящихся Тебе и просящих дати долгов разрешение усопшему (-ей) рабу Твоему (рабе Твоей). Молим Тя, Благость вечная, не вниди в суд с рабом (-ою) Твоим (-ею), яко созданием Твоей Премудрости, но обаче милости сподоби его (ю); яви ему (ей) благоутробие Твое, Всеблагий Владыко, и не отвержи его (ю) от лица Твоего, Человеколюбче Господи: спаси от вечныя смерти, яже есть смерть вторая, душу его (ея), да тако усопший (-шая) не престанет славословити Твою безконечную благость и превозносити имя Твое во веки. Тебе, Сердцеведче, ведомы дела, намерения и помышления усопшаго (-шия) раба Твоего (рабы Твоея) и вся содеянная им (ею) благая же и злая пред Тобою нага и объявлена суть. Но что убо пред Тобою, Святый, вся благая наша? Яко капля в мори, тако суть пред Тобою и пред Твоею правдою и Святостию вся благая наша, содеваемая в веце сем. Безмерны согрешения наши; велико и тяжко бремя грехов наших, но неизреченно велики и безпредельны заслуги Единороднаго Сына Твоего, Господа нашего Иисуса Христа, Егоже по безпредельной любви Твоей дал еси нам грешным, да всяк веруяй в Онь, не погибнет, но имать живот вечный. Зело грешны есмы пред Тобою и недостойны Твоея милости, но Твое человеколюбие соделывает нас достойными; велики беззакония наша, но неистощимо Твое милосердие: безмерныя заслуги Возлюбленнаго Твоего Сына, принесшаго Себя в жертву за грехи наши, побеждают всяк грех, всяко беззаконие; того ради со умилением и со дерзновением молим Тя, Источниче жизни нашея, со дерзновением упования молим, да усвоятся усопшему (-ей) рабу Твоему (рабе Твоей) \myemph{(имя)} сии драгоценныя для нас искупительныя и спасительныя заслуги Единороднаго Сына Твоего; умоляем Твое милосердие слезами сокрушения, покрый грехи усопшаго (-шия) раба Твоего (рабы Твоея) величием сих дражайших заслуг Искупителя нашего; прости все согрешения его (ея) и не погуби его (ю) со беззаконми его (ея) в геенне огненней. Пощади, пощади, Владыко, отшедшаго (-ую) к Тебе раба Твоего (рабу Твою) \myemph{(имя)} и вечных мучений избави, избави ради безмерных и неизглаголанных душевных скорбей и болезней Единороднаго Сына Твоего, понесенных Им пред испитием горькой чаши страданий Своих; пощади его (ю) и спаси от вечныя смерти ради поношения и биения Спасителя нашего, ради заушения и заплевания; помилуй ради дражайшия Крове Его из пречистых ребр Его пролиянныя. Молим Тя, Отче щедрот и всякаго милосердия, пречистыми и животворящими Его ранами, Его святою Кровию уврачуй смертоносныя и греховныя язвы усопшаго (-шия) раба Твоего (рабы Твоея), и тако да исцелеет духовне и да сподобится в неприступном свете Твоем, Светодавче Господи, славословит и прославляти Пресвятое имя Твое во веки. Аминь. 

\end{mymulticols}

\mychapterending

\mychapter{Молитва ко Пресвятой Богородице}\begin{mymulticols}
%http://www.molitvoslov.org/text187.htm 


Пресвятая Владычице Богородице! К Тебе прибегаем, Заступнице наша: Ты бо скорая еси помощница, наша ходатаица у Бога неусыпающая! Наипаче же молим Тя в час сей: помози новопреставленному(-ной) рабу Твоему (рабе Твоей) \myemph{(имя)} прейти страшный и неведомый оный путь; молим Тя, Владычице мира, силою Твоею отжени от движимой страхом души его (ея) страшныя силы темных духов, да смятутся и посрамятся пред Тобою; свободи ю от истязания воздушных мытарей, разори советы их и низложи их, яко зломыслящих врагов. Буди ей, о Всемилостивая Госпоже Богородице, заступницею и защитницею от воздушного князя тьмы, мучителя и страшных путей стоятеля; молим Тя, Пресвятая Богородице, ризою Твоею честною Защити ю, да тако безбоязненно и невозбранно прейдет от земли на небо. Молим Тя, Заступнице наша, заступи за раба Твоего (рабу Твою) Матерним Твоим у Господа дерзновением; молим Тя, Помоще наша, помози ему (ей), имущему (-щей) судитися еще прежде онаго Страшнаго судилища, помози убо оправдатися пред Богом, яко Творцем неба и земли, и умоли Единороднаго Сына Твоего, Господа Бога и Спаса нашего Иисуса Христа, да упокоит усопшаго(-ую) в недрах Авраама с праведными и всеми святыми. Аминь.

\end{mymulticols}

\mychapterending

\mychapter{Молитва Ангелу хранителю}\begin{mymulticols}
%http://www.molitvoslov.org/text186.htm 


Святый Ангеле хранителю, данный усопшему (-шей) рабу Божию (рабе Божией) (имя)! Не преставай охраняти душу его (ея) от злых страшных оных бесов; буди ей пестуном и утешителем и тамо, в оном невидимом мире духов; приими ю под криле своя и преведи ю невозбранно чрез врата воздушных истязателей; предстани ходатаем и молитвенником за ню у Бога,"--- моли Его Преблагаго, да не низведена будет в место мрака, но да вчинит ю, идеже пребывает Свет невечерний. 

\end{mymulticols}

\mychapterending

\mychapter{Канон молебный от лица человека с душею разлучающегося и не могущаго глаголати, глас 6-й}\begin{mymulticols}
%http://www.molitvoslov.org/text179.htm 


\mysubtitle{Песнь 1}

\irmos{Яко по суху пешешествовав Израиль, по бездне стопами, гонителя фараона видя потопляема, Богу победную песнь поим, вопияше.} 

\pripev{Пресвятая Богородице, спаси нас.}

Каплям подобно дождевным, злии и малии дние мои, летним обхождением оскудевающе, помалу исчезают уже, Владычице, спаси мя. 

\pripev{Пресвятая Богородице, спаси нас.}

Твоим благоутробием и многими щедротами Твоими, Владычице, преклоняема естественно, в час сей ужасный предстани ми, Помощнице Непоборимая. 

\pripev{Пресвятая Богородице, спаси нас.}

Содержит ныне душу мою страх велик, трепет неисповедим и болезнен есть, внегда изыти ей от телесе, Пречистая, юже утеши. 

\slavan

Грешным и смиренным известное прибежище, о мне извести Твою милость, Чистая, и бесовския избави руки, якоже бо пси мнози обступиша мя. 

\inynen

Се время помощи, се время Твоего заступления, се, Владычице, время, о немже день и нощь припадах тепле и моляхся Тебе.

\mysubtitle{Песнь 3}

\irmos{Несть свят, якоже Ты, Господи Боже мой, вознесый рог верных Твоих, Блаже, и утвердивый нас на камени исповедания Твоего.} 

\pripev{Пресвятая Богородице, спаси нас.}

Издалеча сего дне, Владычице, провидя, и того яко пришедша помышляяй присно, слезами теплыми моляхся не забыти мене. 

\pripev{Пресвятая Богородице, спаси нас.}

Обыдоша мя мысленнии рыкающе скимны, и ищут восхитити и растерзати мя горце, ихже зубы, Чистая, и челюсти сокруши и спаси мя. 

\pripev{Пресвятая Богородице, спаси нас.}

Угасшу убо отнюд органу словесному, и связавшуся языку, и затворившуся гласу, в сокрушении сердца молю Тя, Спасительнице моя, спаси мя. 

\slavan

Приклони ухо Твое ко мне, Христа Бога моего Мати, от высоты многия славы Твоея, Благая, и услыши стенание конечное, и руку ми подаждь. 

\inynen

Не отврати от мене многия щедроты Твоя, не затвори утробу Твою человеколюбную, Чистая; но предстани ми ныне, и в час судный помяни мя.

\mysubtitle{Песнь 4}

\irmos{Христос моя сила, Бог и Господь, честная Церковь боголепно поет, взывающи, от смысла чиста о Господе празднующи.} 

\pripev{Пресвятая Богородице, спаси нас.}

Умовение согрешением, ток слезный ныне положи, Благая, сердца моего сокрушение приемлющи: о Тебе утверждающу упование, Благая, егда како страшнаго мя избавиши огненнаго мучения, яко Сама благодати еси источник, Богородительнице. 

\pripev{Пресвятая Богородице, спаси нас.}

Непостыдное и непогрешительное всем, иже в нуждах, прибежище, Владычице Пренепорочная, Ты ми буди Заступница в час испытания. 

\pripev{Пресвятая Богородице, спаси нас.}

Простерши пречистеи Твои и всечестнеи руце, яко священнии голубине криле, под кровом и сению тех покрый мя, Владычице. 

\slavan

Воздушнаго князя насильника, мучителя, страшных путей стоятеля и напраснаго сих словоиспытателя, сподоби мя прейти невозбранно отходяща от земли. 

\inynen

Се мя, Владычице, страх усрете, егоже и бояхся: се подвиг велик объят мя, в немже буди ми Помощница, Надежде спасения моего.

\mysubtitle{Песнь 5}

\irmos{Божиим светом Твоим, Блаже, утренюющих Ти души любовию озари, молюся, Тя ведети, Слове Божий, Истиннаго Бога, от мрака греховнаго взывающа.} 

\pripev{Пресвятая Богородице, спаси нас.}

Не забуди мя, Благая, ниже отврати от мене, Твоего отрока, лице Твое, но услыши мя, яко скорблю, и вонми души моей, и сию избави. 

\pripev{Пресвятая Богородице, спаси нас.}

Иже по плоти сродницы мои, и иже по духу братие, и друзи, и обычнии знаемии, плачите, воздохните, сетуйте, се бо от вас ныне разлучаюся. 

\pripev{Пресвятая Богородице, спаси нас.}

Ныне избавляяй никако и помогаяй воистинну никтоже; Ты помози ми, Владычице, да не яко человек безпомощен, в руках враг моих затворен буду. 

\slavan

Вшедше, святии мои Ангели, предстаните судищу Христову, колене свои мысленнии преклоньше, плачевне возопийте Ему: помилуй, Творче всех, дело рук Твоих, Блаже, и не отрини его. 

\inynen

Поклоньшеся Владычице и Пречистей Матери Бога моего помолитеся, яко да преклонит колена с вами, и преклонит Его на милость: Мати бо сущи и Питательница услышана будет.

\mysubtitle{Песнь 6}

\irmos{Житейское море, воздвизаемое зря напастей бурею, к тихому пристанищу Твоему притек, вопию Ти: возведи от тли живот мой, Многомилостиве.} 

\pripev{Пресвятая Богородице, спаси нас.}

Устне мои молчат, и язык не глаголет, но сердце вещает: огнь бо сокрушения сие снедая внутрь возгарается, и гласы неизглаголанными Тебе, Дево, призывает. 

\pripev{Пресвятая Богородице, спаси нас.}

Призри на мя свыше, Мати Божия, и милостивно вонми ныне на мое посещение снити, яко да видев Тя, от телесе изыду радуяся. 

\pripev{Пресвятая Богородице, спаси нас.}

Растерзаеми соузы, раздираеми закони естественнаго сгущения, и составления всего телеснаго, нужду нестерпимую и тесноту сотворяют ми. 

\slavan

Святых Ангел священным и честным рукам преложи мя, Владычице, яко да тех крилы покрывся, не вижу безчестнаго и смраднаго и мрачнаго бесов образа. 

\inynen

Чертоже Божий Всечестне, небесному разумному чертогу сподоби мя, мою угасшую и несияющую свещу вжегши, святым елеем милости Твоея.

\mysubtitle{Кондак, глас 6}

Душе моя, душе моя, востани, что спиши, конец приближается, и нужда ти молвити: воспряни убо, да пощадит тя Христос Бог, Иже везде сый, и вся исполняяй. 

\mysubtitle{Икос} 

Христово врачевство видя отверсто, и от сего Адаму источающе здравие, пострадав уязвися диавол, яко беды приемля рыдаше, и своим другом возопи: что сотворю Сыну Мариину? Убивает мя Вифлеемлянин, Иже везде сый и вся исполняяй.

\mysubtitle{Песнь 7}

\irmos{Росодательну убо пещь содела Ангел, преподобным отроком, халдеи же опаляющее веление Божие, мучителя увеща вопити: благословен еси, Боже отец наших.} 

\pripev{Пресвятая Богородице, спаси нас.}

Нощь смертная мя постиже неготова, мрачна же и безлунна, препущающи неприготовлена к долгому оному пути страшному: да спутешествует ми Твоя милость, Владычице. 

\pripev{Пресвятая Богородице, спаси нас.}

Се вси дние мои исчезоша воистинну в суете, якоже пишется, и лета моя со тщанием, сети же смертныя воистинну и горькия предвариша мою душу, яже мя обдержат. 

\pripev{Пресвятая Богородице, спаси нас.}

Множество грехов моих да не возможет победити Твоего многаго благоутробия, Владычице; но да обыдет мя Твоя милость и вся да покрыет беззакония моя. 

\slavan

Отводящии мя отсюду находят, содержаще мя всюду: душа же моя отлагает и страшится, многа исполнена мятежа, юже утеши, Чистая, явлением Твоим. 

\inynen

Спечалующаго ниединаго в скорби моей, ниже утешающаго обретох, Владычице, ибо друзи мои и знаемии вкупе оставиша мя ныне; но, Надежде моя, никакоже да не оставиши мя.

\mysubtitle{Песнь 8}

\irmos{Из пламене преподобным росу источил еси, и праведнаго жертву водою попалил еси, вся бо твориши, Христе, токмо еже хотети. Тя превозносим во вся веки.} 

\pripev{Пресвятая Богородице, спаси нас.}

Яко Бога Человеколюбца Мати Человеколюбивая, тихим и милостивым вонми оком, егда от тела душа моя отлучается, да Тя во вся веки славлю, Святая Богородице. 

\pripev{Пресвятая Богородице, спаси нас.}

Убегнути ми варвар безплотных полки и воздушныя бездны возникнути, и к Небеси взыти мя сподоби, да Тя во веки славлю, Святая Богородице. 

\pripev{Пресвятая Богородице, спаси нас.}

Рождшая Господа Вседержителя, горьких мытарств начальника миродержца отжени далече от мене, внегда скончатися хощу, да Тя во веки славлю, Святая Богородице. 

\slavan

Великой последней гласящей трубе, в страшное и грозное воскрешение суда, воскресающим всем, помяни мя тогда, Святая Богородице. 

\inynen

Высокая Владыки Христа палато, Твою благодать свыше пославши, предвари мя ныне в день озлобления, да Тя славлю во веки вся, Святая Богородице.

\mysubtitle{Песнь 9}

\irmos{Бога человеком невозможно видети, на Негоже не смеют чини ангельстии взирати. Тобою же, Всечистая, явися человеком Слово Воплощенно. Егоже величающе, с небесными вои Тя ублажаем.} 

\pripev{Пресвятая Богородице, спаси нас.}

О, како узрю Невидимаго; како ужасное оно претерплю видение? Како дерзну отверсти очи? Како моего Владыку смею видети, Егоже не престаях от юности огорчевая присно? 

\pripev{Пресвятая Богородице, спаси нас.}

Святая Отроковице, Богородительнице, на мое смирение милосердно призри, умиленное мое и последнее моление сие приимши, и мучащаго вечнующаго огня потщися избавити мя. 

\pripev{Пресвятая Богородице, спаси нас.}

Храмы святыя осквернившая, скверный и телесный храм оставивши, Тебе, Божий Всечестный Храме, молит, Отроковице Дево Мати, душа моя тьмы кромешния убежати и лютаго геенскаго жжения. 

\slavan

Зря конец близу жития моего, и помышляя безместных мыслей, деяний же, Всечистая, душу мою делательницу, люте уязвляюся стрелами совести, но преклоньшися милостивно, буди ми Предстательница. 

\inynen

Сын дастся нам за милость, Сын Божий, и Ангельский Царь, Превечный от чистых кровей Твоих Человек прошед. Егоже умилостиви, Отроковице, страстней моей душе, исторгаемей люте от окаяннаго моего телесе. 

\myemph{Таже:} Достойно есть яко воистинну блажити Тя Богородицу, Присноблаженную и Пренепорочную и Матерь Бога нашего. Честнейшую Херувим и славнейшую без сравнения Серафим, без истления Бога Слова рождшую, сущую Богородицу Тя величаем.

\mysubtitle{Молитва, от иерея глаголемая на исход души:}

Владыко Господи Вседержителю, Отче Господа нашего Иисуса Христа, Иже всем человеком хотяй спастися и в разум истины приити, не хотяй смерти грешному, но обращения и живота; молимся и милися Ти деем, душу раба Твоего \myemph{(имярек)} от всякия узы разреши и от всякия клятвы свободи, остави прегрешения ему, яже от юности, ведомая и неведомая, в деле и слове, и чисто исповеданная, или забвением, или студом утаеная. Ты бо Един еси разрешаяй связанныя и исправляяй сокрушенныя, надежда неначаемым, могий оставляти грехи всякому человеку, на Тя упование имущему. Ей, человеколюбивый Господи, повели, да отпустится от уз плотских и греховных, и приими в мир душу раба Твоего сего \myemph{(имярек)}, и покой ю в вечных обителех со святыми Твоими, благодатию Единороднаго Сына Твоего, Господа Бога и Спаса нашего Иисуса Христа, с Нимже благословен еси, с Пресвятым и Благим, и Животворящим Твоим Духом, ныне и присно, и во веки веков. Аминь.

\end{mymulticols}

\mychapterending

\mychapter{Канон молебный ко Господу Иисусу Христу и Пречистой Богородице Матери Господни при разлучении души от тела всякого правоверного}\begin{mymulticols}
%http://www.molitvoslov.org/text174.htm 


\myemph{Приходит игумен, к мирскому же отец его духовный, и вопрошает, аще есть кое слово, или дело забвения ради, или студа, или кая злоба к коему брату неисповедана, или непрощена есть, вся должен есть изыскивати и вопрошати по единому умирающаго.}

\myemph{Посем начинает иерей:} Благословен Бог наш, всегда, ныне и присно, и во веки веков. (\myemph{аще ли мирский:} \MolitvamiSviatyhOtecNashih)

\TrisviatoePoOtcheNash

Господи, помилуй \myemph{(12)}.

\priiditepoklonimsia

\mysubtitle{Псалом 50}

\PsalmFifty

\mysubtitle{Песнь 1}

\irmos{Яко по суху пешешествовав Израиль, по бездне стопами, гонителя фараона видя потопляема, Богу победную песнь поим, вопияше.}

\pripev{Пресвятая Богородице, спаси нас.}

Каплям подобно дождевным, злии и малии дние мои, летним обхождением оскудевающе, помалу исчезают уже, Владычице, спаси мя.

\pripev{Пресвятая Богородице, спаси нас.}

Твоим благоутробием и многими щедротами Твоими, Владычице, преклоняема естественно, в час сей ужасный предстани ми, Помощнице Непоборимая.

\pripev{Пресвятая Богородице, спаси нас.}

Содержит ныне душу мою страх велик, трепет неисповедим и болезнен есть, внегда изыти ей от телесе, Пречистая, юже утеши.

\slava

Грешным и смиренным известное прибежище, о мне извести Твою милость, Чистая, и бесовския избави руки, якоже бо пси мнози обступиша мя.

\inyne

Се время помощи, се время Твоего заступления, се, Владычице, время, о немже день и нощь припадах тепле и моляхся Тебе.

\mysubtitle{Песнь 3}

\irmos{Несть свят, якоже Ты, Господи Боже мой, вознесый рог верных Твоих, Блаже, и утвердивый нас на камени исповедания Твоего.}

\pripev[]{Пресвятая Богородице, спаси нас.}

Издалеча сего дне, Владычице, провидя, и того яко пришедша помышляяй присно, слезами теплыми моляхся не забыти мене.

\pripev[]{Пресвятая Богородице, спаси нас.}

Обыдоша мя мысленнии рыкающе скимны, и ищут восхитити и растерзати мя горце, ихже зубы, Чистая, и челюсти сокруши и спаси мя.

\pripev[]{Пресвятая Богородице, спаси нас.}

Угасшу убо отнюд органу словесному, и связавшуся языку, и затворившуся гласу, в сокрушении сердца молю Тя, Спасительнице моя, спаси мя.

\slava

Приклони ухо Твое ко мне, Христа Бога моего Мати, от высоты многия славы Твоея, Благая, и услыши стенание конечное, и руку ми подаждь.

\inyne

Не отврати от мене многия щедроты Твоя, не затвори утробу Твою человеколюбную, Чистая; но предстани ми ныне, и в час судный помяни мя.

\mysubtitle{Песнь 4}

\irmos{Христос моя сила, Бог и Господь, честная Церковь боголепно поет, взывающи, от смысла чиста о Господе празднующи.}

\pripev[]{Пресвятая Богородице, спаси нас.}

Умовение согрешением, ток слезный ныне положи, Благая, сердца моего сокрушение приемлющи: о Тебе утверждающу упование, Благая, егда како страшнаго мя избавиши огненнаго мучения, яко Сама благодати еси источник, Богородительнице.

\pripev[]{Пресвятая Богородице, спаси нас.}

Непостыдное и непогрешительное всем, иже в нуждах, прибежище, Владычице Пренепорочная, Ты ми буди Заступница в час испытания.

\pripev[]{Пресвятая Богородице, спаси нас.}

Простерши пречистеи Твои и всечестнеи руце, яко священнии голубине криле, под кровом и сению тех покрый мя, Владычице.

\slava

Воздушнаго князя насильника, мучителя, страшных путей стоятеля и напраснаго сих словоиспытателя, сподоби мя прейти невозбранно отходяща от земли.

\inyne

Се мя, Владычице, страх усрете, егоже и бояхся: се подвиг велик объят мя, в немже буди ми Помощница, Надежде спасения моего.

\mysubtitle{Песнь 5}

\irmos{Божиим светом Твоим, Блаже, утренюющих Ти души любовию озари, молюся, Тя ведети, Слове Божий, Истиннаго Бога, от мрака греховнаго взывающа.}

\pripev[]{Пресвятая Богородице, спаси нас.}

Не забуди мя, Благая, ниже отврати от мене, Твоего отрока, лице Твое, но услыши мя, яко скорблю, и вонми души моей, и сию избави.

\pripev[]{Пресвятая Богородице, спаси нас.}

Иже по плоти сродницы мои, и иже по духу братие, и друзи, и обычнии знаемии, плачите, воздохните, сетуйте, се бо от вас ныне разлучаюся.

\pripev[]{Пресвятая Богородице, спаси нас.}

Ныне избавляяй никако и помогаяй воистинну никтоже; Ты помози ми, Владычице, да не яко человек безпомощен, в руках враг моих затворен буду.

\slava

Вшедше, святии мои Ангели, предстаните судищу Христову, колене свои мысленнии преклоньше, плачевне возопийте Ему: помилуй, Творче всех, дело рук Твоих, Блаже, и не отрини его.

\inyne

Поклоньшеся Владычице и Пречистей Матери Бога моего помолитеся, яко да преклонит колена с вами, и преклонит Его на милость: Мати бо сущи и Питательница услышана будет.

\mysubtitle{Песнь 6}

\irmos{Житейское море, воздвизаемое зря напастей бурею, к тихому пристанищу Твоему притек, вопию Ти: возведи от тли живот мой, Многомилостиве.}

\pripev[]{Пресвятая Богородице, спаси нас.}

Устне мои молчат, и язык не глаголет, но сердце вещает: огнь бо сокрушения сие снедая внутрь возгарается, и гласы неизглаголанными Тебе, Дево, призывает.

\pripev[]{Пресвятая Богородице, спаси нас.}

Призри на мя свыше, Мати Божия, и милостивно вонми ныне на мое посещение снити, яко да видев Тя, от телесе изыду радуяся.

\pripev[]{Пресвятая Богородице, спаси нас.}

Растерзаеми соузы, раздираеми закони естественнаго сгущения, и составления всего телеснаго, нужду нестерпимую и тесноту сотворяют ми.

\slava

Святых Ангел священным и честным рукам преложи мя, Владычице, яко да тех крилы покрывся, не вижу безчестнаго и смраднаго и мрачнаго бесов образа.

\inyne

Чертоже Божий Всечестне, небесному разумному чертогу сподоби мя, мою угасшую и несияющую свещу вжегши, святым елеем милости Твоея.

\mysubtitle{Кондак, глас 6}

Душе моя, душе моя, востани, что спиши, конец приближается, и нужда ти молвити: воспряни убо, да пощадит тя Христос Бог, Иже везде сый, и вся исполняяй.

\mysubtitle{Икос}

Христово врачевство видя отверсто, и от сего Адаму источающе здравие, пострадав уязвися диавол, яко беды приемля рыдаше, и своим другом возопи: что сотворю Сыну Мариину? Убивает мя Вифлеемлянин, Иже везде сый и вся исполняяй.

\mysubtitle{Песнь 7}

\irmos{Росодательну убо пещь содела Ангел, преподобным отроком, халдеи же опаляющее веление Божие, мучителя увеща вопити: благословен еси, Боже отец наших.}

\pripev[]{Пресвятая Богородице, спаси нас.}

Нощь смертная мя постиже неготова, мрачна же и безлунна, препущающи неприготовлена к долгому оному пути страшному: да спутешествует ми Твоя милость, Владычице.

\pripev[]{Пресвятая Богородице, спаси нас.}

Се вси дние мои исчезоша воистинну в суете, якоже пишется, и лета моя со тщанием, сети же смертныя воистинну и горькия предвариша мою душу, яже мя обдержат.

\pripev[]{Пресвятая Богородице, спаси нас.}

Множество грехов моих да не возможет победити Твоего многаго благоутробия, Владычице; но да обыдет мя Твоя милость и вся да покрыет беззакония моя.

\slava

Отводящии мя отсюду находят, содержаще мя всюду: душа же моя отлагает и страшится, многа исполнена мятежа, юже утеши, Чистая, явлением Твоим.

\inyne

Спечалующаго ниединаго в скорби моей, ниже утешающаго обретох, Владычице, ибо друзи мои и знаемии вкупе оставиша мя ныне; но, Надежде моя, никакоже да не оставиши мя.

\mysubtitle{Песнь 8}

\irmos{Из пламене преподобным росу источил еси, и праведнаго жертву водою попалил еси, вся бо твориши, Христе, токмо еже хотети. Тя превозносим во вся веки.}

\pripev[]{Пресвятая Богородице, спаси нас.}

Яко Бога Человеколюбца Мати Человеколюбивая, тихим и милостивым вонми оком, егда от тела душа моя отлучается, да Тя во вся веки славлю, Святая Богородице.

\pripev[]{Пресвятая Богородице, спаси нас.}

Убегнути ми варвар безплотных полки и воздушныя бездны возникнути, и к Небеси взыти мя сподоби, да Тя во веки славлю, Святая Богородице.

\pripev[]{Пресвятая Богородице, спаси нас.}

Рождшая Господа Вседержителя, горьких мытарств начальника миродержца отжени далече от мене, внегда скончатися хощу, да Тя во веки славлю, Святая Богородице.

\slava

Великой последней гласящей трубе, в страшное и грозное воскрешение суда, воскресающим всем, помяни мя тогда, Святая Богородице.

\inyne

Высокая Владыки Христа палато, Твою благодать свыше пославши, предвари мя ныне в день озлобления, да Тя славлю во веки вся, Святая Богородице.

\mysubtitle{Песнь 9}

\irmos{Бога человеком невозможно видети, на Негоже не смеют чини ангельстии взирати. Тобою же, Всечистая, явися человеком Слово Воплощенно. Егоже величающе, с небесными вои Тя ублажаем.}

\pripev[]{Пресвятая Богородице, спаси нас.}

О, како узрю Невидимаго; како ужасное оно претерплю видение? Како дерзну отверсти очи? Како моего Владыку смею видети, Егоже не престаях от юности огорчевая присно?

\pripev[]{Пресвятая Богородице, спаси нас.}

Святая Отроковице, Богородительнице, на мое смирение милосердно призри, умиленное мое и последнее моление сие приимши, и мучащаго вечнующаго огня потщися избавити мя.

\pripev[]{Пресвятая Богородице, спаси нас.}

Храмы святыя осквернившая, скверный и телесный храм оставивши, Тебе, Божий Всечестный Храме, молит, Отроковице Дево Мати, душа моя тьмы кромешния убежати и лютаго геенскаго жжения.

\slava

Зря конец близу жития моего, и помышляя безместных мыслей, деяний же, Всечистая, душу мою делательницу, люте уязвляюся стрелами совести, но преклоньшися милостивно, буди ми Предстательница.

\inyne

Сын дастся нам за милость, Сын Божий, и Ангельский Царь, Превечный от чистых кровей Твоих Человек прошед. Егоже умилостиви, Отроковице, страстней моей душе, исторгаемей люте от окаяннаго моего телесе.

\myemph{Таже:} \Chestneyshuyu

\end{mymulticols}

\mychapterending

\mychapter{Последование по исходе души от тела}\begin{mymulticols}
%/text179.htm

\TrisviatoePoOtcheNash

Господи, помилуй. \myemph{(12 раз)}

\vspace{\baselineskip}
\mysubtitle{Чтение тропарей:}

Со духи праведных скончавшихся душу раба Твоего, Спасе, упокой, сохраняя ю во блаженной жизни, яже у Тебе, Человеколюбче.

В покоищи Твоем, Господи: идеже вси святии Твои упокоеваются, упокой и душу раба Твоего, яко Един еси Человеколюбец.

\slava

Ты еси Бог, сошедый во ад и узы окованных разрешивый, Сам и душу раба Твоего упокой.

\inyne

Едина Чистая и Непорочная Дево, Бога без семене рождшая, моли спастися души его.

\vspace{\baselineskip}
\mysubtitle{Псалом 90}

Живый в помощи Вышняго, в крове Бога Небеснаго водворится. Речет Господеви: Заступник мой еси, и Прибежище мое, Бог мой, и уповаю на Него. Яко Той избавит тя от сети ловчи и от словесе мятежна, плещма Своима осенит тя, и под криле Его надеешися: оружием обыдет тя истина Его. Не убоишися от страха нощнаго, от стрелы летящия во дни, от вещи во тьме преходящия, от сряща и беса полуденнаго. Падет от страны твоея тысяща, и тма одесную тебе, к тебе же не приближится: обаче очима твоима смотриши, и воздаяние грешников узриши. Яко Ты, Господи, упование мое, Вышняго положил еси прибежище твое. Не приидет к тебе зло, и рана не приближится телеси твоему. Яко ангелом Своим заповесть о тебе, сохранити тя во всех путех твоих. На руках возмут тя, да не когда преткнеши о камень ногу твою. На аспида и василиска наступиши, и попереши льва и змия. Яко на Мя упова, и избавлю и; покрыю и, яко позна имя Мое. Воззовет ко Мне, и услышу его; с ним есмь в скорби, изму его, и прославлю его; долготою дний исполню его, и явлю ему спасение Мое.

\vspace{\baselineskip}
\mysubtitle{Песнь 1}

\irmos{Воду прошед, яко сушу, и египетскаго зла избежав, израильтянин вопияше: Избавителю и Богу нашему поим.}

\pripev[]{Покой, Господи, душу усопшаго раба Твоего.}

Отверз уста моя, Спасе, слово ми подаждь молитися, Милосерде, о ныне преставленном, да покоиши душу его, Владыко.

\pripev[]{Покой, Господи, душу усопшаго раба Твоего.}

Мертв быв плотию, Спасе, и во гробе положен с мертвыми, душу раба Твоего покой в месте злачне, яко Милосерд.

\slava

Молебный глас мой услыши, Боже Триипостасне, и учини душу преставленнаго в недрех Авраамлих, Избавителю.

\inyne

Ты, Пречистая Богородице, Егоже без искуса мужеска заченши родила еси, моли Сына Твоего подати покой рабу Твоему преставленному. 

\vspace{\baselineskip}
\mysubtitle{Песнь 3}

\irmos{Небеснаго круга Верхотворче Господи и Церкви Зиждителю, Ты мене утверди в любви Твоей, желаний краю, верных утверждение, Едине Человеколюбче.}

\pripev[]{Покой, Господи, душу усопшаго раба Твоего.}

В месте злачне, в месте покойне, идеже лицы святых веселятся, душу раба Твоего преставленнаго покой, Христе, Едине Милостиве.

\pripev[]{Покой, Господи, душу усопшаго раба Твоего.}

Идеже лицы святых, тамо вчини, Владыко, послужившаго Тебе всем сердцем и воздвигшаго иго Твое на рамо свое, яко Един Владыка живота и смерти. 

\slava

Небесный Отче Вседержителю, и Сыне Единородный, и Душе Святый Исходный, презри умершаго согрешения и в Церкви первенец всели его славити Тя со всеми угождшими Тебе. 

\inyne

Яко Мати Святая Пресвятаго Бога, Владычице всяческих, Марие Богородице, со всеми святыми Сего моли душу упокоити раба Твоего в Небесных селениих.

\vspace{\baselineskip}
\mysubtitle{Песнь 4} 

\irmos{Услышах, Господи, смотрения Твоего таинство, разумех дела Твоя и прославих Твое Божество.}

\pripev[]{Покой, Господи, душу усопшаго раба Твоего.}

Сошедый в преисподняя, Христе, совоздвигл еси умершия вся, и преставльшагося от нас покой, Спасе, яко Щедр. 

\pripev[]{Покой, Господи, душу усопшаго раба Твоего.}

Никтоже без греха есть, токмо Ты Един, Владыко: сего ради преставленному и грехи остави, и в рай того всели.

\slava

Услыши, Троице Святая, гласы молебныя, приносимыя Тебе в церкви о усопшем, и Богоначальным Твоим светом озари душу, омраченную суетными привержении.

\inyne

Родила еси, Пречистая, без мужеска семене, Бога Совершенна и Человека Совершенна, вземлющаго грехи наша, Дево. Того моли, Госпоже, преставльшемуся рабу Твоему подати покой.

\vspace{\baselineskip}
\mysubtitle{Песнь 5}

\irmos{Просвети нас повелении Твоими, Господи, и мышцею Твоею высокою Твой мир подаждь нам, Человеколюбче.}

\pripev[]{Покой, Господи, душу усопшаго раба Твоего.}

Имый живота и смерти власть, преставленнаго от нас покой, Христе Боже. Ты бо еси всех, Спасе, Покой и Живот.

\pripev[]{Покой, Господи, душу усопшаго раба Твоего.}

На Тя, Спасе, надежду возложь, умерый отыде от нас, Ты же, Господи, ущедри его, яко Бог Многомилостив.

\slava 

Просвети нас, Трисвяте, воспеваемый Владыко, молящихся Тебе, мир Небесный прияти, и в мирных селех душу вчини, отшедшую от временных, в надежде безконечныя жизни. 

\inyne 

Шуияго стояния, Пречистая, избавити преставленнаго умоли Сына Твоего, Дево Госпоже, яко Спаса и Бога нашего Мати Сущая.

\vspace{\baselineskip}
\mysubtitle{Песнь 6} 

\irmos{Молитву пролию ко Господу, и Тому возвещу печали моя, яко зол душа моя исполнися и живот мой аду приближися, и молюся, яко Иона: от тли, Боже, возведи мя.}

\pripev[]{Покой, Господи, душу усопшаго раба Твоего.}

Ада испроверг, Владыко, воскресил еси умершия от века и ныне преставленнаго от нас в недро Авраамле Ты, Боже, всели, прегрешения вся отпустив, яко Милосерд.

\pripev[]{Покой, Господи, душу усопшаго раба Твоего.}

Заповедь, юже ми дал еси, Боже, преступих и смертен бых, но Ты, Боже, сошедый во гроб и души яже от века воскресивый, не возстави мене, Владыко, на мучение, но на покой, преставленный вопиет Тебе нами, Многомилостиве. 

\slava

Молим Тя, Безначальный Отче и Сыне и Душе Святый, злобою душезлобнаго мира озлобленную и к Тебе, Зиждителю, прешедшую душу во адово дно не отрини, Боже, Спасе мой.

\inyne

С небесе Христос Бог наш, яко дождь на руно, Пречистая, сниде на Тя, напаяя весь мир и изсушая вся безбожныя потоки, наводняяй всю землю разумом Своим, Приснодево, Того моли дати покой преставленному рабу Твоему.

\vspace{\baselineskip}
\mysubtitle{Кондак, глас 8}

Со святыми упокой, Христе, душу раба Твоего, идеже несть болезнь, ни печаль, ни воздыхание, но жизнь безконечная. 

\vspace{\baselineskip}
\mysubtitle{Икос}

Сам Един еси Безсмертный, сотворивый и создавый человека: земнии убо от земли создахомся и в землю туюжде пойдем, якоже повелел еси, Создавый мя, и рекий ми: яко земля еси и в землю отыдеши, аможе вси человецы пойдем, надгробное рыдание творяще песнь: аллилуиа, аллилуиа, аллилуиа.

\vspace{\baselineskip}
\mysubtitle{Песнь 7}

\irmos{От Иудеи дошедше отроцы, в Вавилоне иногда, верою Троическою пламень пещный попраша, поюще: отцев Боже, благословен еси.}

\pripev[]{Покой, Господи, душу усопшаго раба Твоего.}

Владыко Христе Боже, егда хощеши судити миру, пощади душу раба Твоего, егоже от нас приял еси, вопиющаго: отцев наших Боже, благословен еси. 

\pripev[]{Покой, Господи, душу усопшаго раба Твоего.}

В пищи райстей, идеже праведных души веселятся послуживших Тебе, причти с ними, Христе, душу раба Твоего, воспевшаго: отец наших Боже, благословен еси. 

\slava 

Иудейския три отроки спасый во огни, в триех лицех воспетый, избави огня вечнаго усопшаго, воспевшаго Ти верно: отец наших Боже, благословен еси. 

\inyne 

Исаиа Тя Жезл нарече, Чистая, Даниил же Гору Несекомую, Иезекииль же Дверь, из Неяже пройде Христос, мы же Тя, Истинную Богородицу именующе, величаем. 

\vspace{\baselineskip}
\mysubtitle{Песнь 8} 

\irmos{Седмерицею пещь, халдейский мучитель, богочестивым неистовно разжже, силою же лучшею спасены сия видев, Творцу и Избавителю вопияше: отроцы, благословите, священницы, воспойте, людие, превозносите во вся веки.} 

\pripev[]{Покой, Господи, душу усопшаго раба Твоего.} 

Скончав течение и к Тебе прибегох, Господи, преставленный вопиет ныне: прегрешения остави, Христе Боже, и не осуди мене, егда хощеши судити всем, верно бо Тебе взывах: вся дела Господня, Господа пойте и превозносите Его во веки.

\pripev[]{Покой, Господи, душу усопшаго раба Твоего.} 

Понесшаго, Владыко, иго Твое на раме Своем, и бремя Твое легкое, аще и не всегда, обаче в месте преподобных Твоих всели душу его, воспевшаго Тебе, Христе Спасе: отроцы, благословите, священницы, воспойте, людие, превозносите Его во веки. 

Благословим Отца и Сына и Святаго Духа Господа. 

Безначальная Троице Святая, Боже Отче и Сыне и Душе Святый, в лице святых причти душу преставленнаго раба Твоего и огня вечнаго избави, да Тя хвалит, воспевая во веки: отроцы, благословите, священницы, воспойте, людие, превозносите Его во веки.

\inyne 

Тя, Дево, пророчестии лицы прорекоша, прозряще бо Тя прозорливыма очима: ов убо Жезл нарече Тя, ин же Дверь Восточную, ов же Гору человеки Несекомую. Мы же исповедуем Тя воистинну Богородицу, Бога всяческих рождшую, Егоже моли упокоити преставленнаго во веки вся.

\vspace{\baselineskip}
\mysubtitle{Песнь 9}

\irmos{Ужасеся о сем небо, и земли удивишася концы, яко Бог явися человеком плотски, и чрево Твое бысть пространнейшее Небес. Тем Тя, Богородицу, Ангелов и человек чиноначалия величают.}

\pripev[]{Покой, Господи, душу усопшаго раба Твоего.} 

Иисусе Боже мой, Спасе, Адамле Ты взял еси преступление и смерти вкусил еси, да человеки от нея свободиши, Милосерде. Темже молим Тя, Милостиве: преставленнаго покой, яко Благ, во дворех святых Твоих, яко Един Всеблагий и Милосердый.

\pripev[]{Покой, Господи, душу усопшаго раба Твоего.} 

Несть никтоже, Милосерде, иже не согреши в человецех, токмо Ты Един, Иисусе Христе, вземляй грехи всего мира. Темже, очистив раба Твоего от прегрешений, вчини во святых Твоих дворех: Ты бо Живот еси и Покой, и Свет, и Веселие всех Тебе благоугодивших.

\slava

Удивися все естество человеческое, како Безначальнаго Отца Сын cый Единородный, плоть от Девы действом Святаго Духа приял еси, и пострадал еси яко человек, да умершия оживиши. Тем и преставленнаго ныне от нас, прилежно молим Тя, во стране живых, яко Благ, всели.

\inyne

Невесту Тя нарицаем, Пречистая, Отца невидимаго и Матерь Сына из Тебе Духом Святым воплощеннаго, и Молебницу Тя о усопшем рабе Твоем, предлагаем: Тебе бо Помощницу имамы земнии, и любовию поюще Тя величаем.

\Chestneyshuyu

\TrisviatoePoOtcheNash 

\vspace{\baselineskip}
\mysubtitle{Тропарь, глас 6} 

Един естеством сый Животворец, Христе, и благости воистинну неизследимая пучина, ныне преставльшагося раба Твоего Царствия Твоего сподоби: Ты бо Един еси имеяй множество щедрот и безсмертие. 

\slavainyne

\myemph{Богородичен:} Источник живота рождшая, Владычице, Избавителя миру Иисуса Господа, Того прилежно моли безконечнаго живота преставльшагося ныне раба Твоего сподобити: Ты бо христиан Едина еси известнейшая Помощница. 

Господи, помилуй \myemph{(12). И молитву сию:} 

Помяни, Господи Боже наш, в вере и надежде живота вечнаго преставльшагося раба Твоего, брата нашего \myemph{(имярек)}, и яко Благ и Человеколюбец, отпущаяй грехи, и потребляяй неправды, ослаби, остави и прости вся вольная его согрешения и невольная, избави его вечныя муки и огня геенскаго, и даруй ему причастие и наслаждение вечных Твоих благих, уготованных любящим Тя: аще бо и согреши, но не отступи от Тебе, и несумненно во Отца и Сына и Святаго Духа, Бога Тя в Троице славимаго, верова, и Единицу в Троице и Троицу во Единстве, православно даже до последняго своего издыхания исповеда. Темже милостив тому буди, и веру, яже в Тя вместо дел вмени, и со святыми Твоими яко Щедр упокой: несть бо человека, иже поживет и не согрешит. Но ты Един еси кроме всякаго греха, и правда Твоя, правда во веки, и Ты еси Един Бог милостей и щедрот, и человеколюбия, и Тебе славу возсылаем Отцу и Сыну и Святому Духу, ныне и присно, и во веки веков.

\end{mymulticols}

\mychapterending

