\mypart{МОЛИТВЫ ЗА ВОИНОВ}\label{_content_za-voinov}
%http://www.molitvoslov.org/content/za-voinov

\mychapter{Во время бедствия и нашествия врагов, иноплеменников и иноверных}
%http://www.molitvoslov.org/text239.htm

\section{Святому благоверному князю Александру Невскому}\begin{mymulticols}
%http://www.molitvoslov.org/text242.htm 

\mysubtitle{Молитва:}

Скорый помощниче всех, усердно к тебе прибегающих, и теплый наш пред Господем предстателю, святый благоверный, великий княже Александре! 

Призри милостивно на ны недостойныя, многими беззаконии непотребны себе сотворшия, к раце мощей твоих ( \myemph{или} ко святей иконе твоей) ныне притекающия и из глубины сердца к тебе взывающия. Ты в житии твоем ревнитель и защитник Православныя веры был еси: и нас в ней теплыми твоими к Богу молитвами непоколебимы утверди. Ты великое возложенное на тя служение тщательно проходил еси: и нас твоею помощию пребывати коегождо, в неже призван есть, настави. 

Ты, победив полки супостатов, от пределов Российских отгнал еси: и на нас ополчающихся всех видимых и невидимых врагов низложи. Ты, оставив тленный венец царства земнаго, избрал еси безмолвное житие, и ныне праведно венцем нетленным увенчанный на Небесех царствуеши: исходатайствуй и нам, смиренно молим тя, житие тихое и безмятежное, и к Вечному Царствию шествие неуклонное твоим предстательством устрой нам. 

Предстоя же со всеми святыми Престолу Божию, молися о всех православных христианех, да сохранит их Господь Бог Своею благодатию в мире, здравии, долгоденствии и всяком благополучии в должайшая лета, да присно славим и благословим Бога в Троице Святей славимаго, Отца и Сына и Святаго Духа, ныне и присно и во веки веков. Аминь.

\end{mymulticols}

\section{Молитва против супостатов}\begin{mymulticols}
%http://www.molitvoslov.org/text246.htm 


Господи Боже наш, послушавый Моисея, простерша к Тебе руце, и люди Израелевы укрепивый на Амалика, ополчивый Иисуса Навина на брань и повелевый солнцу стати: Ты и ныне, Владыко Господи, услыши нас, молящихся Тебе. Посли, Господи, невидимо десницу Твою, рабы Твоя заступающую во всех, а имже судил еси положити на брани души своя за веру, царя и Отечество, тем прости согрешения их, и в день праведнаго воздания Твоего воздай венцы нетления: яко Твоя Держава, Царство и Сила, от Тебе помощь вси приемлем, на Тя уповаем, и Тебе славу возсылаем, Отцу и Сыну и Святому Духу, ныне и присно и во веки веков. Аминь.

\end{mymulticols}

\section{Преподобному Иову, игумену и чудотворцу Почаевскому}\begin{mymulticols}
%http://www.molitvoslov.org/text245.htm 

\myfigure{154_0}

\mysubtitle{Тропарь, глас 4-й:}

Возложь на ся иго Христово от юности, преподобне отче Иове, многолетне свято подвизался еси на поприще благочестия во обители Угорницкой и на острове Дубенстем; и пришед к горе Почаевской, знаменанной цельбоносною стопою Пресвятыя Богородицы, в тесной пещере каменней богомыслия ради и молитвы многократно заключался еси: и благодатию Божиею укрепляяся, мужественно потрудился еси на пользу обители твоея, купно же и противу врагов православия и благочестия христианскаго. И наставив сицевому ополчению иночествующих, победители тех представил еси Владыце своему и Богу: Того моли спастися душам нашим.

\mysubtitle{Молитва:}


О преподобне отче Иове, иноков трудолюбнаго жития богомудрый наставниче, кротости и воздержания, чистоты и целомудрия, братолюбия и нищелюбия, терпения и бдения от ранней юности до поздней старости неутомимый подвижниче, веры православныя великий ревнителю и непреоборимый поборниче, земли Волынския и Галицкия светило благосветлое и святыя Почаевския обители непобедимый защитниче! Призри оком благоутробия твоего на нас недостойных чад твоих, к тебе усердно по вся дни прибегающих и на боголюбивыя люди сия, пред твоими духоносными и многоцелебными мощами собравшияся и благоговейно к тем припадающия, и испроси предстательством твоим ко Всевышнему Владыце им и нам вся, яже к животу и благочестию, полезная и благопотребная: болящия исцели, малодушныя ободри, скорбящия утеши, обидимыя заступи, немощныя подкрепи и поверженныя долу возстави, всем вся, благодатию от Бога данною, даруй, по коегождо нужде и потребе, во спасение души и во здравие телу. Вознеси угодниче Божий всемощную молитву твою о Российской державе, да будет выну мир и тишина, благочестие и благоденствие, в судах правда и милость, в советах мудрость и благое преспеяние, да утверждается же во благих человецех верность, в злых же страх и боязнь, во еже престати им от зла и творити добрая, да тако в державе Российстей царство Христово растет и множится и да прославится в нем Бог, дивный во святых своих: Емуже единому подобает всякая слава, честь и поклонение Отцу и Сыну, и Святому Духу, ныне и присно, и во веки веков. Аминь.

\myemph{В перечисленных обстоятельствах молятся также перед иконами Божией Матери: Владимирская, Казанская, «Знамение», Тихвинская.}

\end{mymulticols}

\section{Молитва перед сражением}\begin{mymulticols}
%http://www.molitvoslov.org/text247.htm 

\myfigure{156}

Спаситель мой! Ты положил за нас душу Свою, дабы спасти нас; Ты заповедал и нам полагать души свои за други наша и за ближних наших. Радостно иду я исполнити святую волю Твою и положити жизнь свою за Царя и Отечество. Вооружи мя крепостию и мужеством на одоление врагов наших, и даруй ми умрети с твердою верою и надеждою вечной блаженной жизни во Царствии Твоем.

\mysubtitle{Молитва Кресту Господню}

Да воскреснет Бог, и расточатся врази Его, и да бежат от лица Его ненавидящии Его. Яко исчезает дым, да исчезнут; яко тает воск от лица огня, тако да погибнут беси от лица любящих Бога и знаменующихся крестным знамением, и в веселии глаголющих: радуйся, Пречестный и Животворящий Кресте Господень, прогоняяй бесы силою на тебе пропятаго Господа нашего Иисуса Христа, во ад сшедшаго и поправшаго силу диаволю, и даровавшаго нам тебе Крест Свой Честный на прогнание всякаго супостата. О, Пречестный и Животворящий Кресте Господень! Помогай ми со Святою Госпожею Девою Богородицею и со всеми святыми во веки. Аминь.

\end{mymulticols}

\mychapterending

\mychapter{Об охранении жизни воинов на поле брани}
%http://www.molitvoslov.org/content/ob-ohranenii-zhizni-voinov

\section{Преподобному Сергию, игумену Радонежскому}\begin{mymulticols}
%http://www.molitvoslov.org/text257.htm 

\myfigure{159_0}

\mysubtitle{Тропарь, глас 4-й:}

И же добродетелей подвижник, яко истинный воин Христа Бога, на страсти вельми подвизался еси в жизни временней, в пениих, бдениих же и пощениих образ быв твоим учеником: темже и вселися в тя Пресвятый Дух, Егоже действием светло украшен еси: но яко имея дерзновение ко Святей Троице, поминай стадо, еже собрал еси, мудре, и не забуди, якоже обещался еси, посещая чад твоих, Сергие Преподобне отче наш.

\mysubtitle{Кондак, глас 8-й:}

Христовою любовию уязвився, Преподобне, и тому невозвратным желанием последовав, всякое наслаждение плотское возненавидел еси, и яко солнце отечеству твоему возсиял еси, тем и Христос даром чудес обогати тя. Поминай нас, чтущих пресветлую память Твою, да зовем ти: Радуйся, Сергие Богомудре.

\mysubtitle{Молитва:}

О, священная главо, пpеподобне и Богоносне отче наш Сеpгие, молитвою твоею, и веpою и любовию, яже к Богy, и чистотою сеpдца, еще на земли во обитель Пpесвятыя Тpоицы дyшy твою yстpоивый, и Ангельскаго общения и Пpесвятыя Богоpодицы посещения сподобивыйся, и даp чyдодейственный благодати пpиемый, по отшествии же твоем от земных наипаче к Богy пpиближивыйся, и небесныя силы пpиобщивыйся; но и от нас дyхом любве своея не отстyпивый, и честныя твоя мощи, яко сосyд благодати полный и пpеизливающийся, нас оставивый! Велие имея деpзновение ко Всемилостивомy Владыце, моли спасти Рабы Его, сyщей в тебе благодати Его веpyющия и к тебе с любовию пpибегающия. Испpоси нам от великодаpовитаго Бога нашего всякий даp, всем и коемyждо благопотpебен, веpы непоpочны соблюдение, гpадов наших yтвеpждение, миpа yмиpение, от глада и пагyбы избавление, от нашествия иноплеменных сохpанение, скоpбящим yтешение, недyгyющим исцеление, падшим возставление, заблyждающимся на пyть истины и спасения возвpащение, подвизающимся yкpепление, благоделающим в делах благих пpеyспеяние и благословение, младенцам воспитание, юным наставление, неведyщим вpазyмление, сиpотам и вдовицам застyпление, отходящим от сего вpеменнаго жития к вечномy благое yготование и напyтствие, отшедшим блаженное yпокоение, и вся ны споспешествyющими твоими молитвами сподоби в день стpашнаго сyда шyия части избавитися, десныя же стpаны общники быти, и блаженный оный глас Владыки Хpиста yслышати: «Пpиидите, благословеннии Отца Моего, наследyйте yготованное вам Цаpствие от сложения миpа». Аминь.

\end{mymulticols}

\section{Святителю Никите, епископу Новгородскому}\begin{mymulticols}
%http://www.molitvoslov.org/text249.htm 

\mysubtitle{Тропарь, глас 4-й:}

Насладився, Богомудре, воздержания и желание плоти твоея обуздав, на престоле святительства сел еси, и яко звезда многосветлая, просвещая верных сердца зарями чудес твоих, отче наш, святителю Никито, и ныне моли Христа Бога, да спасет души наша.

\mysubtitle{Кондак, глас 6-й:}

Архиерейства саном почтився, и чисте чистейшему предстоя, прилежно моление за люди твоя приносил еси; яко и дождь молитвою свел еси, овогда же и града запаления угасил еси. И ныне, святителю Никито, моли Христа Бога, спасти люди твоя молящиеся, да вси вопием ти: радуйся святителю, отче предивный.

\mysubtitle{Молитва:}

О, архиерею Божий, святителю Никито! Услыши нас грешных, днесь во священный храм сей стекшихся, и честному образу твоему поклоняющихся, и ко священней твоей раце припадающих, и умиленно вопиющих: якоже сидя на престоле святительства в Великом Новеграде сем, и единою належащу бездождию, дождь молитвою свел еси, и паки граду сему огненным пламенем обдержиму, молитвою избаву подал еси, тако и ныне молим тя, о святителю Христов Никито: молися ко Господу, еже избавити царствующий грады, Великий Новград сей и вся грады и страны христианские от труса, потопа, глада, огня, града, меча и от всех врагов видимых к невидимых, яко да изрядных ради молитв твоих спасаеми, славим Пресвятую Троицу, Отца и Сына и Святаго Духа, и твое милостивное предстательство, ныне и присно, и во веки веков. Аминь.

\end{mymulticols}

\section{Великомученику Георгию Победоносцу}\begin{mymulticols}
%http://www.molitvoslov.org/text256.htm 

\ifpdf
\else
\myfigure{158_1}
\fi

\mysubtitle{Тропарь, глас 4-й:}

Яко пленных свободитель, и нищих защититель, немощствующих врач, царей поборниче, победоносче великомучениче Георгие, моли Христа Бога, спастися душам нашим.

\mysubtitle{Кондак, глас 4-й:}

Возделан от Бога, показался еси благочестия делатель честнейший, добродетелей рукояти собрав себе: сеяв бо в слезах, веселием жнеши: страдальчествовав же кровию, Христа приял еси: и молитвами, святе, твоими, всем подаеши прегрешений прощение.

\mysubtitle{Величание:}

Величаем тя, страстотерпче святый великомучениче и победоносче Георгие, и чтим страдания твоя, яже за Христа претерпел еси.

\mysubtitle{Молитва:}

Святый, славный и всехвальный великомучениче Георгие! Собраннии в храме твоем и пред иконою твоею святою поклоняющийся людие, молим тя, известный желания нашего ходатаю, моли с нами и о нас умоляемаго от своего благоутробия Бога, да милостивно услышит нас, просящих Его благостыню, и не оставит вся наша ко спасению и житию нуждная прошения, и дарует стране нашей победу на сопротивныя; и паки, припадающе, молим тя, святый победоносче: укрепи данною тебе благодатию во бранех православное воинство, разруши силы востающих врагов, да постыдятся и посрамятся, и дерзость их да сокрушится, и да уведят, яко мы имеем Божественную помощь, и всем, в скорби и обстоянии сущим, многомощное яви свое заступление. Умоли Господа Бога, всея твари Создателя, избавити нас от вечнаго мучения, да прославляем Отца, и Сына, и Святаго Духа и твое исповедуем предстательство ныне, и присно, и во веки веков. Аминь.

\ifpdf
\myfigure{158_1}
\else
\fi

\end{mymulticols}

\section{Молитва о возвращении мирных времен преподобного Ефрема Сирина}\begin{mymulticols}
%http://www.molitvoslov.org/text276.htm 


Куда бегу от Тебя, Господи наш? В какой стране скроюсь от лица Твоего? Небо "--- престол Твой, земля "--- подножие Твое, в море путь Твой, в преисподней владычество Твое. Если близок уже конец мира, то не без щедрот Твоих будет кончина. 

Знаешь Ты, Господи, что неправды наши велики; и мы знаем, что велики щедроты Твои. Если не умилостивят Тебя щедроты Твои, погибли мы за беззакония наши. Не оставь нас, Господи, Господи; потому что вкушали мы Плоть и Кровь Твою. 

Когда дела каждого подвергнутся испытанию пред Тобою, Господи всяческих,"--- в это последнее время не отврати, Господи, лица Твоего от исповедавших святое Имя Твое. Отче, Сыне и Душе Святый, Утешителю, спаси нас и сохрани души наши! 

Умоляем благость Твою, Господи, отпусти нам вины наши, презри беззакония наши; отверзи нам дверь щедрот Твоих, Господи, да приидут к нам времена мирные, и по щедротам Твоим милостиво приими молитву нашу, потому что кающимся, Господи, Ты отверзаешь дверь.

\end{mymulticols}

\mychapterending

\mychapter{Молитва родителей о детях, находящихся на службе в армии}\begin{mymulticols}
%http://www.molitvoslov.org/text881.htm 


Господи Иисусе Христе, Сыне Божий, молитв ради Пречистыя Твоея Матери услыши мя, недостойную рабу [ \myemph{или} раба] \myemph{(имя).} Господи, в милостивой власти Твоей чада моя [ \myemph{или} чадо мое], рабы Твоя \myemph{(имена),} помилуй и спаси их, Имени Твоего ради. Господи, прости им все согрешения вольные и невольные, совершенные ими пред Тобою. Господи, настави их на истинный путь Твоих заповедей, и разум просвети светом Христовым во спасение души и исцеление тела. Господи, благослови их службу в армии, на суше, воздухе и в море, в пути, летании и плавании и на каждом месте Твоего владычества. Господи, сохрани их силою Честного и Животворящего Креста Твоего под кровом Твоим святым от летящей пули, стрелы, меча, огня, от смертоносной раны, водного потопления и напрасной смерти. Господи, огради их от всяких видимых и невидимых врагов, от всякой беды, зол, несчастий, предатель ства и плена. Господи, исцели их от всякой болезни и раны, от всякия скверны и облегчи их душевные страдания. Господи, даруй им благодать Духа Твоего Святаго на многие годы жизни, здравия и целомудрия во всяком благочестии и любви в мире и единодушии с окружающими их начальствующими, ближними и дальними людьми. Господи, умножь и укрепи им умственные способнос ти и телесные силы, здравы и благополучны возврати их в родительский дом. 

Всеблагий Господи, даруй мне, недостойной и грешной рабе Твоей \myemph{(имя)}, родительское благословение на чад моих \myemph{(имена)} в настоящее время утра, дня, ночи, ибо Царствие Твое вечно, всесильно и всемогущественно. Аминь. 

\end{mymulticols}

\mychapterending

