

\mypart{МОЛИТВЫ ЗА ОТЕЧЕСТВО}
%/content/zaotechestvo



\bfseries Смотреть весь раздел &rarr;\normalfont{} 

\mychapter{Молитва покаянная, читаемая в день убиения царской семьи 4 / 17 июля}
%/text275.htm



Благословен еси Господи Боже Отец наших, и хвально и прославлено имя Твое во веки, яко праведен еси о всех, яже сотворил еси нам, и вся дела Твоя истинна и правы путие Твои, и вей Суды Твои истинни, и судьбы истинни сотворил еси по всем, яже навел еси на ны, яко согрешихом и беззаконновахом отступивше от Тебе, и прегрешихом во всех, и заповедей Твоих не послушахом, ниже соблюдохом, ниже сотворихом, якоже заповедал еси нам, да благо нам будет, и предал еси нас в руки врагов беззаконных, мерзких "--- отступников, человеком неправедным и лукавнейшим паче всея земли. 

И ныне несть нам отверсти уст, спуд и поношение быхом рабом Твоим и чтущим Тя. Не предаждь же нас до конца имени Твоего ради и не разори завета Твоего, и не остави милости Твоея от нас, яко, Владыко, умалихомся паче всех язык и есмы смирени по всей земли днесь, грех ради наших и несть во время сие начальг ника, пророка и вождя. И ныне возследуем всем сердцем и боимся Тебе и ищем лица Твоего, не посрами нас, но сотвори с нами по кротости Твоей, и по множеству милости Твоея, и молитв ради Пречистыя Матери Твоея и всех святых Твоих, изми нас по чудесем Твоим, и даждь славу имени Твоему, Господи, и да посрамятся вей являющий рабом Твоим злая, и да постыдятся от всякия силы и крепость их да сокрушится, и да разумеют вси, яко Ты еси Бог наш, един и славен по всей вселенной. Аминь. 





\mychapterending

\mychapter{Молитва покаянная , читаемая в церквах России во дни смуты}
%/text274.htm



Господи, Боже, Вседержителю, призри на нас, грешных и недостойных чад Твоих, согрешивших пред Тобою, прогневавших благость Твою, навлекших гнев Твой праведный на ны, падших во глубину греховную. Ты зриши, Господи, немощь нашу и скорбь душевную, веси растление умов и сердец наших, оскудение веры, отступление от заповедей Твоих, умножение нестроений семейных, разъединения и раздоры церковный, Ты зриши печали и скорби наша, от болезней, гладов, потопления, запаления и междоусобныя брани происходящыя. Но, Премилостивый и Человеколюбивый Господи, вразуми, настави и помилуй нас, недостойных. Исправи жизнь нашу греховную, утоли раздоры и нестроения, собери расточенныя, соедини разсеянныя, подаждь мир стране нашей и благоденствие, избави ю от тяжких бед и несчастий. Всесвятый Владыко, просвети разум наш светом учения Евангельскаго, возгрей сердца наша теплотою благодати Твоея и направи я к деланию заповедей Твоих, да прославится в нас всесвятое и преславное имя Твое, Отца и Сына и Свягаго Духа, ныне и во веки веков. Аминь. 





\mychapterending

\mychapter{Молитва за русский народ}
%/text273.htm



Всемогущий Боже, Ты "--- Кто сотворил небо и землю со всяким дыханием, "--- умилосердись над бедным Русским народом и дай ему познать, на что Ты его сотворил! 

Спаситель мира, Иисусе Христе, Ты отверз очи слепорожденному "--- открой глаза и нашему Русскому народу, дабы он познал Волю Твою святую, отрекся от всего дурного и стал народом богобоязненным, разумным, трезвым, трудолюбивым и честным! 

Душе Святый, Утешителю, Ты "--- что в пятидесятый день сошел на Апостолов "--- прииди и вселися в нас! Согрей святою ревностью сердца духовных пастырей наших и всего народа, дабы свет Божественного учения разлился по Земле Русской, а с ним низошли на нее все блага земные и небесные! Аминь. 


\mychapterending

\mychapter{Молитва о спасении России}
%/text272.htm



Господи Иисусе Христе, Боже наш! Прими от нас, недостойных раб Твоих, усердное моление сие и, простив нам вся согрешения наша, помяни всех врагов наших, ненавидящих и обидящих нас, и не воздаждь им по делам их, но по велицей Твоей милости обрати их неверных ко правоверию и благочестию, верных же во еже уклоншися от зла и творит благое. Нас же всех и Церковь Твою Святую всесильною Твоею крепостию от всякаго злаго обстояния милостивне избави. Отечество наше от любых безбожник и власти их свободи, верных же раб Твоих, в скорби и печали день и ночь вопиющих к Тебе, многоболезненный вопль услыши, многомилостиве Боже наш, и изведи из истления живот их. Подаждь же мир и тишину, любовь и утверждение и скорое примирение людям Твоим, их же Честною Твоею Кровию искупил еси. Но и отступившим от Тебе и Тебе не ищущим явлен буди, воеже ни единому от них погибнути, но всем им спастися и в разум истины прийти, да вей в согласном единомыслии и в непрестанной любви прославят пречестное имя Твое, терпеливодушне, незлобиве Господи, во веки веков. Аминь. 





\mychapterending

\mychapter{Молитва о спасении державы Российской и утолении в ней раздоров и нестроений}
%/text271.htm



\itshape (Святителя Тихона, Патриарха Московского и всея  Руси)

\normalfont{}Господи Боже, Спасителю наш! К Тебе припадаем сокрушен­ным сердцем и исповедуем грехи и беззакония наша, ими же раздражихом Твое благоутробие и затворихом щедроты Твоя. Отступихом бо от Тебе, Владыко, и заповедей Твоих не соблюдохом, ниже сотворихом, яко же заповедал еси нам. Сего ради нестроени­ем поразил еси нас, и дал еси нас на попрание врагом нашим, и ума лихомся паче всех язык, и быхом в притчу и поношение соседом нашим. Боже великий и дивный, каяйся о злобах человеческих, возводяй низверженныя и утверждайя низпадающия! Небесную Твою силу с небесе низпосли, уврачуй язвы душ наших и воздвигни нас от одра болезни, яко наполнишася разслабления чресла наша, яко болим неправдою и рождаем беззаконие. Утоли шатания и раздо­ры в земли нашей, отжени от нас зависти и рвения, убийства и пианства, разжения и соблазны, попали в сердцах наших всяку нечистоту, вражду и злобу, да паки возлюбим друг друга и едино пребудем в Тебе, Господе и Владыце нашем, якоже повелел еси и за­поведал еси нам. Помилуй нас, Господи, помилуй нас, яко исполнихомся уничижения и несмы достойни возвести очеса наша на не­бо. Помяни милости, яже показал еси отцем нашим, преложи гнев Твой на милосердие и даждъ нам помощь от скорби. Молит Тя Твоя Церковь, представляющи Тебе ходатайство другов Твоих: преподобных и богоносных отцев наших Сергия Радонежскаго и Серафима Саровскаго, Петра, Алексия, Ионы, Иова, Филиппа, святителей Московских, священномученика Ермогена, страсто­терпца Царя Николая, страстотерпицы Царицы Александры, Царевича Алексия, великих княжен Ольги, Татианы, Марии, Ана­стасии, и верных слуг их, наипаче святителя Тихона, Патриар­ха Московскаго, исповедника, преподобномученицы великой кня­гини Елисаветы, ...и всех святых в земли нашей просиявших, из­ряднее же Пресвятыя Богородицы и Приснодевы Марии, от лет древних покрывавшия и заступавшия страну нашу. Вразуми всех, иже во власти суть, и возглаголи в них благая о Церкви Тво­ей и о всех людех Твоих. Силою Креста Твоего укрепи воинство на­ше и избави их от всякаго навета вражия. Воздвигни нам мужей силы и разума, и даждь всем нам Духа премудрости и страха Божия, Духа крепости и благочестия. 

Господи, к Тебе прибегаем, научи нас творити волю Твою, яко Ты еси Бог наш, яко у Тебе источник живота, во свете Твоем узрим свет. Пробави милость Твою ведущим Тя во веки веков. Аминь. 


\mychapterending