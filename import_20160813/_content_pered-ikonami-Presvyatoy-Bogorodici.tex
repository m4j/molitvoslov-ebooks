

\mypart{МОЛИТВЫ ПЕРЕД ИКОНАМИ ПРЕСВЯТОЙ БОГОРОДИЦЫ}
%/content/pered-ikonami-Presvyatoy-Bogorodici



\bfseries Смотреть весь раздел &rarr;\normalfont{} 

\mychapter{Богородичное правило}
%/text260.htm



\myfig{img/161.jpg}

Богородице Дево, радуйся, Благодатная Марие, Господь с Тобою; благословена Ты в женах и благословен Плод чрева Твоего, яко Спаса родила еси душ наших. \itshape (Читается ежедневно 150 раз).

\normalfont{}


\itshape Если по непривычке трудно будет одолевать ежедневно 150 раз, следует читать поначалу 50 раз. После каждого десятка надо читать по одному разу "Отче наш" и "Милосердия двери".\normalfont{}


\itshape Ниже приводится схема, в которую вкладывал свои молитвы Приснодеве Марии владыка Серафим (Звездинский). Исполняя Богородичное правило, он молился за весь мир и охватывал этим правилом всю жизнь Царицы Небесной. После каждого десятка читаются дополнительные молитвы, например те, которые указаны далее:\normalfont{}


\medskip


\bfseries Первый десяток.\normalfont{}


\itshape Вспоминаем Рождество Пресвятой Богородицы. Молимся о матерях, отцах и детях.

\normalfont{}О, Пресвятая Владычице Богородице, спаси и сохрани рабов Твоих \itshape (имена родителей и сродников)\normalfont{}, а умерших со святыми упокой в вечной славе Твоей.


\medskip


\bfseries Второй десяток.\normalfont{}


\itshape Вспоминаем Введение во храм Пресвятой Девы Богородицы. Молимся о заблудших и отпавших от Церкви\normalfont{}.

О, пресвятая Владычице Богородице, спаси и сохрани и соедини \itshape (или присоедини)\normalfont{} к Святой Православной Церкви заблудших и отпавших рабов Твоих \itshape (имена).\normalfont{}


\medskip


\bfseries Третий десяток.\normalfont{}


\itshape Вспоминаем Благовещение Пресвятой Богородицы. Молимся об утолении скорбей и утешении скорбящих.

\normalfont{}О, Пресвятая Владычице Богородице, утоли наши скорби и пошли утешение скорбящим и болящим рабам Твоим \itshape (имена)\normalfont{}.


\medskip


\bfseries Четвертый десяток.\normalfont{}


\itshape Вспоминаем Встречу Пресвятой Богородицы с праведной Елисаветою. Молимся о соединении разлученных, у кого близкие или дети в разлуке или пропали без вести.

\normalfont{}О, Пресвятая Владычице Богородице, соедини в разлуке находящихся рабов Твоих \itshape (имена)\normalfont{}.


\medskip


\bfseries Пятый десяток.\normalfont{}


\itshape Вспоминаем Рождество Христово, молимся о возрождении душ, о новой жизни во Христе.\normalfont{}

О, Пресвятая Владычице Богородице, даруй мне, во Христа крестившемуся, во Христа облещися.


\medskip


\bfseries Шестой десяток.\normalfont{}


\itshape Вспоминаем Сретение Господне, и пророченное святым Симеоном слово: "И Тебе Самой оружие пройдет душу". Молимся, чтобы Матерь Божия встретила душу в час кончины и сподобила при последнем издыхании причаститься Святых Таин и провела бы душу через страшные мытарства.

\normalfont{}О, Пресвятая Владычице Богородице, сподоби меня при последнем издыхании причаститься Святых Таин Христовых и Сама проведи душу чрез страшные мытарства.


\medskip


\bfseries Седьмой десяток.\normalfont{}


\itshape Вспоминаем бегство в Египет Божией Матери с Богомладенцем, молимся, чтобы Царица Небесная помогла бы избежать искушений в этой жизни и избавила бы от напастей.

\normalfont{}О, Пресвятая Владычице Богородице, не введи меня во искушение в сей жизни и избави меня от всяких напастей.


\medskip


\bfseries Восьмой десяток.\normalfont{}


\itshape Вспоминаем исчезновение двенадцатилетнего отрока Иисуса в Иерусалиме и скорбь Божией Матери по поводу этого. Молимся, испрашивая у Богоматери постоянную Иисусову молитву.

\normalfont{}О, Пресвятая Владычице Богородице, Пречистая Дево Марие, даруй мне непрестанную Иисусову молитву.


\medskip


\bfseries Девятый десяток.\normalfont{}


\itshape Вспоминаем чудо в Кане Галилейской, когда Господь претворил воду в вино по слову Божией Матери: "Вина нет у них". Просим у Божией Матери помощи в делах и избавлении от нужды.

\normalfont{}О, Пресвятая Владычице Богородице, помоги мне во всех делах и избави меня от всяких нужд и печали.


\medskip


\bfseries Десятый десяток.\normalfont{}


\itshape Вспоминаем стояние Божией Матери у Креста Господня, когда скорбь, как оружие, пронзила Ее душу. Молимся Божией Матери о подкреплении душевных сил и об отгнании уныния.

\normalfont{}О, Пресвятая Владычице Богородице, Преблагословенная Дево Марие, укрепи мои силы душевные и отжени от меня уныние.


\medskip


\bfseries Одиннадцатый десяток.\normalfont{}


\itshape Вспоминаем Воскресение Христово и молитвенно просим Божию Матерь воскресить душу и подать новую бодрость к подвигу.

\normalfont{}О, Пресвятая Владычице Богородице, воскреси душу мою и даруй мне постоянную готовность к подвигу.


\medskip


\bfseries Двенадцатый десяток.\normalfont{}


\itshape Вспоминаем Вознесение Христово, при котором присутствовала Матерь Божия. Молимся и просим Царицу Небесную вознести душу от земных суетных забав и направить на стремление к горнему.

\normalfont{}О, Пресвятая Владычице Богородице, избави меня от помышлений суетных и даруй мне ум и сердце, стремящееся ко спасению души.


\medskip


\bfseries Тринадцатый десяток.\normalfont{}


\itshape Вспоминаем сионскую горницу и сошествие Святаго Духа на апостолов и Божию Матерь и молимся: "Сердце чисто созижди во мне, Боже, и дух прав обнови во утробе моей. Не отвержи мене от лица Твоего и Духа Твоего Святаго не отыми от мене".

\normalfont{}О, Пресвятая Владычице Богородице, ниспосли и укрепи благодать Святаго Духа в сердце моем.


\medskip


\bfseries Четырнадцатый десяток.\normalfont{}


\itshape Вспоминаем Успение Пресвятой Богородицы и просим мирной и безмятежной кончины.

\normalfont{}О, Пресвятая Владычице Богородице, даруй мне мирную и безмятежную кончину.


\medskip


\bfseries Пятнадцатый десяток.\normalfont{}


\itshape Вспоминаем славу Божией Матери, которой увенчивается Она от Господа после переселения Её от земли на Небо, и молим Царицу Небесную не оставлять верных, сущих на земле, но защищать их от всякого зла, покрывая их Честным Своим Омофором.

\normalfont{}О, Пресвятая Владычице Богородице, сохрани меня от всякаго зла и покрый меня Честным Твоим Омофором.


\mychapterending

\mychapter{Пяточисленные молитвы}
%/content/Pyatochislennie-molitvi



\myfig{img/520_0.jpg}

\itshape (Читаются по благословению духовника)\normalfont{}


\bfseries Творение святителя Димитрия Ростовского\normalfont{}


Один старец от Богоносных Отцев, стоя на молитве и быв в восторзе, слышал глас Господа нашего Иисуса Христа, глаголющий с Пречистою и Пресвятою Богородицею, Материю Своею, рекши к Ней: 


"--- Рцы Мне, Мати Моя, колико болезней наибольших, живущи на свете, претерпела еси Мене ради?


И рече Пресвятая:


"--- Сыне и Боже Мой! Наибольших пять болезней Тебе ради претерпела: первая "--- егда услышала от Симеона Пророка, еже имел еси убиен быти; вторая "--- еща, ищущи Тебе во Иерусалиме через три дня, Тя не видела есмь; третья "--- егда Тя поймана и связана от жидов услышала; четвертая "--- егда Тя на кресте между разбойниками Распятаго видела; пятая "--- егда Тя видела во гробе полагаема.


И рече к Ней Господь:


"--- Глаголю Тебе, Мати Моя: аще кто каждую болезнь Твою на кийждо день прочтет с молитвою Моею, сие есть «Отче наш» и Архангельское обрадование, сие есть «Богородице Дево, радуйся» "--- за первую болезнь дам ему познание грехов и жаление о них; за вторую "--- дам ему всех грехов прощение; за третию "--- возвращу ему добродетели его, через грехи изгибшия; за четвертую "--- напитаю его, при смерти его, Телом и Кровию Моею Божественною; за пятую "--- явлюся ему Сам при смерти его и прииму душу его в живот вечный. Аминь.


По видению этого Богоносного старца, святителем Димитрием Ростовским было сложено далее следующее молитвословие.


\bfseries Начало к сим молитвам:\normalfont{}


Слава Тебе, Христу Богу моему, не погубившему мя грешнаго со беззаконьми моими, но даже доселе грехом моим потерпевшему. \itshape (Поклон)\normalfont{}


Сподоби, Господи, в день сей без греха сохранится нам; даруй ми, Господи, да ни словом, ни делом, ни помышлением прогневаю Тя, Создателя моего, но вся дела моя, советы и помышления да будут во славу Пресвятаго имени Твоего. \itshape (Поклон) \normalfont{}


Боже, милостив буди мне, грешному, во всей жизни моей: во исходе моем и по кончине моей не остави мене. \itshape (Поклон)\normalfont{}


\itshape Сие, падши ниц на земли, глаголи:\normalfont{}


Господи Иисусе Христе, Сыне Божий, приими умершаго мя душею и умом. Приими мя грешнаго, блудника, сквернаго душею и телом. Отыми неприязнь безстудную и не отврати лица Твоего от мене, не рцы, Владыко: не вем, кто еси, но вонми гласу моления моего; спаси мя, яко имеяй множество щедрот и не хощеши смерти грешника, не оставлю же Тя, Создателя моего, и не отступлю от Тебе, дондеже послушаеши мя и даси всем грехом моим прощение, молитв ради Пречистыя Матери Твоея, предстательства честных Небесных Сил безплотных, святаго славнаго Ангела хранителя моего, пророка же и предтечи Твоего Крестителя Иоанна, богоглаголивых Апостол, святых и добропобедных мученик, Преподобных и Богоносных отец наших и всех Твоих святых, помилуй и спаси мя грешнаго. Аминь.


Царю Небесный... Трисвятое... Отче наш... Яко Твое есть Царство и Сила и Слава во веки. Аминь. Богородице Дево, радуйся...


\bfseries По сем:\normalfont{}


\bfseries МОЛИТВА 1\normalfont{}


О, Мати Милосердая, Дево Марие, аз грешный и непотребный раб Твой, воспоминаючи Твоя болезни, егда Ты услышала от Симеона пророка о немилостивном убиении Сына Твоего, Господа нашего Иисуса Христа, приношу сию тебе молитву и Архангельское обрадование, приими в честь и память болезней Твоих и умоли Сына Твоего, "--- Господа нашего Иисуса Христа, да дарует ми познание грехов и жаление о них. \itshape (Поклон)\normalfont{}


\bfseries МОЛИТВА 2\normalfont{}


Отче наш... Яко Твое есть Царство... Богородице Дево, радуйся...


О, Богоблаженная и Пренепорочная Отроковице, Мати и Дево, приими от мене, грешнаго и непотребнаго раба Твоего, сию молитву и Архангельское обрадование в честь памяти болезни Твоея, егда позабыла Сына Твоего, Господа нашего Иисуса Христа, в церкви и через три дни Его не видела еси; умоли Его и испроси у Него всех грехов моих прощение и оставление, Едина Благословенная. \itshape (Поклон)\normalfont{}


\bfseries МОЛИТВА 3\normalfont{}


Отче наш... Яко Твое есть Царство... Богородице Дево, радуйся. ..


О, Мати света, Преблагословенная Дево Богородице, приими от мене, грешнаго и непотребнаго раба Твоего, сию молитву и Архангельское обрадование, в честь и память болезни Твоея, егда Сына Твоего, Господа нашего Иисуса Христа, поймана и связана слышала еси. Умоли Его, да возвратит мне добродетели, через грех изгибшия, да Тя, Пречистая, во веки величаю. \itshape (Поклон)\normalfont{}


\bfseries МОЛИТВА 4\normalfont{}


Отче наш... Яко Твое есть Царство... Богородице Дево, радуйся...


О, Милосердия Источниче, Дево Богородице, приими от мене, грешнаго и непотребнаго раба Твоего, сию молитву и Архангельское обрадование в честь и память болезни Твоея, егда на кресте между разбойники видела еси Сына Твоего, Господа нашего Иисуса Христа, Егоже умоли, Владычице, да подаст ми дар милосердия Своего в час смерти моея и да напитает мя Телом и Кровию Своею Божественною, да Тя, Заступницу, славлю во веки.\itshape  (Поклон)\normalfont{}


\bfseries МОЛИТВА 5\normalfont{}


Отче наш... Яко Твое есть Царство... Богородице Дево, радуйся...


О, Надеждо моя, Пречистая Дево Богородице, приими от мене, грешнаго и непотребнаго раба Твоего, сию молитву и Архангельское обрадование в честь и память болезни Твоея, егда видела еси Сына Твоего, Господа нашего Иисуса Христа, во гробе полагаема, Егоже умоли, Владычице, да явится мне в час смерти моея и да приимет душу мою в живот вечный. Аминь.\itshape  (Поклон) \normalfont{}


О, Премилостивая Дево, Госпоже Богородице, чадолюбивая Горлице, небесе и земли Самодержавная Царице, любезная Приемнице всех, простирающих к Тебе свои мольбы, печальных Утешительнице, приими от мене, грешнаго и непотребнаго раба Твоего, пяточисленное сие моление, в нем же воспоминаю земные и небесныя Твоя радости, умильно вопиюща к Тебе сице: 


\itshape (Земные радости)\normalfont{} Радуйся, заченшая во чреве без семене Христа Бога нашего. Радуйся, во чреве без болезни носившая Того. Радуйся, порождшая чудным смотрением. Радуйся, восприимшая от волхвов-царей дары и поклонение. Радуйся, яко обрела еси посреди учителей Своего Сына и Бога. Радуйся, яко рождество Твое преславно из мертвых воста. Радуйся, видевшая возносящагося Своего Создателя, к Нему же сама душею и телом возшла еси.


\itshape (Небесные радости)\normalfont{} Радуйся, девства ради Своего, паче ангел и всех святых Преславная. Радуйся, близ Пресвятыя Троицы просияющая. Радуйся, Миротворице наша. Радуйся, Властительнице, обладающая всеми небесными силами. Радуйся, паче всех дерзновение к Сыну и Богу Своему имущая. Радуйся, милосердая Мати всем, к Тебе прибегающим. Радуйся, яко Твое веселие во веки не скончается!


И мне недостойному, по неложному обещанию Твоему, в день исхода моего милостива предстани, да Твоим руководством управлен буду к горнему Иерусалиму, в нем же прославленно царствуеши с Сыном Твоим и Богом нашим. Ему же подобает всякая слава, честь и поклонение со Отцем и с Пресвятым Духом в безконечныя веки веков. Аминь.


От скверных устен моих, от мерзкаго сердца, от нечистаго языка и от души оскверненной приими, Госпоже Царице, сие похваление, Радосте моя. Приими, якоже вдовий прияла сии две лепты, и даруй ми принести Твоей благости дар достоин. О, Владычице моя, Пречистая Дево, Небесная Царице, якоже хощеши и волиши, паче же научи мя, что ми подобает Тебе, Матери Божией, глаголати к Тебе единой грешных прибежищу и утешению. Радуйся, Владычице, да и аз, многогрешный раб Твой, радостно зову Тебе, воспетой Матери Христа Бога нашего. Аминь.


На Твой Пресвятый образ взирая, яко истинную самую Тя зрю Богородицу, верою сердечною от души припадаю и поклоняюся с Предвечным на руку Твоею держимым Младенцем, Господом нашим Иисусом Христом, боголепно почитаю и молю Тя со слезами, покрый мя покровом Твоим, от враг видимых и невидимых, Ты бо род человеческий ввела еси в Царство Небесное. Аминь.


\bfseries По сем:\normalfont{}


Достойно есть, яко воистину... Слава и Благодарение Господу за все!


\mychapterending

\mychapter{Перед иконой &quot;Скоропослушница&quot;}
%/text823.htm



\bfseries Тропарь, глас 4-й:\normalfont{}


К Богородице притецем сущии в бедах, и святей иконе Ея ныне припадем, с верою зовуще из глубины души: скоро наше услыши моление, Дево, яко Скоропослушница нарекшаяся, Тебе бо ради раби Твои в нужду готовую помощницу имамы.


\medskip


\bfseries Кондак, глас 8-й\normalfont{}:


В мори житейстем обуреваемии, треволнению подпадаем страстей и искушений. Подаждь убо нам, Госпоже, руку помощи, якоже Петрови Сын Твой, и ускори от бед избавити ны, да зовем Ти: радуйся, всеблагая Скоропослушнице.


\medskip


\bfseries Молитва\normalfont{}:


Преблагословенная Владычице, Приснодево Богородице, Бога Слова паче всякаго слова на спасение наше рождшая, и благодать Его преизобильно паче всех приявшая, море явльшаяся Божественных дарований и чудес приснотекущая река, изливающая благость всем, с верою к Тебе прибегающим! Чудотворному Твоему образу припадающе, молимся Тебе, всещедрей Матери человеколюбиваго Владыки: удиви на нас пребогатыя милости Твоя и прошения наша, приносимая Тебе, Скоропослушнице, ускори исполнити все, еже на пользу во утешение и спасение коемуждо устрояющи. Посети, Преблагая, рабы Твоя благодатию Твоей, подаждь недугующим цельбу и совершенное здравие, обуреваемым тишину, плененным свободу, и различными образы страждущих утеши. Избави, Всемилостивая Госпоже, всяк град и страну от глада, язвы, труса, потопа, огня, меча и иныя казни временныя и вечныя, Матерним Твоим дерзновением отвращающи гнев Божий; и душевнаго разслабления, обуревания страстей и грехопадений свободи рабы Твоя, яко да непреткновенно во всяком благочестии поживше в сем веце, и в будущем вечных благ сподобимся благодатию и человеколюбием Сына Твоего и Бога, Емуже подобает всякая слава, честь и поклонение со Безначальным Его Отцем и Пресвятым Духом, ныне и присно и во веки веков. Аминь.


\mychapterending

\mychapter{Перед иконой &quot;Всех скорбящих радость&quot;}
%/text822.htm



\bfseries Тропарь, глас 2-й:\normalfont{}


Всех скорбящих радосте и обидимых заступнице, и алчущих питательнице, странных утешение, обуреваемых пристанище, больных посещение, немощных покрове и заступнице, жезле старости, Мати Бога Вышняго Ты еси, Пречистая: потщися, молимся, спастися рабом Твоим.


\medskip


\bfseries Кондак, глас 6-й:\normalfont{}


Не имамы иныя помощи, не имамы иные надежды, разве Тебе, Владычице. Ты нам помози, на Тебе надеемся и Тобою хвалимся, Твои бо есмы рабы, да не постыдимся.


\medskip


\bfseries Молитва:\normalfont{}


О, Пресвятая Владычице Богородице, Преблагословенная Мати Христа Бога Спасителя нашего, всех скорбящих Радосте, больных посещение, немощных покрове и заступнице, вдовиц и сирых покровительнице, матерей печальных всенадежная утешительнице, младенцев немощных крепосте, и всем беспомощным всегда готовая помоще и верное прибежище! Тебе, о, Всемилостивая, дадеся от Всевышняго благодать во еже всех заступати и избавляти от скорби и болезней, зане Сама лютыя скорби и болезни претерпела еси, взирающи на вольное страдание Сына Твоего возлюбленнаго и Того на кресте распинаема зрящи, егда оружие Симеоном предреченное сердце Твое пройде. Темже убо, о Мати чадолюбивая, вонми гласу моления нашего, утеши нас в скорби сущих, яко верная радости Ходатаица: предстоящи престолу Пресвятыя Троицы, одесную Сына Твоего, Христа Бога нашего, можеши, аще восхощеши, вся нам полезная испросити. Сего ради с верою сердечною и любовию от души припадаем к Тебе яко Царице и Владычице и псаломски вопити Тебе дерзаем: слыша, Дщи, и виждь, и приклони ухо Твое, услыши моление наше, и избави нас от обстоящих бед и скорбей; Ты бо прошения всех верных, яко скорбящих радость, исполняeши, и душам их мир и утешение подавши. Се зриши беду нашу и скорбь: яви нам милость Твою, посли утешение уязвленному печалию сердцу нашему, покажи и удиви на нас грешных богатство милосердия Твоего, подаждь нам слезы покаяния ко очищению грехов наших и утолению гнева Божия, да с чистым сердцем, совестию благою и надеждою несумненною прибегаем ко Твоему ходатайству и заступлению: приими, всемилостивая наша Владычице Богородице, усердное моление наше Тебе приносимое, и не отрини нас, недостойных, от Твоего благосердия, но подаждь нам избавление от скорби и болезни, защити нас от всякаго навета вражия и клеветы человеческая, буди нам помощница неотступная во все дни жизни нашея, яко да под Твоим матерним покровом всегда пребудем цели и сохранена Твоим заступлением и молитвами к Сыну Твоему и Богу Спасителю нашему, Ему же подобает всякая слава, честь и поклонение, со безначальным Его Отцом и Святым Духом, ныне и присно и во веки веков. Аминь.


\mychapterending

\mychapter{Перед иконою &quot;Живоносный источник&quot; (В пяток Светлой седмицы)}
%/text856.htm



\myfig{img/839.jpg}

О Пресвятая Госпоже Владычице Богородице! Вышши еси всех aнгел и aрхангел, и всея твари честнейши: помощнице еси обидимых, ненадеющихся надеяние, убогих заступнице, печальных утешение, алчущих кормительница, нагих одеяние, больных исцеление, грешных спасение, христиан всех поможение и заступление. О Всемилостивая Госпоже, Дево Богородице Владычице! Милостию Твоею спаси и помилуй православный народ наш, Великаго Господина и Отца нашего Святейшаго Патриарха Кирилла, преосвященныя митрополиты, архиепископы и епископы, и весь священнический и иноческий чин, власти и воинство и вся православныя xристианы ризою Твоею честною защити; и умоли, Госпоже, из Тебе без семене воплотившагося Христа Бога нашего, да препояшет нас силою Своею свыше на невидимыя и видимыя враги наша. О Всемилостивая Госпоже Владычице Богородице! Воздвигни нас из глубины греховныя, и избави нас от глада, губительства, от труса и потопа, от огня и меча, от нахождения иноплеменных и междоусобныя брани, и от напрасныя смерти, и от нападения вражия, и от тлетворных ветр, и от смертоносныя язвы, и от всякаго зла. Подаждь, Госпоже, мир и здравие рабом Твоим, всем православным христианом, и просвети им ум и очи сердечныя, еже ко спасению, и сподоби ны, грешныя рабы Твоя, Царствия Сына Твоего, Христа Бога нашего, яко держава Его благословенна и препрославленна, со Безначальным Его Отцем и с Пресвятым и Благим и Животворящим Его Духом, ныне и присно и во веки веков.


\mychapterending

\mychapter{Пред иконою &quot;Прозрение очес&quot; (В пяток Светлыя седмицы)}
%/text855.htm



О Пресвятая Госпоже Владычице Богородице! Приими недостойную молитву нашу, и сохрани всех нас,  с верою и любовию покланяющихся пречистому образу Твоему, от навета злых человек и от внезапныя смерти и даруй нам прежде конца покаяние. На моление наше умилосердися и радость вместо печали даруй. Се бо, на твой образ взирающе, яко живущей Ти с нами, молим Тя усердно: избави нас, о Мати всех скорбящих и обремененных, от всякия беды и напасти, скорби, болезни и печали и от всякого зла. И сподоби ны, недостойныя рабы Твоя, одесную стати Сына Твоего, Христа Бога нашего, во втором и славнем Его пришествии, и наследники быти Царствия Небеснаго и жизни вечныя со всеми Святыми в безконечныя веки веков.


\mychapterending

\mychapter{Перед иконой &quot;Спасительница Утопающих&quot;  (20 декабря / 2 января)}
%/text827.htm



\bfseries Молитва:\normalfont{}


Заступница усердная, Мати Господа Вышняго! Ты еси всем xристианом помощь и заступление, паче же в бедах сущим. Призри ныне с высоты святыя Твоея и на ны, с верою покланяющияся пречистому образу Твоему, и яви, молим Тя, скорую помощь Твою по морю плавающим и от ветров бурных тяжкия скорби терпящим. Подвигни и вся православныя христианы на спасение в водах утопающих, и воздаждь подвизающимся о сем богатыя милости и щедроты Твоя. Се бо, на образ Твой взирающе, Тебе, яко милостивно сущей с нами, приносим смиренная моления наша. Не имамы бо ни иныя помощи, ни инаго предстательства, ни утешения, токмо Тебе, о, Мати всех скорбящих и напаствуемых. Ты по Бозе наша Надежда и Заступница, и на Тя уповающе, сами себе, и друг друга, и весь живот наш Тебе предаем во веки веков. Аминь.


\mychapterending

\mychapter{Перед иконой &quot;Млекопитательница&quot;  (12 / 25 января)}
%/text828.htm



\myfig{img/699.jpg}

\bfseries Тропарь, глас 3-й:\normalfont{}


Без семене от Божественнаго Духа, волею же Отчею зачала еси Сына Божия, от Отца без матери прежде век суща, нас же ради из Тебе без отца бывша, плотию родила еси и Младенца млеком питала еси, темже не престай молити избавитися от бед душам нашим.


\medskip


\bfseries Кондак, глас 5-й:\normalfont{}


Душ наших чувствия очистиши, узрим на иконе таинство преславное, Творца вселенныя и Господа вышних сил, во объятиях держима и от сосцу Твоею яко Младенца питаема, и, со страхом и радостию покланяющеся Тебе и рождшемуся от Тебе Спасу нашему, воззовем: Радуйся, Владычице, жизни нашея Питательница.


\medskip


\bfseries Молитва:\normalfont{}


Приими, Госпоже Богородительнице, слезныя моления рабов Твоих, к Тебе притекающих. Зрим Тя на святей иконе, на руках носящую и млеком питающую Сына Твоего и Бога нашего, Господа Иисуса Христа. Аще и безболезненно родила еси Его, а паче матерние скорби веси и немощи сынов и дщерей человеческих зриши. Темже тепле припадающе к цельбоносному образу Твоему и умиленно сей лобызающе, молим Тя, Всемилостивая Владычице: нас, грешных, осужденных в болезнех родити и в печалех питати чада наша, милостивно пощади и сострадательно заступи, младенцы же наша, такожде и родившия их, от тяжкаго недуга и горькия скорби избави. Даруй им здравие и благомощие, да и питаемии от силы в силу возрастати будут, и питающие их исполнятся радостию и утешением, яко да и ныне предстательством Твоим из уст младенец и ссущих Господь совершит хвалу Свою. О, Мати Сына Божия! Умилосердися на матери сынов человеческих и на немощныя люди Твоя: постигающия нас болезни скоро исцели, належащия на нас скорби и печали утоли, и не презри слез и воздыханий рабов Твоих. Услыши нас в день скорби пред иконою Твоею припадающих и в день радости избавления приими благодарное хваление сердец наших. Вознеси мольбы наша ко престолу Сына Твоего и Бога нашего, да милостив будет ко грехом и немощем нашим и пробавит милость Свою ведущим Имя Его, яко да и мы, и чада наша, прославим Тя, милосердую Заступницу и верную надежду рода нашего, во веки веков. Аминь.


\mychapterending

\mychapter{Молитва пред иконою &quot;Утоли моя печали&quot; (25 января / 7 февраля)}
%/text849.htm



\bfseries Тропарь, глас 5-й:\normalfont{}


Утоли болезни многовоздыхающия души моея, Утолившая всяку слезу от лица земли: Ты бо человеком болезни отгониши и грешных скорби разрушаеши. Тебе бо вси стяжахом надежду и утверждение, Пресвятая Мати Дево.


\medskip


\bfseries Кондак, глас 6-й:\normalfont{}


Не ввери мя человеческому предстательству, Пресвятая Владычице, но приими моление раба Твоего, скорбь бо обдержит мя, терпети не могу демонскаго стреляния, покрова не имам, ниже где прибегну, окаянный, всегда побеждаем, и утешения не имам, разве Тебе, Владычице мира, упование и предстательство верных, не презри моление мое, полезно сотвори.


\medskip


\bfseries Молитва:\normalfont{}


Надеждо всех концев земли, Пречистая Дево, Госпоже Богородице, утешение мое! Не гнушайся нас грешных, на Твою бо милость уповаем, в нас пламень и покаянием ороси изсохшая сердца наша, очисти ум наш от греховных помыслов, приими мольбы, от души и сердца со воздыханием Тебе приносимые. Буди о нас ходатаица к Сыну Твоему и Богу, и отврати гнев Его матерними Твоими молитвами, душевныя и телесныя язвы исцели, Госпоже Владычице, утоли болезнь души и тела, утиши бурю злых нападений вражеских, отъими бремя грехов наших, и не остави нас до конца погибнути, и печалию сокрушенная сердца наша утеши, да славим Тя до последняго издыхания нашего. Аминь.


\mychapterending

\mychapter{Перед иконой Иверскою (13 / 26 Октября, 12 /  25 Февраля и Вторник Светлой Седмицы)}
%/text267.htm



\myfig{img/164_0.jpg}

\bfseries Тропарь, глас 1-й:\normalfont{}


От святыя иконы Твоея, о Владычи­це Богородице, исцеления и цельбы подаются обильно, с верою и любовию приходящим к ней: тако и мою немощь посети, и душу мою помилуй, Благая, и тело исцели благодатию Твоею, Пречистая.


\medskip


\bfseries Кондак, глас 8-й:\normalfont{}


Аше и в море ввержена бысть икона Твоя, Богородице, от вдовицы не могущия спасти ея от врагов: но явилася Хранительница Афона, и Вратарница обители Иверския, врагов устрашающая, и в православной Российстей стране, чтущих Тя от всех бед и напастей избавляющая.


\medskip


\bfseries Величание:\normalfont{}


Величаем Тя, Пресвятая Дево, Богоизбранная Отроковице, и чтим образ Твой святый, имже точиши исцеления всем с верою притекающим.


\medskip


\bfseries Молитва:\normalfont{}


О Пресвятая Дево, Мати Христа Бога нашего, Царице Небесе и земли! Вонми многоболезненному воздыханию душ наших, призри с высоты святыя Твоея на нас, с верою и любовию покланяющихся пречистому (и чудотворному) образу Твоему. Се бо, грехми погружаемии и скорбьми обуреваемии, взирающе на Твой образ, яко живей Тебе сущей с нами, приносим смиренная моления наша. Не имамы бо иныя помощи, ни инаго предстательства, ни утешения, токмо Тебе, о Мати всех скорбящих и обремененных! Помози нам немощным, утоли скорби наша, настави на путь правый нас заблуждающих, уврачуй болезненнаясердца наша и спаси безнадежных, даруй нам прочее время жития нашего в мире и покаянии проводити, подаждь христианскую кончину и на Страшнем Суде Сына Твоего явися нам милосердая предстательница, да всегда поем, величаем и славим Тя, яко благую Заступницу рода христианскаго, со всеми угодившими Богу, во веки веков. 


\mychapterending

\mychapter{Молитва пред иконою &quot;Державная&quot;  (2/15 марта)}
%/text850.htm



\myfig{img/716.jpg}

\bfseries Тропарь, глас 8-й:\normalfont{}


Ангелов лицы благоговейно Тебе служат и вся Небесныя силы немолчными гласы Тя ублажают, Богородице Дево; усердно молим Тя, Владычице, да пребудет Божественная благодать на честней иконе Твоей Державной, и светозарный луч славы чудес Твоих да нисходит от нее на всех, с верою Тебе молящихся и вопиющих к Богу: Аллилуиа.


\medskip


\bfseries Кондак, глас 8-й:\normalfont{}


Взбранной Воеводе Победительныя песни приносим, яко даровася нам держава Твоя, и ничесоже устрашимся, не от мира бо спасение наше, но превознесенныя Владычице милосердием ограждаемся и тому днесь радуемся, яко прииде Заступнице на стражу земли Своея.


\medskip


\bfseries Молитва:\normalfont{}


О, мира Заступнице, Мати Всепетая! Со страхом, верою и любовию припадающе пред честною иконою Твоею Державною, усердно молим Тя: не отврати лица Твоего от прибегающих к Тебе, умоли, милосердная Мати Света, Сына Твоего и Бога нашего, Сладчайшаго Господа Иисуса Христа, да сохранит в мире страну нашу, да утвердит державу нашу в благоденствии, и избавит нас от междоусобныя брани, да укрепит Святую Церковь нашу Православную, и незыблему соблюдет ю от неверия, раскола и ересей. Не имамы иныя помощи, разве Тебе, Пречистая Дево. Ты еси всесильная христиан заступница пред Богом, праведный гнев Его умягчая. Избави всех с верою Тебе молящихся от падений греховных, от навета злых человек, от глада, скорбей и болезней, даруй нам дух сокрушения, смирение сердца, чистоту помышлений, исправление греховныя жизни и оставление согрешений наших; да вси, благодарственне воспевающе величия Твоя, сподобимся Небеснаго Царствия, и тамо со всеми святыми прославим пречестное и великолепое имя в Троице славимаго Бога, Отца и Сына и Святаго Духа. Аминь.


\mychapterending

\mychapter{Молитва пред иконою &quot;Споручница грешных&quot; (7/20 марта)}
%/text851.htm



\myfig{img/717.jpg}

\bfseries Тропарь, глас 4-й:\normalfont{}


Умолкает ныне всякое уныние и страх отчаяния исчезает, грешницы в скорби сердца обретают утешение и небесною любовию озаряются светло: днесь бо Матерь Божия простирает нам спасающую руку и от Пречистаго образа Своего вещает, глаголя: Аз Споручница грешных к Моему Сыну, Сей дал Мне за них руце слышати Мя выну. Темже людие, обремененнии грехи многими, припадите к подножию Ея иконы со слезами вопиюще: Заступнице мира, грешным Споручнице, умоли Матерними молитвами Избавителя всех, да Божественным всепрощением покрыет грехи наша, и светлыя двери райския отверзет нам, Ты бо еси предстательство и спасение рода христианскаго.


\medskip


\bfseries Кондак, глас 1-й:\normalfont{}


Честное Жилище бывши неизреченнаго естества Божественнаго выше слова и паче ума и грешным еси Споручница, подаваеши благодать и исцеления, яко Мати всех Царствующаго, моли Сына Твоего получити нам милость в день судный.


\medskip


\bfseries Молитва:\normalfont{}


О, Владычице Преблагословенная, защитнице рода христианскаго, прибежище и спасение притекающих к Тебе! Вем, воистинну вем, яко зело согреших и прогневах, Премилостивая Госпоже, рожденнаго плотию от Тебе Сына Божия, но имам многия образы прежде мене прогневавших Его милосердие: мытари, блудницы и прочия грешники, имже дадеся прощение грехов их, покаяния ради и исповедания. Тыя убо образы помилованных очесем грешныя души моея представляя и на толикое Божие милосердие, оныя приемшее, взирая, дерзнух и аз грешный прибегнути с покаянием ко Твоему благоутробию. О, Всемилостивая Владычице! Подаждь ми руку помощи и испросиши у Сына Твоего и Бога, Mатерними и святейшими Твоими молитвами, тяжким моим грехом прощение. Верую и исповедую, яко Той, Егоже родила еси, Сын Твой есть воистинну Христос, Сын Бога Живаго, Судия живых и мертвых, воздаяй комуждо по делом его. Верую же паки и исповедую Тебе быти истинную Богородицу, милосердия источник, утешение плачущих, взыскание погибших, сильную и непрестающую к Богу Ходатаицу, зело любящую род христианский, и споручницу покаяния. Воистинну бо несть человеком иныя помощи и покрова, разве Тебе, Госпоже Премилостивая, и никтоже уповая на Тя постыдится когда, и Тобою умоляя Бога, никтоже оставлен бысть. Того ради Твою неисчетную благость молю: отверзи двери милосердия Твоего мене заблуждшему и падшему в тимение глубины, не возгнушайся мене сквернаго, не презри грешнаго моления моего, не остави мене окаяннаго, яко в погибель злобный враг похитити мя ищет, но умоли о мне рожденнаго от Тебе милосердаго Сына Твоего и Бога, да простит великия моя грехи, и избавит мя от пагубы моея, яко да и аз со всеми получившими прощение воспою и прославлю безмерное милосердие Божие и Твое непостыдное о мне заступление в жизни сей и в нескончаемем веце. Аминь.


\mychapterending

\mychapter{Молитва пред иконою &quot;Слове плоть бысть&quot;  (9/22 марта)}
%/text852.htm



Богородице Дево, Пренепорочная Мати Христа Бога нашего, Заступнице рода христианскаго! Пред чудотворною иконою Твоею предстояще, отцы наши молиша Тя, да явиши покров Твой и заступление стране Приамурстей. Темже и мы ныне молим Тя: град наш и страну сию от нахождения иноплеменных соблюди и от междоусобныя брани сохрани. Даруй мирови мир, земли плодов изобилие, сохрани во святыни пастырей наших, во святых храмех труждающихся, осени вседержавным покровом Твоим строителей и благотворителей их. Утверди в правоверии и единомыслии братии наших; заблуждших же и отступльших от православныя веры вразуми и соедини Святей Церкви Сына Твоего. Буди всем притекающим к чудотворней  иконе Твоей покров, утешение и пристанище от всяких зол, бед и обстояний: Ты бо еси больных исцеление, скорбящих утешение, заблуждших исправление и вразумление. Приими моления наша и вознеси я ко Престолу Всевышняго, яко да Твоим предстательством соблюдаеми и покровом Твоим осеняеми, прославим Отца и Сына и Святаго Духа, ныне и присно и во веки веков. Аминь. 


\mychapterending

\mychapter{Молитва пред иконою Феодоровскою (14 / 27 марта)}
%/text853.htm



\bfseries Тропарь, глас 4-й:\normalfont{}


Пришествием честныя Твоея иконы, Богоотроковице, обрадованный днесь, богохранимый град Кострома, якоже древний Израиль к кивоту завета, притекает ко изображению лица Твоего и воплотившагося от Тебе Бога нашего, да Твоим Матерним к нему предстательством присно ходатайствуеши всем под сень крова Твоего прибегающим мир и велию милость.


\medskip


\bfseries Кондак, глас1-й:\normalfont{}


Взбранней Воеводе, Пренепорочней Деве Богородице, Заступнице нашей и Предстательству христиан непостыдному, явлением чудныя иконы Своея радование подавшей земле Российстей и вся верныя чада Церкве просветившей, благодарение усердно приносим Ти, Богородице, и припадающе пречудному образу Твоему, умильно глаголем. Спаси, Госпоже, и помилуй рабы Твоя, зовущия: Радуйся, Матерь Божия, Предстательнице и Заступнице наша усердная.


\medskip


\bfseries Молитва:\normalfont{}


К кому воззову, Владычице, к кому прибегну в печали моей; к кому принесу слезы и воздыхания моя, аще не к Тебе, Царице Небеси и земли: кто исторгнет мя от тины грехов и беззаконий, аще не Ты, о Мати Живота, Заступнице и Прибежище рода человеческаго. Услыши стенание мое, утеши мя и помилуй в горести моей, защити в бедах и напастях, избави от озлоблений и скорбей, и всяких недугов, и болезней, от враг видимых и невидимых, умири вражду стужающих мне, да избавлен буду от клеветы и злобы человеческия; такожде от своея ми плоти гнусных обычаев свободи мя. Укрый мя под сению милости Твоея, да обрящу покой и радость и от грехов очищение. Твоему Матернему заступлению себе вручаю; буди мне Мати и надеждо, покров, и помощь, и заступление, радость и утешение, и скорая во всем Помощнице. О, чудная Владычице! Всяк притекает к Тебе, без Твоея всесильныя помощи не отходит; сего ради и аз недостойный к Тебе прибегаю, да избавлен буду от внезапныя и лютыя смерти, скрежета зубнаго и вечнаго мучения. Небесное же Царствие получити сподоблюся и Тебе во умилении сердца реку: Радуйся, Мати Божия, Предстательница и Заступница наша усердная, во веки веков. Аминь.


\mychapterending

\mychapter{Молитва пред иконою Виленскою, именуемую &quot;Одигитрия&quot;  (14 / 27 Апреля)}
%/text857.htm



О Всемилостивая Госпоже, Царице Богородице, от всех родов избранная и всеми роды небесными ублажаемая! Воззри милостиво на предстоящия пред святою иконою Твоею люди сия, усердно молящияся Тебе, и сотвори предстательством Твоим и заступлением у Сына Твоего и Бога нашего, да никтоже отыдет от места сего тощ упования своего и посрамлен в надежде своей,  но да приимет кийждо от Тебе вся по благому изволению сердца своего, по нужде и потребе своей, во спасение души и во здравие телу. Наипаче же осени и огради покровом Твоим, Милосердная Мати, Церковь Твою Святую, вышним Твоим благословением архиереи наши православныя укрепи, миром огради, и Святей Твоей Церкви целых, здравых, честных, долгоденствующих право правящих слово Своея истины даруй, от всех же видимых и невидимых враг, со всеми православными христианы, милостиво избави, и во православии и твердей вере до конца веков непоступно и неизменно сохрани. Призирай благосердием, Всепетая, и призрением милостиваго Твоего заступления на всю страну нашу Российскую, грады наша и град сей и духовный вертоград зде сущий, и на сия богатыя Твоя милости неоскудно изливай. Ты бо еси всесильная Помощница и Заступница всех нас. Приклонися к молитвам и всех раб Твоих, притекающих зде ко святей иконе Твоей зде притекающих, услыши воздыхания и гласы, имиже раби Твои молятся во святем месте сем. Аще же и иноверный, и иноплеменник, зде преходя, помолится, услыши, чадолюбивая Госпоже, и сего человеколюбне и милостивно соделай, яже к помощи ему и ко спасению. Ожесточенныя же и разсеянныя сердцы своими во странах наших на путь истины настави: отпадшия от благочестивыя веры обрати и паки святей Православней Кафолической Церкви сопричти. В домех людей Твоих и во братии святыя обители сея  мир огради и соблюди, в юных братство и смиренномудрие утверди, старость поддержи, отроки настави, в возрасте совершеннем сущия умудри, сирыя и вдовицы заступи, утесненныя и в скорбех сущия утеши и охрани, младенцы воспитай, болящия уврачуй, плененныя свободи, ограждающи ны присно от всякаго зла благостию Твоею, и утеши милостивным Твоим посещением и вся благодеющия нам. Даруй же, Благая, земли плодоносие, воздуху благорастворение и вся, яже на пользу нашу, дары благовременныя и благопотребныя, всемощным 

Твоим предстательством пред Всесвятою Живоначальною Троицею, в Ея же храме предстоим еси, нам помогающи. Прежде отшедшия отцы и матери, братию и сестры наша, и вся от лет древних ко святей иконе Твоей сей припадавшия, упокой в селениих святых, в месте злачне, в месте покойне, идеже несть печаль и воздыхание. Егда же приспеет и наше от жития сего отшествие и к вечной жизни преселение, предстани нам, Преблагословенная Дево, и даруй христианскую кончину жития нашего, безболезненну, непостыдну, мирну и Святых Тайн причастну, да и в будущем веце сподобимся вси, купно со всеми святыми, безконечныя блаженныя жизни во Царствии возлюбленнаго Сына Твоего, Господа и Бога нашего Иисуса Христа, Емуже подобает слава, честь и поклонение со Отцем и Святым Духом, во веки веков. Аминь.


\mychapterending

\mychapter{Перед иконой &quot;Нечаянная Радость&quot;  (9/22 декабря и 1/14 мая)}
%/text826.htm



\bfseries Тропарь, глас 4-й\normalfont{}:


Днесь вернии людие духовно торжествуем, прославляюще Заступницу усердную рода христианскаго и притекающе к Пречистому Ея образу, взываем сице: о, Премилостивая Владычице Богородице, подаждь нам нечаянную радость, обремененным грехи и скорбьми многими, и избави нас от всякаго зла, молящи Сына Твоего, Христа Бога нашего, спасти души наша.


\medskip


\bfseries Кондак, глас 6-й\normalfont{}:


Не имамы иныя помощи, не имамы иныя надежды, разве Тебе, Владычице, Ты нам помози, на Тебя надеемся и Тобою хвалимся, Твои бо есмы рабы, да не постыдимся.


\medskip


\bfseries Молитва\normalfont{}:


О, Пресвятая Дево, Всеблагаго Сына Мати Всеблагая, града сего Покровительнице, всех сущих во гресех, скорбех, бедах и болезнех верная Предстательнице и Заступнице! Приими молебное пение сие от нас, недостойных рабов Твоих, Тебе возносимое: и якоже древле грешника, на всяк день многажды пред честною иконою Твоею молившагося, не презрела еси, но нечаянную радость покаяния даровала еси, и усердным к Сыну Твоему ходатайством Сего ко прощению грешника преклонила еси тако и ныне не презри моления нас, недостойных раб Твоих, но умоли Сына Твоего и Бога нашего, да и всем нам, с верою и умилением пред цельбоносным образом Твоим покланяющимся, по коегождо потребе нечаянную радость дарует: да вси на небеси и на земли ведят Тя, яко твердую и непостыдную Предстательницу рода христианского, и сие ведуще, славят Тя и Тобою Сына Твоего со Безначальным Его Отцем и Единосущным Его Духом, ныне и присно и во веки веков. Аминь.


\mychapterending

\mychapter{Молитва пред иконою  &quot;Неупиваемая Чаша&quot;  ( 5 / 18 Мая)}
%/text858.htm



\bfseries Тропарь, глас 4-й:\normalfont{}


Днесь притецем вернии к Божественному и пречудному образу Пресвятыя Богоматери, напaяющей верных сердца небесною Неупиваемою Чашею Своего милосердия, и людем верным чудеса показующей. Яже мы видяще и слышаще духовно празднуем и тепле вопием: Владычице Премилостивая, исцели наша недуги и страсти, молящи Сына Твоего, Христа Бога нашего, спасти души наша.


\medskip


\bfseries Кондак:\normalfont{}


Избранное и дивное избавление нам даровася "--- Твой образ честный, Владычице Богородице, яко избавльшеся явлением его от недугов душевных и телесных и скорбных обстояний, благодарственная хваления приносим Ти, Всемилостивая Заступнице. Ты же, Владычице, Неупиваемою Чашею нами именуемая, приклонися благоутробно к нашим воздыханиям и воплем сердечным, и избавление подаждь страждущим недугом пианства, да с верою воззовем Ти: Радуйся, Владычице, Неупиваемая Чаше, духовную жажду нашу утоляющая.


\medskip


\bfseries Молитва:\normalfont{}


О премилосердая Владычице! К Твоему заступлению ныне прибегаем, молений наших не презри, но милостивно услыши нас, жен, детей, матерей и тяжким недугом пианства одержимых, и того ради от матере своея, Церкве Христовы, и спасения отпадающих братий, сестер и сродник наших исцели. О милостивая Мати Божия, коснися сердец их и скоро возстави от падений греховных и ко спасительному воздержанию приведи их. Умоли Сына Твоего Христа Бога нашего, да простит нам согрешения наша и не отвратит милости Своея от людей Своих, но да укрепит нас в трезвении и целомудрии. Приими, Пресвятая Богородице, молитвы матерей, о чадех своих слезы проливающих, жен, о мужех своих рыдающих, чад сирых и убогих, заблуждшими оставленных, и всех нас, к иконе Твоей припадающих, и да приидет сей вопль наш, молитвами Твоими, ко престолу Всевышняго. Покрый и соблюди нас от лукаваго ловления и всех козней вражиих. В страшный же час исхода нашего помози пройти непреткновенно воздушныя мытарства, молитвами Твоими избави нас вечнаго осуждения, да покрыет нас милость Божия в нескончаемые веки веков. Аминь.


\mychapterending

\mychapter{Молитва пред иконою  &quot;Владимирская&quot;  ( 21 Мая / 3 Июня, 23 Июня / 6 Июля и 26 Августа / 8 Сентября)}
%/text859.htm



\myfig{img/727.jpg}

\bfseries Тропарь, глас 4-й:\normalfont{}


Днесь светло красуется славнейший град Москва, яко зарю солнечную восприимши, Владычице, чудотворную Твою икону, к нейже ныне мы, притекающе и молящеся, Тебе взываем сице: о, пречудная Владычице Богородице, молися из Тебе воплощенному Христу Богу нашему, да избавит град сей и вся грады и страны христианския невредимы от всех навет вражиих и спасет души наша, яко Милосерд. 


\medskip


\bfseries Кондак, глас 8-й:\normalfont{}


Взбранной Воеводе победительная, яко избавльшеся от злых пришествием Твоего честнаго образа, Владычице Богородице, светло сотворяем празднество сретения Твоего и обычно зовем Ти: радуйся, Невесто Неневестная.


\medskip


\bfseries Молитва:\normalfont{}


К кому возопием, Владычице? К кому прибегнем в горести нашей, аще не к Тебе, Царице Небесная? Кто плач наш и воздыхание приимет, аще не Ты, Пренепорочная, надеждо христиан и прибежище нам грешным? Кто в паче Тебе, в милости? Приклони ухо Твое к нам, Владычице, Мати Бога нашего, и не презри требующих Твоея помощи: услыши стенание наше, подкрепи нас грешных, вразуми и научи, Царице Небесная, и не отступи от нас, раб Твоих, Владычице, за роптание наше, но буди нам Мати и Заступница, и вручи нас милостивому покрову Сына Твоего. Устрой о нас, како угодно будет святей Твоей воли, и приведи нас грешных к тихой и безмятежной жизни, да плачем о гресех наших, да возрадуемся же с Тобою всегда, ныне и присно и во веки веков. Аминь.


\mychapterending

\mychapter{Молитва пред иконою &quot;Недремлющее Око&quot;  ( 29 Мая / 11 Июня)}
%/text860.htm



О, Пресвятая и Преблагая Владычице Богородице, чистейшая Святаго Духа приемнице и непорочное Слова Божия святилище! Радуйся, яко Ты еси радость ангелом и человеком, Сладчайшего бо Иисуса Христа нашествием Духа Святаго родила еси плотию, и яко Младенца Того пеленами повила еси, яко возлюбленного Сына млеком питала еси и обнимающи лобзала еси. Молим тя, Великая Владычице: не отрини раб Твоих, сему чудотворному образу предстоящих и всю надежду свою по Бозе на Тя возложивших: Ты бо еси надежда и прибежище наше, жизнь и заступление наше, крепость наша и отрада. Умоли возлюбленного Сына Твоего, еже сохранитися отечеству нашему и всякой стране христианстей в мире и безмятежии и утвердитися державе Российстей, соблюдатися же непоколебимой Святей Православней Церкви нашей, неверию искоренитися, ересем и расколом упразднитися, пастырем нашим в ревности укрепитися, еже право правити слово истины. Паки умоли, Госпоже, возлюбленного Сына Твоего, да избавит всех нас от всякия нужды, скорби и болезни, да ниспослет нам тихое и мирное житие, здравие телесное и душевное спасение, да умирит страждущая сердца наша и да исцелит сердечные скорби, да вразумит нас на добрая дела, да очистит умы от помыслов суетных и наставит к исполнению заповедей Своих, и от вечныя муки да избавит, да дарует нам Небесное Царствие, в нем же возвеличим Тя, Преславную Владычицу нашу, и прославим пречестное и великолепное Имя Великаго Бога, Отца и Сына и Святаго Духа, ныне и присно и в бесконечныя веки веков.


\mychapterending

\mychapter{Молитва пред иконою &quot;Достойно есть&quot;  (11 / 24 Июня)}
%/text861.htm



\myfig{img/729.jpg}

О, Пресвятая Дево Богородице, в скорбях наших едино утешение! Тобою, Дево Богородице, пение послушника афонского чудесне дополнися: восполни убо и наша силы духовные, яко да возможем достойно пети Тя, воистину рождшую Спасителя миру. Вознеси моление к Сыну Твоему о нас, да не оскудеет вера в нас грешных, юже диавол исхитити тщится в рабех твоих и во всем христианском роде. О сих сокрушенным сердцем помолимся. Помози нашему маловерию, еже в нас дела злонравныя посеявает, самолюбие же и гордыню, сребролюбие и сластолюбие, невоздержание и пианство, от нихже да избавимся предстательством Твоим, Пречистая, и да сподобимся спасение получити благодатию Господа нашего Иисуса Христа, Ему же подобает честь, благодарение и поклонение, со Безначальным Его Отцем и со Святым и Единосущным Ему Духом, во веки веков.


\mychapterending

\mychapter{Молитва пред иконою Тихвинской  (26 Июня / 9 Июля)}
%/text862.htm



\bfseries Тропарь, глас 4-й:\normalfont{}


Днесь, яко солнце пресветлое, возсия нам на воздусе всечестная икона Твоя, Владычице, лучами милости мир просвещающи, юже великая Россия, яко некий дар Божественный, свыше благоговейно восприемши, прославляет Тя, Богомати, всех Владычицу, и от Тебе Рождшагося Христа Бога нашего величает радостно. Емуже молися, о, Госпоже Царице Богородице, да сохранит вся грады и страны христианския невредимы от всех навет вражиих, и спасет верою покланяющихся Его Божественному и Твоему Пречистому образу, Дево Неискусобрачная.


\medskip


\bfseries Кондак, глас 8-й:\normalfont{}


Притецем, людие, к Деве Богородице Царице, благодаряще Христа Бога, и, к Тоя чудотворней иконе умильно взирающи, припадем и возопием Ей: о, Владычице Марие, присетивши страну сию Твоего честнаго образа явлением чудесно, спасай в мире и благовременстве отечество наше и вся христианы, наследники показующи Небесныя жизни, Тебе бо верно зовем: радуйся, Дево, мира спасение.


\medskip


\bfseries Молитва:\normalfont{}


Благодарим Тя, о Преблагая и Пречистая, Преблагословенная Дево, Владычице, Мати Христа Бога нашего, о всех благодеяниих Твоих, яже показала еси роду человеческому, наипаче же нам, христоименитым людем российским, о нихже ниже самый ангельский язык к похвалению доволен будет. Благодарим Тя, якоже и ныне удивила еси неизреченную Твою милость на нас, недостойных рабех Твоих, преестественным самопришествием Пречистыя Твоея иконы, ею же всю просветила еси Российскую страну. Темже и мы, грешнии, со страхом и радостию покланяющеся, вопием Ти: о Пресвятая Дево, Царице и Богородице, спаси и помилуй люди Твоя, и подаждь им победы на вся враги их, и сохрани царствующия грады, и вся грады и страны христианския, и сей святый храм от всякаго навета вражия избави, и всем вся на пользу даруй, ныне пришедшим с верою и молящимся рабом Твоим, и поклоняющимся Пресвятому образу Твоему, яко благословенна еси с рождшимся от Тебе Сыном и Богом, ныне и присно и во веки веков. Аминь.


\mychapterending

\mychapter{Молитва пред иконою &quot;Троеручица&quot;  (28 Июня / 11 Июля)}
%/text863.htm



\bfseries Тропарь, глас 4-й:\normalfont{}


Днесь всемирная радость возсия нам велия: даровася святей горе Афонстей цельбоносная Твоя, Владычице Богородице, икона, со изображением тричисленно и нераздельно пречистых рук Твоих, в прославление Святыя Троицы, созываеши бо верных и молящихся Тебе о сем познати, яко двема имаши Сына и Господа держиши, третию же яви на прибежище и покров чтущим Тя от всяких напастей и бед избавляти, да вси, притекающий к Тебе верою, приемлют неоскудно от всех зол свобождение, от врагов защищение, сего ради и мы вкупе с Афоном вопием: радуйся, Благодатная, Господь с Тобою. 


\medskip


\bfseries Кондак, глас 8-й:\normalfont{}


Днесь веселое наста ныне Твое торжество, Богомати Пречистая, вси вернии исполнишася радости и веселия, яко сподобльшеся изрядно воспевати предивное явление честнаго образа Твоего и рождшагося от Тебе Младенца, истинна же Бога, Егоже двема рукама объемлеши, и третиею от напастей и бед нас изымаеши и избавляеши от всех зол и обстояний.


\medskip


\bfseries Молитва:\normalfont{}


О, Пресвятая Госпоже Владычице Богородице, велие чудо святому Иоанну Дамаскину явившая, яко веру истинную "--- надежду несумненную показавшему! Услыши нас, грешных, пред чудотворною Твоею иконою усердно молящихся и просящих Твоея помощи: не отрини моления сего многих ради прегрешений наших, но, яко Мати милосердия и щедрот, избави нас от болезней, скорбей и печалей, прости содеянныя нами грехи, исполни радости и веселия всех, чтущих святую икону Твою, да радостно воспоем и любовию прославим имя Твое, яко Ты еси от всех родов избранная и благословенная во веки веков. Аминь.


\medskip


\bfseries Молитва иная:\normalfont{}


О, Пресвятая и Преблагословенная Дево, Богородице Марие! Припадаем и покланяемся Тебе пред святою иконою Твоею, воспоминающе преславное чудо Твое, исцелением усеченныя десницы преподобнаго Иоанна Дамаскина, от иконы сей явленное, егоже знамение доныне видимо есть на ней, во образе третия руки, ко изображению Твоему приложенныя. Молимся Ти и просим Тя, Всеблагую и Всещедрую рода нашего Заступницу: услыши нас, молящихся Тебе, и якоже блаженнаго Иоанна, в скорби и болезни к Тебе возопившаго, услышала еси, тако и нас не презри, скорбящих и болезнующих ранами страстей многоразличных, не презри, усердно к Тебе от души сокрушенныя прибегающих. Ты зриши, Госпоже Всемилостивая, немощи наша, озлобление наше, нужду, во Твоей помощи потребу, яко отвсюду врази окружают нас, и несть помогающего, ниже заступающаго, аще не Ты умилосердишися о нас, Владычице. Ей, молим Ти ся, вонми гласу болезненному нашему и помози нам святоотеческую Православную веру до конца дней наших непорочно сохранити, во всех заповедех Господних неуклонно ходити, покаяние истинное о гресех наших всегда Богу приносити и сподобитися мирныя христианския кончины и добраго ответа на страшнем суде Сына Твоего и Бога нашего. Егоже умоли за нас матернею молитвою Твоею, да не осудит нас по беззаконием нашим, но да помилует нас по велицей и неизреченной милости Своей. О, Всеблагая! Услыши нас и не лиши нас помощи Твоея державныя, да, Тобою спасение получивше, воспоем и прославим Тя на земли живых и Рождшагося от Тебе Искупителя нашего, Господа Иисуса Христа, Емуже подобает слава и держава, честь и поклонение, купно со Отцем и Святым Духом, всегда, ныне и присно и во веки веков. Аминь.


\mychapterending

\mychapter{Перед иконою Казанскою (22 Октября / 4 Ноября и 8/21 июля)}
%/text821.htm



\myfig{img/691.jpg}

\bfseries Тропарь, глас 4-й:\normalfont{}


Заступнице усердная, Мати Господа Вышняго! За всех молиши Сына Твоего, Христа Бога нашего, и всем твориши спастися, в державный Твой покров прибегающим. Всех нас заступи, о. Госпоже, Царице и Владычице, иже в напастех и в скорбех и в болезнех, обремененных грехи многими, предстоящих и молящихся Тебе умиленною душею и сокрушенным сердцем пред Пречистым Твоим образом со слезами, и невозвратно надежду имущих на Тя избавления всех зол. Всем полезная даруй и вся спаси, Богородице Дево: Ты бо еси Божественный покров рабом Твоим.


\medskip


\bfseries Кондак, глас 8-й\normalfont{}:


Притецем, людие, к тихому сему и доброму пристанищу, скорой Помощнице, готовому и теплому спасению, покрову Девы; ускорим на молитву и потщимся на покаяние: источает бо нам неоскудныя милости Пречистая Богородица, предваряет на помощь и избавляет от великих бед и зол благонравныя и богобоящияся рабы Своя.


\medskip


\bfseries Молитва:\normalfont{}


О Пречистая Владычице Богородице, Царице небеси и земли, вышшая ангел и архангел и всея твари честнейшая, чистая Дево Марие, миру благая помощнице и всем людем утверждение и во всяких нуждах избавление! Призри и ныне, Госпоже Всемилостивая, на рабы Твоя, Тебе умиленною душею и сокрушенным сердцем молящияся, со слезами к Тебе припадающия пречистому и цельбоносному образу Твоему, и помощи и заступления Твоего просящия. О Всемилостивая и Премилосердая Дево Богородице чтимая! Воззри, Госпоже, на люди Твоя: мы бо, грешнии, не имамы иныя помощи, разве Тебе и от Тебя рождшагося Христа Бога нашего. Ты еси заступница и предстательница наша, Ты еси обидимым защищение. Скорбящим радование, сирым прибежище, вдовам хранительница, девам слава, плачущим веселие, больным посещение, немощным исцеление, грешным спасение. Сего ради, о Богомати, к Тебе прибегаем и на Твой пречистый образ с превечным на руку Твоею держимым младенцем, Господем нашим Иисусом Христом взирающе, умиленное пение Тебе приносим и вопием: помилуй нас, Мати Божия, и прошение наше исполни, вся бо суть возможна ходатайству Твоему: яко Тебе слава подобает ныне и присно и во веки веков. Аминь.


\mychapterending

\mychapter{Молитва пред иконою &quot;Почаевская&quot;  (23 июля / 5 августа)}
%/text864.htm



\myfig{img/732.jpg}

\bfseries Тропарь, глас 5-й:\normalfont{}


Пред святою Твоею иконою, Владычице, молящиися исцелений сподобляются, веры истинныя познание приемлют, и агарянская нашествия отражают. Темже и нам, к Тебе припадающим, грехов оставление испроси, помыслы благочестия сердца наша просвети, и к Сыну Твоему молитву вознеси о спасении душ наших.


\medskip


\bfseries Кондак, глас 1-й:\normalfont{}


Источник исцелений и веры Православныя утверждение Почаевская Твоя икона, Богородице, явися: тем же и нас, к ней притекающих, от бед и искушений свободи, Лавру Твою невредиму сохрани, Православие во окрест стоящих странах утверди, и грехи разреши молитвенник Твоих: елика бо хощещи, можеши.


\medskip


\bfseries Молитва:\normalfont{}


К Тебе, о Богомати, молитвенно притекаем мы, грешнии, чудеса Твоя во святей Лавре Почаевстей явленная поминающе и о своих сокрушающеся прегрешениих. Вемы, Владычице, вемы, яко не подобаше нам, грешным, чесого просити, токмо о еже праведному Судии беззакония наша оставити нам. Вся бо, нами в житии претерпенная, скорби же, и нужды, и болезни, яко плоды падений наших, прозябоша нам, Богу сия на исправление наше попущающу. Темже вся сия истиною и судом Своим наведе Господь на грешныя рабы Своя, иже в печалех своих к заступлению Твоему, Пречистая, притекоша, и во умилении сердец к Тебе взывают сице: грехов и беззаконий наших, Благая, не помяни, но паче всечестныя руце Твоя воздвигши, к Сыну Твоему и Богу предстани, да люте содеянная нами отпустит нам, да, за премногия неисполненная обещания наша лица Своего от рабов Своих не отвратит, да благодати Своея, спасению нашему пособствующия, от душ наших не отъимет. Ей, Владычице, буди спасению нашему ходатаица и, малодушия нашего не возгнушавшися, призри на стенания наша, яже в бедах и скорбех наших пред чудотворным Твоим образом возносим. Просвети умиленными помыслы умы наша, веру нашу укрепи, надежду утверди, любве сладчайший дар сподоби нас прияти. Сими убо, Пречистая, дарованьми, а не болезньми и скорбьми живот наш ко спасению да возводится, но от уныния и отчаяния души наша ограждающи, избави нас маломощных от находящих на ны бед, и нужд, и клеветы человеческия и болезней нестерпимых. Даруй мир и благоустроение жительству христианскому предстательством Твоим, Владычице, утверди православную веру в стране нашей, во всем в мире, Церковь Апостольскую и Соборную умалению не предаждь, уставы святых отец на веки непоколебимы сохрани, всех к Тебе притекающих от рова погибельнаго спаси. Еще же и ересию прельщенных братии наших или веру спасительную в греховных страстех погубивших паки ко истинней вере и покаянию приведи, да вкупе с нами Твоему чудотворному образу поклоняющеся Твое предстательство исповедят. Сподоби убо нас, Пречистая Госпоже Богородице, еще в животе сем победу истины Твоим заступлением узрети, сподоби нас благодатную радость прежде кончины нашея восприяти, якоже древле насельники почаевстии Твоим явлением победители и просветители агарян показала еси, да вси мы благодарным сердцем вкупе со ангелы, и пророки, и апостолы, и со всеми святыми Твое милосердие прославляюще, воздадим славу, честь и поклонение в Троице певаемому Богу Отцу, и Сыну, и Святому Духу во веки веков. Аминь.


\mychapterending

\mychapter{Молитва пред иконою Смоленскою, именуемою &quot;Одигитрия&quot;  (28 Июля / 10 Августа)}
%/text865.htm



\myfig{img/733.jpg}

\bfseries Тропарь, глас 4-й:\normalfont{}


К Богородице прилежно ныне притецем, грешнии и смиреннии, и припадем, в покаянии зовуще из глубины души: Владычице, помози, на ны милосердовавши, потщися, погибаем от множества прегрешений, не отврати Твоя рабы тщи, Тя бо и едину надежду имамы.


\medskip


\bfseries Кондак, глас 6-й:\normalfont{}


Не имамы иныя помощи, не имамы иныя надежды, разве Тебе, Владычице, Ты нам помози, на Тебе надеемся и Тобою хвалимся: Твои бо есмы раби, да не постыдимся.


\medskip


\bfseries Молитва:\normalfont{}


О, Пречудная и Превышшая всех тварей Царице Богородице, Небеснаго Царя Христа Бога нашего Мати, Пресвятая Одигитрие Марие! Услыши ны грешныя и недостойныя, в час сей молящияся и к Тебе со воздыханием и слезами пред Пречистым образом Твоим припадающия, и сице умильно глаголющия: изведи нас от рова страстей, Одигитрие благая, избави нас от всякия скорби и печали, огради от всякия напасти и злых клевет и от неправеднаго навета вражия: можеши бо, о Благодатная Мати наша, не точию от всякаго зла сохранити люди Твоя, но и всяким благодеянием снабдити и спасти. Разве бо Тебе иныя предстательницы в бедах и обстояниях и теплыя ходатаицы о нас грешных к Сыну Твоему, Христу Богу нашему, не имамы. Егоже умоли, Владычице, спасти нас и Царствия Небеснаго сподобити, да спасеннии Тобою славим Тя и в будущем веце, якоже спасения нашего виновницу, и превозносим Всесвятое и Великолепое имя Отца и Сына и Святаго Духа, в Троице славимаго и покланяемаго Бога, во веки веков. Аминь.


\mychapterending

\mychapter{Молитва пред иконою Киево-Печерской, именуемою &quot;Успение&quot; (15 / 28 Августа)}
%/text867.htm



\bfseries Тропарь, глас 4-й:\normalfont{}


В рождестве девство сохранила еси, во успении мира не оставила еси, Богородице, преставилася еси к животу, Мати сущи Живота, и молитвами Твоими избавляеши от смерти души наша.


\medskip


\bfseries Кондак, глас 2-й:\normalfont{}


В молитвах неусыпающую Богородицу и в предстательствах непреложное упование, гроб и умерщвление не удержаста: якоже бо Живота Матерь, к животу престави, во утробу Вселивыйся приснодевственную.


\medskip


\bfseries Молитва:\normalfont{}


О Пресвятая Богородице Дево, Владычице, вышшая aнгел и aрхангел и всея твари честнейшая, ангельское великое удивление, пророческая высокая проповедь, апостольская преславная похвало, святителей изрядное украшение, мучеников крепкое утверждение, иноков спасительное наставление, постников неизнемогающее воздержание, девствующих чистото и славо, матерей тихое веселие, младенцев мудросте и вразумление, вдовиц и сирых кормительнице, нагих одеяние, болящих здравие, пленников избавление, по морю плавающих тишино, обуреваемых небурное пристанище, блуждающих нетрудная наставнице, путешествующих легкое прохождение, труждающихся благое покоище, в бедах сущих скорая заступнице, обидимых покрове и прибежище, ненадеющихся надеяние, требующих помощнице, печальных присное утешение, ненавидимых любовное смирение, грешников спасение и к Богу присвоение, правоверных всех твердое ограждение, непобедимое поможение и заступление! Тобою нам, Владычице, Невидимый видим бысть, и Тебе мольбу приносим, Госпоже, грешнии раби Твои: О Премилостивая и Пречудная Света умнаго Царице, рождшая Царя Христа, Бога нашего, Живодавца всех, от небесных славимая и от земных хвалимая, ангельский уме, светозарная звездо, святых пресвятейшая, Царице цариц, Владычице всех тварей, боголепная Девице, нескверная Невесто, палато Духа Пресвятаго, огненный престоле невидимаго Царя, небесный кивоте, носило Слова Божия, огнеобразная колеснице, покоище Живаго Бога, неизреченное составление плоти Христовы, гнездо Орла Небеснаго, горлице богогласная, голубице кроткая, тихая и незлобивая, Мати чадолюбивая, милостей бездно, развергающая тучу гнева Божия, неизмеримая глубино, неизреченная тайно, несведомое чудо, не рукотворенная Церковь Единаго Царя всех век, благоуханное кадило, честная багрянице, боготканная порфиро, душевный раю, живоноснаго сада отрасле, цвете прекрасный, процветший нам небесное веселие, грозде спасения нашего, чаше Царя Небеснаго, в нейже растворися от Духа Святаго вино неисчерпаемыя благодати, ходатаице закона, зачало истинныя веры Христовы непоколебимый столпе, еретиков пагубо, мечу ярости Божия на богопротивных, бесов устрашение, во бранех побеждение, христиан всех неложная хранительнице и мира всего известное спасение! О Всемилостивая Госпоже, Дево Владычице Богородице, услыши нас, молящихся Тебе, и яви милость Твою на людей Твоих: моли Сына Своего избавитися нам от всякаго зла и сохрани обитель нашу, и всяку обитель, и град, и страну верных, и люди благочестно прибегающия и призывающия Имя Твое святое, от всякия напасти, губительства, глада, труса, потопа, огня, меча, нашествия иноплеменников и междоусобныя брани, от всякия болезни и всякаго обстояния: да ни ранами, ни прещением, ни мором, ни всяким праведным гневом Божиим умалятся раби Твои, но соблюдай и спасай милостию Твоею, Госпоже, за ны молящися, и полезное воздуха благорастворение ко времени плоднаго приношения нам даруй; oблегчи, возстави и помилуй, Премилостивая Владычице, Богородице препетая, во всякой беде и нужде сущия. Помяни рабы Твоя, и не презри слез и воздыхания с нашего, и обнови нас благостию Твоея милости, да со благодарением утешаемся, обретше Тя Помощницу. Умилосердися, Госпоже Пречистая, на немощныя люди Твоя, Надеждо наша: разсеянныя собери, заблудшия на путь правый настави, отпадшия от благочестивыя отеческия веры паки возврати, старость поддержи, юныя вразуми, младенцы воспитай и прослави славящия Тя; изряднее же "--- Церковь Сына Твоего соблюди и сохрани в долготу дний. О Милостивая и Премилостивая Царице Небесе и земли, Богородице Приснодево! Ходатайством Твоим помилуй вся православныя христианы, сохраняющи их под кровом милости Твоея, ризою Твоею честною защити; и моли из Тебе воплошагося без семене Христа Бога нашего, да препояшет ны свыше силою на вся видимыя и невидимыя враги наша, на иноплеменники и единоплеменники, воюющая на нас и на веру нашу православную. Спаси же и помилуй, Госпоже, Великаго Господина и Отца нашего Святейшего Патриарха Алексия, преосвященныя митрополиты, архиепископы и епископы православныя, иереи же и диаконы, и весь причет церковный, и весь монашеский чин, и вся правоверныя люди, поклоняющияся и молящияся пред честною Твоею иконою. Призри на всех нас призрением милостивнаго Твоего заступления, воздвигни нас из глубины греховныя и просвети очи сердечныя ко зрению спасения: милостива нам буди зде, и на Страшнем Суде Сына Твоего о нас умоли; преставльшияся во благочестии от жития сего рабы Твоя в вечней жизни со aнгелы и aрхангелы и со всеми святыми причти, да одесную Сына Твоего Бога предстанут, и молитвою Твоею сподоби вся православныя христианы со Христом жити и радости ангельския в небесных селениих наслаждатися. Ты бо еси, Госпоже, слава Небесных и упование земных, Ты наша надеждо и заступнице всех притекающих к Тебе и Твоея святыя помощи просящих, Ты молебница наша теплая к Сыну Твоему и Богу нашему; Твоя Матерняя молитва много может на умоление Владыки, и Твоим предстательством ко престолу благодати Пресвятых и Животворящих Его Таин приступати дерзаем, аще и недостойнии. Темже всечестный образ Твой и рукою Твоею держимаго Вседержителя видяще на иконе, радуемся грешнии, со умилением припадающе, и любовию сей целуем, чающе, Госпоже, Твоими святыми Богоприятными молитвами дойти Небесныя безконечныя жизни и непостыдно стати в день судный одесную Сына Твоего и Бога нашего, яко да славим Его, купно со Безначальным Отцем и Пресвятым, Благим, Животворящим и Единосущным Духом, во веки веков. Аминь.


\mychapterending

\mychapter{Перед иконой &quot;Неопалимая Купина&quot;  ( 4 / 27 Сентября)}
%/text266.htm



\myfig{img/163.jpg}

О, Пресвятая и Преблагословенная Мати Сладчайшаго Господа нашего Иисуса Христа! Припадаем и поклоняемся Тебе пред святою и пречестною иконою Твоею, еюже дивны и преславны чудеса содеваеши, от огненнаго запаления и молниеноснаго громе жилища наша спасавши, недужныя исцеляеши и всякое благое прошение наше во благо исполняеши. Смиренно молим Тя, всесильная рода нашего Заступнице, сподоби ны немощныя и грешныя Твоего Матерняго участия и благопопечения. Спаси и сохрани, Владычице, под кровом милости Твоея Церковь Святую, обитель сию, всю страну нашу православную, и вся ны припадающия к Тебе с верою и любовию, и умиленно просящия со слезами Твоего заступления. Ей, Госпоже Всемилостивая, умилосердися на нас, обуреваемых грехами многими и не имущих дерзновения ко Христу Господу, просити Его о помиловании и прощении, но Тебе предлагаем к Нему на умоление Матерь Его по плоти: Ты же, Всеблагая, простри к Нему Богоприимнии руце Твои и предстательствуй за нас пред Благостию Его, просяще нам прощения прегрешений наших, благочестнаго мирнаго жития, благия христианския кончины, и добраго ответа на страшнем Суде Его. В час же грознаго посещения Божия, егда огнем возгорятся домы наша, или молниеносным громом устрашаеми будем, яви нам милостивное Твое заступление и державное вспоможение: да спасаеми всесильными Твоими ко Господу молитвами временнаго наказания Божия зде избегнем и вечное блаженство райское тамо унаследуем: и со всеми святыми воспоем Пречестное и Великолепное Имя поклоняемые Троицы, Отца и Сына и Святаго Духа, и Твое велие к нам милосердие, во веки веков. Аминь.


\mychapterending

\mychapter{Перед иконой &quot;В скорбех и печалех утешение&quot; (19 ноября / 2 декабря)}
%/text824.htm



Царице моя преблагая, надеждо моя Богородице, приятелище сирых и странных предстательнице, скорбящих радосте, обидимых покровительнице! Зриши мою беду, зриши мою скорбь, помози ми яко немощну, окорми мя яко странна. Обиду мою веси, разреши ту, яко волиши: яко не имам иныя помощи разве Тебе, ни иныя предстательницы, ни благия утешительницы, токмо Тебе, о Богомати, яко да сохраниши мя и покрыеши во веки веков. Аминь.


\mychapterending

\mychapter{Перед иконой &quot;Знамение&quot;, еже в велицем Новеграде (27 ноября / 10 декабря)}
%/text825.htm



\bfseries Тропарь, глас 4-й\normalfont{}:


Яко необоримую стену и источник чудес, стяжавше Тя раби Твои, Богородице Пречистая, сопротивных ополчения низлагаем. Темже молим Тя, мир граду Твоему даруй и душам нашим велию милость.


\medskip


\bfseries Кондак, глас 4-й\normalfont{}:


Честнаго образа Твоего знамение празднующе людие Твои, Богородительнице, имже дивную победу на сопротивныя граду Твоему даровала еси, темже Тебе верою взываем: радуйся, Дево, христиан похвало.


\medskip


\bfseries Молитва\normalfont{}:


О, Пресвятая и Преблагословенная Мати Сладчайшаго Господа нашего Иисуса Христа! Припадаем и покланяемся Тебе пред святою чудотворною иконою Твоею, воспоминающе дивное знамение Твоего заступления, великому Новуграду от нея явленное во дни ратнаго на сей град нашествия. Смиренно молим Тя, Всесильная рода нашего Заступнице: якоже древле отцем нашим на помощь тогда ускорила еси, тако и ныне нас немощных и грешных Твоего Матерняго заступления и благопопечения сподоби: спаси и сохрани, Вадычице, под кровом милости Твоея люди Твоя, Церковь святую утверди, град Твой (весь Твою) и всю страну нашу православную и всех нас, припадающих к Тебе с верою и любовию и умиленно просящих со слезами Твоего заступления, помилуй и сохрани. Ей, Госпоже всемилостивая! Умилосердися на ны, обуреваемыя грехми многими, простри ко Христу Богу богоприимнеи руце Твои и предстательствуй за нас пред благостию Его, просящи нам прощения прегрешений наших, благочестнаго, мирнаго жития, благия христианския кончины и добраго ответа на страшнем суде Его: да спасаеми всесильными Твоими к Нему молитвами, блаженство райское унаследуем, и со всеми святыми воспоем Пречестное и Великолепое имя достопокланяемыя Троицы, Отца и Сына и Святаго Духа, и Твое велие к нам милосердие во веки веков. Аминь.


\mychapterending