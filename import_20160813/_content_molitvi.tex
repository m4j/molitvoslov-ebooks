

\mypart{МОЛИТВЫ}
%/content/molitvi



\bfseries Смотреть весь раздел &rarr;\normalfont{} 

\mychapter{Молитва Господня. Отче наш}
%/node/37



\myfig{img/Spasnatrone.jpg}

О́тче наш, И́же еси́ на небесе́х! Да святи́тся и́мя Твое́, да прии́дет Ца́рствие
Твое́, да бу́дет во́ля Твоя́, я́ко на небеси́ и на земли́. Хлеб наш насу́щный
даждь нам днесь; и оста́ви нам до́лги на́ша, я́коже и мы оставля́ем
должнико́м на́шим; и не введи́ нас во искуше́ние, но изба́ви нас от
лука́ваго.


   Отец наш, сущий на небесах! да святится имя Твое; да приидет Царствие
Твое; да будет воля Твоя и на земле, как на небе; хлеб наш насущный
дай нам на сей день; и прости нам долги наши, как и мы прощаем
должникам нашим; и не введи нас в искушение, но избавь нас от
лукавого. Ибо Твое есть Царство и сила и слава во веки. Аминь. (Матф.
6:9–13)

   


\mychapterending

\mychapter{Иисусова молитва}
%/text596.htm



\myfig{img/456.jpg}

Го́споди Иису́се Христе́, Сы́не Бо́жий, поми́луй мя, гре́шнаго.

   


\mychapterending

\mychapter{Молитва Честно́му Кресту (Да воскреснет Бог...)}
%/content/molitva-Chestnomu-Krestu



\myfig{img/cross.jpg}



\bfseries Знаменуй себя крестом и говори молитву Честно́му Кресту\normalfont{}


   Да воскре́снет Бог, и расточа́тся врази́ Его́, и да бежа́т от лица́ Его́
ненави́дящии Его́. Я́ко исчеза́ет дым, да исче́знут; я́ко та́ет воск от лица́
огня́, та́ко да поги́бнут бе́си от лица́ лю́бящих Бо́га и зна́менующихся
кре́стным зна́мением, и в весе́лии глаго́лющих: ра́дуйся, Пречестны́й и
Животворя́щий Кре́сте Госпо́день, прогоня́яй бе́сы си́лою на тебе́ пропя́таго
Го́спода на́шего Иису́са Христа́, во ад сше́дшаго и попра́вшего си́лу диа́волю,
и дарова́вшаго нам тебе́ Крест Свой Честны́й на прогна́ние вся́каго
супоста́та. О, Пречестны́й и Животворя́щий Кре́сте Госпо́день! Помога́й ми
со Свято́ю Госпоже́ю Де́вою Богоро́дицею и со все́ми святы́ми во ве́ки.
Ами́нь.


 \itshape Или кратко:\normalfont{}


   Огради́ мя, Го́споди, си́лою Честна́го и Животворя́щаго Твоего́ Креста́, и
сохра́ни мя от вся́каго зла.

   


\mychapterending

\mychapter{Благодарение за всякое благодеяние Божие}
%/text889.htm



\myfig{img/777.jpg}



\bfseries Тропарь, глас 4-й\normalfont{}


   Благода́рни су́ще недосто́йнии раби́ Твои́, Го́споди, о Твои́х вели́ких
благодея́ниих на нас бы́вших, сла́вяще Тя хва́лим, благослови́м, благодари́м,
пое́м и велича́ем Твое́ благоутро́бие, и ра́бски любо́вию вопие́м Ти:
Благоде́телю Спа́се наш, сла́ва Тебе́.



 

\bfseries Кондак, глас 3-й\normalfont{}


   Твои́х благодея́ний и даро́в ту́не, я́ко раби́ непотре́бнии, сподо́бльшеся,
Влады́ко, к Тебе́ усе́рдно притека́юще, благодаре́ние по си́ле прино́сим, и
Тебе́ я́ко Благоде́теля и Творца́ сла́вяще, вопие́м: сла́ва Тебе́, Бо́же
Всеще́дрый.


   Сла́ва Отцу́ и Сы́ну и Свято́му Ду́ху, и ны́не и при́сно и во ве́ки веко́в.
Ами́нь.


 \itshape Богородичен:\normalfont{} Богоро́дице, христиа́ном Помо́щнице, Твое́ предста́тельство
стяжа́вше раби́ Твои́, благода́рно Тебе́ вопие́м: ра́дуйся, Пречи́стая
Богоро́дице Де́во, и от всех нас бед Твои́ми моли́твами всегда́ изба́ви, Еди́на
вско́ре предста́тельствующая.

   


\mychapterending

\mychapter{Молитвы утренние}
%/text893.htm



\myfig{img/795.jpg}

[1]
\itshape Востав от сна, прежде всякого другого дела, стань благоговейно,
представляя себя пред Всевидящим Богом, и, совершая крестное знамение,
произнеси:\normalfont{}


   Во и́мя Отца́, и Сы́на, и Свята́го Ду́ха, Ами́нь.


 \itshape Затем немного подожди, пока все чувства твои не придут в тишину и
мысли твои не оставят все земное, и тогда произноси следующие молитвы,
без поспешности и со вниманием сердечным:\normalfont{}



 

\bfseries Молитва мытаря \itshape (Евангелие от Луки, глава 18, стих 13)\normalfont{}\normalfont{}


   Бо́же, ми́лостив бу́ди мне гре́шному. \itshape (Поклон)\normalfont{}



 

\bfseries Молитва предначинательная\normalfont{}


   Го́споди Иису́се Христе́, Сы́не Бо́жий, молитв ра́ди Пречи́стыя Твоея́
Ма́тере и всех святы́х, поми́луй нас. Ами́нь.


   Сла́ва Тебе́, Бо́же наш, сла́ва Тебе́.



 

\bfseries Молитва Святому Духу\normalfont{}


   Царю́ Небе́сный, Уте́шителю, Ду́ше и́стины, И́же везде́ сый и
вся исполня́яй, Сокро́вище благи́х и жи́зни Пода́телю, прииди́ и
всели́ся в ны, и очи́сти ны от вся́кия скве́рны, и спаси́, Бла́же, ду́ши
на́ша[2].



 

\bfseries Трисвятое\normalfont{}


   Святы́й Бо́же, Святы́й Кре́пкий, Святы́й Безсме́ртный, поми́луй нас.
\itshape (Читается трижды, с крестным знамением и поясным поклоном.)\normalfont{}


   Сла́ва Отцу́ и Сы́ну и Свято́му Ду́ху, и ны́не и при́сно и во ве́ки веко́в.
Ами́нь[3].



 

\bfseries Молитва ко Пресвятой Тро́ице\normalfont{}


   Пресвята́я Тро́ице, поми́луй нас; Го́споди, очи́сти грехи́ на́ша; Влады́ко,
прости́ беззако́ния на́ша; Святы́й, посети́ и исцели́ не́мощи на́ша, и́мене
Твое́го ра́ди.


   Го́споди, поми́луй. \itshape (Трижды)\normalfont{}.


   Сла́ва Отцу́ и Сы́ну и Свято́му Ду́ху, и ны́не и при́сно и во ве́ки веко́в.
Ами́нь.



 

\bfseries Молитва Господня\normalfont{}


   О́тче наш, И́же еси́ на небесе́х! Да святи́тся и́мя Твое́, да прии́дет
Ца́рствие Твое́, да бу́дет во́ля Твоя́, я́ко на небеси́ и на земли́. Хлеб наш
насу́щный да́ждь нам днесь; и оста́ви нам до́лги на́ша, я́коже и мы оставля́ем
должнико́м на́шим; и не введи́ нас во искуше́ние, но изба́ви нас от
лука́ваго.



 

\bfseries Тропари Троичные\normalfont{}


   Воста́вше от сна, припа́даем Ти, Бла́же, и а́нгельскую песнь вопие́м Ти,
Си́льне: Свят, Свят, Свят еси́, Бо́же, Богоро́дицею поми́луй нас.


   Сла́ва Отцу́ и Сы́ну и Свято́му Ду́ху.


   От одра́ и сна воздви́гл мя еси́, Го́споди, ум мой просвети́ и се́рдце, и
устне́ мои́ отве́рзи, во е́же пе́ти Тя, Свята́я Тро́ице: Свят, Свят, Свят еси́,
Бо́же, Богоро́дицею поми́луй нас.


   И ны́не и при́сно и во ве́ки веко́в. Ами́нь.


   Внеза́пно Судия́ прии́дет, и коего́ждо дея́ния обнажа́тся, но стра́хом
зове́м[4] в
полу́нощи: Свят, Свят, Свят еси́, Бо́же, Богоро́дицею поми́луй нас.


   Го́споди, поми́луй. \itshape (12 раз)\normalfont{}



 

\bfseries Молитва ко Пресвятой Тро́ице\normalfont{}


   От сна воста́в, благодарю́ Тя, Свята́я Тро́ице, я́ко мно́гия ра́ди Твоея́
бла́гости и долготерпе́ния не прогне́вался еси́ на мя, лени́ваго и гре́шнаго,
ниже́ погуби́л мя еси́ со беззако́ньми мои́ми; но человеколю́бствовал еси́
обы́чно и в неча́янии лежа́щаго воздви́гл мя еси́, во е́же у́треневати и
славосло́вити держа́ву Твою́. И ны́не просвети́ мои́ о́чи мы́сленныя, отве́рзи
моя́ уста́ поуча́тися словесе́м Твои́м, и разуме́ти за́поведи Твоя́, и твори́ти
во́лю Твою́, и пе́ти Тя во исповеда́нии серде́чнем, и воспева́ти всесвято́е и́мя
Твое́, Отца́ и Сы́на и Свята́го Ду́ха, ны́не и при́сно и во ве́ки веко́в.
Ами́нь.


   Прииди́те, поклони́мся Царе́ви на́шему Бо́гу. \itshape (Поклон)\normalfont{}


   Прииди́те, поклони́мся и припаде́м Христу́, Царе́ви на́шему Бо́гу.
\itshape (Поклон)\normalfont{}


   Прииди́те, поклони́мся и припаде́м Самому́ Христу́, Царе́ви и Бо́гу
на́шему. \itshape (Поклон)\normalfont{}



 

\bfseries Псалом 50\normalfont{}


   Поми́луй мя, Бо́же, по вели́цей ми́лости Твое́й, и по мно́жеству щедро́т
Твои́х очи́сти беззако́ние мое́. Наипа́че омы́й мя от беззако́ния моего́, и от
греха́ моего́ очи́сти мя; яко беззако́ние мое́ аз зна́ю, и грех мой пре́до мно́ю
есть вы́ну. Тебе́ Еди́ному согреши́х и лука́вое пред Тобо́ю сотвори́х, я́ко да
оправди́шися во словесе́х Твои́х, и победи́ши внегда́ суди́ти Ти. Се бо, в
беззако́ниих зача́т есмь, и во гресе́х роди́ мя ма́ти моя́. Се бо, и́стину
возлюби́л еси́; безве́стная и та́йная прему́дрости Твоея́ яви́л ми еси́.
Окропи́ши мя иссо́пом, и очи́щуся; омы́еши мя, и па́че сне́га убелю́ся. Слу́ху
моему́ да́си ра́дость и весе́лие; возра́дуются ко́сти смире́нныя. Отврати́ лице́
Твое́ от грех мои́х и вся беззако́ния моя́ очи́сти. Се́рдце чи́сто сози́жди во
мне, Бо́же, и дух прав обнови́ во утро́бе мое́й. Не отве́ржи мене́ от
лица́ Твоего́ и Ду́ха Твоего́ Свята́го не отыми́ от мене́. Возда́ждь ми
ра́дость спасе́ния Твое́го и Ду́хом влады́чним утверди́ мя. Научу́
беззако́ныя путе́м Твои́м, и нечести́вии к Тебе́ обратя́тся. Изба́ви мя от
крове́й, Бо́же, Бо́же спасе́ния моего́; возра́дуется язы́к мой пра́вде

Твое́й. Го́споди, устне́ мои отве́рзеши, и уста́ моя́ возвестя́т хвалу́
Твою́. Я́ко а́ще бы восхоте́л еси́ же́ртвы, дал бых у́бо: всесожже́ния не
благоволи́ши. Же́ртва Бо́гу дух сокруше́н; се́рдце сокруше́нно и смире́нно
Бог не уничижи́т. Ублажи́, Го́споди, благоволе́нием Твои́м Сио́на, и
да сози́ждутся сте́ны Иерусали́мския. Тогда́ благоволи́ши же́ртву
пра́вды, возноше́ние и всесожега́емая; тогда́ возложа́т на oлта́рь Твой
тельцы́.



 

\bfseries Символ веры\normalfont{}


   Ве́рую во еди́наго Бо́га Отца́, Вседержи́теля, Творца́ не́бу и земли́,
ви́димым же всем и неви́димым. И во еди́наго Го́спода Иису́са Христа́, Сы́на
Бо́жия, Единоро́днаго, И́же от Отца́ рожде́ннаго пре́жде всех век; Све́та от
Све́та, Бо́га и́стинна от Бо́га и́стинна, рожде́нна, несотворе́нна, единосу́щна
Отцу́, И́мже вся бы́ша. Нас ра́ди челове́к и на́шего ра́ди спасе́ния сше́дшаго с
небе́с и воплоти́вшагося от Ду́ха Свя́та и Мари́и Де́вы и вочелове́чшася.
Распя́таго же за ны при Понти́йстем Пила́те, и страда́вша, и погребе́нна. И
воскре́сшаго в тре́тий день по Писа́нием. И возше́дшаго на небеса́, и
седя́ща одесну́ю Отца́. И па́ки гряду́щаго со сла́вою суди́ти живы́м и
ме́ртвым, Его́же Ца́рствию не бу́дет конца́. И в Ду́ха Свята́го, Го́спода,
Животворя́щаго, И́же от Отца́ исходя́щаго, И́же со Отце́м и Сы́ном
спокланя́ема и ссла́вима, глаго́лавшаго проро́ки. Во еди́ну Святу́ю,
Собо́рную и Апо́стольскую Це́рковь. Испове́дую еди́но креще́ние во
оставле́ние грехо́в. Ча́ю воскресе́ния ме́ртвых, и жи́зни бу́дущаго ве́ка.
Ами́нь.



 

\bfseries Молитва первая, святого Макария Великого\normalfont{}


   Бо́же, очи́сти мя гре́шнаго, я́ко николи́же сотвори́х благо́е пред
Тобо́ю; но изба́ви мя от лука́ваго, и да бу́дет во мне во́ля Твоя́, да
неосужде́нно отве́рзу уста́ моя́ недосто́йная и восхвалю́ и́мя Твое́
свято́е, Отца́ и Сы́на и Свята́го Ду́ха, ны́не и при́сно и во ве́ки веко́в.
Ами́нь.




 

\bfseries  Молитва вторая, того же святого\normalfont{}


   От сна воста́в, полу́нощную песнь приношу́ Ти, Спа́се, и припа́дая вопию́
Ти: не да́ждь ми усну́ти во грехо́вней сме́рти, но уще́дри мя, распны́йся
во́лею, и лежа́щаго мя в ле́ности ускори́в возста́ви, и спаси́ мя в предстоя́нии
и моли́тве, и по сне нощне́м возсия́й ми день безгре́шен, Христе́ Бо́же, и
спаси́ мя.



 

\bfseries Молитва третья, того же святого\normalfont{}


   К Тебе́, Влады́ко Человеколю́бче, от сна воста́в прибега́ю, и на дела́ Твоя́
подвиза́юся милосе́рдием Твои́м, и молю́ся Тебе́: помози́ ми на вся́кое вре́мя,
во вся́кой ве́щи, и изба́ви мя от вся́кия мирски́я злы́я ве́щи и диа́вольскаго
поспеше́ния, и спаси́ мя, и введи́ в Ца́рство Твое́ ве́чное. Ты бо еси́ мой
Сотвори́тель и вся́кому бла́гу Промы́сленник и Пода́тель, о Тебе́ же все
упова́ние мое́, и Тебе́ сла́ву возсыла́ю, ны́не и при́сно и во ве́ки веко́в.
Ами́нь.



 

\bfseries Молитва четвертая, того же святого\normalfont{}


   Го́споди, И́же мно́гою Твое́ю бла́гостию и вели́кими щедро́тами Твои́ми
дал еси́ мне, рабу́ Твоему́, мимоше́дшее вре́мя но́щи сея́ без напа́сти прейти́
от вся́каго зла проти́вна; Ты Сам, Влады́ко, вся́ческих Тво́рче, сподо́би мя
и́стинным Твои́м све́том и просвеще́нным се́рдцем твори́ти во́лю Твою́, ны́не
и при́сно и во ве́ки веко́в. Ами́нь.



 

\bfseries Молитва пятая, святого Василия Великого\normalfont{}


   Го́споди Вседержи́телю, Бо́же сил и вся́кия пло́ти, в вы́шних живы́й и на
смире́нныя призира́яй, се́рдца же и утро́бы испыту́яй и сокрове́нная
челове́ков я́ве предве́дый, Безнача́льный и Присносу́щный Све́те, у Него́ же
несть премене́ние, или́ преложе́ния осене́ние; Сам, Безсме́ртный Царю́,

приими́ моле́ния на́ша, я́же в настоя́щее вре́мя, на мно́жество Твои́х щедро́т
дерза́юще, от скве́рных к Тебе́ усте́н твори́м, и оста́ви нам прегреше́ния
на́ша, я́же де́лом, и сло́вом, и мы́слию, ве́дением, или́ неве́дением
согреше́нная на́ми; и очи́сти ны от вся́кия скве́рны пло́ти и ду́ха. И да́руй
нам бо́дренным се́рдцем и тре́звенною мы́слию всю настоя́щаго жития́ нощь
прейти́, ожида́ющим прише́ствия све́тлаго и явле́ннаго дне Единоро́днаго
Твое́го Сы́на, Го́спода и Бо́га и Спа́са на́шего Иису́са Христа́, в о́ньже
со сла́вою Судия́ всех прии́дет, кому́ждо отда́ти по де́лом его́; да
не па́дше и облени́вшеся, но бо́дрствующе и воздви́жени в де́лание
обря́щемся гото́ви, в ра́дость и Боже́ственный черто́г сла́вы Его́ совни́дем,
иде́же пра́зднующих глас непреста́нный, и неизрече́нная сла́дость
зря́щих Твое́го лица́ добро́ту неизрече́нную. Ты бо еси́ и́стинный Свет,
просвеща́яй и освяща́яй вся́ческая, и Тя пое́т вся тварь во ве́ки веко́в.
Ами́нь.



 

\bfseries Молитва шестая, того же святого\normalfont{}


   Тя благослови́м, вы́шний Бо́же и Го́споди ми́лости, творя́щаго
при́сно с на́ми вели́кая же и неизсле́дованная, сла́вная же и ужа́сная,
и́хже несть числа́, пода́вшаго нам сон во упокое́ние не́мощи на́шея,
и ослабле́ние трудо́в многотру́дныя пло́ти. Благодари́м Тя, якo не
погуби́л еси́ нас со беззако́ньми на́шими, но человеколю́бствовал еси́
обы́чно, и в неча́янии лежа́щия ны воздви́гл еси́, во е́же славосло́вити
держа́ву Твою́. Те́мже мо́лим безме́рную Твою́ бла́гость, просвети́ на́ша
мы́сли, очеса́, и ум наш от тя́жкаго сна ле́ности возста́ви: отве́рзи на́ша
уста́, и испо́лни я Твое́го хвале́ния, я́ко да возмо́жем непоколе́блемо
пе́ти же и испове́датися Тебе́, во всех, и от всех сла́вимому Бо́гу,
Безнача́льному Отцу́, со Единоро́дным Твои́м Сы́ном, и Всесвяты́м и
Благи́м и Животворя́щим Твои́м Ду́хом, ны́не и при́сно и во ве́ки веко́в.
Ами́нь.



 

\bfseries Молитва седьмая, ко Пресвятой Богоро́дице\normalfont{}


   Воспева́ю благода́ть Твою́, Влады́чице, молю́ Тя, ум мой облагодати́.
Ступа́ти пра́во мя наста́ви, путе́м Христо́вых за́поведей. Бде́ти к пе́сни

укрепи́, уны́ния сон отгоня́ющи. Свя́зана плени́цами грехопаде́ний, мольба́ми
Твои́ми разреши́, Богоневе́сто. В нощи́ мя и во дни сохраня́й, борю́щих враг
избавля́ющи мя. Жизнода́теля Бо́га ро́ждшая, умерщвле́на мя страстьми́
оживи́. Я́же Свет невече́рний ро́ждшая, ду́шу мою́ осле́пшую просвети́. О
ди́вная Влады́чня пала́то, дом Ду́ха Боже́ственна мене́ сотвори́. Врача́
ро́ждшая, уврачу́й души́ моея́ многоле́тныя стра́сти. Волну́ющася
жите́йскою бу́рею, ко стези́ мя покая́ния напра́ви. Изба́ви мя огня́
ве́чнующаго, и че́рвия же зла́го, и та́ртара. Да мя не яви́ши бесо́м
ра́дование, и́же мно́гим грехо́м пови́нника. Но́ва сотвори́ мя, обетша́вшаго
нечу́вственными, Пренепоро́чная, согреше́нии. Стра́нна му́ки вся́кия покажи́
мя, и всех Влады́ку умоли́. Небе́сная ми улучи́ти весе́лия, со все́ми
святы́ми, сподо́би. Пресвята́я Де́во, услы́ши глас непотре́бнаго раба́
Твоего́. Струю́ дава́й мне слеза́м, Пречи́стая, души́ моея́ скве́рну
очища́ющи. Стена́ния от се́рдца приношу́ Ти непреста́нно, усе́рдствуй,
Влады́чице. Моле́бную слу́жбу мою́ приими́, и Бо́гу благоутро́бному
принеси́. Превы́шшая А́нгел, мирска́го мя превы́шша сли́тия сотвори́.
Светоно́сная Се́не небе́сная, духо́вную благода́ть во мне напра́ви. Ру́це
возде́ю и устне́ к похвале́нию, оскверне́ны скве́рною, Всенепоро́чная.
Душетле́нных мя па́костей изба́ви, Христа́ приле́жно умоля́ющи;
Ему́же честь и поклоне́ние подоба́ет, ны́не и при́сно и во ве́ки веко́в.
Ами́нь.



 

\bfseries Молитва восьмая, ко Го́споду на́шему Иису́су Христу́\normalfont{}


   Многоми́лостиве и Всеми́лостиве Бо́же мой, Го́споди Иису́се Христе́,
мно́гия ра́ди любве́ сшел и воплоти́лся еси́, я́ко да спасе́ши всех. И па́ки,
Спа́се, спаси́ мя по благода́ти, молю́ Тя; а́ще бо от дел спасе́ши мя,
несть се благода́ть, и дар, но долг па́че. Ей, мно́гий в щедро́тах и
неизрече́нный в ми́лости! Ве́руяй бо в Мя, рекл еси́, о Христе́ мой, жив
бу́дет и не у́зрит сме́рти во ве́ки. А́ще у́бо ве́ра, я́же в Тя, спаса́ет
отча́янныя, се ве́рую, спаси́ мя, я́ко Бог мой еси́ Ты и Созда́тель.
Ве́ра же вме́сто дел да вмени́тся мне, Бо́же мой, не обря́щеши бо
дел отню́д оправда́ющих мя. Но та ве́ра моя́ да довле́ет вме́сто всех,
та да отвеща́ет, та да оправди́т мя, та да пока́жет мя прича́стника
сла́вы Твоея́ ве́чныя. Да не у́бо похи́тит мя сатана́, и похва́лится,
Сло́ве, е́же отто́ргнути мя от Твоея́ руки́ и огра́ды; но или́ хощу́,
спаси́ мя, или́ не хощу́, Христе́ Спа́се мой, предвари́ ско́ро, ско́ро

погибо́х: Ты бо еси́ Бог мой от чре́ва ма́тере моея́. Сподо́би мя, Го́споди,
ны́не возлюби́ти Тя, я́коже возлюби́х иногда́ той са́мый грех; и па́ки
порабо́тати Тебе́ без ле́ности то́щно, я́коже порабо́тах пре́жде сатане́
льсти́вому. Наипа́че же порабо́таю Тебе́, Го́споду и Бо́гу моему́ Иису́су
Христу́, во вся дни живота́ моего́, ны́не и при́сно и во ве́ки веко́в.
Ами́нь.



 

\bfseries Молитва девятая, к Ангелу хранителю\normalfont{}


   Святы́й А́нгеле, предстоя́й окая́нной мое́й души́ и стра́стной мое́й жи́зни,
не оста́ви мене́ гре́шнаго, ниже́ отступи́ от мене́ за невоздержа́ние мое́. Не
да́ждь ме́ста лука́вому де́мону облада́ти мно́ю, наси́льством сме́ртнаго сего́
телесе́; укрепи́ бе́дствующую и худу́ю мою́ ру́ку и наста́ви мя на путь
спасе́ния. Ей, святы́й А́нгеле Бо́жий, храни́телю и покрови́телю окая́нныя
моея́ души́ и те́ла, вся мне прости́, ели́кими тя оскорби́х во вся дни живота́
моего́, и а́ще что согреши́х в преше́дшую нощь сию́, покры́й мя в
настоя́щий день, и сохрани́ мя от вся́каго искуше́ния проти́внаго,
да ни в ко́ем гресе́ прогне́ваю Бо́га, и моли́ся за мя ко Го́споду, да
утверди́т мя в стра́се Свое́м, и досто́йна пока́жет мя раба́ Своея́ бла́гости.
Ами́нь.



 

\bfseries Молитва десятая, ко Пресвятой Богоро́дице\normalfont{}


   Пресвята́я Влады́чице моя́ Богоро́дице, святы́ми Твои́ми и всеси́льными
мольба́ми отжени́ от мене́, смире́ннаго и окая́ннаго раба́ Твоего́, уны́ние,
забве́ние, неразу́мие, нераде́ние, и вся скве́рная, лука́вая и ху́льная
помышле́ния от окая́ннаго моего́ се́рдца и от помраче́ннаго ума́ моего́; и
погаси́ пла́мень страсте́й мои́х, я́ко нищ есмь и окая́нен. И изба́ви мя от
мно́гих и лю́тых воспомина́ний и предприя́тий, и от всех действ злых свободи́
мя. Я́ко благослове́на еси́ от всех родо́в, и сла́вится пречестно́е и́мя Твое́ во
ве́ки веко́в. Ами́нь.



 

\bfseries Молитвенное призывание святого, имя которого носишь\normalfont{}


   Моли́ Бо́га о мне, святы́й уго́дниче Бо́жий \itshape (имя)\normalfont{}, я́ко аз усе́рдно к тебе́
прибега́ю, ско́рому помо́щнику и моли́твеннику о душе́ мое́й.



 

\bfseries Песнь Пресвятой Богородице\normalfont{}


   Богоро́дице Де́во, ра́дуйся, Благода́тная Мари́е, Госпо́дь с Тобо́ю;
благослове́на Ты в жена́х и благослове́н плод чре́ва Твоего́, я́ко Спа́са
родила́ еси́ душ на́ших.



 

\bfseries Тропарь Кресту и молитва за отечество\normalfont{}


   Спаси́, Го́споди, лю́ди Твоя́, и благослови́ достоя́ние Твое́, побе́ды
правосла́вным христиа́ном на сопроти́вныя да́руя, и Твое́ сохраня́я Кресто́м
Твои́м жи́тельство.



 

\bfseries Молитва о живых\normalfont{}


   Спаси́, Го́споди, и поми́луй отца́ моего́ духо́внаго \itshape (имя)\normalfont{}, роди́телей мои́х
\itshape (имена)\normalfont{}, сро́дников \itshape (имена)\normalfont{}, нача́льников, наста́вников, благоде́телей \itshape (имена)\normalfont{}
и всех правосла́вных христиа́н.



 

\bfseries Молитва о усопших\normalfont{}


   Упоко́й, Го́споди, ду́ши усо́пших раб Твои́х: роди́телей мои́х, сро́дников,
благоде́телей \itshape (имена)\normalfont{}, и всех правосла́вных христиа́н, и прости́ им вся
согреше́ния во́льная и нево́льная, и да́руй им Ца́рствие Небе́сное.


 \itshape Если можешь, читай вместо кратких молитв о живых и усопших этот
помянник:\normalfont{}



 

\bfseries О живых\normalfont{}


   Помяни́, Го́споди Иису́се Христе́, Бо́же наш, ми́лости и щедро́ты Твоя́ от
ве́ка су́щия, и́хже ра́ди и вочелове́чился еси́, и распя́тие и смерть, спасе́ния
ра́ди пра́во в Тя ве́рующих, претерпе́ти изво́лил еси́; и воскре́с из ме́ртвых,
возне́слся еси́ на небеса́ и седи́ши одесну́ю Бо́га Отца́, и призира́еши на
смире́нныя мольбы́ всем се́рдцем призыва́ющих Тя: приклони́ у́хо Твое́, и
услы́ши смире́нное моле́ние мене́, непотре́бнаго раба́ Твоего́, в воню́
благоуха́ния духо́внаго, Тебе́ за вся лю́ди Твоя́ принося́щаго. И в пе́рвых
помяни́ Це́рковь Твою́ Святу́ю, Собо́рную и Апо́стольскую, ю́же снабде́л еси́
честно́ю Твое́ю Кро́вию, и утверди́, и укрепи́, и разшири́, умно́жи,
умири́, и непребори́му а́довыми враты́ во ве́ки сохрани́; раздира́ния
Церкве́й утиши́, шата́ния язы́ческая угаси́, и ересе́й воста́ния ско́ро
разори́ и искорени́, и в ничто́же си́лою Свята́го Твоего́ Ду́ха обрати́.
\itshape (Поклон)\normalfont{}


   Спаси́, Го́споди, и поми́луй богохрани́мую страну́ на́шу, вла́сти и
во́инство ея́, огради́ ми́ром держа́ву их, и покори́ под но́зе правосла́вных
вся́каго врага́ и супоста́та, и глаго́ли ми́рная и блага́я в сердца́х их о
Це́ркви Твое́й Святе́й, и о всех лю́дех Твои́х: да ти́хое и безмо́лвное
житие́ поживе́м во правове́рии, и во вся́ком благоче́стии и чистоте́.
\itshape (Поклон)\normalfont{}


   Спаси́, Го́споди, и поми́луй Вели́каго Господи́на и Отца́ на́шего Святе́йшего
Патриа́рха Кири́лла, преосвяще́нныя митрополи́ты, архиепи́скопы и
епи́скопы правосла́вныя, иере́и же и диа́коны, и весь при́чет церко́вный, я́же
поста́вил еси́ пасти́ слове́сное Твое́ ста́до, и моли́твами их поми́луй и спаси́
мя гре́шнаго. \itshape (Поклон)\normalfont{}


   Спаси́, Го́споди, и поми́луй отца́ моего́ духо́внаго \itshape (имя его)\normalfont{}, и святы́ми его́
моли́твами прости́ моя́ согреше́ния. \itshape (Поклон)\normalfont{}


   Спаси́, Го́споди, и поми́луй роди́тели моя́ \itshape (имена их)\normalfont{}, бра́тию и се́стры, и
сро́дники моя́ по пло́ти, и вся бли́жния ро́да моего́, и дру́ги, и да́руй им
ми́рная Твоя́ и преми́рная блага́я. \itshape (Поклон)\normalfont{}


   Спаси́, Го́споди, и поми́луй по мно́жеству щедро́т Твои́х вся священнои́ноки,
и́ноки же и и́нокини, и вся в де́встве же и благогове́нии и по́стничестве
живу́щия в монастыре́х, в пусты́нях, в пеще́рах, гора́х, столпе́х, затво́рех,
разсе́линах ка́менных, острове́х же морски́х, и на вся́ком ме́сте влады́чествия
Твоего́ правове́рно живу́щия, и благоче́стно служа́щия Ти, и моля́щияся
Тебе́: облегчи́ им тяготу́, и уте́ши их скорбь, и к по́двигу о Тебе́ си́лу и
кре́пость им пода́ждь, и моли́твами их да́руй ми оставле́ние грехо́в.
\itshape (Поклон)\normalfont{}


   Спаси́, Го́споди, и поми́луй ста́рцы и ю́ныя, ни́щия и сироты́ и вдови́цы, и
су́щия в боле́зни и в печа́лех, беда́х же и ско́рбех, обстоя́ниих и плене́ниих,
темни́цах же и заточе́ниих, изря́днее же в гоне́ниих, Тебе́ ра́ди и ве́ры
правосла́вныя, от язы́к безбо́жных, от отсту́пник и от еретико́в, су́щия рабы́
Твоя́, и помяни́ я, посети́, укрепи́, уте́ши, и вско́ре си́лою Твое́ю осла́бу,
свобо́ду и изба́ву им пода́ждь. \itshape (Поклон)\normalfont{}


   Спаси́, Го́споди, и поми́луй благотворя́щия нам, ми́лующия и пита́ющия
нас, да́вшия нам ми́лостыни, и запове́давшия нам недосто́йным моли́тися о
них, и упокоева́ющия нас, и сотвори́ ми́лость Твою́ с ни́ми, да́руя им вся,
я́же ко спасе́нию проше́ния, и ве́чных благ восприя́тие. \itshape (Поклон)\normalfont{}


   Спаси́, Го́споди, и поми́луй по́сланныя в слу́жбу, путеше́ствующия, отцы́
и бра́тию на́шу, и вся правосла́вныя христиа́ны. \itshape (Поклон)\normalfont{}


   Спаси́, Го́споди, и поми́луй и́хже аз безу́мием мои́м соблазни́х, и от
пути́ спаси́тельнаго отврати́х, к де́лом злым и неподо́бным приведо́х;
Боже́ственным Твои́м Про́мыслом к пути́ спасе́ния па́ки возврати́.
\itshape (Поклон)\normalfont{}


   Спаси́, Го́споди, и поми́луй ненави́дящия и оби́дящия мя, и творя́щия ми
напа́сти, и не оста́ви их поги́бнути мене́ ра́ди, гре́шнаго. \itshape (Поклон)\normalfont{}


   Отступи́вшия от правосла́вныя ве́ры и поги́бельными ересьми́
ослепле́нныя, све́том Твоего́ позна́ния просвети́ и Святе́й Твое́й Апо́стольстей
Собо́рней Це́ркви причти́. \itshape (Поклон)\normalfont{}



 

\bfseries О усопших\normalfont{}


   Помяни́, Го́споди, от жития́ сего́ отше́дшия правове́рныя цари́ и цари́цы,
благове́рныя кня́зи и княги́ни, святе́йшия патриа́рхи, преосвяще́нныя
митрополи́ты, архиепи́скопы и епи́скопы правосла́вныя, во иере́йстем же и в
при́чте церко́внем, и мона́шестем чи́не Тебе́ послужи́вшия, и в ве́чных Твои́х
селе́ниих со святы́ми упоко́й. \itshape (Поклон.)\normalfont{}


   Помяни́, Го́споди, ду́ши усо́пших рабо́в Твои́х, роди́телей мои́х \itshape (имена
их)\normalfont{}, и всех сро́дников по пло́ти; и прости́ их вся согреше́ния во́льная и
нево́льная, да́руя им Ца́рствие и прича́стие ве́чных Твои́х благи́х и Твоея́
безконе́чныя и блаже́нныя жи́зни наслажде́ние. \itshape (Поклон)\normalfont{}


   Помяни́, Го́споди, и вся в наде́жди воскресе́ния и жи́зни ве́чныя
усо́пшия, отцы́ и бра́тию на́шу, и се́стры, и зде лежа́щия и повсю́ду,
правосла́вныя христиа́ны, и со святы́ми Твои́ми, иде́же присеща́ет свет лица́
Твоего́, всели́, и нас поми́луй, я́ко Благ и Человеколю́бец. Ами́нь.

\itshape (Поклон)\normalfont{}


   Пода́ждь, Го́споди, оставле́ние грехо́в всем пре́жде отше́дшим в ве́ре и
наде́жди воскресе́ния, отце́м, бра́тиям и се́страм на́шим и сотвори́ им ве́чную
па́мять. \itshape (Трижды)\normalfont{}



 

\bfseries Окончание молитв\normalfont{}


   Досто́йно е́сть я́ко вои́стину блажи́ти Тя Богоро́дицу, Присноблаже́нную и
Пренепоро́чную и Ма́терь Бо́га на́шего. Честне́йшую Херуви́м и сла́внейшую без
сравне́ния Серафи́м, без истле́ния Бо́га Сло́ва ро́ждшую, су́щую Богоро́дицу Тя
велича́ем[5]
.


   Сла́ва Отцу́ и Сы́ну и Свято́му Ду́ху, и ны́не и при́сно и во ве́ки веко́в.
Ами́нь.


   Го́споди, поми́луй. \itshape (Трижды)\normalfont{}


   Го́споди, Иису́се Христе́, Сы́не Бо́жий, молитв ра́ди Пречи́стыя Твоея́
Ма́тере, преподо́бных и богоно́сных оте́ц на́ших и всех святы́х поми́луй нас.
Ами́нь.

   


\mychapterending

\mychapter{Молитвы на сон грядущим}
%/text2.htm



\myfig{img/1_1.jpg}

Во и́мя Отца́, и Сы́на, и Свята́го Ду́ха. Ами́нь.


   Го́споди Иису́се Христе́, Сы́не Бо́жий, молитв ра́ди Пречи́стыя Твоея́
Ма́тере, преподо́бных и богоно́сных оте́ц на́ших и всех святы́х, поми́луй нас.
Ами́нь.


   Сла́ва Тебе́, Бо́же наш, сла́ва Тебе́.


   Царю́ Небе́сный, Уте́шителю, Ду́ше и́стины, И́же везде́ сый и вся
исполня́яй, Сокро́вище благи́х и жи́зни Пода́телю, прииди́ и всели́ся в ны, и
очи́сти ны от вся́кия скве́рны, и спаси́, Бла́же, ду́ши на́ша.


   Святы́й Бо́же, Святы́й Кре́пкий, Святы́й Безсме́ртный, поми́луй нас.
\itshape (Tрижды)\normalfont{}


   Сла́ва Отцу́ и Сы́ну и Свято́му Ду́ху, и ны́не и при́сно и во ве́ки веко́в.
Ами́нь.


   Пресвята́я Тро́ице, поми́луй нас; Го́споди, очи́сти грехи́ на́ша; Влады́ко,
прости́ беззако́ния на́ша; Святы́й, посети́ и исцели́ не́мощи на́ша, и́мене
Твоего́ ра́ди.


   Го́споди, поми́луй. \itshape (Трижды)\normalfont{}


   Сла́ва Отцу́ и Сы́ну и Свято́му Ду́ху, и ны́не и при́сно и во ве́ки веко́в.
Ами́нь.


   О́тче наш, И́же еси́ на небесе́х! Да святи́тся и́мя Твое́, да прии́дет
Ца́рствие Твое́, да бу́дет во́ля Твоя́, я́ко на небеси́ и на земли́. Хлеб наш
насу́щный даждь нам днесь; и оста́ви нам до́лги на́ша, я́коже и мы оставля́ем
должнико́м на́шим; и не введи́ нас во искуше́ние, но изба́ви нас от
лука́ваго.
   


 

\bfseries Тропари\normalfont{}


   Поми́луй нас, Го́споди, поми́луй нас; вся́каго бо отве́та недоуме́юще, сию́
Ти моли́тву я́ко Влады́це гре́шнии прино́сим: поми́луй нас.


   Сла́ва Отцу́ и Сы́ну и Свято́му Ду́ху.


   Го́споди, поми́луй нас, на Тя бо упова́хом; не прогне́вайся на ны зело́,
ниже́ помяни́ беззако́ний на́ших, но при́зри и ны́не я́ко благоутро́бен, и
изба́ви ны от враг на́ших; Ты бо еси́ Бог наш, и мы лю́дие Твои́, вси́ дела́
ру́ку Твое́ю, и и́мя Твое́ призыва́ем.


   И ны́не и при́сно и во ве́ки веко́в. Ами́нь.


   Милосе́рдия две́ри отве́рзи нам, благослове́нная Богоро́дице, наде́ющиися
на Тя да не поги́бнем, но да изба́вимся Тобо́ю от бед: Ты бо еси́ спасе́ние
ро́да христиа́нскаго.



   Го́споди, поми́луй. \itshape (12 раз)\normalfont{}



 

\bfseries Молитва 1-я, святого Макария Великого, к Богу Отцу\normalfont{}


   Бо́же ве́чный и Царю́ вся́каго созда́ния, сподо́бивый мя да́же в час сей
доспе́ти, прости́ ми грехи́, я́же сотвори́х в сей день де́лом, сло́вом и
помышле́нием, и очи́сти, Го́споди, смире́нную мою́ ду́шу от вся́кия
скве́рны пло́ти и ду́ха. И да́ждь ми, Го́споди, в нощи́ сей сон прейти́ в
ми́ре, да воста́в от смире́ннаго ми ло́жа, благоугожду́ пресвято́му
и́мени Твоему́, во вся́ дни живота́ моего́, и поперу́ борю́щия мя враги́
плотски́я и безпло́тныя. И изба́ви мя, Го́споди, от помышле́ний су́етных,
оскверня́ющих мя, и по́хотей лука́вых. Я́ко Твое́ е́сть ца́рство, и си́ла и
сла́ва, Отца́ и Сы́на и Свята́го Ду́ха, ны́не и при́сно и во ве́ки веко́в.
Ами́нь.



 

\bfseries Молитва 2-я, святого Антиоха, ко Господу нашему Иисусу Христу\normalfont{}


   Вседержи́телю, Сло́во О́тчее, Сам соверше́н сый, Иису́се Христе́, мно́гаго
ра́ди милосе́рдия Твоего́ никогда́же отлуча́йся мене́, раба́ Твоего́,
но всегда́ во мне почива́й. Иису́се, до́брый Па́стырю Твои́х ове́ц, не
преда́ждь мене́ крамоле́ змии́не, и жела́нию сатанину́ не оста́ви мене́, я́ко
се́мя тли во мне есть. Ты у́бо, Го́споди Бо́же покланя́емый, Царю́
Святы́й, Иису́се Христе́, спя́ща мя сохра́ни немерца́ющим све́том,
Ду́хом Твои́м Святы́м, И́мже освяти́л еси́ Твоя́ ученики́. Даждь,
Го́споди, и мне, недосто́йному рабу́ Твоему́, спасе́ние Твое́ на ло́жи
мое́м: просвети́ ум мой све́том ра́зума свята́го Ева́нгелия Твоего́, ду́шу
любо́вию Креста́ Твоего́, се́рдце чистото́ю словесе́ Твоего́, те́ло мое́ Твое́ю
стра́стию безстра́стною, мысль мою́ Твои́м смире́нием сохра́ни, и
воздви́гни мя во вре́мя подо́бно на Твое́ славосло́вие. Я́ко препросла́влен
еси́ со Безнача́льным Твои́м Отце́м и с Пресвяты́м Ду́хом во ве́ки.
Ами́нь.



 

\bfseries Молитва 3-я, ко Пресвятому Духу\normalfont{}


   Го́споди, Царю́ Небе́сный, Уте́шителю, Ду́ше и́стины, умилосе́рдися и
поми́луй мя гре́шнаго раба́ Твоего́, и отпусти́ ми недосто́йному, и прости́ вся́,
ели́ка Ти согреши́х днесь я́ко челове́к, па́че же и не я́ко челове́к, но и горе́е
скота́, во́льныя мо́я грехи́ и нево́льныя, ве́домыя и неве́домыя: я́же от ю́ности
и нау́ки злы, и я́же суть от на́гльства и уны́ния. А́ще и́менем Твои́м кля́хся,
или́ поху́лих е в помышле́нии мое́м; или́ кого́ укори́х; или́ оклевета́х кого́
гне́вом мои́м, или́ опеча́лих, или́ о чем прогне́вахся; или́ солга́х, или́ безго́дно
спах, или́ нищ прии́де ко мне, и презре́х его́; или́ бра́та моего́ опеча́лих, или́
сва́дих, или́ кого́ осуди́х; или́ развелича́хся, или́ разгорде́хся, или́
разгне́вахся; или́ стоя́щу ми на моли́тве, ум мой о лука́вствии ми́ра сего́
подви́жеся, или́ развраще́ние помы́слих; или́ объядо́хся, или́ опи́хся,
или́ без ума́ смея́хся; или́ лука́вое помы́слих, или́ добро́ту чужду́ю
ви́дев, и то́ю уя́звлен бых се́рдцем; или́ неподо́бная глаго́лах, или́ греху́
бра́та моего́ посмея́хся, мо́я же суть безчи́сленная согреше́ния; или́ о
моли́тве не ради́х, или́ и́но что́ соде́ях лука́вое, не по́мню, та бо вся́ и
бо́льша сих соде́ях. Поми́луй мя, Тво́рче мой Влады́ко, уны́лаго и
недосто́йнаго раба́ Твоего́, и оста́ви ми, и отпусти́, и прости́ мя, я́ко Благ и
Человеколю́бец, да с ми́ром ля́гу, усну́ и почи́ю, блу́дный, гре́шный и
окая́нный аз, и поклоню́ся, и воспою́, и просла́влю пречестно́е и́мя
Твое́, со Отце́м, и Единоро́дным Его́ Сы́ном, ны́не и при́сно и во ве́ки.
Ами́нь.



 

\bfseries Молитва 4-я, святого Макария Великого\normalfont{}


   Что́ Ти принесу́, или́ что́ Ти возда́м, великодарови́тый Безсме́ртный
Царю́, ще́дре и Человеколю́бче Го́споди, я́ко леня́щася мене́ на Твое́
угожде́ние, и ничто́же бла́го сотво́рша, приве́л еси́ на коне́ц мимоше́дшаго
дне сего́, обраще́ние и спасе́ние души́ мое́й стро́я? Ми́лостив ми бу́ди
гре́шному и обнаже́нному вся́каго де́ла бла́га, возста́ви па́дшую мою́ ду́шу,
оскверни́вшуюся в безме́рных согреше́ниих, и отыми́ от мене́ весь по́мысл
лука́вый ви́димаго сего́ жития́. Прости́ мо́я согреше́ния, еди́не Безгре́шне,
я́же Ти согреши́х в сей день, ве́дением и неве́дением, сло́вом, и де́лом, и
помышле́нием, и все́ми мои́ми чу́вствы. Ты Сам, покрыва́я, сохра́ни мя от
вся́каго сопроти́внаго обстоя́ния Боже́ственною Твое́ю вла́стию, и
неизрече́нным человеколю́бием, и си́лою. Очи́сти, Бо́же, очи́сти мно́жество
грехо́в мои́х. Благоволи́, Го́споди, изба́вити мя от се́ти лука́ваго, и спаси́

стра́стную мою́ ду́шу, и осени́ мя све́том лица́ Твоего́, егда́ прии́деши во
сла́ве, и неосужде́нна ны́не сном усну́ти сотвори́, и без мечта́ния, и
несмуще́н по́мысл раба́ Твоего́ соблюди́, и всю сатанину́ де́тель отжени́ от
мене́, и просвети́ ми разу́мныя о́чи серде́чныя, да не усну́ в смерть. И
посли́ ми А́нгела ми́рна, храни́теля и наста́вника души́ и те́лу моему́,
да изба́вит мя от враг мои́х; да воста́в со одра́ моего́, принесу́ Ти
благода́рственныя мольбы́. Ей, Го́споди, услы́ши мя гре́шнаго и убо́гаго раба́
Твоего́, изволе́нием и со́вестию; да́руй ми воста́вшу словесе́м Твои́м
поучи́тися, и уны́ние бесо́вское дале́че от мене́ отгна́но бы́ти сотвори́
Твои́ми А́нгелы; да благословлю́ и́мя Твое́ свято́е, и просла́влю, и
сла́влю Пречи́стую Богоро́дицу Мари́ю, Ю́же дал еси́ нам гре́шным
заступле́ние, и приими́ Сию́ моля́щуюся за ны; вем бо, я́ко подража́ет Твое́
человеколю́бие, и моля́щися не престае́т. Тоя́ заступле́нием, и Честна́го
Креста́ зна́мением, и всех святы́х Твои́х ра́ди, убо́гую ду́шу мою́ соблюди́,
Иису́се Христе́ Бо́же наш, я́ко Свят еси́, и препросла́влен во ве́ки.
Ами́нь.



 

\bfseries Молитва 5-я\normalfont{}


   Го́споди Бо́же наш, е́же согреши́х во дни сем сло́вом, де́лом и
помышле́нием, я́ко Благ и Человеколю́бец прости́ ми. Ми́рен сон и
безмяте́жен да́руй ми. А́нгела Твоего́ храни́теля посли́, покрыва́юща и
соблюда́юща мя от вся́каго зла, я́ко Ты еси́ храни́тель душа́м и телесе́м
на́шим, и Тебе́ сла́ву возсыла́ем, Отцу́ и Сы́ну и Свято́му Ду́ху, ны́не и
при́сно и во ве́ки веко́в. Ами́нь.



 

\bfseries Молитва 6-я\normalfont{}


   Го́споди Бо́же наш, в Него́же ве́ровахом, и Его́же и́мя па́че вся́каго и́мене
призыва́ем, да́ждь нам, ко сну отходя́щим, осла́бу души́ и те́лу, и соблюди́
нас от вся́каго мечта́ния, и те́мныя сла́сти кроме́; уста́ви стремле́ние
страсте́й, угаси́ разжже́ния воста́ния теле́снаго. Даждь нам целому́дренне
пожи́ти де́лы и словесы́; да доброде́тельное жи́тельство восприе́млюще,
обетова́нных не отпаде́м благи́х Твои́х, я́ко благослове́н еси́ во ве́ки.
Ами́нь.




 

\bfseries Молитва 7-я, святого Иоанна Златоуста (24 молитвы, по числу часов дня
и ночи)\normalfont{}


   Го́споди, не лиши́ мене́ небе́сных Твои́х благ. Го́споди, изба́ви мя ве́чных
мук. Го́споди, умо́м ли или́ помышле́нием, сло́вом или́ де́лом согреши́х,
прости́ мя. Го́споди, изба́ви мя вся́каго неве́дения и забве́ния, и малоду́шия, и
окамене́ннаго нечу́вствия. Го́споди, изба́ви мя от вся́каго искуше́ния.
Го́споди, просвети́ мое́ се́рдце, е́же помрачи́ лука́вое похоте́ние. Го́споди,
аз я́ко челове́к согреши́х, Ты же я́ко Бог щедр, поми́луй мя, ви́дя
не́мощь ду́ши моея́. Го́споди, посли́ благода́ть Твою́ в по́мощь мне,
да просла́влю и́мя Твое́ свято́е. Го́споди Иису́се Христе́, напиши́ мя
раба́ Твоего́ в кни́зе живо́тней и да́руй ми коне́ц благи́й. Го́споди,
Бо́же мой, а́ще и ничто́же бла́го сотвори́х пред Тобо́ю, но да́ждь
ми по благода́ти Твое́й положи́ти нача́ло благо́е. Го́споди, окропи́ в
се́рдце мое́м ро́су благода́ти Твоея́. Го́споди небесе́ и земли́, помяни́ мя
гре́шнаго раба́ Твоего́, сту́днаго и нечи́стаго, во Ца́рствии Твое́м.
Ами́нь.


   Го́споди, в покая́нии приими́ мя. Го́споди, не оста́ви мене́. Го́споди, не
введи́ мене́ в напа́сть. Го́споди, да́ждь ми мысль бла́гу. Го́споди, да́ждь ми
сле́зы и па́мять сме́ртную, и умиле́ние. Го́споди, да́ждь ми по́мысл
испове́дания грехо́в мои́х. Го́споди, да́ждь ми смире́ние, целому́дрие и
послуша́ние. Го́споди, да́ждь ми терпе́ние, великоду́шие и кро́тость. Го́споди,
всели́ в мя ко́рень благи́х, страх Твой в се́рдце мое́. Го́споди, сподо́би мя
люби́ти Тя от всея́ души́ моея́ и помышле́ния и твори́ти во всем во́лю Твою́.
Го́споди, покры́й мя от челове́к не́которых, и бесо́в, и страсте́й, и от вся́кия
ины́я неподо́бныя ве́щи. Го́споди, ве́си, я́ко твори́ши, я́коже Ты во́лиши,
да бу́дет во́ля Твоя́ и во мне гре́шнем, я́ко благослове́н еси́ во ве́ки.
Ами́нь.



 

\bfseries Молитва 8-я, ко Господу нашему Иисусу Христу\normalfont{}


   Го́споди Иису́се Христе́, Сы́не Бо́жий, ра́ди честне́йшия Ма́тере Твоея́,
и безпло́тных Твои́х А́нгел, Проро́ка же и Предте́чи и Крести́теля
Твоего́, богоглаго́ливых же апо́стол, све́тлых и добропобе́дных му́ченик,

преподо́бных и богоно́сных оте́ц, и всех святы́х моли́твами, изба́ви мя
настоя́щаго обстоя́ния бесо́вскаго. Ей, Го́споди мой и Тво́рче, не хотя́й
сме́рти гре́шнаго, но я́коже обрати́тися и жи́ву бы́ти ему́, да́ждь и мне
обраще́ние окая́нному и недосто́йному; изми́ мя от уст па́губнаго
зми́я, зия́ющаго пожре́ти мя и свести́ во ад жи́ва. Ей, Го́споди мой,
утеше́ние мое́, И́же мене́ ра́ди окая́ннаго в тле́нную плоть оболки́йся,
исто́ргни мя от окая́нства, и утеше́ние пода́ждь души́ мое́й окая́нней.
Всади́ в се́рдце мое́ твори́ти Твоя́ повеле́ния, и оста́вити лука́вая
дея́ния, и получи́ти блаже́нства Твоя́: на Тя бо, Го́споди, упова́х, спаси́
мя.



 

\bfseries Молитва 9-я, ко Пресвятой Богородице, Петра Студийского\normalfont{}


   К Тебе́ Пречи́стей Бо́жией Ма́тери аз окая́нный припа́дая молю́ся: ве́си,
Цари́це, я́ко безпреста́ни согреша́ю и прогневля́ю Сы́на Твоего́ и Бо́га
моего́, и мно́гажды а́ще ка́юся, лож пред Бо́гом обрета́юся, и ка́юся
трепе́ща: неуже́ли Госпо́дь порази́т мя, и по часе́ па́ки та́яжде творю́;
ве́дущи сия́, Влады́чице мо́я Госпоже́ Богоро́дице, молю́, да поми́луеши,
да укрепи́ши, и блага́я твори́ти да пода́си ми. Ве́си бо, Влады́чице
моя́ Богоро́дице, я́ко отню́д и́мам в не́нависти зла́я моя́ дела́, и все́ю
мы́слию люблю́ зако́н Бо́га моего́; но не вем, Госпоже́ Пречи́стая,
отку́ду я́же ненави́жду, та и люблю́, а блага́я преступа́ю. Не попуща́й,
Пречи́стая, во́ли мое́й соверша́тися, не уго́дна бо есть, но да бу́дет
во́ля Сы́на Твоего́ и Бо́га моего́: да мя спасе́т, и вразуми́т, и пода́ст
благода́ть Свята́го Ду́ха, да бых аз отсе́ле преста́л скверноде́йства, и
про́чее пожи́л бых в повеле́нии Сы́на Твоего́, Ему́же подоба́ет вся́кая
сла́ва, честь и держа́ва, со Безнача́льным Его́ Отце́м, и Пресвяты́м и
Благи́м и Животворя́щим Его́ Ду́хом, ны́не и при́сно, и во ве́ки веко́в.
Aминь.



 

\bfseries Молитва 10-я, ко Пресвятой Богородице\normalfont{}


   Блага́го Царя́ блага́я Ма́ти, Пречи́стая и Благослове́нная Богоро́дице
Мари́е, ми́лость Сы́на Твоего́ и Бо́га на́шего изле́й на стра́стную мою́ ду́шу и
Твои́ми моли́твами наста́ви мя на дея́ния блага́я, да про́чее вре́мя живота́

моего́ без поро́ка прейду́ и Тобо́ю рай да обря́щу, Богоро́дице Де́во, еди́на
Чи́стая и Благослове́нная.



 

\bfseries Молитва 11-я, ко святому Ангелу хранителю\normalfont{}


   А́нгеле Христо́в, храни́телю мой святы́й и покрови́телю души́ и те́ла
моего́, вся́ ми прости́, ели́ка согреши́х во дне́шний день, и от вся́каго
лука́вствия проти́внаго ми врага́ изба́ви мя, да ни в ко́емже гресе́ прогне́ваю
Бо́га моего́; но моли́ за мя гре́шнаго и недосто́йнаго раба́, я́ко да досто́йна мя
пока́жеши бла́гости и ми́лости Всесвяты́я Тро́ицы и Ма́тере Го́спода моего́
Иису́са Христа́ и всех святы́х. Ами́нь.



 

\bfseries Кондак Богородице\normalfont{}


   Взбра́нной Воево́де победи́тельная, я́ко изба́вльшеся от злых,
благода́рственная воспису́ем Ти раби́ Твои́, Богоро́дице, но я́ко иму́щая
держа́ву непобеди́мую, от вся́ких нас бед свободи́, да зове́м Ти; ра́дуйся,
Неве́сто Неневе́стная.


   Пресла́вная Присноде́во, Ма́ти Христа́ Бо́га, принеси́ на́шу моли́тву Сы́ну
Твоему́ и Бо́гу на́шему, да спасе́т Тобо́ю ду́ши на́ша.


   Все упова́ние мое́ на Тя возлага́ю, Ма́ти Бо́жия, сохра́ни мя под кро́вом
Твои́м.


   Богоро́дице Де́во, не пре́зри мене́, гре́шнаго, тре́бующа Твоея́
по́мощи и Твоего́ заступле́ния, на Тя бо упова́ душа́ мо́я, и поми́луй
мя.



 

\bfseries Молитва святого Иоанникия\normalfont{}


   Упова́ние мое́ Оте́ц, прибе́жище мое́ Сын, покро́в мой Дух Святы́й:
Тро́ице Свята́я, сла́ва Тебе́.


   Досто́йно е́сть я́ко вои́стинну блажи́ти Тя, Богоро́дицу, Присноблаже́нную и
Пренепоро́чную и Ма́терь Бо́га на́шего. Честне́йшую Херуви́м и сла́внейшую
без сравне́ния Серафи́м, без истле́ния Бо́га Сло́ва ро́ждшую, су́щую

Богоро́дицу Тя велича́ем.


   Сла́ва Отцу́ и Сы́ну и Свято́му Ду́ху, и ны́не и при́сно и во ве́ки веко́в.
Ами́нь.


   Го́споди, поми́луй. \itshape (Трижды)\normalfont{}


   Го́споди Иису́се Христе́, Сы́не Бо́жий, молитв ра́ди Пречи́стыя Твоея́
Ма́тере, преподо́бных и богоно́сных оте́ц на́ших и всех святы́х, поми́луй нас.
Ами́нь.



 

\bfseries Молитва святого Иоанна Дамаскина\normalfont{}


   Влады́ко Человеколю́бче, неуже́ли мне одр сей гроб бу́дет, или́ еще́
окая́нную мою́ ду́шу просвети́ши днем? Се ми гроб предлежи́т, се ми смерть
предстои́т. Суда́ Твоего́, Го́споди, бою́ся и му́ки безконе́чныя, зло́е же творя́
не престаю́: Тебе́ Го́спода Бо́га моего́ всегда́ прогневля́ю, и Пречи́стую Твою́
Ма́терь, и вся́ Небе́сныя си́лы, и свята́го А́нгела храни́теля моего́. Вем у́бо,
Го́споди, я́ко недосто́ин есмь человеколю́бия Твоего́, но досто́ин есмь вся́каго
осужде́ния и му́ки. Но, Го́споди, или́ хощу́, или́ не хощу́, спаси́ мя. А́ще бо
пра́ведника спасе́ши, ничто́же ве́лие; и а́ще чи́стаго поми́луеши, ничто́же
ди́вно: досто́йни бо суть ми́лости Твоея́. Но на мне гре́шнем удиви́ ми́лость
Твою́: о сем яви́ человеколю́бие Твое́, да не одоле́ет мо́я зло́ба Твое́й
неизглаго́ланней бла́гости и милосе́рдию: и я́коже хо́щеши, устро́й о мне
вещь.


   Просвети́ о́чи мои́, Христе́ Бо́же, да не когда́ усну́ в смерть, да не когда́
рече́т враг мой: укрепи́хся на него́.


   Сла́ва Отцу́ и Сы́ну и Свято́му Ду́ху.


   Засту́пник души́ моея́ бу́ди, Бо́же, я́ко посреде́ хожду́ сете́й мно́гих;
изба́ви мя от них и спаси́ мя, Бла́же, я́ко Человеколю́бец.


   И ны́не и при́сно и во ве́ки веко́в. Ами́нь.


   Пресла́вную Бо́жию Ма́терь, и святы́х А́нгел Святе́йшую, немо́лчно
воспои́м се́рдцем и усты́, Богоро́дицу сию́ испове́дающе, я́ко вои́стинну
ро́ждшую нам Бо́га воплоще́нна, и моля́щуюся непреста́нно о душа́х
на́ших.



 

\bfseries Знаменуй себя крестом и говори молитву Честно́му Кресту:\normalfont{}


   Да воскре́снет Бог, и расточа́тся врази́ Его́, и да бежа́т от лица́ Его́
ненави́дящии Его́. Я́ко исчеза́ет дым, да исче́знут; я́ко та́ет воск от лица́
огня́, та́ко да поги́бнут бе́си от лица́ лю́бящих Бо́га и зна́менующихся
кре́стным зна́мением, и в весе́лии глаго́лющих: ра́дуйся, Пречестны́й и
Животворя́щий Кре́сте Госпо́день, прогоня́яй бе́сы си́лою на тебе́ пропя́таго
Го́спода на́шего Иису́са Христа́, во ад сше́дшаго и попра́вшего си́лу диа́волю,
и дарова́вшаго нам тебе́ Крест Свой Честны́й на прогна́ние вся́каго
супоста́та. О, Пречестны́й и Животворя́щий Кре́сте Госпо́день! Помога́й ми
со Свято́ю Госпоже́ю Де́вою Богоро́дицею и со все́ми святы́ми во ве́ки.
Ами́нь.


 \itshape Или кратко:\normalfont{}


   Огради́ мя, Го́споди, си́лою Честна́го и Животворя́щаго Твоего́ Креста́, и
сохра́ни мя от вся́каго зла.



 

\bfseries Молитва\normalfont{}


   Осла́би, оста́ви, прости́, Бо́же, прегреше́ния на́ша, во́льная и нево́льная,
я́же в сло́ве и в де́ле, я́же в ве́дении и не в ве́дении, я́же во дни и
в нощи́, я́же во уме́ и в помышле́нии: вся́ нам прости́, я́ко Благ и
Человеколю́бец.



 

\bfseries Молитва\normalfont{}


   Ненави́дящих и оби́дящих нас прости́, Го́споди Человеколю́бче.
Благотворя́щим благосотвори́. Бра́тиям и сро́дником на́шим да́руй я́же ко
спасе́нию проше́ния и жизнь ве́чную. В не́мощех су́щия посети́ и исцеле́ние
да́руй. И́же на мо́ри упра́ви. Путеше́ствующим спутеше́ствуй. Правосла́вным
христиа́ном спобо́рствуй. Служа́щим и ми́лующим нас грехо́в оставле́ние
да́руй. Запове́давших нам недосто́йным моли́тися о них поми́луй по вели́цей
Твое́й ми́лости. Помяни́, Го́споди, пре́жде усо́пших оте́ц и бра́тий на́ших и
упоко́й их, иде́же присеща́ет свет лица́ Твоего́. Помяни́, Го́споди,
бра́тий на́ших плене́нных и изба́ви я от вся́каго обстоя́ния. Помяни́,
Го́споди, плодонося́щих и доброде́лающих во святы́х Твои́х це́рквах,
и да́ждь им я́же ко спасе́нию проше́ния и жизнь ве́чную. Помяни́,
Го́споди, и нас, смире́нных и гре́шных и недосто́йных раб Твои́х,

и просвети́ наш ум све́том ра́зума Твоего́, и наста́ви нас на стезю́
за́поведей Твои́х, моли́твами Пречи́стыя Влады́чицы на́шея Богоро́дицы и
Присноде́вы Мари́и и всех Твои́х святы́х: я́ко благослове́н еси́ во ве́ки веко́в.
Ами́нь.



 

\bfseries Исповедание грехов повседневное\normalfont{}


   Испове́даю Тебе́ Го́споду Бо́гу моему́ и Творцу́, во Святе́й Тро́ице
Еди́ному, сла́вимому и покланя́емому, Отцу́ и Сы́ну и Свято́му Ду́ху, вся́
мо́я грехи́, я́же соде́ях во вся́ дни живота́ моего́, и на вся́кий час,
и в настоя́щее вре́мя, и в преше́дшия дни и но́щи, де́лом, сло́вом,
помышле́нием, объяде́нием, пия́нством, тайнояде́нием, праздносло́вием,
уны́нием, ле́ностию, прекосло́вием, непослуша́нием, оклевета́нием,
осужде́нием, небреже́нием, самолю́бием, многостяжа́нием, хище́нием,
неправдоглаго́ланием, скверноприбы́тчеством, мшелои́мством, ревнова́нием,
за́вистию, гне́вом, памятозло́бием, не́навистию, лихои́мством и все́ми
мои́ми чу́вствы: зре́нием, слу́хом, обоня́нием, вку́сом, осяза́нием и
про́чими мои́ми грехи́, душе́вными вку́пе и теле́сными, и́миже Тебе́
Бо́га моего́ и Творца́ прогне́вах, и бли́жняго моего́ онепра́вдовах: о
сих жале́я, ви́нна себе́ Тебе́ Бо́гу моему́ представля́ю, и име́ю во́лю
ка́ятися: то́чию, Го́споди Бо́же мой, помози́ ми, со слеза́ми смире́нно
молю́ Тя: преше́дшая же согреше́ния мо́я милосе́рдием Твои́м прости́
ми, и разреши́ от всех сих, я́же изглаго́лах пред Тобо́ю, я́ко Благ и
Человеколю́бец.



 

\bfseries  \itshape Когда отходишь ко сну, произноси:\normalfont{}\normalfont{}


   В ру́це Твои́, Го́споди Иису́се Христе́, Бо́же мой, предаю́ дух мой:
Ты же мя благослови́, Ты мя поми́луй и живо́т ве́чный да́руй ми.
Ами́нь.

   


\mychapterending

\mychapter{Канон покаянный ко Господу нашему Иисусу Христу}
%/text3.htm



\myfig{img/748.jpg}

\bfseries Глас 6-й, Песнь 1\normalfont{}


\itshape Ирмо&#x301;с:\normalfont{} Яко по суху пешешествовав Израиль, по бездне стопами, гонителя фараона видя потопляема, Богу победную песнь поим, вопияше.


\itshape Припев:\normalfont{} Помилуй мя, Боже, помилуй мя.


Ныне приступих аз грешный и обремененный к Тебе, Владыце и Богу моему; не смею же взирати на небо, токмо молюся, глаголя: даждь ми, Господи, ум, да плачуся дел моих горько.


\itshape Припев:\normalfont{} Помилуй мя, Боже, помилуй мя.


О, горе мне грешному! Паче всех человек окаянен есмь, покаяния несть во мне; даждь ми, Господи, слезы, да плачуся дел моих горько.


Слава Отцу и Сыну и Святому Духу.


Безумне, окаянне человече, в лености время губиши; помысли житие твое, и обратися ко Господу Богу, и плачися о делех твоих горько.


И ныне и присно и во веки веков. Аминь.


Мати Божия Пречистая, воззри на мя грешного, и от сети диаволи избави мя, и на путь покаяния настави мя, да плачуся дел моих горько.




\bfseries Песнь 3\normalfont{}


\itshape Ирмо&#x301;с:\normalfont{} Несть свят, якоже Ты, Господи Боже мой, вознесый рог верных Твоих, Блаже, и утвердивый нас на камени исповедания Твоего.


\itshape Припев:\normalfont{} Помилуй мя, Боже, помилуй мя.


Внегда поставлени будут престоли на судищи страшнем, тогда всех человек дела обличатся; горе тамо будет грешным, в муку отсылаемым; и то ведущи, душе моя, покайся от злых дел твоих.


\itshape Припев:\normalfont{} Помилуй мя, Боже, помилуй мя.


Праведницы возрадуются, а грешнии восплачутся, тогда никтоже возможет помощи нам, но дела наша осудят нас, темже прежде конца покайся от злых дел твоих.


Слава Отцу и Сыну и Святому Духу.


Увы мне великогрешному, иже делы и мысльми осквернився, ни капли слез имею от жестосердия; ныне возникни от земли, душе моя, и покайся от злых дел твоих.


И ныне и присно и во веки веков. Аминь.


Се, взывает, Госпоже, Сын Твой, и поучает нас на доброе, аз же грешный добра всегда бегаю; но Ты, Милостивая, помилуй мя, да покаюся от злых моих дел.




\bfseries Седален, глас 6-й\normalfont{}


Помышляю день страшный и плачуся деяний моих лукавых: како отвещаю Безсмертному Царю, или коим дерзновением воззрю на Судию, блудный аз? Благоутробный Отче, Сыне Единородный и Душе Святый, помилуй мя.




\bfseries Богородичен\normalfont{}


Слава Отцу и Сыну и Святому Духу. И ныне и присно и во веки веков. Аминь.


Связан многими ныне пленицами грехов и содержим лютыми страстьми и бедами, к Тебе прибегаю, моему спасению, и вопию: помози ми, Дево, Мати Божия.




\bfseries Песнь 4\normalfont{}


\itshape Ирмо&#x301;с:\normalfont{} Христос моя сила, Бог и Господь, честная Церковь боголепно поет, взывающи от смысла чиста, о Господе празднующи.


\itshape Припев:\normalfont{} Помилуй мя, Боже, помилуй мя.


Широк путь зде и угодный сласти творити, но горько будет в последний день, егда душа от тела разлучатися будет: блюдися от сих, человече, Царствия ради Божия.


\itshape Припев:\normalfont{} Помилуй мя, Боже, помилуй мя.


Почто убогаго обидиши, мзду наемничу удержуеши, брата твоего не любиши, блуд и гордость гониши? Остави убо сия, душе моя, и покайся Царствия ради Божия.


Слава Отцу и Сыну и Святому Духу.


О, безумный человече, доколе углебаеши, яко пчела, собирающи богатство твое? Вскоре бо погибнет, яко прах и пепел: но более взыщи Царствия Божия.


И ныне и присно и во веки веков. Аминь.


Госпоже Богородице, помилуй мя грешного, и в добродетели укрепи, и соблюди мя, да наглая смерть не похитит мя неготоваго, и доведи мя, Дево, Царствия Божия.




\bfseries Песнь 5\normalfont{}


\itshape Ирмо&#x301;с:\normalfont{} Божиим светом Твоим, Блаже, утренюющих Ти души любовию озари, молюся, Тя ведети, Слове Божий, истиннаго Бога, от мрака греховнаго взывающа.


\itshape Припев:\normalfont{} Помилуй мя, Боже, помилуй мя.


Воспомяни, окаянный человече, како лжам, клеветам, разбою, немощем, лютым зверем, грехов ради порабощен еси; душе моя грешная, того ли восхотела еси?


\itshape Припев:\normalfont{} Помилуй мя, Боже, помилуй мя.


Трепещут ми уди, всеми бо сотворих вину: очима взираяй, ушима слышай, языком злая глаголяй, всего себе геенне предаяй; душе моя грешная, сего ли восхотела еси?


Слава Отцу и Сыну и Святому Духу.


Блудника и разбойника кающася приял еси, Спасе, аз же един леностию греховною отягчихся и злым делом поработихся, душе моя грешная, сего ли восхотела еси?


И ныне и присно и во веки веков. Аминь.


Дивная и скорая помощнице всем человеком, Мати Божия, помози мне недостойному, душа бо моя грешная того восхоте.




\bfseries Песнь 6\normalfont{}


\itshape Ирмо&#x301;с:\normalfont{} Житейское море, воздвизаемое зря напастей бурею, к тихому пристанищу Твоему притек, вопию Ти: возведи от тли живот мой, Многомилостиве.


\itshape Припев:\normalfont{} Помилуй мя, Боже, помилуй мя.


Житие на земли блудно пожих и душу во тьму предах, ныне убо молю Тя, Милостивый Владыко: свободи мя от работы сея вражия, и даждь ми разум творити волю Твою.


\itshape Припев:\normalfont{} Помилуй мя, Боже, помилуй мя.


Кто творит таковая, якоже аз? Якоже бо свиния лежит в калу, тако и аз греху служу. Но Ты, Господи, исторгни мя от гнуса сего и даждь ми сердце творити заповеди Твоя.


Слава Отцу и Сыну и Святому Духу.


Воспряни, окаянный человече, к Богу, воспомянув своя согрешения, припадая ко Творцу, слезя и стеня; Той же, яко милосерд, даст ти ум знати волю Свою.


И ныне и присно и во веки веков. Аминь.


Богородице Дево, от видимаго и невидимаго зла сохрани мя, Пречистая, и приими молитвы моя, и донеси я Сыну Твоему, да даст ми ум творити волю Его.




\bfseries Кондак\normalfont{}


Душе моя, почто грехами богатееши, почто волю диаволю твориши, в чесом надежду полагаеши? Престани от сих и обратися к Богу с плачем, зовущи: милосерде Господи, помилуй мя грешнаго.




\bfseries Икос\normalfont{}


Помысли, душе моя, горький час смерти и страшный суд Творца твоего и Бога: Ангели бо грознии поймут тя, душе, и в вечный огнь введут: убо прежде смерти покайся, вопиющи: Господи, помилуй мя грешнаго.




\bfseries Песнь 7\normalfont{}


\itshape Ирмо&#x301;с:\normalfont{} Росодательну убо пещь содела Ангел преподобным отроком, халдеи же опаляющее веление Божие, мучителя увеща вопити: благословен еси, Боже отец наших.


\itshape Припев:\normalfont{} Помилуй мя, Боже, помилуй мя.


Не надейся, душе моя, на тленное богатство и на неправедное собрание, вся бо сия не веси кому оставиши, но возопий: помилуй мя, Христе Боже, недостойнаго.


\itshape Припев:\normalfont{} Помилуй мя, Боже, помилуй мя.


Не уповай, душе моя, на телесное здравие и на скоромимоходящую красоту, видиши бо, яко сильнии и младии умирают; но возопий: помилуй мя, Христе Боже, недостойнаго.


Слава Отцу и Сыну и Святому Духу.


Воспомяни, душе моя, вечное житие, Царство Небесное, уготованное святым, и тьму кромешную и гнев Божий злым, и возопий: помилуй мя, Христе Боже, недостойнаго.


И ныне и присно и во веки веков. Аминь.


Припади, душе моя, к Божией Матери и помолися Той, есть бо скорая помощница кающимся, умолит Сына Христа Бога, и помилует мя недостойнаго.




\bfseries Песнь 8\normalfont{}


\itshape Ирмо&#x301;с:\normalfont{} Из пламене преподобным росу источил еси и праведнаго жертву водою попалил еси: вся бо твориши, Христе, токмо еже хотети. Тя превозносим во вся веки.


\itshape Припев:\normalfont{} Помилуй мя, Боже, помилуй мя.


Како не имам плакатися, егда помышляю смерть, видех бо во гробе лежаща брата моего, безславна и безобразна? Что убо чаю, и на что надеюся? Токмо даждь ми, Господи, прежде конца покаяние. \itshape (Дважды)\normalfont{}


Слава Отцу и Сыну и Святому Духу.


Верую, яко приидеши судити живых и мертвых, и вси во своем чину станут, старии и младии, владыки и князи, девы и священницы; где обрящуся аз? Сего ради вопию: даждь ми, Господи, прежде конца покаяние.


И ныне и присно и во веки веков. Аминь.


Пречистая Богородице, приими недостойную молитву мою и сохрани мя от наглыя смерти, и даруй ми прежде конца покаяние.




\bfseries Песнь 9\normalfont{}


\itshape Ирмо&#x301;с:\normalfont{} Бога человеком невозможно видети, на Негоже не смеют чини Ангельстии взирати; Тобою же, Всечистая, явися человеком Слово Воплощенно, Егоже величающе, с небесными вои Тя ублажаем.


\itshape Припев:\normalfont{} Помилуй мя, Боже, помилуй мя.


Ныне к вам прибегаю, Ангели, Архангели и вся небесныя силы, у Престола Божия стоящии, молитеся ко Творцу своему, да избавит душу мою от муки вечныя.


\itshape Припев:\normalfont{} Помилуй мя, Боже, помилуй мя.


Ныне плачуся к вам, святии патриарси, царие и пророцы, апостоли и святителие и вси избраннии Христовы: помозите ми на суде, да спасет душу мою от силы вражия.


Слава Отцу и Сыну и Святому Духу.


Ныне к вам воздежу руце, святии мученицы, пустынницы, девственницы, праведницы и вси святии, молящиися ко Господу за весь мир, да помилует мя в час смерти моея.


И ныне и присно и во веки веков. Аминь.


Мати Божия, помози ми, на Тя сильне надеющемуся, умоли Сына Своего, да поставит мя недостойнаго одесную Себе, егда сядет судяй живых и мертвых, аминь.




\bfseries Молитва\normalfont{}


Господи Иисусе Христе, Сыне Божий, помилуй мя грешнаго.


Владыко Христе Боже, Иже страстьми Своими страсти моя исцеливый и язвами Своими язвы моя уврачевавый, даруй мне, много Тебе прегрешившему, слезы умиления; сраствори моему телу от обоняния Животворящаго Тела Твоего, и наслади душу мою Твоею Честною Кровию от горести, еюже мя сопротивник напои; возвыси мой ум к Тебе, долу поникший, и возведи от пропасти погибели: яко не имам покаяния, не имам умиления, не имам слезы утешительныя, возводящия чада ко своему наследию. Омрачихся умом в житейских страстех, не могу воззрети к Тебе в болезни, не могу согретися слезами, яже к Тебе любве. Но, Владыко Господи Иисусе Христе, сокровище благих, даруй мне покаяние всецелое и сердце люботрудное во взыскание Твое, даруй мне благодать Твою и обнови во мне зраки Твоего образа. Оставих Тя, не остави мене; изыди на взыскание мое, возведи к пажити Твоей и сопричти мя овцам избраннаго Твоего стада, воспитай мя с ними от злака Божественных Твоих Таинств, молитвами Пречистыя Твоея Матере и всех святых Твоих. Аминь.


\mychapterending

\mychapter{Канон молебный ко Пресвятой Богородице}
%/text4.htm



\myfig{img/3.jpg}

\bfseries Поемый во всякой скорби душевной и обстоянии.\normalfont{}


\bfseries Tворение Феостирикта монаха.\normalfont{}




\bfseries Тропaрь Богородице, глас 4-й\normalfont{}


К Богородице прилежно ныне притецем, грешнии и смиреннии, и припадем, в покаянии зовуще из глубины души: Владычице, помози, на ны милосердовавши, потщися, погибaем от множества прегрешений, не отврати Твоя рабы тщи, Тя бо и едину надежду имамы. \itshape (Дважды)\normalfont{}


Слава Отцу и Сыну и Святому Духу. И ныне и присно и во веки веков. Аминь.


Не умолчим никогда, Богородице, силы Твоя глаголати, недостойнии: aще бо Ты не бы предстояла молящи, кто бы нас избaвил от толиких бед, кто же бы сохранил до ныне свободны? Не отступим, Владычице, от Тебе: Твоя бо рабы спасaеши присно от всяких лютых.




\bfseries Псалом 50\normalfont{}


Помилуй мя, Боже, по велицей милости Твоей, и по множеству щедрот Твоих очисти беззаконие мое. Наипaче омый мя от беззакония моего, и от греха моего очисти мя; яко беззаконие мое аз знaю, и грех мой предо мною есть выну. Тебе единому согреших и лукaвое пред Тобою сотворих; яко да оправдишися во словесех Твоих, и победиши внегдa судити Ти. Се бо, в беззакониих зачaт есмь, и во гресех роди мя мaти моя. Се бо, истину возлюбил еси; безвестная и тайная премудрости Твоея явил ми еси. Окропиши мя иссопом, и очищуся; омыеши мя, и пaче снега убелюся. Слуху моему дaси рaдость и веселие; возрaдуются кости смиренныя. Отврати лице Твое от грех моих и вся беззакония моя очисти. Сердце чисто созижди во мне, Боже, и дух прав обнови во утробе моей. Не отвержи мене от лица Твоего и Духа Твоего Святaго не отыми от мене. Воздaждь ми рaдость спасения Твоего и Духом Владычним утверди мя. Научу беззаконныя путем Твоим, и нечестивии к Тебе обратятся. Избaви мя от кровей, Боже, Боже спасения моего; возрaдуется язык мой правде Твоей. Господи, устне мои отверзеши, и устa моя возвестят хвалу Твою. Яко aще бы восхотел еси жертвы, дал бых убо: всесожжения не благоволиши. Жертва Богу дух сокрушен; сердце сокрушенно и смиренно Бог не уничижит. Ублажи, Господи, благоволением Твоим Сиона, и да созиждутся стены Иерусалимския. Тогда благоволиши жертву правды, возношение и всесожигaемая; тогда возложaт на олтaрь Твой тельцы.




\bfseries Канон ко Пресвятой Богородице, глас 8-й\normalfont{}




\bfseries Песнь 1\normalfont{}


\itshape Ирмо&#x301;с:\normalfont{} Воду прошед яко сушу, и египетскаго зла избежaв, изрaильтянин вопияше: избaвителю и Богу нашему поим.


\itshape Припев:\normalfont{} Пресвятая Богородице, спаси нас.


Многими содержимь напaстьми, к Тебе прибегаю, спасения иский: о, Мaти Слова и Дево, от тяжких и лютых мя спаси.


\itshape Припев:\normalfont{} Пресвятая Богородице, спаси нас.


Страстей мя смущaют прилози, многаго уныния исполнити мою душу; умири, Отроковице, тишиною Сына и Бога Твоего, Всенепорочная.


Слава Отцу и Сыну и Святому Духу.


Спaса рождшую Тя и Бога, молю, Дево, избaвитися ми лютых: к Тебе бо ныне прибегaя, простирaю и душу и помышление.


И ныне и присно и во веки веков. Аминь.


Недугующа телом и душею, посещения Божественнаго и промышления от Тебе сподоби, едина Богомaти, яко благая, Благaго же Родительница.




\bfseries Песнь 3\normalfont{}


\itshape Ирмо&#x301;с:\normalfont{} Небеснаго круга Верхотворче, Господи, и Церкве Зиждителю, Ты мене утверди в любви Твоей, желaний крaю, верных утверждение, едине Человеколюбче.


\itshape Припев:\normalfont{} Пресвятая Богородице, спаси нас.


Предстaтельство и покров жизни моея полагaю Тя, Богородительнице Дево: Ты мя окорми ко пристaнищу Твоему, благих виновна; верных утверждение, едина Всепетая.


\itshape Припев:\normalfont{} Пресвятая Богородице, спаси нас.


Молю, Дево, душевное смущение и печали моея бурю разорити: Ты бо, Богоневестная, начальника тишины Христа родилa еси, едина Пречистая.


Слава Отцу и Сыну и Святому Духу.


Благодетеля рождши добрых виновнаго, благодеяния богатство всем источи, вся бо можеши, яко сильнаго в крепости Христа рождши, Богоблаженная.


И ныне и присно и во веки веков. Аминь.


Лютыми недуги и болезненными страстьми истязaему, Дево, Ты ми помози: исцелений бо неоскудное Тя знаю сокровище, Пренепорочная, неиждивaемое.


Спаси от бед рабы Твоя, Богородице, яко вси по Бозе к Тебе прибегaем, яко нерушимей стене и предстaтельству.


Призри благосердием, всепетая Богородице, на мое лютое телесе озлобление, и исцели души моея болезнь.




\bfseries Тропарь, глас 2-й\normalfont{}


Моление теплое и стенa необоримая, милости источниче, мирови прибежище, прилежно вопием Ти: Богородице Владычице, предвари, и от бед избaви нас, едина вскоре предстaтельствующая.




\bfseries Песнь 4\normalfont{}


\itshape Ирмо&#x301;с:\normalfont{} Услышах, Господи, смотрения Твоего тaинство, разумех дела Твоя и прослaвих Твое Божество.


\itshape Припев:\normalfont{} Пресвятая Богородице, спаси нас.


Страстей моих смущение, кормчию рождшая Господа, и бурю утиши моих прегрешений, Богоневестная.


\itshape Припев:\normalfont{} Пресвятая Богородице, спаси нас.


Милосердия Твоего бездну призывaющу подaждь ми, яже Благосердаго рождшая и Спaса всех поющих Тя.


\itshape Припев:\normalfont{} Пресвятая Богородице, спаси нас.


Наслаждaющеся, Пречистая, Твоих даровaний, благодaрственное воспевaем пение, ведуще Тя Богомaтерь.


Слава Отцу и Сыну и Святому Духу.


На одре болезни моея и немощи низлежaщу ми, яко Благолюбива, помози, Богородице, едина Приснодево.


И ныне и присно и во веки веков. Аминь.


Надежду и утверждение и спасения стену недвижиму имуще Тя, Всепетая, неудобства всякаго избавляемся.




\bfseries Песнь 5\normalfont{}


\itshape Ирмо&#x301;с:\normalfont{} Просвети нас повелении Твоими, Господи, и мышцею Твоею высокою Твой мир подaждь нам, Человеколюбче.


\itshape Припев:\normalfont{} Пресвятая Богородице, спаси нас.


Исполни, Чистая, веселия сердце мое, Твою нетленную дающи радость, веселия рождшая виновнаго.


\itshape Припев:\normalfont{} Пресвятая Богородице, спаси нас.


Избaви нас от бед, Богородице чистая, вечное рождши избавление, и мир, всяк ум преимущий.


Слава Отцу и Сыну и Святому Духу.


Разреши мглу прегрешений моих, Богоневесто, просвещением Твоея светлости, Свет рождшая Божественный и превечный.


И ныне и присно и во веки веков. Аминь.


Исцели, Чистая, души моея неможение, посещения Твоего сподобльшая, и здрaвие молитвами Твоими подaждь ми.




\bfseries Песнь 6\normalfont{}


\itshape Ирмо&#x301;с:\normalfont{} Молитву пролию ко Господу, и Тому возвещу печали моя, яко зол душа моя исполнися, и живот мой аду приближися, и молюся яко Иона: от тли, Боже, возведи мя.


\itshape Припев:\normalfont{} Пресвятая Богородице, спаси нас.


Смерти и тли яко спасл есть, Сам Ся издaв смерти, тлением и смертию мое естество, ято бывшее, Дево, моли Господа и Сына Твоего, врагов злодействия мя избaвити.


\itshape Припев:\normalfont{} Пресвятая Богородице, спаси нас.


Предстaтельницу Тя живота вем и хранительницу тверду, Дево, и напaстей решaщу молвы, и налоги бесов отгоняющу; и молюся всегда, от тли страстей моих избaвити мя.


Слава Отцу и Сыну и Святому Духу.


Яко стену прибежища стяжaхом, и душ всесовершенное спасение, и прострaнство в скорбех, Отроковице, и просвещением Твоим присно рaдуемся: о, Владычице, и ныне нас от страстей и бед спаси.


И ныне и присно и во веки веков. Аминь.


На одре ныне немощствуяй лежу, и несть исцеления плоти моей: но, Бога и Спaса миру и Избaвителя недугов рождшая, Тебе молюся, Благой: от тли недуг возстaви мя.




\bfseries Кондaк, глас 6-й\normalfont{}


Предстaтельство христиан непостыдное, ходaтайство ко Творцу непреложное, не презри грешных молений глaсы, но предвари, яко Благaя, на помощь нас, верно зовущих Ти; ускори на молитву, и потщися на умоление, предстaтельствующи присно, Богородице, чтущих Тя.




\bfseries Другой кондaк, глас тот же\normalfont{}


 Не имамы иныя помощи, не имамы иныя надежды, разве Тебе, Пречистая Дево. Ты нам помози, на Тебе надеемся, и Тобою хвaлимся, Твои бо есмы рабы, да не постыдимся.




\bfseries Стихира, глас тот же\normalfont{}


Не ввери мя человеческому предстaтельству, Пресвятая Владычице, но приими моление раба Твоего: скорбь бо обдержит мя, терпети не могу демонскаго стреляния, покрова не имам, ниже где прибегну, окаянный, всегда побеждaемь, и утешения не имам, разве Тебе, Владычице мира, уповaние и предстaтельство верных, не презри моление мое, полезно сотвори.




\bfseries Песнь 7\normalfont{}


\itshape Ирмо&#x301;с:\normalfont{} От Иудеи дошедше отроцы, в Вавилоне иногдa, верою Троическою плaмень пещный попрaша, поюще: отцев Боже, благословен еси.


\itshape Припев:\normalfont{} Пресвятая Богородице, спаси нас.


Наше спасение якоже восхотел еси, Спaсе, устроити, во утробу Девыя вселился еси, Юже миру предстaтельницу показал еси: отец наших Боже, благословен еси.


\itshape Припев:\normalfont{} Пресвятая Богородице, спаси нас.


Волителя милости, Егоже родилa еси, Мaти чистая, умоли избaвитися от прегрешений и душевных скверн верою зовущим: отец наших Боже, благословен еси.


Слава Отцу и Сыну и Святому Духу.


Сокровище спасения и Источник нетления, Тя рождшую, и столп утверждения, и дверь покаяния, зовущим показал еси: отец наших Боже, благословен еси.


И ныне и присно и во веки веков. Аминь.


Телесныя слабости и душевныя недуги, Богородительнице, любовию приступaющих к крову Твоему, Дево, исцелити сподоби, Спaса Христа нам рождшая.




\bfseries Песнь 8\normalfont{}


\itshape Ирмо&#x301;с:\normalfont{} Царя Небеснаго, Егоже поют вои aнгельстии, хвалите и превозносите во вся веки.


\itshape Припев:\normalfont{} Пресвятая Богородице, спаси нас.


Помощи яже от Тебе требующия не презри, Дево, поющия и превозносящия Тя во веки.


\itshape Припев:\normalfont{} Пресвятая Богородице, спаси нас.


Неможение души моея исцеляеши и телесныя болезни, Дево, да Тя прослaвлю, Чистая, во веки.


Слава Отцу и Сыну и Святому Духу.


Исцелений богатство изливaеши верно поющим Тя, Дево, и превозносящим неизреченное Твое рождество.


И ныне и присно и во веки веков. Аминь.


Напaстей Ты прилоги отгоняеши и страстей находы, Дево: темже Тя поем во вся веки.




\bfseries Песнь 9\normalfont{}


\itshape Ирмо&#x301;с:\normalfont{} Воистинну Богородицу Тя исповедуем, спасеннии Тобою, Дево чистая, с безплотными лики Тя величaюще.


\itshape Припев:\normalfont{} Пресвятая Богородице, спаси нас.


Тока слез моих не отвратися, Яже от всякаго лица всяку слезу отъемшаго, Дево, Христа рождшая.


\itshape Припев:\normalfont{} Пресвятая Богородице, спаси нас.


Радости мое сердце исполни, Дево, Яже радости приемшая исполнение, греховную печаль потребляющи.


\itshape Припев:\normalfont{} Пресвятая Богородице, спаси нас.


Пристaнище и предстaтельство к Тебе прибегaющих буди, Дево, и стена нерушимая, прибежище же и покров и веселие.


Слава Отцу и Сыну и Святому Духу.


Света Твоего зарями просвети, Дево, мрак неведения отгоняющи, благоверно Богородицу Тя исповедающих.


И ныне и присно и во веки веков. Аминь.


На месте озлобления немощи смирившагося, Дево, исцели, из нездрaвия во здрaвие претворяющи.




\bfseries Стихиры, глас 2-й\normalfont{}


Высшую небес и чистшую светлостей солнечных, избaвльшую нас от клятвы, Владычицу мира песньми почтим.


От многих моих грехов немощствует тело, немощствует и душа моя; к Тебе прибегaю, Благодaтней, надеждо ненадежных, Ты ми помози.


Владычице и Мaти Избaвителя, приими моление недостойных раб Твоих, да ходaтайствуеши к Рождшемуся от Тебе; о, Владычице мира, буди Ходaтаица!


Поем прилежно Тебе песнь ныне, всепетой Богородице, рaдостно: со Предтечею и всеми святыми моли, Богородице, еже ущедрити ны.


Вся aнгелов воинства, Предтече Господень, апостолов двоенадесятице, святии вси с Богородицею, сотворите молитву, во еже спастися нам.




\bfseries Молитвы ко Пресвятой Богородице\normalfont{}


Пресвятая Богородице, спаси мя.


Царице моя преблагaя, надеждо моя Богородице, приятелище сирых и странных предстaтельнице, скорбящих рaдосте, обидимых покровительнице! Зриши мою беду, зриши мою скорбь, помози ми яко немощну, окорми мя яко стрaнна. Обиду мою веси, разреши ту, яко волиши: яко не имам иныя помощи разве Тебе, ни иныя предстaтельницы, ни благия утешительницы, токмо Тебе, о Богомaти, яко да сохраниши мя и покрыеши во веки веков. Аминь.


К кому возопию, Владычице? К кому прибегну в горести моей, aще не к Тебе, Царице Небесная? Кто плач мой и воздыхaние мое приимет, aще не Ты, Пренепорочная, надеждо христиан и прибежище нам, грешным? Кто пaче Тебе в напaстех защитит? Услыши убо стенaние мое, и приклони ухо Твое ко мне, Владычице Мaти Бога моего, и не презри мене, требующаго Твоея помощи, и не отрини мене, грешнаго. Вразуми и научи мя, Царице Небесная; не отступи от мене, раба Твоего, Владычице, за роптaние мое, но буди мне Мaти и заступница. Вручaю себе милостивому покрову Твоему: приведи мя, грешнаго, к тихой и безмятежной жизни, да плaчуся о гресех моих. К кому бо прибегну повинный аз, aще не к Тебе, уповaнию и прибежищу грешных, надеждою на неизреченную милость Твою и щедроты Твоя окриляемь? О, Владычице Царице Небесная! Ты мне уповaние и прибежище, покров и заступление и помощь. Царице моя преблагaя и скорая заступнице! Покрый Твоим ходaтайством моя прегрешения, защити мене от враг видимых и невидимых; умягчи сердца злых человек, возстающих на мя. О, Мaти Господа моего Творцa! Ты еси корень девства и неувядaемый цвет чистоты. О, Богородительнице! Ты подaждь ми помощь немощствующему плотскими страстьми и болезнующему сердцем, едино бо Твое и с Тобою Твоего Сына и Бога нашего имам заступление; и Твоим пречудным заступлением да избaвлюся от всякия беды и напaсти, о пренепорочная и преслaвная Божия Мaти Марие. Темже со уповaнием глаголю и вопию: радуйся, благодaтная, радуйся, обрaдованная; радуйся, преблагословенная, Господь с Тобою.


\mychapterending

\mychapter{Канон Ангелу Хранителю}
%/text5.htm



\myfig{img/8.jpg}



\bfseries Тропарь, глас 6-й\normalfont{}


Ангеле Божий, хранителю мой святый, живот мой соблюди во страсе Христа Бога, ум мой утверди во истиннем пути, и к любви горней уязви душу мою, да тобою направляемь, получу от Христа Бога велию милость.


Слава Отцу и Сыну и Святому Духу. И ныне и присно и во веки веков. Аминь.




\bfseries Богородичен\normalfont{}


Святая Владычице, Христа Бога нашего Мати, яко всех Творца недоуменно рождшая, моли благость Его всегда, со хранителем моим ангелом, спасти душу мою, страстьми одержимую, и оставление грехов даровати ми.




\bfseries Канон, глас 8-й\normalfont{}




\bfseries Песнь 1\normalfont{}


\itshape Ирмо&#x301;с:\normalfont{} Поим Господеви, проведшему люди Своя сквозе Чермное море, яко един славно прославися.


\itshape Иисусу:\normalfont{} Господи Иисусе Христе Боже мой, помилуй мя.


Песнь воспети и восхвалити, Спасе, Твоего раба достойно сподоби, безплотному Aнгелу, наставнику и хранителю моему.


\itshape Припев:\normalfont{} Святый Aнгеле Божий, хранителю мой, моли Бога о мне.


Един аз в неразумии и в лености ныне лежу, наставниче мой и хранителю, не остави мене, погибающа.


Слава Отцу и Сыну и Святому Духу.


Ум мой твоею молитвою направи, творити ми Божия повеления, да получу от Бога отдание грехов, и ненавидети ми злых настави мя, молюся ти.


И ныне и присно и во веки веков. Аминь.


Молися, Девице, о мне, рабе Твоем, ко Благодателю, со хранителем моим Aнгелом, и настави мя творити заповеди Сына Твоего и Творца моего.




\bfseries Песнь 3\normalfont{}


\itshape Ирмо&#x301;с:\normalfont{} Ты еси утверждение притекающих к Тебе, Господи, Ты еси свет омраченных, и поет Тя дух мой.


\itshape Припев:\normalfont{} Святый Aнгеле Божий, хранителю мой, моли Бога о мне.


Все помышление мое и душу мою к тебе возложих, хранителю мой; ты от всякия мя напасти вражия избави.


\itshape Припев:\normalfont{} Святый Aнгеле Божий, хранителю мой, моли Бога о мне.


Враг попирает мя, и озлобляет, и поучает всегда творити своя хотения; но ты, наставниче мой, не остави мене погибающа.


Слава Отцу и Сыну и Святому Духу.


Пети песнь со благодарением и усердием Творцу и Богу даждь ми, и тебе, благому Aнгелу хранителю моему: избавителю мой, изми мя от враг озлобляющих мя.


И ныне и присно и во веки веков. Аминь.


Исцели, Пречистая, моя многонедужныя струпы, яже в души, прожени враги, иже присно борются со мною.




\bfseries Седален, глас 2-й\normalfont{}


От любве душевныя вопию ти, хранителю моея души, всесвятый мой Aнгеле: покрый мя и соблюди от лукаваго ловления всегда, и к жизни настави небесней, вразумляя и просвещая и укрепляя мя.


Слава Отцу и Сыну и Святому Духу. И ныне и присно и во веки веков. Аминь.




\bfseries Богородичен:\normalfont{}


Богородице безневестная Пречистая, Яже без семени рождши всех Владыку, Того со Aнгелом хранителем моим моли, избавити ми ся всякаго недоумения, и дати умиление и свет души моей и согрешением очищение, Яже едина вскоре заступающи.




\bfseries Песнь 4\normalfont{}


\itshape Ирмо&#x301;с:\normalfont{} Услышах, Господи, смотрения Твоего таинство, разумех дела Твоя, и прославих Твое Божество.


\itshape Припев:\normalfont{} Святый Aнгеле Божий, хранителю мой, моли Бога о мне.


Моли Человеколюбца Бога ты, хранителю мой, и не остави мене, но присно в мире житие мое соблюди и подаждь ми спасение необоримое.


\itshape Припев:\normalfont{} Святый Aнгеле Божий, хранителю мой, моли Бога о мне.


Яко заступника и хранителя животу моему прием тя от Бога, Aнгеле, молю тя, святый, от всяких мя бед свободи.


Слава Отцу и Сыну и Святому Духу.


Мою скверность твоею святынею очисти, хранителю мой, и от части шуия да отлучен буду молитвами твоими и причастник славы явлюся.


И ныне и присно и во веки веков. Аминь.


Недоумение предлежит ми от обышедших мя зол, Пречистая, но избави мя от них скоро: к Тебе бо единей прибегох.




\bfseries Песнь 5\normalfont{}


\itshape Ирмо&#x301;с:\normalfont{} Утренююще вопием Ти: Господи, спаси ны; Ты бо еси Бог наш, разве Тебе иного не вемы.


\itshape Припев:\normalfont{} Святый Aнгеле Божий, хранителю мой, моли Бога о мне.


Яко имея дерзновение к Богу, хранителю мой святый, Сего умоли от оскорбляющих мя зол избавити.


\itshape Припев:\normalfont{} Святый Aнгеле Божий, хранителю мой, моли Бога о мне.


Свете светлый, светло просвети душу мою, наставниче мой и хранителю, от Бога данный ми Aнгеле.


Слава Отцу и Сыну и Святому Духу.


Спяща мя зле тяготою греховною, яко бдяща сохрани, Aнгеле Божий, и возстави мя на славословие молением твоим.


И ныне и присно и во веки веков. Аминь.


Марие, Госпоже Богородице безневестная, надеждо верных, вражия возношения низложи, поющия же Тя возвесели.




\bfseries Песнь 6\normalfont{}


\itshape Ирмо&#x301;с:\normalfont{} Ризу ми подаждь светлу, одеяйся светом яко ризою, многомилостиве Христе Боже наш.


\itshape Припев:\normalfont{} Святый Aнгеле Божий, хранителю мой, моли Бога о мне.


Всяких мя напастей свободи, и от печалей спаси, молюся ти, святый Aнгеле, данный ми от Бога, хранителю мой добрый.


\itshape Припев:\normalfont{} Святый Aнгеле Божий, хранителю мой, моли Бога о мне.


Освети ум мой, блаже, и просвети мя, молюся ти, святый Aнгеле, и мыслити ми полезная всегда настави мя.


Слава Отцу и Сыну и Святому Духу.


Устави сердце мое от настоящаго мятежа, и бдети укрепи мя во благих, хранителю мой, и настави мя чудно к тишине животней.


И ныне и присно и во веки веков. Аминь.


Слово Божие в Тя вселися, Богородице, и человеком Тя показа небесную лествицу; Тобою бо к нам Вышний сошел есть.




\bfseries Кондак, глас 4-й\normalfont{}


Явися мне милосерд, святый Aнгеле Господень, хранителю мой, и не отлучайся от мене, сквернаго, но просвети мя светом неприкосновенным и сотвори мя достойна Царствия Небеснаго.




\bfseries Икос\normalfont{}


Уничиженную душу мою многими соблазны, ты, святый предстателю, неизреченныя славы небесныя сподоби, и певец с лики безплотных сил Божиих, помилуй мя и сохрани, и помыслы добрыми душу мою просвети, да твоею славою, Aнгеле мой, обогащуся, и низложи зломыслящия мне враги, и сотвори мя достойна Царствия Небеснаго.




\bfseries Песнь 7\normalfont{}


\itshape Ирмо&#x301;с:\normalfont{} От Иудеи дошедше отроцы, в Вавилоне иногда, верою Троическою пламень пещный попраша, поюще: отцев Боже, благословен еси.


\itshape Припев:\normalfont{} Святый Aнгеле Божий, хранителю мой, моли Бога о мне.


Милостив буди ми, и умоли Бога, Господень Aнгеле, имею бо тя заступника во всем животе моем, наставника же и хранителя, от Бога дарованнаго ми во веки.


\itshape Припев:\normalfont{} Святый Aнгеле Божий, хранителю мой, моли Бога о мне.


Не остави в путь шествующия души моея окаянныя убити разбойником, святый Aнгеле, яже ти от Бога предана бысть непорочне; но настави ю на путь покаяния.


Слава Отцу и Сыну и Святому Духу.


Всю посрамлену душу мою привожду от лукавых ми помысл и дел: но предвари, наставниче мой, и исцеление ми подаждь благих помысл, уклоняти ми ся всегда на правыя стези.


И ныне и присно и во веки веков. Аминь.


Премудрости исполни всех и крепости Божественныя, Ипостасная Премудросте Вышняго, Богородицы ради, верою вопиющих: отец наших Боже, благословен еси.




\bfseries Песнь 8\normalfont{}


\itshape Ирмо&#x301;с:\normalfont{} Царя Небеснаго, Егоже поют вои ангельстии, хвалите и превозносите во вся веки.


\itshape Припев:\normalfont{} Святый Aнгеле Божий, хранителю мой, моли Бога о мне.


От Бога посланный, утверди живот мой, раба твоего, преблагий Aнгеле, и не остави мене во веки.


\itshape Припев:\normalfont{} Святый Aнгеле Божий, хранителю мой, моли Бога о мне.


Ангела тя суща блага, души моея наставника и хранителя, преблаженне, воспеваю во веки.


Слава Отцу и Сыну и Святому Духу.


Буди ми покров и забрало в день испытания всех человек, воньже огнем искушаются дела благая же и злая.


И ныне и присно и во веки веков. Аминь.


Буди ми помощница и тишина, Богородице Приснодево, рабу Твоему, и не остави мене лишена быти Твоего владычества.




\bfseries Песнь 9\normalfont{}


\itshape Ирмо&#x301;с:\normalfont{} Воистинну Богородицу Тя исповедуем, спасеннии Тобою, Дево чистая, с безплотными лики Тя величающе.


\itshape Иисусу:\normalfont{} Господи Иисусе Христе Боже мой, помилуй мя.


Помилуй мя, едине Спасе мой, яко милостив еси и милосерд, и праведных ликов сотвори мя причастника.


\itshape Припев:\normalfont{} Святый Aнгеле Божий, хранителю мой, моли Бога о мне.


Мыслити ми присно и творити, Господень Aнгеле, благая и полезная даруй, яко сильна яви в немощи и непорочна.


Слава Отцу и Сыну и Святому Духу.


Яко имея дерзновение к Царю Небесному, Того моли, с прочими безплотными, помиловати мя, окаяннаго.


И ныне и присно и во веки веков. Аминь.


Много дерзновение имущи, Дево, к Воплощшемуся из Тебе, преложи мя от уз и разрешение ми подаждь и спасение, молитвами Твоими.




\bfseries Молитва к Aнгелу Xранителю\normalfont{}


Святый Aнгеле Божий, хранителю мой, моли Бога о мне.


Ангеле Христов святый, к тебе припадая молюся, хранителю мой святый, приданный мне на соблюдение души и телу моему грешному от святаго крещения, аз же своею леностию и своим злым обычаем прогневах твою пречистую светлость и отгнах тя от себе всеми студными делы: лжами, клеветами, завистию, осуждением, презорством, непокорством, братоненавидением, и злопомнением, сребролюбием, прелюбодеянием, яростию, скупостию, объядением без сытости и опивством, многоглаголанием, злыми помыслы и лукавыми, гордым обычаем и блудным возбешением, имый самохотение на всякое плотское вожделение. О, злое мое произволение, егоже и скоти безсловеснии не творят! Да како возможеши воззрети на мя, или приступити ко мне, аки псу смердящему? Которыма очима, ангеле Христов, воззриши на мя, оплетшася зле во гнусных делех? Да како уже возмогу отпущения просити горьким и злым моим и лукавым деянием, в няже впадаю по вся дни и нощи и на всяк час? Но молюся ти припадая, хранителю мой святый, умилосердися на мя грешнаго и недостойнаго раба твоего \itshape (имя)\normalfont{}, буди ми помощник и заступник на злаго моего сопротивника, святыми твоими молитвами, и Царствия Божия причастника мя сотвори со всеми святыми, всегда, и ныне и присно и во веки веков. Аминь.


\mychapterending

\mychapter{Последование ко Святому Причащению}
%/text207.htm



\myfig{img/132.jpg}

Моли́твами святы́х оте́ц на́ших, Го́споди Иису́се Христе́, Бо́же наш,
поми́луй нас. Ами́нь.


   Царю́ Небе́сный, Уте́шителю, Ду́ше и́стины, И́же везде́ сый и вся
исполня́яй, Сокро́вище благи́х и жи́зни Пода́телю, прииди́ и всели́ся в ны, и
очи́сти ны от вся́кия скве́рны, и спаси́, Бла́же, ду́ши на́ша.


   Святы́й Бо́же, Святы́й Кре́пкий, Святы́й Безсме́ртный, поми́луй нас.
\itshape (Tрижды)\normalfont{}


   Сла́ва Отцу́ и Сы́ну и Свято́му Ду́ху, и ны́не и при́сно и во ве́ки веко́в.
Ами́нь.


   Пресвята́я Тро́ице, поми́луй нас; Го́споди, очи́сти грехи́ на́ша; Влады́ко,
прости́ беззако́ния на́ша; Святы́й, посети́ и исцели́ не́мощи на́ша, и́мене
Твоего́ ра́ди.


   Го́споди, поми́луй. \itshape (Трижды)\normalfont{}


   Сла́ва Отцу́ и Сы́ну и Свято́му Ду́ху, и ны́не и при́сно и во ве́ки веко́в.
Ами́нь.


   О́тче наш, И́же еси́ на небесе́х! Да святи́тся и́мя Твое́, да прии́дет
Ца́рствие Твое́, да бу́дет во́ля Твоя́, я́ко на небеси́ и на земли́. Хлеб наш
насу́щный даждь нам днесь; и оста́ви нам до́лги на́ша, я́коже и мы оставля́ем
должнико́м на́шим; и не введи́ нас во искуше́ние, но изба́ви нас от
лука́ваго.


   Го́споди, поми́луй.\itshape (12 раз)\normalfont{}


   Сла́ва Отцу́ и Сы́ну и Свято́му Ду́ху, и ны́не и при́сно и во ве́ки веко́в.
Ами́нь.


   Прииди́те, поклони́мся Царе́ви на́шему Бо́гу. \itshape (Поклон)\normalfont{}


   Прииди́те, поклони́мся и припаде́м Христу́, Царе́ви на́шему Бо́гу.
\itshape (Поклон)\normalfont{}


   Прииди́те, поклони́мся и припаде́м Самому́ Христу́, Царе́ви и Бо́гу
на́шему. \itshape (Поклон)\normalfont{}



 

\bfseries Псалом 22\normalfont{}


   Госпо́дь пасе́т мя, и ничто́же мя лиши́т. На ме́сте зла́чне, та́мо всели́ мя,
на воде́ поко́йне воспита́ мя. Ду́шу мою́ обрати́, наста́ви мя на стези́ пра́вды,
и́мене ра́ди Своего́. А́ще бо и пойду́ посреде́ се́ни сме́ртныя, не убою́ся зла,
я́ко Ты со мно́ю еси́, жезл Твой и па́лица Твоя́, та мя уте́шиста. Угото́вал
еси́ пре́до мно́ю трапе́зу сопроти́в стужа́ющим мне, ума́стил еси́ еле́ом главу́
мою́, и ча́ша Твоя́ упоява́ющи мя, я́ко держа́вна. И ми́лость Твоя́ пожене́т

мя вся дни живота́ моего́, и е́же всели́ти ми ся в дом Госпо́день, в долготу́
дний.



 

\bfseries Псалом 23\normalfont{}


   Госпо́дня земля́, и исполне́ние ея́, вселе́нная, и вси́ живу́щии на ней. Той
на моря́х основа́л ю есть, и на река́х угото́вал ю есть. Кто взы́дет на го́ру
Госпо́дню? Или́ кто ста́нет на ме́сте святе́м Его́? Непови́нен рука́ма и чист
се́рдцем, и́же не прия́т всу́е ду́шу свою́, и не кля́тся ле́стию и́скреннему
своему́. Сей прии́мет благослове́ние от Го́спода, и ми́лостыню от Бо́га, Спа́са
своего́. Сей род и́щущих Го́спода, и́щущих лице́ Бо́га Иа́ковля. Возми́те
врата́, кня́зи ва́ша, и возми́теся врата́ ве́чная; и вни́дет Царь Сла́вы.
Кто есть сей Царь Сла́вы? Госпо́дь кре́пок и си́лен, Госпо́дь си́лен в
бра́ни. Возми́те врата́, кня́зи ва́ша, и возми́теся врата́ ве́чная; и вни́дет
Царь Сла́вы. Кто есть сей Царь Сла́вы? Госпо́дь сил, Той есть Царь
Сла́вы.



 

\bfseries Псалом 115\normalfont{}


   Ве́ровах, те́мже возглаго́лах, аз же смири́хся зело́. Аз же рех во
изступле́нии мое́м: всяк челове́к ложь. Что возда́м Го́сподеви о всех, яже
воздаде́ ми? Ча́шу спасе́ния прииму́, и и́мя Госпо́дне призову́, моли́твы моя́
Го́сподеви возда́м пред все́ми людьми́ Его́. Честна́ пред Го́сподем смерть
преподо́бных Его́. О, Го́споди, аз раб Твой, аз раб Твой и сын рабы́ни Твоея́;
растерза́л еси́ у́зы моя́. Тебе́ пожру́ же́ртву хвалы́, и во и́мя Госпо́дне
призову́. Моли́твы моя́ Го́сподеви возда́м пред все́ми людьми́ Его́, во дво́рех
до́му Госпо́дня, посреде́ тебе́, Иерусали́ме.


   Сла́ва Отцу́ и Сы́ну и Свято́му Ду́ху, и ны́не и при́сно и во ве́ки веко́в.
Ами́нь.


   Аллилу́ия. \itshape (Трижды с тремя поклонами)\normalfont{}



 

\bfseries Тропари, глас 8-й\normalfont{}


   Беззако́ния моя́ пре́зри, Го́споди, от Де́вы рожде́йся, и се́рдце мое́ очи́сти,
храм то творя́ пречи́стому Твоему́ Те́лу и Кро́ви, ниже́ отри́ни мене́ от
Твоего́ лица́, без числа́ име́яй ве́лию ми́лость.


   Сла́ва Отцу́ и Сы́ну и Свято́му Ду́ху.


   Во прича́стие святы́нь Твои́х ка́ко дерзну́ [вни́ду], недосто́йный? А́ще бо
дерзну́ к Тебе́ приступи́ти с досто́йными, хито́н мя облича́ет, я́ко несть
вече́рний, и осужде́ние исхода́тайствую многогре́шной души́ мое́й. Очи́сти,
Го́споди, скве́рну души́ моея́, и спаси́ мя, я́ко Человеколю́бец.


   И ны́не и при́сно и во ве́ки веко́в. Ами́нь.


   Мно́гая мно́жества мои́х, Богоро́дице, прегреше́ний, к Тебе́ прибего́х,
Чи́стая, спасе́ния тре́буя: посети́ немощству́ющую мою́ ду́шу, и моли́ Сы́на
Твоего́ и Бо́га на́шего, да́ти ми оставле́ние, я́же соде́ях лю́тых, Еди́на
благослове́нная.



 

\bfseries [Во Святу́ю же Четыредеся́тницу:\normalfont{}


   Егда́ сла́внии ученицы́ на умове́нии ве́чери просвеща́хуся, тогда́ Иу́да
злочести́вый сребролю́бием неду́говав омрача́шеся, и беззако́нным судия́м
Тебе́ пра́веднаго Судию́ предае́т. Виждь, име́ний рачи́телю, сих ра́ди
удавле́ние употреби́вша: бежи́ несы́тыя души́, Учи́телю такова́я дерзну́вшия.
И́же о всех благи́й Го́споди, сла́ва Тебе́.]



 

\bfseries Псалом 50\normalfont{}


   Поми́луй мя, Бо́же, по вели́цей ми́лости Твое́й, и по мно́жеству щедро́т
Твои́х очи́сти беззако́ние мое́. Наипа́че омы́й мя от беззако́ния моего́, и от
греха́ моего́ очи́сти мя; яко беззако́ние мое́ аз зна́ю, и грех мой пре́до мно́ю
есть вы́ну. Тебе́ Еди́ному согреши́х и лука́вое пред Тобо́ю сотвори́х, я́ко да
оправди́шися во словесе́х Твои́х, и победи́ши внегда́ суди́ти Ти. Се бо, в
беззако́ниих зача́т есмь, и во гресе́х роди́ мя ма́ти моя́. Се бо, и́стину
возлюби́л еси́; безве́стная и та́йная прему́дрости Твоея́ яви́л ми еси́.
Окропи́ши мя иссо́пом, и очи́щуся; омы́еши мя, и па́че сне́га убелю́ся. Слу́ху
моему́ да́си ра́дость и весе́лие; возра́дуются ко́сти смире́нныя. Отврати́ лице́
Твое́ от грех мои́х и вся беззако́ния моя́ очи́сти. Се́рдце чи́сто сози́жди во
мне, Бо́же, и дух прав обнови́ во утро́бе мое́й. Не отве́ржи мене́ от

лица́ Твоего́ и Ду́ха Твоего́ Свята́го не отыми́ от мене́. Возда́ждь ми
ра́дость спасе́ния Твое́го и Ду́хом влады́чним утверди́ мя. Научу́
беззако́ныя путе́м Твои́м, и нечести́вии к Тебе́ обратя́тся. Изба́ви мя от
крове́й, Бо́же, Бо́же спасе́ния моего́; возра́дуется язы́к мой пра́вде
Твое́й. Го́споди, устне́ мои отве́рзеши, и уста́ моя́ возвестя́т хвалу́
Твою́. Я́ко а́ще бы восхоте́л еси́ же́ртвы, дал бых у́бо: всесожже́ния не
благоволи́ши. Же́ртва Бо́гу дух сокруше́н; се́рдце сокруше́нно и смире́нно
Бог не уничижи́т. Ублажи́, Го́споди, благоволе́нием Твои́м Сио́на, и
да сози́ждутся сте́ны Иерусали́мския. Тогда́ благоволи́ши же́ртву
пра́вды, возноше́ние и всесожега́емая; тогда́ возложа́т на oлта́рь Твой
тельцы́.



 

\bfseries Канон, глас 2-й\normalfont{}


 

\bfseries Песнь 1\normalfont{}


 \itshape Ирмо́с:\normalfont{} Гряди́те лю́дие, пои́м песнь Христу́ Бо́гу, разде́льшему мо́ре, и
наста́вльшему лю́ди, я́же изведе́ из рабо́ты еги́петския, я́ко просла́вися.


 \itshape Припев:\normalfont{} Се́рдце чи́сто сози́жди во мне, Бо́же, и дух прав обнови́ во утро́бе
мое́й.


   Хлеб живота́ ве́чнующаго да бу́дет ми Те́ло Твое́ Свято́е, благоутро́бне
Го́споди, и Честна́я Кровь, и неду́г многообра́зных исцеле́ние.


 \itshape Припев:\normalfont{} Не отве́ржи мене́ от лица́ Твоего́, и Ду́ха Твоего́ Свята́го не отыми́ от
мене́.


   Оскверне́н де́лы безме́стными окая́нный, Твоего́ Пречи́стаго Те́ла и
Боже́ственныя Кро́ве недосто́ин есмь, Христе́, причаще́ния, его́же мя
сподо́би.


 \itshape Припев:\normalfont{} Пресвята́я Богоро́дице, спаси́ нас.


 \itshape Богородичен:\normalfont{} Земле́ блага́я, благослове́нная Богоневе́сто, клас
прозя́бшая неора́нный и спаси́тельный ми́ру, сподо́би мя сей яду́ща
спасти́ся.



 

\bfseries Песнь 3\normalfont{}


 \itshape Ирмо́с:\normalfont{} На камени мя веры утвердив, разширил еси уста моя на враги моя.
Возвесели бо ся дух мой, внегда пети: несть свят, якоже Бог наш, и несть праведен
паче Тебе, Господи.


 \itshape Припев:\normalfont{} Се́рдце чи́сто сози́жди во мне, Бо́же, и дух прав обнови́ во утро́бе
мое́й.


   Сле́зныя ми пода́ждь, Христе́, ка́пли, скве́рну се́рдца моего́ очища́ющия:
я́ко да благо́ю со́вестию очище́н, ве́рою прихожду́ и стра́хом, Влады́ко, ко
причаще́нию Боже́ственных Даро́в Твои́х.


 \itshape Припев:\normalfont{} Не отве́ржи мене́ от лица́ Твоего́, и Ду́ха Твоего́ Свята́го не отыми́ от
мене́.


   Во оставле́ние да бу́дет ми прегреше́ний Пречи́стое Те́ло Твое́, и
Боже́ственная Кровь, Ду́ха же Свята́го обще́ние, и в жизнь ве́чную,
Человеколю́бче, и страсте́й и скорбе́й отчужде́ние.


 \itshape Припев:\normalfont{} Пресвята́я Богоро́дице, спаси́ нас.


 \itshape Богородичен:\normalfont{} Хле́ба живо́тнаго Tрапе́за Пресвята́я, свы́ше ми́лости ра́ди
сше́дшаго, и ми́рови но́вый живо́т даю́щаго, и мене́ ны́не сподо́би
недосто́йнаго, со стра́хом вкуси́ти сего́, и жи́ву бы́ти.



 

\bfseries Песнь 4\normalfont{}


 \itshape Ирмо́с:\normalfont{} Прише́л еси́ от Де́вы, не хода́тай, ни А́нгел, но Сам, Го́споди,
воплощься, и спасл еси́ всего́ мя челове́ка. Тем зову́ Ти: сла́ва си́ле Твое́й,
Го́споди.


 \itshape Припев:\normalfont{} Се́рдце чи́сто сози́жди во мне, Бо́же, и дух прав обнови́ во утро́бе
мое́й.


   Восхоте́л еси́, нас ра́ди вопло́щся, Многоми́лостиве, за́клан бы́ти
я́ко овча́, грех ра́ди челове́ческих: те́мже молю́ Тя, и моя́ очи́сти
согреше́ния.


 \itshape Припев:\normalfont{} Не отве́ржи мене́ от лица́ Твоего́, и Ду́ха Твоего́ Свята́го не отыми́ от
мене́.


   Исцели́ души́ моея́ я́звы, Го́споди, и всего́ освяти́: и сподо́би, Влады́ко,
я́ко да причащу́ся та́йныя Твоея́ Боже́ственныя ве́чери, окая́нный.


 \itshape Припев:\normalfont{} Пресвята́я Богоро́дице, спаси́ нас.


 \itshape Богородичен:\normalfont{} Уми́лостиви и мне Су́щаго от утро́бы Твоея́, Влады́чице, и
соблюди́ мя нескве́рна раба́ Твое́го и непоро́чна, я́ко да прие́м у́мнаго би́сера,

освящу́ся.



 

\bfseries Песнь 5\normalfont{}


 \itshape Ирмо́с:\normalfont{} Све́та Пода́телю и веко́в Тво́рче, Го́споди, во све́те Твои́х повеле́ний
наста́ви нас; ра́зве бо Тебе́ ино́го бо́га не зна́ем.


 \itshape Припев:\normalfont{} Се́рдце чи́сто сози́жди во мне, Бо́же, и дух прав обнови́ во утро́бе
мое́й.


   Я́коже предре́кл еси́, Христе́, да бу́дет у́бо худо́му рабу́ Твоему́, и во мне
пребу́ди, я́коже обеща́лся еси́: се бо Те́ло Твое́ ям Боже́ственное, и пию́
Кровь Твою́.


 \itshape Припев:\normalfont{} Не отве́ржи мене́ от лица́ Твоего́, и Ду́ха Твоего́ Свята́го не отыми́ от
мене́.


   Сло́ве Бо́жий и Бо́же, угль Те́ла Твое́го да бу́дет мне помраче́нному в
просвеще́ние, и очище́ние оскверне́нной души́ мое́й Кровь Твоя́.


 \itshape Припев:\normalfont{} Пресвята́я Богоро́дице, спаси́ нас.


 \itshape Богородичен:\normalfont{} Мари́е, Ма́ти Бо́жия, благоуха́ния честно́е селе́ние, Твои́ми
моли́твами сосу́д мя избра́нный соде́лай, я́ко да освяще́ний причащу́ся Сы́на
Твое́го.



 

\bfseries Песнь 6\normalfont{}


 \itshape Ирмо́с:\normalfont{} В бе́здне грехо́вней валя́яся, неизсле́дную милосе́рдия Твое́го призыва́ю
бе́здну: от тли, Бо́же, мя возведи́.


 \itshape Припев:\normalfont{} Се́рдце чи́сто сози́жди во мне, Бо́же, и дух прав обнови́ во утро́бе
мое́й.


   Ум, ду́шу и се́рдце освяти́, Спа́се, и те́ло мое́, и сподо́би неосужде́нно,
Влады́ко, к стра́шным Та́йнам приступи́ти.


 \itshape Припев:\normalfont{} Не отве́ржи мене́ от лица́ Твоего́, и Ду́ха Твоего́ Свята́го не отыми́ от
мене́.


   Да бых устрани́лся от страсте́й, и Твоея́ благода́ти име́л бы приложе́ние,
живота́ же утвержде́ние, причаще́нием Святы́х, Христе́, Та́ин Твои́х.


 \itshape Припев:\normalfont{} Пресвята́я Богоро́дице, спаси́ нас.


 \itshape Богородичен:\normalfont{} Бо́жие, Бо́же, Сло́во Свято́е, всего́ мя освяти́, ны́не
приходя́щаго к Боже́ственным Твои́м Та́йнам, Святы́я Ма́тере Твоея́

мольба́ми.



 

\bfseries Кондак, глас 2-й\normalfont{}


   Хлеб, Христе́, взя́ти не пре́зри мя, Те́ло Твое́, и Боже́ственную Твою́
ны́не Кровь, пречи́стых, Влады́ко, и стра́шных Твои́х Та́ин причасти́тися
окая́ннаго, да не бу́дет ми в суд, да бу́дет же ми в живо́т ве́чный и
безсме́ртный.



 

\bfseries Песнь 7\normalfont{}


 \itshape Ирмо́с:\normalfont{} Те́лу злато́му прему́дрыя де́ти не послужи́ша, и в пла́мень са́ми
поидо́ша, и бо́ги их обруга́ша, среди́ пла́мене́ возопи́ша, и ороси́ я А́нгел: услы́шася
уже́ уст ва́ших моли́тва.


 \itshape Припев:\normalfont{} Се́рдце чи́сто сози́жди во мне, Бо́же, и дух прав обнови́ во утро́бе
мое́й.


   Исто́чник благи́х, причаще́ние, Христе́, безсме́ртных Твои́х ны́не Та́инств
да бу́дет ми свет, и живо́т, и безстра́стие, и к преспея́нию же и умноже́нию
доброде́тели Боже́ственнейшия хода́тайственно, еди́не Бла́же, я́ко да сла́влю
Тя.


 \itshape Припев:\normalfont{} Не отве́ржи мене́ от лица́ Твоего́, и Ду́ха Твоего́ Свята́го не отыми́ от
мене́.


   Да изба́влюся от страсте́й, и враго́в, и ну́жды, и вся́кия ско́рби, тре́петом
и любо́вию со благогове́нием, Человеколю́бче, приступа́яй ны́не к Твои́м
безсме́ртным и Боже́ственным Та́йнам, и пе́ти Тебе́ сподо́би: благослове́н еси́,
Го́споди, Бо́же оте́ц на́ших.


 \itshape Припев:\normalfont{} Пресвята́я Богоро́дице, спаси́ нас.


 \itshape Богородичен:\normalfont{} Спа́са Христа́ ро́ждшая па́че ума́, Богоблагода́тная,
молю́ Тя ны́не, раб Твой, Чи́стую нечи́стый: хотя́щаго мя ны́не к
пречи́стым Та́йнам приступи́ти, очи́сти всего́ от скве́рны пло́ти и
ду́ха.



 

\bfseries Песнь 8\normalfont{}


 \itshape Ирмо́с:\normalfont{} В пещь о́гненную ко отроко́м евре́йским снизше́дшаго, и пла́мень в
ро́су прело́жшаго Бо́га, по́йте дела́ я́ко Го́спода, и превозноси́те во вся
ве́ки.


 \itshape Припев:\normalfont{} Се́рдце чи́сто сози́жди во мне, Бо́же, и дух прав обнови́ во утро́бе
мое́й.


   Небе́сных, и стра́шных, и святы́х Твои́х, Христе́, ны́не Та́ин, и
Боже́ственныя Твоея́ и та́йныя ве́чери о́бщника бы́ти и мене́ сподо́би
отча́яннаго, Бо́же, Спа́се мой.


 \itshape Припев:\normalfont{} Не отве́ржи мене́ от лица́ Твоего́, и Ду́ха Твоего́ Свята́го не отыми́ от
мене́.


   Под Твое́ прибе́г благоутро́бие, Бла́же, со стра́хом зову́ Ти: во мне
пребу́ди, Спа́се, и аз, я́коже рекл еси́, в Тебе́; се бо дерза́я на ми́лость Твою́,
ям Те́ло Твое́, и пию́ Кровь Твою́.


 \itshape Припев:\normalfont{} Пресвята́я Тро́ице, Бо́же наш, сла́ва Тебе́.


 \itshape Тро́ичен:\normalfont{} Трепе́щу, прие́мля огнь, да не опалю́ся я́ко воск и я́ко трава́; о́ле
стра́шнаго та́инства! о́ле благоутро́бия Бо́жия! Ка́ко Боже́ственнаго Те́ла и
Кро́ве бре́ние причаща́юся, и нетле́нен сотворя́юся?



 

\bfseries Песнь 9\normalfont{}


 \itshape Ирмо́с:\normalfont{} Безнача́льна Роди́теля Сын, Бог и Госпо́дь, вопло́щся от Де́вы нам
яви́ся, омраче́нная просвети́ти, собра́ти расточе́нная: тем всепе́тую Богоро́дицу
велича́ем.


 \itshape Припев:\normalfont{} Се́рдце чи́сто сози́жди во мне, Бо́же, и дух прав обнови́ во утро́бе
мое́й.


   Христо́с е́сть, вкуси́те и ви́дите: Госпо́дь нас ра́ди, по нам бо дре́вле
бы́вый, еди́ною Себе́ прине́с, я́ко приноше́ние Отцу́ Своему́, при́сно
закала́ется, освяща́яй причаща́ющияся.


 \itshape Припев:\normalfont{} Не отве́ржи мене́ от лица́ Твоего́, и Ду́ха Твоего́ Свята́го не отыми́ от
мене́.


   Душе́ю и те́лом да освящу́ся, Влады́ко, да просвещу́ся, да спасу́ся, да
бу́ду дом Твой причаще́нием свяще́нных Та́ин, живу́щаго Тя име́я в себе́ со
Отце́м и Ду́хом, Благоде́телю Многоми́лостиве.


 \itshape Припев:\normalfont{} Возда́ждь ми ра́дость спасе́ния Твое́го и Ду́хом Владычним утверди́
мя.


   Я́коже огнь да бу́дет ми, и я́ко свет, Те́ло Твое́ и Кровь, Спа́се мой,

пречестна́я, опаля́я грехо́вное вещество́, сжига́я же страсте́й те́рние, и всего́
мя просвеща́я, покланя́тися Божеству́ Твоему́.


 \itshape Припев:\normalfont{} Пресвята́я Богоро́дице, спаси́ нас.


 \itshape Богородичен:\normalfont{} Бог воплоти́ся от чи́стых крове́й Твои́х; те́мже вся́кий род
пое́т Тя, Влады́чице, у́мная же мно́жества сла́вят, я́ко Тобо́ю я́ве узре́ша
все́ми Влады́чествующаго, осуществова́вшагося челове́чеством.



 

\bfseries Далее\normalfont{}


   Досто́йно е́сть я́ко вои́стинну блажи́ти Тя, Богоро́дицу, Присноблаже́нную и
Пренепоро́чную и Ма́терь Бо́га на́шего. Честне́йшую Херуви́м и сла́внейшую
без сравне́ния Серафи́м, без истле́ния Бо́га Сло́ва ро́ждшую, су́щую
Богоро́дицу Тя велича́ем.


   Святы́й Бо́же, Святы́й Кре́пкий, Святы́й Безсме́ртный, поми́луй нас.
\itshape (Tрижды)\normalfont{}


   Сла́ва Отцу́ и Сы́ну и Свято́му Ду́ху, и ны́не и при́сно и во ве́ки веко́в.
Ами́нь.


   Пресвята́я Тро́ице, поми́луй нас; Го́споди, очи́сти грехи́ на́ша; Влады́ко,
прости́ беззако́ния на́ша; Святы́й, посети́ и исцели́ не́мощи на́ша, и́мене
Твоего́ ра́ди.


   Го́споди, поми́луй. \itshape (Трижды)\normalfont{}


   Сла́ва Отцу́ и Сы́ну и Свято́му Ду́ху, и ны́не и при́сно и во ве́ки веко́в.
Ами́нь.


   О́тче наш, И́же еси́ на небесе́х! Да святи́тся и́мя Твое́, да прии́дет
Ца́рствие Твое́, да бу́дет во́ля Твоя́, я́ко на небеси́ и на земли́. Хлеб наш
насу́щный даждь нам днесь; и оста́ви нам до́лги на́ша, я́коже и мы оставля́ем
должнико́м на́шим; и не введи́ нас во искуше́ние, но изба́ви нас от
лука́ваго.


 \itshape Если неделя, тропарь воскресный по гласу. Если же нет, настоящие
тропари, глас 6-й:\normalfont{}


   Поми́луй нас, Го́споди, поми́луй нас; вся́каго бо отве́та недоуме́юще, сию́
Ти моли́тву, я́ко Влады́це, гре́шнии прино́сим: поми́луй нас.


   Сла́ва Отцу́ и Сы́ну и Свято́му Ду́ху.


   Го́споди, поми́луй нас, на Тя бо упова́хом; не прогне́вайся на ны зело́,
ниже́ помяни́ беззако́ний на́ших, но при́зри и ны́не я́ко благоутро́бен, и
изба́ви ны от враг на́ших. Ты бо еси́ Бог наш, и мы лю́дие Твои́, вси́ дела́
руку́ Твое́ю, и и́мя Твое́ призыва́ем.



   И ны́не и при́сно и во ве́ки веко́в. Ами́нь.


   Милосе́рдия две́ри отве́рзи нам, благослове́нная Богоро́дице, наде́ющиися
на Тя да не поги́бнем, но да изба́вимся Тобо́ю от бед: Ты бо еси́ спасе́ние
ро́да христиа́нскаго.


   Го́споди, поми́луй. \itshape (40 раз) И поклоны, сколько хочешь.\normalfont{}


 \itshape И стихи:\normalfont{}



 

\bfseries Молитва 1-я, Василия Великого\normalfont{}


   Влады́ко Го́споди Иису́се Христе́, Бо́же наш, Исто́чниче жи́зни и
безсме́ртия, всея́ тва́ри ви́димыя и неви́димыя Соде́телю, безнача́льнаго Отца́
соприсносу́щный Сы́не и собезнача́льный, премно́гия ра́ди бла́гости в
после́дния дни в плоть оболки́йся, и распны́йся, и погребы́йся за ны
неблагода́рныя и злонра́вныя, и Твое́ю Кро́вию обнови́вый растле́вшее
грехо́м естество́ на́ше, Сам, Безсме́ртный Царю́, приими́ и мое́ гре́шнаго
покая́ние, и приклони́ у́хо Твое́ мне, и услы́ши глаго́лы моя́. Согреши́х бо,
Го́споди, согреши́х на не́бо и пред Тобо́ю, и несмь досто́ин воззре́ти на
высоту́ сла́вы Твоея́: прогне́вах бо Твою́ бла́гость, Твоя́ за́поведи
преступи́в, и не послу́шав Твои́х повеле́ний. Но Ты, Го́споди, незло́бив сый,
долготерпели́в же и многоми́лостив, не пре́дал еси́ мя поги́бнути со
беззако́ньми мои́ми, моего́ вся́чески ожида́я обраще́ния. Ты бо рекл еси́,
Человеколю́бче, проро́ком Твои́м: я́ко хоте́нием не хощу́ сме́рти гре́шника, но

е́же обрати́тся и жи́ву бы́ти ему́. Не хо́щеши бо, Влады́ко, созда́ния Твое́ю
руку́ погуби́ти, ниже́ благоволи́ши о поги́бели челове́честей, но хо́щеши всем
спасти́ся, и в ра́зум и́стины приити́. Те́мже и аз, а́ще и недосто́ин есмь небесе́
и земли́, и сея́ привре́менныя жи́зни, всего́ себе́ повину́в греху́, и сласте́м
порабо́тив, и Твой оскверни́в о́браз; но творе́ние и созда́ние Твое́ быв, не
отчаява́ю своего́ спасе́ния окая́нный, на Твое́ же безме́рное благоутро́бие
дерза́я, прихожду́. Приими́ у́бо и мене́, Человеколю́бче Го́споди, я́коже
блудни́цу, я́ко разбо́йника, я́ко мытаря́ и я́ко блу́днаго, и возми́ мое́ тя́жкое
бре́мя грехо́в, грех взе́мляй ми́ра, и не́мощи челове́ческия исцеля́яй,
тружда́ющияся и обремене́нныя к Себе́ призыва́яй и упокоева́яй, не
прише́дый призва́ти пра́ведныя, но гре́шныя на покая́ние. И очи́сти мя
от вся́кия скве́рны пло́ти и ду́ха, и научи́ мя соверша́ти святы́ню
во стра́се Твое́м: я́ко да чи́стым све́дением со́вести моея́, святы́нь
Твои́х часть прие́мля, соединю́ся свято́му Те́лу Твоему́ и Кро́ви, и
име́ю Тебе́ во мне живу́ща и пребыва́юща, со Отце́м, и Святы́м Твои́м
Ду́хом. Ей, Го́споди Иису́се Христе́, Бо́же мой, и да не в суд ми бу́дет
прича́стие пречи́стых и животворя́щих Та́ин Твои́х, ниже́ да не́мощен бу́ду
душе́ю же и те́лом, от е́же недо́стойне тем причаща́тися, но даждь ми,
да́же до коне́чнаго моего́ издыха́ния, неосужде́нно восприима́ти часть
святы́нь Твои́х, в Ду́ха Свята́го обще́ние, в напу́тие живота́ ве́чнаго, и
во благоприя́тен отве́т на Стра́шнем суди́щи Твое́м: я́ко да и аз со
все́ми избра́нными Твои́ми о́бщник бу́ду нетле́нных Твои́х благ, я́же
угото́вал еси́ лю́бящим Тя, Го́споди, в ни́хже препросла́влен еси́ во ве́ки.
Ами́нь.



 

\bfseries Молитва 2-я, святого Иоанна Златоустого\normalfont{}


   Го́споди Бо́же мой, вем, я́ко несмь досто́ин, ниже́ дово́лен, да под кров
вни́деши хра́ма души́ моея́, зане́же весь пуст и па́лся е́сть, и не и́маши во мне
ме́ста досто́йна е́же главу́ подклони́ти: но я́коже с высоты́ нас ра́ди смири́л
еси́ Себе́, смири́ся и ны́не смире́нию моему́; и я́коже восприя́л еси́ в верте́пе и
в я́слех безслове́сных возлещи́, си́це восприими́ и в я́слех безслове́сныя моея́
души́, и во оскверне́нное мое́ те́ло вни́ти. И я́коже не неудосто́ил еси́ вни́ти, и
свечеря́ти со гре́шники в до́му Си́мона прокаже́ннаго, та́ко изво́ли вни́ти
и в дом смире́нныя моея́ души́, прокаже́нныя и гре́шныя; и я́коже
не отри́нул еси́ подо́бную мне блудни́цу и гре́шную, прише́дшую и
прикосну́вшуюся Тебе́, си́це умилосе́рдися и о мне гре́шнем, приходя́щем и

прикаса́ющем Ти ся; и я́коже не возгнуша́лся еси́ скве́рных ея́ уст и
нечи́стых, целу́ющих Тя, ниже́ мои́х возгнуша́йся скве́рнших о́ныя уст
и нечи́стших, ниже́ ме́рзких мои́х и нечи́стых усте́н, и скве́рнаго и
нечи́стейшаго моего́ язы́ка. Но да бу́дет ми угль пресвята́го Твое́го Те́ла, и
честны́я Твоея́ Кро́ве, во освяще́ние и просвеще́ние и здра́вие смире́нней мое́й
души́ и те́лу, во облегче́ние тя́жестей мно́гих мои́х согреше́ний, в
соблюде́ние от вся́каго диа́вольскаго де́йства, во отгна́ние и возбране́ние
зла́го моего́ и лука́ваго обы́чая, во умерщвле́ние страсте́й, в снабде́ние
за́поведей Твои́х, в приложе́ние Боже́ственныя Твоея́ благода́ти, и Твое́го
Ца́рствия присвое́ние. Не бо я́ко презира́яй прихожду́ к Тебе́, Христе́
Бо́же, но я́ко дерза́я на неизрече́нную Твою́ бла́гость, и да не на мно́зе
удаля́яйся обще́ния Твое́го, от мы́сленнаго во́лка звероуловле́н бу́ду.
Те́мже молю́ся Тебе́: я́ко еди́н сый Свят, Влады́ко, освяти́ мою́ ду́шу и
те́ло, ум и се́рдце, чревеса́ и утро́бы, и всего́ мя обнови́, и вкорени́
страх Твой во удесе́х мои́х, и освяще́ние Твое́ неотъе́млемо от мене́
сотвори́; и бу́ди ми помо́щник и засту́пник, окормля́я в ми́ре живо́т
мой, сподобля́я мя и одесну́ю Тебе́ предстоя́ния со святы́ми Твои́ми,
моли́твами и моле́ньми Пречи́стыя Твоея́ Ма́тере, невеще́ственных Твои́х
служи́телей и пречи́стых сил, и всех святы́х, от ве́ка Тебе́ благоугоди́вших.
Ами́нь.



 

\bfseries Молитва 3-я, Симеона Метафраста\normalfont{}


   Еди́не чи́стый и нетле́нный Го́споди, за неизрече́нную ми́лость
человеколю́бия на́ше все восприе́мый смеше́ние, от чи́стых и де́вственных
крове́й па́че естества́ ро́ждшия Тя, Ду́ха Боже́ственнаго наше́ствием, и
благоволе́нием Отца́ присносу́щнаго, Христе́ Иису́се, прему́дросте Бо́жия, и
ми́ре, и си́ло; Твои́м восприя́тием животворя́щая и спаси́тельная страда́ния
восприе́мый, крест, гво́здия, копие́, смерть, умертви́ моя́ душетле́нныя
стра́сти теле́сныя. Погребе́нием Твои́м а́дова плени́вый ца́рствия, погреби́
моя́ благи́ми по́мыслы лука́вая сове́тования, и лука́вствия ду́хи разори́.
Тридне́вным Твои́м и живоно́сным воскресе́нием па́дшаго пра́отца
возста́вивый, возста́ви мя грехо́м попо́лзшагося, о́бразы мне покая́ния
предлага́я. Пресла́вным Твои́м вознесе́нием плотско́е обожи́вый восприя́тие,
и сие́ десны́м Отца́ седе́нием почты́й, сподо́би мя прича́стием святы́х
Твои́х Та́ин десну́ю часть спаса́емых получи́ти. Сни́тием Уте́шителя
Твое́го Ду́ха сосу́ды че́стны свяще́нныя Твоя́ ученики́ соде́лавый,

прия́телище и мене́ покажи́ Того́ прише́ствия. Хотя́й па́ки прийти суди́ти
вселе́нней пра́вдою, благоволи́ и мне усре́сти Тя на о́блацех, Судию́ и
Созда́теля моего́, со все́ми святы́ми Твои́ми: да безконе́чно славосло́влю и
воспева́ю Тя, со безнача́льным Твои́м Отце́м, и Пресвяты́м и Благи́м
и Животворя́щим Твои́м Ду́хом, ны́не и при́сно, и во ве́ки веко́в.
Ами́нь.



 

\bfseries Молитва 4-я, его же\normalfont{}  Я́ко на Стра́шнем Твое́м и нелицеприе́мнем
предстоя́й Суди́лищи, Христе́ Бо́же, и осужде́ния подъе́мля, и сло́во творя́ о
соде́янных мно́ю злых; си́це днесь, пре́жде да́же не приити́ дне́ви осужде́ния
моего́, у свята́го Твоего́ Же́ртвенника предстоя́ пред Тобо́ю и пред
стра́шными и святы́ми А́нгелы Твои́ми, преклоне́н от своея́ со́вести, приношу́
лука́вая моя́ и беззако́нная дея́ния, явля́яй сия́ и облича́яй. Виждь, Го́споди,
смире́ние мое́, и оста́ви вся грехи́ моя́; виждь, я́ко умно́жишася па́че влас
главы́ моея́ беззако́ния моя́. Ко́е у́бо не соде́ях зло? Кий грех не сотвори́х?
Ко́е зло не вообрази́х в души́ мое́й? Уже́ бо и де́лы соде́ях: блуд,
прелюбоде́йство, го́рдость, киче́ние, укоре́ние, хулу́, праздносло́вие, смех
неподо́бный, пия́нство, гортанобе́сие, объяде́ние, не́нависть, за́висть,
сребролю́бие, любостяжа́ние, лихои́мство, самолю́бие, славолю́бие, хище́ние,
непра́вду, злоприобре́тение, ре́вность, оклевета́ние, беззако́ние; вся́кое мое́
чу́вство и вся́кий уд оскверни́х, растли́х, непотре́бен сотвори́х, де́лателище
быв вся́чески диа́воле. И ве́м, Го́споди, я́ко беззако́ния моя́ превзыдо́ша
главу́ мою; но безме́рно есть мно́жество щедро́т Твои́х, и ми́лость
неизрече́нна незло́бивыя Твоея́ бла́гости, и несть грех побежда́ющ
человеколю́бие Твое́. Те́мже, пречу́дный Царю́, незло́биве Го́споди, удиви́ и
на мне, гре́шнем, ми́лости Твоя́, покажи́ бла́гости Твоея́ си́лу и яви́
кре́пость благоутро́бнаго милосе́рдия Твоего́, и обраща́ющася приими́
мя гре́шнаго. Приими́ мя, я́коже прия́л еси́ блу́днаго, разбо́йника,
блудни́цу. Приими́ мя, пребезме́рне и сло́вом, и де́лом, и по́хотию
безме́стною, и помышле́нием безслове́сным согреши́вша Тебе́. И я́коже во
единонадеся́тый час прише́дших прия́л еси́, ничто́же досто́йно соде́лавших,
та́ко приими́ и мене́, гре́шнаго: мно́го бо согреши́х и оскверни́хся, и
опеча́лих Ду́ха Твоего́ Свята́го, и огорчи́х человеколю́бную утро́бу Твою́
и де́лом, и сло́вом, и помышле́нием, в нощи́ и во дни, явле́нне же
и неявле́нне, во́лею же и нево́лею. И вем, я́ко предста́виши грехи́
моя́ пре́до мно́ю таковы́, я́ковы же мно́ю соде́яшася, и истя́жеши

сло́во со мно́ю о и́хже ра́зумом непроще́нно согреши́х. Но Го́споди,
Го́споди, да не пра́ведным судо́м Твои́м, ниже́ я́ростию Твое́ю обличи́ши
мя, ниже́ гне́вом Твои́м нака́жеши мя; поми́луй мя, Го́споди, я́ко
не то́кмо не́мощен е́смь, но и Твое́ есмь созда́ние. Ты у́бо, Го́споди,
утверди́л еси́ на мне страх Твой, аз же лука́вое пред Тобо́ю сотвори́х.
Тебе́ у́бо еди́ному согреши́х, но молю́ Тя, не вни́ди в суд с рабо́м
Твои́м. Аще бо беззако́ния на́зриши, Го́споди, Го́споди, кто постои́т?
Аз бо есмь пучи́на греха́, и несмь досто́ин, ниже́ дово́лен воззре́ти
и ви́дети высоту́ небе́сную, от мно́жества грехо́в мои́х, и́хже несть
числа́: вся́кое бо злодея́ние и кова́рство, и ухищре́ние сатанино́, и
растле́ния, злопомне́ния, сове́тования ко греху́ и ины́е тьмочи́сленныя
стра́сти не оскуде́ша от мене́. Ки́ими бо не растли́хся грехи́? Ки́ими не
содержа́хся злы́ми? Всяк грех соде́ях, вся́кую нечистоту́ вложи́х в ду́шу
мою́, непотре́бен бых Тебе́, Бо́гу моему́, и челове́ком. Кто возста́вит
мя, в сицева́я зла́я и толи́ка па́дшаго согреше́ния? Го́споди Бо́же
мой, на Тя упова́х; а́ще есть ми спасе́ния упова́ние, а́ще побежда́ет
человеколю́бие Твое́ мно́жества беззако́ний мои́х, бу́ди ми спаси́тель, и по
щедро́там Твои́м и ми́лостем Твои́м, осла́би, оста́ви, прости́ ми вся, ели́ка
Ти согреши́х, яко мно́гих зол испо́лнися душа́ моя́, и несть во мне
спасе́ния наде́жды. Поми́луй мя, Бо́же, по вели́цей ми́лости Твое́й и не
возда́ждь ми по дело́м мои́м, и не осуди́ мя по дея́ниям мои́м, но
обрати́, заступи́, изба́ви ду́шу мою от совозраста́ющих ей зол и лю́тых
восприя́тий. Спаси́ мя ра́ди ми́лости Твоея́, да иде́же умно́жится грех,
преизоби́лует благода́ть Твоя́; и восхвалю́ и просла́влю Тя всегда́, вся
дни живота́ моего́. Ты бо еси́ Бог ка́ющихся и Спас согреша́ющих; и
Тебе́ сла́ву возсыла́ем со Безнача́льным Твои́м Отце́м и Пресвяты́м и
Благи́м, и Животворя́щим Твои́м Ду́хом ны́не и при́сно, и во ве́ки веко́в.
Ами́нь.



 

\bfseries Молитва 5-я, святого Иоанна Дамаскина\normalfont{}


   Влады́ко Го́споди Иису́се Христе́, Бо́же наш, еди́не име́яй власть
челове́ком оставля́ти грехи́, я́ко благ и Человеколю́бец пре́зри моя́ вся
в ве́дении и не в ве́дении прегреше́ния, и сподо́би мя неосужде́нно
причасти́тися Боже́ственных, и пресла́вных, и пречи́стых, и животворя́щих
Твои́х Та́ин, не в тя́жесть, ни в му́ку, ни в приложе́ние грехо́в, но во
очище́ние, и освяще́ние, и обруче́ние бу́дущаго Живота и ца́рствия, в сте́ну и

по́мощь, и в возраже́ние сопроти́вных, во истребле́ние мно́гих мои́х
согреше́ний. Ты бо еси́ Бог ми́лости, и щедро́т, и человеколю́бия, и Тебе́
сла́ву возсыла́ем, со Отце́м, и Святы́м Ду́хом, ны́не и при́сно, и во ве́ки
веко́в. Ами́нь.



 

\bfseries Молитва 6-я, святого Василия Великого\normalfont{}


   Вем, Го́споди, я́ко недо́стойне причаща́юся пречи́стаго Твое́го Те́ла и
честны́я Твоея́ Кро́ве, и пови́нен есмь, и суд себе́ ям и пию́, не разсужда́я
Те́ла и Кро́ве Тебе́ Христа́ и Бо́га моего́, но на щедро́ты Твоя́ дерза́я
прихожду́ к Тебе́ ре́кшему: яды́й Мою́ плоть, и пия́й Мою́ кровь, во
Мне пребыва́ет, и Аз в нем. Умилосе́рдися у́бо, Го́споди, и не обличи́
мя гре́шнаго, но сотвори́ со мно́ю по ми́лости Твое́й; и да бу́дут ми
свя́тая сия́ во исцеле́ние, и очище́ние, и просвеще́ние, и сохране́ние, и
спасе́ние, и во освяще́ние души́ и те́ла; во отгна́ние вся́каго мечта́ния, и
лука́ваго дея́ния, и де́йства диа́вольскаго, мы́сленнe во удесе́х мои́х
де́йствуемаго, в дерзнове́ние и любо́вь, я́же к Тебе́; во исправле́ние жития́ и
утвержде́ние, в возраще́ние доброде́тели и соверше́нства; во исполне́ние
за́поведей, в Ду́ха Свята́го обще́ние, в напу́тие живота́ ве́чнаго, во
отве́т благоприя́тен на Стра́шнем суди́щи Твое́м: не в суд или́ во
осужде́ние.



 

\bfseries Молитва 7-я,святого Симеона Нового Богослова\normalfont{}


   От скве́рных усте́н, от ме́рзкаго се́рдца, от нечи́стаго язы́ка, от души́
оскверне́ны, приими́ моле́ние, Христе́ мой, и не пре́зри мои́х ни слове́с, ниже́
образо́в, ниже́ безсту́дия. Даждь ми дерзнове́нно глаго́лати, я́же хощу́,
Христе́ мой, па́че же и научи́ мя, что ми подоба́ет твори́ти и глаго́лати.
Согреши́х па́че блудни́цы, я́же уве́де, где обита́еши, ми́ро купи́вши, прии́де
де́рзостне пома́зати Твои́ но́зе, Бо́га моего́, Влады́ки и Христа́ моего́. Я́коже
о́ну не отри́нул еси́ прише́дшую от се́рдца, ниже́ мене́ возгнуша́йся,
Сло́ве: Твои́ же ми пода́ждь но́зе, и держа́ти и целова́ти, и струя́ми
сле́зными, я́ко многоце́нным ми́ром, сия́ де́рзостно пома́зати. Омы́й
мя слеза́ми мои́ми, очи́сти мя и́ми, Сло́ве. Оста́ви и прегреше́ния
моя́, и проще́ние ми пода́ждь. Ве́си зол мно́жество, ве́си и стру́пы

моя́, и я́звы зри́ши моя́, но и ве́ру ве́си, и произволе́ние зри́ши, и
воздыха́ние слы́шиши. Не таи́тся Тебе́, Бо́же мой, Тво́рче мой, Изба́вителю
мой, ниже́ ка́пля сле́зная, ниже́ ка́пли часть не́кая. Несоде́ланное мое́
ви́десте о́чи Твои́, в кни́зе же Твое́й и еще́ несоде́янная напи́сана Тебе́
суть. Виждь смире́ние мое́, виждь труд мой ели́к, и грехи́ вся оста́ви
ми, Бо́же вся́ческих: да чи́стым се́рдцем, притре́петною мы́слию, и
душе́ю сокруше́нною, нескве́рных Твои́х причащу́ся и пресвяты́х
Та́ин, и́миже оживля́ется и обожа́ется всяк яды́й же и пия́й чи́стым
се́рдцем; Ты бо рекл еси́, Влады́ко мой: всяк яды́й Мою́ Плоть, и пия́й
Мою́ Кровь, во Мне у́бо сей пребыва́ет, в не́мже и Аз есмь. Истинно
сло́во вся́ко Влады́ки и Бо́га моего́: боже́ственных бо причаща́яйся и
боготворя́щих благода́тей, не у́бо есмь еди́н, но с Тобо́ю, Христе́ мой,
Све́том трисо́лнечным, просвеща́ющим мир. Да у́бо не еди́н пребу́ду
кроме́ Тебе́ Живода́вца, дыха́ния моего́, живота́ моего́, ра́дования
моего́, спасе́ния ми́ру. Сего́ ра́ди к Тебе́ приступи́х, я́коже зри́ши,
со слеза́ми, и душе́ю сокруше́нною избавле́ния мои́х прегреше́ний
прошу́ прия́ти ми, и Твои́х живода́тельных и непоро́чных Та́инств
причасти́тися неосужде́нно, да пребу́деши, я́коже рекл еси́, со мно́ю
треокая́нным: да не кроме́ обре́т мя Твоея́ благода́ти, преле́стник восхи́тит
мя льсти́вне, и прельсти́в отведе́т боготворя́щих Твои́х слове́с. Сего́ ра́ди
к Тебе́ припа́даю, и те́пле вопию́ Ти: я́коже блу́днаго прия́л еси́, и
блудни́цу прише́дшую, та́ко приими́ мя блу́днаго и скве́рнаго, Ще́дре.
Душе́ю сокруше́нною, ны́не бо к Тебе́ приходя́, вем, Спа́се, я́ко ины́й,
я́коже аз, не прегреши́ Тебе́, ниже́ соде́я дея́ния, я́же аз соде́ях. Но
сие́ па́ки вем, я́ко не вели́чество прегреше́ний, ни грехо́в мно́жество
превосхо́дит Бо́га моего́ мно́гое долготерпе́ние, и человеколю́бие кра́йнее; но
ми́лостию состра́стия те́пле ка́ющияся, и чи́стиши, и све́тлиши, и
све́та твори́ши прича́стники, о́бщники Божества́ Твое́го соде́ловаяй
незави́стно, и стра́нное и Ангелом, и челове́ческим мы́слем, бесе́дуеши им
мно́гажды, я́коже друго́м Твои́м и́стинным. Сия́ де́рзостна творя́т
мя, сия́ вперя́ют мя, Христе́ мой. И дерза́я Твои́м бога́тым к нам
благодея́нием, ра́дуяся вку́пе и трепе́ща, огне́ви причаща́юся трава́
сый, и стра́нно чу́до, ороша́ем неопа́льно, я́коже у́бо купина́ дре́вле
неопа́льне горя́щи. Ны́не благода́рною мы́слию, благода́рным же се́рдцем,
благода́рными удесы́ мои́ми, души́ и те́ла моего́, покланя́юся и велича́ю, и
славосло́влю Тя, Бо́же мой, я́ко благослове́нна су́ща, ны́не же и во
ве́ки.




 

\bfseries Молитва 8-я, святого Иоанна Златоустого\normalfont{}


   Бо́же, осла́би, оста́ви, прости́ ми согреше́ния моя́, ели́ка Ти согреши́х, а́ще
сло́вом, а́ще де́лом, а́ще помышле́нием, во́лею или́ нево́лею, ра́зумом или́
неразу́мием, вся ми прости́ я́ко благ и Человеколю́бец, и моли́твами
Пречи́стыя Твоея́ Ма́тере, у́мных Твои́х служи́телей и святы́х сил, и всех
святы́х от ве́ка Тебе́ благоугоди́вших, неосужде́нно благоволи́ прия́ти ми
свято́е и пречи́стое Твое́ Те́ло и честну́ю Кровь, во исцеле́ние души́ же и
те́ла, и во очище́ние лука́вых мои́х помышле́ний. Я́ко Твое́ е́сть ца́рство и
си́ла и сла́ва, со Отце́м и Святы́м Ду́хом, ны́не и при́сно, и во ве́ки веко́в.
Ами́нь.



 

\bfseries Его же, 9-я\normalfont{}


   Несмь дово́лен, Влады́ко Го́споди, да вни́деши под кров души́ моея́; но
поне́же хо́щеши Ты, я́ко Человеколю́бец, жи́ти во мне, дерза́я приступа́ю;
повелева́еши, да отве́рзу две́ри, я́же Ты еди́н созда́л еси́, и вни́деши со
человеколю́бием я́коже еси́, вни́деши и просвеща́еши помраче́нный мой
по́мысл. Ве́рую, я́ко сие́ сотвори́ши: не бо блудни́цу, со слеза́ми прише́дшую к
Тебе́, отгна́л еси́; ниже́ мытаря́ отве́ргл еси́ пока́явшася; ниже́ разбо́йника,
позна́вша Ца́рство Твое́, отгна́л еси́; ниже́ гони́теля пока́явшася оста́вил еси́,
е́же бе: но от покая́ния Тебе́ прише́дшия вся, в ли́це Твои́х друго́в вчини́л
еси́, Еди́н сый благослове́нный всегда́, ны́не и в безконе́чныя ве́ки.
Ами́нь.



 

\bfseries Его же, 10-я\normalfont{}


   Го́споди Иису́се Христе́ Бо́же мой, осла́би, оста́ви, очи́сти и прости́ ми
гре́шному, и непотре́бному, и недосто́йному рабу́ Твоему́, прегреше́ния, и
согреше́ния, и грехопаде́ния моя́, ели́ка Ти от ю́ности моея́, да́же до
настоя́щего дне и часа́ согреши́х: а́ще в ра́зуме и в неразу́мии, а́ще в словесе́х
или́ де́лех, или́ помышле́ниих и мы́слех, и начина́ниих, и всех мои́х чу́вствах.
И моли́твами безсе́менно ро́ждшия Тя Пречи́стыя и Присноде́вы Мари́и,
Ма́тере Твоея́, еди́ныя непосты́дныя наде́жды и предста́тельства и спасе́ния

моего́, сподо́би мя неосужде́нно причасти́тися пречи́стых, безсме́ртных,
животворя́щих и стра́шных Твои́х Та́инств, во оставле́ние грехо́в и в жизнь
ве́чную: во освяще́ние, и просвеще́ние, кре́пость, исцеле́ние, и здра́вие души́
же и те́ла, и в потребле́ние и всесоверше́нное погубле́ние лука́вых мои́х
помысло́в, и помышле́ний, и предприя́тий, и нощны́х мечта́ний, те́мных и
лука́вых духо́в; я́ко Твое́ е́сть ца́рство, и си́ла, и сла́ва, и честь, и
поклоне́ние, со Отце́м и Святы́м Твои́м Ду́хом, ны́не и при́сно, и во ве́ки
веко́в. Ами́нь.



 

\bfseries Молитва 11-я, святого Иоанна Дамаскина\normalfont{}


   Пред две́рьми хра́ма Твое́го предстою́ и лю́тых помышле́ний не отступа́ю;
но Ты, Христе́ Бо́же, мытаря́ оправди́вый, и ханане́ю поми́ловавый, и
разбо́йнику рая́ две́ри отве́рзый, отве́рзи ми утро́бы человеколю́бия Твое́го и
приими́ мя приходя́ща и прикаса́ющася Тебе́, я́ко блудни́цу, и кровоточи́вую:
о́ва у́бо кра́я ри́зы Твоея́ косну́вшися, удо́бь исцеле́ние прия́т, о́ва же
пречи́стеи Твои́ но́зе удержа́вши, разреше́ние грехо́в понесе́. Аз же,
окая́нный, все Твое́ Те́ло дерза́я восприя́ти, да не опале́н бу́ду; но приими́ мя,
я́коже о́ныя, и просвети́ моя́ душе́вныя чу́вства, попаля́я моя́ грехо́вныя
вины́, моли́твами безсе́менно Р́ождшия Тя, и Небе́сных сил; я́ко благослове́н
еси́ во ве́ки веко́в. Ами́нь.



 

\bfseries Молитва святого Иоанна Златоустого\normalfont{}


   Ве́рую, Го́споди, и испове́дую, я́ко Ты еси́ вои́стинну Христо́с, Сын Бо́га
жива́го, прише́дый в мир гре́шныя спасти́, от ни́хже пе́рвый есмь аз. Еще́
ве́рую, я́ко сие́ е́сть са́мое пречи́стое Те́ло Твое́, и сия́ са́мая е́сть
честна́я Кровь Твоя́. Молю́ся у́бо Тебе́: поми́луй мя, и прости́ ми
прегреше́ния моя́, во́льная и нево́льная, я́же сло́вом, я́же де́лом, я́же
ве́дением и неве́дением, и сподо́би мя неосужде́нно причасти́тися
пречи́стых Твои́х Та́инств, во оставле́ние грехо́в, и в жизнь ве́чную.
Ами́нь.



 

\bfseries Приходя же причаститься, произноси мысленно эти стихи Метафраста:\normalfont{}



 

\bfseries Затем:\normalfont{}


   Ве́чери Твоея́ та́йныя днесь, Сы́не Бо́жий, прича́стника мя приими́;
не бо враго́м Твои́м та́йну пове́м, ни лобза́ния Ти дам, я́ко Иу́да,
но я́ко разбо́йник испове́даю Тя: помяни́ мя, Го́споди, во Ца́рствии
Твое́м.



 

\bfseries И стихи:\normalfont{}



 

\bfseries Потом тропари:\normalfont{}


   Услади́л мя еси́ любо́вию, Христе́, и измени́л мя еси́ Боже́ственным
Твои́м раче́нием; но попали́ огне́м невеще́ственным грехи́ моя́, и насы́титися
е́же в Тебе́ наслажде́ния сподо́би: да лику́я возвелича́ю, Бла́же, два
прише́ствия Твоя́.


   Во све́тлостех Святы́х Твои́х ка́ко вни́ду недосто́йный? А́ще бо дерзну́
совни́ти в черто́г, оде́жда мя облича́ет, я́ко несть бра́чна, и свя́зан изве́ржен
бу́ду от А́нгелов. Очи́сти, Го́споди, скве́рну души́ моея́, и спаси́ мя, я́ко
Человеколю́бец.



 

\bfseries Также молитву:\normalfont{}


   Влады́ко Человеколю́бче, Го́споди Иису́се Христе́ Бо́же мой, да не в суд
ми бу́дут Свята́я сия́, за е́же недосто́йну ми бы́ти: но во очище́ние и
освяще́ние души́ же и те́ла, и во обруче́ние бу́дущия жи́зни и ца́рствия. Мне
же, е́же прилепля́тися Бо́гу, бла́го е́сть, полага́ти во Го́споде упова́ние
спасе́ния моего́.



 

\bfseries И еще:\normalfont{}


   Ве́чери Твоея́ та́йныя днесь, Сы́не Бо́жий, прича́стника мя приими́;
не бо враго́м Твои́м та́йну пове́м, ни лобза́ния Ти дам, я́ко Иу́да,
но я́ко разбо́йник испове́даю Тя: помяни́ мя, Го́споди, во Ца́рствии
Твое́м.
   


   Желающий причаститься должен достойно приготовится к этому
святому таинству. Приготовление это (в церковной практике оно называется
говением) продолжается несколько дней и касается как телесной, так и
духовной жизни человека. Телу предписывается воздержание, т. е. телесная
чистота (воздержание от супружеских отношений) и ограничение в пищи
(пост). В дни поста исключается пища животного происхождения "---
мясо, молоко, яйца и, про строгом посте, рыба. Хлеб, овощи, фрукты
употребляются в умеренном количестве. Ум не должен рассеиваться по

мелочам житейским и развлекаться.


   В дни говения надлежит посещать богослужения в храме, если позволят
обстоятельства, и более прилежно выполнять домашнее молитвенное
правило: кто читает обычно не все утренние и вечерние молитвы, пусть
читает все полностью, кто не читает каноны, пусть в эти дни читает хотя бы
по одному канону. Накануне причащения надо быть на вечернем
богослужении и прочитать дома, кроме обычных молитв на сон грядущим,
канон покаянный, канон Богородице и Ангелу хранителю. Каноны читают
или один за другим полностью, или соединяя таким образом: читается
ирмос первой песни покаянного канона («Яко по суху петешествовав
Израиль, по бездне стопами, гонителя фараона видя потопляема, Богу
победную песнь поим, вопияше») и тропари, затем тропари первой
песни канона Богородице («Многими содержимь напастьми, к Тебе
прибегаю, спасения иский: о, Мати Слова и Дево, от тяжких и лютых мя
спаси»), опуская ирмос «Воду прошед…», и тропари канона Ангелу
хранителю, тоже без ирмоса («Поим Господеви, проведшему люди Своя
сквозе Чермное море, яко един славно прославися»). Так же читают
и следующие песни. Тропари перед каноном Богородице и Ангелу
хранителю, а также стихиры после канона Богородице в таком случае
опускаются.


   Читается также канон ко причащению и, кто пожелает,"--- акафист Иисусу
Сладчайшему. После полуночи уже не едят и не пьют, ибо принято
приступать к Таинству Причащения натощак. Утром прочитываются
утренние молитвы и все последование ко Святому Причащению, кроме
канона, прочитанного накануне.


   Перед причащением необходима исповедь "--- вечером ли, или утром, перед
литургией.

   


\mychapterending

\mychapter{Акафист Иисусу Сладчайшему}
%/text11.htm



\myfig{img/B-2757_01-IV.jpg}

\bfseries Кондак 1\normalfont{}


Возбранный Воеводо и Господи, ада победителю, яко избавлься от вечныя смерти, похвальная восписую Ти, создание и раб Твой; но, яко имеяй милосердие неизреченное, от всяких мя бед свободи, зовуща: Иисусе, Сыне Божий, помилуй мя.


\medskip


\bfseries Икос 1\normalfont{}


Ангелов Творче и Господи cил, отверзи ми недоуменный ум и язык на похвалу пречистаго Твоего имене, якоже глухому и гугнивому древле слух и язык отверзл еси, и, глаголаше зовый таковая: Иисусе пречудный, aнгелов удивление; Иисусе пресильный, прародителей избавление. Иисусе пресладкий, патриархов величание; Иисусе преславный, царей укрепление. Иисусе прелюбимый, пророков исполнение; Иисусе предивный, мучеников крепосте. Иисусе претихий, монахов радосте; Иисусе премилостивый, пресвитеров сладосте. Иисусе премилосердый, постников воздержание; Иисусе пресладостный, преподобных радование. Иисусе пречестный, девственных целомудрие; Иисусе предвечный, грешников спасение. Иисусе, Сыне Божий, помилуй мя.


\medskip


\bfseries Кондак 2\normalfont{}


Видя вдовицу зельне плачущу, Господи, якоже бо тогда умилосердився, сына ея на погребение несома воскресил еси; сице и о мне умилосердися, Человеколюбче, и грехми умерщвленную мою душу воскреси, зовущую: Аллилуиа.


\medskip


\bfseries Икос 2\normalfont{}


Разум неуразуменный разумети Филипп ища, Господи, покажи нам Отца, глаголаше; Ты же к нему: толикое время сый со Мною, не познал ли еси, яко Отец во Мне, и Аз во Отце есмь? Темже, Неизследованне, со страхом зову Ти: Иисусе, Боже предвечный; Иисусе, Царю пресильный. Иисусе, Владыко долготерпеливый; Иисусе, Спасе премилостивый. Иисусе, хранителю мой преблагий; Иисусе, очисти грехи моя. Иисусе, отыми беззакония моя; Иисусе, отпусти неправды моя. Иисусе, надеждо моя, не остави мене; Иисусе, помощниче мой, не отрини мене. Иисусе, Создателю мой, не забуди мене; Иисусе, Пастырю мой, не погуби мене. Иисусе, Сыне Божий, помилуй мя.


\medskip


\bfseries Кондак 3\normalfont{}


Силою свыше апостолы облекий, Иисусе, во Иерусалиме седящия, облецы и мене, обнаженнаго от всякаго благотворения, теплотою Духа Святаго Твоего и даждь ми с любовью пети Тебе: Аллилуиа.


\medskip


\bfseries Икос 3\normalfont{}


Имеяй богатство милосердия, мытари и грешники, и неверныя призвал еси, Иисусе; не презри и мене ныне, подобнаго им, но, яко многоценное миро, приими песнь сию: Иисусе, сило непобедимая; Иисусе, милосте безконечная. Иисусе, красото пресветлая; Иисусе, любы неизреченная. Иисусе, Сыне Бога Живаго; Иисусе, помилуй мя грешнаго. Иисусе, услыши мя в беззакониих зачатаго; Иисусе, очисти мя во гресех рожденнаго. Иисусе, научи мя непотребнаго; Иисусе, освети мя темнаго. Иисусе, очисти мя сквернаго; Иисусе, возведи мя блуднаго. Иисусе, Сыне Божий, помилуй мя.


\medskip


\bfseries Кондак 4\normalfont{}


Бурю внутрь имеяй помышлений сумнительных, Петр утопаше; узрев же во плоти Тя суща, Иисусе, и по водам ходяща, позна Тя Бога истиннаго и, руку спасения получив, рече: Аллилуиа.


\medskip


\bfseries Икос 4\normalfont{}


Слыша слепый мимоходяща Тя, Господи, путем вопияше: Иисусе, Сыне Давидов, помилуй мя! И, призвав, отверзл еси очи его. Просвети убо милостию Твоею очи мысленныя сердца и мене, вопиюща Ти и глаголюща: Иисусе, вышних Создателю; Иисусе, нижних Искупителю. Иисусе, преисподних потребителю; Иисусе, всея твари украсителю. Иисусе, души моея утешителю; Иисусе, ума моего просветителю. Иисусе, сердца моего веселие; Иисусе, тела моего здравие. Иисусе, Спасе мой, спаси мя; Иисусе, свете мой, просвети мя. Иисусе, муки всякия избави мя; Иисусе, спаси мя, недостойнаго. Иисусе, Сыне Божий, помилуй мя.


\medskip


\bfseries Кондак 5\normalfont{}


Боготочною Кровию якоже искупил еси нас древле от законныя клятвы, Иисусе, сице изми нас от сети, еюже змий запят ны страстьми плотскими, и блудным наваждением, и злым унынием, вопиющия Ти: Аллилуиа.


\medskip


\bfseries Икос 5\normalfont{}


Видевше отроцы еврейстии во образе человечестем Создавшаго рукою человека, и Владыку разумевше Его, потщашася ветвьми угодити Ему, осанна вопиюще. Мы же песнь приносим Ти, глаголюще: Иисусе, Боже истинный; Иисусе, Сыне Давидов. Иисусе, Царю преславный; Иисусе, Агнче непорочный. Иисусе, Пастырю предивный; Иисусе, хранителю во младости моей. Иисусе, кормителю во юности моей; Иисусе, похвало в старости моей. Иисусе, надежде в смерти моей; Иисусе, животе по смерти моей. Иисусе, утешение мое на суде Твоем; Иисусе, желание мое, не посрами мене тогда. Иисусе, Сыне Божий, помилуй мя.


\medskip


\bfseries Кондак 6\normalfont{}


Проповедник богоносных вещание и глаголы исполняя, Иисусе, на земли явлься и с человеки Невместимый пожил еси, и болезни наша подъял еси, отнюдуже ранами Твоими мы исцелевше, пети навыкохом: Аллилуиа.


\medskip


\bfseries Икос 6\normalfont{}


Возсия вселенней просвещение истины Твоея, и отгнася лесть бесовская: идоли бо, Спасе наш, не терпяще Твоея крепости, падоша. Мы же, спасение получивше, вопием Ти: Иисусе, истино, лесть отгонящая; Иисусе, свете, превышший всех светлостей. Иисусе, Царю, премогаяй всех крепости; Иисусе, Боже, пребываяй в милости. Иисусе, Хлебе Животный, насыти мя алчущаго; Иисусе, источниче разума, напой мя жаждущаго. Иисусе, одеждо веселия, одей мя тленнаго; Иисусе, покрове радости, покрый мя недостойнаго. Иисусе, подателю просящим, даждь мне плач за грехи моя; Иисусе, обретение ищущим, обрящи душу мою. Иисусе, отверзителю толкущим, отверзи сердце мое окаянное; Иисусе, Искупителю грешных, очисти беззакония моя. Иисусе, Сыне Божий, помилуй мя.


\medskip


\bfseries Кондак 7\normalfont{}


Хотя сокровенную тайну от века открыти, яко овча на заколение веден был еси, Иисусе, и яко агнец прямо стригущаго его безгласен, и яко Бог из мертвых воскресл еси, и со славою на небеса вознеслся еси, и нас совоздвигл еси, зовущих: Аллилуиа.


\medskip


\bfseries Икос 7\normalfont{}


Дивную показа тварь, явлейся Творец нам: без семене от Девы воплотися, из гроба, печати не рушив, воскресе, и ко апостолом, дверем затворенным, с плотию вниде. Темже чудящеся, воспоим: Иисусе, Слове необыменный; Иисусе, Слове несоглядаемый. Иисусе, сило непостижимая; Иисусе, мудросте недомыслимая. Иисусе, Божество неописанное; Иисусе, господство неисчетное. Иисусе, царство непобедимое; Иисусе, владычество безконечное. Иисусе, крепосте высочайшая; Иисусе, власте вечная. Иисусе, Творче мой, ущедри мя; Иисусе, Спасе мой, спаси мя. Иисусе, Сыне Божий, помилуй мя.


\medskip


\bfseries Кондак 8\normalfont{}


Странно Бога вочеловечшася видяще, устранимся суетнаго мира и ум на Божественная возложим. Сего бо ради Бог на землю сниде, да нас на небеса возведет, вопиющих Ему: Аллилуиа.


\medskip


\bfseries Икос 8\normalfont{}


Весь бе в нижних, и вышних никакоже отступи Неисчетный, егда волею нас ради пострада, и смертию Своею нашу смерть умертви, и воскресением живот дарова поющим: Иисусе, сладосте сердечная; Иисусе, крепосте телесная. Иисусе, светлосте душевная; Иисусе, быстрото умная. Иисусе, радосте совестная; Иисусе, надеждо известная. Иисусе, памяте предвечная; Иисусе, похвало высокая. Иисусе, славо моя превознесенная; Иисусе, желание мое, не отрини мене. Иисусе, Пастырю мой, взыщи мене; Иисусе, Спасе мой, спаси мене. Иисусе, Сыне Божий, помилуй мя.


\medskip


\bfseries Кондак 9\normalfont{}


Все естество ангельское безпрестани славит пресвятое имя Твое, Иисусе, на небеси: Свят, Свят, Свят, вопиюще; мы же, грешнии на земли бренными устнами вопием: Аллилуиа.


\medskip


\bfseries Икос 9\normalfont{}


Ветия многовещанны, якоже рыбы безгласныя видим о Тебе, Иисусе, Спасе наш: недоумеют бо глаголати, како Бог непреложний и человек совершенный пребываеши? Мы же таинству дивящеся, вопием верно: Иисусе, Боже предвечный; Иисусе, Царю царствующих. Иисусе, Владыко владеющих; Иисусе, Судие живых и мертвых. Иисусе, надеждо ненадежных; Иисусе, утешение плачущих. Иисусе, славо нищих; Иисусе, не осуди мя по делом моим. Иисусе, очисти мя по милости Твоей; Иисусе, отжени от мене уныние. Иисусе, просвети моя мысли сердечныя; Иисусе, даждь ми память смертную. Иисусе, Сыне Божий, помилуй мя.


\medskip


\bfseries Кондак 10\normalfont{}


Спасти хотя мир, Восточе востоком, к темному западу "--- естеству нашему пришед, смирился еси до смерти; темже превознесеся имя Твое паче всякаго имене, и от всех колен небесных и земных слышиши: Аллилуиа.


\medskip


\bfseries Икос 10\normalfont{}


Царю Превечный, Утешителю, Христе истинный, очисти ны от всякия скверны, якоже очистил еси десять прокаженных, и исцели ны, якоже исцелил еси сребролюбивую душу Закхеа мытаря, да вопием Ти, во умилении зовуще: Иисусе, сокровище нетленное; Иисусе, богатство неистощимое. Иисусе, пище крепкая; Иисусе, питие неисчерпаемое. Иисусе, нищих одеяние; Иисусе, вдов заступление. Иисусе, сирых защитниче; Иисусе, труждающихся помоще. Иисусе, странных наставниче; Иисусе, плавающих кормчий. Иисусе, бурных отишие; Иисусе Боже, воздвигни мя падшаго. Иисусе, Сыне Божий, помилуй мя.


\medskip


\bfseries Кондак 11\normalfont{}


Пение всеумиленное приношу Ти недостойный, вопию Ти яко хананеа: Иисусе, помилуй мя; не дщерь бо, но плоть имам страстьми люте бесящуюся и яростию палимую, и исцеление даждь вопиющу Ти: Аллилуиа.


\medskip


\bfseries Икос 11\normalfont{}


Светоподательна светильника сущим во тьме неразумия, прежде гоняй Тя Павел, богоразумнаго гласа силу внуши и душевную быстроту уясни; сице и мене темныя зеницы душевныя просвети, зовуща: Иисусе, Царю мой прекрепкий; Иисусе, Боже мой пресильный. Иисусе, Господи мой пребезсмертный; Иисусе, Создателю мой преславный. Иисусе, Наставниче мой предобрый; Иисусе, Пастырю мой прещедрый. Иисусе, Владыко мой премилостивый; Иисусе, Спасе мой премилосердый. Иисусе, просвети моя чувствия, потемненныя страстьми; Иисусе, исцели мое тело, острупленное грехми. Иисусе, очисти мой ум от помыслов суетных; Иисусе, сохрани сердце мое от похотей лукавых. Иисусе, Сыне Божий, помилуй мя.


\medskip


\bfseries Кондак 12\normalfont{}


Благодать подаждь ми, всех долгов решителю, Иисусе, и приими мя кающася, якоже приял еси Петра, отвергшагося Тебе, и призови мя унывающаго, якоже древле Павла, гоняща Тя, и услыши мя, вопиюща Ти: Аллилуиа.


\medskip


\bfseries Икос 12\normalfont{}


Поюще Твое вочеловечение, восхваляем Тя вси, и веруем со Фомою, яко Господь и Бог еси, седяй со Отцем и хотяй судити живым и мертвым. Тогда убо сподоби мя деснаго стояния, вопиющаго: Иисусе, Царю предвечный, помилуй мя; Иисусе, цвете благовонный, облагоухай мя. Иисусе, теплото любимая, огрей мя; Иисусе, храме предвечный, покрый мя. Иисусе, одеждо светлая, украси мя; Иисусе, бисере честный, осияй мя. Иисусе, каменю драгий, просвети мя; Иисусе, солнце правды, освети мя. Иисусе, свете святый, облистай мя; Иисусе, болезни душевныя и телесныя избави мя. Иисусе, из руки сопротивныя изми мя; Иисусе, огня неугасимаго и прочих вечных мук свободи мя. Иисусе, Сыне Божий, помилуй мя.


\medskip


\bfseries Кондак 13\normalfont{}


О, пресладкий и всещедрый Иисусе! Приими ныне малое моление сие наше, якоже приял еси вдовицы две лепте, и сохрани достояние Твое от враг видимых и невидимых, от нашествия иноплеменних, от недуга и глада, от всякия скорби и смертоносныя раны, и грядущия изми муки всех, вопиющих Ти: Аллилуиа, aллилуиа, aллилуиа.


\medskip


\bfseries (Kондак читается трижды)\normalfont{}


\medskip


\bfseries Икос 1\normalfont{}


Ангелов Творче и Господи cил, отверзи ми недоуменный ум и язык на похвалу пречистаго Твоего имене, якоже глухому и гугнивому древле слух и язык отверзл еси, и, глаголаше зовый таковая: Иисусе пречудный, aнгелов удивление; Иисусе пресильный, прародителей избавление. Иисусе пресладкий, патриархов величание; Иисусе преславный, царей укрепление. Иисусе прелюбимый, пророков исполнение; Иисусе предивный, мучеников крепосте. Иисусе претихий, монахов радосте; Иисусе премилостивый, пресвитеров сладосте. Иисусе премилосердый, постников воздержание; Иисусе пресладостный, преподобных радование. Иисусе пречестный, девственных целомудрие; Иисусе предвечный, грешников спасение. Иисусе, Сыне Божий, помилуй мя.


\medskip


\bfseries Кондак 1\normalfont{}


Возбранный Воеводо и Господи, ада победителю, яко избавлься от вечныя смерти, похвальная восписую Ти, создание и раб Твой; но, яко имеяй милосердие неизреченное, от всяких мя бед свободи, зовуща: Иисусе, Сыне Божий, помилуй мя.


\medskip


\bfseries Молитва\normalfont{}


Владыко Господи Иисусе Христе Боже мой, иже неизреченнаго ради Твоего человеколюбия на конец веков во плоть оболкийся от Приснодевы Марии, славлю о мне Твое спасительное промышление, раб Твой, Владыко; песнословлю Тя, яко Тебе ради Отца познах; благословлю Тя, Егоже ради и Дух Святый в мир прииде; покланяюся Твоей по плоти Пречистой Матери, таковей страшней тайне послужившей; восхваляю Твоя Ангельская ликостояния, яко воспеватели и служители Твоего величествия; ублажаю Предтечу Иоанна, Тебе крестившаго, Господи; почитаю и провозвестившия Тя пророки, прославляю апостолы Твоя святыя; торжествую же и мученики, священники же Твоя славлю; покланяюся преподобным Твоим, и вся Твоя праведники пестунствую. Таковаго и толикаго многаго и неизреченнаго лика Божественнаго в молитву привожду Тебе, всещедрому Богу, раб Твой, и сего ради прошу моим согрешением прощения, еже даруй ми всех Твоих ради святых, изряднее же святых Твоих щедрот, яко благословен еси во веки. Аминь


\mychapterending

\mychapter{Акафист Пресвятой Богородице}
%/text13.htm



\myfig{img/5_0.jpg}

\bfseries Кондак 1\normalfont{}

Взбранной Воеводе победительная, яко избавльшеся от злых, благодарственная восписуем Ти раби Твои, Богородице; но яко имущая державу непобедимую, от всяких нас бед свободи, да зовем Ти: радуйся, Невесто Неневестная.


\bfseries Икос 1\normalfont{}

Ангел предстатель с небесе послан бысть рещи Богородице: радуйся, и со безплотным гласом воплощаема Тя зря, Господи, ужасашеся и стояше, зовый к Ней таковая: Радуйся, Еюже рaдocть возсияет; радуйся, Еюже клятва изчезнет. Радуйся, падшаго Адама воззвание; радуйся, слез Евиных избавление. Радуйся, высото неудобовосходимая человеческими помыслы; радуйся, глубино неудобозримая и ангельскима очима. Радуйся, яко еси Царево седалище; радуйся, яко носиши Носящаго вся. Радуйся, Звездо, являющая Солнце; радуйся, утробо Божественнаго воплощения. Радуйся, Еюже обновляется тварь; радуйся, Еюже покланяемся Творцу. Радуйся, Невесто Неневестная.


\bfseries Кондак 2\normalfont{}

Видящи Святая Себе в чистоте, глаголет Гавриилу дерзостно: преславное твоего гласа неудобоприятельно души Моей является: безсеменнаго бо зачатия рождество како глаголеши, зовый: Аллилуиа.


\bfseries Икос 2\normalfont{}

Разум недоразумеваемый разумети Дева ищущи, возопи к служащему: из боку чисту, Сыну како есть родитися мощно, рцы Ми? К Нейже он рече со страхом, обаче зовый сице: Радуйся, совета неизреченнаго Таиннице; радуйся, молчания просящих веро. Радуйся, чудес Христовых начало; радуйся, велений Его главизно. Радуйся, лествице небесная, Еюже сниде Бог; радуйся, мосте, преводяй сущих от земли на небо. Радуйся, Ангелов многословущее чудо; радуйся, бесов многоплачевное поражение. Радуйся, Свет неизреченно родившая; радуйся, еже како, ни единаго же научившая. Радуйся, премудрых превосходящая разум; радуйся, верных озаряющая смыслы. Радуйся, Невесто Неневестная.


\bfseries Кондак 3\normalfont{}


Сила Вышняго осени тогда к зачатию Браконеискусную, и благоплодная Тоя ложесна, яко село показа сладкое, всем хотящим жати спасение, всегда пети сице: Аллилуиа.


\bfseries Икос 3\normalfont{}


Имущи Богоприятную Дева утробу, востече ко Елисавети: младенец же оноя абие познав Сея целование, радовашеся, и играньми яко песньми вопияше к Богородице: Радуйся, отрасли неувядаемыя розго; радуйся, Плода безсмертнаго стяжание. Радуйся, Делателя делающая Человеколюбца; радуйся, Садителя жизни нашея рождшая. Радуйся, ниво, растящая гобзование щедрот; радуйся, трапезо, носящая обилие очищения. Радуйся, яко рай пищный процветаеши; радуйся, яко пристанище душам готовиши. Радуйся, приятное молитвы кадило; радуйся, всего мира очищение. Радуйся, Божие к смертным благоволение; радуйся, смертных к Богу дерзновение. Радуйся, Невесто Неневестная.


\bfseries Кондак 4\normalfont{}


Бурю внутрь имея помышлений сумнительных, целомудренный Иосиф смятеся, к Тебе зря небрачней, и бракоокрадованную помышляя, Непорочная; уведев же Твое зачатие от Духа Свята, рече: Аллилуиа.


\bfseries Икос 4\normalfont{}


Слышаша пастырие Ангелов поющих плотское Христово пришествие, и текше яко к Пастырю видят Сего яко агнца непорочна, во чреве Мариине упасшася, Юже поюще реша: Радуйся, Агнца и Пастыря Мати; радуйся, дворе словесных овец. Радуйся, невидимых врагов мучение; радуйся, райских дверей отверзение. Радуйся, яко небесная срадуются земным; радуйся, яко земная сликовствуют небесным. Радуйся, апостолов немолчная уста; радуйся, страстотерпцев непобедимая дерзосте. Радуйся, твердое веры утверждение; радуйся, светлое благодати познание. Радуйся, Еюже обнажися ад; радуйся, Еюже облекохомся славою. Радуйся, Невесто Неневестная.


\bfseries Кондак 5\normalfont{}


Боготечную звезду узревше волсви, тоя последоваша зари, и яко светильник держаще ю, тою испытаху крепкаго Царя, и достигше Непостижимаго, возрадовашася, Ему вопиюще: Аллилуиа.


\bfseries Икос 5\normalfont{}


Видеша отроцы халдейстии на руку Девичу Создавшаго руками человеки, и Владыку разумевающе Его, аще и рабий прият зрак, потщашася дарми послужити Ему, и возопити Благословенней: Радуйся, Звезды незаходимыя Мати; радуйся, заре таинственнаго дне. Радуйся, прелести пещь угасившая; радуйся, Троицы таинники просвещающая. Радуйся, мучителя безчеловечнаго изметающая от начальства; радуйся, Господа Человеколюбца показавшая Христа. Радуйся, варварскаго избавляющая служения; радуйся, тимения изымающая дел. Радуйся, огня поклонение угасившая; радуйся, пламене страстей изменяющая. Радуйся, верных наставнице целомудрия; радуйся, всех родов веселие. Радуйся, Невесто Неневестная.


\bfseries Кондак 6\normalfont{}


Проповедницы богоноснии, бывше волсви, возвратишася в Вавилон, скончавше Твое пророчество, и проповедавше Тя Христа всем, оставиша Ирода яко буесловна, не ведуща пети: Аллилуиа.


\bfseries Икос 6\normalfont{}


Возсиявый во Египте просвещение истины, отгнал еси лжи тьму: идоли бо его, Спасе, не терпяще Твоея крепости, падоша, сих же избавльшиися вопияху к Богородице: Радуйся, исправление человеков; радуйся, низпадение бесов. Радуйся, прелести державу поправшая; радуйся, идольскую лесть обличившая. Радуйся, море, потопившее фараона мысленнаго; радуйся, каменю, напоивший жаждущия жизни. Радуйся, огненный столпе, наставляяй сущия во тьме; радуйся, покрове миру, ширший облака. Радуйся, пище, манны приемнице; радуйся, сладости святыя служительнице. Радуйся, земле обетования; радуйся, из неяже течет мед и млеко. Радуйся, Невесто Неневестная.


\bfseries Кондак 7\normalfont{}


Хотящу Симеону от нынешняго века преставитися прелестнаго, вдался еси яко младенец тому, но познался еси ему и Бог совершенный. Темже удивися Твоей неизреченней премудрости, зовый: Аллилуиа.


\bfseries Икос 7\normalfont{}


Новую показа тварь, явлься Зиждитель нам от Него бывшим, из безсеменныя прозяб утробы, и сохранив Ю, якоже бе, нетленну, да чудо видяще, воспоем Ю, вопиюще: Радуйся, цвете нетления; радуйся, венче воздержания. Радуйся, воскресения образ облистающая; радуйся, ангельское житие являющая. Радуйся, древо светлоплодовитое, от негоже питаются вернии; радуйся, древо благосеннолиственное, имже покрываются мнози. Радуйся, во чреве носящая Избавителя плененным; радуйся, рождшая Наставника заблудшим. Радуйся, Судии праведнаго умоление; радуйся, многих согрешений прощение. Радуйся, одеждо нагих дерзновения; радуйся, любы, всякое желание побеждающая. Радуйся, Невесто Неневестная.


\bfseries Кондак 8\normalfont{}


Странное рождество видевше, устранимся мира, ум на небеса преложше: сего бо ради высокий Бог на земли явися смиренный человек, хотяй привлещи к высоте Тому вопиющия: Аллилуиа.


\bfseries Икос 8\normalfont{}


Весь бе в нижних и вышних никакоже отступи неописанное Слово: снизхождение бо Божественное, не прехождение же местное бысть, и рождество от Девы Богоприятныя, слышащия сия: Радуйся, Бога невместимаго вместилище; радуйся, честнаго таинства двери. Радуйся, неверных сумнительное слышание; радуйся, верных известная похвало. Радуйся, колеснице пресвятая Сущаго на Херувимех; радуйся, селение преславное Сущаго на Серафимех. Радуйся, противная в тожде собравшая; радуйся, девство и рождество сочетавшая. Радуйся, Еюже разрешися преступление; радуйся, Еюже отверзеся рай. Радуйся, ключу Царствия Христова; радуйся, надеждо благ вечных. Радуйся, Невесто Неневестная.


\bfseries Кондак 9\normalfont{}


Всякое естество ангельское удивися великому Твоего вочеловечения делу; неприступнаго бо яко Бога, зряще всем приступнаго Человека, нам убо спребывающа, слышаща же от всех: Аллилуиа.


\bfseries Икос 9\normalfont{}


Ветия многовещанныя, яко рыбы безгласныя видим о Тебе, Богородице, недоумевают бо глаголати, еже како и Дева пребываеши, и родити возмогла еси. Мы же, таинству дивящеся, верно вопием: Радуйся, премудрости Божия приятелище, радуйся, промышления Его сокровище. Радуйся, любомудрыя немудрыя являющая; радуйся, хитрословесныя безсловесныя обличающая. Радуйся, яко обуяша лютии взыскателе; радуйся, яко увядоша баснотворцы. Радуйся, афинейская плетения растерзающая; радуйся, рыбарския мрежи исполняющая. Радуйся, из глубины неведения извлачающая; радуйся, многи в разуме просвещающая. Радуйся, кораблю хотящих спастися; радуйся, пристанище житейских плаваний. Радуйся, Невесто Неневестная.


\bfseries Кондак 10\normalfont{}


Спасти хотя мир, Иже всех Украситель, к сему самообетован прииде, и Пастырь сый, яко Бог, нас ради явися по нам человек: подобным бо подобное призвав, яко Бог слышит: Аллилуиа.


\bfseries Икос 10\normalfont{}


Стена еси девам, Богородице Дево, и всем к Тебе прибегающим: ибо небесе и земли Творец устрои Тя, Пречистая, вселься во утробе Твоей, и вся приглашати Тебе научи: Радуйся, столпе девства; радуйся, дверь спасения. Радуйся, начальнице мысленнаго наздания; радуйся, подательнице Божественныя благости. Радуйся, Ты бо обновила еси зачатыя студно; радуйся, Ты бо наказала еси окраденныя умом. Радуйся, тлителя смыслов упраждняющая; радуйся, Сеятеля чистоты рождшая. Радуйся, чертоже безсеменнаго уневещения; радуйся, верных Господеви сочетавшая. Радуйся, добрая младопитательнице девам; радуйся, невестокрасительнице душ святых. Радуйся, Невесто Неневестная.


\bfseries Кондак 11\normalfont{}


Пение всякое побеждается, спростретися тщащееся ко множеству многих щедрот Твоих: равночисленныя бо песка песни аще приносим Ти, Царю Святый, ничтоже совершаем достойно, яже дал еси нам, Тебе вопиющим: Аллилуиа.


\bfseries Икос 11\normalfont{}


Светоприемную свещу, сущим во тьме явльшуюся, зрим Святую Деву, невещественный бо вжигающи огнь, наставляет к разуму Божественному вся, зарею ум просвещающая, званием же почитаемая, сими: Радуйся, луче умнаго Солнца; радуйся, светило незаходимаго Света. Радуйся, молние, души просвещающая; радуйся, яко гром враги устрашающая. Радуйся, яко многосветлое возсияваеши просвещение; радуйся, яко многотекущую источаеши реку. Радуйся, купели живописующая образ; радуйся, греховную отъемлющая скверну. Радуйся, бане, омывающая совесть; радуйся, чаше, черплющая рaдocть. Радуйся, обоняние Христова благоухания; радуйся, животе тайнаго веселия. Радуйся, Невесто Неневестная.


\bfseries Кондак 12\normalfont{}


Благодать дати восхотев, долгов древних, всех долгов Решитель человеком, прииде Собою ко отшедшим Того благодати, и раздрав рукописание, слышит от всех сице: Аллилуиа.


\bfseries Икос 12\normalfont{}


Поюще Твое Рождество, хвалим Тя вси, яко одушевленный храм, Богородице: во Твоей бо вселився утробе содержай вся рукою Господь, освяти, прослави и научи вопити Тебе всех: Радуйся, селение Бога и Слова; радуйся, святая святых большая. Радуйся, ковчеже, позлащенный Духом; радуйся, сокровище живота неистощимое. Радуйся, честный венче людей благочестивых; радуйся, честная похвало иереев благоговейных. Радуйся, церкве непоколебимый столпе; радуйся, Царствия нерушимая стено. Радуйся, Еюже воздвижутся победы; радуйся, Еюже низпадают врази. Радуйся, тела моего врачевание; радуйся, души моея спасение. Радуйся, Невесто Неневестная.


\bfseries Кондак 13\normalfont{}


О, Всепетая Мати, рождшая всех святых Святейшее Слово! Нынешнее приемши приношение, от всякия избави напасти всех, и будущия изми муки, о Тебе вопиющих: Аллилуиа, aллилуиа, aллилуиа.


\itshape (Kондак читается трижды)\normalfont{}


\bfseries Икос 1\normalfont{}


Ангел предстатель с небесе послан бысть рещи Богородице: радуйся, и со безплотным гласом воплощаема Тя зря, Господи, ужасашеся и стояше, зовый к Ней таковая: Радуйся, Еюже рaдocть возсияет; радуйся, Еюже клятва изчезнет. Радуйся, падшаго Адама воззвание; радуйся, слез Евиных избавление. Радуйся, высото неудобовосходимая человеческими помыслы; радуйся, глубино неудобозримая и ангельскима очима. Радуйся, яко еси Царево седалище; радуйся, яко носиши Носящаго вся. Радуйся, Звездо, являющая Солнце; радуйся, утробо Божественнаго воплощения. Радуйся, Еюже обновляется тварь; радуйся, Еюже покланяемся Творцу. Радуйся, Невесто Неневестная.


\bfseries Кондак 1\normalfont{}


Взбранной Воеводе победительная, яко избавльшеся от злых, благодарственная восписуем Ти раби Твои, Богородице; но яко имущая державу непобедимую, от всяких нас бед свободи, да зовем Ти: радуйся, Невесто Неневестная.


\bfseries Молитвы\normalfont{}


О, Пресвятая Госпоже Владычице Богородице, вышши еси всех Ангел и Архангел, и всея твари честнейши, помощнице еси обидимых, ненадеющихся надеяние, убогих заступнице, печальных утешение, алчущих кормительнице, нагих одеяние, больных исцеление, грешных спасение, христиан всех поможение и заступление. О, Всемилостивая Госпоже, Дево Богородице Владычице, милостию Твоею спаси и помилуй святейшия патриархи православныя, преосвященныя митрополиты, архиепископы и епископы и весь священнический и иноческий чин, и вся православныя христианы ризою Твоею честною защити; и умоли, Госпоже, из Тебе без семене воплотившагося Христа Бога нашего, да препояшет нас силою Своею свыше, на невидимыя и видимыя враги наша. О, Всемилостивая Госпоже Владычице Богородице! Воздвигни нас из глубины греховныя и избави нас от глада, губительства, от труса и потопа, от огня и меча, от нахождения иноплеменных и междоусобныя брани, и от напрасныя смерти, и от нападения вражия, и от тлетворных ветр, и от смертоносныя язвы, и от всякаго зла. Подаждь, Госпоже, мир и здравие рабом Твоим, всем православным христианом, и просвети им ум, и очи сердечнии, еже ко спасению; и сподоби ны, грешныя рабы Твоя, Царствия Сына Твоего, Христа Бога нашего; яко держава Его благословена и препрославлена, со Безначальным Его Отцем, и с Пресвятым, и Благим, и Животворящим Его Духом, ныне и присно, и во веки веков. Аминь.


О, Пресвятая Дево Мати Господа, Царице Небесе и земли! Вонми многоболезненному воздыханию души нашея, призри с высоты святыя Твоея на нас, с верою и любовию покланяющихся пречистому образу Твоему. Се бо грехми погружаемии и скорбьми обуреваемии, взирая на Твой образ, яко живей Ти сущей с нами, приносим смиренныя моления наша. Не имамы бо ни иныя помощи, ни инаго предстательства, ни утешения, токмо Тебе, о, Мати всех скорбящих и обремененных. Помози нам немощным, утоли скорбь нашу, настави на путь правый нас, заблуждающих, уврачуй и спаси безнадежных, даруй нам прочее время живота нашего в мире и тишине проводити, подаждь христианскую кончину, и на страшнем суде Сына Твоего явися нам милосердая Заступница, да всегда поем, величаем и славим Тя, яко благую Заступницу рода христианскаго, со всеми угодившими Богу. Аминь.

\mychapterending

\mychapter{Благодарственные молитвы по Святом Причащении}
%/text206.htm



Сла́ва Тебе́, Бо́же.
Сла́ва Тебе́, Бо́же. Сла́ва Тебе́, Бо́же.





\bfseries Благодарственная моли́тва, 1-я\normalfont{}


   Благодарю́ Тя, Го́споди, Бо́же мой, я́ко не отри́нул мя еси́ гре́шнаго, но
о́бщника мя бы́ти святы́нь Твои́х сподо́бил еси́. Благодарю́ Тя, я́ко
мене́ недосто́йнаго причасти́тися Пречи́стых Твои́х и Небе́сных Даро́в
сподо́бил еси́. Но Влады́ко Человеколю́бче, нас ра́ди уме́рый же и
воскресы́й, и дарова́вый нам стра́шная сия́ и животворя́щая Та́инства во
благодея́ние и освяще́ние душ и теле́с на́ших, да́ждь бы́ти сим и мне
во исцеле́ние души́ же и те́ла, во отгна́ние вся́каго сопроти́внаго, в
просвеще́ние о́чию се́рдца моего́, в мир душе́вных мои́х сил, в ве́ру
непосты́дну, в любо́вь нелицеме́рну, во исполне́ние прему́дрости, в
соблюде́ние за́поведей Твои́х, в приложе́ние Боже́ственныя Твоея́
благода́ти и Твоего́ Ца́рствия присвое́ние; да во святы́ни Твое́й те́ми
сохраня́емь, Твою́ благода́ть помина́ю всегда́, и не ктому́ себе́ живу́, но Тебе́,
на́шему Влады́це и Благоде́телю; и та́ко сего́ жития́ изше́д о наде́жди
живота́ ве́чнаго, в присносу́щный дости́гну поко́й, иде́же пра́зднующих
глас непреста́нный, и безконе́чная сла́дость, зря́щих Твоего́ лица́
добро́ту неизрече́нную. Ты бо еси́ и́стинное жела́ние, и неизрече́нное
весе́лие лю́бящих Тя, Христе́ Бо́же наш, и Тя пое́т вся́ тварь во ве́ки.
Ами́нь.



 

\bfseries Молитва 2-я, святого Василия Великого\normalfont{}


   Влады́ко Христе́ Бо́же, Царю́ веко́в, и Соде́телю всех, благодарю́ Тя о
всех, я́же ми по́дал благи́х, и о причаще́нии пречи́стых и животворя́щих
Твои́х Та́инств. Молю́ у́бо Тя, Бла́же и Человеколю́бче: сохра́ни мя под
кро́вом Твои́м, и в се́ни крилу́ Твое́ю; и да́руй ми чи́стою со́вестию, да́же до
после́дняго моего́ издыха́ния, досто́йно причаща́тися святы́нь Твои́х, во
оставле́ние грехо́в, и в жизнь ве́чную. Ты бо еси́ хлеб живо́тный, исто́чник
святы́ни, Пода́тель благи́х, и Тебе́ сла́ву возсыла́ем, со Отце́м и Святы́м
Ду́хом, ны́не и при́сно, и во ве́ки веко́в. Ами́нь.




 

\bfseries Молитва 3-я, Симеона Метафраста\normalfont{}


   Да́вый пи́щу мне плоть Твою́ во́лею, огнь сый и опаля́яй недосто́йныя, да
не опали́ши мене́, Соде́телю мой; па́че же пройди́ во у́ды мо́я, во вся́ соста́вы,
во утро́бу, в се́рдце. Попали́ те́рние всех мои́х прегреше́ний. Ду́шу очи́сти,
освяти́ помышле́ния. Соста́вы утверди́ с костьми́ вку́пе. Чувств просвети́
просту́ю пятери́цу. Всего́ мя спригвозди́ стра́ху Твоему́. При́сно покры́й,
соблюди́ же, и сохра́ни мя от вся́каго де́ла и сло́ва душетле́ннаго. Очи́сти и
омы́й, и украси́ мя; удобри́, вразуми́, и просвети́ мя. Покажи́ мя Твое́ селе́ние
еди́наго Ду́ха, и не ктому́ селе́ние греха́. Да я́ко Твоего́ до́му, вхо́дом
причаще́ния, я́ко огня́ мене́ бежи́т всяк злоде́й, вся́ка страсть. Моли́твенники
Тебе́ приношу́ вся́ святы́я, чинонача́лия же безпло́тных, Предте́чу Твоего́,
прему́дрыя Апостолы, к сим же Твою́ нескве́рную чи́стую Ма́терь, и́хже
мольбы́ Благоутро́бне приими́, Христе́ мой, и сы́ном све́та соде́лай Твоего́
служи́теля. Ты бо еси́ освяще́ние и Еди́ный на́ших, Бла́же, душ и све́тлость;
и Тебе́ лепоподо́бно, я́ко Бо́гу и Влады́це, сла́ву вси́ возсыла́ем на всяк
день.



 

\bfseries Молитва 4-я\normalfont{}


   Те́ло Твое́ Свято́е, Го́споди, Иису́се Христе́, Бо́же наш, да бу́дет ми в
живо́т ве́чный, и Кровь Твоя́ Честна́я во оставле́ние грехо́в: бу́ди же ми
благодаре́ние сие́ в ра́дость, здра́вие и весе́лие; в стра́шное же и второ́е
прише́ствие Твое́ сподо́би мя гре́шнаго ста́ти одесну́ю сла́вы Твоея́,
моли́твами Пречи́стыя Твоея́ Ма́тере, и всех святы́х.



 

\bfseries Молитва 5-я, ко Пресвятой Богородице\normalfont{}


   Пресвята́я Влады́чице Богоро́дице, све́те помраче́нныя моея́ души́,
наде́ждо, покро́ве, прибе́жище, утеше́ние, ра́дование мое́, благодарю́ Тя, я́ко
сподо́била мя еси́ недосто́йнаго, прича́стника бы́ти Пречи́стаго Те́ла и
Честны́я Кро́ве Сы́на Твоего́. Но ро́ждшая и́стинный Свет, просвети́ мо́я
у́мныя о́чи се́рдца; Я́же Исто́чник безсме́ртия ро́ждшая, оживотвори́ мя
умерщвле́ннаго грехо́м; Я́же ми́лостиваго Бо́га любоблагоутро́бная

Ма́ти, поми́луй мя, и да́ждь ми умиле́ние и сокруше́ние в се́рдце мое́м,
и смире́ние в мы́слех мои́х, и воззва́ние в плене́ниих помышле́ний
мои́х; и сподо́би мя до после́дняго издыха́ния, неосужде́нно приима́ти
пречи́стых Та́ин освяще́ние, во исцеле́ние ду́ши же и те́ла. И пода́ждь ми
сле́зы покая́ния и испове́дания, во е́же пе́ти и сла́вити Тя во вся́ дни
живота́ моего́, я́ко благослове́нна и препросла́вленна еси́ во ве́ки.
Ами́нь.


   Ны́не отпуща́еши раба́ Твоего́, Влады́ко, по глаго́лу Твоему́, с
ми́ром: я́ко ви́деста о́чи мои́ спасе́ние Твое́, е́же еси́ угото́вал пред
лице́м всех люде́й, свет во открове́ние язы́ков и сла́ву люде́й Твои́х,
Изра́иля.


   Святы́й Бо́же, Святы́й Кре́пкий, Святы́й Безсме́ртный, поми́луй нас.
\itshape (Tрижды)\normalfont{}


   Сла́ва Отцу́ и Сы́ну и Свято́му Ду́ху, и ны́не и при́сно и во ве́ки веко́в.
Ами́нь.


   Пресвята́я Тро́ице, поми́луй нас; Го́споди, очи́сти грехи́ на́ша; Влады́ко,
прости́ беззако́ния на́ша; Святы́й, посети́ и исцели́ не́мощи на́ша, и́мене
Твоего́ ра́ди.


   Го́споди, поми́луй. \itshape (Трижды)\normalfont{}


   Сла́ва Отцу́ и Сы́ну и Свято́му Ду́ху, и ны́не и при́сно и во ве́ки веко́в.
Ами́нь.


   О́тче наш, И́же еси́ на небесе́х! Да святи́тся и́мя Твое́, да прии́дет
Ца́рствие Твое́, да бу́дет во́ля Твоя́, я́ко на небеси́ и на земли́. Хлеб наш
насу́щный даждь нам днесь; и оста́ви нам до́лги на́ша, я́коже и мы оставля́ем
должнико́м на́шим; и не введи́ нас во искуше́ние, но изба́ви нас от
лука́ваго.



 

\bfseries Тропарь св. Иоанну Златоустому, глас 8-й:\normalfont{}


   Уст твои́х, я́коже све́тлость огня́, возсия́вши благода́ть, вселе́нную
просвети́: не сребролю́бия ми́рови сокро́вища сниска́, высоту́ нам
смиренному́дрия показа́, но твои́ми словесы́ наказу́я, о́тче Иоа́нне Златоу́сте,
моли́ Сло́ва Христа́ Бо́га спасти́ся душа́м на́шим.



 

\bfseries Кондак, глас 6-й:\normalfont{}


   Сла́ва Отцу́ и Сы́ну и Свято́му Ду́ху.


   От небе́с прия́л еси́ Боже́ственную благода́ть, и твои́ми устна́ма вся́
учи́ши покланя́тися в Тро́ице единому Бо́гу, Иоа́нне Златоу́сте, всеблаже́нне
преподо́бне, досто́йно хва́лим тя: еси́ бо наста́вник, я́ко боже́ственная
явля́я.


   И ны́не и при́сно и во ве́ки веко́в. Ами́нь.


 \itshape Богородичен:\normalfont{} Предста́тельство христиа́н непосты́дное, хода́тайство ко
Творцу́ непрело́жное, не пре́зри гре́шных моле́ний гла́сы, но предвари́, я́ко
Блага́я, на по́мощь нас, ве́рно зову́щих Ти: ускори́ на моли́тву, и
потщи́ся на умоле́ние, предста́тельствующи при́сно, Богоро́дице, чту́щих
Тя.


 \itshape Если совершалась литургия святого Василия Великого, читай\normalfont{}



 

\bfseries Тропарь Василию Великому, глас 1-й:\normalfont{}


   Во всю зе́млю изы́де веща́ние твое́, я́ко прие́мшую сло́во твое́, и́мже
боголе́пно научи́л еси́, естество́ су́щих уясни́л еси́, челове́ческия обы́чаи
украси́л еси́, ца́рское свяще́ние, о́тче преподо́бне, моли́ Христа́ Бо́га, спасти́ся
душа́м на́шим.



 

\bfseries Кондак, глас 4-й:\normalfont{}


   Сла́ва Отцу́ и Сы́ну и Свято́му Ду́ху.


   Яви́лся еси́ основа́ние непоколеби́мое́ Це́ркве, подая́ всем некрадо́мое
госпо́дство челове́ком, запечатле́я твои́ми веле́ньми, небоявле́нне Васи́лие
преподо́бне.


   И ны́не и при́сно и во ве́ки веко́в. Ами́нь.


 \itshape Богородичен:\normalfont{} Предста́тельство христиа́н непосты́дное, хода́тайство ко
Творцу́ непрело́жное, не пре́зри гре́шных моле́ний гла́сы, но предвари́, я́ко
Блага́я, на по́мощь нас, ве́рно зову́щих Ти: ускори́ на моли́тву, и
потщи́ся на умоле́ние, предста́тельствующи при́сно, Богоро́дице, чту́щих
Тя.


 \itshape Если совершалась Литургия Преждеосвященных Даров, читай\normalfont{}



 

\bfseries Тропарь святому Григорию Двоеслову, глас 4-й:\normalfont{}


   И́же от Бо́га свы́ше боже́ственную благода́ть восприе́м, сла́вне Григо́рие,
и Того́ си́лою укрепля́емь, ева́нгельски ше́ствовати изво́лил еси́, отону́дуже у
Христа́ возме́здие трудо́в прия́л еси́ всеблаже́нне: Его́же моли́, да спасе́т
ду́ши на́ша.



 

\bfseries Кондак, глас 3-й:\normalfont{}


   Сла́ва Отцу́ и Сы́ну и Свято́му Ду́ху.


   Подобонача́льник показа́лся еси́ Нача́льника па́стырем Христа́, и́ноков
чреды́, о́тче Григо́рие, ко огра́де небе́сней наставля́я, и отту́ду научи́л еси́
ста́до Христо́во за́поведем Его́: ны́не же с ни́ми ра́дуешися, и лику́еши в
небе́сных кро́вех.


   И ны́не и при́сно и во ве́ки веко́в. Ами́нь.


 \itshape Богородичен:\normalfont{} Предста́тельство христиа́н непосты́дное, хода́тайство ко
Творцу́ непрело́жное, не пре́зри гре́шных моле́ний гла́сы, но предвари́, я́ко
Блага́я, на по́мощь нас, ве́рно зову́щих Ти: ускори́ на моли́тву, и
потщи́ся на умоле́ние, предста́тельствующи при́сно, Богоро́дице, чту́щих
Тя.


   Го́споди, поми́луй. \itshape (12 раз)\normalfont{}


   Сла́ва Отцу́ и Сы́ну и Свято́му Ду́ху, и ны́не и при́сно и во ве́ки веко́в.
Ами́нь.


   Честне́йшую Херуви́м и сла́внейшую без сравне́ния Серафи́м, без
истле́ния Бо́га Сло́ва ро́ждшую, су́щую Богоро́дицу Тя велича́ем.


 \itshape После Причащения да пребывает каждый в чистоте, воздержании и
немногословии, чтобы достойно сохранить в себе принятого Христа.\normalfont{}



\mychapterending

\mychapter{Символ веры}
%/content/Simvol-very



Ве́рую во еди́наго Бо́га Отца́, Вседержи́теля, Творца́ не́бу и
земли́, ви́димым же всем и неви́димым. И во еди́наго Го́спода Иису́са Христа́,
Сы́на Бо́жия, Единоро́днаго, И́же от Отца́ рожде́ннаго пре́жде всех век;
Све́та от Све́та, Бо́га и́стинна от Бо́га и́стинна, рожде́нна, несотворе́нна,
единосу́щна Отцу́, И́мже вся бы́ша. Нас ра́ди челове́к и на́шего ра́ди
спасе́ния сше́дшаго с небе́с и воплоти́вшагося от Ду́ха Свя́та и Мари́и Де́вы и
вочелове́чшася. Распя́таго же за ны при Понти́йстем Пила́те, и страда́вша, и
погребе́нна. И воскре́сшаго в тре́тий день по Писа́нием. И возше́дшаго на
небеса́, и седя́ща одесну́ю Отца́. И па́ки гряду́щаго со сла́вою суди́ти живы́м
и ме́ртвым, Его́же Ца́рствию не бу́дет конца́. И в Ду́ха Свята́го, Го́спода,
Животворя́щаго, И́же от Отца́ исходя́щаго, И́же со Отце́м и Сы́ном
спокланя́ема и ссла́вима, глаго́лавшаго проро́ки. Во еди́ну Святу́ю,
Собо́рную и Апо́стольскую Це́рковь. Испове́дую еди́но креще́ние во
оставле́ние грехо́в. Ча́ю воскресе́ния ме́ртвых, и жи́зни бу́дущаго ве́ка.
Ами́нь.


\mychapterending