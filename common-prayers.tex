\newcommand\pripev[2][Припев:]{{\small\myemph{#1} #2}}
\newcommand\irmos[1]{\pripev[Ирм\'{о}с:]{#1}}
\newcommand\Bogorodichen[1]{\myemph{Богородичен:} #1}
\newcommand\slavan{Слава Отцу и Сыну и Святому Духу.}
\newcommand\slava{{\small\slavan}}
\newcommand\inynen{И ныне и присно и во веки веков. Аминь.}
\newcommand\inyne{{\small\inynen}}
\newcommand\slavainynen{Слава Отцу и Сыну и Святому Духу. И ныне и присно и во веки веков. Аминь.}
\newcommand\slavainyne{{\small\slavainynen}}

\newcommand{\pripevc}[1]{{\small \centerline{#1} \nopagebreak}}
\newcommand{\pripevmskipc}[1]{\medskip\pripevc{#1}}
\newcommand{\pripevpomiluj}{\pripevmskipc{\pripev{\firstletter{П}омилуй мя, Боже, помилуй мя.}}}
\newcommand{\slavac}{\pripevmskipc{\slavan}}
\newcommand{\inynec}{\pripevmskipc{\inynen}}

\newcommand{\TsariuNebesnyj}{%
Царю Небесный, Утешителю, Душе истины, Иже везде сый и вся исполняяй, Сокровище благих и жизни Подателю, прииди и вселися в ны, и очисти ны от всякия скверны, и спаси, Блаже, души наша.}

\newcommand{\TrisviatoePoOtcheNash}{%
Святый Боже, Святый Крепкий, Святый Безсмертный, помилуй нас. \myemph{ (Tрижды)}

Слава Отцу и Сыну и Святому Духу, и ныне и присно и во веки веков. Аминь.

Пресвятая Троице, помилуй нас; Господи, очисти грехи наша; Владыко, прости беззакония наша; Святый, посети и исцели немощи наша, имене Твоего ради.

Господи, помилуй. \myemph{ (Трижды)}

Слава Отцу и Сыну и Святому Духу, и ныне и присно и во веки веков. Аминь.

Отче наш, Иже еси на небесех! Да святится имя Твое, да приидет Царствие Твое, да будет воля Твоя, яко на небеси и на земли. Хлеб наш насущный даждь нам днесь; и остави нам долги наша, якоже и мы оставляем должником нашим; и не введи нас во искушение, но избави нас от лукаваго.
}

\newcommand{\priiditepoklonimsia}{%
Приидите, поклонимся Цареви нашему Богу. \myemph{(Поклон)}

Приидите, поклонимся и припадем Христу, Цареви нашему Богу. \myemph{(Поклон)}

Приидите, поклонимся и припадем Самому Христу, Цареви и Богу нашему. \myemph{(Поклон)}}


\newcommand{\PsalmFifty}{%
Помилуй мя, Боже, по велицей милости Твоей, и по множеству щедрот Твоих очисти беззаконие мое. Наипаче омый мя от беззакония моего, и от греха моего очисти мя; яко беззаконие мое аз знаю, и грех мой предо мною есть выну. Тебе Единому согреших и лукавое пред Тобою сотворих, яко да оправдишися во словесех Твоих, и победиши внегда судити Ти. Се бо, в беззакониих зачат есмь, и во гресех роди мя мати моя. Се бо, истину возлюбил еси; безвестная и тайная премудрости Твоея явил ми еси. Окропиши мя иссопом, и очищуся; омыеши мя, и паче снега убелюся. Слуху моему даси радость и веселие; возрадуются кости смиренныя. Отврати лице Твое от грех моих и вся беззакония моя очисти. Сердце чисто созижди во мне, Боже, и дух прав обнови во утробе моей. Не отвержи мене от лица Твоего и Духа Твоего Святаго не отыми от мене. Воздаждь ми радость спасения Твоего и Духом владычним утверди мя. Научу беззаконыя путем Твоим, и нечестивии к Тебе обратятся. Избави мя от кровей, Боже, Боже спасения моего; возрадуется язык мой правде Твоей. Господи, устне мои отверзеши, и уста моя возвестят хвалу Твою. Яко аще бы восхотел еси жертвы, дал бых убо: всесожжения не благоволиши. Жертва Богу дух сокрушен; сердце сокрушенно и смиренно Бог не уничижит. Ублажи, Господи, благоволением Твоим Сиона, и да созиждутся стены Иерусалимския. Тогда благоволиши жертву правды, возношение и всесожегаемая; тогда возложат на oлтарь Твой тельцы.\par}

\newcommand{\Chestneyshuyu}{%
Достойно есть яко воистинну блажити Тя, Богородицу, Присноблаженную и Пренепорочную и Матерь Бога нашего. Честнейшую Херувим и славнейшую без сравнения Серафим, без истления Бога Слова рождшую, сущую Богородицу Тя величаем.}

\newcommand{\MolitvamiSviatyhOtecNashih}{%
Молитвами святых отец наших, Господи Иисусе Христе, Боже наш, помилуй нас. Аминь.}

\newcommand{\tolkopoblagosloveniyu}{%
{\centering\myemph{\normalfont Читаются только по благословению духовника}\par}}

\newcommand{\tolkosviashennikom}{%
{\centering\myemph{\normalfont Молитва читается священником}\par}}

\newcommand{\SymbolOfFaith}{%
  Верую во единаго Бога Отца, Вседержителя, Творца небу и земли, видимым же всем и невидимым.
  И во единаго Господа Иисуса Христа, Сына Божия, Единороднаго, Иже от Отца рожденнаго прежде всех век; Света от Света, Бога истинна от Бога истинна, рожденна, несотворенна, единосущна Отцу, Имже вся быша.
  Нас ради человек и нашего ради спасения сшедшаго с небес и воплотившагося от Духа Свята и Марии Девы и вочеловечшася.
  Распятаго же за ны при Понтийстем Пилате, и страдавша, и погребенна.
  И воскресшаго в третий день по Писанием.
  И возшедшаго на небеса, и седяща одесную Отца.
  И паки грядущаго со славою судити живым и мертвым, Егоже Царствию не будет конца.
  И в Духа Святаго, Господа, Животворящаго, Иже от Отца исходящего, Иже со Отцем и Сыном спокланяема и сславима, глаголавшаго пророки.
  Во едину Святую, Соборную и Апостольскую Церковь.
  Исповедую едино крещение во оставление грехов.
  Чаю воскресения мертвых, и жизни будущаго века. Аминь.}

\newcommand{\TroparPomilujNas}{Помилуй нас, Господи, помилуй нас; всякаго бо ответа недоумеюще, сию Ти молитву яко Владыце грешнии приносим: помилуй нас.

\slavan

 Господи, помилуй нас, на Тя бо уповахом; не прогневайся на ны зело, ниже помяни беззаконий наших, но призри и ныне яко благоутробен, и избави ны от враг наших; Ты бо еси Бог наш, и мы людие Твои, вси дела руку Твоею, и имя Твое призываем.

\inynen

 Милосердия двери отверзи нам, благословенная Богородице, надеющиися на Тя да не погибнем, но да избавимся Тобою от бед: Ты бо еси спасение рода христианскаго.}

