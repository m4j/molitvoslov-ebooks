\newcommand\pripev[2][Припев:]{\mysmall{\myemph{#1} #2}}
\newcommand\irmos[1]{\pripev[Ирм\'{о}с:]{#1}}
\newcommand\Bogorodichen[1]{\myemph{Богородичен:} #1}
\newcommand\slavan{Сл\'{а}ва Отц\'{у} и С\'{ы}ну и Свят\'{о}му Д\'{у}ху.}
\newcommand\slava{\mysmall{\slavan}}
\newcommand\inynen{И н\'{ы}не и пр\'{и}сно и во в\'{е}ки век\'{о}в. Ам\'{и}нь.}
\newcommand\inyne{\mysmall{\inynen}}
\newcommand\slavainynen{Сл\'{а}ва Отц\'{у} и С\'{ы}ну и Свят\'{о}му Д\'{у}ху, и н\'{ы}не и пр\'{и}сно и во в\'{е}ки век\'{о}в. Ам\'{и}нь.}
\newcommand\slavainyne{\mysmall{\slavainynen}}

\newcommand{\pripevc}[1]{\mysmall{ \centerline{#1} \nopagebreak}}
\newcommand{\pripevmskipc}[1]{\medskip\pripevc{#1}}
\newcommand{\pripevpomiluj}{\pripevmskipc{\pripev{\firstletter{П}омилуй мя, Боже, помилуй мя.}}}
\newcommand{\slavac}{\pripevmskipc{\slavan}}
\newcommand{\inynec}{\pripevmskipc{\inynen}}

\newcommand{\TsariuNebesnyj}{%
Цар\'{ю} Неб\'{е}сный, Ут\'{е}шителю, Д\'{у}ше \'{и}стины, \'{И}же везд\'{е} сый и вся исполн\'{я}яй, Сокр\'{о}вище благ\'{и}х и ж\'{и}зни Под\'{а}телю, приид\'{и} и всел\'{и}ся в ны, и оч\'{и}сти ны от вс\'{я}кия скв\'{е}рны, и спас\'{и}, Бл\'{а}же, д\'{у}ши н\'{а}ша.}

\newcommand{\TrisviatoePoOtcheNash}{%
Свят\'{ы}й Б\'{о}же, Свят\'{ы}й Кр\'{е}пкий, Свят\'{ы}й Безсм\'{е}ртный, пом\'{и}луй нас. \myemph{(Tрижды)}

Сл\'{а}ва Отц\'{у} и С\'{ы}ну и Свят\'{о}му Д\'{у}ху, и н\'{ы}не и пр\'{и}сно и во в\'{е}ки век\'{о}в. Ам\'{и}нь.

Пресвят\'{а}я Тр\'{о}ице, пом\'{и}луй нас; Г\'{о}споди, оч\'{и}сти грех\'{и} н\'{а}ша; Влад\'{ы}ко, прост\'{и} беззак\'{о}ния н\'{а}ша; Свят\'{ы}й, посет\'{и} и исцел\'{и} н\'{е}мощи н\'{а}ша, \'{и}мене Твоег\'{о} р\'{а}ди.

Г\'{о}споди, пом\'{и}луй. \myemph{(Трижды)}

Сл\'{а}ва Отц\'{у} и С\'{ы}ну и Свят\'{о}му Д\'{у}ху, и н\'{ы}не и пр\'{и}сно и во в\'{е}ки век\'{о}в. Ам\'{и}нь.

\'{О}тче наш, \'{И}же ес\'{и} на небес\'{е}х! Да свят\'{и}тся \'{и}мя Тво\'{е}, да при\'{и}дет Ц\'{а}рствие Тво\'{е}, да б\'{у}дет в\'{о}ля Тво\'{я}, \'{я}ко на небес\'{и} и на земл\'{и}. Хлеб наш нас\'{у}щный даждь нам днесь; и ост\'{а}ви нам д\'{о}лги н\'{а}ша, \'{я}коже и мы оставл\'{я}ем должник\'{о}м н\'{а}шим; и не введ\'{и} нас во искуш\'{е}ние, но изб\'{а}ви нас от лук\'{а}ваго.
}

\newcommand{\priiditepoklonimsia}{%
Приид\'{и}те, поклон\'{и}мся Цар\'{е}ви н\'{а}\-ше\-му Б\'{о}гу. \myemph{(Поклон)}

Приид\'{и}те, поклон\'{и}мся и припад\'{е}м Христ\'{у}, Цар\'{е}ви н\'{а}шему Б\'{о}гу. \myemph{(Поклон)}

Приид\'{и}те, поклон\'{и}мся и припад\'{е}м Самом\'{у} Христ\'{у}, Цар\'{е}ви и Б\'{о}гу н\'{а}шему. \myemph{(Поклон)}}


\newcommand{\PsalmFifty}{%
Пом\'{и}луй мя, Б\'{о}же, по вел\'{и}цей м\'{и}лости Тво\'{е}й, и по мн\'{о}жеству щедр\'{о}т Тво\'{и}х оч\'{и}сти беззак\'{о}ние мо\'{е}. Наип\'{а}че ом\'{ы}й мя от беззак\'{о}ния моег\'{о}, и от грех\'{а} моег\'{о} оч\'{и}сти мя; яко беззак\'{о}ние мо\'{е} аз зн\'{а}ю, и грех мой пр\'{е}до мн\'{о}ю есть в\'{ы}ну. Теб\'{е} Ед\'{и}ному согреш\'{и}х и лук\'{а}вое пред Тоб\'{о}ю сотвор\'{и}х, \'{я}ко да оправд\'{и}шися во словес\'{е}х Тво\'{и}х, и побед\'{и}ши внегд\'{а} суд\'{и}ти Ти. Се бо, в беззак\'{о}ниих зач\'{а}т есмь, и во грес\'{е}х род\'{и} мя м\'{а}ти мо\'{я}. Се бо, \'{и}стину возлюб\'{и}л ес\'{и}; безв\'{е}стная и т\'{а}йная прем\'{у}дрости Твое\'{я} яв\'{и}л ми ес\'{и}. Окроп\'{и}ши мя исс\'{о}пом, и оч\'{и}щуся; ом\'{ы}еши мя, и п\'{а}че сн\'{е}га убел\'{ю}ся. Сл\'{у}ху моем\'{у} д\'{а}си р\'{а}дость и вес\'{е}лие; возр\'{а}дуются к\'{о}сти смир\'{е}нныя. Отврат\'{и} лиц\'{е} Тво\'{е} от грех мо\'{и}х и вся беззак\'{о}ния мо\'{я} оч\'{и}сти. С\'{е}рдце ч\'{и}сто соз\'{и}жди во мне, Б\'{о}же, и дух прав обнов\'{и} во утр\'{о}бе мо\'{е}й. Не отв\'{е}ржи мен\'{е} от лиц\'{а} Твоег\'{о} и Д\'{у}ха Твоег\'{о} Свят\'{а}го не отым\'{и} от мен\'{е}. Возд\'{а}ждь ми р\'{а}дость спас\'{е}ния Тво\'{е}го и Д\'{у}хом влад\'{ы}чним утверд\'{и} мя. Науч\'{у} беззак\'{о}ныя пут\'{е}м Тво\'{и}м, и нечест\'{и}вии к Теб\'{е} обрат\'{я}тся. Изб\'{а}ви мя от кров\'{е}й, Б\'{о}же, Б\'{о}же спас\'{е}ния моег\'{о}; возр\'{а}дуется яз\'{ы}к мой пр\'{а}вде Тво\'{е}й. Г\'{о}споди, устн\'{е} мои отв\'{е}рзеши, и уст\'{а} мо\'{я} возвест\'{я}т хвал\'{у} Тво\'{ю}. \'{Я}ко \'{а}ще бы восхот\'{е}л ес\'{и} ж\'{е}ртвы, дал бых \'{у}бо: всесожж\'{е}ния не благовол\'{и}ши. Ж\'{е}ртва Б\'{о}гу дух сокруш\'{е}н; с\'{е}рдце сокруш\'{е}нно и смир\'{е}нно Бог не уничиж\'{и}т. Ублаж\'{и}, Г\'{о}споди, благовол\'{е}нием Тво\'{и}м Си\'{о}на, и да соз\'{и}ждутся ст\'{е}ны Иерусал\'{и}мския. Тогд\'{а} благовол\'{и}ши ж\'{е}ртву пр\'{а}вды, вознош\'{е}ние и всесожег\'{а}емая; тогд\'{а} возлож\'{а}т на oлт\'{а}рь Твой тельц\'{ы}.\par}

\newcommand{\PsalmNinety}{%
Жив\'{ы}й в п\'{о}мощи В\'{ы}шняго, в кр\'{о}ве Б\'{о}га Неб\'{е}снаго водвор\'{и}тся. Реч\'{е}т Г\'{о}сподеви: Заст\'{у}пник мой ес\'{и} и Приб\'{е}жище мо\'{е}, Бог мой, и упов\'{а}ю на Нег\'{о}. \'{Я}ко Той изб\'{а}вит тя от с\'{е}ти л\'{о}вчи и от словес\'{е} мят\'{е}жна, плещм\'{а} Сво\'{и}ма осен\'{и}т тя, и под крил\'{е} Ег\'{о} над\'{е}ешися: ор\'{у}жием об\'{ы}дет тя \'{и}стина Ег\'{о}. Не убо\'{и}шися от стр\'{а}ха нощн\'{а}го, от стрел\'{ы} лет\'{я}щия во дни, от в\'{е}щи во тме преход\'{я}щия, от ср\'{я}ща и б\'{е}са пол\'{у}деннаго. Пад\'{е}т от стран\'{ы} твое\'{я} т\'{ы}сяща, и тма одесн\'{у}ю теб\'{е}, к теб\'{е} же не прибл\'{и}жится, об\'{а}че оч\'{и}ма тво\'{и}ма см\'{о}триши, и возда\'{я}ние гр\'{е}шников \'{у}зриши. \'{Я}ко Ты, Г\'{о}споди, упов\'{а}ние мо\'{е}, В\'{ы}шняго полож\'{и}л ес\'{и} приб\'{е}жище тво\'{е}. Не при\'{и}дет к теб\'{е} зло и р\'{а}на не прибл\'{и}жится телес\'{и} твоем\'{у}, \'{я}ко \'{А}нгелом Сво\'{и}м запов\'{е}сть о теб\'{е}, сохран\'{и}ти тя во всех пут\'{е}х тво\'{и}х. На рук\'{а}х в\'{о}змут тя, да не когд\'{а} преткн\'{е}ши о к\'{а}мень н\'{о}гу тво\'{ю}, на \'{а}спида и васил\'{и}ска наст\'{у}пиши, и попер\'{е}ши льва и зм\'{и}я. \'{Я}ко на Мя упов\'{а} и изб\'{а}влю \'{и}, покр\'{ы}ю \'{и}, \'{я}ко позн\'{а} \'{и}мя Мо\'{е}. Воззов\'{е}т ко Мне и усл\'{ы}шу eг\'{о}, с ним есмь в ск\'{о}рби, изм\'{у} eг\'{о}, и просл\'{а}влю eг\'{о}, долгот\'{о}ю дний исп\'{о}лню eг\'{о}, и явл\'{ю} eм\'{у} спас\'{е}ние Мо\'{е}.\par}

\newcommand{\Predstatelstvo}{%
Предст\'{а}тельство христи\'{а}н непост\'{ы}дное, ход\'{а}тайство ко Творц\'{у} непрел\'{о}жное, не пр\'{е}зри гр\'{е}шных мол\'{е}ний гл\'{а}сы, но предвар\'{и}, \'{я}ко Благ\'{а}я, на п\'{о}мощь нас, в\'{е}рно зов\'{у}щих Ти: ускор\'{и} на мол\'{и}тву, и потщ\'{и}ся на умол\'{е}ние, предст\'{а}тельствующи пр\'{и}сно, Богор\'{о}дице, чт\'{у}щих Тя.}

\newcommand{\Chestneyshuyu}{%
Дост\'{о}йно \'{е}сть \'{я}ко во\'{и}стинну бла\-ж\'{и}\-ти Тя, Богор\'{о}дицу, Присноблаж\'{е}нную и Пренепор\'{о}чную и М\'{а}терь Б\'{о}га н\'{а}шего. Честн\'{е}йшую Херув\'{и}м и сл\'{а}внейшую без сравн\'{е}ния Сераф\'{и}м, без истл\'{е}ния Б\'{о}га Сл\'{о}ва р\'{о}ждшую, с\'{у}щую Богор\'{о}дицу Тя велич\'{а}ем.}

\newcommand{\MolitvamiSviatyhOtecNashih}{%
Мол\'{и}твами свят\'{ы}х от\'{е}ц н\'{а}ших, Г\'{о}споди Иис\'{у}се Христ\'{е}, Б\'{о}же наш, пом\'{и}луй нас. Ам\'{и}нь.}

\newcommand{\LMolitvamiSviatyhOtecNashih}{%
\lettrine{М}{}ол\'{и}твами свят\'{ы}х от\'{е}ц н\'{а}ших, Г\'{о}споди Иис\'{у}се Христ\'{е}, Б\'{о}же наш, пом\'{и}луй нас. Ам\'{и}нь.}

\newcommand{\tolkopoblagosloveniyu}{%
{\centering\myemph{\normalfont Читаются только по благословению духовника}\par}}

\newcommand{\tolkosviashennikom}{%
{\centering\myemph{\normalfont Молитва читается священником}\par}}

\newcommand{\SymbolOfFaith}{%
В\'{е}рую во ед\'{и}наго Б\'{о}га Отц\'{а}, Вседерж\'{и}теля, Творц\'{а} н\'{е}бу и земл\'{и}, в\'{и}димым же всем и нев\'{и}димым.
И во ед\'{и}наго Г\'{о}спода Иис\'{у}са Христ\'{а}, С\'{ы}на Б\'{о}жия, Единор\'{о}днаго, \'{И}же от Отц\'{а} рожд\'{е}ннаго пр\'{е}жде всех век; Св\'{е}та от Св\'{е}та, Б\'{о}га \'{и}стинна от Б\'{о}га \'{и}стинна, рожд\'{е}нна, несотвор\'{е}нна, единос\'{у}щна Отц\'{у}, \'{И}мже вся б\'{ы}ша.
Нас р\'{а}ди челов\'{е}к и н\'{а}шего р\'{а}ди спас\'{е}ния сш\'{е}дшаго с неб\'{е}с и воплот\'{и}вшагося от Д\'{у}ха Св\'{я}та и Мар\'{и}и Д\'{е}вы и вочелов\'{е}чшася.
Расп\'{я}таго же за ны при Понт\'{и}йстем Пил\'{а}те, и страд\'{а}вша, и погреб\'{е}нна.
И воскр\'{е}сшаго в тр\'{е}тий день по Пис\'{а}нием.
И возш\'{е}дшаго на небес\'{а}, и сед\'{я}ща одесн\'{у}ю Отц\'{а}.
И п\'{а}ки гряд\'{у}щаго со сл\'{а}вою суд\'{и}ти жив\'{ы}м и м\'{е}ртвым, Ег\'{о}же Ц\'{а}рствию не б\'{у}дет конц\'{а}.
И в Д\'{у}ха Свят\'{а}го, Г\'{о}спода, Животвор\'{я}щаго, \'{И}же от Отц\'{а} исход\'{я}щаго, \'{И}же со Отц\'{е}м и С\'{ы}ном споклан\'{я}ема и ссл\'{а}вима, глаг\'{о}лавшаго прор\'{о}ки.
Во ед\'{и}ну Свят\'{у}ю, Соб\'{о}рную и Ап\'{о}стольскую Ц\'{е}рковь.
Испов\'{е}дую ед\'{и}но крещ\'{е}ние во оставл\'{е}ние грех\'{о}в.
Ч\'{а}ю воскрес\'{е}ния м\'{е}ртвых, и ж\'{и}зни б\'{у}дущаго в\'{е}ка. Ам\'{и}нь.}

\newcommand{\TroparPomilujNas}{Пом\'{и}луй нас, Г\'{о}споди, пом\'{и}луй нас; вс\'{я}каго бо отв\'{е}та недоум\'{е}юще, си\'{ю} Ти мол\'{и}тву \'{я}ко Влад\'{ы}це гр\'{е}шнии прин\'{о}сим: пом\'{и}луй нас.

\slavan

Г\'{о}споди, пом\'{и}луй нас, на Тя бо упов\'{а}хом; не прогн\'{е}вайся на ны зел\'{о}, ниж\'{е} помян\'{и} беззак\'{о}ний н\'{а}ших, но пр\'{и}зри и н\'{ы}не \'{я}ко благоутр\'{о}бен, и изб\'{а}ви ны от враг н\'{а}ших; Ты бо ес\'{и} Бог наш, и мы л\'{ю}дие Тво\'{и}, вс\'{и} дел\'{а} р\'{у}ку Тво\'{е}ю, и \'{и}мя Тво\'{е} призыв\'{а}ем.

\inynen

Милос\'{е}рдия дв\'{е}ри отв\'{е}рзи нам, благослов\'{е}нная Богор\'{о}дице, над\'{е}ющиися на Тя да не пог\'{и}бнем, но да изб\'{а}вимся Тоб\'{о}ю от бед: Ты бо ес\'{и} спас\'{е}ние р\'{о}да христи\'{а}нскаго.}

