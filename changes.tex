\renewcommand{\ornament}{uzor_begin_9}
\mychapter{Список изменений}
\markright{Список изменений}

Этот раздел содержит информацию об изменениях и исправлениях в материалах сайта \url{http://www.molitvoslov.com}, вошедших в эту и предыдущие публикации. Информация о более новых версиях располагается в начале, а о более старых "--- в конце раздела.

\mysubtitle{Версия 20160406}

Обновление текстов от 4-го апреля 2016 г.

Исправлены опечатки в следующих молитвах и разделах:

\begin{itemize}

\item 115-м псалом в Последовании ко Святому Причащению.
\item Благодарственные молитвы по Святому Причащению.
\item Величание преподобному Александру Свирскому.
\item Канон за болящего.
\item Канон преподобному Паисию Великому об избавлении от мук умерших без покаяния.
\item Моление о том, чтобы Бог даровал нам усердие к молитве за усопших и принял бы её.
\item Молитва cвятителю Нектарию Эгинскому.
\item Молитва «Вопль к Богоматери».
\item Молитва Пресвятой Богородице в честь Ее иконы «Казанская».
\item Молитва архистратигу Михаилу.
\item Молитва великомученику Георгию Победоносцу.
\item Молитва за всякого усопшего.
\item Молитва за усопших внезапною (скоропостижною) смертию.
\item Молитва мученику Уару.
\item Молитва последних Оптинских старцев на начало дня.
\item Молитва преподобному Моисею Мурину.
\item Молитва св. праведного Иоанна Кронштадского об укреплении в Православной вере и единстве.
\item Молитва святителю Тихону, Патриарху Московскому и Всея Руси.
\item Молитва священномученику Ипатию, епископу Гангрскому.
\item Правило от осквернения.
\item Пяточисленные молитвы.
\item Раздел БОЖЕСТВЕННАЯ ЛИТУРГИЯ: О Литургии, Литургия оглашенных и Литургия верных.
\item Раздел ЕДИНОВЕРЧЕСКИЕ МОЛИТВЫ: Келейное правило, Павечерица великая и малая.
\item Тропарь преподобному Иоанну Дамаскину.
\item Тропарь святителю Спиридону, Тримифунтскому чудотворцу.
\item Тропарь священномученику Ипатию, епископу Гангрскому.

\end{itemize}

Добавлены тексты:

\begin{itemize}

\item Последование по исходе души от тела.
\item В раздел «Молитвы» добавлена отдельно молитва Честному Кресту (Да воскреснет Бог\ldots).
\item Молитва «При немилосердии и раздражении на ближнего» отделена от текста предыдущей молитвы.
\item В Последование ко Святому Причащению добавлена 4-я молитва святого Симеона Метафраста («Яко на Страшнем Твоем и нелицеприемнем предстояй Судилищи\ldots»).

\end{itemize}

\mysubtitle{Версия 20130227}

Обновление текстов от 27-го февраля 2013

\begin{itemize}

\item Незначительные изменения в оформлении после переезда сайта на новый домен \texttt{molitvoslov.org}.
\item Исправлена опечатка в Благодарственных молитвах по Святом Причащении.
\item Исправлена опечатка в Литургии верных.
\item Добавлено житие блаженной Ксении Петербургской.
\item Исправление в Акафисте за единоумершего.

\end{itemize}

\mysubtitle{Версия 20121214}

\begin{itemize}

\item Обновление текстов с сайта от 14-го декабря 2012 г.
\item Исправлен русско-язычный поиск, копирование и вставка из PDF в программе Adobe Acrobat Reader.
\item В текст Великого Канона включены припевы и добавлен русский перевод.

\item В текстах Божественной Литургии, канонах и молитвах добавлены положенные между стихами припевы, которые довольно часто из экономии места опускаются в печатных или интернетных изданиях, что может сбить с толку начинающих. По этой же причине часто употребляемые молитвы "--- «Слава:», «И ныне:», «Трисвятое по Отче наш», «Приидите, поклонимся...» и другие "--- приводятся каждый раз полностью, а не обозначаются сокращённо.

\item Исправлены опечатки (большая часть которых была обнаружена посетителями сайта) либо добавлены недостающие фрагменты молитв или изображения икон в следующих текстах:
\begin{itemize}
\item Молитва пред иконою «Неупиваемая Чаша».
\item Молитва пред иконою «Почаевская».
\item Пресвятой Богородице в честь Ее иконы «Казанская».
\item Молитвы задержания.
\item Молитва против антихриста.
\item Молитва о спасении державы Российской и утолении в ней раздоров и нестроений.
\item Тропарь Кресту и молитва за отечество (Молитвы утренние).
\item Молитвы на сон грядущим.
\item Последование ко Святому Причащению.
\item Единоверческие молитвы.
\item Молитва по исходе души из тела.
\item Правило от осквернения.
\item Молитва перед учением.
\item Молитва после учения.
\item Апостолу Симону Зилоту.
\item Святому праведному Иоанну, Кронштадтскому чудотворцу.
\item Молитвы ко Господу Св. Иоанна Кронштадского.
\item Преподобному Серафиму Саровскому.
\item Святым равноапостольным Кириллу и Мефодию.
\item Святой равноапостольной княгине Ольге.
\item Преподобному Даниилу Московскому.
\item Святым равноапостольным Константину и Елене.
\item Святителю Иоанну Златоустому, архиепископу Константинопольскому.
\item Поучение святителя Игнатия Брянчанинова о молитвенном правиле.
\item Молитвы ко Господу Иисусу Христу.
\item Молитва на сон.
\end{itemize}

\end{itemize}

\mysubtitle{Версия 20111108}

Обновление текстов с сайта от 8-го ноября 2011 г.

\begin{itemize}

\item Добавлен раздел «Утешение сродникам самоубийц».
\item Исправлены опечатки в следующих текстах:

\begin{itemize}

\item «Богородичное правило».
\item «Канон за болящего, глас 3-й».
\item Молитва «Пророку, Предтече и Крестителю Господню Иоанну».
\item Молитва «Святителю Димитрию, Ростовскому чудотворцу».
\item Молитва «Святой блаженной Ксении Петербургской».
\item «Молитва покаянная, читаемая в церквах России во дни смуты».
\item «КАНОН ПРЕПОДОБНОМУ ПАИСИЮ ВЕЛИКОМУ ОБ ИЗБАВЛЕНИИ ОТ МУК УМЕРШИХ БЕЗ ПОКАЯНИЯ».
\item Псалом 50-й, в разделе «ЧАСЫ ТРЕТИЙ И ШЕСТОЙ».
\item Псалом 50-й, в «Молитвы утренние».
\item Псалом 50-й и 26-й, в разделе «ЕДИНОВЕРЧЕСКИЕ МОЛИТВЫ».
\item Псалом 26-й, в разделе «МОЛИТВЕННОЕ ПРАВИЛО ПРЕПОДОБНОГО АМВРОСИЯ OПТИНСКОГО».
\item Молитва «Святому преподобному Иосифу Волоцкому».
\item «Молитва от осквернения».
\item Молитва «Великомученику Георгию Победоносцу».
\end{itemize}

\end{itemize}

\mysubtitle{Версия 20110709}

Обновление текстов с сайта от 9 июля 2011 г.

\begin{itemize}

\item Добавлен новый раздел «О Молитве», а также «Словарь и термины» (около 2500 слов и выражений).

\item Небольшое изменение в разделе «Божественная литургия».

\item Добавлены заглавия тропарей, Богородичных, канонов и песен в

\begin{itemize}

\item «Канон Ангелу Хранителю» («Молитвы»).

\item «Канон молебный ко Господу Иисусу Христу и Пречистой Богородице Матери Господни при разлучении души от тела всякого правоверного» («НАПУТСТВИЕ ХРИСТИАНИНА ПЕРЕД СМЕРТЬЮ И ЗАУПОКОЙНЫЕ МОЛИТВЫ»).

\item в разделе «КАНОН ПРЕПОДОБНОМУ ПАИСИЮ ВЕЛИКОМУ ОБ ИЗБАВЛЕНИИ ОТ МУК УМЕРШИХ БЕЗ ПОКАЯНИЯ».

\end{itemize}
\end{itemize}

\mysubtitle{Версия 20110616}

Обновление текстов с сайта от 16 июня 2011 г.
\begin{itemize}

\item В разделе «РАЗЛИЧНЫЕ МОЛИТВЫ»
\begin{itemize}
\item Исправлены опечатки в «Молитвах ко Господу Св. Иоанна Кронштадского».
\end{itemize}

\item В разделе «МОЛИТВЫ СВЯТЫМ»
\begin{itemize}
\item Убраны повторы молитв прп. Александру Свирскому и прп. Серафиму Саровскому.
\item Добавлены молитвы св. первомученику архидиакону Стефану; св. царице Грузинской Тамаре; святителю Иоанну Шанхайскому и Сан-Францисскому.
\item Исправлены опечатки в молитвах св. Царю Страстотерпцу Николаю; св. прп. Силуану Афонскому; cвятителю Тихону, Патриарху Московскому и Всея Руси.
\end{itemize}

\item В разделе «МОЛИТВЫ В СКОРБЯХ И ИСКУШЕНИЯХ ТВОРИМЫЕ»
\begin{itemize}
\item Добавлена «Молитва преследуемого человеками (свт. Игнатия Брянчанинова)».
\end{itemize}

\item В разделе «МОЛИТВЫ ЗА БОЛЯЩИХ»
\begin{itemize}
\item Исправлена опечатка в «Каноне за болящего, глас 3-й», Песнь 9.
\end{itemize}

\item В разделе «НАПУТСТВИЕ ХРИСТИАНИНА ПЕРЕД СМЕРТЬЮ И ЗАУПОКОЙНЫЕ МОЛИТВЫ»
\begin{itemize}
\item Исправлен «Чин литии, совершаемой мирянином дома и на кладбище».
\item Исправлен тропарь прп. Паисию Великому в молитве «Об ослаблении вечных мук умерших без покаяния», а также в разделе «КАНОН ПРЕПОДОБНОМУ ПАИСИЮ ВЕЛИКОМУ ОБ ИЗБАВЛЕНИИ ОТ МУК УМЕРШИХ БЕЗ ПОКАЯНИЯ».
\item Исправлен «Канон молебный ко Господу Иисусу Христу и Пречистой Богородице Матери Господни при разлучении души от тела всякого правоверного».
\end{itemize}

\item В разделе «В ПОМОЩЬ КАЮЩЕМУСЯ»
\begin{itemize}
\item Исправлены опечатки в «Десяти заповедях».
\end{itemize}


\end{itemize}


\mysubtitle{Версия 20110503.02}

Содержимое сайта от 03 мая 2011 г.

\mychapterending
%\normalfont\nopagebreak\bigskip\bigskip\begin{center}\includegraphics[width=0.15\textwidth]{uzor_end_4}\end{center}

