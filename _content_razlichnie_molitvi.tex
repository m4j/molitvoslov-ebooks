

\mypart{РАЗЛИЧНЫЕ МОЛИТВЫ}\label{_content_razlichnie_molitvi}
%http://www.molitvoslov.org/content/razlichnie_molitvi



\mychapter{Призывание помощи Духа Святаго на всякое дело доброе}\begin{mymulticols}
%http://www.molitvoslov.org/text888.htm



\mysubtitle{Тропарь, Глас 4}


Творче и Создателю всяческих, Боже, дела рук наших, ко славе Твоей начинаемая, Твоим благословением спешно исправи, и нас от всякаго зла избави, яко един всесилен и Человеколюбец.


\mysubtitle{Кондак, Глас 3}


Скорый в заступление и крепкий в помощь, предстани благодатию силы Твоея ныне, и благословив укрепи, и в совершение намерения благаго дела рабов Твоих произведи: вся бо, елика хощеши, яко сильный Бог творити можеши.

\end{mymulticols}

\mychapterending


\mychapter{Молитвы ко Господу Иисусу Христу}\begin{mymulticols}
%http://www.molitvoslov.org/content/Molitvy-ko-Gospodu-Iisusu-Khristu



\mysubtitle{Молитва первая}


Господи Иисусе Христе, Боже мой, покрый мя и раб твоих \myemph{(имена)} от злобы супостата нашего, зане сила его крепка, естество же наше страстно и сила наша немощна. Ты убо, о Благий, сохрани мя от смущения помыслов и потопа страстей. Господи, Пресладкий мой Иисусе, помилуй и спаси мя и рабов Твоих \myemph{(имена)}.


\mysubtitle{Молитва вторая}


О, Господи, Иисусе Христе! Не отврати лица Твоего от нас, рабов твоих \myemph{(имена)} и уклонися гневом от рабов Твоих: помошник нам буди, не отрини нас и не остави нас.


\mysubtitle{Молитва третья}


Помилуй мя, Господи, и не даждь мне погибнуть! Помилуй мя, Господи, яко немощен есмь! посрами, Господи, борющаго мя беса. Упование мое, осени над главою моею в день брани бесовския! борющаго мя врага побори, Господи, и обуревающия мя помыслы укроти тишиною Твоею, Слове Божий!


\mysubtitle{Молитва четвертая}


Господи! Се сосуд Твой есмь: наполни мя дарованиями Духа Твоего Святого, без Тебя я пуст всякого блага, или паче полн всякого греха. Господи! Се корабль Твой есмь: исполни мя грузом добрых дел. Господи! Се ковчег Твой: исполни его не прелестью сребролюбия и сластей, а любовию к Тебе и к одушевленному образу Твоему "--- человеку.


\end{mymulticols}

\mychapterending


\mychapter{Молитва водителя}\begin{mymulticols}
%http://www.molitvoslov.org/text935.htm



Боже, Всеблагий и Всемилостивый, всех охраняяй Своею милостию и человеколюбием, смиренно молю Тя, предстательством Богородицы и всех святых, сохрани от внезапной смерти и всякой напасти мене, грешнаго, и вверенных мне человек и помози невредимых доставлять каждого по его потребе.


Боже Милостивый! Избави мене от злаго духа лихачества, нечистой силы пианства, вызывающих несчастия и внезапную смерть без покаяния.


Сподоби мене, Господи, с чистой совестию дожить до глубокой старости без бремени убитых и искалеченных по моему нерадению людей, и да прославится имя Твое Святое, ныне и присно, и во веки веков. Аминь.

\end{mymulticols}

\mychapterending


\mychapter{О путешествующих}\begin{mymulticols}
%http://www.molitvoslov.org/text887.htm



\mysubtitle{Тропарь, глас 2}


Путь и истина сый, Христе, спутника Ангела Твоего рабом Твоим ныне, якоже Товии иногда, посли сохраняюща, и невредимых, к славе Твоей, от всякаго зла во всяком благополучии соблюдающа, молитвами Богородицы, Едине Человеколюбче.


\mysubtitle{Кондак, глас 2}


Луце и Клеопе во Еммаус спутешествовавый, Спасе, сшествуй и ныне рабом Твоим, путешествовати хотящим, от всякаго избавляя их злаго обстояния: вся бо Ты, яко Человеколюбец, можеши хотяй.


\mysubtitle{Молитва}


Господи Иисусе Христе Боже наш, истинный и живый путю, состранствовати мнимому Твоему отцу Иосифу и Пречистей Ти Деве Матери во Египет изволивый, и Луце и Клеопе во Еммаус спутешествовавый! И ныне смиренно молим Тя, Владыко Пресвятый, и рабом Твоим сим Твоею благодатию спутешествуй. И якоже рабу Твоему Товии, Ангела Хранителя и наставника посли, сохраняюща и избавляюща их от всякаго злаго обстояния видимых и невидимых врагов, и ко исполнению заповедей Твоих наставляюща, мирно же и благополучно и здраво препровождающа, и паки цело и безмятежно возвращающа; и даждь им все благое свое намерение ко благоугождению Твоему благополучно в славу Твою исполнити. Твое бо есть, еже миловати и спасати нас, и Тебе славу возсылаем со Безначальным Твоим Отцем и со Пресвятым и Благим и Животворящим Твоим Духом, ныне и присно и во веки веков. Аминь.


\mysubtitle{Молитва ко Пресвятой Богородице «Скоропослушнице»}


Преблагословенная Владычице, Приснодево Богородице, Бога Слова паче всякаго слова на спасение наше рождшая, и благодать Его преизобильно паче всех приявшая, море явльшаяся Божественных дарований и чудес приснотекущая река, изливающая благость всем, с верою к Тебе прибегающим! Чудотворному Твоему образу припадающе, молимся Тебе всещедрей Матери Человеколюбиваго Владыки: удиви на нас пребогатыя милости Твоя и прошения наша, приносимая Тебе, Скоропослушнице, ускори исполнити все, еже на пользу во утешение и спасение, коемуждо устрояющи. Посети, Преблагая, рабы Твоя благодатию Твоею, подаждь недугующим цельбу и совершенное здравие, обуреваемым тишину, плененным свободу и различными образы страждущих утеши. Избави, Всемилостивая Госпоже, всяк град и страну от глада, язвы, труса, потопа, огня, меча и иныя казни временныя и вечныя, Матерним Твоим дерзновением отвращающи гнев Божий: и душевнаго расслабления, обуревания страстей и грехопадений свободи рабы Твоя, яко да непреткновенно во всяком благочестии поживше в сем веце, и в будущем вечных благ сподобимся благодатию и человеколюбием Сына Твоего и Бога, Емуже подобает всякая слава, честь и поклонение со Безначальным Его Отцем и Пресвятым Духом ныне и присно и во веки веков. Аминь.

\end{mymulticols}

\mychapterending


\mychapter{Молитвы ко Господу Св. Иоанна Кронштадского}\begin{mymulticols}
%http://www.molitvoslov.org/content/Molitvy-ko-Gospodu-Sv-Ioanna-Kronshtadskogo



\mysubtitle{Молитва первая}


Господи! Имя Тебе Любовь: не отвергни меня, заблуждающегося человека! Имя Тебе "--- Сила: подкрепи меня изнемогающего и падающего! Имя Тебе "--- Свет: просвети мою душу, омраченную житейскими страстями! Имя Тебе "--- Мир: умири мятущуюся душу мою! Имя Тебе "--- Милость: не переставай миловать меня.


\mysubtitle{Молитва вторая}


Господи, Ты Судия земли, Ты не любишь неправду, услыши молитву мою недостойную и пошли силу Твою, и пошли помощь Твою там, где встретят меня враги видимые и невидимые, и чтобы стали столпы на месте том, где их встретит сила Твоя. Аминь.


\mysubtitle{Молитва третья}


Тебе, Господи, единому Благому и Непамятозлобному, исповедую грехи моя; Тебе припадаю, вопия, недостойный: согреших, Господи, согреших, и несмь достоин воззрети на высоту небесную от множества неправд моих. Но, Господи мой, Господи, даруй ми слезы умиления, единый Блаже и Милостивый, яко да ими Тя умолю, очиститися прежде конца от всякаго греха: страшно бо и грозно место имам пройти, тела разлучився, и множество мя мрачное и безчеловечное демонов срящет, и никтоже в помощь спутствуяй, или избавляяй. Тем припадаю Твоей благости, не предаждь обидящим мя, ниже да похвалятся о мне врази мои, Благий Господи, ниже да рекут: в руки наша пришел еси и нам предан еси. Ни, Господи, не забуди щедрот Твоих и не воздаждь ми по беззаконием моим, и не отврати лица Твоего от мене; но Ты, Господи, накажи мя, обаче милостию и щедротами, враг же мой да не возрадуется о мне, но угаси его на мя прещения и все упраздни его действо. И даждь ми к Тебе путь неукорный, Благий Господи, занеже, и согрешив, не прибегох ко иному врачу и не прострех руки моея к богу чуждему. Не отрини убо моления моего, но услыши мя Твоею благостию и утверди мое сердце страхом Твоим; и да будет благодать Твоя на мне, Господи, яко огнь попаляяй нечистыя во мне помыслы. Ты бо еси, Господи, свет паче всякаго света, радость паче всякия радости, упокоение паче всякаго упокоения, жизнь истинная и спасение, пребывающее во веки веков. Аминь.

\end{mymulticols}

\mychapterending


\mychapter{Молитва св. праведного Иоанна Кронштадского об укреплении в Православной вере и единстве}\begin{mymulticols}
%http://www.molitvoslov.org/text650.htm




«Господи, утверди в вере сей сердце мое и сердца всех православных христиан; сей веры и сего чаяния жити достойно вразуми; соедини в вере сей все великие общества христианские, бедственно отпадшие от единства Святой Православной Кафолической Церкви, яже есть Тело Твое и ее же Глава еси Ты и Спаситель Тела; низложи гордыню и противление учителей их и последующих им; даруй им сердцем уразуметь истину и спасительность Церкви Твоей и нелестно ей соединиться; совокупи Твоей Святой Церкви и недугующих невежеством, заблуждением и упорством раскола, сломив силою благодати Духа Твоего упорство их и противление Истине Твоей, да не погибнут люте в своем противлении, якоже Корей, Дафан и Авирон, противившиеся Моисею и Аарону, рабам Твоим. К сей вере привлецы все языцы, населяющие землю, да единым сердцем и едиными усты все языцы прославят Тебя, Единого всех Бога и Благодетеля; в сей вере и нас всех соедини духом кротости, смирения, незлобия, простоты, бесстрастия, терпения, долготерпения, милосердия, соболезнования и сорадования». АМИНЬ.




\end{mymulticols}

\mychapterending


\mychapter{Песнь хвалебная святого Амвросия, епископа Медиоланского}\begin{mymulticols}
%http://www.molitvoslov.org/text543.htm




Тебе Бога хвалим, Тебе Господа исповедуем, Тебе Превечнаго Отца вся земля величает; Тебе вси aнгели, Тебе небеса и вся Силы, Тебе Херувими и Серафими непрестанными гласы взывают: Свят, Свят, Свят, Господь Бог Саваоф, полны суть небеса и земля величества славы Твоея, Тебе преславный Апостольский лик, Тебе пророческое хвалебное число, Тебе хвалит пресветлое мученическое воинство, Тебе по всей вселенней исповедует Святая Церковь, Отца непостижимаго величества, покланяемаго Твоего истиннаго и Единороднаго Сына и Святаго Утешителя Духа. Ты, Царю славы, Христе, Ты Отца Присносущный Сын еси: Ты, ко избавлению приемля человека, не возгнушался еси Девическаго чрева; Ты, одолев смерти жало, отверзл еси верующим Царство Небесное. Ты одесную Бога седиши во славе Отчей, Судия приити веришися. Тебе убо просим: помози рабом Твоим, ихже Честною Кровию искупил еси. Сподоби со святыми Твоими в вечной славе Твоей царствовати. Спаси люди Твоя, Господи, и благослови достояние Твое, исправи я и вознеси их во веки; во вся дни благословим Тебе и восхвалим имя Твое во век и в века века. Сподоби, Господи, в день сей без греха сохранитися нам. Помилуй нас, Господи, помилуй нас: буди милость Твоя, Господи, на нас, якоже уповахом на Тя. На Тя, Господи, уповахом, да не постыдимся во веки. Аминь.




\end{mymulticols}

\mychapterending


\mychapter{Mолитва по соглашению}\begin{mymulticols}
%http://www.molitvoslov.org/text542.htm



Господи, Иисусе Христе, Сыне Божий, Ты бо рекл еси пречистыми усты Твоими: «Аминь глаголю вам, яко аще двое от вас совещаются на земли всякой вещи, ее же аще просите, будете иметь от Отца Моего, Иже на Небесех: где же два или трое собрались во имя Мое, ту есмь Аз посреде их». Непреложны словеса Твоя, Господи, милосердие Твое безприкладно и человеколюбию Твоему несть конца. Сего ради молим Тя: даруй нам, рабам Твоим \myemph{(имена)}, согласившимся просить Тя \myemph{(просьба)}, исполнения нашего прошения. Но обаче не якоже мы хотим, но якоже Ты. Да будет во веки воля Твоя. Аминь.

\end{mymulticols}

\mychapterending


\mychapter{Молитва иерея (Прав. Иоанна Кронштадского)}\begin{mymulticols}
%http://www.molitvoslov.org/text546.htm




О Дражайший и Сладчайший Иисусе мой! Благодарю Тебя, что Ты меня недостойнейшего облек великою благодатью священства! Но прости, Иисусе мой, что я не усугубил дары благодати Твоея и погубил драгоценное время жизни моея! Даруй мне, Сладчайший Иисусе мой, поне от ныне положить начало благое: ежедневно нудить себя к Царствию Небесному и ежедневно усвоять умом, сердцем и жизнию святые Твоя истины. Соделай меня недостойнейшего истинным, добрым и мудрым пастырем и руководителем народа Твоего; приложи мне веру крепкую и непоколебимую и облеки меня во глубину святаго смирения. Ими же веси судьбами спаси меня недостойного, сродников по плоти и чад моих духовных. Аминь.

\end{mymulticols}

\mychapterending


\mychapter{Молитвы иерея (св. Иннокентия, митрополита Московского)}\begin{mymulticols}
%http://www.molitvoslov.org/text545.htm




Иисусе Слове Божий! Даждь разум и слово мне, Твоему слабейшему служителю, дабы я мог и ныне и всегда достойно возвещать слово Царствия Твоего, и коснись Твоею благодатью сердец наших, дабы всякое слово Твое доходило до нашего сознания сердечного.


Господи Иисусе Христе Боже мой, молитвами Пречистыя Твоея Матери, святых небесных Сил и всех Святых приими мою грешную и недостойную молитву за всех моих чад духовных, живых и умерших. Услыши молитву мою и даруй всем милость Твою: живых спаси и соблюди в мире и благосостоянии, усопшим остави грехи и даруй им вечный покой и бесконечную радость.




\end{mymulticols}

\mychapterending


\mychapter{Молитва неграмотного после причащения Святых Таин (прот. И. Европейцева)}\begin{mymulticols}
%http://www.molitvoslov.org/text544.htm




Господи Иисусе Христе Пресладкий мой Искупителю, я чувствую, что Твоего пресвятаго Тела и Крови я недостоин, но по Твоей благости и я принял Твою Чашу, как мои братья: благодарю Тебя от всего сердца за Твою Небесную Милость ко мне и благодать. Молю Тебя, Господи, да будет мне сие приобщение в очищение грехов и здравие тела, в исправление жизни и будущее вечное блаженство.

\end{mymulticols}

\mychapterending


\mychapter{Молитва в день рождения}\begin{mymulticols}
%http://www.molitvoslov.org/text539.htm




Господи Боже, Владыка всего мира видимого и невидимого. От Твоей святой воли зависят все дни и лета моей жизни. Благодарю Тебя, премилосердный Отче, что Ты дозволил мне прожить еще один год; знаю, что по грехам моим я недостоин этой милости, но Ты оказываешь мне ее по неизреченному человеколюбию Твоему. Продли и еще милости Твои мне, грешному; продолжи жизнь мою в добродетели, спокойствии, в здравии, в мире со всеми сродниками и в согласии со всеми ближними. Подай мне изобилие плодов земных и все, что к удовлетворению нужд моих потребно. Наипаче же очисти совесть мою, укрепи меня на пути спасения, чтобы я, следуя по нему, после многолетней в мире сем жизни, перейдя в жизнь вечную, удостоился быть наследником Царства Твоего Небесного. Сам, Господи, благослови начинаемый мною год и все дни жизни моей. Аминь.

\end{mymulticols}

\mychapterending


\mychapter{Молитва в день Hового года}\begin{mymulticols}
%http://www.molitvoslov.org/text538.htm




Господи Боже, всех видимых и невидимых тварей, Творец и Зиждитель, сотворивший времена и лета, Сам благослови начинающийся сего дня Новый Год, который мы считаем от воплощения Твоего для нашего спасения. Дозволь нам провести сей год и многие по нем в мире и согласии с ближними нашими; укрепи и распространи Святую Вселенскую Церковь, которую Ты Сам основал, и спасительною жертвою святoго Тела и пречистой Крови освятил. Отечество наше возвыси, сохрани и прославь; долгоденствие, здравие, изобилие плодов земных и благорастворение воздухoв дай нам; меня, грешного раба Твоего, всех родственников и ближних моих и всех благоверных христиан, как истинный наш верховный Пастырь, упаси, огради и на пути спасения утверди, чтобы мы, следуя по нему после долговременной и благополучной жизни в мире сем, достигнули Царства Твоего Небесного и удостоились вечного блаженства со святыми Твоими. Аминь.

\end{mymulticols}

\mychapterending


\mychapter{Молитва перед чтением духовных книг}\begin{mymulticols}
%http://www.molitvoslov.org/text534.htm




Господи, Иисусе Христе, открой мои очи сердечные, чтобы я, услыша Слово Твое, уразумел оное и исполнил волю Твою. Не скрой от меня заповедей Твоих, но отверзи очи мои, чтобы я уразумел чудеса от закона Твоего. Скажи мне безвестное и тайное премудрости Твоей! На Тебя уповаю, Боже мой, и верую, что Ты просветишь ум мой и смысл светом разума Твоего и что тогда я не только прочту написанное, но и исполню оное. Соделай, чтобы я не в грех себе Жития Святых и Слово Твое прочитал, но во обновление и просвещение, и в святыню, и во спасение души, и в наследие жизни вечной. Ибо Ты, Господи, просвещение лежащих во тьме и от Тебя есть всякое даяние благое и всякий дар совершенный. Аминь.

\end{mymulticols}

\mychapterending


\mychapter{Молитва на освящение всякой вещи (священником)}\begin{mymulticols}
%http://www.molitvoslov.org/text533.htm




Создателю и Содетелю человеческаго рода, Дателю благодати духовныя, Подателю вечнаго спасения, Сам, Господи, посли Духа Твоего Святаго с вышним благословением на вещь сию, яко да вооружена силою небеснаго заступления хотящим ю употребляти, помощна будет к телесному спасению и заступлению и помощи, о Христе Иисусе Господе нашем. Аминь.


\myemph{(И кропить вещь святой водой трижды).}




\end{mymulticols}

\mychapterending


\mychapter{Молитва любимому святому}\begin{mymulticols}
%http://www.molitvoslov.org/text537.htm



Угодниче Божий \myemph{(имярек)}. Поминай в благоприятных твоих молитвах перед Христом Богом, да сохранит Он нас от искушений, болезней и скорбей, да дарует нам смирение, любовь, рассуждение и кротость, и да сподобит Он нас, недостойных, Царствия Своего. Аминь.




\end{mymulticols}

\mychapterending


\mychapter{Молитва перед выходом из дома}\begin{mymulticols}
%http://www.molitvoslov.org/text532.htm





Отрицаюся тебе, сатано, гордыни твоей и служению тебе, и сочетаюся Тебе, Христе, во имя Отца и Сына и Святаго Духа. Аминь. \myemph{(И оградить себя крестным знамением).}




\end{mymulticols}

\mychapterending


\mychapter{Молитва на вхождение в новый дом}\begin{mymulticols}
%http://www.molitvoslov.org/text541.htm



Боже Спасителю наш, изволивый под сень Закхееву внити и спасение тому и всему дому того бывый, Сам и ныне зде жити восхотевшия, и нами, недостойными мольбы Тебе и моления приносящия, от всякаго вреда соблюди невредимы, благословляя тех зде жилище, и ненаветен тех живот сохраняяй. Аминь.




\end{mymulticols}

\mychapterending


\mychapter{Две утренние малоизвестные молитвы}\begin{mymulticols}
%http://www.molitvoslov.org/text536.htm



Тебе, Богу и Творцу моему, в Троице Святей славимому Отцу и Сыну и Святому Духу, поклоняюся и вручаю душу и тело мое, и молюся: Ты мя благослови, Ты мя помилуй, и от всякаго мирскаго, диавольскаго и телеснаго зла избави. И даждь в мире без греха прейти день сей, в славу Твою, и во спасение души моея. Аминь.


Слава Тебе, Царю, Боже Вседержителю, Иже Божественным Твоим и человеколюбным промыслом, сподобил мя еси, грешнаго и недостойнаго, от сна встати и получити вход святаго дому Твоего: прими, Господи, и глас моления моего, якоже святых и умных Твоих сил, и благоволи сердцем чистым, и духом смиренным приносити Тебе хвалу от скверных устен моих, яко да и аз общник буду мудрым девам, со светлою свещею души моея, и славлю Тя во Отце и Дусе славимаго Бога Слова. Аминь.




\end{mymulticols}

\mychapterending


\mychapter{Молитвы ко Пресвятой Богородице от человека, собирающегося в путь}\begin{mymulticols}
%http://www.molitvoslov.org/text531.htm

\myfigure{397}


О, Пресвятая Владычице моя, Дево Богородице, Одигитрие, покровительнице и упование спасения моего! Се в путь, мне предлежащий, ныне хощу отлучитися и на время сие вручаю Тебе, премилосердой Матери моей, душу и тело мое, вся умныя моя и вещественныя силы, всего себе вверяя в крепкое Твое смотрение и всесильную Твою помощь. О, благая Спутнице и Защитнице моя! Усердно молю Тя, да не ползок путь мой сей будет, руководствуй мя на нем, и направи его, Всесвятая Одигитрие, якоже Сама веси, ко славе Сына Твоего, Господа моего Иисуса Христа, буди ми во всем помощнице, наипаче же в сем дальнем и многотрудном путешествии соблюди мя под державным покровом Твоим от всяких находящих бед и скорбей, от враг видимых и невидимых, и моли о мне, Госпоже моя, Сына Твоего Христа Бога нашего, да послет в помощь мне Ангела Своего мирна, верна наставника и хранителя, да якоже древле даровал есть рабу Своему Товии Рафаила, на всяком месте и во всякое время хранивша его в пути от всякаго зла: тако и мой путь благополучно управив и сохранив мя небесною силою,"--- здрава да возвратит мя, мирна и всецела к жилищу моему во славу имени Своего Святаго, славяща и благословяща Его во вся дни живота моего и Тебе величающа ныне и присно, и во веки веков. Аминь.

\end{mymulticols}

\mychapterending


\mychapter{Молитва на сон}\begin{mymulticols}
%http://www.molitvoslov.org/text540.htm



И даждь нам, Владыко, на сон грядущим, покой тела и души, и сохрани нас от мрачнаго сна греховнаго, и всякаго темнаго и нощнаго сладострастия. Утиши стремление страстей, и угаси разжженныя стрелы лукаваго, яже на ны льстиво движимыя. Плоти нашея востания утоли, и всяко земное и вещественное наше мудрование успи. И даруй нам, Боже, бодр ум, целомудр помысл, сердце трезвящеeся, сон легок, и всякаго сатанина мечтания изменен. Возстави же нас во время молитвы, утверждены в заповедех Твоих, и память судеб Твоих в себе тверду имуща. Всенощное славословие нам даруй, во еже пети и благословити и славити пречестное и великолепное имя Твое, Отца и Сына и Святаго Духа, ныне и присно и во веки веков. Aминь.




\end{mymulticols}

\mychapterending


\mychapter{Молитва против антихриста}
%http://www.molitvoslov.org/text535.htm

{\centering \myemph{(составленная оптинским старцем схииеромонахом Анатолием Потаповым)}\par}

\begin{mymulticols}

Избави мя, Господи, от обольщения богомерзкого и злохитрого антихриста, близгрядущего, и укрой меня от сетей его в сокровенной пустыне Твоего спасения. Даждь ми, Господи, крепость и мужество твердаго исповедания имени Твоего святого, да не отступлю страха ради дьявольского, да не отрекусь от Тебя, Спасителя и Искупителя моего, от Святой Твоей Церкви. Но даждь мне, Господи, день и ночь плач и слезы о грехах моих, и пощади мя, Господи, в час Страшного Суда Твоего. Аминь.

\end{mymulticols}

\mychapterending


\mychapter{Молитва на благословение пищи и пития мирянам}
%http://www.molitvoslov.org/text530.htm




Господи, Иисусе Христе, Боже наш, благослови нам пищу и питие молитвами Пречистыя Твоея Матере и всех святых Твоих, яко благословен во веки веков. Аминь. \myemph{(И перекрестить пищу и питие.)}






\mychapterending


\mychapter{Перед вкушением пищи}\begin{mymulticols}
%http://www.molitvoslov.org/text529.htm



Oтче наш, Иже еси на небесех! Да святится имя Твое, да приидет Царствие Твое, да будет воля Твоя, яко на небеси и на земли. Хлеб наш насущный даждь нам днесь; и остави нам долги наша, якоже и мы оставляем должником нашим; и не введи нас во искушение, но избави нас от лукаваго.


\myemph{Или:} Очи всех на Тя, Господи, уповают, и Ты даеши им пищу во благовремении, отверзаеши Ты щедрую руку Твою и исполняеши всякое животное благоволения.

\end{mymulticols}

\mychapterending


\mychapter{После вкушения пищи}
%http://www.molitvoslov.org/text527.htm



Благодарим Тя, Христе Боже наш, яко насытил еси нас земных Твоих благ; не лиши нас и Небеснаго Твоего Царствия, но яко посреде учеников Твоих пришел еси, Спасе, мир даяй им, прииди к нам и спаси нас.


\mychapterending


\mychapter{Перед началом всякого дела}\begin{mymulticols}
%http://www.molitvoslov.org/text522.htm

\TsariuNebesnyj

Благослови', Го'споди, и помоги' мне, гре'шному, соверши'ть начина'емое мно'ю де'ло, во сла'ву Твою'.


Го'споди, Иису'се Христе', Сы'не Единоро'дный Безнача'льнаго Твоего' Отца', Ты бо рекл еси' пречи'стыми усты' Твои'ми, я'ко без Мене' не мо'жете твори'ти ничесо'же. Го'споди мой, Го'споди, ве'рою объе'м в ду'ши мое'й и се'рдце Тобо'ю рече'нная, припа'даю Твое'й бла'гости: помози' ми, гре'шному, сие' дело, мно'ю начина'емое, о Тебе' Само'м соверши'ти, во и'мя Отца' и Сы'на и Свята'го Ду'ха, моли'твами Богоро'дицы и всех Твои'х святы'х. Ами'нь.

\end{mymulticols}

\mychapterending


\mychapter{По окончании дела}\begin{mymulticols}
%http://www.molitvoslov.org/text523.htm




Исполнение всех благих Ты еси, Христе мой, исполни радости и веселия душу мою и спаси мя, яко един Многомилостив, Господи, слава Тебе.


Достойно есть яко воистинну блажити Тя Богородицу, Присноблаженную и Пренепорочную и Матерь Бога нашего. Честнейшую Херувим и славнейшую без сравнения Серафим, без истления Бога Слова рождшую, сущую Богородицу Тя величаем.

\end{mymulticols}

\mychapterending


\mychapter{Молитва последних Оптинских старцев на начало дня}\begin{mymulticols}
%http://www.molitvoslov.org/text521.htm



Господи, дай мне с душевным спокойствием встретить все, что даст мне сей день. Господи, дай мне вполне предаться воле Твоей Святой. Господи, на всякий час сего дня во всем наставь и поддержи меня. Господи, открой мне волю Твою для меня и окружающих меня. Господи, какие бы я ни получил известия в течение дня, дай мне принять их с покойной душой и твердым убеждением, что на все Святая воля Твоя. Господи, Великий, Милосердный, во всех моих делах и словах руководи моими мыслями и чувствами, во всех непредвиденных обстоятельствах не дай мне забыть, что все ниспослано Тобой. Господи, дай мне разумно действовать с каждым из ближних моих, никого не огорчая и никого не смущая. Господи, дай мне силу перенести утомления сего дня и все события в течение его. Руководи моею волею и научи молиться и любить всех нелицемерно. Аминь.

\end{mymulticols}

\mychapterending


\mychapter{На принятие просфоры и святой воды}\begin{mymulticols}
%http://www.molitvoslov.org/text525.htm

\myfigure[0.8]{389}


Господи Боже мой, да будет дар Твой святый святая Твоя просфора и святая Твоя вода во оставление грехов моих, в просвещение ума моего, в укрепление душевных и телесных сил моих, во здравие души и тела моего, в покорение страстей и немощей моих по безпредельному милосердию Твоему молитвами Пречистыя Твоея Матери и всех святых Твоих. Аминь.



\end{mymulticols}

\mychapterending


\mychapter{О даровании молитвы}
%http://www.molitvoslov.org/text520.htm




Научи мя, Господи, усердно молиться Тебе со вниманием и любовью, без которых молитва не бывает услышана! Да не будет у меня небрежной молитвы во грех мне!


\mychapterending


\mychapter{Молитва идущего в церковь}\begin{mymulticols}
%http://www.molitvoslov.org/text524.htm




Возвеселихся о рекших мне: в дом Господень пойдем. Аз же множеством милости Твоея, Господи, вниду в дом Твой, поклонюся ко храму святому Твоему в страсе Твоем. Господи, настави мя правдою Твоею, враг моих ради исправи пред Тобою путь мой; да без преткновения прославлю Едино Божество, Отца и Сына и Святаго Духа, ныне и присно и во веки веков. Аминь.

\end{mymulticols}

\mychapterending


\mychapter{О даровании кротости и смирения в служении ближним. Молитва св. Иоанна Кронштадтского.}\begin{mymulticols}
%http://www.molitvoslov.org/content/O-darovanii-krotosti-i-smireniya-v-sluzhenii-blizhnim-Molitva-sv-Ioanna-Kronshtadtskogo



О, кроткий и смиренный сердцем Творче, Жизнодавче, Искупителю, Кормителю и Хранителю наш, Господи Иисусе! Научи Ты нас любви, кротости и смирению Духом Твоим Святым и укрепи нас в сих достолюбезнейших Тебе добродетелях, да не надмевают нашего сердца дары Твои богатые, да ни мним мы, что мы питаем, довольствуем и поддерживаем коголибо: Ты "--- общий всех Кормилец "--- питаешь, довольствуешь и хранишь; все под крылами Твоея благости, щедрот и человеколюбия довольствуются и покоятся, а не под нашими, ибо мы сами имеем нужду укрываться в тени крыл Твоих,"--- каждое мгновение нашей жизни. Наши очи устремлены к Тебе, Богу нашему, якоже очи раб в руку господий, очи рабыни в руку госпожи своея, дондеже ущедриши нас. Аминь.

\end{mymulticols}

\mychapterending


\mychapter{В печали по Богу. Молитвенный плач прп. Силуана Афонского}\begin{mymulticols}
%http://www.molitvoslov.org/content/V-pechali-po-Bogu-Molitvennyi-plach-prp-Siluana-Afonskogo



Сердце мое возлюбило Тебя, Господи, и потому скучаю по Тебе и слезно ищу Тебя. Ты украсил небо звездами, воздух "--- облаками, землю же "--- морями, реками и зелеными садами, где поют птицы, но душа моя возлюбила Тебя и не хочет смотреть на этот мир, хотя он и прекрасен. Только Тебя желает душа моя, Господи. Приди, и вселись, и очисти меня от грехов моих. Ты видишь с высоты святой славы Твоей, как скучает душа моя по Тебе. Не оставь меня, раба Твоего; услышь меня, вопиющего, как пророк Давид: «Помилуй мя, Боже, по велицей Твоей милости».

\end{mymulticols}

\mychapterending


\mychapter{Молитвы ко Пресвятой Богородице}\begin{mymulticols}
%http://www.molitvoslov.org/content/Molitvy-ko-Presvyatoi-Bogoroditse



\mysubtitle{Молитва первая}


О Пресвятая Госпоже Владычице Богородице! Воздвигни нас, раб Божиих \myemph{(имена)} из глубины греховныя и избави нас от смерти внезапныя и от всякого зла. Подаждь, Госпоже, нам мир и здравие и просвети нам ум и очи сердечныя, еже ко спасению, и сподоби ны, грешныя рабы Твоя, Царствия Сына Твоего, Христа Бога нашего: яко держава Его благословенна со Отцем и Пресвятым Его Духом.


\mysubtitle{Молитва вторая}


Пресвятая Дево, Мати Господа, покажи на мне, убозем, и рабах Божиих \myemph{(имена)} древния милости Твоя: низпосли дух разума и благочестия, дух милосердия и кротости, дух чистоты и правды. Ей, Госпоже Пречистая! Милостива мне буди зде и на Страшнем Суде. Ты бо еси, Госпоже, слава небесных и упование земных. Аминь.


\mysubtitle{Молитва третья}


Нескверная, Неблазная, Нетленная, Пречистая, Неневестная Богоневесто, Богородице Марие, Госпоже мира и Надежде моя! Призри на мя, грешнаго, в час сей и Егоже из чистых кровей Твоих неискусомужно родила еси Господа Иисуса Христа, милостива мне соделай матерними Твоими молитвами; Того зревшая осужденна и оружием печали в сердце уязвившаяся, уязви душу мою Божественною любовию! Того в узах и поруганиях горце оплакавшая, слезы сокрушения мне даруй; при вольном Того ведении на смерть душею тяжце поболевшая, болезней мя свободи, да Тя славлю, достойно славимую во веки. Аминь.


\mysubtitle{Молитва четвертая}


Заступнице усердная, благоутробная Господа Мати! К Тебе прибегаю аз, окаянный и паче всех человек грешнейший: вонми гласу моления моего, и вопль мой и стенание услыши. Яко беззакония моя превзыдоша главу мою и аз, якоже корабль в пучине, погружаюся в море грехов моих. Но Ты, Всеблагая и Милосердая Владычице, не презри мене, отчаяннаго и во гресех погибающаго; помилуй мя, кающагося во злых делех моих, и обрати на путь правый заблуждшую окаянную душу мою. На Тебе, Владычице моя Богородице, возлагаю все упование мое. Ты, Мати Божия, сохрани и соблюди мя под кровом Твоим, ныне и присно и во веки веков. Аминь.


\mysubtitle{Молитва пятая}


Пресвятая Владычице Богородице, единая чистейшая душею и телом, единая превысшая всякой чистоты, целомудрия и девства, единая всецело соделавшаяся обителию всецелой благодати всясвятаго Духа, самыя невещественыя силы здесь еще несравненно превзошедшая чистотою и святынею души и тела, призри на мя мерзкаго, нечистаго, душу и тело очернившаго скверною страстей жизни моей, очисти страстный мой ум, непорочными соделай и благоустрой блуждающие и слепотствующие помыслы мои, приведи в порядок чувства мои и руководствуй ими, освободи меня от мучительствующаго надо мною злаго и гнуснаго навыка к нечистым предразсудкам и страстям, останови всякий действующий во мне грех, омраченному и окаянному уму моему даруй трезвение и разсудительность для исправления своих поползновений и падений, чтобы, освободившись от греховной тьмы, сподобился я с дерзновением прославлять и песнословить Тебя, единую Матерь истиннаго Света "--- Христа, Бога нашего; потому что Тебя одну с Ним и о Нем благословляет и славит всякая невидимая и видимая тварь ныне, и всегда, и во веки веков. Аминь.


\mysubtitle{Молитва шестая}


О Пресвятая Дево, Мати Господа Вышняго, Заступнице и Покрове всех, к Тебе прибегающих! Призри с высоты святыя Твоея на мя, грешнаго (имя), припадающаго ко пречистому образу Твоему; услыши мою теплую молитву и принеси ю пред Возлюбленнаго Сына Твоего, Господа нашего Иисуса Христа; умоли Его, да озарит мрачную душу мою светом Божественныя благодати Своея, да избавит мя от всякия нужды, скорби и болезни, да низпослет мне тихое и мирное житие, здравие телесное и душевное, да умирит страждущее сердце мое и исцелит раны его, да наставит мя на добрая дела, ум мой убо от помышлений суетных да очистит, исполнению же заповедей Своих мя научив, от вечныя муки да избавит и Своего Небеснаго Царствия да не лишит мя. О Пресвятая Богородице! Ты, «Всех скорбящих Радосте», услыши и мя, скорбнаго; Ты, именуемая «Утоление печали», утоли и мою печаль; Ты, «Купино Неопалимая», сохрани мир и всех нас от вредоносных огненных стрел врага; Ты, «Взыскание погибших», не попусти и мне погибнути в бездне грехов моих. На Тя бо по Бозе вся моя надежда и упование. Буди мне в жизни временней Заступница, и о жизни вечней пред Возлюбленным Сыном Твоим, Господем нашим Иисусом Христом, Ходатаица. Научи мя Тому убо с верою и любовию служити, Тебе же, Пресвятая Мати Божия, Преблагословенная Марие, благоговейно чтити до скончания дней моих. Аминь.

\end{mymulticols}

\mychapterending

\mychapter{Молитва прп. Нектария Оптинского}
%http://www.molitvoslov.org/content/Molitva-prp-Nektariya-Optinskogo



Господи Иисусе Христе Сыне Божий, грядый судити живых и мертвых помилуй нас, грешных, прости грехопадения всей нашей жизни, и имиже веси судьбами сокрый нас от лица антихриста в сокровенной пустыне спасения Твоего.


\mychapterending

