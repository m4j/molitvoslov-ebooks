\mypart{МОЛИТВЫ В СКОРБЯХ И ИСКУШЕНИЯХ ТВОРИМЫЕ}\label{_content_molitvi-vskorbyah}
%http://www.molitvoslov.org/content/molitvi-vskorbyah



\mychapter{Об избавлении от искушения}
%http://www.molitvoslov.org/content/ob-izbavlenii-viskusheniyah

\ifpdf
% optimization for a4 paper size page
{\centering\myemph{(преподобного Симеона Нового Богослова)}\par}
\else
\section{Преподобного Симеона Нового Богослова}
\fi
%http://www.molitvoslov.org/text625.htm 

\begin{mymulticols}

Не попусти на меня, Владыко, Господи, искушение, или скорби, или болезнь свыше силы моей, но избавь от них или даруй мне крепость перенести их с благодарностью.

\end{mymulticols}

\section{О ненавидящих и обидящих нас}\begin{mymulticols}
%http://www.molitvoslov.org/text587.htm 

\ifpdf
\myemph{\normalsize\textbf{Тропарь, глас 4-й:}}
\else
\mysubtitle{Тропарь, глас 4-й:}
\fi

О распенших Тя моливыйся, Любодушне Господи, и рабом Твоим о вразех молитися повелевый, ненавидящих и обидящих нас прости, и от всякаго зла и лукавства к братолюбному и добродетельному настави жительству, смиренно мольбу Тебе приносим; да в согласном единомыслии славим Тя, Единаго Человеколюбца.

\ifpdf
\myemph{\normalsize\textbf{Кондак, глас 5-й:}}
\else
\mysubtitle{Кондак, глас 5-й:}
\fi

Якоже первомученик Твой Стефан о убивающих его моляше Тя, Господи, и мы припадающе молим, ненавидящих всех и обидящих нас прости, во еже ни единому от них нас ради погибнути, но всем спастися благодатию Твоею, Боже Всещедрый.

\end{mymulticols}

\section{Просьба защиты от обидчика. Мученику Иоанну-воину}\begin{mymulticols}
%http://www.molitvoslov.org/text589.htm 


\mysubtitle{Тропарь, глас 8-й:}

Блаженство Евангельское возлюбив, Богомудре Иоанне, чистоту сердца почтил еси. Темже суету мира сего пренебрегл, устремился еси зрети Бога, Иже тя прослави чудесы во врачевании различне страждущих. Сего ради молим тя: проси нам у Христа Господа всяких скорбей избавления и получения Царства Небеснаго.

\mysubtitle{Кондак, глас 6-й:}

Благочестиваго воина Христова, победившаго враги душевныя и телесныя богомудренно, Иоанна мученика, достодолжно песньми восхвалим, чудодействуя бо, подает обильная исцеления страждущим людем, и молится Господу Богу от всяких бед спасти правоверныя.

\mysubtitle{Молитва:}

О, великий Христов мучениче Иоанне, правоверных поборниче, врагов прогонителю и обидимых заступниче. Услыши нас, в бедах и скорбях молящихся тебе, яко дана тебе бысть благодать от Бога печальныя утешати, немощным помогати, неповинныя от напрасныя смерти избавляти и за всех зле страждущих молитися. Буди убо и нам поборник крепок на вся видимыя и невидимыя враги наша, яко да твоею помощию и поборством по нас посрамятся вси являющии нам злая. Умоли Господа нашего, да сподобит ны грешныя и недостойныя рабы Своя получити от Него неизреченная благая, яже уготова любящим Его, в Троице Святей славимаго Бога, всегда, ныне и присно и во веки веков. Аминь. 

\myemph{Этому же святому молятся об отыскании украденного.}

\end{mymulticols}

\section{Молитва за неверующих, гонителей и презрителей правды}\begin{mymulticols}
%http://www.molitvoslov.org/text588.htm 


Господи, обрати к Тебе и сердца врагов наших, если же не возможно ожесточенным обратиться, то положи преграду зла и защити от них избранных Твоих. Аминь.

\end{mymulticols}

\section{Об умножении любви и искоренении ненависти и всякой злобы}\begin{mymulticols}
%http://www.molitvoslov.org/text592.htm 


\mysubtitle{Тропарь, глас 4-й:}

Союзом любве апостолы Твоя связавый, Христе, и нас, Твоих верных рабов к Себе тем крепко связав, творити заповеди Твоя и друг друга любити нелицемерно сотвори, молитвами Богородицы, Едине Человеколюбче.

\mysubtitle{Кондак, глас 5-й:}

Пламенем любве распали к Тебе сердца наша, Христе Боже, да тою разжигаеми, сердцем, мыслию же и душею, и всею крепостию нашею возлюбим Тя, и искренняго своего яко себе, и повеления Твоя храняще славим Тя, всех благ Дателя.

\mysubtitle{Прокимен, глас 7-й:}

Возлюблю Тя, Господи, крепосте моя, Господь утверждение мое.

\mysubtitle{Евангелие от Иоанна, гл.13, ст. 31-3}

Егда изыде, глагола Иисус: ныне прославися Сын Человеческий, и Бог прославися о нем; аще Бог прославися о нем, и Бог прославит его в себе, и абие прославит его. Чадца, еще с вами мало есмь: взыщете Мене, и якоже рех Иудеом, яко аможе Аз иду, вы не можете приити: и вам глаголю ныне. Заповедь новую даю вам, да любите друг друга: якоже возлюбих вы, да и вы любите себе; о сем разумеют вси, яко мои ученицы есте, аще любовь имате между собою.

\vspace{\baselineskip}
\mysubtitle{Из ектении:}

Господи, Боже наш, милостиво, яко Благ, призри на изсохшую в любви землю сердца нашего, и тернием ненависти, самолюбия же, и неисчетных беззаконий люте оледеневшую: и каплю благодати Пресвятого Твоего Духа испустив, богато ороси ю, во еже плодоносити, и возрастати от горящия к Тебе любве, всех добродетель корень страх Твой, и о искренняго спасении неленостное попечение, всех же страстей и многообразных лукавств, и лицемерия искоренение, прилежно яко всех Благодетеля молим, скоро услыши и человеколюбно помилуй…

\myemph{Причастен:} Заповедь новую даю вам, да любите друг друга, якоже Аз возлюбих вы, рече Господь.

\end{mymulticols}

\section{Ежедневная молитва святителя Филарета, митрополита Московского}\begin{mymulticols}
%http://www.molitvoslov.org/text583.htm 

Господи! Не знаю, что мне просить у Тебя. Ты Един ведаешь, что мне потребно. Ты любишь меня паче, нежели я умею любить себя. Отче! Даждь рабу Твоему чего сам я просить не смею. Не дерзаю просить ни креста, ни утешения: только предстою пред Тобою. Сердце мое Тебе отверзто; Ты зришь нужды, которых я не знаю. Зри и сотвори по милости Твоей. Порази и исцели, низложи и подыми меня. Благоговею и безмолвствую пред Твоею Святою Волею и непостижимыми для меня Твоими судьбами. Приношу себя в жертву Тебе. Предаюсь Тебе. Нет у меня другого желания, кроме желания исполнить Волю Твою. Научи меня молиться! Сам во мне молись. Аминь.

\end{mymulticols}

\section{При непрощении обид и памятовании злого}\begin{mymulticols}
%http://www.molitvoslov.org/content/Pri-neproshchenii-obid-i-pamyatovanii-zlogo 


Спасителю мой, научи меня простить от всей души всех, кто чем-либо меня обидел. Я знаю, что не могу предстать пред Тобой с чувствами вражды, которые таятся в моей душе. Очерствело мое сердце! Нет во мне любви! Помоги мне, Господи! Молю Тебя, научи меня простить обидящих меня, как Сам Ты, Бог мой, на кресте простил врагов Твоих!

\end{mymulticols}

\section{При немилосердии и раздражении на ближнего}\begin{mymulticols}
%/content/pri-nemiloserdii-i-razdrazhenii-na-blizhnego

Милосердный, милостивый, благий, долготерпеливый, любвеобильный, благосердный Отец Небесный! Оплакиваю и исповедаю пред Тобою прирожденное злонравие и нечувствительность моего сердца, что я часто немилосердием и недружелюбием согрешил пред бедным ближним моим, не принимал участия в его бедности и несчастиях, с ним случающихся, не имел к нему должнаго человеческаго, христианскаго и братскаго сострадания, оставлял его в бедствии, не посещал, не утешал, не помогал ему. В этом поступил я не как чадо Божие, потому что не был милосерден, как Ты, Небесный мой Отче, и не помышлял о том, что говорит Христос Господь мой: блаженны милостивые, ибо они помилованы будут. Не помышлял я о последнем приговоре на Страшном суде: подите от Меня, проклятые, в огонь вечный; ибо алкал Я, и вы не дали Мне есть; был наг, и вы не одели Меня; болен был, и не посетили Меня. Милосердный Отче! Прости мне этот тяжкий грех, и не вмени мне его. Отврати от меня тяжкое и праведное наказание, и соделай, чтобы не исполнился надо мной суд без милости, но прикрой и забудь немилосердие мое ради милосердия возлюбленнаго Сына Твоего. Даруй мне милосердное сердце, которое бы скорбело о бедствии ближняго моего, и соделай, чтобы скоро и легко побуждался я к состраданию. Даруй мне благодать, чтобы я содействовал к облегчению, а не к увеличению скорбей и бедствия, претерпеваемаго ближним моим; чтобы я утешал его в печали его и всем скорбным духом являл милосердие "--- больным, странникам, вдовам и сиротам; чтобы помогал им охотно и любил не на словах только, но делом и истиною. Боже мой! Милости хочешь Ты, а не жертвы. Соделай, чтоб я облекся в сердечное милосердие, благость, смирение, терпение, и охотно прощал, как Христос простил мне. Соделай, чтоб познал великое милосердие Твое во мне, потому что я слишком мал пред всем милосердием, которое Ты от дня рождения моего являл надо мною. Милосердие Твое предварило меня, когда я лежал во грехах; оно объемлет меня, оно следует за мною, куда бы я ни пошел, и наконец восприимет меня к себе в жизнь вечную. Аминь. 

\end{mymulticols}

\section{При охлаждении любви к ближним}\begin{mymulticols}
%http://www.molitvoslov.org/content/Pri-okhlazhdenii-lyubvi-k-blizhnim 


Господи, повсюду всем дающий жизнь и дыхание, и непрестанно чрез служение тварей доказывающий любовь Свою ко всем людям. Соделай в том и меня подобным образу Твоему, чтобы и я неутомимо, Тебя ради и по Твоему примеру, любил благороднейшее создание Твое "--- ближнего моего, и поступал с ним, как любовь того требует, чрез Иисуса Христа, Господа нашего. Аминь.

\end{mymulticols}

\section{При клевете и несправедливых притеснениях}\begin{mymulticols}
%http://www.molitvoslov.org/content/Pri-klevete-i-nespravedlivykh-pritesneniyakh 


Ты, Господи, принес Себя в крестную жертву для моего спасения. Могу ли и я отвергнуться тягостей несправедливых суждений о мне людей? Соделай, Господи Иисусе, чтобы, помышляя о тех поруганиях и злословиях, какия Ты за меня претерпел, научилось сердце мое терпению, и не только без раздражения, но даже с благодарением, охотно переносило обиды и осуждения других. Об одном умоляю Тебя, Господи, не оставь врагов моих навсегда в слепоте их, но озари наконец и их лучем благодати Твоей. Аминь.

\end{mymulticols}

\section{В скорби от поношений и непонимания}\begin{mymulticols}
%http://www.molitvoslov.org/content/V-skorbi-ot-ponoshenii-i-neponimaniya 


Вечный Боже и Отче! Даруй мне Твоего Духа силы, да укрепит Он меня в немощи моей, да утвердит и соделает непоколебимым, дабы ни скорбь, ни гонение, ни нагота не отлучали меня от любви Твоей, но дабы я все преодолевал ради Возлюбившаго меня. Даруй, да во всем являю себя, как служитель Христов, с великим терпением в бедствиях, в нуждах, в тесных обстоятельствах, в ранах, в темницах, в трудах, в бдениях, в постах, с чистотою, с ведением, с долготерпением, с добротою, с Духом Святым, с нелицемерною любовию, со словом истины, с силою Божиею; да последую Ему в чести и безчестии, среди порицаний и похвал; охотно с Ним и ради имени Его допуская здесь ругаться над собою, ибо Он, если пребуду верен до смерти, увенчает меня венцом жизни. Аминь.

\end{mymulticols}

\section{Против вражьего страха}
%http://www.molitvoslov.org/content/Protiv-vrazhego-strakha

{\centeringГосподи, от страха вражьего изми душу мою.\par}

\section{О безропотном перенесении насмешек и гонений Блаженному Василию, Христа ради юродивому}\begin{mymulticols}
%http://www.molitvoslov.org/content/O-bezropotnom-perenesenii-nasmeshek-i-gonenii-Blazhennomu-Vasiliyu-Khrista-radi-yurodivomu 


О, великий угодниче Христов, истинный друже и верный рабе Всетворца Господа Бога, преблаженне Василие! Услыши ны, многогрешныя \myemph{(имена)}, ныне вопиющия к тебе и призывающия имя твое святое, помилуй ны, припадающия днесь к пречистому образу твоему, приими малое наше и недостойное сие моление, умилосердися над убожеством нашим, и молитвами твоими исцели всяк недуг и болезнь души и тела нашего грешнаго, и сподоби ны течение жизни сея невредимо от видимых и невидимых врагов и безгрешно прейти, и христианскую кончину непостыдну, мирну, безмятежну, и Небеснаго Царствия наследие получить со всеми святыми во веки веков. Аминь.

\end{mymulticols}


\mychapterending

\mychapter{В скорби или недуге}
%http://www.molitvoslov.org/content/vskorbi-ili-neduge



\section{Молитва против крамолы}\begin{mymulticols}
%http://www.molitvoslov.org/text605.htm 


Боже сил, Царю царствующих и Господи господствующих! В Твоей руце сердце царево и власть всея земли. Ты посаждаеши царей на престолы их и глаголеши о них: “Мною царие царствуют, не прикасайтеся убо помазанным Моим”. Призри милостивным оком Твоим на люте страждущую страну нашу, в нейже за беззакония наша зело умножишася нестроения и раздоры, междоусобия и противления царю нашему и властем, от него поставленным. Еще же и мор, и глад, и болезни всякия постигоша ны, и несть мира, несть ослабы, несть успокоения в домех наших, ниже во градех и весех наших. О Всеведущий и Премилосердый, Ты веси беду нашу, зриши озлобления, слышиши рыдания убогих, стенания сирых и вдовиц и гласы неповинных младенцев, во общенуждии страждущих. Ты пощадил еси Ниневию, град великий, исполнен беззакония, ради покаяния его и ради младенец, иже не познаша десницы своея и шуйцы своея; пощади и нас грешных, умилосердися над отечеством нашим, пощади ради младенец наших, сирых и вдовиц, пред Тобою, о Милосердый, слезы своя проливающих! Видиши, Господи, како людие, имже несть разума, не точною глаголют в сердцах своих “несть Бог”, но и проповедуют нам, паче же юношам и девам нашим, малоопытным сущим, учения развращенная, и тщатся отвратити их от пути Заповедей Твоих на пути стропотныя и погибельныя. Вся возможна Тебе суть, о, Всемогущий Владыко! Якоже убо единым cловом Твоим не точию слепых, глухих и немых исцеляеши, но и мертвых воскресаeши: тако возглаголи и ныне всесильным Словом Твоим в сердцах сих сеятелей нечестия: просвети разум их, гордынею омраченный, пробуди совесть их, суетными мудрованьми, завистию и страстьми пагубными усыпленную, обрати волю их ко исполнению животворящих Твоих Заповедей, да и тии познают, коль сладка суть Словеса Твоя сердцу человеческому, и коль иго Твое благо и бремя Твое легко есть. Да прославится убо Имя Твое, Господи Спасителю наш, и в сих погибающих братиях наших, и да посрамится лукавый супостат наш диавол, сеяй плевелы на ниве Твоей, яже есть Церковь Твоя Святая! О Господи, Боже Милосердый, Боже Премудрый, Боже Всемогущий! Паки и паки припадаем Тебе и слезно в покаянии и умилении сердца вопием: согрешихом, беззаконновахом, неправдовахом пред Тобою, и воистинну праведно по делом нашим наказуеми есмы. Помяни убо веру и смирение отец наших, услыши теплая моления святых угодников Твоих, в земли нашей просиявших: помилуй землю Русскую, утоли вся крамолы, раздоры и нестроения, умири сердца, страстьми обуреваемая, сохрани возлюбленнаго Тобою раба Твоего, царя нашего, огради престол его правдою и миром, возглаголи в сердце его благая и мирная о Церкви Твоей и о людех Твоих, поели ему советников, мудростию исполненных и волю его свято исполняющих, соблюди суды его немздоимны и нелицеприятны, вдохни мужество в сердца стоящих на страже благоустроения государственнаго: а имже судил еси за веру, царя и Oтечество во дни скорби сея душу свою положити, и тем прости вся согрешения их, венцы мученическими венчай во Царствии Твоем. Всех же нас озари светом Закона Твоего евангельскаго, возгрей сердца наша теплотою благодати Твоея, утверди волю нашу в воле Твоей, да якоже древле, тако и ныне на земли нашей, и в нас, и чрез нас прославится Всесвятое Имя Твое, Отца и Сына и Святаго Духа. Аминь. 

\end{mymulticols}

\section{Молитва ко Господу о прощении, заступлении и помощи}\begin{mymulticols}
%http://www.molitvoslov.org/text604.htm 


В руце Твоего превеликаго милосердия, о Боже мой, вручаю душу и тело мое, чувства и глаголы моя, советы и помышления моя, дела моя и вся тела и души моея движения. Вход и исход мой, веру и жительство мое, течение и кончину живота моего, день и час издыхания моего, преставление мое, упокоение души и тела моего. Ты же, о Премилосерде Боже, всего мира грехами непреодолеваемая Благосте, незлобиве, Господи, мене, паче всех человеков грешнейшаго, приими в руце защищения Твоего и избави от всякаго зла, очисти многое множество беззаконий моих, подаждь исправление злому и окаянному моему житию и от грядущих грехопадений лютых всегда восхищай мя, да ни в чемже когда прогневаю Твое человеколюбие, имже покрывай немощь мою от бесов, страстей и злых человеков. Врагом видимым и невидимым запрети, руководствуя мя спасенным путем, доведи к Тебе, пристанищу моему и желаний моих краю. Даруй ми кончину христианску, непостыдну, мирну, от воздушных духов злобы соблюди, на Страшном Твоем Суде милостив рабу Твоему буди и причти мя одесную благословенным Твоим овцам, да с ними Тебе, Творца моего, славлю во веки. Аминь.

\end{mymulticols}

\section{Молитвенное воздыхание ко Господу}

{\centering\myemph{(Предсмертная молитва иеросхимонаха Парфения Киевского)}\par}

\begin{mymulticols}
%http://www.molitvoslov.org/text585.htm 

\begin{enumerate}

\item Когда я, удрученный болезнью, восчувствую, приближение кончины земного бытия моего: Господи, помилуй меня. 

\item Когда бедное сердце мое при последних ударах своих будет изнывать и томиться смертными муками: Господи, помилуй меня. 

\item Когда очи мои в последний раз орошатся слезами при мысли, что в течение моей жизни оскорблял я Тебя, Боже, грехами моими: Господи, помилуй меня. 

\item Когда частое биение сердца станет ускорять исход души моей: Господи, помилуй меня. 

\item Когда смертная бледность лица моего и холодеющее тело мое поразит страхом близких моих: Господи, помилуй меня. 

\item Когда зрение мое помрачится и пресечется голос, окаменеет язык мой: Господи, помилуй меня. 

\item Когда страшные призраки и видения станут доводить меня до отчаяния в Твоем милосердии: Господи, помилуй меня. 

\item Когда душа моя, пораженная воспоминаниями моих преступлений и страхом суда Твоего изнеможет в борьбе с врагами моего спасения, силящимися увлечь меня в область мрака мучений: Господи, помилуй меня. 

\item Когда смертный пот оросит меня, и душа с болезненными страданиями будет отдаляться от тела: Господи, помилуй меня. 

\item Когда смертный мрак закроет от мутного взора моего все предметы мира сего: Господи, помилуй меня. 

\item Когда в теле моем прекратится все ощущение, оцепенеют жилы и окаменеют мышцы мои: Господи, помилуй меня. 

\item Когда до слуха моего не будут уже доходить людские речи и звуки земные: Господи, помилуй меня. 

\item Когда душа предстанет лицу Твоему, Боже, в ожидании Твоего назначения: Господи, помилуй меня. 

\item Когда стану внимать праведному приговору суда Твоего, определяющего вечную участь мою: Господи, помилуй меня. 

\item Когда тело, оставленное душею, сделается добычей червей и тления и, наконец, весь состав мой превратится в горсть праха: Господи, помилуй меня. 

\item Когда трубный глас возбудит всех при втором Твоем пришествии и раскроется книга деяний моих: Господи Иисусе Христе, Сыне Божий, помилуй мя грешного раба Твоего \myemph{(имя)}. В руце Твои, Господи, предаю дух мой. Аминь. 

\end{enumerate}

\end{mymulticols}

\section{Вопль к Богоматери}\begin{mymulticols}
%http://www.molitvoslov.org/text584.htm 


О чем молить Тебя, чего просить у Тебя? Ты ведь все видишь, знаешь Сама, посмотри мне в душу и дай ей то, что ей нужно. Ты, все претерпевшая, премогшая,"--- все поймёшь. Ты повившая Младенца в яслях и принявшая Его Своими руками со Креста, Ты одна знаешь всю высоту радости, весь гнёт горя. Ты, получившая в усыновление весь род человеческий, взгляни и на меня с материнской заботой. Из тенет греха приведи меня к Своему Сыну. Я вижу слезу, оросившую Твой лик. Это надо мной Ты пролила еe и пусть смоет она следы моих прегрешений. Вот я пришла, я стою, я жду Твоего отклика, о, Богоматерь, о, Всепетая, о, Владычице! Ничего не прошу, только стою пред Тобой. Только сердце моe, бедное человеческое сердце, изнемогшее в тоске по правде, бросаю к Пречистым ногам Твоим, Владычицe! Дай всем, кто зовёт Тебя, достигнуть Тобою вечного дня и лицем к лицу поклониться Тебе.

\end{mymulticols}

\vspace{-1.5\baselineskip}\section{Об укрощении людского гнева}\begin{mymulticols}
%http://www.molitvoslov.org/text594.htm 

\begin{enumerate}

\item \myemph{Молиться святому пророку и царю Давиду. (Царь Давид непоколебимо веровал в Бога и старался исполнять Его волю. Он подвергался многочисленным преследованиям, и Господь избавлял его от всех врагов.)}

\item \myemph{Читать молитву} «Владыко Человеколюбче…» ( \myemph{молитва святого Иоанна Дамаскина из молитв на сон грядущим.}

\item \myemph{Читать молитву} «Богородице Дево, радуйся, Благодатная Марие, Господь с Тобою; благословена Ты в женах и благословен плод чрева Твоего, яко Спаса родила еси душ наших.»

\item \myemph{Читать псалмы 26 и 131.}

\item \myemph{Подходя к дверям гневливого начальника, стоит говорить:} «Помяни, Господи, Царя Давида, и всю кротость его».

\end{enumerate}

\end{mymulticols}

\vspace{-1.5\baselineskip}\section{Mолитва святителя Димитрия Ростовского}\begin{mymulticols}
%http://www.molitvoslov.org/text582.htm 


Спаси мя, Спасе мой, по Твоей благости, а не по моим делом! Ты хочеши мя спасти, Ты веси, коим образом мя спасти: спаси убо мне, яко хощеши, яко можеши, яко веси: имиже веси судьбами спаси мя! Аз на Тя, Господа моего, надеюся, и Твоей воле Святой себе вручаю: твори со мною еже хощеши! Аще хощеши мя имети во свете: буди благословен. Аще хощеши мя имети во тьме: буди паки благословен. Аще отверзеши ми двери милосердия Твоего: добро убо и благо. Аще затвориши ми двери милосердия Твоего: благословен еси, затворивый ми в правду. Аще не погубиши мя со беззаконьми моими, слава безмерному милосердию Твоему. Аще погубиши мя со беззаконьми моими, слава праведному суду Твоему: якоже хощеши, устрой о мне вещь!

\end{mymulticols}

\vspace{-1.5\baselineskip}\section{В скорби или недуге}\begin{mymulticols}
%http://www.molitvoslov.org/text581.htm 


Спаси, Господи, и помилуй раба Твоего \myemph{(имярек)} словесами Божественного Евангелия Твоего, читаемыми о спасении раба твоего \myemph{(имярек)}, Попали, Господи, терние всех согрешений его, и да вселится в него благодать Твоя, опаляющая, очищающая, освящающая всякого человека во имя Отца и Сына и Святого Духа. Аминь.

\myemph{Каждый день нужно читать по одной главе Евангелия, а перед главой и после нее "--- эту молитву}

\end{mymulticols}

\mychapterending{}

\mychapter{При опасности потопления}
%http://www.molitvoslov.org/content/pri-opasnosti-potopleniya

\section{Пресвятой Богородице перед Ее иконой Леньковскою, или «Спасительница Утопающих»}\begin{mymulticols}
%http://www.molitvoslov.org/text607.htm 


\mysubtitle{Молитва:}

Заступница усердная, Мати Господа Вышняго! Ты еси всем xристианом помощь и заступление, паче же в бедах сущим. Призри ныне с высоты святыя Твоея и на ны, с верою покланяющияся пречистому образу Твоему, и яви, молим Тя, скорую помощь Твою по морю плавающим и от ветров бурных тяжкия скорби терпящим. Подвигни и вся православныя христианы на спасение в водах утопающих, и воздаждь подвизающимся о сем богатыя милости и щедроты Твоя. Се бо, на образ Твой взирающе, Тебе, яко милостивно сущей с нами, приносим смиренная моления наша. Не имамы бо ни иныя помощи, ни инаго предстательства, ни утешения, токмо Тебе, о Мати всех скорбящих и напаствуемых. Ты по Бозе наша Надежда и Заступница, и на Тя уповающе, сами себе, и друг друга, и весь живот наш Тебе предаем во веки веков. Аминь.

\end{mymulticols}

\section{Преподобным Зосиме и Саватию Соловецким чудотворцам}\begin{mymulticols}
%http://www.molitvoslov.org/text608.htm 


\mysubtitle{Тропарь, глас 4-й:}

Постническое и равноангельское житие ваше, преподобнии Зосимо и Савватие, вселенней познаны сотвори вас: чудодеяния различными, богоноснии, просвещаете верою призывающия вы и чтущия честную память вашу.

\mysubtitle{Иной тропарь, глас 8-й:}

Яко светильницы явистеся всесветлии, во отце окиана моря, преподобнии отцы Зосимо и Савватие: вы бо крест Христов на рамо всемше, усердно Тому последовасте, и чистотою Богови преближившеся, отонудуже притекаем к ракам честных мощей ваших, и умильно глаголем: о преподобнии, молите Христа Бога спастися душам нашим.

\mysubtitle{Величание:}

Ублажаем вас, преподобнии отцы Зосиме и Саватие, и чтим святую память вашу, наставницы монахов и собеседницы ангелов.

\mysubtitle{Молитва:}

О, преподобнии отцы, велиции заступницы и скории услышателие молитв, угодницы Божии и чудотворцы Зосимо и Савватие! Не забудите, якоже обещастеся, посещати чада ваша. Аще бо и отъидосте от нас телом, но духом присно с нами пребываете. Молим убо вас, о преподобнии, избавите ны от огня и меча, от нашествия иноплеменников и междоусобныя брани, от тлетворных ветров и от внезапныя смерти, и от всех прилог бесовских, находящих на ны. Услышите нас, грешных \myemph{(имена)}, и приимите молитву сию и моление наше, яко кадило благовонное, яко жертву благоугодную, и души наша, злыми делы и советы и помыслы умерщвленныя, оживите, и якоже умершую отроковицу возстависте, и неисцельныя раны многих исцелисте, и от духов нечистых зле мучимыя избависте: тако и нас, содержимых во узах вражиих, изьмите, и от сетей диавола избавите, из глубины прегрешений изведите, и милостивым вашим посещением и ходатайством от враг видимых и невидимых оградите ны благодатию и силою Всесвятыя Троицы, всегда, ныне и присно и во веки веков. Аминь.

\end{mymulticols}

\section{Святителю Николаю, архиепископу Мирликийскому}\begin{mymulticols}
%http://www.molitvoslov.org/text609.htm 


\mysubtitle{Тропарь, глас 4-й:}

Правило веры и образ кротости, воздержания учителя яви тя стаду твоему, яже вещей истина; сего ради стяжал еси смирением высокая, нищетою богатая, отче священноначальниче Николае, моли Христа Бога спастися душам нашим.

\mysubtitle{Кондак, глас 3-й:}

В Мирех, святе, священнодействователь показался еси: Христово бо, преподобне, Евангелие исполнив, положил еси душу твою о людех твоих, и спасл еси неповинныя от смерти; сего ради освятился еси, яко великий таинник Божия благодати.

\mysubtitle{Величание:}

Величаем тя, святителю отче Николае, и чтим святую память твою, ты бо молиши за нас Христа Бога нашего.

\mysubtitle{Молитва:}

О, всехвальный и всечестный архиерею, великий чудотворче, святителю Христов, отче Николае, человече Божий и верный рабе, муже желаний, сосуде избранный, крепкий столпе церковный, светильниче пресветлый, звездо, осиявающая и освещающая всю вселенную: ты еси праведник, яко финикс процветший, насажденный во дворех Господа своего, живый в Мирех, миром облагоухал еси и миро приснотекущее благодати Божия источаяй. Твоим шествием, пресвятый отче, море освятися, егда многочудесныя твоя мощи шествоваху во град Барский, от востока до запада хвалити имя Господне. О, преизрядный и предивный чудотворче, скорый помощниче, теплый заступниче, пастырю предобрый, спасающий словесное стадо от всяких бед! Тебе прославляем и тебе величаем, яко надежду всех христиан, источника чудес, защитителя верных, премудраго учителя, алчущих кормителя, плачущих веселие, нагих одеяние, болящих врача, по морю плавающих управителя, пленников свободителя, вдов и сирот питателя и заступника, целомудрия хранителя, младенцев кроткаго наказателя, старых укрепление, постников наставника, труждающихся упокоение, нищих и убогих изобильное богатство. Услыши нас, молящихся тебе и прибегающих под кров твой, яви предстательство твое о нас к Вышнему и исходатайствуй твоими богоприятными молитвами вся полезная ко спасению душ и телес наших: сохрани святую обитель сию \myemph{(или храм сей)}, всякий град и весь, и всякую страну христианскую, и люди живущия от всякаго озлобления помощию твоею, вемы бо, вемы, яко много может молитва праведнаго, поспешествующая во благое: тебе же праведнаго, по Преблагословенней Деве Марии, предстателя ко Всемилостивому Богу имамы, и к твоему, преблагий отче, теплому ходатайству и заступлению смиренно притекаем. Ты нас соблюди, яко бодрый и теплый пастырь, от всяких врагов, губительства, труса, града, глада, потопа, огня, меча, нашествия иноплеменников, и во всяких бедах и скорбех наших подай нам руку помощи, и отверзи двери милосердия Божия: понеже недостойни есмы зрети высоту небесную, от множества неправд наших, связани узами греховными, и николиже воли Создателя нашего сотворихом, ни сохранихом повелений Его. Темже преклоняем колена сокрушенна и смиренна сердца нашего к Зиждителю своему, и твоего отеческаго заступления к Нему просим: помози нам, угодниче Божий, да не погибнем со беззаконии нашими, избави нас от всякаго зла и от всякия вещи сопротивныя, управи ум наш и укрепи сердце наше в правой вере, в нейже твоим предстательством и ходатайством, ни ранами, ни прещением, ни мором, ни коим гневом от Создателя своего умалени будем, но мирное зде поживем житие и да сподобимся видети благая на земли живых, славяще Отца и Сына и Святаго Духа, Единаго в Троице славимаго и покланяемаго Бога, ныне и присно и во веки веков. Аминь.

\end{mymulticols}

\mychapterending

\mychapter{В беде, печали и скорби}
%http://www.molitvoslov.org/content/vbede-pechali-iskorbi


\section{Святому мученику Трифону}\begin{mymulticols}
%http://www.molitvoslov.org/text586.htm 

\myfigure[0.85]{446}

\mysubtitle{Тропарь, глас 8-й:}

Мученик Твой, Господи, Трифон во страдании своем венец прият нетленный от Тебе, Бога нашего: имеяй бо крепость Твою, мучителей низложи, сокруши и демонов немощныя дерзости. Того молитвами спаси души наша.

\mysubtitle{Кондак, глас 8-й:}

Троическою твердостию многобожие разрушил еси от конец, всеславне, честен во Христе быв, и, победив мучители во Христе Спасителе, венец приял еси мученичества твоего и дарования Божественных исцелений, яко непобедим.

\mysubtitle{Молитва:}

О, святый мучениче Христов Трифоне, скорый помощниче всем, к тебе прибегающим и молящимся пред святым твоим образом скоропослушный предстателю! Услыши убо ныне и на всякий час моление нас, недостойных рабов твоих, почитающих святую память твою. Ты убо, угодниче Христов, сам обещался еси прежде исхода твоего от жития сего тленнаго молитися за ны ко Господу и испросил еси у Него дар сей: аще кто в коей-либо нужде и печали своей призывати начнет святое имя твое, той да избавлен будет от всякаго прилога злаго. И якоже ты иногда дщерь цареву в Риме граде от диавола мучиму исцелил еси, сице и нас от лютых его козней сохрани во вся дни живота нашего, наипаче же в день страшный последняго нашего, издыхания предстательствуй о нас, егда темнии зраки лукавых бесов окружати и устрашати нас начнут. Буди нам тогда помощник и скорый прогонитель лукавых бесов, и к Царствию Небесному предводитель, идеже ты ныне предстоиши с ликом святых у Престола Божия, моли Господа, да сподобит и нас причастниками быти присносущнаго веселия и радости, да с тобою купно удостоимся славити Отца и Сына и Святаго Утешителя Духа во веки. Аминь. \myemph{(Ему же молятся об избавлении от храпа.)}

\end{mymulticols}

\mychapterending

\mychapter{При нападении грабителей}
%http://www.molitvoslov.org/content/prinapadeniigrabiteley


\section{Святому праведному Иосифу, обручнику Пресвятой Девы Марии}\begin{mymulticols}
%http://www.molitvoslov.org/text613.htm 


\mysubtitle{Тропарь, глас 4-й:} 

Благовествуй, Иосифе, Давиду чудеса Богоотцу: Деву видел еси рождшую, с пастыри славословил еси. C волхвы поклонился еси. Aнгелом весть прием: моли Христа Бога спасти души наша. 

\mysubtitle{Кондак, глас 3-й:} 

Веселия днесь Давид исполняется Божественный, Иосиф же хваление со Иаковом приносит: венец бо сродством Христовым приемше, радуются, и неизреченно на земли Родшагося воспевают, и вопиют: Щедре, спасай Тебе чтущия.

\mysubtitle{Молитва:}

О святый праведный Иосифе! Ты еще на земли сый, велие имел еси дерзновение к Сыну Божию, иже благоизволи именовати тя отца Своего, яко обрученика Своея Матери, и послушати тя: веруем, яко ныне с лики праведных во обителех небесных водворяяся, услышан будеши во всяком твоем прошении к Богу и Спасителю нашему. Темже, ко твоему покрову и заступлению прибегающе, смиренно молим тя: якоже сам от бури сумнительных помышлений избавлен был еси, сице избави и нас, волнами смущений и страстей обуреваемых: якоже ограждал еси Всенепорочную Деву от клеветы человеческия, огради и нас такожде от всякия клеветы напрасныя: якоже хранил еси от всякого вреда и озлобления воплотившагося Господа, сице сохраняй твоим заступлением Церковь Его Православную и всех нас от всякаго озлобления и вреда. Веси, святче Божий яко и Сын Божий во днех плоти Своея в телесных потребах нужду имеяши, и ты послужил еси им: того ради молим тя, и нашим временным нуждам благопоспеши твоим ходатайством, подая нам вся благая, в житии сем потребная. Изряднее же просим тя, исходатайствуй нам оставление грехов прияти от нареченнаго ти Сына, Единороднаго же Сына Божия, Господа нашего Иисуса Христа и достойны быти наследия Царства Небеснаго нас предстательством твоим сотвори, да и мы, в горних селениих с тобою водворяющеся, прославим Единаго Триипостаснаго Бога, Отца и Сына и Святаго Духа, ныне и во веки веков. Аминь.

\end{mymulticols}

\mychapterending

\mychapter{Молитва о находящихся в запрещении и себя клятвою связавших} 
%http://www.molitvoslov.org/content/molitva-o-nakhodyashchikhsya-v-zapreshchenii-i-sebya-klyatvoyu-svyazavshikh

\tolkosviashennikom\begin{mymulticols}

Владыко Господи Боже наш, Единородный Сыне и Слово Отчее, иже всяк соуз грех наших, Твоею страстию растерзавый, Иже дунувый на лица Твоим Апостолам и рек им: Приимите Духа Святаго, и ихже аще оставите грехи, оставятся им, и ихже аще держите, удержатся. Ты Сам, Владыко, святыми Твоими Апостолы даровал еси по времени священнодействующым во святей Твоей Церкви на земле оставляти грехи, и вязати и решити всякий соуз неправды: молимся убо и ныне о брате нашем (имя) предстоящем пред Тобою, подаждь ему Твою милость, разреши его соуз греховный, аще что в неведении, или небрежении глагола, или от малодушия содела, ведый человеческую немощь, яко Человеколюбец и благ, Владыка, вся вольныя и невольныя грехи прости ему, яко Ты еси милуяй окованныя, возставляяй низверженныя, надежде ненадеющихся, упокоение падших, и раба Твоего сего свободи от соуза греховнаго. Яко прославися Твое всесвятое имя, со Безначальным Твоим Отцем, и Святым Духом, ныне и присно, и во веки веков. Аминь.

\end{mymulticols}

\mychapterending

\mychapter{Молитвы о примирении враждующих}
%http://www.molitvoslov.org/content/oprimireniivrazhduyuchih

\section{Молитва о примирении враждующих}\begin{mymulticols}
%http://www.molitvoslov.org/text590.htm 

Владыко Человеколюбче, Царю веков и Подателя благих, разрушивший вражды средостения и мир подавший роду человеческому, даруй и ныне мир рабом Твоим, вкорени в них страх Твой и друг к другу любовь утверди: угаси всяку распрю, отъими вся разгласия и соблазны. Яко Ты еси мир наш и Тебе славу возсылаем, Отцу и Сыну и Святому Духу, ныне и присно и во веки веков. Аминь.

\end{mymulticols}

\section{Пресвятой Богородице перед Ее иконой «Умягчение Злых Сердец», или «Семистрельная»}\begin{mymulticols}
%http://www.molitvoslov.org/text591.htm 

\myfigure{780}

\mysubtitle{Тропарь, глас 5-й:}

Умягчи наша злая сердца, Богородице, и напасти ненавидящих нас угаси, и всякую тесноту души нашея разреши. Hа Твой святый образ взирающи, Твоим страданием и милосердием о нас умиляемся и раны Твоя лобызаем, стрел же наших, Тя терзающих, ужасаемся. Не даждь нам, Мати благосердная, в жестокосердии нашем и от жестокосердия ближних погибнути, Ты бо еси воистину злых сердец Умягчение.

\mysubtitle{Кондак:}

Избранной Деве Марии, превысшей всех дщерей земли, Матери Сына Божия, Его же даде спасению мира, со умилением взываем: воззри на многоскорбное житие наше, воспомяни скорби и болезни, ихже претерпела еси, яко наша земнородная, и сотвори с нами по милосердию Твоему, да зовем Ти: Радуйся, многоскорбная Мати Божия, печаль нашу в радость претворяющая.

\mysubtitle{Молитва:}

О многострадальная Мати Божия, Превысшая всех дщерей земли, по чистоте Своей и по множеству страданий, Тобою на земли перенесенных, приими многоболезненныя воздыхания наша и сохрани нас под кровом Твоея милости. Иного бо прибежища и теплаго предстательства разве Тебе не вемы, но, яко дерзновение имущая ко Иже из Тебе рожденному, помози и спаси ны молитвами Своими, да непреткновенно достигнем Царствия Небеснаго, идеже со всеми святыми будем воспевать в Троице единому Богу ныне и присно, и во веки веков. Аминь.

\end{mymulticols}

\mychapterending

\mychapter{Об умирении вражды между ближними}
%http://www.molitvoslov.org/content/ob-umirenii-vrazhdi



\section{Святым благоверным князьям Борису и Глебу, в крещении Роману и Давиду}\begin{mymulticols}
%http://www.molitvoslov.org/text595.htm 

{\small\myfigure{455}

%\mysubtitle{Святым благоверным князьям Борису и Глебу, в крещении Роману и Давиду}

\mysubtitle{Тропарь, глас 2-й:}

Правдивая страстотерпца, и истинная Евангелия Христова послушателя, целомудренный Романе с незлобивым Давидом, не сопротив стаста врагу сущу брату, убивающему телеса ваша, душам же коснутся не могущу. Да плачется убо злый властолюбец, вы же радующеся с лики ангельскими, предстояще Святей Троице, молитеся о державе сродников ваших, богоугодней быти, и сыновом Российским спастися.

\mysubtitle{Кондак, глас 3-й:}

Возсия днесь преславная память ваша, благороднии страстотерпцы Христовы, Романе и Давиде, созывающи нас к похвалению Христа Бога нашего. Тем, притекающе к раце мощей ваших, исцеления дар приемлем молитвами вашими, святии: вы бо Божественнии врачеве есте.

\mysubtitle{Величание:}

Величаем вас, страстотерпцы святии, и чтим честная страдания ваша, яже за Христа претерпели есте.

\mysubtitle{Молитва:}

О, двоице священная, братия прекрасная, доблии страстотерпцы Борисе и Глебе, от юности Христу верою, чистотою и любовию послуживший, и кровьми своими, яко багряницею, украсившиися, и ныне со Христом царствующий! Не забудите и нас, сущих на земли, но яко теплии заступницы, вашим сильным ходатайством пред Христом Богом сохраните юных во святей вере и чистоте неврежденными от всякаго прилога неверия и нечистоты, оградите всех нас от всякия скорби, озлоблений и напрасныя смерти, укротите всякую вражду и злобу, действом диавола воздвизаемую от ближних и чуждих. Молим вас, христолюбивии страстотерпцы, испросите у Великодаровитаго Владыки всем нам оставление прегрешений наших, единомыслие и здравие, избавление от нашествия иноплеменных, междоусобныя брани, язвы и глада. Снабдевайте своим заступлением \myemph{(град сей и)} всех, чтущих святую память вашу, во веки веков. Аминь.}

\end{mymulticols}

\mychapterending

\mychapter{О помощи в бедности и нужде}
%http://www.molitvoslov.org/content/opomochi-vbednosti



\section{Святителю Иоанну Милостивому}\begin{mymulticols}
%http://www.molitvoslov.org/text602.htm 


\mysubtitle{Тропарь, глас 8-й:}

В терпении твоем стяжал еси мзду твою, отче преподобне, в молитвах непрестанно терпевый, нищия возлюбивый и сия удовливый, но молися Христу Богу, Иоанне милостиве, блаженне, спастися душам нашим.

\mysubtitle{Кондак, глас 2-й:}

Богатство твое расточил еси убогим и Небесное богатство ныне восприял еси, Иоанне всемудре, сего ради вси тя почитаем, совершающе память твою, милостыни о тезоимените!

\mysubtitle{Молитва:}

Святителю Божий Иоанне, милостивый защитниче сирых и сущих в напастех! К тебе прибегаем и тебе молимся, яко скорому покровителю всех ищущих от Бога утешения в бедах и скорбех. Hе престай, моляся ко Господу о всех с верою притекающих к тебе! Ты, преисполнен сый Христовы любви и благости, явился еси яко чудный чертог добродетели милосердия и стяжал еси себе имя «милостивый». Ты был еси яко река, непрестанно текущая щедрыми милостьми и всех жаждущих обильно напаяющая. Веруем, яко по переселении от земли на небо, усугубися в тебе дар благодати сея и яко соделался еси неисчерпаемый сосуд всякия благостыни. Сотвори убо твоим ходатайством и заступлением пред Богом “всякия утехи”, да вси, прибегающий к тебе, обретают мир и безмятежие: даруй им утешение в печалех временных и пособие в нуждах житейских, всели в них надежду вечнаго упокоения во Царствии Небеснем. В житии твоем на земли ты был еси пристанище всем сущим во всякой беде и нужде, обидимым и недугующим, и ни един от притекавших к тебе и просивших у тебе милости лишен бысть твоея благостыни. Тожде и ныне, царствуя со Христом на Небеси, яви всем покланяющимся пред честною твоею иконою и молящимся о помощи и заступлении. Не точию сам ты творил еси милость безпомощным, но и сердца других воздвигал еси ко утешению немощных и ко призрению убогих. Подвигни убо и ныне сердца верных к заступлению сирых, ко утешению скорбящих и успокоению неимущих. Да не оскудевают в них дары милости, паче же да вселится в них и в дому сем, призревающем страждущих, мир и радость о Дуxе Святе, во славу Господа и Спаса нашего Иисуса Христа, во веки веков. Аминь.

\end{mymulticols}

\mychapterending{}

\mychapter{При нападении зверей}
%http://www.molitvoslov.org/content/prinapadenii-zverey

\section{Великомученику Георгию Победоносцу}\begin{mymulticols}
%http://www.molitvoslov.org/text611.htm 

\ifpdf
{\centering\myemph{\normalsize\textbf{Тропарь, глас 4-й:}\par}}
\else
\mysubtitle{Тропарь, глас 4-й:}
\fi

Яко пленных свободитель, и нищих защититель, немощствующих врач, царей поборниче, победоносче великомучениче Георгие, моли Христа Бога, спастися душам нашим.

\mysubtitle{Кондак, глас 4-й:}

Возделан от Бога, показался еси благочестия делатель честнейший, добродетелей рукояти собрав себе: сеяв бо в слезах, веселием жнеши: страдальчествовав же кровию, Христа приял еси: и молитвами, святе, твоими, всем подаеши прегрешений прощение.

\mysubtitle{Величание:}

Величаем тя, страстотерпче святый великомучениче и победоносче Георгие, и чтим страдания твоя, яже за Христа претерпел еси.

\mysubtitle{Молитва:}

Святый, славный и всехвалъный великомучениче Христов Георгие! Собраннии во храме твоем и перед иконою твоею святою покланяющиися людие, молим тя, известный желания нашего ходатаю: моли с нами и о нас умоляемаго от Своего благоутробия Бога, да милостивно услышит нас, просящих Его благостыню, и не оставит вся наша ко спасению и житию нуждная прошения, и да укрепит же данною тебе благодатию во бранех православное воинство, разрушит силы возстающих враг, да постыдятся и посрамятся, и дерзость их да сокрушится, и да уведят, яко мы имамы Божественную помощь; и всем в скорби и обстоянии сущим многомощное яви твое заступление. Умоли Господа Бога, всея твари Создателя, избавити нас от вечнаго мучения, да всегда прославляем Отца и Сына и Святаго Духа, и твое исповедуем предстательство, ныне и присно и во веки веков. Аминь.

Да воскреснет Бог, и расточатся врази Его, и да бежат от лица Его ненавидящии Его. Яко исчезает дым, да исчезнут; яко тает воск от лица огня, тако да погибнут беси от лица любящих Бога и знаменующихся крестным знамением, и в веселии глаголющих: радуйся, Пречестный и Животворящий Кресте Господень, прогоняяй бесы силою на тебе пропятаго Господа нашего Иисуса Христа, во ад сшедшаго и поправшего силу диаволю, и даровавшаго нам тебе Крест Свой Честный на прогнание всякаго супостата. О, Пречестный и Животворящий Кресте Господень! Помогай ми со Святою Госпожею Девою Богородицею и со всеми святыми во веки. Аминь.

Богородице Дево, радуйся, Благодатная Марие, Господь с Тобою; благословена Ты в женах и благословен плод чрева Твоего, яко Спаса родила еси душ наших.

\end{mymulticols}

\mychapterending

\mychapter{От укушения гада}
%http://www.molitvoslov.org/content/ot-ukusheniya-gada



\section{Преподобному Алексию, человеку Божию}\begin{mymulticols}
%http://www.molitvoslov.org/text612.htm 


\mysubtitle{Тропарь, глас 4-й:}

Возвысився на добродетель и ум очистив, к желанному и крайнему достигл еси, безстрастием же украсив житие твое, и пощение изрядное восприим совестию чистою, в молитвах, яко безплотен, пребывая, возсиял еси, яко солнце, в мире, преблаженне Алексие.

\mysubtitle{Кондак, глас 2-й:}

Дом родителей твоих яко чужд, имев, водворился еси в нем нищеобразно и, по преставлении венец прием славы, дивен на земли явился еси, Алексие, человече Божий, ангелом и человеком радование.

\mysubtitle{Молитва:}

О, великий Христов угодниче, святый человече Божий Алексие, душею на Небеси Престолу Господню предстояй, на земли же данною ти свыше благодатию различная совершаяй чудеса! Призри милостивно на предстоящия святей иконе твоей люди, умиленно молящияся и просящия от тебе помощи и заступления. Простри молитвенно ко Господу Богу честнии руце твои, и испроси нам от Него оставление согрешений наших вольных и невольных, в недузех страждущим исцеление, напаствуемым заступление, скорбящим утешение, бедствующим скорую помощь, всем же чтущим тя мирную и христианскую живота кончину и добрый ответ на страшнем суде Христове. Ей, угодниче Божий, не посрами упования нашего, еже на тя по Бозе и Богородице возлагаем, но буди нам помощник и покровитель во спасение, да, твоими молитвами получивше благодать и милость от Господа, прославим человеколюбие Отца и Сына и Святаго Духа, в Троице славимаго и покланяемаго Бога, и твое святое заступление, ныне и присно и во веки веков. Аминь.

\end{mymulticols}

\mychapterending

\mychapter{О спасении от насилия}
%http://www.molitvoslov.org/content/ospasenii-ot-nasiliya



\section{Мученице Фомаиде Египетской}\begin{mymulticols}
%http://www.molitvoslov.org/text610.htm 


\mysubtitle{Тропарь, глас 4-й:}

Агница Твоя, Иисусе, Фомаида, зовет велиим гласом: Тебе, Женише мой, люблю, и Тебе ищущи страдальчествую и сраспинаюся, и спогребаюся Крещению Твоему, и стражду Тебе ради, яко да царствую в Тебе, и умираю за Тя, да и живу с Тобою; но яко жертву непорочную приими мя, с любовию пожершуюся Тебе. Тоя молитвами, яко Милостив, спаси души наша.

\mysubtitle{Кондак, глас 2-й:}

Храм твой всечестный яко цельбу душевную обретше, вси вернии велегласно вопием ти: дево мученице, Фомаидо великоименитая, Христа Бога моли непрестанно о всех нас.

\mysubtitle{Молитва:}

О, всехвальная мученице Фомаидо! За чистоту супружества даже до крове подвизавшися и целомудрия ради душу свою положивши, достойна пред Господем обрелася еси, во еже в лице преподобных дев почтенней тебе быти. Услыши нас молящихся тебе, и якоже древле боримии от плоти целительницу тя имеяху, по дарованной тебе от Бога благодати, сице и ныне прибегающим ко предстательству твоему подаждь отраду и свобождение от плотския брани, и целомудренное житие и незазорное в супружестве и девстве пребывание твоими богоприятными молитвами всем исходатайствовати потщися, яко да телеса наша храм живущаго в нас Святаго Духа будут. О, преизбранная в женах и верная рабо Христова! Помози нам, да не погибнем со страстьми и похотьми нашими, но да управимся умом нашим и укрепимся сердцем нашим во всяком благочестии и чистоте, славяще твою помощь и предстательство, благодать же и милость Триединаго Бога, Отца и Сына и Святаго Духа, во веки веков. Аминь. 

\end{mymulticols}

\mychapterending

\mychapter{Об обращении заблудшего}
%http://www.molitvoslov.org/content/ob-obrawenii-zabludshego



\section{Ко Господу Иисусу Христу}\begin{mymulticols}
%http://www.molitvoslov.org/content/Ko-Gospodu-Iisusu-Khristu 


Господи! Просвети заблудших \myemph{(имена)}, обрати их в святую Церковь Твою и спаси их всесильною Твоею благодатию! Нас же в Православии и истенней вере соблюди яко благословен еси во веки веков.

\end{mymulticols}

\section{Об обращении заблудшего. Молитва к Божией Матери (Святителя Гавриила Новгородского)}\begin{mymulticols}
%http://www.molitvoslov.org/text615.htm 

\myfigure[0.85]{474}

О, Всемилостивая Госпоже, Дево, Владычице Богородице, Царица Небесная! Ты рождеством Своим спасла род человеческий от вечного мучительства диавола: ибо от Тебя родился Христос, Спаситель наш. Призри Своим милосердием и на сего \myemph{(имя)}, лишенного милости Божией и благодати, исходатайствуй Матерним Своим дерзновением и Твоими молитвами у Сына Твоего, Христа Бога нашего, дабы ниспослал благодать Свою свыше на сего погибающего. О, Преблагословенная! Ты надежда ненадежных, Ты отчаянных спасение, да не порадуется враг о душе его! 

\end{mymulticols}

\mychapterending

\mychapter{Молитвы от голода}
%http://www.molitvoslov.org/content/molitvi-ot-goloda



\section{Пророку Божию Илии}\begin{mymulticols}
%http://www.molitvoslov.org/node/287 


{\small\mysubtitle{Тропарь, глас 4-й:}

Во плоти Ангел, пророков основание, второй предтеча пришествия Христова Илиа славный, свыше пославый Елисееви благодать недуги отгоняти и прокаженныя очищати, темже и почитающим его точит исцеления.

\vspace{-\baselineskip}\mysubtitle{Кондак, глас 3-й:}

Пророче и провидче великих дел Бога нашего, Илие великотеимените, вещанием твоим уставивый водоточныя облаки, моли о нас единого Человеколюбца.

\mysubtitle{Молитва:}

О святый пророче Божий Илие, моли о нас Человеколюбца Бога, да подаст нам, рабам Божиим \myemph{(имена)}, дух покаяния и сокрушения о гресех наших, и всесильною Своею благодатию да поможет нам пути нечестия оставити, преспевати же во всяком деле блазе, и в борьбе со страстьми и похотьми нашими да укрепит нас; да всадит в сердца наши дух смирения и кротости, дух братолюбия и незлобия, дух терпения и целомудрия, дух ревности ко славе Божией и о спасении своем и ближних доброе попечение. Отврати от нас предстательством твоим праведный гнев Божий, да тако в мире и благочестии поживше в сем веце, сподобимся причастия вечных благ во Царствии Господа и Спаса нашего Иисуса Христа, Ему же подобает честь и поклонение, со Безначальным Его Отцем и Пресвятым Духом, во веки веков. Аминь.

\myfigure{487}}

\end{mymulticols}

\vspace{-\baselineskip}\section{Молитва от голода пророку Божию Илии}\begin{mymulticols}
%http://www.molitvoslov.org/text638.htm 


{\smallВо плоти ангел, пророков основание, второй предтеча пришествия Христова, Илия славный, от ангела пищу приемый и вдовицу в годину глада напитавый, и нам, почитающим тя, благодатный питатель будь.

\par
\myemph{Открыта в 1915 г. в Москве одному подвижнику. С этой молитвой владыка Афанасий Сахаров прошел все тюрьмы и лагеря, везде соблюдая пост. («Литературная учеба», 1991, No 6)}
}

\end{mymulticols}

\mychapterending

\mychapter{Во время голода}
%http://www.molitvoslov.org/content/vo-vremya-goloda


\section{Святителю Спиридону, Тримифунтскому чудотворцу}\begin{mymulticols}
%http://www.molitvoslov.org/text643.htm 


\mysubtitle{Тропарь, глас 1-й:}

Собора Перваго показался еси поборник и чудотворец, богоносе Спиридоне, отче наш. Темже мертву ты во гробе возгласив, и змию в злато претворил еси; и внегда пети тебе святыя молитвы, ангелы сослужащия тебе, имел еси, священнейший. Слава Давшему тебе крепость, слава Венчавшему тя, слава Действующему тобою всем исцеления.

\mysubtitle{Кондак, глас 2-й:}

Любовию Христовою уязвився, священнейший, ум вперив зарею Духа, детельным видением твоим деяние обрел еси, Богоприятне, жертвенник Божественный быв, прося всем Божественнаго сияния.

\mysubtitle{Молитва:}

О, преблаженне святителю Спиридоне, великий угодниче Христов и преславный чудотворче! Предстоя на небеси Престолу Божию с лики ангелов, призри милостивым оком на предстоящия зде люди \myemph{(имярек)} и просящия сильныя твоея помощи. Умоли благосердие Человеколюбца Бога, да не осудит нас по беззаконием нашим, но да сотворит с нами по милости Своей! Испроси нам у Христа и Бога нашего мирное и безмятежное житие, здравие душевное и телесное, земли благоплодие и во всем всякое изобилие и благоденствие, и да не во зло обратим благая, даруемая нам от щедраго Бога, но во славу Его и в прославление твоего заступления! Избави всех верою несумненною к Богу приходящих, от всяких бед душевных и телесных, от всех томлений и диавольских наветов! Буди печальным утешитель, недугующим врач, в напастех помощник, нагим покровитель, вдовицам заступник, сирым защитник, младенцем питатель, старым укрепитель, странствующим путевождь, плавающим кормчий, и исходатайствуй всем, крепкия помощи твоея требующим, вся, яже ко спасению полезная! Яко да твоими молитвами наставляеми и соблюдаеми, достигнем вечнаго покоя и купно с тобою прославим Бога, в Троице Святей славимаго, Отца и Сына и Святаго Духа, ныне и присно и во веки веков. Аминь.

\end{mymulticols}\newpage

\section{Святому священномученику Харалампию}\begin{mymulticols}
%http://www.molitvoslov.org/text642.htm 


\mysubtitle{Тропарь, глас 2-й:}

Молитвами твоими ко Господу от зол, бед, напастей свободи, верно к тебе прибегающих и сетей избави вражиих, великий страстотерпче Харалампие.

\mysubtitle{Кондак, глас 6-й:}

Яко мученик и исповедник цвет, двойным венцем Христос тя увенча, возвеличив безчисленными чудесы на земли, но яко благодать Господню являя нам, верно к тебе прибегающим даруй мир, помози в напастех и лютых бедах, святейший Харалампие.

\mysubtitle{Молитва:}

О священная и многострадальная главо, пастырю добрый словесных овец Христовых, священномучениче Христов Харалампие, Магнисийская похвало и славо вселенныя, всемирный светильниче и великий наш заступниче и помощниче в скорбех, бедах и всяких нуждах! Услыши нас грешных, к тебе прибегающих и молящихся, и избави нас от всякаго злаго обстояния. Tы бо приял от Господа благодать и силу велику за многия и нуждныя страдания твоя и терпение, еже всюду и во всем нам помогати, и наипаче идеже память твоя почитаема будет бдением, пением и усердным молением. Сицевая благодать дадеся ти, священномучениче Христов, от явльшагося ти Господа Царя славы по твоему испрошению. Eгда на мечное усечение тя осудиша, услышал еси превожделенный и сладчайший оный глас, глаголющий тебе: «прииди, Харалампие, друже Мой, многия муки имене ради Моего претерпевый, и проси у Мене, еже хощеши, и Аз дам ти». Ты ж великий светильниче, рекл еси Христу Господу усты своими: «Господи мой, велико убо мне есть от Тебе, Света невечерняго, дарование сие. Аще угодно есть величеству Твоему, молю Тя, благоволи даровати ми сицевую милость Свою: да идеже положены будут мощи моя и почитаема будет память страдания моего, не будет на месте том ни глада, ни мора, ни тлетворнаго воздуха, погубляющаго плоды, но да будет паче на месте оном мир, здравие телесем и душам спасение, изобилие пшеницы, вина и елеа и умножение скотов, яже на потребу человеком». Ты же паки, угодниче Христов, Харалампие многострадальниче, глас Господень услышал еси, вещающий тебе: «буди по прошению твоему, преславный Мой воине». И абие, по глаголании сем Господнем, предал еси душу свою без мечнаго посечения, и пойде во след Христа Господа со славою многою, ангелом убо сретающим и провождающим ю до Престола Господня с радостию великою. И тако приял еси венец славы от Божественныя руки Его с лики святых, вечно славящих Пресвятое имя Господа. И тамо, во славе небесней пребывая, призри, угодниче Божий, и на нас грешных, молящихся тебе, и воспомяни нас пред Господем, еже даровати нам на потребу велия Его милости в безконечныя веки. Аминь.

\end{mymulticols}

\mychapterending

\mychapter{От засухи, грозы, града}
%http://www.molitvoslov.org/content/ot-zasuhi-grozi-grada



\section{Пророку Божию Илии}\begin{mymulticols}
%http://www.molitvoslov.org/text640.htm 

\myfigure{487}

\mysubtitle{Тропарь, глас 4-й:}

Во плоти Ангел, пророков основание, второй предтеча пришествия Христова Илиа славный, свыше пославый Елисееви благодать недуги отгоняти и прокаженныя очищати, темже и почитающим его точит исцеления.

\mysubtitle{Кондак, глас 3-й:}

Пророче и провидче великих дел Бога нашего, Илие великотеимените, вещанием твоим уставивый водоточныя облаки, моли о нас единого Человеколюбца.

\mysubtitle{Величание:}

Величаем тя, святый пророче Божий Илие, и почитаем еже на колеснице огненней преславное восхождение твое.

\mysubtitle{Молитва:}

Во плоти ангел, пророков основание, второй предтеча пришествия Христова, Илия славный, от ангела пищу приемый и Сарептскую вдовицу в годину глада напитавый, сам и нам, почитающим тя, благодатный питатель будь. Аминь.

\end{mymulticols}

\section{Блаженному Прокопию, Христа ради юродивому, Устюжскому чудотворцу}\begin{mymulticols}
%http://www.molitvoslov.org/text646.htm 


\mysubtitle{Тропарь, глас 4-й:}

Просветився божественною благодатию, богомудре, и весь разум и сердце от суетного мира сего к Зиждителю неуклонно возложив целомудрием и многим терпением, во временней жизни течение добре скончал еси и веру соблюл еси непорочну. Темже и по смерти явися светлость жития твоего: источаеши бо чудесем источник неисчерпаемый верою притекающим ко святому твоему гробу, Прокопие всеблаженне, моли Христа Бога, да спасет души наша.

\mysubtitle{Кондак, глас 4-й:}

Христа ради юродством воздушная мытарства на руках ангельских неприкосновенно пришед, Царского достиг еси Престола и Царя всех Христа Бога дар прием благодать исцелений, многими бо чудесы твоими и знамением страшным удивил еси град твой Великий Устюг: людем твоим милость испросив, миро от честнаго образа Пресвятыя Богородицы молитвою извел еси и недужным подал еси цельбы. Тем же молим тя, чудоносче Прокопие: моли Христа Бога непрестанно подати грехов наших прощение.

\mysubtitle{Молитва:}

О, великий угодниче Божий и чудотворче, святый блаженный Прокопие! Тебе молимся и тебе просим: молися о нас ко всемилостивому Богу и Спасу нашем Иисусу Христу, да проявит милость Свою к нам недостойным и дарует нам вся к животу и благочестию потребная: веры убо и любве преспеяние, благочестия умножение, мира утверждение, земли плодоносие, воздухов благорастворение и во всем благом благое поспешение. Вся грады и веси Российския предстательством твоим соблюди невредимы от всякаго зла. Всем православным христианом, тя молитвенно призывающим, комуждо по нуждам их, потребная даруй: болящим исцеление, скорбящим утешение, бедствующим поможение, унывающим ободрение, нищим снабдение, сирым призрение. Нам же всем дух покаяния и страха Божия испроси, да благочестно временное сие житие скончавше, сподобимся благую христианскую кончину получити и царствие небесное со избранными Божиими наследовати. Ей, праведниче Божий! Не посрами упования нашего, еже на тя смиренно возлагаем, но буди нам помощник и заступник в жизни, при смерти и по смерти нашей; да, твоим предстательством спасение улучивше, купно с тобою прославим Отца и Сына и Святаго Духа, и твое крепкое заступление о нас, во веки веков. Аминь.

\end{mymulticols}

\mychapterending

\mychapter{На брань блуда}\begin{mymulticols}
%http://www.molitvoslov.org/content/Na-bran-bluda 


Боже Вседержителю, всю тварь премудростию создавый, мене, падшаго многими согрешеньми, Твоею воздвигни рукою: подаждь ми Твою помощь, и сподоби мя от мирских свободитися искушений, от диавольских сетей, и от плотских похотей. Умилосердися и прости ми вся, елика Ти согреших во вся дни живота моего; помажи душу мою елеем благодати и щедрот единороднаго Сына Твоего, Господа Бога и Спаса нашего Иисуса Христа, с Ним же Тебе и Святому Духу подобает всякая слава во веки. Аминь.

\end{mymulticols}

\mychapterending

\mychapter{Молитва преследуемого человеками (свт. Игнатия Брянчанинова)}\begin{mymulticols}
%http://www.molitvoslov.org/content/Molitva-presleduemogo-chelovekami-svt-Ignatiya-Bryanchaninova 


Благодарю Тебя, Господь и Бог мой, за все совершившееся надо мною! Благодарю Тебя за все скорби и искушения, которые посылал Ты мне для очищения оскверненных грехами, для исцеления изъязвленных грехами моих души и тела! "--- Помилуй и спаси те орудия, которыя Ты употреблял для моего врачевания: тех людей, которые наносили мне оскорбления. Благослови их в этом и будущем веке! Вмени им в добродетели то, что они делали для меня! Назначь им из вечных твоих сокровищ и обильныя награды. "--- Что же я приносил Тебе? Какие благоугодныя жертвы? "--- Я приносил одни грехи, одни нарушения Твоих Божественнейших заповеданий. Прости меня, Господи, прости виновнаго пред Тобою и пред человеками! Прости безответного! Даруй мне увериться и искренно сознаться, что я грешник! Даруй мне отвергнуть лукавыя оправдания! Даруй мне покаяние! Даруй мне сокрушение сердца! Даруй мне кротость и смирение! Даруй любовь к ближним, любовь непорочную, одинаковую ко всем, и утешающим и оскорбляющим меня! Даруй мне терпение во всех скорбях моих! Умертви меня для мира! Отъими от меня мою греховную волю, и насади в сердце мое Твою святую волю, да творю ее единую и делами, и словами, и помышлениями, и чувствованиями моими. "--- Тебе за все подобает слава! Тебе единому принадлежит слава! Мое единственное достояние "--- стыдение лица и молчание уст. Предстоя страшному суду Твоему в убогой молитве моей, не обретаю в себе ни единаго достоинства, и предстою, лишь объятый отвсюду безчисленным множеством грехов моих, как бы густым облаком и мглою, с единым утешением в душе моей: с упованием на неограниченную милость и благость Твою. Аминь.

\end{mymulticols}

\mychapterending

