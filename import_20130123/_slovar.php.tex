

\mypart{Словарь и термины}
%http://www.molitvoslov.org/slovar.php 
 



А
Б
В
Г
Д
Е
Ж
З
И
К
Л
М
Н
О
П
Р
С
Т
У
Ф
Х
Ц
Ч
Ш
Щ
Ю
Я





\bfseries Абие \normalfont{} "--- немедленно, тотчас. 




\bfseries Авва \normalfont{} "--- отец. 




\bfseries Аввадон \normalfont{} "--- евр. "Погубитель"; имя ангела бездны. 




\bfseries Авраамово недро, лоно \normalfont{} "--- иносказательно: рай, место вечного блаженства. 




\bfseries Агаряне \normalfont{} "--- потомки Исмаила, сына Агари, наложницы Авраама, иносказательно "--- кочевые восточные племена. 




\bfseries Агиасма \normalfont{} "--- освященная по церковному чину вода. Освященная на празднике Богоявления вода называется Великой агиасмой. 




\bfseries Агиос \normalfont{} "--- надписание на древних иконах; греч. "святой". 




\bfseries Агкира \normalfont{} (читается "анкира") "--- якорь. 




\bfseries Агнец \normalfont{} "--- ягненок; чистое, кроткое существо; изымаемая на проскомидии часть просфоры для Евхаристии; мн. ч. "--- " \bfseries агнцы \normalfont{}" "--- иногда значит "христиане". 




\bfseries Агница \normalfont{} "--- овечка. 




\bfseries Агня \normalfont{} "--- ягненок. 




\bfseries Ад \normalfont{} "--- место нахождения душ умерших до освобождения их Господом Иисусом Христом; место вечного мучения грешников; жилище диавола. 




\bfseries Адамант \normalfont{} "--- алмаз; бриллиант; драгоценный камень. 




\bfseries Адамантовый \normalfont{} "--- твердый; крепкий; драгоценный. 




\bfseries Адов \normalfont{} "--- адский. 




\bfseries Адонаи \normalfont{} "--- евр. "Господь мой". 




\bfseries Аер \normalfont{} "--- покровец, полагаемый сверху священных сосудов на Литургии. 




\bfseries Аермонский \normalfont{} "--- связанный с горой Аермон. 




\bfseries Аз \normalfont{} "--- я. 




\bfseries Аиромантия \normalfont{} "--- воздуховолхвование, т.е. суеверное гадание на основании атмосферных явлений. 




\bfseries Акафист \normalfont{} "--- греч. "неседальное"; церковная служба, во время которой возбраняется сидеть. 




\bfseries Аки \normalfont{} "--- как будто, как бы. 




\bfseries Акриды \normalfont{} "--- пища Иоанна Крестителя; по мнению одних "--- род съедобной саранчи, или кузнечиков; по мнению других "--- какое-то растение. 




\bfseries Аксиос \normalfont{} "--- греч. "достоин". 




\bfseries Алавастр \normalfont{} "--- каменный сосуд. 




\bfseries Алектор \normalfont{} "--- петух. 




\bfseries Алкати \normalfont{} "--- голодать; хотеть есть, сильно желать чего-либо. 




\bfseries Алкота \normalfont{} "--- голод. 




\bfseries Аллилуия \normalfont{} "--- евр. "хвалите Бога"; "слава Богу!" 




\bfseries Аллилуия красная \normalfont{} "--- пение "аллилуйя" на особый умилительный распев. См. Триодь постную. 




\bfseries Аллилуиарий, аллилуиар \normalfont{} "--- стих, возглашаемый чтецом после чтения Апостола на Литургии. При этом возглашении на клиросах поют "аллилуия". 




\bfseries Алой, алое \normalfont{} "--- сок благовонного дерева, употреблявшийся для каждения и бальзамирования. 




\bfseries Алтабас \normalfont{} "--- самая лучшая старинная парча. 




\bfseries Амалик \normalfont{} "--- народ, живший между Палестиною и Египтом. В церковной поэзии это имя часто прилагается к диаволу. 




\bfseries Амвон \normalfont{} "--- возвышенная часть храма перед царскими вратами. 




\bfseries Амвросия \normalfont{} "--- неистлеваемая пища. 




\bfseries Амигдал \normalfont{} "--- миндаль. 




\bfseries Аминь \normalfont{} "--- евр. "да будет так"; "истинно"; "подлинно"; "да". 




\bfseries Амо, аможе \normalfont{} "--- куда. 




\bfseries Аможе аще \normalfont{} "--- куда бы ни. 




\bfseries Аналой \normalfont{} (правильнее " 




\bfseries аналогий \normalfont{}") "--- возвышенный стол, на который полагаются церковные книги при чтении и иконы. 




\bfseries Анафема \normalfont{} "--- отлучение от общины верных и предание суду Божию; тот, кто подвергся такому отлучению. 




\bfseries Анафематствовати \normalfont{} "--- предавать анафеме. 




\bfseries Анахорет \normalfont{} "--- отшельник. 




\bfseries Ангел \normalfont{} "--- вестник. 




\bfseries Ангеловидный \normalfont{} "--- внешне напоминающий Ангела. 




\bfseries Ангелозрачный \normalfont{} "--- внешне напоминающий Ангела. 




\bfseries Ангелоименитый \normalfont{} "--- знаменитый, почитаемый в лике ангельском; носящий имя какого-либо Ангела. 




\bfseries Ангелолепный \normalfont{} "--- приличный Ангелу. 




\bfseries Ангеломудренный \normalfont{} "--- имеющий мудрость Ангела. 




\bfseries Ангельское житие, ангельский образ \normalfont{} "--- высшая степень монашеского совершенства; греч. "схима". 




\bfseries Анепсий \normalfont{} "--- племянник, родственник. 




\bfseries Антидор \normalfont{} "--- благословенный хлеб, т.е. остатки той просфоры, из которой на проскомидии был изъят Агнец. 




\bfseries Антиминс \normalfont{} "--- греч. "вместопрестолие", освященный плат с изображением Иисуса Христа во гробе и вшитыми св. мощами. Только на антиминсе может быть совершаема Литургия. 




\bfseries Антифон \normalfont{} "--- греч. "противугласник"; песнопение, которое должно быть пето попеременно на обоих клиросах. 




\bfseries Антихрист \normalfont{} "--- греч. "противник Христа". 




\bfseries Антологион \normalfont{} "--- греч. "Цветослов"; название "Минеи праздничной". 




\bfseries Анфипат \normalfont{} "--- наместник, проконсул. 




\bfseries Анфракс \normalfont{} "--- яхонт. 




\bfseries Апокалипсис \normalfont{} "--- греч. "откровение". 




\bfseries Аполлион \normalfont{} "--- греч. "Погубитель"; имя ангела бездны. 




\bfseries Апостол \normalfont{} "--- греч. "посланник". 




\bfseries Апостасис \normalfont{} "--- отступничество. 




\bfseries Апостата \normalfont{} "--- отступник. 




\bfseries Априллий \normalfont{} "--- апрель. 




\bfseries Ариил \normalfont{} "--- горн у жертвенника всесожжения в храме Иерусалимском. 




\bfseries Армония \normalfont{} "--- гармония. 




\bfseries Ароматы \normalfont{} "--- душистая мазь. 




\bfseries Артос \normalfont{} "--- греч.хлеб квасной; он освящается с особой молитвой в день св. Пасхи. 




\bfseries Архангел \normalfont{} "--- начальствующий у Ангелов, название одного из чинов ангельских. 




\bfseries Архиерей \normalfont{} "--- первосвященник, епископ. 




\bfseries Архимагир \normalfont{} "--- главный повар. 




\bfseries Архипастырь \normalfont{} "--- первенствующий епископ. 




\bfseries Архисинагог \normalfont{} "--- начальник синагоги. 




\bfseries Архистратиг \normalfont{} "--- военачальник, полководец. 




\bfseries Архитектон \normalfont{} "--- архитектор, художник-строитель; главный строитель. 




\bfseries Архитриклин \normalfont{} "--- распорядитель пира. 




\bfseries Асмодей, Азмодеос \normalfont{} "--- "губитель", имя бесовское. 




\bfseries Аспид \normalfont{} "--- ядовитая змея. 




\bfseries Аспид парящий \normalfont{} "--- летающий ящер. 




\bfseries Ассарий \normalfont{} "--- мелкая медная монетка. 




\bfseries Астерикс \normalfont{} "--- звездица, поставляемая на дискосе при совершении Литургии. 




\bfseries Афарим \normalfont{} "--- соглядатаи ; лазутчики. 




\bfseries Афедрон \normalfont{} "--- задний проход (Мф. 15, 17). 




\bfseries Афинеи \normalfont{} "--- афиняне. 




\bfseries Африкия \normalfont{} "--- Африка. 




\bfseries Аще \normalfont{} "--- если; хотя; или; ли. 




\bfseries Аще убо \normalfont{} "--- поскольку; потому что. 




 





\bfseries Баальник \normalfont{} "--- волшебник. 




\bfseries Баба \normalfont{} "--- повивальная женщина. 




\bfseries Бабити \normalfont{} "--- помогать при родах. 




\bfseries Багряница \normalfont{} "--- ткань темно-красного цвета; порфира, пурпурная одежда высокопоставленных особ. 




\bfseries Балия \normalfont{} "--- колдунья; волшебница. 




\bfseries Баня пакибытия \normalfont{} "--- таинство св. Крещения. 




\bfseries Баснословити \normalfont{} "--- рассказывать небылицы; лгать. 




\bfseries Баснь \normalfont{} "--- ложное и бесполезное учение. 




\bfseries Бдение \normalfont{} "--- бодрствование; продолжительное ночное богослужение. 




\bfseries Бденно \normalfont{} "--- неусыпно, бодрственно. 




\bfseries Бденный \normalfont{} "--- неусыпный. 




\bfseries Бдети \normalfont{} "--- бодрствовать; не спать. 




\bfseries Бедне \normalfont{} "--- трудно; несносно; тяжело. 




\bfseries Бесоватися \normalfont{} "--- бесноваться, неистоваться. 




\bfseries Бедник \normalfont{} "--- калека; увечный. 




\bfseries Бедный \normalfont{} "--- иногда: увечный; калека. 




\bfseries Безведрие \normalfont{} "--- ненастье. 




\bfseries Безвидный \normalfont{} "--- не имеющий вида, образа. 




\bfseries Безвиновный \normalfont{} "--- не имеющий начала или причины для своего бытия. Одно из Божественных определений. 




\bfseries Безвозрастное \normalfont{} "--- младенец. 




\bfseries Безвременне \normalfont{} "--- некстати; неблаговременно. 




\bfseries Безгласие \normalfont{} "--- немота; молчание. 




\bfseries Безгодие \normalfont{} "--- бедствие; несчастье; тяжелый период в жизни. 




\bfseries Безквасный \normalfont{} "--- пресный; не кислый. 




\bfseries Безкнижный \normalfont{} "--- неученый. 




\bfseries Безлетно \normalfont{} "--- бесконечно; вечно; прежде всех времен. 




\bfseries Безматерен \normalfont{} "--- не имеющий матери. 




\bfseries Безмездник \normalfont{} "--- не принимающий мзды, платы. 




\bfseries Безмилостивный \normalfont{} "--- не чувствующий или не оказывающий милости, жалости. 




\bfseries Безмолвник \normalfont{} "--- пустынножитель; отшельник. 




\bfseries Безмолвный \normalfont{} "--- иногда значит: безопасный; спокойный. 




\bfseries Безневестный \normalfont{} "--- безбрачный; девственный. 




\bfseries Безочство \normalfont{} "--- нахальство; бесстыдство; наглость. 




\bfseries Безпрестани \normalfont{} "--- всегда; непрерывно. 




\bfseries Безпреткновенный \normalfont{} "--- не имеющий претыкания, соблазна, препятствия. 




\bfseries Безпутие \normalfont{} "--- совращение с пути; развращение. 




\bfseries Безсквернен \normalfont{} "--- не имеющий скверны или порока. 




\bfseries Безсловеснство \normalfont{} "--- скотство; глупость; безумие. 




\bfseries Безсловесные \normalfont{} "--- животные, скоты. 




\bfseries Безсребреник \normalfont{} "--- человек, трудящийся даром, бесплатно. 




\bfseries Безстудие \normalfont{} "--- бесстыдство. 




\bfseries Безцарный \normalfont{} "--- не имеющий над собой царя. 




\bfseries Безчадие \normalfont{} "--- неимение, лишение детей. 




\bfseries Безчаствовати \normalfont{} "--- лишать положенной части; обделять. 




\bfseries Безчестен \normalfont{} "--- обесчещенный. 




\bfseries Безчиние \normalfont{} "--- беспорядок; неустройство; смешение. 




\bfseries Безчинновати \normalfont{} "--- вести беспорядочную жизнь. 




\bfseries Безчисльство \normalfont{} "--- бесчисленное множество. 




\bfseries Бервенный \normalfont{} "--- деревянный. 




\bfseries Бесный \normalfont{} "--- бесноватый. 




\bfseries Бийца \normalfont{} "--- драчун. 




\bfseries Било \normalfont{} "--- колотушка, при помощи которой в монастырях созывают на молитву. 




\bfseries Бисер \normalfont{} "--- жемчуг. 




\bfseries Благий \normalfont{} "--- хороший; добрый. 




\bfseries Благовест \normalfont{} "--- удары колокола, созывающие христиан на молитву в храм. От 




\bfseries "звона" \normalfont{} отличается тем, что благовестят в один колокол, а звонят во многие. 




\bfseries Благовестити \normalfont{} "--- возвещать доброе; проповедовать. 




\bfseries Благоверный \normalfont{} "--- исповедующий правую веру; православный. 




\bfseries Благовещение \normalfont{} "--- добрая весть. 




\bfseries Благоволити \normalfont{} "--- хорошо относиться к кому-нибудь; принимать участие в ком-либо. 




\bfseries Благовоние \normalfont{} "--- благоухание, хороший запах. 




\bfseries Благовременне \normalfont{} "--- в удобное время. 




\bfseries Благогласник \normalfont{} "--- проповедник слова Божия. 




\bfseries Благодатный \normalfont{} "--- преисполненный Божественной благодати. 




\bfseries Благоделие \normalfont{} "--- доброе, богоугодное дело. 




\bfseries Благодушествовати \normalfont{} "--- радоваться. 




\bfseries Благоискусный \normalfont{} "--- имеющий знание в полезных вещах. 




\bfseries Благокласный \normalfont{} "--- приносящий обильную жатву. 




\bfseries Благоключимый \normalfont{} "--- случившийся вовремя. 




\bfseries Благокрасный \normalfont{} "--- очень красивый. 




\bfseries Благолепие \normalfont{} "--- красота; великолепие; богатое убранство. 




\bfseries Благолепно \normalfont{} "--- красиво; прилично. 




\bfseries Благолозный \normalfont{} "--- приносящий обильные, хорошие плоды. 




\bfseries Благолюбец \normalfont{} "--- склонный к добру. 




\bfseries Благомилостивый \normalfont{} "--- милосердный. 




\bfseries Благомощие \normalfont{} "--- крепость; сила. 




\bfseries Благомужство \normalfont{} "--- благоразумная храбрость, доблесть. 




\bfseries Благонаказательный \normalfont{} "--- направляющий к благонравию. 




\bfseries Благоодеждный \normalfont{} "--- украшенный изящной одеждой. 




\bfseries Благоотдатливый \normalfont{} "--- воздающий добром за зло. 




\bfseries Благопитание \normalfont{} "--- сладкая, вкусная пища. 




\bfseries Благопослушливый \normalfont{} "--- слушающий со вниманием; послушный. 




\bfseries Благопослушный \normalfont{} "--- внимательный; легко, хорошо слышимый. 




\bfseries Благопотребный \normalfont{} "--- хорошо устроенный; угодный; необходимый. 




\bfseries Благоразтворение \normalfont{} "--- очищение; оздоровление; прояснение. 




\bfseries Благоразтворити \normalfont{} "--- очищать; оздоровлять. 




\bfseries Благорасленный \normalfont{} "--- хорошо растущий. 




\bfseries Благорозгный \normalfont{} "--- ветвистый. 




\bfseries Благосеннолиственный \normalfont{} "--- тенистый. 




\bfseries Благосенный \normalfont{} "--- производящий обильную тень. 




\bfseries Благословенный \normalfont{} "--- прославляемый; восхваляемый; превозносимый. 




\bfseries Благословити \normalfont{} "--- посвятить Богу; желать добра; хвалить; помолиться о ниспослании Божией благодати на кого-либо; дозволить; пожелать добра. 




\bfseries Благословная вина \normalfont{} "--- уважительная причина. 




\bfseries Благостояние \normalfont{} "--- твердость, крепость (в добре, против зла). 




\bfseries Благостыня \normalfont{} "--- благодеяние; милосердие; добродетель, доброе дело. 




\bfseries Благость \normalfont{} "--- доброта. 




\bfseries Благотещи \normalfont{} "--- быстро идти. 




\bfseries Благоуветие \normalfont{} "--- снисхождение. 




\bfseries Благоуветливый \normalfont{} "--- снисходительный. 




\bfseries Благоутишие \normalfont{} "--- тихая, ясная погода. 




\bfseries Благоутробие \normalfont{} "--- милосердие. 




\bfseries Благохваление \normalfont{} "--- откровенная похвала. 




\bfseries Благоцветный \normalfont{} "--- испещренный; изобилующий цветами. 




\bfseries Благочествовати \normalfont{} "--- благоговеть; благоговейно почитать кого-либо. 




\bfseries Благочестие \normalfont{} "--- истинное Богопочитание. 




\bfseries Благочестивый, благочестный \normalfont{} "--- богобоязненный; благоговейный; почитающий Бога. 




\bfseries Блаженный \normalfont{} "--- счастливый. 




\bfseries Блажити \normalfont{} "--- ублажать; прославлять. 




\bfseries Блазнити \normalfont{} "--- соблазнять. 




\bfseries Блед, бледый \normalfont{} "--- бледный. 




\bfseries Блещатися \normalfont{} "--- блистать; сиять; светить. 




\bfseries Близна \normalfont{} "--- рубец; морщина; складка. 




\bfseries Блистание \normalfont{} "--- сверкание; излияние света, блеска. 




\bfseries Блудилище \normalfont{} "--- непотребный дом. 




\bfseries Блудодей \normalfont{} "--- нарушитель супружества. 




\bfseries Блудопитие \normalfont{} "--- побуждающая к блуду попойка. 




\bfseries Блужение \normalfont{} "--- неверность Богу истинному, служение идолам (Исх. 34, 15; Суд. 8, 33). Как нарушение брачного союза есть блудодеяние, так в духовном смысле и нарушение союза с Богом есть служение идолам, хождение во след богов иных, то есть блужение, тем более что некоторые виды идолослужения сопровождались блудом в собственном смысле слова. 




\bfseries Блюдомый \normalfont{} "--- сохраняемый. 




\bfseries Блюсти \normalfont{} "--- хранить; беречь; соблюдать. 




\bfseries Блядение \normalfont{} "--- суесловие; ложные слова; вранье. 




\bfseries Бо \normalfont{} "--- потому что; так как; ибо; поскольку. 




\bfseries Богатити \normalfont{} "--- обогащать. 




\bfseries Богоглаголивый \normalfont{} "--- говорящий по внушению от Бога или от Его Имени. 




\bfseries Богодельне \normalfont{} "--- по действию Бога. 




\bfseries Боголепно \normalfont{} "--- так, как прилично Богу. 




\bfseries Боголепный \normalfont{} "--- имеющий Божественную красоту, достоинство. 




\bfseries Богомужный \normalfont{} "--- Богочеловеческий. 




\bfseries Богоначальный \normalfont{} "--- имеющий в Боге свое начало. 




\bfseries Богоотец \normalfont{} "--- это название в церковных книгах усвояется Давиду, от рода которого родился Христос. 




\bfseries Бодренно \normalfont{} "--- бдительно; неусыпно. 




\bfseries Болезновати \normalfont{} "--- терпеть боль; страдать. 




\bfseries Болий \normalfont{} "--- больший. 




\bfseries Борзе \normalfont{} "--- скоро. 




\bfseries Борзитися \normalfont{} "--- спешить. 




\bfseries Боритель \normalfont{} "--- противник. 




\bfseries Брада \normalfont{} "--- борода. 




\bfseries Брадатый \normalfont{} "--- бородатый. 




\bfseries Бразда \normalfont{} "--- борозда. 




\bfseries Бракоокрадованная \normalfont{} "--- лишенная целомудрия, девственности. 




\bfseries Бранити \normalfont{} "--- запрещать; оборонять; препятствовать. 




\bfseries Брань \normalfont{} "--- война; битва. 




\bfseries Братися \normalfont{} "--- бороться; воевать. 




\bfseries Брашно \normalfont{} "--- пища; еда. 




\bfseries Бремя \normalfont{} "--- ноша; тяжесть. 




\bfseries Брение \normalfont{} "--- глина; грязь. 




\bfseries Бренный \normalfont{} "--- взятый из земли; слабый; непрочный. 




\bfseries Брещи \normalfont{} "--- стеречь; хранить. 




\bfseries Брозда \normalfont{} "--- удила (часть конской сбруи). 




\bfseries Бряцати \normalfont{} "--- звенеть. 




\bfseries Будильник \normalfont{} "--- один из монахов в обители, будящий на молитву братию. 




\bfseries Буесловие \normalfont{} "--- глупые речи; вранье. 




\bfseries Буесловити \normalfont{} "--- говорить глупые речи. 




\bfseries Буй, (буий) \normalfont{} "--- безумный; сумасшедший; глупый. 




\bfseries Буйство \normalfont{} "--- глупость; безумие; сумасшествие. 




\bfseries Былие \normalfont{} "--- трава. 




 





\bfseries Вавилонское семя \normalfont{} "--- племя нечестивцев. 




\bfseries Вага \normalfont{} "--- весы; тяжесть. 




\bfseries Вадити \normalfont{} "--- делать ложный донос; клеветать; обвинять; приманивать; привлекать. 




\bfseries Ваия \normalfont{} "--- ветви; листья. 




\bfseries Вайный \normalfont{} "--- состоящий из ваий. 




\bfseries Валсамный \normalfont{} "--- благовонный; пахучий; ароматический. 




\bfseries Вап (а) \normalfont{} "--- краска. 




\bfseries Вар \normalfont{} "--- зной; жара; кипяток. 




\bfseries Варити \normalfont{} "--- предварять; упреждать; опереживать, предостерегать. 




\bfseries Василиск \normalfont{} "--- большая ядовитая змея. 




\bfseries Вборзе \normalfont{} "--- скоро. 




\bfseries Ввергати \normalfont{} "--- вбрасывать. 




\bfseries Вдавати \normalfont{} "--- поручать; передавать; доверять. 




\bfseries Веглас \normalfont{} "--- знающий; искусный. 




\bfseries Ведети \normalfont{} "--- знать. 




\bfseries Ведунство \normalfont{} "--- волхвование; ворожба; чародейство. 




\bfseries Веельзевул \normalfont{} "--- "повелитель мух"; начальник злых духов; одно из имен сатаны. 




\bfseries Вежди \normalfont{} "--- веки; ресницы. 




\bfseries Веие \normalfont{} "--- ветвь; сучок. 




\bfseries Велелепие \normalfont{} "--- красота; великолепие; украшение. 




\bfseries Велелепота \normalfont{} "--- красота; великолепие; украшение. 




\bfseries Велемудренно \normalfont{} "--- высокомудренно. 




\bfseries Веление \normalfont{} "--- указ; повеление; заповедь; учение. 




\bfseries Велеречивый \normalfont{} "--- многословный; хвастливый. 




\bfseries Велеречити \normalfont{} "--- много говорить; хвастать; гордиться. 




\bfseries Велиар \normalfont{} (или 




\bfseries Велиал \normalfont{}) "--- одно из имен диавола. 




\bfseries Велий \normalfont{} "--- великий; сильный. 




\bfseries Великовыйный \normalfont{} "--- гордый. 




\bfseries Великое \normalfont{} "--- самый большой, главный колокол. 




\bfseries Великодарный, великодаровный, великодаровитый \normalfont{} "--- щедро награждающий. 




\bfseries Величатися \normalfont{} "--- гордиться; хвалиться; кичиться. 




\bfseries Велми \normalfont{} "--- весьма; очень. 




\bfseries Вельблуд, велбуд \normalfont{} "--- верблюд; толстый канат. 




\bfseries Вельблуждь \normalfont{} "--- верблюжий. 




\bfseries Вено \normalfont{} "--- плата жениха за невесту. 




\bfseries Венчати \normalfont{} "--- возлагать венок или венец; удостаивать; сподоблять. 




\bfseries Вепрь \normalfont{} "--- дикий кабан. 




\bfseries Вербие \normalfont{} "--- ива; лоза. 




\bfseries Вервица \normalfont{} "--- четки. 




\bfseries Верея \normalfont{} "--- дверь; перекладина; столб у ворот. 




\bfseries Вержение \normalfont{} "--- кидание; метание; бросание. 




\bfseries Вержение камени \normalfont{} "--- расстояние, равное тому, на какое можно бросить камень. 




\bfseries Верзити \normalfont{} "--- кинуть. 




\bfseries Верзитися \normalfont{} "--- упасть. 




\bfseries Вериги \normalfont{} "--- цепи; оковы. 




\bfseries Верт, вертоград \normalfont{} "--- сад. 




\bfseries Вертеп \normalfont{} "--- пещера. 




\bfseries Вертоградарь \normalfont{} "--- садовник. 




\bfseries Верху \normalfont{} "--- на; над; сверху. 




\bfseries Весь \normalfont{} "--- селение, деревушка. 




\bfseries Ветия \normalfont{} "--- оратор; ритор. 




\bfseries Ветрило \normalfont{} "--- парус. 




\bfseries Ветхий деньми \normalfont{} "--- имя Божие в Дан. 7, 9. На основании этого пророческого видения в иконографической традиции новозаветной Церкви образ Бога Отца изображается в виде старца. 




\bfseries Вечеря \normalfont{} "--- ужин; пир. 




\bfseries Вечеряти \normalfont{} "--- ужинать. 




\bfseries Вещенеистовный \normalfont{} "--- пристрастившийся до безумия к тленным благам. 




\bfseries Вещь \normalfont{} "--- дело; событие. 




\bfseries Взаим \normalfont{} "--- в долг; взаймы. 




\bfseries Взиматися \normalfont{} "--- подниматься. 




\bfseries Взыгратися \normalfont{} "--- играть; скакать; веселиться. 




\bfseries Взыскати \normalfont{} "--- стремиться, искать. 




\bfseries Взятися \normalfont{} "--- взяться; отвориться; совершиться. 




\bfseries Вина \normalfont{} "--- причина; обвинение; извинение. 




\bfseries Винарь \normalfont{} "--- виноградарь. 




\bfseries Винничина \normalfont{} "--- виноградная лоза. 




\bfseries Винопийца \normalfont{} "--- пьяница. 




\bfseries Винопитие \normalfont{} "--- употребление вина. 




\bfseries Висети \normalfont{} "--- висеть; держаться на чем-либо. 




\bfseries Виссон \normalfont{} "--- драгоценная тонкая пряжа желтоватого цвета или одежда из этой ткани. 




\bfseries Виталище \normalfont{} "--- место жительства; жилище. 




\bfseries Виталница \normalfont{} "--- комната; гостиница; постоялый двор; ночлег. 




\bfseries Витати \normalfont{} "--- обитать; пребывать; проживать; ночевать. 




\bfseries Вкупе \normalfont{} "--- вместе. 




\bfseries Влагалище \normalfont{} "--- мешок; карман; ларец. 




\bfseries Владычный \normalfont{} "--- господский или Господний. 




\bfseries Владящий \normalfont{} "--- обладающий; господствующий. 




\bfseries Власти \normalfont{} "--- имя одного из чинов ангельских. 




\bfseries Власяница \normalfont{} "--- одежда из жесткого, колючего волоса. 




\bfseries Влаятися \normalfont{} "--- мыкаться; колебаться; волноваться; носиться по волнам. 




\bfseries Влещи \normalfont{} "--- тащить. 




\bfseries Влещися \normalfont{} "--- брести; медленно идти; тащиться. 




\bfseries Вмале \normalfont{} "--- вскоре; немного спустя; почти; едва. 




\bfseries Вне(уду) \normalfont{} "--- извне; снаружи. 




\bfseries Внегда \normalfont{} "--- когда. 




\bfseries Внезапу \normalfont{} "--- вдруг; неожиданно. 




\bfseries Внемшийся \normalfont{} "--- загоревшийся. 




\bfseries Внитие \normalfont{} "--- вхождение; явление; пришествие. 




\bfseries Внове \normalfont{} "--- недавно. 




\bfseries Внутрь (уду) \normalfont{} "--- внутри. 




\bfseries Вняти \normalfont{} "--- обратить внимание; услышать. 




\bfseries Вняти от \normalfont{} "--- остерегаться. 




\bfseries Внятися \normalfont{} "--- загореться. 




\bfseries Воврещи \normalfont{} "--- бросить во что-либо; ввергнуть; внести. 




\bfseries Водрузити \normalfont{} "--- утвердить; укрепить. 




\bfseries Во еже \normalfont{} "--- чтобы; ради; для. 




\bfseries Вожделети \normalfont{} "--- сильно желать. 




\bfseries Возбеситися \normalfont{} "--- сделаться неистовым. 




\bfseries Возбранный, взбранный \normalfont{} "--- военный; храбрый в бранях; победоносный. 




\bfseries Возбраняти \normalfont{} "--- препятствовать; удерживать. 




\bfseries Возбряцати \normalfont{} "--- воспеть; хвалить в песнях. 




\bfseries Возвлачити \normalfont{} "--- затащить наверх. 




\bfseries Возврещи, возвергати \normalfont{} "--- возложить, возлагать. 




\bfseries Возглавие \normalfont{} "--- подушка; изголовье. 




\bfseries Возглас \normalfont{} "--- окончательные слова молитвы, тайно творимой священником. 




\bfseries Возглашение \normalfont{} "--- громкое пение или чтение; Возглас. 




\bfseries Возглядати \normalfont{} "--- взирать; смотреть. 




\bfseries Возгнещати \normalfont{} "--- разводить огонь. 




\bfseries Воздвигнути \normalfont{} "--- поднять; возвысить. 




\bfseries Воздвижение \normalfont{} "--- поднятие, возвышение. 




\bfseries Воздвизатися \normalfont{} "--- иногда: отправляться в путь. 




\bfseries Воздвизати \normalfont{} "--- поднимать; возвышать. 




\bfseries Воздеяние \normalfont{} "--- поднятие, возвышение. 




\bfseries Воздух \normalfont{} "--- покровец, полагаемый сверху священных сосудов на Литургии. 




\bfseries Возлежати \normalfont{} "--- лежать облокотившись; полулежать. 




\bfseries Возмущение \normalfont{} "--- смятение; бунт. 




\bfseries Возмятати \normalfont{} "--- возмущать; производить раздор. 




\bfseries Возниспослати \normalfont{} "--- послать свыше; наградить. 




\bfseries Возничати \normalfont{} "--- поднять голову. 




\bfseries Возобразитися \normalfont{} "--- принять образ; олицетвориться; вселиться в видимый образ; вновь быть изображенным. 




\bfseries Возраст \normalfont{} "--- возраст (число лет); рост. 




\bfseries Возрастити \normalfont{} "--- вырастить; увеличить. 




\bfseries Возреяти \normalfont{} "--- поколебать; потрясти. 




\bfseries Возставити \normalfont{} "--- восстановить; поставить на прежнее место. 




\bfseries Волити \normalfont{} "--- хотеть; желать; требовать. 




\bfseries Волна \normalfont{} "--- шерсть; руно; овчина. 




\bfseries Волхв \normalfont{} "--- мудрец; звездочет; чародей; предсказатель. 




\bfseries Волчец \normalfont{} "--- колючая трава. 




\bfseries Вонь \normalfont{} "--- в него. 




\bfseries Воня \normalfont{} "--- запах; курение. 




\bfseries Воня злая \normalfont{} "--- смрад. 




\bfseries Вопити, вопияти \normalfont{} "--- громко кричать; взывать. 




\bfseries Ворожити \normalfont{} "--- колдовать; предсказывать будущее. 




\bfseries Ворожея \normalfont{} "--- волшебник; колдун; отравитель. 




\bfseries Воскликновение \normalfont{} "--- хоровое пение. 




\bfseries Восклонятися \normalfont{} "--- выпрямляться; подниматься; разгибаться. 




\bfseries Воскресати \normalfont{} "--- восставать; оживать; возвращаться к жизни. 




\bfseries Воскресение \normalfont{} "--- восстание из мертвых. 




\bfseries Воскрешати \normalfont{} "--- оживлять. 




\bfseries Воскрилие \normalfont{} "--- подол; край одежды; пола верхней одежды. 




\bfseries Восперяти \normalfont{} "--- оперять; окрылять (надеждой). 




\bfseries Восписовати \normalfont{} "--- изображать письменно; изъявлять. 




\bfseries Воспросити \normalfont{} "--- попросить. 




\bfseries Воспрянути \normalfont{} "--- вскочить; вспрыгнуть; приходить в себя. 




\bfseries Востерзати \normalfont{} "--- извлекать; выдергивать. 




\bfseries Восторгати \normalfont{} "--- рвать; щипать; полоть. 




\bfseries Востягнути \normalfont{} "--- подтянуть; укрепить; подтащить. 




\bfseries Востязати, востязовати \normalfont{} "--- исследовать; испытывать; интересоваться. 




\bfseries Восхитити \normalfont{} "--- изловить; поймать; не законно захватить; похитить; увлечь в высоту; привести в восторг. 




\bfseries Вотще \normalfont{} "--- понапрасну; впустую; даром; тщетно. 




\bfseries Воутрие, воутрий \normalfont{} "--- на другой день. 




\bfseries Впасти \normalfont{} "--- упасть; попасть; ввалиться; подвергнуться; подпасть. 




\bfseries Вперити \normalfont{} "--- возвысить; поднять; устремить вверх наподобие пера. 




\bfseries Вперитися \normalfont{} "--- воспарить; взлететь. 




\bfseries Вперсити \normalfont{} "--- внутрь себя принять. 




\bfseries Впрямо \normalfont{} "--- прямо; напротив. 




\bfseries Врабий \normalfont{} "--- воробей. 




\bfseries Вран \normalfont{} "--- ворон. 




\bfseries Врата красныя \normalfont{} "--- западные церковные двери. 




\bfseries Вратарь, вратник \normalfont{} "--- сторож у ворот. 




\bfseries Врачба \normalfont{} "--- лекарство; врачевание. 




\bfseries Врачебница \normalfont{} "--- больница. 




\bfseries Вреды \normalfont{} "--- кожное заболевание. 




\bfseries Вресноту \normalfont{} "--- вправду; по достоинству; пристойно. 




\bfseries Вретище \normalfont{} "--- плохая, грубая одежда; дерюга; скорбное одеяние. 




\bfseries Вреяти \normalfont{} "--- кипеть; пениться; разгорячаться; бить ключом; выкипать. 




\bfseries Вреяти \normalfont{} "--- ввергать; вметать; вталкивать. 




\bfseries Всеблаголепный \normalfont{} "--- великолепный. 




\bfseries Всевидно \normalfont{} "--- всенародно. 




\bfseries Вседетельный \normalfont{} "--- все создавший. 




\bfseries Всеконечне \normalfont{} "--- совершенно. 




\bfseries Всекрасный \normalfont{} "--- самый красивый. 




\bfseries Вселукавый \normalfont{} "--- самый коварный, т.е. диавол. 




\bfseries Всеоружие \normalfont{} "--- полное вооружение. 




\bfseries Всепетый \normalfont{} "--- препрославленный; всеми или всюду восхваляемый. 




\bfseries Всесожжение \normalfont{} "--- жертвоприношение, при котором жертва сжигалась целиком. 




\bfseries Всечреждение \normalfont{} "--- богатое угощение. 




\bfseries Всеядец \normalfont{} "--- тот, кто всех поедает, т.е. ад, или смерть. 




\bfseries Всеяти \normalfont{} "--- посеять. 




\bfseries Вскрай \normalfont{} "--- по краю; близ; около. 




\bfseries Вскую \normalfont{} "--- почему? из-за чего? за что? 




\bfseries Вспять \normalfont{} "--- назад. 




\bfseries Всуе \normalfont{} "--- напрасно. 




\bfseries Всяко \normalfont{} "--- совершенно; совсем; вовсе. 




\bfseries Вторицею \normalfont{} "--- вторично; усиленно. 




\bfseries Вчиняти \normalfont{} "--- учреждать; узаконять. 




\bfseries Выну \normalfont{} "--- всегда. 




\bfseries Выспренний \normalfont{} "--- высокий; гордый. 




\bfseries Выспрь \normalfont{} "--- вверх. 




\bfseries Высоковыйный \normalfont{} "--- гордый; надменный; кичливый. 




\bfseries Высокомудрствовати \normalfont{} "--- высокоумствовать; гордиться. 




\bfseries Вышелетный \normalfont{} "--- предвечный. 




\bfseries Выя \normalfont{} "--- шея. 




\bfseries Вящший \normalfont{} "--- больший. 




\bfseries Вящше \normalfont{} "--- больше. 




 





\bfseries Гаггрена \normalfont{} (читается "гангрена") "--- гангрена; антонов огонь; рак. 




\bfseries Гадание \normalfont{} "--- загадка; неясность. 




\bfseries Гади \normalfont{} (мн. ч.) "--- пресмыкающиеся. 




\bfseries Гаждение \normalfont{} "--- поношение; бесчестие; ругань. 




\bfseries Газофилакия \normalfont{} "--- казнохранилище во храме Иерусалимском. 




\bfseries Ганание \normalfont{} "--- загадка; притча. 




\bfseries Гастримаргия \normalfont{} "--- чревобесие; обжорство. 




\bfseries Гащи \normalfont{} "--- штаны; нижнее мужское белье. 




\bfseries Геенна \normalfont{} "--- долина Гинном около Иерусалима, где идолопоклонствующие иудеи при царе Ахазе сжигали своих детей в честь идола Молоха. Иносказательно: место будущих мучений, загробных наказаний. 




\bfseries Генварь \normalfont{} "--- январь. 




\bfseries Гибель \normalfont{} "--- трата; расход. 




\bfseries Главизна \normalfont{} "--- глава; начало; причина. 




\bfseries Главотяж \normalfont{} "--- головная повязка у иудеев. 




\bfseries Глагол \normalfont{} "--- слово; речь. 




\bfseries Глаголати \normalfont{} "--- говорить; рассказывать. 




\bfseries Глаголемый \normalfont{} "--- называемый; так называемый. 




\bfseries Глас \normalfont{} "--- голос; напев. 




\bfseries Глезна \normalfont{} "--- голень; ступня. 




\bfseries Глоба \normalfont{} "--- казнь; наказание. 




\bfseries Глумец \normalfont{} "--- кощун; пересмешник. 




\bfseries Глумилище \normalfont{} "--- место для скачек, плясок, маскарадов и т. п. 




\bfseries Глумитися \normalfont{} "--- забавляться; тешиться; получать удовольствие. 




\bfseries Глумы \normalfont{} (мн. ч.) "--- шутки; смех; игры. 




\bfseries Глядати \normalfont{} "--- смотреть; глядеть. 




\bfseries Гнати \normalfont{} "--- гнать; преследовать; идти; следовать за кем или чем-либо. 




\bfseries Гной, гноище \normalfont{} "--- навозная куча; раны. 




\bfseries Гнушатися \normalfont{} "--- считать гнусным; презирать. 




\bfseries Гобзование \normalfont{} "--- изобилие, довольство. 




\bfseries Гобзовати \normalfont{} "--- изобиловать; избыточествовать; быть богатым. 




\bfseries Гобзующий \normalfont{} "--- живущий в довольстве. 




\bfseries Говение \normalfont{} "--- почитание (например, поста). 




\bfseries Говети \normalfont{} "--- чтить; почитать (например, пост). 




\bfseries Говядо \normalfont{} "--- рогатый скот. 




\bfseries Годе \normalfont{} "--- угодно; приятно; подходяще. 




\bfseries Година, год \normalfont{} "--- час; время; пора. 




\bfseries Голоть \normalfont{} "--- гололедица, лед. 




\bfseries Гомола \normalfont{} "--- ком; комок; катыш; кусок. 




\bfseries Гонзати \normalfont{} "--- убегать; спасаться бегством. 




\bfseries Гонзнутие \normalfont{} "--- избежание. 




\bfseries Гонитель \normalfont{} "--- преследователь. 




\bfseries Горе \normalfont{} "--- ввысь; вверх. 




\bfseries Горее \normalfont{} "--- хуже; бедственнее. 




\bfseries Горлица, горличишь \normalfont{} "--- дикий голубь. 




\bfseries Горнец \normalfont{} "--- горшок; котелок; жаровня. 




\bfseries Горнило \normalfont{} "--- кузнечный горн; плавильня; место для плавки или очищения огнем. 




\bfseries Горница \normalfont{} "--- верхняя комната; столовая. 




\bfseries Горний \normalfont{} "--- высокий; вышний; небесный. 




\bfseries Горохищный \normalfont{} "--- пасущийся; блуждающий по горам; украденный диким зверем с горного пастбища. 




\bfseries Гортанобесие \normalfont{} "--- пристрастие к лакомствам. 




\bfseries Горушный \normalfont{} "--- горчичный. 




\bfseries Горший \normalfont{} "--- злейший; худший. 




\bfseries Господствия \normalfont{} "--- один из чинов ангельских. 




\bfseries Господыня \normalfont{} "--- госпожа. 




\bfseries Гостинник \normalfont{} "--- корчмарь; содержатель постоялого двора. 




\bfseries Градарь \normalfont{} "--- садовник; огородник. 




\bfseries Градеж \normalfont{} "--- оплот; забор. 




\bfseries Грезн \normalfont{} "--- гроздь виноградная. 




\bfseries Гривна \normalfont{} "--- ожерелье; носимая на шее цепь. 




\bfseries Гроздие \normalfont{} "--- кисть плодов; ветвь (винограда). 




\bfseries Грясти \normalfont{} "--- идти; шествовать. 




\bfseries Гугнивый \normalfont{} "--- гнусавый; заика; косноязычный; картавый; говорящий в нос. 




\bfseries Гудение \normalfont{} "--- игра на гуслях или арфе. 




\bfseries Гудец \normalfont{} "--- гуслист; музыкант. 




 





\bfseries Да \normalfont{} "--- пусть; чтобы. 




\bfseries Дабы \normalfont{} "--- чтобы. 




\bfseries Далечайше \normalfont{} "--- гораздо далее. 




\bfseries Далечен \normalfont{} "--- далекий; трудный. 




\bfseries Даннословие \normalfont{} "--- обещание; обязательство. 




\bfseries Двакраты \normalfont{} "--- дважды. 




\bfseries Дващи \normalfont{} "--- дважды. 




\bfseries Дверь адова \normalfont{} "--- смерть. 




\bfseries Двизати \normalfont{} "--- двигать; шевелить. 




\bfseries Двоедушный \normalfont{} "--- нетвердый в вере. 




\bfseries Двоица \normalfont{} "--- пара. 




\bfseries Дворище \normalfont{} "--- небольшой или запустелый дом. 




\bfseries Дебельство \normalfont{} "--- тучность; полнота; дородность. 




\bfseries Дебрь \normalfont{} "--- долина; ложбина; овраг; ущелье. 




\bfseries Девствовати \normalfont{} "--- хранить девство, целомудрие. 




\bfseries Действо \normalfont{} "--- действие; представление. 




\bfseries Декемврий \normalfont{} "--- декабрь. 




\bfseries Делатилище \normalfont{} "--- купеческая лавка; орудие в чьих-либо руках. 




\bfseries Делва \normalfont{} "--- бочка; кадка. 




\bfseries Делма \normalfont{} "--- для. 




\bfseries Деля \normalfont{} "--- для; ради. 




\bfseries Демественник \normalfont{} "--- певчий. 




\bfseries Демоноговение \normalfont{} "--- почитание бесов. 




\bfseries Демоночтец \normalfont{} "--- идолопоклонник. 




\bfseries Денница \normalfont{} "--- утренняя заря, утренняя звезда; отпадший ангел. 




\bfseries Денносветлый \normalfont{} "--- подобный дневному свету. 




\bfseries Держава \normalfont{} "--- сила; крепость; власть; государство. 




\bfseries Державно \normalfont{} "--- властно; могущественно. 




\bfseries Дерзать \normalfont{} "--- осмеливаться; полагаться. 




\bfseries Дерзновение \normalfont{} "--- смелость. 




\bfseries Дерзословие \normalfont{} "--- наглая речь. 




\bfseries Дерзостник \normalfont{} "--- наглец; нахал. 




\bfseries Дерзый \normalfont{} "--- смелый; бесстыдный; дерзкий. 




\bfseries Десница \normalfont{} "--- правая рука. 




\bfseries Десный \normalfont{} "--- правый; находящийся с правой стороны. 




\bfseries Десятина \normalfont{} "--- десятая часть. 




\bfseries Десятословие \normalfont{} "--- десять заповедей Божиих, данных через Моисея. 




\bfseries Детищ \normalfont{} "--- младенец; дитя; отроча. 




\bfseries Детосаждение \normalfont{} "--- зачатие во утробе младенца. 




\bfseries Диадима \normalfont{} "--- венец; диадема. 




\bfseries Дивий \normalfont{} "--- дикий; лесной. 




\bfseries Дивьячитися \normalfont{} "--- зверствовать. 




\bfseries Дидрахма \normalfont{} "--- греч.двойная драхма", древнегреч. серебряная монета. 




\bfseries Динарий \normalfont{} "--- монета. 




\bfseries Длань \normalfont{} "--- ладонь. 




\bfseries Дмение \normalfont{} "--- гордость. 




\bfseries Дмитися \normalfont{} "--- гордиться; кичиться. 




\bfseries Дне \normalfont{} "--- относящийся к числу песнопений из Октоиха, а в дни пения Триоди "--- из этой книги. 




\bfseries Дненощно \normalfont{} "--- в течение целых суток. 




\bfseries Днесь \normalfont{} "--- сегодня, ныне; теперь. 




\bfseries Днешний \normalfont{} "--- нынешний; сегодняшний. 




\bfseries Доблий, добльственный, доблестный \normalfont{} "--- крепкий в добре; твердый в добродетели. 




\bfseries Доброзрачие \normalfont{} "--- красота; благообразие. 




\bfseries Доброкласный \normalfont{} "--- приносящий обильную жатву. 




\bfseries Добропобедный \normalfont{} "--- прославленный победами. 




\bfseries Доброта \normalfont{} "--- красота. 




\bfseries Доброхвальный \normalfont{} "--- заслуживающий похвалы; похвальный. 




\bfseries Довлесотворити \normalfont{} "--- удовлетворить. 




\bfseries Довлети \normalfont{} "--- доставать; быть достаточным; хватать. 




\bfseries Доволний \normalfont{} "--- достаточный. 




\bfseries Догмат \normalfont{} "--- греч. одно из основных положений веры. 




\bfseries Дождити \normalfont{} "--- посылать дождь; кропить; орошать. 




\bfseries Дозде \normalfont{} "--- доселе; до сего дня; досюда. 




\bfseries Дозела \normalfont{} "--- чрезвычайно. 




\bfseries Доилица \normalfont{} "--- кормилица; мамка. 




\bfseries Доити \normalfont{} "--- кормить грудью. 




\bfseries Доколе \normalfont{} "--- до какого времени? долго ли? 




\bfseries Долний \normalfont{} "--- нижний; земной (как противоп. "небесный, горний"). 




\bfseries Долу, доле \normalfont{} "--- внизу; вниз. 




\bfseries Долувлекущий \normalfont{} "--- тянущий вниз. 




\bfseries Дондеже \normalfont{} "--- пока. 




\bfseries Донележе \normalfont{} "--- пока. 




\bfseries Дориносити \normalfont{} "--- сопровождать кого-либо в качестве стражи, свиты. 




\bfseries Досаждение \normalfont{} "--- делание неугодного; нечестие; оскорбление. 




\bfseries Достижно \normalfont{} "--- понятно. 




\bfseries Достояние \normalfont{} "--- имение; наследство; власть. 




\bfseries Драхма \normalfont{} "--- древнегреч. серебряная монета. 




\bfseries Драчие \normalfont{} "--- сорная трава. 




\bfseries Древле \normalfont{} "--- давно. 




\bfseries Древодель \normalfont{} "--- плотник; столяр. 




\bfseries Дреколие \normalfont{} "--- колья. 




\bfseries Дрождие \normalfont{} "--- дрожжи; отстой. 




\bfseries Другиня \normalfont{} "--- подруга. 




\bfseries Дружина \normalfont{} "--- общество (товарищей, сверстников). 




\bfseries Дручити \normalfont{} "--- удручать; томить; изнурять. 




\bfseries Дряселовати \normalfont{} "--- быть пасмурным, мрачным, печальным. 




\bfseries Дряхлование \normalfont{} "--- печаль. 




\bfseries Дряхлый \normalfont{} "--- печальный. 




\bfseries Дска, дщица \normalfont{} "--- доска; дощечка. 




\bfseries Дуга \normalfont{} "--- радуга. 




\bfseries Дхнути \normalfont{} "--- дохнуть; дунуть. 




\bfseries Дщи, дщерь \normalfont{} "--- дочь. 




 





\bfseries Евнух \normalfont{} "--- скопец; сторож при гареме; придворный. 




\bfseries Егда \normalfont{} "--- когда. 




\bfseries Егов \normalfont{} "--- его (притяжательный падеж от местоимения "он"). 




\bfseries Еда \normalfont{} "--- разве? неужели? 




\bfseries Едем \normalfont{} "--- Эдем; рай земной. 




\bfseries Единако \normalfont{} "--- согласно; одинаково. 




\bfseries Единаче \normalfont{} "--- одинаково; равно; еще. 




\bfseries Единаче ли \normalfont{} "--- неужели еще? 




\bfseries Единовидный \normalfont{} "--- одновидный; однообразный. 




\bfseries Единою \normalfont{} "--- однажды. 




\bfseries Еже \normalfont{} "--- что; кое. 




\bfseries Езеро \normalfont{} "--- озеро. 




\bfseries Ей \normalfont{} "--- да; истинно; верно. 




\bfseries Ексапсалмы \normalfont{} "--- шестопсалмие. 




\bfseries Ектения \normalfont{} "--- усиленное моление; прошение. 




\bfseries Елей \normalfont{} "--- оливковое, деревянное масло. 




\bfseries Елень \normalfont{} "--- олень; лань. 




\bfseries Елеонский \normalfont{} "--- оливковый. 




\bfseries Елижды аще \normalfont{} "--- когда бы ни. 




\bfseries Елижды, еликожды \normalfont{} "--- всегда как; всякий раз, когда. 




\bfseries Еликий \normalfont{} "--- кто; который. 




\bfseries Елико \normalfont{} "--- сколько. 




\bfseries Елико-елико \normalfont{} "--- через короткое время; очень скоро. 




\bfseries Еликомощно \normalfont{} "--- по возможности; сколько дозволяют силы. 




\bfseries Еллин \normalfont{} "--- грек; язычник; прозелит иудаизма. 




\bfseries Елма \normalfont{} "--- поскольку; насколько. 




\bfseries Епендит \normalfont{} "--- верхнее платье. 




\bfseries Епистолия \normalfont{} "--- письмо; послание. 




\bfseries Еродий \normalfont{} "--- цапля. 




\bfseries Есмирнисменный \normalfont{} "--- смешанный вместе со смирной. 




\bfseries Ехидна \normalfont{} "--- ядовитая змея. 




 





\bfseries Жаждати \normalfont{} "--- хотеть пить; сильно желать чего-либо. 




\bfseries Жалость \normalfont{} "--- ревность; рвение. 




\bfseries Жатель \normalfont{} "--- жнец. 




\bfseries Жегомый \normalfont{} "--- тот, кого жгут огнем; больной огнем; больной огневицей, горячкой. 




\bfseries Жезл \normalfont{} "--- посох; трость; палка. 




\bfseries Женитва \normalfont{} "--- бракосочетание; супружество; брак. 




\bfseries Женонеистовый \normalfont{} "--- похотливый; блудный; сластолюбивый. 




\bfseries Жестоковыйный \normalfont{} "--- бесчувственный; упрямый. 




\bfseries Живити \normalfont{} "--- животворить; давать жизнь; оживлять. 




\bfseries Живодавец \normalfont{} "--- податель жизни. 




\bfseries Живоначалие \normalfont{} "--- начало; причина жизни. 




\bfseries Живот \normalfont{} "--- жизнь. 




\bfseries Животный \normalfont{} "--- живущий; одушевленный. 




\bfseries Жребя \normalfont{} "--- жеребенок. 




\bfseries Жрети \normalfont{} "--- заколать; приносить жертвоприношение. 




\bfseries Жупел \normalfont{} "--- горячая сера. 




 





\bfseries Забавати \normalfont{} "--- заговаривать; заколдовывать. 




\bfseries Забавление \normalfont{} "--- промедление; мешкание; ожидание. 




\bfseries Забавляти \normalfont{} "--- удерживать; замедлять. 




\bfseries Забобоны \normalfont{} "--- самовольная служба, бесчиние. 




\bfseries Забрало \normalfont{} "--- стена; забор. 




\bfseries Завет \normalfont{} "--- союз; договор; условие. 




\bfseries Завида \normalfont{} "--- зависть. 




\bfseries Завистно \normalfont{} "--- мало; недостаточно. 




\bfseries За еже \normalfont{} "--- для того, чтобы. 




\bfseries Заздати \normalfont{} "--- загородить. 




\bfseries Зазрети \normalfont{} "--- заглянуть; заметить; осудить; упрекнуть. 




\bfseries Заимование \normalfont{} "--- заем; долг. 




\bfseries Заимовати \normalfont{} "--- занимать; заимствовать. 




\bfseries Заклание \normalfont{} "--- жертвоприношение. 




\bfseries Заклеп \normalfont{} "--- запор; замок; задвижка. 




\bfseries Заколение \normalfont{} "--- жертвоприношение. 




\bfseries Законописец \normalfont{} "--- составитель законов. 




\bfseries Законополагати \normalfont{} "--- давать закон. 




\bfseries Закров \normalfont{} "--- место для укрытия. 




\bfseries Залещи \normalfont{} "--- быть в засаде; скрываться. 




\bfseries Заматорети \normalfont{} "--- устареть; зачерстветь; состариться. 




\bfseries Замреженый \normalfont{} "--- пойманный в сети. 




\bfseries Зане \normalfont{} "--- так как; потому что. 




\bfseries Занеже \normalfont{} "--- поскольку. 




\bfseries Зань \normalfont{} "--- за него. 




\bfseries Запаление \normalfont{} "--- загорание; пожар. 




\bfseries Запев \normalfont{} "--- краткий стих, предваряющий стихиры (на "Господи, воззвах", хвалитны, стиховны) или тропари канона. 




\bfseries Запечатствовати \normalfont{} "--- запечатать; утвердить; связать; скрепить. 




\bfseries Запинание \normalfont{} "--- враждебное действие. 




\bfseries Запойство \normalfont{} "--- пьянство. 




\bfseries Запона \normalfont{} "--- завеса. 




\bfseries Запрение \normalfont{} "--- отрицание; запирание. 




\bfseries Запретити \normalfont{} "--- запретить; опечалиться; скорбеть. 




\bfseries Запустение \normalfont{} "--- опустение; пустыня. 




\bfseries Запустети \normalfont{} "--- придти в запущение или запустение, запустеть. 




\bfseries Запяти, запнути \normalfont{} "--- остановить; задержать; обольститься. 




\bfseries Запятие \normalfont{} "--- препинание; препятствие; преткновение. 




\bfseries Заревидный \normalfont{} "--- подобный заре. 




\bfseries Зарелучный \normalfont{} "--- лучезарный. 




\bfseries Застояти \normalfont{} "--- останавливать на дороге; удерживать; наскучивать; утруждать. 




\bfseries За ся \normalfont{} "--- за себя. 




\bfseries Затвор \normalfont{} "--- замок; запор; место молитвенного подвига некоторых иноков, давших обет не исходить из своей келлии. 




\bfseries Заточаемый \normalfont{} "--- обуреваемый ветром; носимый; гонимый. 




\bfseries Затулити \normalfont{} "--- закрыть; спрятать; укрыть. 




\bfseries Затуне \normalfont{} "--- даром; без причины. 




\bfseries Зауститися \normalfont{} "--- закрыть уста; замолчать. 




\bfseries Заутра \normalfont{} "--- до восхода солнца; поутру; рано; завтра. 




\bfseries Заутрие \normalfont{} "--- завтрашний день. 




\bfseries Заушение \normalfont{} "--- пощечина; удар рукой по лицу. 




\bfseries Заушати \normalfont{} "--- заграждать уста; запрещать говорить. 




\bfseries Захленутися \normalfont{} "--- погрузиться. 




\bfseries Заходный \normalfont{} "--- западный. 




\bfseries Зачало \normalfont{} "--- начало; название отрезков текста в книгах Священного Писания Нового Завета. 




\bfseries Заяти \normalfont{} "--- взять взаймы; занять. 




\bfseries Звездоблюститель \normalfont{} "--- астроном. 




\bfseries Звездоволхвовати \normalfont{} "--- гадать по звездам; заниматься астрологией. 




\bfseries Звездозаконие \normalfont{} "--- астрономия. 




\bfseries Звездослов \normalfont{} "--- астролог. 




\bfseries Звездословие \normalfont{} "--- астрология. 




\bfseries Звездословити \normalfont{} "--- заниматься астрологией. 




\bfseries Звероядина \normalfont{} "--- скот, поврежденный хищным зверем. 




\bfseries Звиздание \normalfont{} "--- свист; посвист. 




\bfseries Звиздати \normalfont{} "--- свистеть. 




\bfseries Звонец \normalfont{} "--- колокольчик. 




\bfseries Звонница \normalfont{} "--- колокольня. 




\bfseries Звяцати \normalfont{} "--- звенеть; бренчать. 




\bfseries Здати \normalfont{} "--- строить. 




\bfseries Зде \normalfont{} "--- здесь. 




\bfseries Здо \normalfont{} "--- здание; стена; крыша. 




\bfseries Зелейник \normalfont{} "--- знахарь, лечащий травами и заговором. 




\bfseries Зелейничество \normalfont{} "--- напоение отравой. 




\bfseries Зелейный \normalfont{} "--- состоящий из зелия, т.е. травы или других растений. 




\bfseries Зелие \normalfont{} "--- трава; растение. 




\bfseries Зело, зельне \normalfont{} "--- весьма; очень сильно. 




\bfseries Зельный \normalfont{} "--- сильный; великий. 




\bfseries Земен \normalfont{} "--- земной. 




\bfseries Земстий \normalfont{} "--- земной. 




\bfseries Зеница \normalfont{} "--- зрачок в глазе. 




\bfseries Зепь \normalfont{} "--- карман; мешок. 




\bfseries Зерцало \normalfont{} "--- зеркало. 




\bfseries Зиждитель \normalfont{} "--- создатель; творец. 




\bfseries Зиждити \normalfont{} "--- строить. 




\bfseries Зима \normalfont{} "--- зима; холод; плохая погода. 




\bfseries Злак \normalfont{} "--- растение; зелень; овощ. 




\bfseries Златарь \normalfont{} "--- золотых дел мастер. 




\bfseries Златица, златница \normalfont{} "--- золотая монета. 




\bfseries Златозарный \normalfont{} "--- яркоблестящий. 




\bfseries Злато \normalfont{} "--- золото. 




\bfseries Златокованный \normalfont{} "--- отчеканенный из золота. 




\bfseries Златокровный \normalfont{} "--- имеющий позлащенную крышу. 




\bfseries Злачный \normalfont{} "--- травный; богатый растительностью, злаками. 




\bfseries Зле \normalfont{} "--- зло; жестоко; худо. 




\bfseries Злоба \normalfont{} "--- забота. 




\bfseries Злокозненный \normalfont{} "--- исполненный злобы; лукавства. 




\bfseries Злокоман \normalfont{} "--- злодей; зложелатель; враг. 




\bfseries Злонравие \normalfont{} "--- развратный или дурной нрав. 




\bfseries Злообстояние \normalfont{} "--- беда ;несчастье. 




\bfseries Злопомнение \normalfont{} "--- злопамятство. 




\bfseries Злоречети \normalfont{} "--- бранить; ругать; злословить; поносить. 




\bfseries Злосердный \normalfont{} "--- безжалостный. 




\bfseries Злосмрадие \normalfont{} "--- зловоние. 




\bfseries Злосоветие \normalfont{} "--- злой умысел. 




\bfseries Злострастие \normalfont{} "--- сильные и порочные страсти. 




\bfseries Злостужати \normalfont{} "--- сильно досаждать. 




\bfseries Злостужный \normalfont{} "--- причиняющий большое беспокойство, мучение. 




\bfseries Злотечение \normalfont{} "--- развратные или злые поступки. 




\bfseries Злоумерший \normalfont{} "--- претерпевший тяжелую смерть. 




\bfseries Злоухищряти \normalfont{} "--- замышлять зло. 




\bfseries Злохитренный \normalfont{} "--- коварный. 




\bfseries Злохудожный \normalfont{} "--- лукавый; злобный; беззаконный. 




\bfseries Злый \normalfont{} "--- злой; плохой; негодный; худой; жестокий. 




\bfseries Знаемый \normalfont{} "--- знакомый, близкий человек. 




\bfseries Знаменательне \normalfont{} "--- прообразовательно. 




\bfseries Знаменательный \normalfont{} "--- прообразовательный; обозначающий нечто. 




\bfseries Знаменати \normalfont{} "--- обозначать знаком; помечать; изображать; показывать; являть. 




\bfseries Знамение \normalfont{} "--- знак; признак; явление; чудо. 




\bfseries Знаменоносец \normalfont{} "--- чудотворец. 




\bfseries Знаменоносный \normalfont{} "--- чудотворный. 




\bfseries Зобати \normalfont{} "--- наполнять зоб; клевать; есть; поглощать. 




\bfseries Зрак \normalfont{} "--- лицо; вид; образ. 




\bfseries Зрети \normalfont{} "--- смотреть. 




\bfseries Зрети к смерти \normalfont{} "--- находиться при последнем издыхании. 




\bfseries Зыбати \normalfont{} "--- шевелить; двигать; качать. 




 





\bfseries И \normalfont{} "--- его. 




\bfseries Игемон \normalfont{} "--- вождь; начальник; правитель. 




\bfseries Иго \normalfont{} "--- ярмо; ноша. 




\bfseries Игралище \normalfont{} "--- место для представления. 




\bfseries Игрище \normalfont{} "--- смешное или непристойное представление. 




\bfseries Идеже \normalfont{} "--- где; когда. 




\bfseries Идолобесие \normalfont{} "--- неистовое идолопоклонство. 




\bfseries Иерей \normalfont{} "--- священник. 




\bfseries Иждивати \normalfont{} "--- проживать; тратить; издерживать. 




\bfseries Иже \normalfont{} "--- который. 




\bfseries Изблистати \normalfont{} "--- осиять; облистать; излить свет. 




\bfseries Избодати \normalfont{} "--- пропороть; поразить; пронзить; проколоть; выколоть. 




\bfseries Изборение \normalfont{} "--- поражение. 




\bfseries Изборати \normalfont{} "--- побеждать; поражать. 




\bfseries Избременяти \normalfont{} "--- облегчать; освобождать от бремени; выгружать. 




\bfseries Избутелый \normalfont{} "--- согнивший; испортившийся. 




\bfseries Избыти \normalfont{} "--- остаться в избытке, излишестве; изобиловать; освободиться. 




\bfseries Избыток \normalfont{} "--- довольство; изобилие. 




\bfseries Изваяние \normalfont{} "--- идол; кумир. 




\bfseries Извержение \normalfont{} "--- исключение из церковного клира или лишение сана. 




\bfseries Извесити \normalfont{} "--- свесить; вывесить. 




\bfseries Извествовати \normalfont{} "--- объявлять; оглашать; удостоверять. 




\bfseries Известно \normalfont{} "--- точно; тщательно. 




\bfseries Извет \normalfont{} "--- донос; извещение. 




\bfseries Извещен \normalfont{} "--- уверен. 




\bfseries Извещение \normalfont{} "--- удостоверение. 




\bfseries Извитие словес \normalfont{} "--- красноречие; витийство. 




\bfseries Извитийствовати \normalfont{} "--- красноречиво рассказать. 




\bfseries Извлачитися \normalfont{} "--- раздеться; разоблачиться. 




\bfseries Изволение \normalfont{} "--- воля; желание. 




\bfseries Изволити \normalfont{} "--- дозволять; захотеть; пожелать. 




\bfseries Извращати \normalfont{} "--- выворачивать; изменять; превращать. 




\bfseries Изврещи \normalfont{} "--- выбросить; вымести. 




\bfseries Извыкати \normalfont{} "--- научиться, познавать. 




\bfseries Изгвоздити \normalfont{} "--- выдернуть, вынуть гвозди. 




\bfseries Изгибающий \normalfont{} "--- погибающий, пропадающий. 




\bfseries Изгибнути \normalfont{} "--- погибнуть; пропасть; потеряться. 




\bfseries Изглаждати \normalfont{} "--- исключать; уничтожать. 




\bfseries Издетска \normalfont{} "--- с детства. 




\bfseries Издревле \normalfont{} "--- издавна; исстари. 




\bfseries Издручитися \normalfont{} "--- изнурить себя. 




\bfseries Изженяти \normalfont{} "--- изгонять; выгонять. 




\bfseries Излазити \normalfont{} "--- выходить; сходить (например, с корабля). 




\bfseries Излиха \normalfont{} "--- чрезмерно; еще более. 




\bfseries Излишше \normalfont{} "--- до излишества; паче меры. 




\bfseries Изляцати \normalfont{} "--- протягивать; простирать. 




\bfseries Измерети \normalfont{} "--- умереть. 




\bfseries Изметати \normalfont{} "--- извергать; выкидывать; выбрасывать. 




\bfseries Измена \normalfont{} "--- замена; перемена; выкуп. 




\bfseries Изменяти \normalfont{} "--- заменять; переменять. 




\bfseries Изменяти лице \normalfont{} "--- притворяться. 




\bfseries Измлада \normalfont{} "--- смолоду. 




\bfseries Измовение \normalfont{} "--- омытие; очищение. 




\bfseries Измолкати \normalfont{} "--- перестать говорить; замолкать. 




\bfseries Изнесение, изношение \normalfont{} "--- вынос. 




\bfseries Изницати \normalfont{} "--- возникать; появляться. 




\bfseries Износити \normalfont{} "--- выносить; произносить; производить; произращать; приносить. 




\bfseries Изнуждати \normalfont{} "--- выводить из нужды. 




\bfseries Изобнажати \normalfont{} "--- обнаруживать; являть; открывать. 




\bfseries Изостати \normalfont{} "--- остаться где-либо. 




\bfseries Изощряти \normalfont{} "--- обострить; наточить. 




\bfseries Изращение \normalfont{} "--- вырощение; произведение; порождение. 




\bfseries Изриновение \normalfont{} "--- выбрасывание; извержение; исключение. 




\bfseries Изриновенный \normalfont{} "--- изверженный; выкинутый; прогнанный. 




\bfseries Изринути \normalfont{} "--- столкнуть; опрокинуть; повалить; погубить. 




\bfseries Изрок \normalfont{} "--- изречение; осуждение. 




\bfseries Изрыти \normalfont{} "--- вырыть; выкопать. 




\bfseries Изрядно \normalfont{} "--- особенно; преимущественно. 




\bfseries Изсунути \normalfont{} "--- вынуть; исторгнуть; вырвать; изъять. 




\bfseries Изступление \normalfont{} "--- изумление; восторг. 




\bfseries Изуведети \normalfont{} "--- уразуметь; познать. 




\bfseries Изуздитися \normalfont{} "--- освободиться; получить волю. 




\bfseries Изумевати \normalfont{} "--- недоумевать; не понимать. 




\bfseries Изумителен \normalfont{} "--- буйствующий; беснующийся. 




\bfseries Изумитися \normalfont{} "--- сойти с ума; обезуметь. 




\bfseries Изути \normalfont{} "--- разуть; снять обувь. 




\bfseries Изчленити \normalfont{} "--- лишить членов; сокрушить члены; изуродовать. 




\bfseries Изъядати \normalfont{} "--- проматывать; растрачивать. 




\bfseries Иконом \normalfont{} "--- домоправитель. 




\bfseries Иконоратный \normalfont{} "--- иконоборственный. 




\bfseries Икос \normalfont{} "--- пространная песнь, написанная в похвалу святого или праздника. 




\bfseries Имати \normalfont{} "--- брать. 




\bfseries Иматисма \normalfont{} "--- верхнее платье, плащ. 




\bfseries Именный \normalfont{} "--- сокровищный; касающийся имения. 




\bfseries Имуществительно \normalfont{} "--- преимущественно. 




\bfseries Ин \normalfont{} "--- иной; другой. 




\bfseries Инамо \normalfont{} "--- в ином месте. 




\bfseries Иноковати \normalfont{} "--- жить по-иночески. 




\bfseries Инуде, инде \normalfont{} "--- в ином месте, в иное место. 




\bfseries Ипакои \normalfont{} "--- песнопение, положенное по малой ектении после полиелея на воскресной утрени. 




\bfseries Ипарх \normalfont{} "--- начальник области; градоначальник; наместник. 




\bfseries Ипостась \normalfont{} "--- лицо. 




\bfseries Ирмос \normalfont{} "--- песнопение, стоящее в начале каждой из песен канона. 




\bfseries Иродианы \normalfont{} "--- сторонники Ирода. 




\bfseries Ирой \normalfont{} "--- греч. миф. герой. 




\bfseries Исказити \normalfont{} "--- испортить; оскопить. 




\bfseries Искапати \normalfont{} "--- источать; испускать каплями; истечь. 




\bfseries Исковати \normalfont{} "--- выковать. 




\bfseries Искони \normalfont{} "--- изначала; вначале; всегда. 




\bfseries Исконный \normalfont{} "--- бывший искони; всегдашний. 




\bfseries Искренний \normalfont{} "--- ближний. 




\bfseries Искус \normalfont{} "--- испытание; искушение; проверка. 




\bfseries Исперва \normalfont{} "--- сначала; искони. 




\bfseries Исплевити \normalfont{} "--- выполоть; вырвать; исторгнуть; выдернуть; собрать. 




\bfseries Исплести \normalfont{} "--- сплести; сложить; составить. 




\bfseries Исповедатися \normalfont{} "--- признаваться; открыто выражать свою веру. 




\bfseries Исповедник \normalfont{} "--- человек, подвергавшийся страданиям или гонению за веру Христову. 




\bfseries Исполнение \normalfont{} "--- полнота; наполнение; совершение. 




\bfseries Исполнь \normalfont{} "--- наполненный; исполненный. 




\bfseries Исполняти \normalfont{} "--- наполнять; совершать. 




\bfseries Исполу \normalfont{} "--- вполовину; пополам; частию. 




\bfseries Исправити \normalfont{} "--- выпрямить; исправить; направить; укрепить. 




\bfseries Исправление \normalfont{} "--- восстановление; правый образ жизни. 




\bfseries Испразднити \normalfont{} "--- ниспровергнуть; уничтожить; умалить. 




\bfseries Испрати, исперити \normalfont{} "--- вытоптать; вымыть. 




\bfseries Испытно \normalfont{} "--- тщательно. 




\bfseries Испытовати \normalfont{} "--- выведывать. 




\bfseries Иссоп \normalfont{} "--- растение, употребляемое в пучках для кропления. 




\bfseries Истее \normalfont{} "--- точнее; яснее. 




\bfseries Истесы \normalfont{} "--- чресла, лядвеи. 




\bfseries Истицание \normalfont{} "--- истечение; истечение семени; поллюция. 




\bfseries Истаевати \normalfont{} "--- растаять; исчезать. 




\bfseries Исторгнути \normalfont{} "--- вырвать; вывести. 




\bfseries Истощание \normalfont{} "--- изнурение; унижение; снисхождение. 




\bfseries Истрезвлятися \normalfont{} "--- протрезвляться. 




\bfseries Истукан \normalfont{} "--- статуя; болван; идол. 




\bfseries Истый, истовый \normalfont{} "--- точный; подлинный; истинный. 




\bfseries Истязати \normalfont{} "--- вытягивать; получать; допрашивать. 




\bfseries Исходище \normalfont{} "--- место выхода; исток; начало. 




\bfseries Исходище вод(ное) \normalfont{} "--- ручей; поток; река. 




\bfseries Исходище путей \normalfont{} "--- распутье; перекресток. 




\bfseries Исчадие \normalfont{} "--- детище; плод; род; потомки. 




\bfseries Иулий \normalfont{} "--- июль. 




\bfseries Иуний \normalfont{} "--- июнь. 




 





\bfseries Кадило \normalfont{} "--- возносимое во славу Божию благовонное курение. 




\bfseries Кадильница \normalfont{} "--- сосуд, в котором на горящие угли возлагается фимиам для совершения каждения. 




\bfseries Кадь \normalfont{} "--- кадка, ушат. 




\bfseries Каженик \normalfont{} "--- скопец; сторож при гареме; придворный. 




\bfseries Казатель \normalfont{} "--- учитель, наставник. 




\bfseries Казати \normalfont{} "--- наставлять; поучать. 




\bfseries Казити \normalfont{} "--- искажать; повреждать. 




\bfseries Како \normalfont{} "--- как. 




\bfseries Камара \normalfont{} "--- шатер; скиния; горница; покои. 




\bfseries Камо \normalfont{} "--- куда? 




\bfseries Кампан \normalfont{} "--- колокол. 




\bfseries Камы, камык \normalfont{} "--- камень. 




\bfseries Камык горящ \normalfont{} "--- сера. 




\bfseries Кандило \normalfont{} "--- лампада. 




\bfseries Кандиловжигатель \normalfont{} "--- пономарь. 




\bfseries Кандия \normalfont{} "--- небольшая чаша. 




\bfseries Капище \normalfont{} "--- идольский храм. 




\bfseries Катапетасма \normalfont{} "--- завеса. 




\bfseries Кафисма \normalfont{} "--- один из 20 разделов, на которые разделена Псалтирь. 




\bfseries Кацея \normalfont{} "--- кадильница не на цепочках, а на ручке. 




\bfseries Кацы \normalfont{} "--- каковые; которые; какие. 




\bfseries Квас \normalfont{} "--- закваска; дрожжи. 




\bfseries Квасный \normalfont{} "--- приготовленный на дрожжах. 




\bfseries Келарня, келарница \normalfont{} "--- помещение в монастыре для сохранения вещей, необходимых келарю. 




\bfseries Келарь \normalfont{} "--- старшая хозяйственная должность в монастыре. 




\bfseries Кивот \normalfont{} "--- ящик для икон. 




\bfseries Кидар \normalfont{} "--- головной убор ветхозаветного первосвященника. 




\bfseries Кимвал \normalfont{} "--- музыкальный инструмент. 




\bfseries Кимин \normalfont{} "--- тмин. 




\bfseries Киновия \normalfont{} "--- общежительный монастырь. 




\bfseries Кинсон \normalfont{} "--- дань; подать; ценз. 




\bfseries Кириопасха \normalfont{} "--- название праздника Пасхи, пришедшегося на день Благовещения Пресвятой Богородицы 25 марта. 




\bfseries Кичение \normalfont{} "--- гордость. 




\bfseries Клада \normalfont{} "--- колода (орудие пытки). 




\bfseries Кладенец \normalfont{} "--- яма; клад. 




\bfseries Кладязь \normalfont{} "--- колодец. 




\bfseries Клас \normalfont{} "--- колос. 




\bfseries Клеврет \normalfont{} "--- товарищ; собрат. 




\bfseries Клепало \normalfont{} "--- колотушка, при помощи которой в монастырях созывают на молитву. 




\bfseries Клепати \normalfont{} "--- звонить; стучать или бить в клепало. 




\bfseries Клеть \normalfont{} "--- изба; покои; кладовая; комната. 




\bfseries Клирос \normalfont{} "--- возвышение в храме, на котором располагаются певчие. 




\bfseries Клич \normalfont{} "--- крик, гам. 




\bfseries Клобук \normalfont{} "--- покрывало, носимое монашествующими поверх камилавки. 




\bfseries Ключимый \normalfont{} "--- годный; хороший; случившийся кстати; полезный. 




\bfseries Ключитися \normalfont{} "--- приключиться; случиться. 




\bfseries Книгочий \normalfont{} "--- судья; приставник. 




\bfseries Книжник \normalfont{} "--- ученый. 




\bfseries Ков \normalfont{} "--- умысел; заговор. 




\bfseries Ковчег \normalfont{} "--- кованый ящик: сундук; ларец. 




\bfseries Кодрант \normalfont{} "--- мелкая римская монета. 




\bfseries Козлогласование \normalfont{} "--- бесчинные крики на пиршестве. 




\bfseries Козни \normalfont{} "--- лукавство; хитрость. 




\bfseries Кокош \normalfont{} "--- наседка. 




\bfseries Колено \normalfont{} "--- род; поколение. 




\bfseries Колесницегонитель \normalfont{} "--- возница; преследователь на колеснице. 




\bfseries Коливо \normalfont{} "--- вареная пшеница с медом, приносимая для благословения в церковь на праздники. 




\bfseries Колиждо \normalfont{} "--- когда; как. 




\bfseries Колико \normalfont{} "--- сколько. 




\bfseries Колия \normalfont{} "--- яма; ров. 




\bfseries Колми \normalfont{} "--- сколько. 




\bfseries Колми паче \normalfont{} "--- тем более; особенно. 




\bfseries Коло \normalfont{} "--- колесо. 




\bfseries Колобродити \normalfont{} "--- ходить вокруг; уклоняться. 




\bfseries Коль \normalfont{} "--- сколько; насколько; как. 




\bfseries Колькраты \normalfont{} "--- сколько раз; как часто. 




\bfseries Комбоста \normalfont{} "--- сырая капуста. 




\bfseries Кондак \normalfont{} "--- короткая песнь в честь святого или праздника. 




\bfseries Коноб \normalfont{} "--- котел; горшок; умывальница. 




\bfseries Конура \normalfont{} "--- небольшой мешочек, носимый суеверными людьми вместе с кореньями или другими амулетами. 




\bfseries Копр \normalfont{} "--- укроп; анис. 




\bfseries Кораблец \normalfont{} "--- небольшой корабль. 




\bfseries Корван \normalfont{} "--- дар; жертва Богу. 




\bfseries Корвана \normalfont{} "--- казнохранилище при храме Иерусалимском. 




\bfseries Кормило \normalfont{} "--- руль. 




\bfseries Кормильствовати \normalfont{} "--- править; руководить. 




\bfseries Кормление \normalfont{} "--- правление; управление. 




\bfseries Корчаг \normalfont{} "--- лохань. 




\bfseries Корчемница \normalfont{} "--- корчма;кабак. 




\bfseries Коснити \normalfont{} "--- медлить. 




\bfseries Косноязычный \normalfont{} "--- медленноязычный; заика. 




\bfseries Косный \normalfont{} "--- медленный; нерешительный; упорно остающийся в одном и том же состоянии. 




\bfseries Котва \normalfont{} "--- якорь. 




\bfseries Кош \normalfont{} "--- кошель; корзина. 




\bfseries Кошница \normalfont{} "--- кошель, корзина. 




\bfseries Кощунник \normalfont{} "--- шут, балагур. 




\bfseries Кощунница \normalfont{} "--- актриса; танцовщица. 




\bfseries Кощуны \normalfont{} "--- смехотворство. 




\bfseries Крабица \normalfont{} "--- коробочка; ящичек; ковчежец; ларчик. 




\bfseries Крава \normalfont{} "--- корова. 




\bfseries Краегранесие, краестрочие \normalfont{} "--- акростих, т.е. поэтическое произведение, в котором начальные буквы каждой строчки составляют слово, фразу или следуют порядку алфавита. 




\bfseries Крамола \normalfont{} "--- смута; заговор; бунт. 




\bfseries Красный \normalfont{} "--- красивый; прекрасный; непорочный. 




\bfseries Красовул \normalfont{} "--- мерная чаша в монастырях, вмещающая более 200 г . 




\bfseries Крастель \normalfont{} "--- перепел. 




\bfseries Крата \normalfont{} "--- раз. 




\bfseries Крепкий \normalfont{} "--- сильный; крепкий. 




\bfseries Креплий \normalfont{} "--- крепчайший, сильнейший. 




\bfseries Кресати \normalfont{} "--- извлекать; высекать огонь; оживлять. 




\bfseries Крин \normalfont{} "--- лилия. 




\bfseries Кроме \normalfont{} "--- вне; извне; отдельно; кроме. 




\bfseries Кромешный \normalfont{} "--- внешний; запредельный; отдаленный; лишенный. 




\bfseries Кропило \normalfont{} "--- кисть для окропления освященной водой. 




\bfseries Ктитор \normalfont{} "--- создатель; строитель или снабдитель храма или монастыря; церковный староста. 




\bfseries Ктому \normalfont{} "--- впредь; затем; еще; уже; более. 




\bfseries Купа \normalfont{} "--- кипа; груда; куча; ворох. 




\bfseries Купель \normalfont{} "--- озеро; пруд; садок; сосуд для совершения Таинства Крещения. 




\bfseries Купина \normalfont{} "--- соединение нескольких однородных предметов: куст, сноп; терновый куст. 




\bfseries Купно \normalfont{} "--- вместе. 




\bfseries Купный \normalfont{} "--- совместный. 




\bfseries Кустодия \normalfont{} "--- стража; караул; охрана; 




\bfseries Кутия \normalfont{} "--- вареная пшеница с медом, приносимая в церковь на поминовение усопших христиан. 




\bfseries Куща \normalfont{} "--- шатер; палатка; шалаш. 




\bfseries Кущник \normalfont{} "--- человек, делающий палатки или живущий в шалаше. 




 





\bfseries Ладия \normalfont{} "--- небольшое судно; кораблик; ладья. 




\bfseries Ладан \normalfont{} "--- благоуханная смола, влагаемая в кадильницу на горящие угли для благовонного курения. 




\bfseries Лазарома \normalfont{} "--- гробная одежда; повой; плащаница, в которую повивали усопших у иудеев. 




\bfseries Лазня \normalfont{} "--- баня. 




\bfseries Лай \normalfont{} "--- хула; поношение. 




\bfseries Лакать \normalfont{} "--- евр. мера длины. 




\bfseries Ланита \normalfont{} "--- щека. 




\bfseries Лаятель \normalfont{} "--- ругатель; хулитель; седящий в засаде. 




\bfseries Лвичищ \normalfont{} "--- львенок. 




\bfseries Левиафан \normalfont{} "--- крокодил. 




\bfseries Легеон \normalfont{} "--- полк; толпа; множество. 




\bfseries Лежание \normalfont{} "--- лежание; опочивание. 




\bfseries Лемаргия \normalfont{} "--- гортанобесие, т.е. гурманство. 




\bfseries Лентион, лентий \normalfont{} "--- полотенце. 




\bfseries Лепо \normalfont{} "--- красиво. 




\bfseries Лепоподобно \normalfont{} "--- благопристойно; по достоинству. 




\bfseries Лепота \normalfont{} "--- красота; изящество. 




\bfseries Лепта \normalfont{} "--- мелкая монетка. 




\bfseries Лествица \normalfont{} "--- лестница. 




\bfseries Лестчий \normalfont{} "--- льстивый, ложный. 




\bfseries Лесть \normalfont{} "--- обман; хитрость; коварность. 




\bfseries Лето \normalfont{} "--- год; время. 




\bfseries Леторасль \normalfont{} "--- выросшее за год, годовой побег дерева. 




\bfseries Леть \normalfont{} "--- льзя; можно. 




\bfseries Леха \normalfont{} "--- гряда, ряд. 




\bfseries Лечба \normalfont{} "--- лекарство; врачевство. 




\bfseries Лечец \normalfont{} "--- лекарь; врач. 




\bfseries Лжа \normalfont{} "--- ложь. 




\bfseries Лжесловесие \normalfont{} "--- лживые речи. 




\bfseries Лив \normalfont{} "--- полдень; юг; юго-западный ветер. 




\bfseries Ливан \normalfont{} "--- иногда значит то же, что и Ладан. 




\bfseries Лик \normalfont{} "--- собрание; хор. 




\bfseries Ликование \normalfont{} "--- многолюдное пение; пляска; танцы. 




\bfseries Ликоватися \normalfont{} "--- приветствовать чрез соприкосновение правой щекой. 




\bfseries Ликовне \normalfont{} "--- с ликованием. 




\bfseries Ликостояние \normalfont{} "--- бдение на молитве церковной. 




\bfseries Лития \normalfont{} "--- исхождение из церкви на молитву. 




\bfseries Литра \normalfont{} "--- мера веса. 




\bfseries Литургисати \normalfont{} "--- совершать Литургию. 




\bfseries Лихва \normalfont{} "--- прибыль; проценты. 




\bfseries Лихоимец \normalfont{} "--- ростовщик; сребролюбец. 




\bfseries Лице \normalfont{} "--- лицо; вид; человек. 




\bfseries Личина \normalfont{} "--- маскарадная или шутовская маска. 




\bfseries Лишатися \normalfont{} "--- нуждаться. 




\bfseries Лишше \normalfont{} "--- больше, сверх того. 




\bfseries Лобзание \normalfont{} "--- устное целование. 




\bfseries Ловитва \normalfont{} "--- ловля; охота; сети; добыча; грабеж. 




\bfseries Ловительство \normalfont{} "--- засада, ловушка. 




\bfseries Ложе \normalfont{} "--- постель, одр. 




\bfseries Ложесна \normalfont{} "--- утроба женщины. 




\bfseries Лоза \normalfont{} "--- виноград. 




\bfseries Ломимый \normalfont{} "--- преломляемый. 




\bfseries Лоно \normalfont{} "--- пазуха; грудь; колени. 




\bfseries Луновение \normalfont{} "--- месячный цикл у женщин. 




\bfseries Лысто \normalfont{} "--- голень; икры; лытка. 




\bfseries Льстивый \normalfont{} "--- обманчивый. 




\bfseries Льщение \normalfont{} "--- обман; коварство; лесть. 




\bfseries Любо \normalfont{} "--- либо, или. 




\bfseries Любомудрие \normalfont{} "--- философия. 




\bfseries Любомятежный \normalfont{} "--- склонный к мятежу. 




\bfseries Любоначалие \normalfont{} "--- властолюбие. 




\bfseries Любопразднственный \normalfont{} "--- любящий празднствовать. 




\bfseries Любопрение \normalfont{} "--- любовь состязаться, спорить. 




\bfseries Любосластие \normalfont{} "--- сластолюбие; любовь к плотским утехам. 




\bfseries Любочестие \normalfont{} "--- почитание; чествование. 




\bfseries Любочестный \normalfont{} "--- достойный похвалы, чести. 




\bfseries Любы \normalfont{} "--- любовь. 




\bfseries Люте \normalfont{} "--- жестоко; тяжко. 




\bfseries Лютый \normalfont{} "--- свирепый; жестокий; злой; мучительный. 




\bfseries Лядвея \normalfont{} "--- ляжка; верхняя половина ноги; промежность. 




\bfseries Лярва \normalfont{} "--- маска; личина. 




 





\bfseries Маание \normalfont{} "--- знак рукой, головою, глазами или иного рода, содержащий приказание; повеление; воля. 




\bfseries Маий \normalfont{} "--- май. 




\bfseries Малакия \normalfont{} "--- грех рукоблудия. 




\bfseries Малимый \normalfont{} "--- уменьшаемый. 




\bfseries Малобрещи \normalfont{} "--- нерадеть о чем-либо. 




\bfseries Мамона \normalfont{} "--- богатство; имение. 




\bfseries Мандра \normalfont{} "--- ограда. 




\bfseries Мание \normalfont{} "--- знак рукой, головою, глазами или иного рода, содержащий приказание; повеление; воля. 




\bfseries Манна \normalfont{} "--- небесный хлеб, данный израильтянам в пустыне. 




\bfseries Манноприемный \normalfont{} "--- содержащий манну. 




\bfseries Маслина \normalfont{} "--- олива; оливковое дерево. 




\bfseries Масличный \normalfont{} "--- оливковый. 




\bfseries Мастити \normalfont{} "--- намазывать. 




\bfseries Маститый \normalfont{} "--- обильный; тучный; заслуженный. 




\bfseries Масть \normalfont{} "--- мазь; масло. 




\bfseries Матеродевственный \normalfont{} "--- одновременно относящийся и к матери, и к деве. 




\bfseries Матеролепне \normalfont{} "--- по-матерински. 




\bfseries Матерский \normalfont{} "--- материнский. 




\bfseries Матерь градовом \normalfont{} "--- столица; первопрестольный град. 




\bfseries Мгляный \normalfont{} "--- окруженный или покрытый мглой. 




\bfseries Медвен \normalfont{} "--- медовый. 




\bfseries Медленоязычный \normalfont{} "--- косноязычный; заика. 




\bfseries Медница \normalfont{} "--- медная монетка. 




\bfseries Медовина \normalfont{} "--- вареный мед с хмелем. 




\bfseries Медоточный \normalfont{} "--- источающий, изливающий мед. 




\bfseries Медоязычный \normalfont{} "--- сладкословесный. 




\bfseries Междорамие \normalfont{} "--- пространство между плечами. 




\bfseries Мездник \normalfont{} "--- наемник. 




\bfseries Мерзость \normalfont{} "--- скверна; гнусность; беззаконие;нечестие; иногда "--- идол. 




\bfseries Мерило \normalfont{} "--- мера; весы. 




\bfseries Меск \normalfont{} "--- полуосел; мул; лошак. 




\bfseries Мессия \normalfont{} "--- евр. помазанник. 




\bfseries Метание \normalfont{} "--- поясной поклон. 




\bfseries Мех \normalfont{} "--- кожаный мешок для сохранения и перевоза жидкостей. 




\bfseries Мжа \normalfont{} "--- мигание; прищур. 




\bfseries Мжати \normalfont{} "--- жмурить глаза; щуриться; плохо видеть. 




\bfseries Мзда \normalfont{} "--- награда; плата. 




\bfseries Мздовоздаятель \normalfont{} "--- оплачивающий работу, дающий награду. 




\bfseries Мздоимание \normalfont{} "--- взяточничество. 




\bfseries Мила ся деяти \normalfont{} "--- низко припадать к земле; просить сжалиться над собой. 




\bfseries Милоть \normalfont{} "--- овчина; грубый шерстяной плащ из овечьей шерсти. 




\bfseries Милый \normalfont{} "--- жалкий; заслуживающий сожаления. 




\bfseries Мимотещи \normalfont{} "--- идти, проходить мимо, не останавливаясь. 




\bfseries Мирная \normalfont{} "--- название великой ектении. 




\bfseries Миро \normalfont{} "--- благовонная жидкость или мазь. 




\bfseries Мироподательне \normalfont{} "--- подавая мир. 




\bfseries Мироточец \normalfont{} "--- источающий чудотворное миро. 




\bfseries Мироявленный \normalfont{} "--- явленный, открытый миру. 




\bfseries Мирсина \normalfont{} "--- название красивого дерева. 




\bfseries Младодеяти, младодействовати \normalfont{} "--- принимать образ младенца; облекаться в плоть. 




\bfseries Младоумие \normalfont{} "--- незрелость ума. 




\bfseries Млат \normalfont{} "--- молот. 




\bfseries Млеко \normalfont{} "--- молоко. 




\bfseries Мнас \normalfont{} "--- мина, древнегреч. серебряная монета. 




\bfseries Мнее \normalfont{} "--- менее. 




\bfseries Мнети, мнити \normalfont{} "--- думать; предполагать; казаться. 




\bfseries Мний \normalfont{} "--- меньший. 




\bfseries Мних \normalfont{} "--- монах. 




\bfseries Многажды, множицею \normalfont{} "--- часто; много раз. 




\bfseries Многобезсловесие \normalfont{} "--- невежество. 




\bfseries Многобогатый \normalfont{} "--- изобилующий во всем. 




\bfseries Многоболезненный \normalfont{} "--- подъявший многие труды, подвиги, беды, страдания. 




\bfseries Многоборимый \normalfont{} "--- подвергаемый сильным искушениям, нападениям. 




\bfseries Многобурный \normalfont{} "--- тревожный. 




\bfseries Многогобзенный \normalfont{} "--- весьма обильный. 




\bfseries Многогубо \normalfont{} "--- многократно. 




\bfseries Многокласный \normalfont{} "--- колосистый. 




\bfseries Многомятущий \normalfont{} "--- преисполненный суетою. 




\bfseries Многонарочитый \normalfont{} "--- весьма знаменитый. 




\bfseries Многообразне \normalfont{} "--- во многих видах; различно. 




\bfseries Многооранный \normalfont{} "--- многократно возделанный. 




\bfseries Многоочитый \normalfont{} "--- имеющий множество глаз. 




\bfseries Многоплодие \normalfont{} "--- плодоносие; многочадие. 




\bfseries Многоплотие \normalfont{} "--- тучность. 




\bfseries Многопрелестный \normalfont{} "--- исполненный прелестей и соблазнов. 




\bfseries Многосветлый \normalfont{} "--- радостный; торжественный. 




\bfseries Многослезный \normalfont{} "--- исполненный печали и горя. 




\bfseries Многоснедный \normalfont{} "--- изобилующий многообразием пищи. 




\bfseries Многосугубый \normalfont{} "--- усугубленный; умноженный; усиленный. 




\bfseries Многосуетный \normalfont{} "--- совершенно пустой, бесполезный. 




\bfseries Многоуветливый \normalfont{} "--- очень снисходительный. 




\bfseries Многоцелебный \normalfont{} "--- подающий многие исцеления. 




\bfseries Многочастне \normalfont{} "--- много раз. 




\bfseries Многочудесный \normalfont{} "--- источающий многие чудеса; прославленный чудотворениями. 




\bfseries Многоязычный \normalfont{} "--- состоящий из множества племен. 




\bfseries Молва \normalfont{} "--- говор; ропот; слух; забота; волнение. 




\bfseries Молвити \normalfont{} "--- заботиться; суетиться; волноваться; роптать. 




\bfseries Молие \normalfont{} "--- моль. 




\bfseries Молниезрачный \normalfont{} "--- напоминающий молнию. 




\bfseries Мочащийся к стене \normalfont{} "--- пес. 




\bfseries Мощи \normalfont{} "--- нетленное тело угодника Божия. 




\bfseries Мравий \normalfont{} "--- муравей. 




\bfseries Мраз \normalfont{} "--- мороз. 




\bfseries Мрежа \normalfont{} "--- рыболовная сеть. 




\bfseries Мужатая \normalfont{} "--- замужняя. 




\bfseries Мужатица \normalfont{} "--- замужняя женщина. 




\bfseries Муженеискусная \normalfont{} "--- не познавшая мужа; не причастная браку. 




\bfseries Мурин \normalfont{} "--- эфиоп; арап; негр; чернокожий; дух тьмы; бес. 




\bfseries Мусийский, мусикийский \normalfont{} "--- музыкальный. 




\bfseries Мусикия \normalfont{} "--- музыка. 




\bfseries Мшела \normalfont{} "--- взятка. 




\bfseries Мшелоимство \normalfont{} "--- корыстолюбие. 




\bfseries Мшица \normalfont{} "--- мошка; мошкара. 




\bfseries Мытарь \normalfont{} "--- сборщик подати. 




\bfseries Мытница \normalfont{} "--- таможня; дом или двор для сбора пошлин. 




\bfseries Мыто \normalfont{} "--- пошлина; сбор; налог. 




\bfseries Мышца \normalfont{} "--- рука; плечо; сила. 




\bfseries Мясопуст \normalfont{} "--- последний день вкушения мясной пищи. 




\bfseries Мясоястие, мясоед \normalfont{} "--- время, когда Устав разрешает вкушение мяса. 




\bfseries Мятва \normalfont{} "--- мята. 




 





\bfseries Набдевати \normalfont{} "--- снабжать; наделять; хранить. 




\bfseries Наваждати \normalfont{} "--- научать; подстрекать. 




\bfseries Навет \normalfont{} "--- наговор; клевета; козни. 




\bfseries Навклир \normalfont{} "--- хозяин корабля. 




\bfseries Навыкнути \normalfont{} "--- приучиться; привыкнуть. 




\bfseries Наготовати \normalfont{} "--- ходить без одежды. 




\bfseries Нагствовати \normalfont{} "--- см. Наготовати. 




\bfseries Надходити \normalfont{} "--- внезапно постигнуть, случиться. 




\bfseries Наздати \normalfont{} "--- надстроить; укрепить; утвердить. 




\bfseries Назирати \normalfont{} "--- примечать; наблюдать. 




\bfseries Назнаменовати \normalfont{} "--- назначать; обозначать; осенять Крестом. 




\bfseries Наипаче \normalfont{} "--- особенно; преимущественно. 




\bfseries Наитие \normalfont{} "--- нисшествие; нашествие; сошествие. 




\bfseries Наказание \normalfont{} "--- иногда: учение. 




\bfseries Наляцати \normalfont{} "--- натянуть. 




\bfseries На мале \normalfont{} "--- малое время; дешево. 




\bfseries Намащати \normalfont{} "--- намазывать; втирать. 




\bfseries На мнозе \normalfont{} "--- на долгое время; дорого. 




\bfseries Наопак \normalfont{} "--- наоборот; вопреки. 




\bfseries Напаствуемый \normalfont{} "--- находящийся в напасти. 




\bfseries Наперсник \normalfont{} "--- друг, доверенное лицо. 




\bfseries Напоследок \normalfont{} "--- недавно. 




\bfseries Нард \normalfont{} "--- колосистое ароматическое растение. 




\bfseries Нарекованный \normalfont{} "--- предопределенный; предуставленный; назначенный. 




\bfseries Нарицати \normalfont{} "--- называть. 




\bfseries Нарок \normalfont{} "--- определенное или назначенное время. 




\bfseries Нарочитый \normalfont{} "--- особый; славный. 




\bfseries Наругатися \normalfont{} "--- насмеяться; пренебречь; опозорить. 




\bfseries Насмертник \normalfont{} "--- осужденный на смерть 




\bfseries Насущный \normalfont{} "--- настоящий; нынешний; существенный; необходимый. 




\bfseries На толице \normalfont{} "--- в такое время; за такую цену, за столько. 




\bfseries Началозлобный \normalfont{} "--- виновник зла. 




\bfseries Начаток \normalfont{} "--- начало; первый плод. 




\bfseries Начертавати \normalfont{} "--- изобразить. 




\bfseries Наясне \normalfont{} "--- наружу; открыто. 




\bfseries Наяти \normalfont{} "--- нанять. 




\bfseries Неблазненный \normalfont{} "--- безопасный; непогрешимый. 




\bfseries Неблазный \normalfont{} "--- непрельщаемый. 




\bfseries Небрещи \normalfont{} "--- нерадеть; пренебрегать. 




\bfseries Невеглас \normalfont{} "--- невежда; простак; неученый. 




\bfseries Невеститель \normalfont{} "--- снабжающий бедных невест приданым. 




\bfseries Невестоукрасити \normalfont{} "--- украсить как невесту. 




\bfseries Невечерний \normalfont{} "--- непомрачаемый; светлый. 




\bfseries Невиновный \normalfont{} "--- беспричинный; самобытный. 




\bfseries Невозбранно \normalfont{} "--- беспрепятственно. 




\bfseries Невозносительно \normalfont{} "--- смиренно. 




\bfseries Негли \normalfont{} "--- неужели; может быть; авось. 




\bfseries Неделя \normalfont{} "--- церковное название воскресного дня. 




\bfseries Недремлющий \normalfont{} "--- неусыпный. 




\bfseries Недристый \normalfont{} "--- имеющий широкую грудь. 




\bfseries Недро \normalfont{} "--- нутро; утроба; грудь; внутренность; залив. 




\bfseries Недуг \normalfont{} "--- болезнь. 




\bfseries Неже \normalfont{} "--- нежели; чем. 




\bfseries Независтный \normalfont{} "--- неиспорченный; невредимый; довольный; обильный. 




\bfseries Неиждиваемый \normalfont{} "--- не могущий быть истрачен или использован до конца. 




\bfseries Неизводимый \normalfont{} "--- непрекращаемый. 




\bfseries Неизгиблемый \normalfont{} "--- не подлежащий тлению или времени. 




\bfseries Неизреченный \normalfont{} "--- невыразимый. 




\bfseries Неискусобрачный \normalfont{} "--- не испытавший брака. 




\bfseries Неискусомужная \normalfont{} "--- не познавшая мужа. 




\bfseries Неиспытный \normalfont{} "--- сокровенный; тайный. 




\bfseries Неистовно \normalfont{} "--- с ожесточением; с яростию. 




\bfseries Неистовый \normalfont{} "--- вышедший из себя; находящийся не в должном состоянии. 




\bfseries Неисследимый \normalfont{} "--- непостижимый. 




\bfseries Неключимый \normalfont{} "--- бесполезный; негодный. 




\bfseries Некосненно \normalfont{} "--- немедленно. 




\bfseries Не ктому \normalfont{} "--- более не; еще не; уже не. 




\bfseries Нелестный \normalfont{} "--- необманчивый; нелукавый. 




\bfseries Нелеть \normalfont{} "--- нельзя. 




\bfseries Неможение \normalfont{} "--- болезнь; немощь; бессилие. 




\bfseries Немокренно \normalfont{} "--- по суху. 




\bfseries Немощствующий \normalfont{} "--- больной. 




\bfseries Необименный \normalfont{} "--- необъятный. 




\bfseries Не обинутися \normalfont{} "--- поступать смело. 




\bfseries Необинуяся \normalfont{} "--- смело; дерзновенно. 




\bfseries Неопальный \normalfont{} "--- несгораемый. 




\bfseries Неописанне \normalfont{} "--- изобразимо. 




\bfseries Неопределенный \normalfont{} "--- беспредельный. 




\bfseries Неоранный \normalfont{} "--- непаханный; невозделанный; нетронутый. 




\bfseries Неотметный \normalfont{} "--- неотчужденный. 




\bfseries Неплоды, неплодовь \normalfont{} "--- бесплодная женщина. 




\bfseries Неподобный \normalfont{} "--- непристойный. 




\bfseries Непорочны \normalfont{} "--- название 17-й кафизмы псалма 118. 




\bfseries Непорочный \normalfont{} "--- беспорочный; святой; чистый. 




\bfseries Неправдовати \normalfont{} "--- поступать нечестиво. 




\bfseries Непраздная \normalfont{} "--- беременная. 




\bfseries Непревратный \normalfont{} "--- непременный; неизменяемый. 




\bfseries Непреложно \normalfont{} "--- неизменно; без изменения. 




\bfseries Непременный \normalfont{} "--- неизменяемый. 




\bfseries Непреоборимый \normalfont{} "--- неодолимый; непобедимый. 




\bfseries Непщевание \normalfont{} "--- мнение; подлог; выдумка. 




\bfseries Непщевати \normalfont{} "--- думать; придумывать; считать. 




\bfseries Неразседный \normalfont{} "--- неразрушаемый. 




\bfseries Нерешимый \normalfont{} "--- несокрушимый; неразвязываемый. 




\bfseries Неседальное \normalfont{} "--- церковная служба, во время которой возбраняется сидеть. 




\bfseries Несланый \normalfont{} "--- несоленый. 




\bfseries Неслиянне \normalfont{} "--- неслитно. 




\bfseries Несмесне \normalfont{} "--- не смешиваясь. 




\bfseries Нестареемый \normalfont{} "--- вечный; неизменный. 




\bfseries Несть \normalfont{} "--- нет. 




\bfseries Нестояние \normalfont{} "--- непостоянство; смущение. 




\bfseries Несумненный \normalfont{} "--- несомненный; надежный; беспристрастный. 




\bfseries Несущий \normalfont{} "--- не имеющий бытия. 




\bfseries Нетление \normalfont{} "--- неуничтожимость; вечность; несокрушимость. 




\bfseries Нетребе \normalfont{} "--- не нужно. 




\bfseries Нетреный \normalfont{} "--- непротертый; непроходимый. 




\bfseries Нетесноместно \normalfont{} "--- удобовместительно. 




\bfseries Нетяжестне \normalfont{} "--- без труда. 




\bfseries Не у \normalfont{} "--- еще не. 




\bfseries Неудобоприятный \normalfont{} "--- невместимый; непонятный; непостижимый. 




\bfseries Неудобь \normalfont{} "--- неудобно; трудно. 




\bfseries Неумытный \normalfont{} "--- неподкупный. 




\bfseries Нечаяние \normalfont{} "--- неожиданность; беспечность. 




\bfseries Неясыть \normalfont{} "--- пеликан. 




\bfseries Ниже \normalfont{} "--- тем более не...; ни даже...; и не... 




\bfseries Николиже \normalfont{} "--- никогда. 




\bfseries Ни ли \normalfont{} "--- разве не? неужели? или не? 




\bfseries Ниц \normalfont{} "--- вниз; лицем на землю. 




\bfseries Нищетный \normalfont{} "--- нищенский; униженный; бедный. 




\bfseries Новемврий \normalfont{} "--- ноябрь. 




\bfseries Новина \normalfont{} "--- новость. 




\bfseries Новозданный \normalfont{} "--- вновь построенный. 




\bfseries Новопросвещенный \normalfont{} "--- недавно крещеный. 




\bfseries Новосаждение \normalfont{} "--- почки; отпрыски 




\bfseries Ноемврий \normalfont{} "--- ноябрь. 




\bfseries Ножница \normalfont{} "--- ножны. 




\bfseries Нощный вран \normalfont{} "--- филин; сова. 




\bfseries Нудитися \normalfont{} "--- неволиться; принуждаться; достигаться с усилием. 




\bfseries Нудить \normalfont{} "--- пытать. 




\bfseries Нудма \normalfont{} "--- насильно. 




\bfseries Нуждник \normalfont{} "--- употребляющий усилие. 




\bfseries Нырище \normalfont{} "--- развалины; руины; нежилое место. 




\bfseries Ню \normalfont{} "--- ее. 




 





\bfseries Обавание \normalfont{} "--- ворожба; нашептывание; волхвование; колдовство. 




\bfseries Обаватель \normalfont{} "--- обаятель; чародей; ворожея. 




\bfseries Обавати \normalfont{} "--- обаять; очаровывать; ворожить; колдовать; заговаривать. 




\bfseries Обада \normalfont{} "--- оболгание; оклеветание. 




\bfseries Обажаемый \normalfont{} "--- оклеветаемый. 




\bfseries Обанадесять \normalfont{} "--- двенадцать. 




\bfseries Обапо \normalfont{} "--- с обеих сторон; по обеим сторонам. 




\bfseries Обаче \normalfont{} "--- однако; впрочем; но. 




\bfseries Обвеселити \normalfont{} "--- обрадовать. 




\bfseries Обвечеряти \normalfont{} "--- ночевать; переночевать. 




\bfseries Обглядати \normalfont{} "--- смотреть; оглядывать. 




\bfseries Обдержание \normalfont{} "--- сдерживание; управление; стеснение; грусть; впадение. 




\bfseries Обдесноручный \normalfont{} "--- человек, свободно владеющий как правой, так и левой рукой. 




\bfseries Обезвинити \normalfont{} "--- остаться без наказания; не знать за собой вины. 




\bfseries Обезжилити \normalfont{} "--- лишить сил, крепости. 




\bfseries Обезплодствити \normalfont{} "--- лишить плода, успеха. 




\bfseries Обезтлити \normalfont{} "--- сделать нетленным. 




\bfseries Обесити \normalfont{} "--- повесить на чем-либо. 




\bfseries Обет, обетование \normalfont{} "--- обещание. 




\bfseries Обетшати \normalfont{} "--- придти в ветхость; состариться; сделаться негодным; ослабеть; сокрушиться. 




\bfseries Обещник \normalfont{} "--- сообщник; товарищ. 




\bfseries Обжадати \normalfont{} "--- доносить; клеветать. 




\bfseries Обзорище \normalfont{} "--- высокая башня для наблюдения за местностью. 




\bfseries Обидитель \normalfont{} "--- обидчик. 




\bfseries Обиматель \normalfont{} "--- собиратель винограда. 




\bfseries Обиноватися \normalfont{} "--- колебаться; сомневаться; робеть; говорить непрямо, намеками. 




\bfseries Обиновение \normalfont{} "--- отступление. 




\bfseries Обиталище \normalfont{} "--- жилище. 




\bfseries Обитель \normalfont{} "--- гостиница. 




\bfseries Облагати \normalfont{} "--- ублажать; говорить ласково. 




\bfseries Облагодатити \normalfont{} "--- ниспослать благодать. 




\bfseries Облагоухати \normalfont{} "--- исполнить благовонием. 




\bfseries Облазнити \normalfont{} "--- направить по ложному следу; ввести в заблуждение. 




\bfseries Облазнитися \normalfont{} "--- впасть в заблуждение. 




\bfseries Область \normalfont{} "--- власть; сила; господство. 




\bfseries Облачити \normalfont{} "--- одеть. 




\bfseries Облещи \normalfont{} "--- облечь; одеть; лечь вокруг; окружить; сделать привал; остановиться; остаться. 




\bfseries Облистание \normalfont{} "--- озарение; яркий свет. 




\bfseries Облистати \normalfont{} "--- осветить; озарить. 




\bfseries Обличати \normalfont{} "--- показывать чье-либо подлинное лицо; выказывать; обнаруживать. 




\bfseries Обложити \normalfont{} "--- окружить. 




\bfseries Обноществовати \normalfont{} "--- ночевать; препроводить ночь. 




\bfseries Обнощь \normalfont{} "--- всю ночь. 




\bfseries Обожати \normalfont{} "--- обоготворять; чествовать как Бога; делать причастным Божественной благодати. 




\bfseries Оболгати \normalfont{} "--- обмануть. 




\bfseries Обон пол \normalfont{} "--- по ту сторону; за. 




\bfseries Обочие \normalfont{} "--- висок. 




\bfseries Обоюду \normalfont{} "--- по обе стороны; с обеих сторон. 




\bfseries Обрадованный \normalfont{} "--- приветствованный. 




\bfseries Образовати \normalfont{} "--- изображать; приобретать образ. 




\bfseries Обращати \normalfont{} "--- поворачивать; перевертывать; перемещать; вращать. 




\bfseries Обрести \normalfont{} "--- найти. 




\bfseries Обретаемый \normalfont{} "--- находимый. 




\bfseries Обретение \normalfont{} "--- находка; открытие. 




\bfseries Оброк \normalfont{} "--- плата за службу. 




\bfseries Обручник \normalfont{} "--- жених, помолвленный с невестой, но еще не вступивший с ней в брак. 




\bfseries Обсолонь \normalfont{} "--- против солнца. 




\bfseries Обстояние \normalfont{} "--- осада; беда; напасть. 




\bfseries Обушие \normalfont{} "--- мочка у уха. 




\bfseries Обуяти \normalfont{} "--- обезуметь; испортиться; обессилить. 




\bfseries Объюродити \normalfont{} "--- обезуметь; поглупеть. 




\bfseries Ов \normalfont{} "--- иной; один. 




\bfseries Овамо \normalfont{} "--- там; туда. 




\bfseries Овен \normalfont{} "--- баран. 




\bfseries Ово \normalfont{} "--- или; либо. 




\bfseries Овогда \normalfont{} "--- иногда. 




\bfseries Овоуду \normalfont{} "--- с другой стороны; оттуда. 




\bfseries Огласити \normalfont{} "--- объявить всенародно; научить; просветить. 




\bfseries Оглохновение \normalfont{} "--- глухота. 




\bfseries Огневица \normalfont{} "--- горячка. 




\bfseries Огненосный \normalfont{} "--- носимый в вихрях огня. 




\bfseries Огнепальный \normalfont{} "--- пылающий; горящий; палящий. 




\bfseries Огребатися \normalfont{} "--- удаляться; остерегаться. 




\bfseries Огустети \normalfont{} "--- сгустить; сделать густым; свернуться (о молоке). 




\bfseries Одебелети \normalfont{} "--- растолстеть; огрубеть. 




\bfseries Одесную \normalfont{} "--- справа; по правую руку. 




\bfseries Одесятствовати \normalfont{} "--- выделять десятую часть. 




\bfseries Одигитрия \normalfont{} "--- путеводительница. 




\bfseries Одождити \normalfont{} "--- окропить; оросить; послать в виде дождя; в большом количестве. 




\bfseries Одр \normalfont{} "--- постель; кровать. 




\bfseries Ожестети \normalfont{} "--- сделаться жестким; засохнуть. 




\bfseries Озимение \normalfont{} "--- зимовка. 




\bfseries Озлобление \normalfont{} "--- несчастье; гнев. 




\bfseries Озлобляти \normalfont{} "--- причинять несчастье; гневить; распалять гневом. 




\bfseries Озобати \normalfont{} "--- пожирать. 




\bfseries Окаивати \normalfont{} "--- признавать отверженным. 




\bfseries Окаляти \normalfont{} "--- пачкать; осквернять; марать. 




\bfseries Окаменяти \normalfont{} "--- делать каменным. 




\bfseries Окаянный \normalfont{} "--- достойный проклятия; нечестивый; грешник. 




\bfseries Окаянство \normalfont{} "--- преступность; богоборчество; грех. 




\bfseries Око \normalfont{} "--- глаз. 




\bfseries Окованный \normalfont{} "--- обложенный оковами. 




\bfseries Окормитель \normalfont{} "--- кормчий; правитель. 




\bfseries Окормляти \normalfont{} "--- направлять; руководить; править. 




\bfseries Окоявленне \normalfont{} "--- очевидно; откровенно. 




\bfseries Окрастовети \normalfont{} "--- покрыться коростою. 




\bfseries Окрест \normalfont{} "--- кругом; около. 




\bfseries Окриляемый \normalfont{} "--- ограждаемый крыльями. 




\bfseries Оле \normalfont{} "--- О! 




\bfseries Оловина \normalfont{} "--- любое хмельное питие, отличное от виноградного вина. 




\bfseries Олтарь \normalfont{} "--- алтарь, жертвенник. 




\bfseries Оляденети \normalfont{} "--- зарасти тернием, сорняками. 




\bfseries Омакати \normalfont{} "--- обливать. 




\bfseries Ометы \normalfont{} "--- полы; края одежды. 




\bfseries Она \normalfont{} "--- они (двое). 




\bfseries Онагр \normalfont{} "--- дикий осел. 




\bfseries Онамо, онуду \normalfont{} "--- там; туда. 




\bfseries Онде \normalfont{} "--- в ином месте; там. 




\bfseries Онема \normalfont{} "--- им (двоим). 




\bfseries Он пол \normalfont{} "--- противоположный берег. 




\bfseries Онсица \normalfont{} "--- такой-то. 




\bfseries Опасно \normalfont{} "--- осмотрительно; тщательно; осторожно; опасно. 




\bfseries Оплазивый \normalfont{} "--- любопытный; пустословный; лазутчик. 




\bfseries Оплазнство \normalfont{} "--- ухищрение; пустословие. 




\bfseries Оплот \normalfont{} "--- ограда; забор; тын. 




\bfseries Ополчатися \normalfont{} "--- готовиться к сражению. 




\bfseries Оправдание \normalfont{} "--- заповедь; устав; закон. 




\bfseries Опреснок \normalfont{} "--- пресный хлеб, испеченный без использования дрожжей. 




\bfseries Орало \normalfont{} "--- плуг; соха. 




\bfseries Оранный \normalfont{} "--- распаханный. 




\bfseries Оратай \normalfont{} "--- пахарь. 




\bfseries Орати \normalfont{} "--- пахать. 




\bfseries Орган \normalfont{} "--- орган, музыкальный инструмент. 




\bfseries Осанна \normalfont{} "--- молитвенное восклицание у евреев-"спасение (от Бога)". 




\bfseries Оселский \normalfont{} "--- ослиный. 




\bfseries Жернов оселский \normalfont{} "--- верхний большой жернов в мельнице, приводимый в движение ослом. 




\bfseries Осенити \normalfont{} "--- покрыть тенью. 




\bfseries Осклабитися \normalfont{} "--- усмехнуться; улыбнуться. 




\bfseries Оскорбети \normalfont{} "--- опечалиться; соскучиться. 




\bfseries Оскорд \normalfont{} "--- топор. 




\bfseries Ослаба \normalfont{} "--- облегчение; льгота. 




\bfseries Осля \normalfont{} "--- молодой осел. 




\bfseries Осмица \normalfont{} "--- восемь. 




\bfseries Осмоктати \normalfont{} "--- обсосать; облизать. 




\bfseries Оставити \normalfont{} "--- оставить; простить; позволить. 




\bfseries Остенити \normalfont{} "--- огородить стеной, защитить. 




\bfseries Острастший \normalfont{} "--- обидящий. 




\bfseries Острог \normalfont{} "--- земляной вал. 




\bfseries Острупити \normalfont{} "--- поразить проказой. 




\bfseries Осуществовати \normalfont{} "--- осуществлять; давать бытие. 




\bfseries Осьмерицею \normalfont{} "--- восемь раз. 




\bfseries Отай \normalfont{} "--- тайно; скрытно. 




\bfseries Отверзати \normalfont{} "--- открывать; отворять. 




\bfseries Отвнеуду \normalfont{} "--- снаружи. 




\bfseries Отдати \normalfont{} "--- иногда: простить. 




\bfseries Отдоенное \normalfont{} "--- грудной младенец. 




\bfseries Отдоитися \normalfont{} "--- воскормить грудью. 




\bfseries Отерпати \normalfont{} "--- делаться твердым (терпким); деревенеть; отвердевать; неметь. 




\bfseries Отити в путь всея земли \normalfont{} "--- умереть. 




\bfseries Откосненно \normalfont{} "--- наискось. 




\bfseries Откровение \normalfont{} "--- открытие; просветление; просвещение. 




\bfseries Отлог \normalfont{} "--- ущерб; урон. 




\bfseries Отложение \normalfont{} "--- отвержение; отступление. 




\bfseries Отметатися \normalfont{} "--- отрекаться; не признавать; отвергаться; отпадать. 




\bfseries Отметный \normalfont{} "--- отвергнутый; запрещенный. 




\bfseries Отнелиже \normalfont{} "--- с тех пор как; с того времени как. 




\bfseries Отнюд \normalfont{} "--- совершенно; отнюдь. 




\bfseries Отнюдуже, отонюдуже \normalfont{} "--- откуда; почему. 




\bfseries Отобоюду \normalfont{} "--- с той и с другой стороны. 




\bfseries Отонуду \normalfont{} "--- с другой стороны. 




\bfseries Отполу \normalfont{} "--- от половины; с середины. 




\bfseries Отре \normalfont{} "--- сор; мякина; кожура. 




\bfseries Отребить \normalfont{} "--- очистить; ощипать. 




\bfseries Отрешати \normalfont{} "--- отвязывать; освобождать. 




\bfseries Отрешатися \normalfont{} "--- разлучаться. 




\bfseries Отреяти \normalfont{} "--- отбрасывать; отвергать. 




\bfseries Отрицатися \normalfont{} "--- отвергать; отметать. 




\bfseries Отрождение \normalfont{} "--- возрождение. 




\bfseries Отрок \normalfont{} "--- раб; служитель; мальчик до двенадцати лет; ученик; воин. 




\bfseries Отроковица \normalfont{} "--- девица до двенадцати лет. 




\bfseries Отроча \normalfont{} "--- дитя; младенец. 




\bfseries Отрыгнути \normalfont{} "--- извергнуть. 




\bfseries Отрыгнуть слово \normalfont{} "--- произнести. 




\bfseries Оттоле \normalfont{} "--- с того времени. 




\bfseries Отторгати \normalfont{} "--- открывать; отталкивать. 




\bfseries Оцеждати \normalfont{} "--- процеживать. 




\bfseries Оцет \normalfont{} "--- уксус. 




\bfseries Отщетевати \normalfont{} "--- отнимать; удалять. 




\bfseries Отщетити \normalfont{} "--- потерять; погубить. 




\bfseries Очепие \normalfont{} "--- ошейник. 




\bfseries Очеса \normalfont{} "--- очи, глаза. 




\bfseries Ошаяватися \normalfont{} "--- устраняться, удаляться. 




\bfseries Ошиб \normalfont{} "--- хвост. 




\bfseries Ошуюю \normalfont{} "--- слева; по левую руку. 




 





\bfseries Павечерня, павечерница \normalfont{} "--- малая вечерня. 




\bfseries Паволока \normalfont{} "--- покрывало; чехол; пелена; покров. 




\bfseries Пагуба \normalfont{} "--- гибель; моровая язва. 




\bfseries Пажить \normalfont{} "--- луг; нива; пастбище; поле; корм для скота. 




\bfseries Пазнокти \normalfont{} (мн. ч.) "--- копыта; когти; ногти. 




\bfseries Паки \normalfont{} "--- опять; еще; снова. 




\bfseries Пакибытие \normalfont{} "--- духовное обновление. 




\bfseries Пакости деяти \normalfont{} "--- бить руками; ударять по щеке; оскорблять; вредить. 




\bfseries Пакостник \normalfont{} "--- причинитель зла, вреда; болезнь; боль; жало. 




\bfseries Пакость \normalfont{} "--- гадость; нечистота; мерзость. 




\bfseries Палата \normalfont{} "--- дворец. 




\bfseries Иже в палате суть \normalfont{} "--- правительство. 




\bfseries Палестра \normalfont{} "--- место для соревнований. 




\bfseries Палителище \normalfont{} "--- сильный огонь. 




\bfseries Палительный \normalfont{} "--- сожигающий. 




\bfseries Палица \normalfont{} "--- трость; дубина; палка. 




\bfseries Паличник \normalfont{} "--- ликтор; телохранитель; полицейский пристав. 




\bfseries Памятозлобие \normalfont{} "--- злопамятство. 




\bfseries Панфирь \normalfont{} "--- пантера или лев. 




\bfseries Пара \normalfont{} "--- пар; мгла; дым. 




\bfseries Параекклесиарх \normalfont{} "--- кандиловжигатель; пономарь. 




\bfseries Параклис \normalfont{} "--- усердная молитва. 




\bfseries Параклит \normalfont{} "--- утешитель. 




\bfseries Паримия \normalfont{} "--- притча; чтения из Священного Писания на вечерне или царских часах. 




\bfseries Парити \normalfont{} "--- лететь; висеть в воздухе (подобно пару). 




\bfseries Парусия \normalfont{} "--- торжественное шествие; второе славное пришествие Господа нашего Иисуса Христа; торжественное архиерейское богослужение. 




\bfseries Пасомый \normalfont{} "--- пасущийся; находящийся в ведении пастыря. 




\bfseries Паствити \normalfont{} "--- пасти. 




\bfseries Паствуемый \normalfont{} "--- имеющий пастыря. 




\bfseries Пастися \normalfont{} "--- согрешить (особенно против седьмой заповеди). 




\bfseries Пастыреначальник \normalfont{} "--- начальник над пастырями. 




\bfseries Пастырь \normalfont{} "--- пастух. 




\bfseries Паучина \normalfont{} "--- паутина. 




\bfseries Паче \normalfont{} "--- лучше; больше. 




\bfseries Паче естества \normalfont{} "--- сверхъестественно. 




\bfseries Паче слова \normalfont{} "--- невыразимо. 




\bfseries Паче ума \normalfont{} "--- непостижимо. 




\bfseries Певк, певг \normalfont{} "--- хвойное дерево. 




\bfseries Педагогон \normalfont{} "--- детородный член. 




\bfseries Пекло \normalfont{} "--- горючая сера, смола; неперестающий огонь. 




\bfseries Пентикостарий \normalfont{} "--- название "Триоди цветной". 




\bfseries Пентикостия \normalfont{} "--- Пятидесятница. 




\bfseries Пеняжник \normalfont{} "--- меняла. 




\bfseries Пенязь \normalfont{} "--- мелкая монета. 




\bfseries Первее \normalfont{} "--- прежде; сперва; вначале; наперед. 




\bfseries Первоверховный \normalfont{} "--- первый из верховных. 




\bfseries Первовозлежание \normalfont{} "--- возлежание, восседание на первых, почетных местах в собраниях. 




\bfseries Первоначаток \normalfont{} "--- первородное животное или первоснятый плод. 




\bfseries Первостоятель \normalfont{} "--- первенствующий священнослужитель. 




\bfseries Пернатый \normalfont{} "--- имеющий перья. 




\bfseries Перси \normalfont{} (мн. ч.) "--- грудь; передняя часть тела. 




\bfseries Перст \normalfont{} "--- палец. 




\bfseries Перст возложити на уста \normalfont{} "--- замолчать. 




\bfseries Перстный \normalfont{} "--- земляной; сделанный из земли. 




\bfseries Перстосозданный \normalfont{} "--- сотворенный из персти. 




\bfseries Персть \normalfont{} "--- прах; земля; пыль. 




\bfseries Песнопети \normalfont{} "--- прославлять в песнях. 




\bfseries Песнословити \normalfont{} "--- см. Песнопети. 




\bfseries Пестовати \normalfont{} "--- нянчить; воспитывать. 




\bfseries Пестротный \normalfont{} "--- разноцветный; нарядный. 




\bfseries Пестун \normalfont{} "--- воспитатель; педагог; дядька. 




\bfseries Петель \normalfont{} "--- петух. 




\bfseries Петлоглашение \normalfont{} "--- пение петуха; раннее утро; время от 12 до 3 часов ночи, по народному счету времени у иудеев. 




\bfseries Печаловати \normalfont{} (ся) "--- сетовать, тужить; печалиться. 




\bfseries Печаловник \normalfont{} "--- опекун. 




\bfseries Печатствовати \normalfont{} "--- запечатывать; утверждать; сокрывать. 




\bfseries Печать \normalfont{} "--- перстень. 




\bfseries Пешешествовати \normalfont{} "--- идти пешком. 




\bfseries Пещися \normalfont{} "--- заботиться; иметь попечение. 




\bfseries Пивный \normalfont{} "--- то, что можно выпить. 




\bfseries Пиво \normalfont{} "--- питие; напиток. 




\bfseries Пиган \normalfont{} "--- рута, трава. 




\bfseries Пира \normalfont{} "--- сума; котомка. 




\bfseries Пирга \normalfont{} "--- башня; столп. 




\bfseries Писало \normalfont{} "--- остроконечная трость для писания на вощаной дощечке. 




\bfseries Писание ставильное \normalfont{} "--- ставленная грамота, даваемая архиереем новопосвященному пресвитеру или диакону. 




\bfseries Пискати \normalfont{} "--- играть на свирели. 




\bfseries Писмя \normalfont{} "--- буква; графический знак; буквальный смысл. 




\bfseries Пистикия \normalfont{} "--- чистый; беспримесный. 




\bfseries Питенный \normalfont{} "--- возлелеянный; выращенный в неге. 




\bfseries Питомый \normalfont{} "--- откормленный; дебелый. 




\bfseries Пищный \normalfont{} "--- содержащий обильную пищу; питательный. 




\bfseries Пиянство \normalfont{} "--- пьянство. 




\bfseries Плавы \normalfont{} "--- нивы. 




\bfseries Плавый \normalfont{} "--- зрелый; спелый, соломенного цвета. 




\bfseries Плат \normalfont{} "--- лоскут; заплатка. 




\bfseries Плащаница \normalfont{} "--- погребальные пелены; покрывало; полотно; плащ. 




\bfseries Плевел \normalfont{} "--- сорняк; негодная трава. 




\bfseries Плежити \normalfont{} "--- ползать на чреве; пресмыкаться. 




\bfseries Плежущий \normalfont{} "--- пресмыкающийся. 




\bfseries Пленица \normalfont{} "--- косичка; цепочка; ожерелье; корзина; цепь; оковы; узы. 




\bfseries Плескати \normalfont{} "--- бить в ладоши; аплодировать. 




\bfseries Плесна \normalfont{} "--- стопа; ступня. 




\bfseries Плесница \normalfont{} "--- обувь типа сандалий. 




\bfseries Плещи \normalfont{} "--- плечи. 




\bfseries Плещущий \normalfont{} "--- ударяющий в ладоши. 




\bfseries Плинфа \normalfont{} "--- кирпич. 




\bfseries Плинфоделание \normalfont{} "--- обжигание кирпичей. 




\bfseries Плищ \normalfont{} "--- крик; шум. 




\bfseries Плодствовати \normalfont{} "--- приносить плоды. 




\bfseries Плод устен \normalfont{} "--- слово; голос. 




\bfseries Плод чрева \normalfont{} "--- ребенок; дети. 




\bfseries Плотолюбие \normalfont{} "--- забота о теле. 




\bfseries Плотски \normalfont{} "--- плотью; телесно. 




\bfseries Плотский \normalfont{} "--- плотский; чувственный; телесный. 




\bfseries Плоть \normalfont{} "--- тело; человек; немощь или слабость человека; страсть. 




\bfseries Плюновение \normalfont{} "--- слюна. 




\bfseries Плясалище \normalfont{} "--- балаган. 




\bfseries Плясица, плясавица \normalfont{} "--- танцовщица; актриса. 




\bfseries Победительно \normalfont{} "--- торжественно; победоносно. 




\bfseries Поболети \normalfont{} "--- тужить; сожалеть. 




\bfseries Поборник \normalfont{} "--- защитник. 




\bfseries Повапленный \normalfont{} "--- покрашенный; побеленный. 




\bfseries Повергнути \normalfont{} "--- бросить; опрокинуть. 




\bfseries Повесть \normalfont{} "--- рассказ. 




\bfseries Повити \normalfont{} "--- принять роды или обвить пеленами. 




\bfseries Повои \normalfont{} "--- повязка; пелена. 




\bfseries Поглумитися \normalfont{} "--- рассуждать; размышлять; подумать. 




\bfseries Подвигнуться \normalfont{} "--- трепетать; двинуться. 




\bfseries Подвизати \normalfont{} "--- побуждать; поощрять. 




\bfseries Подвизатися \normalfont{} "--- совершать подвиги; трудиться. 




\bfseries Подвои \normalfont{} "--- косяки дверей. 




\bfseries Подникати \normalfont{} "--- наклоняться; нагибаться. 




\bfseries Подобитися \normalfont{} "--- напоминать что-либо. 




\bfseries Подобник \normalfont{} "--- подражатель. 




\bfseries Подобозрачен \normalfont{} "--- внешне похожий. 




\bfseries Подточилие \normalfont{} "--- сосуд для собирания выжатого сока. 




\bfseries Подъяремник \normalfont{} "--- находящийся под ярмом (например, осел). 




\bfseries Подъяремничий \normalfont{} "--- принадлежащий подъяремнику. 




\bfseries Поелику \normalfont{} "--- поскольку; потому что; так как; насколько. 




\bfseries Поелику аще \normalfont{} "--- сколько бы ни. 




\bfseries Пожрети \normalfont{} "--- принести в жертву. 




\bfseries Позде \normalfont{} "--- поздно; не рано. 




\bfseries Позобати \normalfont{} "--- склевать. 




\bfseries Позорище \normalfont{}




\bfseries , позор \normalfont{} "--- многолюдное зрелище. 




\bfseries Поимати \normalfont{} "--- брать. 




\bfseries Полма \normalfont{} "--- пополам; надвое. 




\bfseries Польский \normalfont{} "--- полевой. 




\bfseries Помавати, поманути \normalfont{} "--- делать знаки; изъясняться без слов. 




\bfseries Помале \normalfont{} "--- вскоре; немного погодя. 




\bfseries Поматы \normalfont{} "--- скрижали на мантиях архиерейских. 




\bfseries Пометати \normalfont{} "--- мести; выметати; бросать. 




\bfseries Помизати \normalfont{} "--- мигать. 




\bfseries Поне \normalfont{} "--- хотя; по крайней мере; так как. 




\bfseries Понеже \normalfont{} "--- потому что; так как. 




\bfseries Понос, поношение \normalfont{} "--- позор; бесславие. 




\bfseries Понт \normalfont{} "--- море; большое озеро. 




\bfseries Понява, понявица \normalfont{} "--- полотенце. 




\bfseries Пооблещися \normalfont{} "--- надеть сверху другую одежду. 




\bfseries Поострити \normalfont{} "--- наточить. 




\bfseries Поползнутися \normalfont{} "--- поскользнуться; совратиться; соблазниться. 




\bfseries Поприще \normalfont{} "--- мера длины, равная тысяче шагов или суточному переходу. 




\bfseries Пореватися \normalfont{} "--- порываться; стремиться; двигаться. 




\bfseries Поречение \normalfont{} "--- обвинение; жалоба; упрек; попрек. 




\bfseries Порещи \normalfont{} "--- обвинить; укорить; осудить. 




\bfseries Поругание \normalfont{} "--- бесчестие; поношение; воспаление; язва. 




\bfseries Поругати \normalfont{} "--- обесчестить. 




\bfseries Порфира \normalfont{} "--- ткань темно-красного цвета; порфира, пурпурная одежда высокопоставленных особ. 




\bfseries Порча \normalfont{} "--- яд; отрава. 




\bfseries Поряду \normalfont{} "--- по порядку. 




\bfseries Поскору \normalfont{} "--- скоро; бегло; без пения (о службе). 




\bfseries Последи \normalfont{} "--- затем; в конце концов. 




\bfseries Последний \normalfont{} "--- остальной; конечный; окончательный. 




\bfseries Последование \normalfont{} "--- изложение молитвословий только одного рода, т.е. или изменяемых, или неизменяемых. 




\bfseries Последовати \normalfont{} "--- исследовать; следовать. 




\bfseries Послушествовати \normalfont{} "--- свидетельствовать; давать показания. 




\bfseries Посолонь \normalfont{} "--- по-солнечному; как солнце; от востока на запад. 




\bfseries Поспешествовати \normalfont{} "--- помогать; пособлять. 




\bfseries Поспешник \normalfont{} "--- пособник; помощник. 




\bfseries Посреде \normalfont{} "--- посередине. 




\bfseries Поставление \normalfont{} "--- посвящение в сан. 




\bfseries Постриг \normalfont{} "--- пострижение в монашество. 




\bfseries Посупление \normalfont{} "--- наклонение головы в печали; печаль; грусть; сетование. 




\bfseries Посягати \normalfont{} "--- вступать в брак. 




\bfseries Потворник \normalfont{} "--- угодник; льстец; чародей; колдун. 




\bfseries Потворы \normalfont{} "--- чародейство; колдовство. 




\bfseries Поткнутися \normalfont{} "--- споткнуться. 




\bfseries Потреба \normalfont{} "--- потребность; необходимость; случай. 




\bfseries Потребник \normalfont{} "--- блин; лепешка. 




\bfseries Потщитися \normalfont{} "--- поспешить; постараться. 




\bfseries Поуститель \normalfont{} "--- подстрекатель. 




\bfseries Поущати \normalfont{} "--- поощрять; побуждать; наставлять; поучать. 




\bfseries Похотствовати \normalfont{} "--- иметь вожделение, похоть. 




\bfseries Почерпало \normalfont{} "--- бадья; кошель; ведро. 




\bfseries Починатися \normalfont{} "--- начинаться. 




\bfseries Почити \normalfont{} "--- успокоиться. 




\bfseries Пояти \normalfont{} "--- взять. 




\bfseries Правый \normalfont{} "--- прямой; истинный; правильный; праведный. 




\bfseries Праг \normalfont{} "--- порог. 

$<$;p$>$\bfseries Празднословие \normalfont{} "--- пустой; вздорный разговор. 



\bfseries Праздный, празден \normalfont{} "--- беспредельный; ленивый; пустой; незанятый. 




\bfseries Прати \normalfont{} "--- попирать; давить. 




\bfseries Превзятися \normalfont{} "--- превознестись; возгордиться. 




\bfseries Превитати \normalfont{} "--- странствовать. 




\bfseries Превозвышенное око \normalfont{} "--- высокоумие; гордость. 




\bfseries Превратити \normalfont{} "--- изменить; поворотить; разрушить. 




\bfseries Превременный \normalfont{} "--- предвечный, существовавший до начала времени. 




\bfseries Предвзыграти \normalfont{} "--- предвозвестить радостью. 




\bfseries Предвозгласити \normalfont{} "--- начать пение; предвозвестить. 




\bfseries Предградие \normalfont{} "--- пригород; оплот; защита; ограда. 




\bfseries Преддворие \normalfont{} "--- передний, внешний двор в восточном доме. 




\bfseries Предзаклатися \normalfont{} "--- прежде других вкусить смерть, принести себя в жертву. 




\bfseries Преди \normalfont{} "--- впереди. 




\bfseries Предитещи \normalfont{} "--- бежать впереди. 




\bfseries Предложение \normalfont{} "--- жертвенник; то место в алтаре, где стоит жертвенник и хранятся священные сосуды. 




\bfseries Предначинательный псалом \normalfont{} "--- название псалма 103, поскольку им начинается вечерня. 




\bfseries Предний \normalfont{} "--- первый; вящий; изящный; старший. 




\bfseries Предпряда \normalfont{} "--- ткань темно-красного цвета; порфира, пурпурная одежда высокопоставленных особ. 




\bfseries Предстательство \normalfont{} "--- ходатайство; заступничество; усердная молитва. 




\bfseries Предстолпие \normalfont{} "--- укрепление. 




\bfseries Предстоятель \normalfont{} "--- настоятель. 




\bfseries Предтеча, предитеча \normalfont{} "--- идущий или бегущий впереди. 




\bfseries Предуставити \normalfont{} "--- предназначить. 




\bfseries Предусрести \normalfont{} "--- встретить заранее. 




\bfseries Предуведети \normalfont{} "--- предвидеть; знать заранее. 




\bfseries Предъявленне \normalfont{} "--- предображая. 




\bfseries Презорливый \normalfont{} "--- гордый; надменный. 




\bfseries Преизбыточествовати \normalfont{} "--- быть довольну; жить в изобилии. 




\bfseries Преизлиха \normalfont{} "--- сильно; очень; жестоко. 




\bfseries Преимение \normalfont{} "--- преимущество; превосходство. 




\bfseries Преиспещренный \normalfont{} "--- разукрашенный. 




\bfseries Преисподний \normalfont{} "--- самый низкий. 




\bfseries Преисподняя \normalfont{} "--- место нахождения душ умерших до освобождения их Господом Иисусом Христом; место вечного мучения грешников; жилище диавола. 




\bfseries Преитие \normalfont{} "--- превосхождение. 




\bfseries Прелагати пределы \normalfont{} "--- портить межи; нарушать границы. 




\bfseries Прелесть \normalfont{} "--- обман. 




\bfseries Прелюбы \normalfont{} "--- прелюбодейство. 




\bfseries Премудрость \normalfont{} "--- высшее знание; мудрость. 




\bfseries Преначальный \normalfont{} "--- доначальный; превышающий всякое начало. 




\bfseries Преогорчити \normalfont{} "--- противиться; быть непокорным; упрямиться. 




\bfseries Преодеян \normalfont{} "--- обильно украшен. 




\bfseries Преоруженный \normalfont{} "--- слишком вооруженный; гордый. 




\bfseries Препирати \normalfont{} "--- опровергать; отражать; отбивать; одолевать; увещевать. 




\bfseries Преподобие \normalfont{} "--- святость. 




\bfseries Преполовение \normalfont{} "--- половина; середина. 




\bfseries Преполовити \normalfont{} "--- переполовинить; разделить пополам; пройти половину пути. 




\bfseries Препона \normalfont{} "--- препятствие. 




\bfseries Препоясатися \normalfont{} "--- подпоясаться; приготовиться к чему-либо. 




\bfseries Препретельный \normalfont{} "--- спорный; убедительный. 




\bfseries Препростый \normalfont{} "--- неученый; невежда. 




\bfseries Препяти \normalfont{} "--- остановить. 




\bfseries Пререкаемый \normalfont{} "--- спорный. 




\bfseries Пререкати \normalfont{} "--- прекословить; говорить наперекор; перечить. 




\bfseries Пресецающий \normalfont{} "--- пересекающий; перерубающий. 




\bfseries Преслушание \normalfont{} "--- неповиновение. 




\bfseries Пресмыкаться \normalfont{} "--- ползти по земле. 




\bfseries Преспевати \normalfont{} "--- иметь успех. 




\bfseries Преставити \normalfont{} "--- переставить; переместить; переселить в вечность. 




\bfseries Престоли \normalfont{} "--- один из чинов ангельских. 




\bfseries Пресущественный \normalfont{} "--- предвечный; исконный. 




\bfseries Пресущный \normalfont{} "--- сверхъестественный. 




\bfseries Претися \normalfont{} "--- спорить; тягаться. 




\bfseries Претити \normalfont{} "--- запрещать; скорбеть; смущаться. 




\bfseries Преткновение \normalfont{} "--- помеха; соблазн; задержка; остановка. 




\bfseries Претор \normalfont{} "--- претория, резиденция представителя римской власти в Иерусалиме. 




\bfseries Претыкание \normalfont{} "--- помеха; соблазн; задержка; остановка. 




\bfseries Прещати \normalfont{} "--- грозить; устрашать. 




\bfseries Прещение \normalfont{} "--- угроза; страх; запрет. 




\bfseries Прибежище \normalfont{} "--- убежище; приют; покров; спасение. 




\bfseries Приведение \normalfont{} "--- доступ. 




\bfseries Привещевати \normalfont{} "--- приветствовать. 




\bfseries Привлещи \normalfont{} "--- притащить; позвать; призвать. 




\bfseries Привременный \normalfont{} "--- временный; непостоянный. 




\bfseries Придевати \normalfont{} "--- прицеплять; приближаться; подносить. 




\bfseries Придел \normalfont{} "--- небольшая церковь, пристроенная к главному храму. 




\bfseries Приделати \normalfont{} "--- прирастить; увеличить; принести. 




\bfseries Придеяти \normalfont{} "--- подносить; приносить. 




\bfseries Призрети \normalfont{} "--- милостиво посмотреть; принять; приютить. 




\bfseries Прииждивати \normalfont{} "--- расходовать; издерживать. 




\bfseries Приискренне \normalfont{} "--- точно так же; равно; точь-в-точь. 




\bfseries Прикровение \normalfont{} "--- прикрытие; предлог; выдуманная причина для сокрытия чего-либо. 




\bfseries Прикуп \normalfont{} "--- барыш; прибыль. 




\bfseries Прикупование \normalfont{} "--- купечество; торговля. 




\bfseries Прилог \normalfont{} "--- приложение; желание сделать зло; злоба; клевета. 




\bfseries Приложение \normalfont{} "--- заплатка; лоскут. 




\bfseries Приметати \normalfont{} "--- прибрасывать; отдавать; уступать. 




\bfseries Приметатися \normalfont{} "--- припадать; отдаваться; лежать у порога. 




\bfseries Примешатися \normalfont{} "--- присоединяться. 




\bfseries Приникнути \normalfont{} "--- пригнуться; наклониться; припасть; проникнуть. 




\bfseries Приобряща \normalfont{} "--- польза; плод; корысть. 




\bfseries Приразитися \normalfont{} "--- напасть; удариться. 




\bfseries Приревание \normalfont{} "--- устремление. 




\bfseries Приристати \normalfont{} "--- подбегать. 




\bfseries Пририщущий \normalfont{} "--- подбегающий. 




\bfseries Присвянути \normalfont{} "--- завянуть; засохнуть. 




\bfseries Приседение \normalfont{} "--- угнетение; окружение. 




\bfseries Приседети \normalfont{} "--- находиться около чего-либо; замышлять зло; нападать. 




\bfseries Присно \normalfont{} "--- непрестанно; всегда. 




\bfseries Присноживотный \normalfont{} "--- всегда живущий. 




\bfseries Присносущий \normalfont{} "--- вечный; всегдашний. 




\bfseries Присносущный \normalfont{} "--- всегда существующий. 




\bfseries Приснотекущий \normalfont{} "--- неиссякаемый. 




\bfseries Присный \normalfont{} "--- родной; близкий. 




\bfseries Приставление \normalfont{} "--- заплатка; назначение; управление; присмотр. 




\bfseries Пристанище \normalfont{} "--- приют; убежище; пристань. 




\bfseries Пристати \normalfont{} "--- прибегнуть; подбежать. 




\bfseries Пристрашен \normalfont{} "--- испуган. 




\bfseries Притвор \normalfont{} "--- вход в храм. 




\bfseries Прителный \normalfont{} "--- спорный. 




\bfseries Притча \normalfont{} "--- иносказание; загадка. 




\bfseries Причаститися \normalfont{} "--- стать участником. 




\bfseries Причастник \normalfont{} "--- участник. 




\bfseries Пришлец \normalfont{} "--- приезжий; пришелец. 




\bfseries Приятилище \normalfont{} "--- вместилище; поместилище; хранилище. 




\bfseries Пробавити \normalfont{} "--- продолжить; протянуть. 




\bfseries Продерзивый \normalfont{} "--- дерзкий. 




\bfseries Прозябение \normalfont{} "--- произрастание; росток. 




\bfseries Прозябнути \normalfont{} "--- расцвести; вырасти; произрастить. 




\bfseries Произникнути \normalfont{} "--- произойти; вырасти. 




\bfseries Пролитися стопам \normalfont{} "--- поскользнуться; иносказательно "--- согрешить. 




\bfseries Пронарещи \normalfont{} "--- предсказать; предназначить. 




\bfseries Проникнути \normalfont{} "--- вырасти; процвесть. 




\bfseries Проничение \normalfont{} "--- племя; род; стебель; росток. 




\bfseries Проповедати \normalfont{} "--- учить; провозглашать; проповедовать. 




\bfseries Прорещи \normalfont{} "--- предсказать. 




\bfseries Пророкованный \normalfont{} "--- предсказанный; предвозвещенный. 




\bfseries Пророковещательный \normalfont{} "--- говоримый пророком. 




\bfseries Проручествовати \normalfont{} "--- посвящать; рукополагать. 




\bfseries Просаждатися \normalfont{} "--- разрываться. 




\bfseries Просветительный \normalfont{} "--- светлый; просвещающий. 




\bfseries Просветити лице$<$ \normalfont{} "--- весело или милостивно взглянуть. 




\bfseries Проскомисати \normalfont{} "--- совершать проскомидию. 




\bfseries Прослутие \normalfont{} "--- притча; пословица; осмеяние. 




\bfseries Простый \normalfont{} "--- стоящий прямо; прямой; чистый; несмешанный. 




\bfseries Простыня \normalfont{} "--- сострадание. 




\bfseries Просядати \normalfont{} "--- разрываться; разваливаться; трескаться. 




\bfseries Протерзатися \normalfont{} "--- прорываться. 




\bfseries Противозрети \normalfont{} "--- смотреть прямо. 




\bfseries Противу, прямо \normalfont{} "--- против; напротив. 




\bfseries Проуведети \normalfont{} "--- узнать заранее; предвидеть. 




\bfseries Проявленне \normalfont{} "--- явно. 




\bfseries Пругло \normalfont{} "--- силок; петля; сеть. 




\bfseries Прудный \normalfont{} "--- неровный; каменистый. 




\bfseries Пружатися \normalfont{} "--- сопротивляться (отсюда "--- пружина); биться в припадке. 




\bfseries Пружие \normalfont{} "--- пища Иоанна Крестителя; по мнению одних "--- род съедобной саранчи, или кузнечиков; по мнению других "--- какое-то растение. 




\bfseries Пря \normalfont{} "--- спор; тяжба; беспорядок. 




\bfseries Пряжмо \normalfont{} "--- жареная пища. 




\bfseries Прямный \normalfont{} "--- находящийся напротив. 




\bfseries Пустити \normalfont{} "--- отпустить; развестись. 




\bfseries Пустыня \normalfont{} "--- уединенное, малообитаемое место. 




\bfseries Пустыня \normalfont{} "--- монастырь, расположенный вдалеке от населенных мест. 




\bfseries Путесотворити \normalfont{} "--- сохранять в пути; проложить дорогу. 




\bfseries Путы \normalfont{} "--- узы; кандалы; цепи; оковы. 




\bfseries Пучина \normalfont{} "--- водоворот; море. 




\bfseries Пучинородный \normalfont{} "--- морской; родившийся в море. 




\bfseries Пущеница \normalfont{} "--- разведенная с мужем женщина. 




\bfseries Птицеволхвование \normalfont{} "--- суеверие, состоящее в гадании по полету птиц или по их внутренностям. 




\bfseries Пядь, пядень \normalfont{} "--- мера длины, равная трем дланям, а каждая длань равна четырем перстам, а перст равен четырем граням или зернам. 




\bfseries Пясть \normalfont{} "--- кулак. 




\bfseries Пяток \normalfont{} "--- пятница. 




 





\bfseries Рабий \normalfont{} "--- рабский. 




\bfseries Работа \normalfont{} "--- рабство. 




\bfseries Работен \normalfont{} "--- покорен; порабощен. 




\bfseries Равви, раввуни \normalfont{} "--- учитель. 




\bfseries Равноангельно \normalfont{} "--- подобно Ангелам. 




\bfseries Равноапостольный \normalfont{} "--- сравниваемый с апостолами. 




\bfseries Равнодушный, равнодушевный \normalfont{} "--- единодушный; имеющий одинаковое усердие. 




\bfseries Равночестный \normalfont{} "--- достойный равного почитания. 




\bfseries Радованный \normalfont{} "--- радостный. 




\bfseries Радоватися \normalfont{} "--- радоваться; наслаждаться. 




\bfseries Радощи \normalfont{} "--- радости (мн. ч.); веселье. 




\bfseries Радуйся \normalfont{} "--- здравствуй; прощай. 




\bfseries Раждежение \normalfont{} "--- горение; воспламенение. 




\bfseries Разботети \normalfont{} "--- растолстеть; разбухнуть. 




\bfseries Разве \normalfont{} "--- кроме. 




\bfseries Развет \normalfont{} "--- мятеж; заговор. 




\bfseries Разврат \normalfont{} "--- волнение; возмущение. 




\bfseries Разгбенный \normalfont{} "--- разогнутый. 




\bfseries Разгнутие \normalfont{} "--- разгибание; раскрытие книги. 




\bfseries Раздолие \normalfont{} "--- долина. 




\bfseries Разжизати \normalfont{} "--- разжигать; раскалять; расплавлять. 




\bfseries Размыслити \normalfont{} "--- усомниться; задуматься; остановиться. 




\bfseries Разнство \normalfont{} "--- различие. 




\bfseries Разрешити \normalfont{} "--- развязать; освободить. 




\bfseries Разслабленный \normalfont{} "--- паралитик. 




\bfseries Разум \normalfont{} "--- ум; познание; разумение. 




\bfseries Разумети телом \normalfont{} "--- почувствовать. 




\bfseries Рака \normalfont{} "--- евр. дурак; пустой человек. 




\bfseries Рака \normalfont{} "--- гробница; ковчег с мощами святого угодника Божия. 




\bfseries Рало \normalfont{} "--- соха; плуг. 




\bfseries Рамо \normalfont{} "--- плечо. 




\bfseries Рамена \normalfont{} "--- плечи. 




\bfseries Расплощатися \normalfont{} "--- развертываться. 




\bfseries Распростирати \normalfont{} "--- расстеливать; разворачивать. 




\bfseries Распудити \normalfont{} "--- распугать; разогнать; рассеять. 




\bfseries Распутие \normalfont{} "--- перекресток. 




\bfseries Раст \normalfont{} "--- росток. 




\bfseries Растерзати \normalfont{} "--- разорвать. 




\bfseries Растнити \normalfont{} "--- рассечь. 




\bfseries Расточати \normalfont{} "--- рассеивать; рассыпать; проматывать; беспутно проживать. 




\bfseries Растренный \normalfont{} "--- перепиленный. 




\bfseries Расчинити \normalfont{} "--- расположить по порядку. 




\bfseries Ратай \normalfont{} "--- воин. 




\bfseries Ратовати \normalfont{} "--- воевать; сражаться; отстаивать. 




\bfseries Ратовище \normalfont{} "--- древко копья. 




\bfseries Рать \normalfont{} "--- война; воинство. 




\bfseries Рачитель \normalfont{} "--- попечитель; любитель. 




\bfseries Рачительный \normalfont{} "--- заботливый; достойный заботы. 




\bfseries Рвение, ревность \normalfont{} "--- ярость; страстное желание; страсть. 




\bfseries Ребра северова \normalfont{} "--- северный склон горы Сион. 




\bfseries Ревновати \normalfont{} "--- завидовать. 




\bfseries Рек \normalfont{} "--- ты, он сказал. 




\bfseries Рекла \normalfont{} "--- сказала. 




\bfseries Рекомый \normalfont{} "--- прозываемый. 




\bfseries Рекох \normalfont{} "--- я сказал. 




\bfseries Репие \normalfont{} "--- репейник; колючее растение. 




\bfseries Ресно \normalfont{} "--- ресницы; глаз. 




\bfseries Реснота \normalfont{} "--- действительность; истина. 




\bfseries Реть \normalfont{} "--- ссора; спор. 




\bfseries Рещи \normalfont{} "--- сказать; говорить. 




\bfseries Реяти \normalfont{} "--- отталкивать; отбрасывать. 




\bfseries Риза \normalfont{} "--- одежда; священное облачение. 




\bfseries Ризница \normalfont{} "--- помещение для сохранения риз. 




\bfseries Ризничий \normalfont{} "--- начальник над ризницей; хранитель церковной утвари. 




\bfseries Ристалище \normalfont{} "--- стадион; цирк. 




\bfseries Ристати \normalfont{} "--- рыскать; бегать. 




\bfseries Рог \normalfont{} "--- рок; иносказательно: сила; власть; защита. 




\bfseries Род \normalfont{} "--- происхождение; племя; поколение. 




\bfseries Родостама \normalfont{} "--- розовая вода, которой по обычаю в праздник Воздвижения производят омовение Честнаго и Животворящего Креста Господня при его воздвизании. 




\bfseries Рожаный \normalfont{} "--- роговой; напоминающий рог. 




\bfseries Рожец \normalfont{} "--- сладкий стручок. 




\bfseries Розга \normalfont{} "--- молодая ветвь; побег; отпрыск. 




\bfseries Росодательный \normalfont{} "--- росоносный; дающий росу. 




\bfseries Рота \normalfont{} "--- божба; клятва. 




\bfseries Ротитель \normalfont{} "--- клятвопреступник. 




\bfseries Ротитися \normalfont{} "--- клясться; божиться. 




\bfseries Ругатися \normalfont{} "--- насмехаться. 




\bfseries Рукописание \normalfont{} "--- список; письмо; письменный договор; свиток; расписка; обязательство. 




\bfseries Рукоять \normalfont{} "--- горсть; охапка. 




\bfseries Руно \normalfont{} "--- шерсть; овчина. 




\bfseries Ручка \normalfont{} "--- сосуд. 




\bfseries Рцем \normalfont{} "--- скажем (повел.наклонение). 




\bfseries Рцы \normalfont{} "--- скажи. 




\bfseries Рыбарь \normalfont{} "--- рыбак. 




\bfseries Рясно \normalfont{} "--- ожерелье; подвески. 




 





\bfseries Самвик \normalfont{} "--- музыкальный инструмент. 




\bfseries Самовидец \normalfont{} "--- очевидец. 




\bfseries Самогласная стихира \normalfont{} "--- имеющая свой особый распев. 




\bfseries Самоохотие \normalfont{} "--- по собственному желанию. 




\bfseries Самоподобен \normalfont{} "--- стихира, имеющая свой особый распев. 




\bfseries Самочиние \normalfont{} "--- бесчиние; беспорядок. 




\bfseries Сата \normalfont{} "--- мера сыпучих тел. 




\bfseries Сбодати \normalfont{} "--- пронзить; заколоть. 




\bfseries Свара \normalfont{} "--- ссора; брань. 




\bfseries Сваритися \normalfont{} "--- ссориться. 




\bfseries Сведети \normalfont{} "--- ведать; знать. 




\bfseries Светильничное \normalfont{} "--- начало вечерни. 




\bfseries Светлопозлащен \normalfont{} "--- великолепно украшен. 




\bfseries Светлость \normalfont{} "--- светящаяся красота. 




\bfseries Светозарный \normalfont{} "--- озаряющий светом. 




\bfseries Светолитие \normalfont{} "--- сияние. 




\bfseries Светоначальник \normalfont{} "--- создатель светил. 




\bfseries Светоносец \normalfont{} "--- несущий свет. 




\bfseries Свечеряти \normalfont{} "--- совместно с кем-либо участвовать в пиру. 




\bfseries Свидение \normalfont{} "--- наставление; приказание. 




\bfseries Свирепоустие \normalfont{} "--- необузданность языка. 




\bfseries Свиток \normalfont{} "--- сверток; рукопись, намотанная на палочку. 




\bfseries Связание злата \normalfont{} "--- впрядение золотых нитей. 




\bfseries Связень \normalfont{} "--- узник; невольник. 




\bfseries Святилище \normalfont{} "--- алтарь; храм. 




\bfseries Святитель \normalfont{} "--- архиерей; епископ. 




\bfseries Святотатство \normalfont{} "--- похищение священных вещей. 




\bfseries Святцы \normalfont{} "--- месяцеслов (книга, содержащая имена святых, расположенных по дням года); икона "Всех святых". 




\bfseries Священнотаинник \normalfont{} "--- посвященный в Божественные тайны. 




\bfseries Се \normalfont{} "--- вот. 




\bfseries Седмерицею \normalfont{} "--- семикратно. 




\bfseries Седмица \normalfont{} "--- семь дней, которые в современном языке принято называть "неделя". 




\bfseries Седмичный \normalfont{} "--- относящийся к любому из дней седмицы, кроме Недели (воскресного дня); будничный. 




\bfseries Секира \normalfont{} "--- топор. 




\bfseries Секраты \normalfont{} "--- недавно; только что. 




\bfseries Секратый \normalfont{} "--- свежий; новый. 




\bfseries Селный \normalfont{} "--- полевой; дикий. 




\bfseries Село \normalfont{} "--- поле. 




\bfseries Семидал \normalfont{} "--- мелкая пшеничная мука; крупчатка. 




\bfseries Семо \normalfont{} "--- сюда. 




\bfseries Семя \normalfont{} "--- семя; потомки; род. 




\bfseries Сеннописание \normalfont{} "--- неясное изображение. 




\bfseries Сень \normalfont{} "--- тень; покров над престолом. 




\bfseries Септемврий \normalfont{} "--- сентябрь. 




\bfseries Серповидец \normalfont{} "--- наименование святого пророка Захарии. 




\bfseries Серядь \normalfont{} "--- монашеское рукоделие; пряжа. 




\bfseries Сеть \normalfont{} "--- западня. 




\bfseries Сечиво \normalfont{} "--- топор. 




\bfseries Сигклит \normalfont{} (читается "синклит") "--- собрание, сенат. 




\bfseries Сиесть \normalfont{} "--- то есть. 




\bfseries Сикарий \normalfont{} "--- убийца; разбойник. 




\bfseries Сикелия \normalfont{} "--- о. Сицилия. 




\bfseries Сикер \normalfont{} "--- хмельной напиток, изготовленный не из винограда. 




\bfseries Силы \normalfont{} "--- название одного из чинов ангельских; иногда значит чудеса. 




\bfseries Синаксарий \normalfont{} "--- сокращенное изложение житий святых или праздников. 




\bfseries Синедрион \normalfont{} "--- верховное судилище у иудеев. 




\bfseries Синфрог \normalfont{} "--- сопрестолие, т.е. скамьи по обе стороны горнего места для сидения сослужащих архиерею священников. 




\bfseries Сиречь \normalfont{} "--- то есть; именно. 




\bfseries Сирини \normalfont{} "--- (в Ис. 13, 21) "--- страусы; сирены. 




\bfseries Сирт \normalfont{} "--- отмель; мель. 




\bfseries Сирый \normalfont{} "--- сиротливый; одинокий; беспомощный; бедный. 




\bfseries Сице \normalfont{} "--- так; таким образом. 




\bfseries Сицевый \normalfont{} "--- такой; таковой. 




\bfseries Скверна \normalfont{} "--- нечистота; грязь; порок. 




\bfseries Сквозе \normalfont{} "--- сквозь; через. 




\bfseries Скимен \normalfont{} "--- молодой лев; львенок. 




\bfseries Скиния \normalfont{} "--- палатка; шатер. 




\bfseries Скинотворец \normalfont{} "--- делатель палаток. 




\bfseries Скит \normalfont{} "--- маленький монастырь. 




\bfseries Склабитися \normalfont{} "--- улыбаться, усмехаться. 




\bfseries Скнипа \normalfont{} "--- вошь. 




\bfseries Сковник \normalfont{} "--- соучастник; сообщник. 




\bfseries Скоктание \normalfont{} "--- щекотание; подстрекательство. 




\bfseries Скопчий \normalfont{} "--- скопческий. 




\bfseries Скоротеча \normalfont{} "--- скороход; гонец. 




\bfseries Скорпия \normalfont{} "--- скорпион. 




\bfseries Скрания \normalfont{} "--- висок. 




\bfseries Скрижаль \normalfont{} "--- доска; таблица. 




\bfseries Скудель \normalfont{} "--- глина; то, что сделано из глины; кувшин; черепица. 




\bfseries Скудельник \normalfont{} "--- горшечник. 




\bfseries Скудный \normalfont{} "--- бедный; тощий. 




\bfseries Скураты \normalfont{} "--- маски; личины. 




\bfseries Славник \normalfont{} "--- молитвословие, положенное по уставу после "Славы". 




\bfseries Славословие \normalfont{} "--- прославление. 




\bfseries Сладковонный \normalfont{} "--- благоуханный. 




\bfseries Сладкогласие \normalfont{} "--- стройное пение. 




\bfseries Сладкопение \normalfont{} "--- тихое, умильное пение. 




\bfseries Слана \normalfont{} "--- гололедица; мороз; ледник; замерзший иней. 




\bfseries Сланость \normalfont{} "--- соленая морская вода; солончак, т.е. сухая, пропитанная солью земля; гололед. 




\bfseries Сластотворный \normalfont{} "--- обольщающий плотскими удовольствиями. 




\bfseries Сликовствовати \normalfont{} "--- совместно играть; веселиться. 




\bfseries Словесный \normalfont{} "--- разумный. 




\bfseries Словоположение \normalfont{} "--- договор; условия. 




\bfseries Сложитися \normalfont{} "--- уговориться; определить. 




\bfseries Слота \normalfont{} "--- ненастье; дурная погода. 




\bfseries Слух \normalfont{} "--- слава; народная молва. 




\bfseries Слякий, слукий \normalfont{} "--- согнутый; скорченный; горбатый. 




\bfseries Сляцати \normalfont{} "--- сгибать; горбить. 




\bfseries Смарагд \normalfont{} "--- изумруд. 




\bfseries Смежити \normalfont{} "--- сблизить; соединять края, межи; закрывать. 




\bfseries Смерчие \normalfont{} "--- кедр. 




\bfseries Смеситися \normalfont{} "--- перемещаться; совокупиться плотски. 




\bfseries Смиряти \normalfont{} "--- унижать. 




\bfseries Смоква \normalfont{} "--- плод фигового дерева. 




\bfseries Смотрение \normalfont{} "--- промысел; попечение; забота. 




\bfseries Смясти \normalfont{} "--- привести в смятение; встревожить. 




\bfseries Снабдевати \normalfont{} "--- сберегать; сохранять. 




\bfseries Снедати \normalfont{} "--- съедать; разорять; сокрушать. 




\bfseries Снедь \normalfont{} "--- пища. 




\bfseries Сниматися \normalfont{} "--- сходиться; собираться. 




\bfseries Снисхождение \normalfont{} "--- снисшествие. 




\bfseries Снитися \normalfont{} "--- вступить в брак; сойтись. 




\bfseries Снуждею \normalfont{} "--- поневоле; насильно; по принуждению. 




\bfseries Собесити \normalfont{} "--- вместе повесить. 




\bfseries Соблюдение \normalfont{} "--- точное исполнение; темница. 




\bfseries С соблюдением \normalfont{} "--- видимым образом; явно. 




\bfseries Совет \normalfont{} "--- совет; решение; определение. 




\bfseries Советный \normalfont{} "--- рассудительный. 




\bfseries Совлачити \normalfont{} "--- разоблачить; снять. 




\bfseries Совлечение \normalfont{} "--- раздевание. 




\bfseries Совлещися \normalfont{} "--- раздеться. 




\bfseries Совозвести \normalfont{} "--- возвести вместе с собой. 




\bfseries Совоздыхати \normalfont{} "--- печалиться вместе. 




\bfseries Совопрошатися \normalfont{} "--- беседовать; состязаться в споре. 




\bfseries Совоспитанный \normalfont{} "--- воспитанный совместно с кем-либо. 




\bfseries Согласно \normalfont{} "--- единодушно. 




\bfseries Соглядатай \normalfont{} "--- разведчик; шпион. 




\bfseries Соглядати \normalfont{} "--- рассматривать; наблюдать; разведывать. 




\bfseries Сограждати \normalfont{} "--- сооружать; строить. 




\bfseries Содеватися \normalfont{} "--- сделаться. 




\bfseries Соделование \normalfont{} "--- дело; превращение. 




\bfseries Содетель \normalfont{} "--- творец. 




\bfseries Содетельный \normalfont{} "--- творческий. 




\bfseries Сокровище \normalfont{} "--- потаенное место; задняя комната; хранилище; клад; драгоценность; погреб. 




\bfseries Сокровиществовати \normalfont{} "--- собирать сокровища. 




\bfseries Сокрушение \normalfont{} "--- уничтожение. 




\bfseries Сокрушение сердца \normalfont{} "--- раскаяние. 




\bfseries Солило \normalfont{} "--- солонка; чаша; блюдо. 




\bfseries Соние \normalfont{} "--- сон; сновидение. 




\bfseries Сонм \normalfont{} "--- собрание; множество. 




\bfseries Сонмище \normalfont{} "--- синагога. 




\bfseries Сопель \normalfont{} "--- свирель, дудка. 




\bfseries Сопети \normalfont{} "--- играть на дудке. 




\bfseries Сопец \normalfont{} "--- сопельщик-музыкант, играющий на сопели, флейте (при похоронах у иудеев). 




\bfseries Сопретися \normalfont{} "--- ссориться; тягаться. 




\bfseries Соприбывати \normalfont{} "--- увеличиваться. 




\bfseries Соприсносущный \normalfont{} "--- совместно существующий в вечности. 




\bfseries Сопрягати \normalfont{} "--- соединять браком. 




\bfseries Сорокоустие \normalfont{} "--- пшеница, вино, фимиам, свечи и пр., приносимые в церковь на 40 дней поминовения усопших христиан. 




\bfseries Соскание \normalfont{} "--- шнурок; веревочка. 




\bfseries Сосканый \normalfont{} "--- витой; крученый 




\bfseries Соскутовати \normalfont{} "--- спеленать; окутать. 




\bfseries Состреляти \normalfont{} "--- поразить стрелой. 




\bfseries Сосудохранительница \normalfont{} "--- помещение для сохранения церковной утвари. 




\bfseries Сосуды смертные \normalfont{} "--- орудия смерти. 




\bfseries Сосцы \normalfont{} "--- иногда иносказательно так называются водные источники. 




\bfseries Сотница \normalfont{} "--- сотня; пение "Господи, помилуй" сто раз при воздвизании Честнаго Креста Господня. 




\bfseries Сотово тело \normalfont{} "--- мед. 




\bfseries Соуз \normalfont{} "--- союз; связь. 




\bfseries Сочетаватися \normalfont{} "--- вступать в союз, в брак. 




\bfseries Сочиво \normalfont{} "--- чечевица; вареная пшеница с медом. 




\bfseries Сочинение \normalfont{} "--- составление; собрание. 




\bfseries Спекулатор \normalfont{} "--- телохранитель. 




\bfseries Спира \normalfont{} "--- отряд; рота; полк. 




\bfseries Сплавати \normalfont{} "--- сопутствовать в плавании. 




\bfseries Споболети \normalfont{} "--- вместе печалиться; тужить. 




\bfseries Споборать \normalfont{} "--- вместе воевать. 




\bfseries Спод \normalfont{} "--- ряд; куча; отдел. 




\bfseries Спона \normalfont{} "--- препятствие. 




\bfseries Спослушествовати \normalfont{} "--- свидетельствовать; подтверждать. 




\bfseries Споспешник \normalfont{} "--- помощник. 




\bfseries Спостник \normalfont{} "--- вместе постящийся. 




\bfseries Спострадати \normalfont{} "--- вместе страдать. 




\bfseries Споющий \normalfont{} "--- вместе или одновременно поющий. 




\bfseries Спротяженный \normalfont{} "--- продолжительный. 




\bfseries Спуд \normalfont{} "--- сосуд; ведерко; мера сыпучих тел; покрышка; плита. 




\bfseries Спяти \normalfont{} "--- низвергнуть; опрокинуть 




\bfseries Срамословие \normalfont{} "--- сквернословие. 




\bfseries Срасленный \normalfont{} "--- сросшийся. 




\bfseries Срачица \normalfont{} "--- сорочка; рубаха. 




\bfseries Сребреник \normalfont{} "--- серебряная монета. 




\bfseries Сребропозлащенный \normalfont{} "--- позолоченный по серебру. 




\bfseries Средоградие \normalfont{} "--- перегородка; простенок; преграда. 




\bfseries Средостение \normalfont{} "--- перегородка; средняя стена. 




\bfseries Сретение \normalfont{} "--- встреча. 




\bfseries Сристатися \normalfont{} "--- стекаться; сбегаться. 




\bfseries Срящь \normalfont{} "--- неприятная встреча; нападение; зараза; мор; гаданье; приметы. 




\bfseries Ставленник \normalfont{} "--- человек, подготовляемый к посвящению в духовный сан. 




\bfseries Стадия \normalfont{} "--- мера длины, равная 100-125 шагам. 




\bfseries Стаинник \normalfont{} "--- сопричастный с кем-либо одной тайне. 




\bfseries Стакти \normalfont{} "--- благовонный сок. 




\bfseries Стамна \normalfont{} "--- сосуд; ведерко; кувшин. 




\bfseries Старей \normalfont{} "--- начальник; старший. 




\bfseries Статир \normalfont{} "--- серебряная или золотая монета. 




\bfseries Статия \normalfont{} "--- глава; подраздел. 




\bfseries Стегно \normalfont{} "--- верхняя половина ноги; бедро; ляжка. 




\bfseries Стезя \normalfont{} "--- тропинка; дорожка. 




\bfseries Стень, сень \normalfont{} "--- тень; отражение; образ. 




\bfseries Степени \normalfont{} "--- ступени. 




\bfseries Стерти \normalfont{} "--- стереть; разрушить. 




\bfseries Стихира \normalfont{} "--- песнопение. 




\bfseries Стих началу \normalfont{} "--- первый возглас священника при богослужении общественном или частном. 




\bfseries Стихологисати \normalfont{} "--- петь избранные стихи из Псалтири при богослужении. 




\bfseries Стихология \normalfont{} "--- чтение или пение Псалтири. 




\bfseries Стихословити \normalfont{} "--- петь избранные стихи из Псалтири при богослужении. 




\bfseries Стицатися \normalfont{} "--- стекаться; сходиться. 




\bfseries Сткляница \normalfont{} "--- стакан. 




\bfseries Сткляный \normalfont{} "--- стеклянный. 




\bfseries Столп \normalfont{} "--- башня; крепость. 




\bfseries Столпостена \normalfont{} "--- башня; крепость. 




\bfseries Стомах \normalfont{} "--- желудок. 




\bfseries Стопа \normalfont{} "--- ступня. 




\bfseries Сторицею \normalfont{} "--- во сто раз. 




\bfseries Стогна \normalfont{} "--- улица, дорога. 




\bfseries Страдальчество \normalfont{} "--- мученичество. 




\bfseries Стража \normalfont{} "--- караул; охрана; мера времени для ночи. 




\bfseries Страннолепный \normalfont{} "--- необычный. 




\bfseries Странный \normalfont{} "--- сторонний; чужой; прохожий; необычайный. 




\bfseries Странь \normalfont{} "--- напротив; против. 




\bfseries Страсть \normalfont{} "--- страдание; страсть; душевный порыв. 




\bfseries Стратиг \normalfont{} "--- военачальник. 




\bfseries Стратилат \normalfont{} "--- военачальник; воевода. 




\bfseries Страхование \normalfont{} "--- угроза; страх; ужас. 




\bfseries Стрекало \normalfont{} "--- спица; палочка с колючкой для управления скотом. 




\bfseries Стрещи \normalfont{} "--- стеречь. 




\bfseries Стрищи \normalfont{} "--- стричь; подстригать. 




\bfseries Стропотный \normalfont{} "--- кривой; извилистый; строптивый; упрямый; злой. 




\bfseries Стрыти \normalfont{} "--- стереть; сокрушить. 




\bfseries Студ \normalfont{} "--- стыд; срам. 




\bfseries Студенец \normalfont{} "--- колодец; родник; источник. 




\bfseries Студень \normalfont{} "--- холод; стужа; мороз. 




\bfseries Стужаемый \normalfont{} "--- беспокоимый. 




\bfseries Стужание \normalfont{} "--- стеснение; гонение; досаждение. 




\bfseries Стужати \normalfont{} "--- докучать; надоедать; теснить. 




\bfseries Стужатися \normalfont{} "--- скорбеть; печалиться. 




\bfseries Стужный \normalfont{} "--- тревожный. 




\bfseries Стягнути \normalfont{} "--- обвязать; собрать; исцелить. 




\bfseries Стязатися \normalfont{} "--- спорить; препираться. 




\bfseries Сугубый \normalfont{} "--- двойной; удвоенный; увеличенный; усиленный. 




\bfseries Сударь \normalfont{} "--- плат; пелена. 




\bfseries Судище \normalfont{} "--- суд; приговор. 




\bfseries Суемудренный \normalfont{} "--- софистический; пустословный. 




\bfseries Суеслов \normalfont{} "--- пустослов. 




\bfseries Суета \normalfont{} "--- пустота; ничтожность; мелочность; бессмысленность. 




\bfseries Суетие \normalfont{} "--- суетность; суета. 




\bfseries Сулица \normalfont{} "--- копье; кинжал; кортик. 




\bfseries Супостат \normalfont{} "--- противник; враг. 




\bfseries Супротивный \normalfont{} "--- противник. 




\bfseries Супруг \normalfont{} "--- пара; чета. 




\bfseries Супря \normalfont{} "--- спор; тяжба. 




\bfseries Суровый \normalfont{} "--- зеленый; свежий; сырой. 




\bfseries Сходник \normalfont{} "--- лазутчик; разведчик; шпион. 




\bfseries Счиневати \normalfont{} "--- соединять. 




\bfseries Сыноположение \normalfont{} "--- усыновление. 




 





\bfseries Таже \normalfont{} "--- потом; затем. 




\bfseries Тай \normalfont{} "--- тайно; скрытно. 




\bfseries Таинник \normalfont{} "--- посвященный в чьи-либо тайны. 




\bfseries Тайноядение \normalfont{} "--- тайное невоздержание от пищи в пост. 




\bfseries Таити \normalfont{} "--- скрывать. 




\bfseries Тако \normalfont{} "--- так. 




\bfseries Такожде \normalfont{} "--- равно; также. 




\bfseries Талант \normalfont{} "--- древнегреч. мера веса и монета. 




\bfseries Тамо \normalfont{} "--- там; туда. 




\bfseries Тартар \normalfont{} "--- место нахождения душ умерших до освобождения их Господом Иисусом Христом; место вечного мучения грешников; жилище диавола. 




\bfseries Татаур \normalfont{} "--- ремень для привешивания языка к колоколу; кожаный ремень, носимый монашествующими . 




\bfseries Тать \normalfont{} "--- вор. 




\bfseries Татьба \normalfont{} "--- кража; воровство. 




\bfseries Тафта \normalfont{} "--- тонкая шелковая материя. 




\bfseries Тацы \normalfont{} "--- таковы. 




\bfseries Таче \normalfont{} "--- для того; также; тогда. 




\bfseries Тварь \normalfont{} "--- творение; создание; произведение. 




\bfseries Твердыня \normalfont{} "--- крепость; цитадель; тюрьма. 




\bfseries Твердь \normalfont{} "--- основание; видимый небосклон, принимаемый глазом за твердую сферу, купол небес. 




\bfseries Твержа \normalfont{} "--- крепость; цитадель; тюрьма. 




\bfseries Тезоименитство \normalfont{} "--- одноименность; именины; день Ангела. 




\bfseries Тектон \normalfont{} "--- плотник; столяр. 




\bfseries Телец \normalfont{} "--- теленок; бычок. 




\bfseries Темже, темже убо \normalfont{} "--- поэтому; следовательно; итак. 




\bfseries Темник \normalfont{} "--- начальник над десятью тысячами человек. 




\bfseries Темнозрачный \normalfont{} "--- черный. 




\bfseries Темнонеистовство \normalfont{} "--- мракобесие; непросвещенность. 




\bfseries Теревинф \normalfont{} "--- дубрава; чаща; лес; большое ветвистое дерево с густой листвой. 




\bfseries Терние \normalfont{} "--- терновник; колючее растение. 




\bfseries Терноносный \normalfont{} "--- плодоносящий терние; иносказательно: не имеющий добрых дел. 




\bfseries Терпкий \normalfont{} "--- кислый; вяжущий; суровый. 




\bfseries Теснина \normalfont{} "--- узкий проход. 




\bfseries Теснота \normalfont{} "--- беда; напасть. 




\bfseries Тетива \normalfont{} "--- туго натянутая веревка. 




\bfseries Тещи \normalfont{} "--- быстро идти. 




\bfseries Тещити \normalfont{} "--- источать; испускать. 




\bfseries Тимение \normalfont{} "--- болото; топь; тина. 




\bfseries Тимпан \normalfont{} "--- литавра; бубен. 




\bfseries Тимпанница \normalfont{} "--- девушка, играющая на тимпане. 




\bfseries Тирон \normalfont{} "--- молодой воин, солдат. 




\bfseries Титло \normalfont{} "--- надпись; ярлык; знак для сокращения слова. 




\bfseries Тихообразно \normalfont{} "--- спокойно; кротко. 




\bfseries Тление, тля \normalfont{} "--- гниение; уничтожение; разрушение. 




\bfseries Тлети \normalfont{} "--- растлевать; гнить; разрушаться. 




\bfseries Тлити \normalfont{} "--- повреждать; губить. 




\bfseries Тма \normalfont{} "--- темнота; мрак; десять тысяч. 




\bfseries Тоболец \normalfont{} "--- мешок; котомка; сумка. 




\bfseries Ток \normalfont{} "--- течение. 




\bfseries Токмо \normalfont{} "--- только. 




\bfseries Толико \normalfont{} "--- столько. 




\bfseries Толк \normalfont{} "--- толкование; учение; особое мнение. 




\bfseries Толковник \normalfont{} "--- переводчик; истолкователь. 




\bfseries Толковый \normalfont{} "--- объясняющий; содержащий объяснения. 




\bfseries Толмач \normalfont{} "--- переводчик. 




\bfseries Толь \normalfont{} "--- столько. 




\bfseries Томитель \normalfont{} "--- мучитель. 




\bfseries Томити \normalfont{} "--- мучить; пытать. 




\bfseries Томление \normalfont{} "--- мучение; пытка. 




\bfseries Топазий \normalfont{} "--- топаз. 




\bfseries Торжище \normalfont{} "--- площадь; рынок. 




\bfseries Торжник \normalfont{} "--- меняла; торговец. 




\bfseries Торчаный \normalfont{} "--- растерзанный. 




\bfseries Точило \normalfont{} "--- пресс для выжимания виноградного сока. 




\bfseries Точию \normalfont{} "--- только. 




\bfseries Тощно \normalfont{} "--- усердно; точно. 




\bfseries Трапеза \normalfont{} "--- стол; кушание; столовая, трапезная; святой престол. 




\bfseries Требе \normalfont{} "--- потребно; надобно. 




\bfseries Требище \normalfont{} "--- жертвенник; языческий храм. 




\bfseries Треблаженный \normalfont{} "--- весьма блаженный. 




\bfseries Требование \normalfont{} "--- нужда; потребность. 




\bfseries Требовати \normalfont{} "--- нуждаться; иметь потребность. 




\bfseries Трегубо \normalfont{} "--- трояко; трижды. 




\bfseries Трекровник \normalfont{} "--- третий этаж. 




\bfseries Тресна \normalfont{} "--- украшение на одежде. 




\bfseries Третицею \normalfont{} "--- трижды; в третий раз. 




\bfseries Тридевять \normalfont{} "--- двадцать семь. 




\bfseries Трисиянный \normalfont{} "--- светящий от трех Светил. 




\bfseries Тристат \normalfont{} "--- военачальник. 




\bfseries Трищи \normalfont{} "--- трижды. 




\bfseries Тропарь \normalfont{} "--- краткое песнопение, выражающее характеристику праздника или события в жизни святого. 




\bfseries Трость \normalfont{} "--- тростник (использовавшийся в качестве пишущего инструмента). 




\bfseries Труд \normalfont{} "--- болезнь; недуг. 




\bfseries Труждатися \normalfont{} "--- трудиться; затрудняться. 




\bfseries Трус \normalfont{} "--- землетрясение. 




\bfseries Трыти \normalfont{} "--- тереть; омывать. 




\bfseries Ту \normalfont{} "--- тут; там; здесь. 




\bfseries Туга \normalfont{} "--- скорбь. 




\bfseries Тук \normalfont{} "--- жир; сало; богатство; пресыщение. 




\bfseries Тул \normalfont{} "--- колчан для стрел. 




\bfseries Туне \normalfont{} "--- напрасно; даром; впустую. 




\bfseries Тунегиблемый \normalfont{} "--- истрачиваемый noнапрасну. 




\bfseries Тщание \normalfont{} "--- усердие; старание. 




\bfseries Тщатися \normalfont{} "--- стараться; спешить. 




\bfseries Тщета \normalfont{} "--- урон; вред; убыток. 




\bfseries Тщий \normalfont{} "--- пустой; бесполезный; неудовлетворенный. 




\bfseries Тягота \normalfont{} "--- тяжесть; обременение. 




\bfseries Тяжание \normalfont{} "--- работа; дело; пашня; поле. 




\bfseries Тяжатель \normalfont{} "--- работник. 




\bfseries Тяжати \normalfont{} "--- работать. 




\bfseries Тяжкосердый \normalfont{} "--- бесчувственный. 




 





\bfseries У \normalfont{} "--- еще; 




\bfseries не у \normalfont{} "--- еще не. 




\bfseries Убо \normalfont{} "--- а; же; вот; хотя; почему; поистине; подлинно. 




\bfseries Убрус \normalfont{} "--- плат; полотенце. 




\bfseries Убудитися \normalfont{} "--- пробудиться; очнуться. 




\bfseries Уведети \normalfont{} "--- узнать. 




\bfseries Увет \normalfont{} "--- увещание. 




\bfseries Увязение \normalfont{} "--- возложение на голову венца. 




\bfseries Увясло \normalfont{} "--- головная повязка. 




\bfseries Углебнути \normalfont{} "--- тонуть; погружаться; погрязать. 




\bfseries Угобзити \normalfont{} "--- обогатить, одарить. 




\bfseries Угонзнути \normalfont{} "--- убежать; ускользнуть; уйти. 




\bfseries Уготоватися \normalfont{} "--- приготовиться. 




\bfseries Угрызнути \normalfont{} "--- укусить зубами. 




\bfseries Уд \normalfont{} "--- телесный член. 




\bfseries Удава \normalfont{} "--- веревка. 




\bfseries Удица \normalfont{} "--- удочка. 




\bfseries Удобие \normalfont{} "--- удобнее. 




\bfseries Удобострастие \normalfont{} "--- склонность к угождению страстям. 




\bfseries Удобрение \normalfont{} "--- украшение. 




\bfseries Удобрити \normalfont{} "--- наполнить; украсить. 




\bfseries Удобь \normalfont{} "--- легко; удобно. 




\bfseries Удовлитися \normalfont{} "--- удовлетвориться. 




\bfseries Удолие, удоль, юдоль \normalfont{} "--- долина. 




\bfseries Удручати \normalfont{} "--- утомлять; оскорблять. 




\bfseries Уже \normalfont{} "--- веревка; цепь; узы. 




\bfseries Ужик \normalfont{} "--- родственник; родственница. 




\bfseries Узилище \normalfont{} "--- тюрьма. 




\bfseries Укорение \normalfont{} "--- бесславие; бесчестие. 




\bfseries Укрой \normalfont{} "--- повязка; пелена. 




\bfseries Укроп \normalfont{} "--- теплота, т.е. горячая вода, вливаемая во святой потир на Литургии. 




\bfseries Укрух \normalfont{} "--- ломоть; кусок. 




\bfseries Улучити \normalfont{} "--- застать; найти; получить. 




\bfseries Умащати \normalfont{} "--- намазывать; натирать. 




\bfseries Умет \normalfont{} "--- помет; кал; сор. 




\bfseries Умовредие \normalfont{} "--- безумие. 




\bfseries Умучити \normalfont{} "--- укротить. 




\bfseries Уне \normalfont{} "--- лучше. 




\bfseries Унзнути \normalfont{} "--- воткнуть; вонзить. 




\bfseries Унше \normalfont{} "--- лучше; полезнее. 




\bfseries Упитанная \normalfont{} "--- откормленные животные. 




\bfseries Упование \normalfont{} "--- твердая надежда. 




\bfseries Упразднити \normalfont{} "--- уничтожить; отменить; исчезнуть. 




\bfseries Уранити \normalfont{} "--- встать рано утром. 




\bfseries Урок \normalfont{} "--- урок; подать; оброк. 




\bfseries Усекновение \normalfont{} "--- отсечение. 




\bfseries Усерязь \normalfont{} "--- серьга. 




\bfseries Усма \normalfont{} "--- выделанная кожа. 




\bfseries Усмарь \normalfont{} "--- кожевенник; скорняк. 




\bfseries Усмен \normalfont{} "--- кожаный. 




\bfseries Уста \normalfont{} "--- рот; губы; речь. 




\bfseries Устранити \normalfont{} "--- лишить; избежать; устранить. 




\bfseries Устудити \normalfont{} "--- охладить; остудить. 




\bfseries Усырити \normalfont{} "--- сделать сырым, твердым, мокрым. 




\bfseries Утварне \normalfont{} "--- по порядку; нарядно. 




\bfseries Утварь \normalfont{} "--- одежда; украшение; убранство. 




\bfseries Утверждение \normalfont{} "--- основание; опора. 




\bfseries Утешение \normalfont{} "--- угощение. 




\bfseries Утолити \normalfont{} "--- успокоить; утешить; умерить. 




\bfseries Утреневати \normalfont{} "--- рано вставать; совершать утреннюю молитву. 




\bfseries Утроба \normalfont{} "--- чрево; живот; сердце; душа. 




\bfseries Ухание \normalfont{} "--- обоняние; запах. 




\bfseries Ухлебити \normalfont{} "--- накормить. 




\bfseries Учреждати \normalfont{} "--- угостить. 




\bfseries Учреждение \normalfont{} "--- пир; обед; угощение. 




\bfseries Ушеса \normalfont{} "--- уши. 




\bfseries Ущедрити \normalfont{} "--- обогатить; помиловать; пожалеть. 




 





\bfseries Факуд \normalfont{} "--- евр. начальник. 




\bfseries Фарос \normalfont{} "--- маяк. 




\bfseries Фаска \normalfont{} "--- Пасха. 




\bfseries Февруарий \normalfont{} "--- февраль. 




\bfseries Фелонь \normalfont{} "--- плащ; верхняя одежда; одно из священных облачений пресвитера. 




\bfseries Фиала \normalfont{} "--- чаша; бокал с широким дном. 




\bfseries Фимиам \normalfont{} "--- благовонная смола для воскурения при каждении. 




\bfseries Финикс \normalfont{} "--- пальма. 




 





\bfseries Халван \normalfont{} "--- благовонная смола. 




\bfseries Халколиван \normalfont{} "--- ливанская медь; янтарь. 




\bfseries Халуга \normalfont{} "--- плетень; забор; закоулок. 




\bfseries Харатейный \normalfont{} "--- написанный на прегаменте или папирусной бумаге. 




\bfseries Хартия \normalfont{} "--- пергамент или папирусная бумага; рукописный список. 




\bfseries Харя \normalfont{} "--- маска; личина. 




\bfseries Хврастие \normalfont{} "--- хворост. 




\bfseries Херет \normalfont{} "--- училище, темница. 




\bfseries Хитон \normalfont{} "--- нижняя одежда; рубашка. 




\bfseries Хитрец \normalfont{} "--- художник; ремесленник. 




\bfseries Хитрогласница \normalfont{} "--- риторика. 




\bfseries Хитрословесие \normalfont{} "--- риторика. 




\bfseries Хитрость \normalfont{} "--- художество; ремесло. 




\bfseries Хитрость, ухищренная вымыслом \normalfont{} "--- стенобитные и метательные военные машины. 




\bfseries Хищноблудие \normalfont{} "--- насильственное привлечение к блуду. 




\bfseries Хламида \normalfont{} "--- верхнее мужское платье; плащ; мантия. 




\bfseries Хлептати \normalfont{} "--- лакать. 




\bfseries Хлябь \normalfont{} "--- водопад; пропасть; бездна; простор; подъемная дверь. 




\bfseries Ходатай \normalfont{} "--- посредник, примиритель. 




\bfseries Хоругвь \normalfont{} "--- военное знамя. 




\bfseries Хотение \normalfont{} "--- воля. 




\bfseries Храмляти \normalfont{} "--- хромать. 




\bfseries Храм набдящий \normalfont{} "--- казнохранилище. 




\bfseries Храм, храмина \normalfont{} "--- дом; помещение; место богослужения. 




\bfseries Хранилище \normalfont{} "--- повязка на лбу или на руках со словами Закона. 




\bfseries Худогий \normalfont{} "--- искусный; умелый. 




\bfseries Худогласие \normalfont{} "--- косноязычие; заикание. 




\bfseries Художество \normalfont{} "--- наука; причуда; выкрутаса. 




\bfseries Худость \normalfont{} "--- скудость; недостоинство. 




\bfseries Хула \normalfont{} "--- злословие; нарекание. 




 





\bfseries Цветник \normalfont{} "--- луг. 




\bfseries Цевница \normalfont{} "--- флейта; свирель. 




\bfseries Целование \normalfont{} "--- приветствие. 




\bfseries Целовати \normalfont{} "--- приветствовать. 




\bfseries Целомудрие \normalfont{} "--- благоразумие, непорочность и чистота телесная. 




\bfseries Целый \normalfont{} "--- здоровый, невредимый. 




\bfseries Цельбоносный \normalfont{} "--- врачебный; целительный. 




 





\bfseries Чадо \normalfont{} "--- дитя. 




\bfseries Чадородие \normalfont{} "--- рождение детей. 




\bfseries Чадце \normalfont{} "--- деточка. 




\bfseries Чарование \normalfont{} "--- яд; отрава; волхвование; заговаривание. 




\bfseries Чаровник \normalfont{} "--- отравитель; волхв; колдун. 




\bfseries Чары \normalfont{} "--- волшебство; колдовство; заговаривание. 




\bfseries Часть \normalfont{} "--- часть; жребий; участь. 




\bfseries Чаяти \normalfont{} "--- надеяться; ждать. 




\bfseries Чван \normalfont{} "--- сосуд; штоф; кружка; фляжка. 




\bfseries Чванец \normalfont{} "--- сосуд; штоф; кружка; фляжка. 




\bfseries Чело \normalfont{} "--- лоб. 




\bfseries Челядь \normalfont{} "--- слуги; домочадцы. 




\bfseries Чепь \normalfont{} "--- цепь. 




\bfseries Червленица \normalfont{} "--- ткань темно-красного цвета; порфира, пурпурная одежда высокопоставленных особ. 




\bfseries Червленый \normalfont{} "--- багряный. 




\bfseries Чермноватися \normalfont{} "--- краснеть. 




\bfseries Чермный \normalfont{} "--- красный. 




\bfseries Чернец \normalfont{} "--- монах. 




\bfseries Черничие \normalfont{} "--- лесная смоква. 




\bfseries Чертог \normalfont{} "--- палата; покои. 




\bfseries Чесати \normalfont{} "--- собирать плоды. 




\bfseries Чесо \normalfont{} "--- чего; что. 




\bfseries Чести \normalfont{} "--- читать. 




\bfseries Честный \normalfont{} "--- уважаемый; прославляемый. 




\bfseries Четверовластник \normalfont{} "--- управляющий четвертою частью страны. 




\bfseries Чин \normalfont{} "--- порядок; полное изложение или указание всех молитвословий. 




\bfseries Чинити \normalfont{} "--- составлять; делать. 




\bfseries Чревобесие \normalfont{} "--- объядение; обжорство. 




\bfseries Чревоношение \normalfont{} "--- зачатие и ношение в утробе младенца. 




\bfseries Чреда \normalfont{} "--- порядок; очередь; черед. 




\bfseries Чреждение \normalfont{} "--- угощение. 




\bfseries Чресла \normalfont{} "--- поясница; бедра; пах. 




\bfseries Чтилище \normalfont{} "--- идол; кумир. 




\bfseries Чудитися \normalfont{} "--- удивляться. 




\bfseries Чудотворити \normalfont{} "--- творить чудеса. 




\bfseries Чути, чуяти \normalfont{} "--- чувствовать; слышать; ощущать. 




 





\bfseries Шелом \normalfont{} "--- шлем; каска. 




\bfseries Шепотник \normalfont{} "--- наушник; клеветник. 




\bfseries Шептание \normalfont{} "--- клевета. 




\bfseries Шипок \normalfont{} "--- цветок шиповника. 




\bfseries Шуий \normalfont{} "--- левый. 




\bfseries Шуйца \normalfont{} "--- левая рука. 




 





\bfseries Щедрота \normalfont{} "--- милость; великодушие; снисхождение. 




\bfseries Щедрый \normalfont{} "--- милостивый. 




\bfseries Щогла \normalfont{} "--- мачта; веха; жердь. 




 





\bfseries Ю \normalfont{} "--- ее. 




\bfseries Юг \normalfont{} "--- зной; название южного ветра; иносказательно; несчастие. 




\bfseries Юдоль \normalfont{} "--- долина. 




\bfseries Юдоль плачевная \normalfont{} "--- мир сей. 




\bfseries Юдуже \normalfont{} "--- там, где. 




\bfseries Южик (а) \normalfont{} "--- родственник; родственница. 




\bfseries Юзник \normalfont{} "--- узник; заключенный. 




\bfseries Юзы \normalfont{} "--- кандалы; узы. 




\bfseries Юнец \normalfont{} "--- телок; молодой бычок. 




\bfseries Юница \normalfont{} "--- телка; молодая корова. 




\bfseries Юнота \normalfont{} "--- молодой человек. 




\bfseries Юродивый \normalfont{} "--- глупый; принявший духовный подвиг юродства. 




 





\bfseries Ягодичина \normalfont{} "--- фиговое дерево. 




\bfseries Ядрило \normalfont{} "--- мачта. 




\bfseries Ядца \normalfont{} "--- лакомка; гурман; обжора. 




\bfseries Ядь \normalfont{} "--- пища; еда. 




\bfseries Язвина \normalfont{} "--- нора. 




\bfseries Язвити \normalfont{} "--- жечь; ранить. 




\bfseries Язык \normalfont{} "--- народ; племя; орган речи. 




\bfseries Язя, язва \normalfont{} "--- рана; ожог. 




\bfseries Яко \normalfont{} "--- ибо; как; так как; потому что; когда. 




\bfseries Яковый \normalfont{} "--- каковой. 




\bfseries Якоже \normalfont{} "--- так чтобы; как; так как. 




\bfseries Яннуарий \normalfont{} "--- январь. 




\bfseries Ярем \normalfont{} "--- ярмо; груз; тяжесть; служение. 




\bfseries Ярина \normalfont{} "--- волна; шерсть. 




\bfseries Ясти \normalfont{} "--- есть; кушать. 




\bfseries Яти \normalfont{} "--- брать. 




 

