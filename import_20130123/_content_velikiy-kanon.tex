

\mypart{ВЕЛИКИЙ КАНОН (ТВОРЕНИЕ СВЯТОГО АНДРЕЯ КРИТСКОГО)}
%http://www.molitvoslov.org/content/velikiy-kanon

\bfseries Смотреть весь раздел &rarr;\normalfont{} 

\mychapter{В начале Великопостного пути — Великий Канон св. Андрея Критского}
%http://www.molitvoslov.org/content/v-nachale-velikopostnogo-puti 
 
\myfig{img/300px-Andreas_cretensis_0_1.jpg}

Первая седмица Великого Поста с древних времен
называется «зарей воздержания» и «неделей
чистой». В эту неделю Церковь убеждает своих чад
выйти из того греховного состояния, в которое
невоздержанием наших прародителей ниспал весь
род человеческий, утратив райское блаженство, и
которое каждый из нас умножает сам своими
грехами, "--- выйти путем веры, молитвы, смирения и
Богоугодного поста. \itshape Се время покаяния, говорит
Церковь, се день спасительный, пощения вход: душе,
бодрствуй, и страстей входа затвори, ко Господу
взирающе\normalfont{}  (из первой песни трипеснца на
утрени понедельника первой седмицы Великого
Поста).


Подобно ветхозаветной церкви, которая особенно
святила первый и последний день некоторых
великих праздников, православные христиане,
приготовленные и воодушевленные матерними
внушениями своей Церкви, издревле, согласно ее
уставу, с особенным усердием и строгостью
проводят первую и последнюю седмицу Великого
Поста.


На первой седмице совершаются особенно
продолжительные богослужения и подвиг телесного
воздержания значительно более строгий, чем в
последующие дни Святой Четыредесятницы. В первые
четыре дня Великого Поста великое повечерие
совершается с чтением Великого Покаянного
Канона преп. Андрея Критского, который как бы
задает «тон», определяет всю последующую
тональность, «мелодию» Великого Поста. На первой
седмице Поста Канон делится на четыре части.
Дивное творение св. Андрея Критского полностью
предлагается нашему вниманию и в четверг (точнее
в среду вечером) пятой седмицы Св.
Четыредесятницы для того, чтобы мы, видя
приближающееся окончание Поста, не разленились к
духовным подвигам, не сделались небрежными, не
забылись и не перестали во всем строго следить за
собой.


Каждый стих Великого Канона сопровождается
псаломским припевом \itshape Помилуй мя, Боже, помилуй
мя!\normalfont{} К канону присоединяется несколько тропарей
в честь самого автора "--- св. Андрея и преп. Марии
Египетской. Еще при жизни св. Андрея
Иерусалимская Церковь ввела у себя в
употребление Великий Канон. Отправляясь в 680 году
на Шестой Вселенский Собор в Константинополь, св.
Андрей принес туда и сделал известным свое
великое творение и житие преп. Марии Египетской,
написанное его соотечественником и учителем
патриархом Иерусалимским Софронием. Житие
египетской подвижницы читается совместно с
Великим Каноном на утрени в среду пятой седмицы
Великого Поста.


Из всех молитвословий Великого Поста, больше
всех других поражает душу Великий Канон. Великий
Канон "--- это чудо церковной гимнографии, это
тексты удивительной силы и поэтической красоты.
Канон составлен в 7-м веке св. Андреем,
Архиепископом Критским, составившим также много
других канонов, которые Церковь использует в
течение всего богослужебного года. Церковь
именовала этот канон великим, не столько из-за
его объема (в нем 250 тропарей или стихов), сколько
по его внутреннему достоинству и силе. 


Великий канон представляет собой беседу
кающегося с собственной душой. Вот как он
начинается:


\itshape Откуда начну плакати окаянного моего жития
деяний? Кое ли положу начало, Христе, нынешнему
рыданию? Но яко благоутробен даждь ми
прегрешений оставление \normalfont{}"--- с чего же начать мне
каяться, ведь это так трудно. 


Затем следует чудный тропарь:


\itshape Гряди, окаянная душе, с плотию твоею. Зиждителю
всех исповеждься и останися прочее преждняго
безсловесия, и принеси Богу в покаянии слезы. \normalfont{}


Удивительные слова, тут и христианская
антропология, и аскетика: плоть тоже должна
участвовать в покаянии, как неотъемлемая часть
человеческого естества. 


Своего апогея эта беседа с душой, постоянные
уговоры ее, призывы покаяться, достигают в
кондаке, который поется после 6-й песни Канона: 


\itshape 
Душе моя, душе моя, востани, что спиши? Конец
приближается, и имаши смутитися; воспряни убо, да
пощадит тя Христос Бог, везде сый, и вся
исполняяй.\normalfont{}


Эти слова произносит, обращаясь к себе, великий
светильник Церкви, тот, к кому самому применимо
было бы выражение, употребленное им относительно
преп. Марии Египетской, которая действительно
была «ангел во плоти». И вот он так к себе
обратился, упрекая себя за то, что душа его спит.
Если он видел себя таким, то какими должны видеть
себя мы? Погруженные уже не только в беспробудный
духовный сон, но в какое-то омертвение…


Когда мы внимаем словам кондака из канона
святителя Андрея Критского, нужно спросить себя:
что мне делать? Если бы человек как должно
исполнял Божий закон, его жизнь была бы заполнена
совсем другим содержанием. Вот для того Церковь и
предлагает нам этот глубокий, проникновенный
великопостный покаянный канон, чтобы мы
заглянули поглубже в свою душу и посмотрели бы,
что там. А душа-то спит… В этом и горе наше и беда
наша. 


В замечательной молитве преп. Ефрема Сирина,
которую мы повторяем в продолжение всего
Великого Поста, говорится: \itshape Господи Царю, даруй
ми зрети моя прегрешения!\normalfont{} "--- Я их не вижу, душа
моя заснула, задремала и я этих грехов, как
должно, даже и не вижу. Как же я буду каяться в них!
И вот потому нужно в дни Великого Поста побольше
сосредотачиваться на себе, оценивая свою жизнь и
ее содержание евангельской мерой, а не
какой-нибудь другой. 


К основным особенностям Великого канона
относится очень широкое использование образов и
сюжетов из Священного Писания, как Ветхого, так и
Нового Завета. Жаль, что мы плохо знакомы со
Святой Библией. Многим из нас имена людей,
упомянутых в Великом Каноне, ничего не говорят,
потому что мы плохо знаем Библию. 


А между тем Библия "--- это не только история
израильского народа, но и грандиозная летопись
души человеческой "--- души, которая падала и
вставала перед лицом Бога, которая согрешала и
каялась. Если мы посмотрим на жизнь людей, о
которых говорится в Библии, то увидим, что каждый
из них представлен не столько как исторический
персонаж, не столько как личность, совершившая те
или иные дела, сколько как человек, предстоящий
перед лицом Живого Бога. Исторические и иные
заслуги человека отходят на второй план,
остается то, что всего важнее: сохранил человек
верность Богу или нет. Если мы будем читать
Библию и Великий Канон под таким углом зрения, то
увидим, что многое из того, что говорится о
древних праведниках и грешниках, является не чем
иным, как летописью нашей души, наших падений и
восстаний, наших грехов и покаяния.


Один церковный писатель по этому поводу очень
кстати замечает: «Если в наши дни столь многие
находят его (Великий Канон) скучным и не
относящимся к нашей жизни, это происходит оттого,
что вера их не питается из источника Священного
Писания, которое для Отцов Церкви было \itshape именно\normalfont{}
источником их веры. Мы должны вновь научиться
воспринимать мир таким, каким он открывается нам
в Библии, научиться жить в этом библейском мире; и
нет лучшего способа научиться этому, как именно
через церковное богослужение, которое не только
передает нам библейское учение, но и открывает
нам библейский образ жизни» (Протопр. Александр
Шмеман, «Великий Пост», стр. 97).


Итак, в Великом Каноне перед нами проходит в
лицах и событиях вся ветхозаветная и
новозаветная история. Автор указывает на
грехопадение прародителей и растление
первобытного мира, на добродетели Ноя и
нераскаянность и ожесточение жителей Содома и
Гоморры, воскрешает перед нами память
благочестивых патриархов и доблестных мужей:
Моисея, Иисуса Навина, Гедеона и Иефая,
представляет нашему взору благочестие царя
Давида, его падение и умиленное покаяние,
указывает на нечестие Ахава и Иезавели и на
великие образцы покаяния "--- неневитян, Манассию,
блудницу и благоразумного разбойника, и в
особенности на Марию Египетскую, неоднократно
останавливает читателя у Креста и Гроба Господня
"--- везде поучая покаянию, смирению, молитве,
самоотвержению. На этих примерах постоянно
происходит увещание души "--- вспомни этого
праведника, он так угодил Богу, вспомни и этого
праведника, он так угодил "--- ты ничего подобного
не сделала. 


Об одних перснонажах Библии говорится в
положительном смысле, о других в отрицательном,
кому-то нужно подражать, а кому-то нет. 


\itshape Колесничник Илия, колесницею добродетелей
вшед, яко на Небеса, ношашеся превыше иногда от
земных. Сего убо, душе моя, восход помышляй\normalfont{} "---
помышляй, душа моя, о восхождении ветхозаветных
праведников. 


\itshape Гиезиев подражала еси, окаянная, разум скверный
всегда, душе, егоже сребролюбие отложи поне на
старость, бегай геенскаго огня, отступивши злых
твоих\normalfont{} "--- хотя бы в старости отринь сребролюбие
Гиезии, душа, и оставив свои злодеяния, избегай
геенского огня. 


Как видите, тексты довольно трудные, поэтому к
восприятию Великого канона необходимо заранее
готовиться. 


В заключительной песни первого дня после всех
воспоминаний следуют тропари удивительной силы: 


\itshape Закон изнеможе, празднует Евангелие, писание же
все в тебе небрежено бысть, пророцы изнемогоша, и
все праведное слово: струпи твои, о душе,
умножишася, не сущу врачу исцеляющему тя\normalfont{} "---
нечего вспоминать из Ветхого Завета, все
бесполезно. Буду приводить тебе примеры из
Нового Завета, может быть, тогда ты покаешься: 


\itshape Новаго привожду ти писания указания, вводящая
тя, душе, ко умилению: праведным убо поревнуй,
грешных же отвращайся, и умилостиви Христа
молитвами же и пощеньми и чистотою и говением. \normalfont{}


Наконец, духовный писатель, представив все
ветхозаветное, восходит к Жизнодавцу, Спасителю
душ наших, восклицая, как разбойник: \itshape Помяни мя!\normalfont{},
взывая, как мытарь: \itshape Боже милостив буди мне
грешному!\normalfont{}, подражая в неотступности Хананеянке
и слепцам на распутии: \itshape Помилуй мя, сыне Давидов!\normalfont{},
источая слезы, вместо мира, на главу и ноги
Христа, подобно блуднице, и горько плача над
собою, как Марфа и Мария над Лазарем. 


Далее в Каноне подчеркивается, что самые
страшные грешники покаялись и придут в Царство
Небесное прежде нас: \itshape Христос вочеловечися,
призвав к покаянию разбойники и блудницы: душе,
покайся, дверь отверзеся Царствия уже, и
предвосхищают о фарисее и мытарии и прелюбодеи
кающиися.\normalfont{} 


Когда же, в некоем духовном ужасе, следуя издали
за чудесами Спасителя и умиляясь над каждым
подвигом Его земной жизни, автор Канона доходит
до страшного заколения Христа, "--- силы сердца его
оскудевает и, вместе со всей тварью, он умолкает
на трепещущей Голгофе, в последний раз
воскликнув: \itshape Судие мой и ведче мой, хотяй паки
приити со ангелы, судити миру всему, милостивным
твоим оком тогда, видев мя пощади и ущедри мя,
Иисусе, паче всякого естества человеча
согрешивша.\normalfont{}


Великий канон, всеми средствами подвигая нас к
покаянию, в последних тропарях как бы открывает
нам свою «методику»: как я с тобой беседовал,
душа, и праведников ветхозаветных тебе
напоминал, и новозаветные образы тебе в пример
приводил, и все напрасно: \itshape ихже не поревновала
еси, душе, ни деянием, ни житию: но горе тебе,
внегда будеши судитися\normalfont{} "--- горе тебе, когда
предстанешь на суд! 


Вслушиваясь в слова Великого Канона,
всматриваясь в историю жизни людей, бежавших от
Бога, но настигнутых Им, людей, которые
оказывались в безднах, но которых Бог выводил
оттуда, подумаем о том, как Бог выводит каждого из
нас из бездны греха и отчаяния для того, чтобы мы
принесли Ему плоды покаяния.


Не нужно думать, что покаяние заключается в том,
чтобы копаться в собственных грехах, заниматься
самобичеванием, стараться выявить в себе как
можно больше злого и темного. Истинное покаяние
"--- это когда мы обращаемся от тьмы к свету, от
греха к праведности; когда мы понимаем, что
прежняя наша жизнь была недостойна высокого
призвания, когда перед лицом Бога мы сознаем, как
ничтожны мы сами, и сознаем, что единственная
наша надежда "--- Сам Бог. Истинное покаяние "--- это
когда перед лицом Бога, по слову Апостола Петра
«призвавшего нас из тьмы в чудный Свой свет» (1
Пет. 2: 9), мы понимаем, что жизнь дана нам для того,
чтобы стать детьми Божиими, чтобы приобщиться к
Божественному свету. Истинное покаяние "--- то,
которое выражается не столько в словах, но и в
делах, в готовности прийти на помощь людям, в
открытости по отношению к ближним, а не в
обращенности на себя. Истинное покаяние "--- это
когда мы понимаем, что, хотя и не в наших силах
стать настоящими христианами, Бог в силах
сделать нас таковыми. Как говорится в Великом
Каноне, \itshape идеже бо хощет Бог, побеждается
естества чин\normalfont{}. Другими словами там, где Бог
хочет, происходят сверхъестественные события:
Савл становится Павлом, Иона изводится из чрева
кита, Моисей проходит через море по суше, умерший
Лазарь воскресает, Мария Египетская из блудницы
превращается в великую праведницу. Ибо, по словам
Спасителя, «человекам это невозможно, Богу же все
возможно» (Мф. 19: 26). 


\itshape ©Протоиерей Виктор Потапов\normalfont{}


\itshape  февраль 2001 г.\normalfont{}
http://www.stjohndc.org/Russian/feasts/fasts/grlent/r_post_akritsk.htm


\medskip

\medskip

\medskip

\mychapterending

\mychapter{Весь Канон целиком на церковнославянском языке (pdf)}
%http://www.molitvoslov.org/content/Ves-Kanon-tselikom-na-tserkovnoslayavyanskom-yazyke-pdf 
 
\myfig{img/300px-Andreas_cretensis_0.jpg}

\mychapterending

\mychapter{В понедельник первой седмицы Великого поста}
%http://www.molitvoslov.org/text571.htm 
 






\bfseries Песнь 1\normalfont{}


\itshape Ирмо́с:\normalfont{}


Помощник и Покровитель бысть мне во спасение, Сей мой Бог, и прославлю Его, Бог Отца моего, и вознесу Его: славно бо прославися.


\itshape Помощник и Покровитель явился мне ко спасению, Он Бог мой, и прославлю Его, Бога отца моего, и превознесу Его, ибо Он торжественно прославился. Исх. 15:1–2\normalfont{}


\itshape Припев:\normalfont{} Помилуй мя, Боже, помилуй мя.


Откуду начну плакати окаяннаго моего жития деяний? кое ли положу начало, Христе, нынешнему рыданию? но, яко благоутробен, даждь ми прегрешений оставление.


\itshape С чего начну я оплакивать деяния злосчастной моей жизни? Какое начало положу, Христе, я нынешнему моему сетованию? Но Ты, как милосердный, даруй мне оставление прегрешений.\normalfont{}


\itshape Припев:\normalfont{} Помилуй мя, Боже, помилуй мя.


Гряди, окаянная душе, с плотию твоею, Зиждителю всех исповеждься и останися прочее преждняго безсловесия, и принеси Богу в покаянии слезы.


\itshape Прииди, несчастная душа, с плотию своею, исповедайся Создателю всего, воздержись, наконец, от прежнего безрассудства и с раскаянием принеси Богу слезы.\normalfont{}


\itshape Припев:\normalfont{} Помилуй мя, Боже, помилуй мя.


Первозданнаго Адама преступлению поревновав, познах себе обнажена от Бога и присносущнаго Царствия и сладости, грех ради моих.


\itshape Подражая в преступлении первозданному Адаму, я сознаю себя лишенным Бога, вечного Царства и блаженства за мои грехи. Быт. 3:6–7\normalfont{}


\itshape Припев:\normalfont{} Помилуй мя, Боже, помилуй мя.


Увы мне, окаянная душе, что уподобилася еси первей Еве? видела бо еси зле, и уязвилася еси горце, и коснулася еси древа, и вкусила еси дерзостно безсловесныя снеди.


\itshape Горе мне, моя несчастная душа, для чего ты уподобилась первосозданной Еве? Не с добром ты посмотрела и уязвилась жестоко, прикоснулась к дереву и дерзостно вкусила бессмысленного плода. Быт. 3:6\normalfont{}


\itshape Припев:\normalfont{} Помилуй мя, Боже, помилуй мя.


Вместо Евы чувственныя мысленная ми бысть Ева, во плоти страстный помысл, показуяй сладкая и вкушаяй присно горькаго напоения.


\itshape Вместо чувственной Евы восстала во мне Ева мысленная "--- плотский страстный помысел, обольщающий приятным, но при вкушении всегда напояющий горечью.\normalfont{}


\itshape Припев:\normalfont{} Помилуй мя, Боже, помилуй мя.


Достойно из Едема изгнан бысть, яко не сохранив едину Твою, Спасе, заповедь Адам: аз же что постражду, отметая всегда животная Твоя словеса.


\itshape Достойно был изгнан из Едема Адам, как не сохранивший одной Твоей заповеди, Спаситель. Что же должен претерпеть я, всегда отвергающий Твои животворные повеления?\normalfont{}


Слава Отцу и Сыну и Святому Духу.


Пресущная Тро́ице, во Еди́нице покланяемая, возьми бремя от мене тяжкое греховное и, яко благоутробна, даждь ми слезы умиления.


\itshape Пресущественная Троица, Которой мы поклоняемся в одном Существе, сними с меня тяжкое бремя греховное и по Своему милосердию, даруй мне слезы умиления.\normalfont{}


И ныне и присно и во веки веков. Аминь.


Богородице, Надежде и Предстательство Тебе поющих, возьми бремя от мене тяжкое греховное, и, яко Владычица Чистая, кающася приими мя.


\itshape Богородице, надежда и защита воспевающих Тебя, сними с меня тяжкое бремя греховное и, как Владычица Чистая, приими меня кающегося.\normalfont{}





\bfseries Песнь 2\normalfont{}


\itshape Ирмо́с:\normalfont{}


Вонми, Небо, и возглаголю, и воспою Христа, от Девы плотию пришедшаго.


\itshape Внемли, небо, я буду возвещать и воспевать Христа, пришедшего во плоти от Девы.\normalfont{}


\itshape Припев:\normalfont{} Помилуй мя, Боже, помилуй мя.


Вонми, Небо, и возглаголю, земле, внушай глас, кающийся к Богу и воспевающий Его.


\itshape Внемли, небо, я буду возвещать: земля, услышь голос, кающийся Богу и прославляющий Его.\normalfont{}


\itshape Припев:\normalfont{} Помилуй мя, Боже, помилуй мя.


Вонми ми, Боже, Спасе мой, милостивым Твоим оком и приими мое теплое исповедание.


\itshape Воззри на меня Боже, Спаситель мой, милостивым Твоим оком и прими мою пламенную исповедь.\normalfont{}


\itshape Припев:\normalfont{} Помилуй мя, Боже, помилуй мя.


Согреших паче всех человек, един согреших Тебе; но ущедри, яко Бог, Спасе, творение Твое.


\itshape Согрешил я более всех людей, один я согрешил пред Тобою; но, как Бог, сжалься, Спаситель, над Твоим созданием. 1 Тим. 1:15\normalfont{}


\itshape Припев:\normalfont{} Помилуй мя, Боже, помилуй мя.


Вообразив моих страстей безобразие, любосластными стремленьми погубих ума красоту.


\itshape Отобразив в себе безобразие моих страстей, сластолюбивыми стремлениями исказил я красоту ума.\normalfont{}


\itshape Припев:\normalfont{} Помилуй мя, Боже, помилуй мя.


Буря мя злых обдержит, благоутробне Господи; но яко Петру и мне руку простри.


\itshape Буря зла окружает меня, Милосердный Господи, но, как Петру, и мне Ты простри руку. Мф. 14:31\normalfont{}


\itshape Припев:\normalfont{} Помилуй мя, Боже, помилуй мя.


Оскверних плоти моея ризу и окалях, еже по образу, Спасе, и по подобию.


\itshape Осквернил я одежду плоти моей и очернил в себе, Спаситель, то, что было создано по Твоему образу и подобию.\normalfont{}


\itshape Припев:\normalfont{} Помилуй мя, Боже, помилуй мя.


Омрачих душевную красоту страстей сластьми и всячески весь ум персть сотворих.


\itshape Помрачил я красоту души страстными удовольствиями и весь ум совершенно превратил в прах.\normalfont{}


\itshape Припев:\normalfont{} Помилуй мя, Боже, помилуй мя.


Раздрах ныне одежду мою первую, юже ми изтка Зиждитель из начала, и оттуду лежу наг.


\itshape Разодрал я первую одежду мою, которую вначале соткал мне Создатель, и оттого лежу обнаженным.\normalfont{}


\itshape Припев:\normalfont{} Помилуй мя, Боже, помилуй мя.


Облекохся в раздранную ризу, юже изтка ми змий советом, и стыждуся.


\itshape Облекся я в разодранную одежду, которую соткал мне змий коварством, и стыжусь. Быт. 3:21\normalfont{}


\itshape Припев:\normalfont{} Помилуй мя, Боже, помилуй мя.


Слезы блудницы, Щедре, и аз предлагаю, очисти мя, Спасе, благоутробием Твоим.


\itshape Как блудница, и я проливаю слезы, Милосердный; смилуйся надо мною, Спаситель, по благоснисхождению Твоему. Лк. 7:38\normalfont{}


\itshape Припев:\normalfont{} Помилуй мя, Боже, помилуй мя.


Воззрех на садовную красоту и прельстихся умом: и оттуду лежу наг и срамляюся.


\itshape Взглянул я на красоту дерева и прельстился в уме; оттого лежу обнаженным и стыжусь.\normalfont{}


\itshape Припев:\normalfont{} Помилуй мя, Боже, помилуй мя.


Делаша на хребте моем вси начальницы страстей, продолжающе на мя беззаконие их.


\itshape На хребте моем пахали все вожди страстей, проводя вдоль по мне беззаконие свое. Пс. 128:3\normalfont{}


Слава Отцу и Сыну и Святому Духу.


Единаго Тя в Триех Лицех, Бога всех пою, Отца и Сына и Духа Святаго.


\itshape Воспеваю Тебя, Единого в трех Лицах, Бога всех, Отца, Сына и Святого Духа.\normalfont{}


И ныне и присно и во веки веков. Аминь.


Пречистая Богородице Дево, Едина Всепетая, моли прилежно, во еже спастися нам.


\itshape Пречистая Богородица Дева, Ты Одна, всеми воспеваемая, усердно моли о нашем спасении.\normalfont{}





\bfseries Песнь 3\normalfont{}


\itshape Ирмо́с:\normalfont{}


На недвижимом, Христе, камени заповедей Твоих утверди мое помышление.


\itshape На неподвижном камне заповедей Твоих, Христе, утверди мое помышление.\normalfont{}


\itshape Припев:\normalfont{} Помилуй мя, Боже, помилуй мя.


Огнь от Господа иногда Господь одождив, землю содомскую прежде попали.


\itshape Пролив дождем огонь от Господа, Господь попалил некогда землю содомлян. Быт. 19:24\normalfont{}


\itshape Припев:\normalfont{} Помилуй мя, Боже, помилуй мя.


На горе спасайся, душе, якоже Лот оный, и в Сигор угонзай.


\itshape Спасайся на горе, душа, как праведный Лот и спеши укрыться в Сигор. Быт. 19:22–23\normalfont{}


\itshape Припев:\normalfont{} Помилуй мя, Боже, помилуй мя.


Бегай запаления, о душе, бегай содомскаго горения, бегай тления Божественнаго пламене.


\itshape Беги, душа, от пламени, беги от горящего Содома, беги от истребления Божественным огнем.\normalfont{}


\itshape Припев:\normalfont{} Помилуй мя, Боже, помилуй мя.


Согреших Тебе един аз, согреших паче всех, Христе Спасе, да не презриши мене.


\itshape Согрешил я один пред Тобою, согрешил более всех, Христос Спаситель "--- не презирай меня.\normalfont{}


\itshape Припев:\normalfont{} Помилуй мя, Боже, помилуй мя.


Ты еси Пастырь добрый, взыщи мене, агнца, и заблуждшаго да не презриши мене.


\itshape Ты "--- Пастырь Добрый, отыщи меня "--- агнца, и не презирай меня, заблудившегося. Ин. 10:11–14\normalfont{}


\itshape Припев:\normalfont{} Помилуй мя, Боже, помилуй мя.


Ты еси сладкий Иисусе, Ты еси Создателю мой, в Тебе, Спасе, оправдаюся.


\itshape Ты "--- вожделенный Иисус; Ты "--- Создатель мой, Спаситель, Тобою я оправдаюсь.\normalfont{}


\itshape Припев:\normalfont{} Помилуй мя, Боже, помилуй мя.


Исповедаюся Тебе, Спасе, согреших, согреших Ти; но ослаби, остави ми, яко благоутробен.


\itshape Исповедуюсь Тебе, Спаситель; согрешил я, согрешил пред Тобою, но отпусти, прости меня, как Милосердный.\normalfont{}


Слава Отцу и Сыну и Святому Духу.


О Тро́ице Еди́нице Боже, спаси нас от прелести, и искушений, и обстояний.


\itshape О, Троица, Единица, Боже, спаси нас от обольщений, от искушений и опасностей.\normalfont{}


И ныне и присно и во веки веков. Аминь.


Радуйся, Богоприятная утробо, радуйся, престоле Господень, радуйся, Мати Жизни нашея.


\itshape Радуйся, чрево, вместившее Бога; радуйся, Престол Господень; радуйся, Матерь Жизни нашей.\normalfont{}





\bfseries Песнь 4\normalfont{}


\itshape Ирмо́с:\normalfont{}


Услыша пророк пришествие Твое, Господи, и убояся, яко хощеши от Девы родитися и человеком явитися, и глаголаше: услышах слух Твой и убояхся, слава силе Твоей, Господи.


\itshape Услышал пророк о пришествии Твоем, Господи, и устрашился, что Тебе угодно родиться от Девы и явиться людям, и сказал: услышал я весть о Тебе и устрашился; слава силе Твоей, Господи.\normalfont{}


\itshape Припев:\normalfont{} Помилуй мя, Боже, помилуй мя.


Дел Твоих да не презриши, создания Твоего да не оставиши, Правосуде. Аще и един согреших, яко человек, паче всякаго человека, Человеколюбче; но имаши, яко Господь всех, власть оставляти грехи.


\itshape Не презри творений Твоих, не оставь создания Твоего, Праведный Судия, ибо хотя я, как человек, один согрешил более всякого человека, но Ты, Человеколюбец, как Господь всего мира, имеешь власть отпускать грехи. Мф. 9:6; Мк. 2:10\normalfont{}


\itshape Припев:\normalfont{} Помилуй мя, Боже, помилуй мя.


Приближается, душе, конец, приближается, и нерадиши, ни готовишися, время сокращается: востани, близ при дверех Судия есть. Яко соние, яко цвет, время жития течет: что всуе мятемся?


\itshape Конец приближается, душа, приближается, и ты не заботишься, не готовишься; время сокращается "--- восстань: Судия уже близко "--- при дверях; время жизни проходит, как сновиденье, как цвет. Для чего мы напрасно суетимся? Мф. 24:33; Мк. 13:29; Лк. 21:31\normalfont{}


\itshape Припев:\normalfont{} Помилуй мя, Боже, помилуй мя.


Воспряни, о душе моя, деяния твоя, яже соделала еси, помышляй, и сия пред лице твое принеси, и капли испусти слез твоих; рцы со дерзновением деяния и помышления Христу и оправдайся.


\itshape Пробудись, душа моя, размысли о делах своих, которые ты сделала, представь их пред своими очами, и пролей капли слез твоих, безбоязненно открой Христу дела и помышления твои и оправдайся.\normalfont{}


\itshape Припев:\normalfont{} Помилуй мя, Боже, помилуй мя.


Не бысть в житии греха, ни деяния, ни злобы, еяже аз, Спасе, не согреших, умом, и словом, и произволением, и предложением, и мыслию, и деянием согрешив, яко ин никтоже когда.


\itshape Нет в жизни ни греха, ни деяния, ни зла, в которых я не был бы виновен, Спаситель, умом, и словом, и произволением, согрешив и намерением, и мыслью, и делом так, как никто другой никогда.\normalfont{}


\itshape Припев:\normalfont{} Помилуй мя, Боже, помилуй мя.


Отсюду и осужден бых, отсюду препрен бых аз, окаянный, от своея совести, еяже ничтоже в мире нужнейше: Судие, Избавителю мой и Ведче, пощади, и избави, и спаси мя, раба Твоего.


\itshape Потому и обвиняюсь, потому и осуждаюсь я, несчастный, своею совестью, строже которой нет ничего в мире; Судия, Искупитель мой и Испытатель, пощади, избавь и спаси меня, раба Твоего.\normalfont{}


\itshape Припев:\normalfont{} Помилуй мя, Боже, помилуй мя.


Лествица, юже виде древле великий в патриарсех, указание есть, душе моя, деятельнаго восхождения, разумнаго возшествия: аще хощеши убо деянием, и разумом, и зрением пожити, обновися.


\itshape Лестница, которую в древности видел великий из патриархов, служит указанием, душа моя, на восхождение делами, на возвышение разумом; поэтому, если хочешь жить в деятельности и в разумении и созерцании, то обновляйся. Быт. 28:12\normalfont{}


\itshape Припев:\normalfont{} Помилуй мя, Боже, помилуй мя.


Зной дневный претерпе лишения ради патриарх и мраз нощный понесе, на всяк день снабдения творя, пасый, труждаяйся, работаяй, да две жене сочетает.


\itshape Патриарх по нужде терпел дневной зной и переносил ночной холод, ежедневно сокращая время, пася стада, трудясь и служа, чтобы получить себе две жены. Быт. 31:7, Быт. 31:40\normalfont{}


\itshape Припев:\normalfont{} Помилуй мя, Боже, помилуй мя.


Жены ми две разумей, деяние же и разум в зрении, Лию убо деяние, яко многочадную, Рахиль же разум, яко многотрудную; ибо кроме трудов ни деяние, ни зрение, душе, исправится.


\itshape Под двумя женами понимай деятельность и разумение в созерцании: под Лиею, как многочадною, "--- деятельность, а под Рахилью, как полученной через многие труды, "--- разумение, ибо без трудов, душа, ни деятельность, ни созерцание не усовершенствуются.\normalfont{}


Слава Отцу и Сыну и Святому Духу.


Нераздельное Существом, Неслитное Лицы богословлю Тя, Троическое Едино Божество, яко Единоцарственное и Сопрестольное, вопию Ти песнь великую, в вышних трегубо песнословимую.


\itshape Нераздельным по существу, неслиянным по Лицам богословски исповедую Тебя, Троичное Единое Божество, Соцарственное и Сопрестольное; возглашаю Тебе великую песнь, в небесных обителях троекратно воспеваемую. Ис. 6:1–3\normalfont{}


И ныне и присно и во веки веков. Аминь.


И раждаеши, и девствуеши, и пребываеши обоюду естеством Дева, Рождейся обновляет законы естества, утроба же раждает нераждающая. Бог идеже хощет, побеждается естества чин: творит бо, елика хощет.


\itshape И рождаешь Ты, и остаешься Девою, в обоих случаях сохраняя по естеству девство. Рожденный Тобою обновляет закон природы, а девственное чрево рождает; когда пожелает Бог, то нарушается порядок природы, ибо Он творит, что хочет.\normalfont{}





\bfseries Песнь 5\normalfont{}


\itshape Ирмо́с:\normalfont{}


От нощи утренююща, Человеколюбче, просвети, молюся, и настави и мене на повеления Твоя, и научи мя, Спасе, творити волю Твою.


\itshape От ночи бодрствующего, просвети меня, молю, Человеколюбец, путеводи меня в повелениях Твоих и научи меня, Спаситель, исполнять Твою волю. Пс. 62:2; Пс. 118:35\normalfont{}


\itshape Припев:\normalfont{} Помилуй мя, Боже, помилуй мя.


В нощи житие мое преидох присно, тьма бо бысть, и глубока мне мгла, нощь греха, но яко дне сына, Спасе, покажи мя.


\itshape Жизнь свою я постоянно проводил в ночи, ибо мраком и глубокою мглою была для меня ночь греха; но покажи меня сыном дня, Спаситель. Еф. 5:8\normalfont{}


\itshape Припев:\normalfont{} Помилуй мя, Боже, помилуй мя.


Рувима подражая, окаянный аз, содеях беззаконный и законопреступный совет на Бога Вышняго, осквернив ложе мое, яко отчее он.


\itshape Подобно Рувиму я, несчастный, совершил преступное и беззаконное дело пред Всевышним Богом, осквернив ложе мое, как тот "--- отчее. Быт. 35:22; Быт. 49:3–4\normalfont{}


\itshape Припев:\normalfont{} Помилуй мя, Боже, помилуй мя.


Исповедаюся Тебе, Христе Царю: согреших, согреших, яко прежде Иосифа братия продавшии, чистоты плод и целомудрия.


\itshape Исповедаюсь Тебе, Христос-Царь: согрешил я, согрешил, как некогда братья, продавшие Иосифа, "--- плод чистоты и целомудрия. Быт. 37:28\normalfont{}


\itshape Припев:\normalfont{} Помилуй мя, Боже, помилуй мя.


От сродников праведная душа связася, продася в работу сладкий, во образ Господень: ты же вся, душе, продалася еси злыми твоими.


\itshape Сродниками предана была душа праведная; возлюбленный продан в рабство, прообразуя Господа; ты же, душа, сама всю продала себя своим порокам.\normalfont{}


\itshape Припев:\normalfont{} Помилуй мя, Боже, помилуй мя.


Иосифа праведнаго и целомудреннаго ума подражай, окаянная и неискусная душе, и не оскверняйся безсловесными стремленьми, присно беззаконнующи.


\itshape Подражай праведному Иосифу и уму его целомудренному, несчастная и невоздержанная душа, не оскверняйся и не беззаконствуй всегда безрассудными стремлениями.\normalfont{}


\itshape Припев:\normalfont{} Помилуй мя, Боже, помилуй мя.


Аще и в рове поживе иногда Иосиф, Владыко Господи, но во образ погребения и востания Твоего: аз же что Тебе когда сицевое принесу?


\itshape Владыко Господи, Иосиф был некогда во рву, но в прообраз Твоего погребения и воскресения; принесу ли когда-либо что подобное Тебе я?\normalfont{}


Слава Отцу и Сыну и Святому Духу.


Тя, Тро́ице, славим Единаго Бога: Свят, Свят, Свят еси, Отче, Сыне и Душе, Про́стое Существо, Еди́нице присно покланяемая.


\itshape Тебя, Пресвятая Троица, прославляем за Единого Бога: Свят, Свят, Свят Ты, Отец, Сын и Дух "--- простая Сущность, Единица вечно поклоняемая.\normalfont{}


И ныне и присно и во веки веков. Аминь.


Из Тебе облечеся в мое смешение, нетленная, безмужная Мати Дево, Бог, создавый веки, и соедини Себе человеческое естество.


\itshape В Тебе, нетленная, не познавшая мужа Матерь-Дево, сотворивший мир Бог облекся в мой состав и соединил с Собою человеческую природу.\normalfont{}





\bfseries Песнь 6\normalfont{}


\itshape Ирмо́с:\normalfont{}


Возопих всем сердцем моим к щедрому Богу, и услыша мя от ада преисподняго, и возведе от тли живот мой.


\itshape От всего сердца моего я воззвал к милосердному Богу, и Он услышал меня из ада преисподнего и воззвал жизнь мою от погибели.\normalfont{}


\itshape Припев:\normalfont{} Помилуй мя, Боже, помилуй мя.


Слезы, Спасе, очию моею и из глубины воздыхания чисте приношу, вопиющу сердцу: Боже, согреших Ти, очисти мя.


\itshape Искренно приношу Тебе, Спаситель, слезы очей моих и воздыхания из глубины сердца, взывающего: Боже, согрешил я пред Тобою, умилосердись надо мною.\normalfont{}


\itshape Припев:\normalfont{} Помилуй мя, Боже, помилуй мя.


Уклонилася еси, душе, от Господа твоего, якоже Дафан и Авирон, но пощади, воззови из ада преисподняго, да не пропасть земная тебе покрыет.


\itshape Уклонилась ты, душа, от Господа своего, как Дафан и Авирон, но воззови из ада преисподнего: пощади!, чтобы пропасть земная не поглотила тебя. Чис. 16:32\normalfont{}


\itshape Припев:\normalfont{} Помилуй мя, Боже, помилуй мя.


Яко юница, душе, разсвирепевшая, уподобилася еси Ефрему, яко серна от тенет сохрани житие, вперивши деянием ум и зрением.


\itshape Рассвирепев, как телица, ты, душа, уподобилась Ефрему, но как серна спасай от тенет свою жизнь, окрылив ум деятельностью и созерцанием. Иер. 31:18; Ос. 10:11\normalfont{}


\itshape Припев:\normalfont{} Помилуй мя, Боже, помилуй мя.


Рука нас Моисеова да уверит, душе, како может Бог прокаженное житие убелити и очистити, и не отчайся сама себе, аще и прокаженна еси.


\itshape Моисеева рука да убедит нас, душа, как Бог может убелить и очистить прокаженную жизнь, и не отчаивайся сама за себя, хотя ты и поражена проказою. Исх. 4:6–7\normalfont{}


Слава Отцу и Сыну и Святому Духу.


Тро́ица есмь Про́ста, Нераздельна, раздельна Личне, и Еди́ница есмь естеством соединена, Отец глаголет, и Сын, и Божественный Дух.


\itshape Я "--- Троица несоставная, нераздельная, раздельная в лицах, и Единица, соединенная по существу; свидетельствует Отец, Сын и Божественный Дух.\normalfont{}


И ныне и присно и во веки веков. Аминь.


Утроба Твоя Бога нам роди, воображена по нам: Егоже, яко Создателя всех, моли, Богородице, да молитвами Твоими оправдимся.


\itshape Чрево Твое родило нам Бога, принявшего наш образ; Его, как Создателя всего мира, моли, Богородица, чтобы по молитвам Твоим нам оправдаться.\normalfont{}





\bfseries Кондак, глас 6:\normalfont{}


Душе моя, душе моя, востани, что спиши? конец приближается, и имаши смутитися: воспряни убо, да пощадит тя Христос Бог, везде сый и вся исполняяй.


\itshape Душа моя, душа моя, восстань, что ты спишь? Конец приближается, и ты смутишься; пробудись же, чтобы пощадил тебя Христос Бог, Вездесущий и все наполняющий.\normalfont{}





\bfseries Песнь 7\normalfont{}


\itshape Ирмо́с:\normalfont{}


Согрешихом, беззаконновахом, неправдовахом пред Тобою, ниже соблюдохом, ниже сотворихом, якоже заповедал еси нам; но не предаждь нас до конца, отцев Боже.


\itshape Мы согрешили, жили беззаконно, неправо поступали пред Тобою, не сохранили, не исполнили, что Ты заповедал нам; но не оставь нас до конца, Боже отцов. Дан. 9:5–6\normalfont{}


\itshape Припев:\normalfont{} Помилуй мя, Боже, помилуй мя.


Согреших, беззаконновах и отвергох заповедь Твою, яко во гресех произведохся, и приложих язвам струпы себе; но Сам мя помилуй, яко благоутробен, отцев Боже.


\itshape Я согрешил, жил в беззакониях и нарушил заповедь Твою, ибо я рожден в грехах и к язвам своим приложил еще раны, но Сам Ты помилуй меня, как Милосердный Боже отцов.\normalfont{}


\itshape Припев:\normalfont{} Помилуй мя, Боже, помилуй мя.


Тайная сердца моего исповедах Тебе, Судии моему, виждь мое смирение, виждь и скорбь мою, и вонми суду моему ныне, и Сам мя помилуй, яко благоутробен, отцев Боже.


\itshape Тайны сердца моего я открыл пред Тобою, Судьей моим; воззри на смирение мое, воззри и на скорбь мою, обрати внимание на мое ныне осуждение и Сам помилуй меня, как Милосердный, Боже отцов. Пс. 37:19; Пс. 24:18; Пс. 34:23\normalfont{}


\itshape Припев:\normalfont{} Помилуй мя, Боже, помилуй мя.


Саул иногда, яко погуби отца своего, душе, ослята, внезапу царство обрете к прослутию; но блюди, не забывай себе, скотския похоти твоя произволивши паче Царства Христова.


\itshape Саул, некогда потеряв ослиц своего отца, неожиданно с известием о них получил царство; душа, не забывайся, предпочитая свои скотские стремления Христову Царству. 1 Цар. 9:1–27; 1 Цар. 10:1\normalfont{}


\itshape Припев:\normalfont{} Помилуй мя, Боже, помилуй мя.


Давид иногда Богоотец, аще и согреши сугубо, душе моя, стрелою убо устрелен быв прелюбодейства, копием же пленен быв убийства томлением; но ты сама тяжчайшими делы недугуеши, самохотными стремленьми.


\itshape Если богоотец Давид некогда и вдвойне согрешил, будучи уязвлен стрелою прелюбодеяния, сражен был копьем мщения за убийства; но ты, душа моя, сама страдаешь более тяжко, нежели этими делами, произвольными стремлениями. 2 Цар. 11:14–15\normalfont{}


\itshape Припев:\normalfont{} Помилуй мя, Боже, помилуй мя.


Совокупи убо Давид иногда беззаконию беззаконие, убийству же любодейство растворив, покаяние сугубое показа абие; но сама ты, лукавнейшая душе, соделала еси, не покаявшися Богу.


\itshape Давид некогда присовокупил беззаконие к беззаконию, ибо с убийством соединил прелюбодеяние, но скоро принес и усиленное покаяние, а ты, коварнейшая душа, совершив бОльшие грехи, не раскаялась пред Богом.\normalfont{}


\itshape Припев:\normalfont{} Помилуй мя, Боже, помилуй мя.


Давид иногда вообрази, списав яко на иконе песнь, еюже деяние обличает, еже содея, зовый: помилуй мя, Тебе бо единому согреших всех Богу, Сам очисти мя.


\itshape Давид некогда, изображая как бы на картине, начертал песнь, которой обличает совершенный им проступок, взывая: помилуй мя, ибо согрешил я пред Тобою, Одним, Богом всех; Сам очисти меня. Пс. 50:3–6\normalfont{}


Слава Отцу и Сыну и Святому Духу.


Тро́ице Про́стая, Нераздельная, Единосущная и Естество Едино, Светове и Свет, и Свята Три, и Едино Свято поется Бог Тро́ица; но воспой, прослави Живот и Животы, душе, всех Бога.


\itshape Троица простая, нераздельная, единосущная, и одно Божество, Светы и Свет, три Святы и одно лицо Свято, Бог-Троица, воспеваемая в песнопениях; воспой же и ты, душа, прославь Жизнь и Жизни "--- Бога всех.\normalfont{}


И ныне и присно и во веки веков. Аминь.


Поем Тя, благословим Тя, покланяемся Ти, Богородительнице, яко Нераздельныя Тро́ицы породила еси Единаго Христа Бога, и Сама отверзла еси нам, сущим на земли, Небесная.


\itshape Воспеваем Тебя, благословляем Тебя, поклоняемся Тебе, Богородительница, ибо Ты родила Одного из Нераздельной Троицы, Христа Бога, и Сама открыла для нас, живущих на земле, небесные обители.\normalfont{}





\bfseries Песнь 8\normalfont{}


\itshape Ирмо́с:\normalfont{}


Егоже воинства Небесная славят, и трепещут херувими и серафими, всяко дыхание и тварь, пойте, благословите и превозносите во вся веки.


\itshape Кого прославляют воинства небесные и пред Кем трепещут Херувимы и Серафимы, Того, все существа и творения, воспевайте, благословляйте и превозносите во все века.\normalfont{}


\itshape Припев:\normalfont{} Помилуй мя, Боже, помилуй мя.


Согрешивша, Спасе, помилуй, воздвигни мой ум ко обращению, приими мя кающагося, ущедри вопиюща: согреших Ти, спаси, беззаконновах, помилуй мя.


\itshape Помилуй меня, грешника, Спаситель, пробуди мой ум к обращению, приими кающегося, умилосердись над взывающим: я согрешил пред Тобою, спаси; я жил в беззакониях, помилуй меня.\normalfont{}


\itshape Припев:\normalfont{} Помилуй мя, Боже, помилуй мя.


Колесничник Илия колесницею добродетелей вшед, яко на небеса, ношашеся превыше иногда от земных: сего убо, душе моя, восход помышляй.


\itshape Везомый на колеснице Илия, взойдя на колесницу добродетелей, некогда вознесся как бы на небеса, превыше всего земного; помышляй, душа моя, об его восходе. 4 Цар. 2:11\normalfont{}


\itshape Припев:\normalfont{} Помилуй мя, Боже, помилуй мя.


Елиссей иногда прием милоть Илиину, прият сугубую благодать от Бога; ты же, о душе моя, сея не причастилася еси благодати за невоздержание.


\itshape Некогда Елисей, приняв милоть (плащ) Илии, получил сугубую благодать от Господа; но ты, душа моя, не получила этой благодати за невоздержание. 4 Цар. 2:9,4 Цар. 12–13\normalfont{}


\itshape Припев:\normalfont{} Помилуй мя, Боже, помилуй мя.


Иорданова струя первее милотию Илииною Елиссеем ста сюду и сюду; ты же, о душе моя, сея не причастилася еси благодати за невоздержание.


\itshape Елисея милотию Илии некогда разделил поток Иордана на ту и другую сторону; но ты, душа моя, не получила этой благодати за невоздержание. 4 Цар. 2:14\normalfont{}


\itshape Припев:\normalfont{} Помилуй мя, Боже, помилуй мя.


Соманитида иногда праведнаго учреди, о душе, нравом благим; ты же не ввела еси в дом ни странна, ни путника. Темже чертога изринешися вон, рыдающи.


\itshape Соманитянка некогда угостила праведника с добрым усердием; а ты, душа, не приняла в свой дом ни странника, ни пришельца; за то будешь извержена вон из брачного чертога с рыданием. 4 Цар. 4:8\normalfont{}


\itshape Припев:\normalfont{} Помилуй мя, Боже, помилуй мя.


Гиезиев подражала еси, окаянная, разум скверный всегда, душе, егоже сребролюбие отложи поне на старость; бегай геенскаго огня, отступивши злых твоих.


\itshape Ты, несчастная душа, непрестанно подражала нечистому нраву Гиезия; хотя в старости отвергни его сребролюбие и, оставив свои злодеяния, избегни огня геенского. 4 Цар. 5:20–27\normalfont{}


Слава Отцу и Сыну и Святому Духу.


Безначальне Отче, Сыне Собезначальне, Утешителю Благий, Душе Правый, Слова Божия Родителю, Отца Безначальна Слове, Душе Живый и Зиждяй, Тро́ице Еди́нице, помилуй мя.


\itshape Безначальный Отче, Собезначальный Сын, Утешитель Благий, Дух Правый, Родитель Слова Божия, Безначальное Слово Отца, Дух, Животворящий и Созидающий, Троица Единая, помилуй меня.\normalfont{}


И ныне и присно и во веки веков. Аминь.


Яко от оброщения червленицы, Пречистая, умная багряница Еммануилева внутрь во чреве Твоем плоть исткася. Темже Богородицу воистинну Тя почитаем.


\itshape Мысленная порфира "--- плоть Еммануила соткалась внутри Твоего чрева, Пречистая, как бы из вещества пурпурного; потому мы почитаем Тебя, Истинную Богородицу.\normalfont{}





\bfseries Песнь 9\normalfont{}


\itshape Ирмо́с:\normalfont{}


Безсеменнаго зачатия Рождество несказанное, Матере безмужныя нетленен Плод, Божие бо Рождение обновляет естества. Темже Тя вси роди, яко Богоневестную Матерь, православно величаем.


\itshape Рождество от бессеменного зачатия неизъяснимо, безмужной Матери нетленен Плод, ибо рождение Бога обновляет природу. Поэтому Тебя, как Богоневесту-Матерь мы, все роды, православно величаем.\normalfont{}


\itshape Припев:\normalfont{} Помилуй мя, Боже, помилуй мя.


Ум острупися, тело оболезнися, недугует дух, слово изнеможе, житие умертвися, конец при дверех. Темже, моя окаянная душе, что сотвориши, егда приидет Судия испытати твоя.


\itshape Ум изранился, тело расслабилось, дух болезнует, слово потеряло силу, жизнь замерла, конец при дверях. Что же сделаешь ты, несчастная душа, когда придет Судия исследовать дела твои?\normalfont{}


\itshape Припев:\normalfont{} Помилуй мя, Боже, помилуй мя.


Моисеово приведох ти, душе, миробытие, и от того все заветное Писание, поведающее тебе праведныя и неправедныя: от нихже вторыя, о душе, подражала еси, а не первыя, в Бога согрешивши.


\itshape Я воспроизвел пред тобою, душа, сказание Моисея о бытии мира и затем все Заветное Писание, повествующее о праведных и неправедных; из них ты, душа, подражала последним, а не первым, согрешая пред Богом.\normalfont{}


\itshape Припев:\normalfont{} Помилуй мя, Боже, помилуй мя.


Закон изнеможе, празднует Евангелие, Писание же все в тебе небрежено бысть, пророцы изнемогоша и все праведное слово; струпи твои, о душе, умножишася, не сущу врачу, исцеляющему тя.


\itshape Ослабел закон, не воздействует Евангелие, пренебрежено все Писание тобою, пророки и всякое слово праведника потеряли силу; язвы твои, душа, умножились, без Врача, исцеляющего тебя.\normalfont{}


\itshape Припев:\normalfont{} Помилуй мя, Боже, помилуй мя.


Новаго привожду ти Писания указания, вводящая тя, душе, ко умилению: праведным убо поревнуй, грешных же отвращайся и умилостиви Христа молитвами же, и пощеньми, и чистотою, и говением.


\itshape Из Новозаветного Писания привожу тебе примеры, душа, возбуждающие в тебе умиление; так подражай праведным и отвращайся примера грешных и умилостивляй Христа молитвою, постом, чистотою и непорочностью.\normalfont{}


\itshape Припев:\normalfont{} Помилуй мя, Боже, помилуй мя.


Христос вочеловечися, призвав к покаянию разбойники и блудницы; душе, покайся, дверь отверзеся Царствия уже, и предвосхищают е фарисее и мытари и прелюбодеи кающиися.


\itshape Христос, сделавшись человеком, призвал к покаянию разбойников и блудниц; покайся, душа, дверь Царства уже открылась, и прежде тебя входят в нее кающиеся фарисеи, мытари и прелюбодеи. Мф. 11:12; Мф. 21:31; Лк. 16:16\normalfont{}


\itshape Припев:\normalfont{} Помилуй мя, Боже, помилуй мя.


Христос вочеловечися, плоти приобщився ми, и вся елика суть естества хотением исполни греха кроме, подобие тебе, о душе, и образ предпоказуя Своего снисхождения.


\itshape Христос, сделался человеком, приобщившись ко мне плотию, и добровольно испытал все, что свойственно природе, за исключением греха, показывая тебе, душа, пример и образец Своего снисхождения.\normalfont{}


\itshape Припев:\normalfont{} Помилуй мя, Боже, помилуй мя.


Христос волхвы спасе, пастыри созва, младенец множества показа мученики, старцы прослави и старыя вдовицы, ихже не поревновала еси, душе, ни деянием, ни житию, но горе тебе, внегда будеши судитися.


\itshape Христос спас волхвов, призвал к Себе пастухов, множество младенцев сделал мучениками, прославил старца и престарелую вдовицу[1]; их деяниям и жизни ты не подражала, душа, но горе тебе, когда будешь судима! Мф. 2:1, 16; Лк. 2:4–8 и след.; Лк. 2:25–26 и след.; Лк. 2:36–38\normalfont{}


\itshape Припев:\normalfont{} Помилуй мя, Боже, помилуй мя.


Постився Господь дний четыредесять в пустыни, последи взалка, показуя человеческое; душе, да не разленишися, аще тебе приложится враг, молитвою же и постом от ног твоих да отразится.


\itshape Господь, постившись сорок дней в пустыне, наконец взалкал, обнаруживая в Себе человеческую природу. Не унывай, душа, если враг устремится на тебя, но да отразится он от ног твоих молитвами и постом. Исх. 34:28; Мф. 4:2; Лк. 4:2; Мк. 1:13\normalfont{}


Слава Отцу и Сыну и Святому Духу.


Отца прославим, Сына превознесем, Божественному Духу верно поклонимся, Тро́ице Нераздельней, Еди́нице по существу, яко Свету и Светом, и Животу и Животом, животворящему и просвещающему концы.


\itshape Прославим Отца, превознесем Сына, с верою поклонимся Божественному Духу, Нераздельной Троице, Единой по существу, как Свету и Светам, Жизни и Жизням, животворящему и просвещающему пределы вселенной.\normalfont{}


И ныне и присно и во веки веков. Аминь.


Град Твой сохраняй, Богородительнице Пречистая, в Тебе бо сей верно царствуяй, в Тебе и утверждается, и Тобою побеждаяй, побеждает всякое искушение, и пленяет ратники, и проходит послушание.


\itshape Сохраняй град Свой, Пречистая Богородительница. Под Твоею защитою он царствует с верою, и от Тебя получает крепость, и при Твоем содействии неотразимо побеждает всякое бедствие, берет в плен врагов и держит их в подчинении.\normalfont{}


\itshape Припев:\normalfont{} Преподобне отче Андрее, моли Бога о нас.


Андрее честный и отче треблаженнейший, пастырю Критский, не престай моляся о воспевающих тя: да избавимся вси гнева и скорби, и тления, и прегрешений безмерных, чтущии твою память верно.


\itshape Андрей досточтимый, отец преблаженный, пастырь Критский, не переставай молиться за воспевающих тебя, чтобы избавиться от гнева, скорби, погибели и бесчисленных прегрешений нам всем, искренно почитающим память твою.\normalfont{}


\itshape Таже оба лика вкупе поют Ирмо́с:\normalfont{}


Безсеменнаго зачатия Рождество несказанное, Матере безмужныя нетленен Плод, Божие бо Рождение обновляет естества. Темже Тя вси роди, яко Богоневестную Матерь, православно величаем.


\itshape Рождество от бессеменного зачатия неизъяснимо, безмужной Матери нетленен Плод, ибо рождение Бога обновляет природу. Поэтому Тебя, как Богоневесту-Матерь мы, все роды, православно величаем.\normalfont{}


________________


[1]Здесь имеются в виду Симеон Богоприимец и Анна-пророчица.


\mychapterending

\mychapter{Во вторник первой седмицы Великого Поста}
%http://www.molitvoslov.org/text572.htm 
 






\bfseries Песнь 1\normalfont{}


\itshape Ирмо́с:\normalfont{}


Помощник и Покровитель бысть мне во спасение, Сей мой Бог, и прославлю Его, Бог отца моего, и вознесу Его: славно бо прославися.


\itshape Помощник и Покровитель явился мне ко спасению, Он Бог мой, и прославлю Его, Бога отца моего, и превознесу Его, ибо Он торжественно прославился. Исх. 15:1–2\normalfont{}


\itshape Припев:\normalfont{} Помилуй мя, Боже, помилуй мя.


Каиново прешед убийство, произволением бых убийца совести душевней, оживив плоть и воевав на ню лукавыми моими деяньми.


\itshape Превзойдя Каиново убийство, сознательным произволением, через оживление греховной плоти, я сделался убийцею души, вооружившись против нее злыми моими делами. Быт. 4:8\normalfont{}


\itshape Припев:\normalfont{} Помилуй мя, Боже, помилуй мя.


Авелеве, Иисусе, не уподобихся правде, дара Тебе приятна не принесох когда, ни деяния божественна, ни жертвы чистыя, ни жития непорочнаго.


\itshape Авелевой праведности, Иисусе, я не подражал, никогда не приносил Тебе приятных даров, ни дел богоугодных, ни жертвы чистой, ни жизни непорочной. Быт. 4:4\normalfont{}


\itshape Припев:\normalfont{} Помилуй мя, Боже, помилуй мя.


Яко Каин и мы, душе окаянная, всех Содетелю деяния скверная, и жертву порочную, и непотребное житие принесохом вкупе: темже и осудихомся.


\itshape Как Каин, так и мы, несчастная душа, принесли Создателю всего жертву порочную "--- дела нечестивые и жизнь невоздержанную: поэтому мы и осуждены. Быт. 4:3–5\normalfont{}


\itshape Припев:\normalfont{} Помилуй мя, Боже, помилуй мя.


Брение Здатель живосоздав, вложил еси мне плоть и кости, и дыхание, и жизнь; но, о Творче мой, Избавителю мой и Судие, кающася приими мя.


\itshape Оживотворивший земной прах, Скудельник, Ты даровал мне плоть и кости, дыхание и жизнь; но, Творец мой, Искупитель мой и Судия, приими меня кающегося! Быт. 2:7\normalfont{}


\itshape Припев:\normalfont{} Помилуй мя, Боже, помилуй мя.


Извещаю Ти, Спасе, грехи, яже содеях, и души и тела моего язвы, яже внутрь убийственнии помыслы разбойнически на мя возложиша.


\itshape Пред Тобою, Спаситель, открываю грехи, сделанные мною, и раны души и тела моего, которые разбойнически нанесли мне внутренние убийственные помыслы. Лк. 10:30\normalfont{}


\itshape Припев:\normalfont{} Помилуй мя, Боже, помилуй мя.


Аще и согреших, Спасе, но вем, яко Человеколюбец еси, наказуеши милостивно и милосердствуеши тепле: слезяща зриши и притекаеши, яко отец, призывая блуднаго.


\itshape Хотя я и согрешил, Спаситель, но знаю, что Ты человеколюбив; наказываешь с состраданием и милуешь с любовью, взираешь на плачущего и спешишь, как Отец, призвать блудного. Лк. 15:20\normalfont{}


Слава Отцу и Сыну и Святому Духу.


Пресущная Тро́ице, во Еди́нице покланяемая, возьми бремя от мене тяжкое греховное и, яко благоутробна, даждь ми слезы умиления.


\itshape Пресущественная Троица, Которой мы поклоняемся как Единому существу, сними с меня тяжелое бремя греховное и, как Милосердная, даруй мне слезы умиления.\normalfont{}


И ныне и присно и во веки веков. Аминь.


Богородице, Надежде и Предстательство Тебе поющих, возьми бремя от мене тяжкое греховное и, яко Владычица Чистая, кающася приими мя.


\itshape Богородице, Надежда и Помощь всем воспевающих Тебя, сними с меня тяжелое бремя греховное и, как Владычица Непорочная, прими меня кающегося.\normalfont{}





\bfseries Песнь 2\normalfont{}


\itshape Ирмо́с:\normalfont{}


Вонми, Небо, и возглаголю, и воспою Христа, от Девы плотию пришедшаго.


\itshape Внемли, небо, я буду возвещать и воспевать Христа, пришедшего во плоти от Девы.\normalfont{}


\itshape Припев:\normalfont{} Помилуй мя, Боже, помилуй мя.


Сшиваше кожныя ризы грех мне, обнаживый мя первыя боготканныя одежды.


\itshape «Кожаные ризы» сшил мне грех, сняв с меня прежнюю Богом сотканную одежду.\normalfont{}


\itshape Припев:\normalfont{} Помилуй мя, Боже, помилуй мя.


Обложен есмь одеянием студа, якоже листвием смоковным, во обличение моих самовластных страстей.


\itshape Облекся я одеянием стыда, как листьями смоковницы, во обличение самовольных страстей моих. Быт. 3:7\normalfont{}


\itshape Припев:\normalfont{} Помилуй мя, Боже, помилуй мя.


Одеяхся в срамную ризу и окровавленную студно течением страстнаго и любосластнаго живота.


\itshape Оделся я в одежду, постыдно запятнанную и окровавленную нечистотой страстной и сластолюбивой жизни.\normalfont{}


\itshape Припев:\normalfont{} Помилуй мя, Боже, помилуй мя.


Впадох в страстную пагубу и в вещественную тлю, и оттоле до ныне враг мне досаждает.


\itshape Подвергся я мучению страстей и вещественному тлению, и оттого ныне враг угнетает меня.\normalfont{}


\itshape Припев:\normalfont{} Помилуй мя, Боже, помилуй мя.


Любовещное и любоименное житие невоздержанием, Спасе, предпочет ныне, тяжким бременем обложен есмь.


\itshape Предпочтя нестяжательности жизнь, привязанную к земным вещам и любостяжательную, Спаситель, я теперь нахожусь под тяжким бременем.\normalfont{}


\itshape Припев:\normalfont{} Помилуй мя, Боже, помилуй мя.


Украсих плотский образ скверных помышлений различным обложением и осуждаюся.


\itshape Украсил я кумир плоти разноцветным одеянием гнусных помыслов и подвергаюсь осуждению.\normalfont{}


\itshape Припев:\normalfont{} Помилуй мя, Боже, помилуй мя.


Внешним прилежно благоукрашением единем попекохся, внутреннюю презрев Богообразную скинию.


\itshape Усердно заботясь об одном внешнем благолепии, я пренебрег внутренней скинией, устроенной по образу Божию.\normalfont{}


\itshape Припев:\normalfont{} Помилуй мя, Боже, помилуй мя.


Погребох перваго образа доброту, Спасе, страстьми, юже, яко иногда драхму, взыскав, обрящи.


\itshape Засыпал страстями красоту первобытного образа, Спаситель; ее, как некогда драхму, Ты взыщи и найди. Лк. 15:8\normalfont{}


\itshape Припев:\normalfont{} Помилуй мя, Боже, помилуй мя.


Согреших, якоже блудница, вопию Ти: един согреших Тебе, яко миро, приими, Спасе, и моя слезы.


\itshape Согрешил, и, как блудница, взываю к Тебе: один я согрешил пред Тобою, приими, Спаситель, и от меня слезы вместо мира. Лк. 7:37–38\normalfont{}


\itshape Припев:\normalfont{} Помилуй мя, Боже, помилуй мя.


Очисти, якоже мытарь, вопию Ти, Спасе, очисти мя: никтоже бо сущих из Адама, якоже аз, согреших Тебе.


\itshape Умилостивись, как мытарь, взываю к Тебе, Спаситель, смилуйся надо мною: ибо как никто из потомков Адамовых я согрешил пред Тобою. Лк. 18:13\normalfont{}


Слава Отцу и Сыну и Святому Духу.


Единаго Тя в Триех Лицех, Бога всех пою, Отца и Сына и Духа Святаго.


\itshape Воспеваю Тебя, Одного в трех Лицах Бога всех, Отца, Сына и Святого Духа.\normalfont{}


И ныне и присно и во веки веков. Аминь.


Пречистая Богородице Дево, Едина Всепетая, моли прилежно, во еже спастися нам.


\itshape Пречистая Богородице Дево, Ты Одна всеми воспеваемая, усердно моли о нашем спасении.\normalfont{}





\bfseries Песнь 3\normalfont{}


\itshape Ирмо́с:\normalfont{}


Утверди, Господи, на камени заповедей Твоих подвигшееся сердце мое, яко Един Свят еси и Господь.


\itshape Утверди, Господи, на камне Твоих заповедей поколебавшееся сердце мое, ибо Ты один свят и Господь.\normalfont{}


\itshape Припев:\normalfont{} Помилуй мя, Боже, помилуй мя.


Источник живота стяжах Тебе, смерти Низложителя, и вопию Ти от сердца моего прежде конца: согреших, очисти и спаси мя.


\itshape Источник жизни нашел я в Тебе, Разрушитель смерти, и прежде кончины взываю к Тебе от сердца моего: согрешил я, умилостивись, спаси меня.\normalfont{}


\itshape Припев:\normalfont{} Помилуй мя, Боже, помилуй мя.


Согреших, Господи, согреших Тебе, очисти мя: несть бо иже кто согреши в человецех, егоже не превзыдох прегрешеньми.


\itshape Согрешил я, Господи, согрешил пред Тобою, смилуйся надо мною, ибо нет грешника между людьми, которого я не превзошел бы прегрешениями.\normalfont{}


\itshape Припев:\normalfont{} Помилуй мя, Боже, помилуй мя.


При Нои, Спасе, блудствовавшия подражах, онех наследствовах осуждение в потопе погружения.


\itshape Я подражал, Спаситель, развращенным современникам Ноя и наследовал осуждение их на потопление в потопе.


Быт. 6:1–17\normalfont{}


\itshape Припев:\normalfont{} Помилуй мя, Боже, помилуй мя.


Хама онаго, душе, отцеубийца подражавши, срама не покрыла еси искренняго, вспять зря возвратившися.


\itshape Подражая отцеубийце Хаму, ты, душа, не прикрыла срамоты ближнего с лицом, обращенным назад. Быт. 9:22–23\normalfont{}


\itshape Припев:\normalfont{} Помилуй мя, Боже, помилуй мя.


Запаления, якоже Лот, бегай, душе моя, греха: бегай Содомы и Гоморры, бегай пламене всякаго безсловеснаго желания.


\itshape Беги, душа моя, от пламени греха; как Лот; беги от Содома и Гоморры; беги от огня всякого безрассудного пожелания. Быт. 19:15–17\normalfont{}


\itshape Припев:\normalfont{} Помилуй мя, Боже, помилуй мя.


Помилуй, Господи, помилуй мя, вопию Ти, егда приидеши со ангелы Твоими воздати всем по достоянию деяний.


\itshape Помилуй, Господи, взываю к Тебе, помилуй меня, когда придешь с Ангелами Своими воздать всем по достоинству их дел.\normalfont{}


Слава Отцу и Сыну и Святому Духу.


Тро́ице Про́стая, Несозданная, Безначальное Естество, в Тро́ице певаемая Ипостасей, спаси ны, верою покланяющияся державе Твоей.


\itshape Троица несоставная, несозданная, Существо Безначальная, в троичности лиц воспеваемая, спаси нас, с верою поклоняющихся силе Твоей.\normalfont{}


И ныне и присно и во веки веков. Аминь.


От Отца безлетна Сына в лето, Богородительнице, неискусомужно родила еси, странное чудо, пребывши Дева доящи.


\itshape Ты, Богородительница, не испытавши мужа, во времени родила Сына от Отца вне времени и "--- дивное чудо: питая молоком, пребыла Девою.\normalfont{}





\bfseries Песнь 4\normalfont{}


\itshape Ирмо́с:\normalfont{}


Услыша пророк пришествие Твое, Господи, и убояся, яко хощеши от Девы родитися и человеком явитися, и глаголаше: услышах слух Твой и убояхся, слава силе Твоей, Господи.


\itshape Услышал пророк о пришествии Твоем, Господи, и устрашился, что Тебе угодно родиться от Девы и явиться людям, и сказал: услышал я весть о Тебе и устрашился; слава силе Твоей, Господи.\normalfont{}


\itshape Припев:\normalfont{} Помилуй мя, Боже, помилуй мя.


Бди, о душе моя, изрядствуй, якоже древле великий в патриарсех, да стяжеши деяние с разумом, да будеши ум, зряй Бога, и достигнеши незаходящий мрак в видении, и будеши великий купец.


\itshape Бодрствуй, душа моя, будь мужественна, как великий из патриархов, чтобы приобрести себе дело по разуму, чтобы обогатиться умом, видящим Бога, и проникнуть в неприступный мрак в созерцании и получить великое сокровище. Быт. 32:28\normalfont{}


\itshape Припев:\normalfont{} Помилуй мя, Боже, помилуй мя.


Дванадесять патриархов великий в патриарсех детотворив, тайно утверди тебе лествицу деятельнаго, душе моя, восхождения: дети, яко основания, степени, яко восхождения, премудренно подложив.


\itshape Великий из патриархов, родив двенадцать патриархов, таинственно представил тебе, душа моя, лестницу деятельного восхождения, премудро расположив детей как ступени, а свои шаги, как восхождения вверх.\normalfont{}


\itshape Припев:\normalfont{} Помилуй мя, Боже, помилуй мя.


Исава возненавиденнаго подражала еси, душе, отдала еси прелестнику твоему первыя доброты первенство и отеческия молитвы отпала еси, и дважды поползнулася еси, окаянная, деянием и разумом: темже ныне покайся.


\itshape Подражая ненавиденному Исаву, душа, ты отдала соблазнителю своему первенство первоначальной красоты и лишилась отеческого благословения и, несчастная, пала дважды, деятельностью и разумением, поэтому ныне покайся. Быт. 25:32; Быт. 27:37; Мал. 1:2–3\normalfont{}


\itshape Припев:\normalfont{} Помилуй мя, Боже, помилуй мя.


Едом Исав наречеся, крайняго ради женонеистовнаго смешения: невоздержанием бо присно разжигаем и сластьми оскверняем, Едом именовася, еже глаголется разжжение души любогреховныя.


\itshape Исав был назван Едомом за крайнее пристрастие к женолюбию; он непрестанно разжигаясь невоздержанием и оскверняясь любострастием, назван Едомом, что значит "--- «распаление души грехолюбивой». Быт. 25:30\normalfont{}


\itshape Припев:\normalfont{} Помилуй мя, Боже, помилуй мя.


Иова на гноищи слышавши, о душе моя, оправдавшагося, того мужеству не поревновала еси, твердаго не имела еси предложения во всех, яже веси, и имиже искусилася еси, но явилася еси нетерпелива.


\itshape Слышав об Иове, сидевшем на гноище, ты, душа моя, не подражала ему в мужестве, не имела твердой воли во всем, что узнала, что видела, что испытала, но оказалась нетерпеливою. Иов. 1:1–22\normalfont{}


\itshape Припев:\normalfont{} Помилуй мя, Боже, помилуй мя.


Иже первее на престоле, наг ныне на гноище гноен, многий в чадех и славный, безчаден и бездомок напрасно: палату убо гноище и бисерие струпы вменяше.


\itshape Бывший прежде на престоле, теперь "--- на гноище, обнаженный и изъязвленный; имевший многих детей и знаменитый, внезапно стал бездетным и бездомным; гноище считал он своим чертогом и язвы "--- драгоценными камнями. Иов. 2:11–13\normalfont{}


Слава Отцу и Сыну и Святому Духу.


Нераздельное Существом, Неслитное Лицы богословлю Тя, Троическое Едино Божество, яко Единоцарственное и Сопрестольное, вопию Ти песнь великую, в вышних трегубо песнословимую.


\itshape Нераздельным по существу, неслиянным в Лицах богословски исповедую Тебя, Троичное Единое Божество, Соцарственное и Сопрестольное; возглашаю Тебе великую песнь, в небесных обителях троекратно воспеваемую. Ис. 6:1–3\normalfont{}


И ныне и присно и во веки веков. Аминь.


И раждаеши, и девствуеши, и пребываеши обоюду естеством Дева, Рождейся обновляет законы естества, утроба же раждает нераждающая. Бог идеже хощет, побеждается естества чин: творит бо, елика хощет.


\itshape И рождаешь Ты, и остаешься Девою, в обоих случаях сохраняя по естеству девство. Рожденный Тобою обновляет законы природы, а девственное чрево рождает; когда хочет Бог, то нарушается порядок природы, ибо Он творит, что хочет.\normalfont{}





\bfseries Песнь 5\normalfont{}


\itshape Ирмо́с:\normalfont{}


От нощи утренююща, Человеколюбче, просвети, молюся, и настави и мене на повеления Твоя, и научи мя, Спасе, творити волю Твою.


\itshape От ночи бодрствующего, просвети меня, молю, Человеколюбец, путеводи меня в повелениях Твоих и научи меня, Спаситель, исполнять Твою волю. Пс. 62:2; Пс. 118:35\normalfont{}


\itshape Припев:\normalfont{} Помилуй мя, Боже, помилуй мя.


Моисеов слышала еси ковчежец, душе, водами, волнами носим речными, яко в чертозе древле бегающий дела, горькаго совета фараонитска.


\itshape Ты слышала, душа, о корзинке с Моисеем, в древности носимом водами в волнах реки, как в чертоге, избегшем горестного последствия замысла фараонова. Исх. 2:3\normalfont{}


\itshape Припев:\normalfont{} Помилуй мя, Боже, помилуй мя.


Аще бабы слышала еси, убивающия иногда безвозрастное мужеское, душе окаянная, целомудрия деяние, ныне, яко великий Моисей, сси премудрость.


\itshape Если ты слышала, несчастная душа, о повивальных бабках, некогда умерщвлявших новорожденных младенцев мужского пола, то теперь, подобно Моисею, млекопитайся мудростью. Исх. 1:8–22\normalfont{}


\itshape Припев:\normalfont{} Помилуй мя, Боже, помилуй мя.


Яко Моисей великий египтянина, ума, уязвивши, окаянная, не убила еси, душе; и како вселишися, глаголи, в пустыню страстей покаянием.


\itshape Подобно великому Моисею, поразившему египтянина, ты, не умертвила, несчастная душа, гордого ума; как же, скажи, вселишься ты в пустыню от страстей через покаяние? Исх. 2:11–12\normalfont{}


\itshape Припев:\normalfont{} Помилуй мя, Боже, помилуй мя.


В пустыню вселися великий Моисей; гряди убо, подражай того житие, да и в купине Богоявления, душе, в видении будеши.


\itshape Великий Моисей поселился в пустыне; иди и ты, душа, подражай его жизни, чтобы и тебе увидеть в терновом кусте явление Бога. Исх. 3:2–3\normalfont{}


\itshape Припев:\normalfont{} Помилуй мя, Боже, помилуй мя.


Моисеов жезл воображай, душе, ударяющий море и огустевающий глубину, во образ Креста Божественнаго: имже можеши и ты великая совершити.


\itshape Изобрази, душа, Моисеев жезл, поражающий море и огустевающий глубину, в знамение Божественного Креста, которым и ты можешь совершить великое. Исх. 14:21–22\normalfont{}


\itshape Припев:\normalfont{} Помилуй мя, Боже, помилуй мя.


Аарон приношаше огнь Богу непорочный, нелестный; но Офни и Финеес, яко ты, душе, приношаху чуждее Богу, оскверненное житие.


\itshape Аарон приносил Богу огонь чистый, беспримесный, но Офни и Финеес принесли, как ты, душа, отчужденную от Бога нечистую жизнь. 1 Цар. 2:12–13\normalfont{}


Слава Отцу и Сыну и Святому Духу.


Тя, Тро́ице, славим Единаго Бога: Свят, Свят, Свят еси, Отче, Сыне и Душе, Про́стое Существо, Еди́нице присно покланяемая.


\itshape Тебя, Пресвятая Троица, прославляем за Единого Бога: Свят, Свят, Свят Отец, Сын и Дух, Простое Существо, Единица вечно поклоняемая.\normalfont{}


И ныне и присно и во веки веков. Аминь.


Из Тебе облечеся в мое смешение, нетленная, безмужная Мати Дево, Бог, создавый веки, и соедини Себе человеческое естество.


\itshape В Тебе, Нетленная, не познавшая мужа Матерь-Дево, облекся в мой состав сотворивший мир Бог и соединил с Собою человеческую природу.\normalfont{}





\bfseries Песнь 6\normalfont{}


\itshape Ирмо́с:\normalfont{}


Возопих всем сердцем моим к щедрому Богу, и услыша мя от ада преисподняго, и возведе от тли живот мой.


\itshape От всего сердца моего я воззвал к милосердному Богу, и Он услышал меня из ада преисподнего и воззвал жизнь мою от погибели.\normalfont{}


\itshape Припев:\normalfont{} Помилуй мя, Боже, помилуй мя.


Волны, Спасе, прегрешений моих, яко в мори Чермнем возвращающеся, покрыша мя внезапу, яко египтяны иногда и тристаты.


\itshape Волны грехов моих, Спаситель, обратившись, как в Чермном море, внезапно покрыли меня, как некогда египтян и их всадников. Исх. 14:26–28; Исх. 15:4–5\normalfont{}


\itshape Припев:\normalfont{} Помилуй мя, Боже, помилуй мя.


Неразумное, душе, произволение имела еси, яко прежде Израиль: Божественныя бо манны предсудила еси безсловесно любосластное страстей объядение.


\itshape Нерассудителен твой выбор, душа, как у древнего Израиля, ибо ты безрассудно предпочла Божественной манне сластолюбивое пресыщение страстями. Чис. 21:5\normalfont{}


\itshape Припев:\normalfont{} Помилуй мя, Боже, помилуй мя.


Кладенцы, душе, предпочла еси хананейских мыслей паче жилы камене, из негоже премудрости река, яко чаша, проливает токи богословия.


\itshape Колодцы хананейских помыслов ты, душа, предпочла камню с источником, из которого река премудрости, как чаша, изливает струи богословия. Быт. 21:25; Исх. 17:6\normalfont{}


\itshape Припев:\normalfont{} Помилуй мя, Боже, помилуй мя.


Свиная мяса и котлы и египетскую пищу, паче Небесныя, предсудила еси, душе моя, якоже древле неразумнии людие в пустыни.


\itshape Свиное мясо, котлы и египетскую пищу ты предпочла пище небесной, душа моя, как древний безрассудный народ в пустыне. Исх. 16:3\normalfont{}


\itshape Припев:\normalfont{} Помилуй мя, Боже, помилуй мя.


Яко удари Моисей, раб Твой, жезлом камень, образно животворивая ребра Твоя прообразоваше, из нихже вси питие жизни, Спасе, почерпаем.


\itshape Как Моисей, раб Твой, ударив жезлом о камень, таинственно предызобразил животворное ребро Твое, Спаситель, из которого все мы почерпаем питие жизни.\normalfont{}


\itshape Припев:\normalfont{} Помилуй мя, Боже, помилуй мя.


Испытай, душе, и смотряй, якоже Иисус Навин, обетования землю, какова есть, и вселися в ню благозаконием.


\itshape Исследуй, душа, подобно Иисусу Навину, и обозри обещанную землю, какова она, и поселись в ней путем исполнения закона.\normalfont{}


Слава Отцу и Сыну и Святому Духу.


Тро́ица есмь Про́ста, Нераздельна, раздельна Личне и Еди́ница есмь естеством соединена, Отец глаголет, и Сын, и Божественный Дух.


\itshape Я "--- Троица Простая, Нераздельная, раздельная в Лицах и Единица, соединенная по существу; свидетельствует Отец, Сын и Божественный Дух.\normalfont{}


И ныне и присно и во веки веков. Аминь.


Утроба Твоя Бога нам роди, воображена по нам: Егоже, яко Создателя всех, моли, Богородице, да молитвами Твоими оправдимся.


\itshape Чрево Твое родило нам Бога, принявшего наш образ; Его, как Создателя всего мира, моли, Богородица, чтобы по молитвам Твоим нам оправдаться.\normalfont{}


Господи, помилуй. \itshape Трижды.\normalfont{}


Слава Отцу и Сыну и Святому Духу. И ныне и присно и во веки веков. Аминь.





\bfseries Кондак, глас 6:\normalfont{}


Душе моя, душе моя, востани, что спиши? конец приближается, и имаши смутитися: воспряни убо, да пощадит тя Христос Бог, везде сый и вся исполняяй.


\itshape Душа моя, душа моя, восстань, что ты спишь? Конец приближается, и ты смутишься; пробудись же, чтобы пощадил тебя Христос Бог, Вездесущий и все наполняющий.\normalfont{}





\bfseries Песнь 7\normalfont{}


\itshape Ирмо́с:\normalfont{}


Согрешихом, беззаконновахом, неправдовахом пред Тобою, ниже соблюдохом, ниже сотворихом, якоже заповедал еси нам; но не предаждь нас до конца, отцев Боже.


\itshape Мы согрешили, жили беззаконно, неправо поступали пред Тобою, не сохранили, не исполнили, что Ты заповедал нам; но не оставь нас до конца, Боже отцов. Дан. 9:5–6\normalfont{}


\itshape Припев:\normalfont{} Помилуй мя, Боже, помилуй мя.


Кивот яко ношашеся на колеснице, Зан оный, егда превращшуся тельцу, точию коснуся, Божиим искусися гневом; но того дерзновения убежавши, душе, почитай Божественная честне.


\itshape Когда ковчег везли на колеснице, то Оза, когда вол свернул в сторону, лишь только прикоснулся, испытал на себе гнев Божий, но, душа, избегая его дерзости, благоговейно почитай Божественное. 2 Цар. 6:6–7\normalfont{}


\itshape Припев:\normalfont{} Помилуй мя, Боже, помилуй мя.


Слышала еси Авессалома, како на естество воста, познала еси того скверная деяния, имиже оскверни ложе Давида отца; но ты подражала еси того страстная и любосластная стремления.


\itshape Ты слышала об Авессаломе, как он восстал на самую природу, знаешь гнусные его деяния, которыми он обесчестил ложе отца "--- Давида; но ты сама подражала его страстным и сластолюбивым порывам. 2 Цар. 15:1–37; 2 Цар. 16:21\normalfont{}


\itshape Припев:\normalfont{} Помилуй мя, Боже, помилуй мя.


Покорила еси неработное твое достоинство телу твоему, иного бо Ахитофела обретше врага, душе, снизшла еси сего советом; но сия разсыпа Сам Христос, да ты всяко спасешися.


\itshape Свободное свое достоинство ты, душа, подчинила своему телу, ибо, нашедши другого Ахитофела-врага, ты склонилась на его советы, но их рассеял Сам Христос, чтобы ты спасена была. 2 Цар. 16:20–21\normalfont{}


\itshape Припев:\normalfont{} Помилуй мя, Боже, помилуй мя.


Соломон чудный и благодати премудрости исполненный, сей лукавое иногда пред Богом сотворив, отступи от Него; емуже ты проклятым твоим житием, душе, уподобилася еси.


\itshape Чудный Соломон, будучи преисполнен дара премудрости, некогда, сотворив злое пред Богом, отступил от Него; ему ты уподобилась, душа, своей жизнью, достойной проклятия. 3 Цар. 3:12; 3 Цар. 11:4–6\normalfont{}


\itshape Припев:\normalfont{} Помилуй мя, Боже, помилуй мя.


Сластьми влеком страстей своих, оскверняшеся, увы мне, рачитель премудрости, рачитель блудных жен и странен от Бога: егоже ты подражала еси умом, о душе, сладострастьми скверными.


\itshape Увлекшись сластолюбивыми страстями, осквернился, увы, ревнитель премудрости, возлюбив нечестивых женщин и отчуждившись от Бога; ему, душа, ты сама подражала в уме постыдным сладострастием. 3 Цар. 11:6–8\normalfont{}


\itshape Припев:\normalfont{} Помилуй мя, Боже, помилуй мя.


Ровоаму поревновала еси, не послушавшему совета отча, купно же и злейшему рабу Иеровоаму, прежнему отступнику, душе, но бегай подражания и зови Богу: согреших, ущедри мя.


\itshape Ты поревновала, душа, Ровоаму, не послушавшему совета отеческого, и вместе злейшему рабу Иеровоаму, древнему мятежнику; избегай подражание им и взывай к Богу: согрешила я, умилосердись надо мною. 3 Цар. 12:13–14, 3 Цар. 12:20\normalfont{}


Слава Отцу и Сыну и Святому Духу.


Тро́ице Про́стая, Нераздельная, Единосущная и Естество Едино, Светове и Свет, и Свята Три, и Едино Свято поется Бог Тро́ица; но воспой, прослави Живот и Животы, душе, всех Бога.


\itshape Троица Простая, Нераздельная, Единосущная, Единая Естеством, Светы и Свет и Три Святы и Едино (Лицо) Свято, Бог-Троица, воспеваемая в песнопениях; воспой же и ты, душа, прославь Жизнь и Жизни "--- Бога всех.\normalfont{}


И ныне и присно и во веки веков. Аминь.


Поем Тя, благословим Тя, покланяемся Ти, Богородительнице, яко Неразлучныя Тро́ицы породила еси Единаго Христа Бога и Сама отверзла еси нам, сущим на земли, Небесная.


\itshape Воспеваем Тебя, благословляем Тебя, поклоняемся Тебе, Богородительница, ибо Ты родила Единого из Нераздельной Троицы Христа Бога и Сама открыла для нас, живущих на земле, небесные обители.\normalfont{}





\bfseries Песнь 8\normalfont{}


\itshape Ирмо́с:\normalfont{}


Егоже воинства Небесная славят, и трепещут херувими и серафими, всяко дыхание и тварь, пойте, благословите и превозносите во вся веки.


\itshape Кого прославляют воинства небесные и пред Кем трепещут Херувимы и Серафимы, Того, все существа и творения, воспевайте, благословляйте и превозносите во все века.\normalfont{}


\itshape Припев:\normalfont{} Помилуй мя, Боже, помилуй мя.


Ты Озии, душе, поревновавши, сего прокажение в себе стяжала еси сугубо: безместная бо мыслиши, беззаконная же дееши; остави, яже имаши, и притецы к покаянию.


\itshape Соревновав Озии, душа, ты получила себе вдвойне его проказу, ибо помышляешь недолжное и делаешь беззаконное; оставь, что у тебя есть и приступи к покаянию. 4 Цар. 15:5; 2 Пар. 26:19\normalfont{}


\itshape Припев:\normalfont{} Помилуй мя, Боже, помилуй мя.


Ниневитяны, душе, слышала еси кающияся Богу, вретищем и пепелом, сих не подражала еси, но явилася еси злейшая всех, прежде закона и по законе прегрешивших.


\itshape Ты слышала, душа, о ниневитянах, в рубище и пепле каявшихся Богу; им ты не подражала, но оказалась упорнейшею всех, согрешивших до закона и после закона. Иона 3:5\normalfont{}


\itshape Припев:\normalfont{} Помилуй мя, Боже, помилуй мя.


В рове блата слышала еси Иеремию, душе, града Сионя рыданьми вопиюща и слез ищуща: подражай сего плачевное житие и спасешися.


\itshape Ты слышала, душа, как Иеремия, в нечистом рве с рыданиями взывал к городу Сиону и искал слез; подражай плачевной его жизни и спасешься. Иер. 38:6\normalfont{}


\itshape Припев:\normalfont{} Помилуй мя, Боже, помилуй мя.


Иона в Фарсис побеже, проразумев обращение ниневитянов, разуме бо, яко пророк, Божие благоутробие: темже ревноваше пророчеству не солгатися.


\itshape Иона побежал в Фарсис, предвидя обращение ниневитян, ибо он, как пророк, знал милосердие Божие и вместе ревновал, чтобы пророчество не оказалось ложным. Иона 1:3\normalfont{}


\itshape Припев:\normalfont{} Помилуй мя, Боже, помилуй мя.


Даниила в рове слышала еси, како загради уста, о душе, зверей; уведела еси, како отроцы, иже о Азарии, погасиша верою пещи пламень горящий.


\itshape Ты слышала, душа, как Даниил во рве заградил уста зверей; ты узнала, как юноши, бывшие с Азариею, верою угасили разожженный пламень печи. Дан. 14:31; Дан. 3:24\normalfont{}


\itshape Припев:\normalfont{} Помилуй мя, Боже, помилуй мя.


Ветхаго Завета вся приведох ти, душе, к подобию; подражай праведных боголюбивая деяния, избегни же паки лукавых грехов.


\itshape Из Ветхого Завета всех я привел тебе в пример, душа; подражай богоугодным деяниям праведных, и избегай грехов людей порочных.\normalfont{}


Слава Отцу и Сыну и Святому Духу.


Безначальне Отче, Сыне Собезначальне, Утешителю Благий, Душе Правый, Слова Божия Родителю, Отца Безначальна Слове, Душе Живый и Зиждяй, Тро́ице Еди́нице, помилуй мя.


\itshape Безначальный Отче, Собезначальный Сын, Утешитель благий, Дух правды, Родитель Бога Слова, Слово Безначальное Отца, Дух Животворящий и Созидающий, Троица Единая, помилуй меня.\normalfont{}


И ныне и присно и во веки веков. Аминь.


Яко от оброщения червленицы, Пречистая, умная багряница Еммануилева внутрь во чреве Твоем плоть исткася. Темже Богородицу воистинну Тя почитаем.


\itshape Мысленная порфира "--- плоть Еммануила соткалась внутри Твоего чрева, Пречистая, как бы из вещества пурпурного; потому мы почитаем Тебя, Истинную Богородицу.\normalfont{}





\bfseries Песнь 9\normalfont{}


\itshape Ирмо́с:\normalfont{}


Безсеменнаго зачатия Рождество несказанное, Матере безмужныя нетленен Плод, Божие бо Рождение обновляет естества. Темже Тя вси роди, яко Богоневестную Матерь, православно величаем.


\itshape Рождество от бессеменного зачатия неизъяснимо, безмужной Матери нетленен Плод, ибо рождение Бога обновляет природу. Поэтому Тебя, как Богоневесту-Матерь мы, все роды, православно величаем.\normalfont{}


\itshape Припев:\normalfont{} Помилуй мя, Боже, помилуй мя.


Христос искушашеся, диавол искушаше, показуя камение, да хлеби будут, на гору возведе видети вся царствия мира во мгновении; убойся, о душе, ловления, трезвися, молися на всякий час Богу.


\itshape Христос был искушаем; диавол искушал, показывая камни, чтобы они обратились в хлебы; возвел Его на гору, чтобы видеть все царства мира в одно мгновение; бойся, душа, этого обольщения, бодрствуй и ежечасно молись Богу. Мф. 4:1–9; Мк. 1; 12–13; Лк. 4:1–12\normalfont{}


\itshape Припев:\normalfont{} Помилуй мя, Боже, помилуй мя.


Горлица пустыннолюбная, глас вопиющаго возгласи, Христов светильник, проповедуяй покаяние, Ирод беззаконнова со Иродиадою. Зри, душе моя, да не увязнеши в беззаконныя сети, но облобызай покаяние.


\itshape Пустыннолюбивая горлица, голос вопиющего, Христов светильник взывал, проповедуя покаяние, а Ирод беззаконствовал с Иродиадою; смотри, душа моя, чтобы не впасть тебе в сети беззаконных, но возлюби покаяние. Песн. 2:12; Ис. 40:3; Мф. 3:8; Мф. 14:3; Мк. 6:17; Лк. 3:19–20\normalfont{}


\itshape Припев:\normalfont{} Помилуй мя, Боже, помилуй мя.


В пустыню вселися благодати Предтеча, и Иудея вся и Самария, слышавше, течаху и исповедаху грехи своя, крещающеся усердно: ихже ты не подражала еси, душе.


\itshape Предтеча благодати обитал в пустыне и все иудеи и самаряне стекались слушать его и исповедовали грехи свои, с усердием принимая крещение. Но ты, душа, не подражала им. Мф. 3:1–6; Мк. 1:3–6\normalfont{}


\itshape Припев:\normalfont{} Помилуй мя, Боже, помилуй мя.


Брак убо честный и ложе нескверно, обоя бо Христос прежде благослови, плотию ядый и в Кане же на браце воду в вино совершая, и показуя первое чудо, да ты изменишися, о душе.


\itshape Брак честен и ложе непорочно, ибо Христос благословил их некогда, в Кане на браке вкушая пищу плотию и претворяя воду в вино, совершая первое чудо, чтобы ты, душа, изменилась. Евр. 13:4; Ин. 2:1–11\normalfont{}


\itshape Припев:\normalfont{} Помилуй мя, Боже, помилуй мя.


Разслабленнаго стягну Христос, одр вземша, и юношу умерша воздвиже, вдовиче рождение, и сотнича отрока, и самаряныне явися, в дусе службу тебе, душе, предживописа.


\itshape Христос укрепил расслабленного, взявшего постель свою; воскресил умершего юного сына вдовы, исцелил слугу сотника и, открыв Себя самарянке, предначертал тебе, душа, служение Богу духом. Мф. 9:6; Мф. 8:13; Лк. 7:14; Ин. 4:7–24\normalfont{}


\itshape Припев:\normalfont{} Помилуй мя, Боже, помилуй мя.


Кровоточивую исцели прикосновением края ризна Господь, прокаженныя очисти, слепыя и хромыя просветив, исправи, глухия же, и немыя, и ничащия низу исцели словом: да ты спасешися, окаянная душе.


\itshape Господь исцелил кровоточивую через прикосновение к одежде Его, очистил прокаженных, дал прозрение слепым, исправил хромых, глухих, немых и уврачевал словом скорченную, чтобы ты спаслась, несчастная душа. Мф. 9:20; Мф. 11:5; Лк. 13:11–13\normalfont{}


Слава Отцу и Сыну и Святому Духу.


Отца прославим, Сына превознесем, Божественному Духу верно поклонимся, Тро́ице Нераздельней, Еди́нице по существу, яко Свету и Светом, и Животу и Животом, Животворящему и Просвещающему концы.


\itshape Прославим Отца, превознесем Сына, с верою поклонимся Божественному Духу, Нераздельной Троице, Единой по существу, как Свету и Светам, Жизни и Жизням, животворящему и просвещающему пределы вселенной.\normalfont{}


И ныне и присно и во веки веков. Аминь.


Град Твой сохраняй, Богородительнице Пречистая, в Тебе бо сей верно царствуяй, в Тебе и утверждается, и Тобою побеждаяй, побеждает всякое искушение, и пленяет ратники, и проходит послушание.


\itshape Сохраняй град Свой, Пречистая Богородительница. Под Твоею защитою он царствует с верою, и от Тебя получает крепость, и при Твоем содействии неотразимо побеждает всякое бедствие, берет в плен врагов и держит их в подчинении.\normalfont{}


\itshape Припев:\normalfont{} Преподобне отче Андрее, моли Бога о нас.


Андрее честный и отче треблаженнейший, пастырю Критский, не престай моляся о воспевающих тя, да избавимся вси гнева, и скорби, и тления, и прегрешений безмерных, чтущии твою память верно.


\itshape Андрей досточтимый, отец преблаженный, пастырь Критский, не переставай молиться за воспевающих тебя, чтобы избавиться от гнева, скорби, погибели и бесчисленных прегрешений нам всем, искренно почитающим память твою.\normalfont{}


\itshape Таже оба лика вкупе поют Ирмо́с:\normalfont{}


Безсеменнаго зачатия Рождество несказанное, Матере безмужныя нетленен Плод, Божие бо Рождение обновляет естества. Темже Тя вси роди, яко Богоневестную Матерь, православно величаем.


\itshape Рождество от бессеменного зачатия неизъяснимо, безмужной Матери нетленен Плод, ибо рождение Бога обновляет природу. Поэтому Тебя, как Богоневесту-Матерь мы, все роды, православно величаем.\normalfont{}

\mychapterending

\mychapter{В среду первой седмицы Великого Поста}
%http://www.molitvoslov.org/text573.htm 
 






\bfseries Песнь 1\normalfont{}


\itshape Ирмо́с:\normalfont{}


Помощник и Покровитель бысть мне во спасение, Сей мой Бог, и прославлю Его, Бог отца моего, и вознесу Его: славно бо прославися.


\itshape Помощник и Покровитель явился мне ко спасению, Он Бог мой, и прославлю Его, Бога отца моего, и превознесу Его, ибо Он торжественно прославился. Исх. 15:1–2\normalfont{}


\itshape Припев:\normalfont{} Помилуй мя, Боже, помилуй мя.


От юности, Христе, заповеди Твоя преступих, всестрастно небрегий, унынием преидох житие. Темже зову Ти, Спасе: поне на конец спаси мя.


\itshape С юности, Христе, я пренебрегал Твоими заповедями, всю жизнь провел в страстях, беспечности и нерадении. Поэтому и взываю к Тебе, Спаситель: хотя при кончине спаси меня.\normalfont{}


\itshape Припев:\normalfont{} Помилуй мя, Боже, помилуй мя.


Повержена мя, Спасе, пред враты Твоими, поне на старость не отрини мене во ад тща, но прежде конца, яко Человеколюбец, даждь ми прегрешений оставление.


\itshape Поверженного пред вратами Твоими, Спаситель, хотя в старости, не низринь меня в ад, как невоздержанного, но прежде кончины, как Человеколюбец, даруй мне оставление прегрешений.\normalfont{}


\itshape Припев:\normalfont{} Помилуй мя, Боже, помилуй мя.


Богатство мое, Спасе, изнурив в блуде, пуст есмь плодов благочестивых, алчен же зову: Отче щедрот, предварив, Ты мя ущедри.


\itshape Расточив богатство мое в распутстве, Спаситель, я чужд плодов благочестия, но, чувствуя голод, взываю: Отец Милосердный, поспеши и умилосердись надо мною.\normalfont{}


\itshape Припев:\normalfont{} Помилуй мя, Боже, помилуй мя.


В разбойники впадый аз есмь помышленьми моими, весь от них уязвихся ныне и исполнихся ран, но, Сам ми представ, Христе Спасе, исцели.


\itshape По помыслам моим я человек, попавшийся разбойникам; теперь я весь изранен ими, покрыт язвами, но Ты Сам, Христос Спаситель, приди и исцели меня. Лк. 10:30\normalfont{}


\itshape Припев:\normalfont{} Помилуй мя, Боже, помилуй мя.


Священник, мя предвидев, мимо иде, и левит, видя в лютых нага, презре, но из Марии возсиявый Иисусе, Ты, представ, ущедри мя.


\itshape Священник, заметив меня, прошел мимо, и левит, видя меня в беде обнаженного, презрел; но Ты, воссиявший от Марии Иисусе, прииди и умилосердись надо мною. Лк. 10:31–32\normalfont{}


\itshape Припев:\normalfont{} Преподобная мати Марие, моли Бога о нас.


Ты ми даждь светозарную благодать от Божественнаго свыше промышления избежати страстей омрачения и пети усердно Твоего, Марие, жития красная исправления.


\itshape Даруй мне, Мария, ниспосланную тебе свыше Божественным Промыслом светозарную благодать "--- избежать мрака страстей и усердно воспеть прекрасные подвиги твоей жизни.\normalfont{}


Слава Отцу и Сыну и Святому Духу.


Пресущная Тро́ице, во Еди́нице покланяемая, возьми бремя от мене тяжкое греховное и, яко благоутробна, даждь ми слезы умиления.


\itshape Пресущественная Троица, Которой мы поклоняемся как Единому Существу, сними с меня тяжелое бремя греховное и, как Милосердная, даруй мне слезы умиления.\normalfont{}


И ныне и присно и во веки веков. Аминь.


Богородице, Надежде и Предстательство Тебе поющих, возьми бремя от мене тяжкое греховное и, яко Владычица Чистая, кающася приими мя.


\itshape Богородице, Надежда и Помощь всем воспевающих Тебя, сними с меня тяжелое бремя греховное и, как Владычица Непорочная, прими меня кающегося.\normalfont{}





\bfseries Песнь 2\normalfont{}


\itshape Ирмо́с:\normalfont{}


Вонми, Небо, и возглаголю, и воспою Христа, от Девы плотию пришедшаго.


\itshape Внемли, небо, я буду возвещать и воспевать Христа, пришедшего во плоти от Девы.\normalfont{}


\itshape Припев:\normalfont{} Помилуй мя, Боже, помилуй мя.


Поползохся, яко Давид, блудно и осквернихся, но омый и мене, Спасе, слезами.


\itshape От невоздержания, как Давид, я пал и осквернился, но омой и меня, Спаситель, слезами. 2 Цар. 11:4\normalfont{}


\itshape Припев:\normalfont{} Помилуй мя, Боже, помилуй мя.


Ни слез, ниже покаяния имам, ниже умиления. Сам ми сия, Спасе, яко Бог, даруй.


\itshape Ни слез, ни покаяния, ни умиления нет у меня; Сам Ты, Спаситель, как Бог, даруй мне это.\normalfont{}


\itshape Припев:\normalfont{} Помилуй мя, Боже, помилуй мя.


Погубих первозданную доброту и благолепие мое и ныне лежу наг и стыждуся.


\itshape Погубил я первозданную красоту и благообразие мое и теперь лежу обнаженным и стыжусь.\normalfont{}


\itshape Припев:\normalfont{} Помилуй мя, Боже, помилуй мя.


Дверь Твою не затвори мне тогда, Господи, Господи, но отверзи ми сию, кающемуся Тебе.


\itshape Не затвори предо мною теперь дверь Твою, Господи, Господи, но отвори ее для меня, кающегося Тебе. Мф. 7:21–23; Мф. 25:11\normalfont{}


\itshape Припев:\normalfont{} Помилуй мя, Боже, помилуй мя.


Внуши воздыхания души моея и очию моею приими капли, Спасе, и спаси мя.


\itshape Внемли, Спаситель, стенаниям души моей, прими слезы очей моих и спаси меня.\normalfont{}


\itshape Припев:\normalfont{} Помилуй мя, Боже, помилуй мя.


Человеколюбче, хотяй всем спастися, Ты воззови мя и приими, яко благ, кающагося.


\itshape Человеколюбец, желающий всем спасения, Ты призови меня и прими, как Благий, кающегося. 1 Тим. 2:4\normalfont{}


\itshape Припев:\normalfont{} Пресвятая Богородице, спаси нас.


Пречистая Богородице Дево, Едина Всепетая, моли прилежно, во еже спастися нам.


\itshape Пречистая Богородице Дева, Ты Одна, всеми воспеваемая, усердно моли о нашем спасении.\normalfont{}


\itshape Иный Ирмо́с:\normalfont{}


Видите, видите, яко Аз есмь Бог, манну одождивый и воду из камене источивый древле в пустыни людем Моим, десницею единою и крепостию Моею.


\itshape Видите, видите, что Я "--- Бог, в древности ниспославший манну и источивший воду из камня народу Моему в пустыне "--- одним Своим всемогуществом. Исх. 16:14; Исх. 17:6\normalfont{}


\itshape Припев:\normalfont{} Помилуй мя, Боже, помилуй мя.


Видите, видите, яко Аз есмь Бог, внушай, душе моя, Господа вопиюща, и удалися прежняго греха, и бойся, яко неумытнаго и яко Судии и Бога.


\itshape Видите, видите, что Я "--- Бог. Внимай, душа моя, взывающему Господу, оставь прежний грех и убойся как праведного Судию и Бога.\normalfont{}


\itshape Припев:\normalfont{} Помилуй мя, Боже, помилуй мя.


Кому уподобилася еси, многогрешная душе? токмо первому Каину и Ламеху оному, каменовавшая тело злодействы и убившая ум безсловесными стремленьми.


\itshape Кому уподобилась ты, многогрешная душа, как не первому Каину и тому Ламеху, жестоко окаменив тело злодеяниями и убив ум безрассудными стремлениями. Быт. 4:1–26\normalfont{}


\itshape Припев:\normalfont{} Помилуй мя, Боже, помилуй мя.


Вся прежде закона претекши, о душе, Сифу не уподобилася еси, ни Еноса подражала еси, ни Еноха преложением, ни Ноя, но явилася еси убога праведных жизни.


\itshape Имея в виду всех, живших до закона, о душа, не уподобилась ты Сифу, не подражала ни Еносу, ни Еноху через преселение духовное, ни Ною, но оказалась чуждой жизни праведников. Быт. 5:1–32\normalfont{}


\itshape Припев:\normalfont{} Помилуй мя, Боже, помилуй мя.


Едина отверзла еси хляби гнева Бога Твоего, душе моя, и потопила еси всю, якоже землю, плоть, и деяния, и житие, и пребыла еси вне спасительнаго ковчега.


\itshape Ты одна, душа моя, открыла бездны гнева Бога своего и потопила, как землю, всю плоть, и дела, и жизнь, и осталась вне спасительного ковчега. Быт. 7:1–24\normalfont{}


\itshape Припев:\normalfont{} Преподобная мати Марие, моли Бога о нас.


Всем усердием и любовию притекла еси Христу, первый греха путь отвращши, и в пустынях непроходимых питающися, и Того чисте совершающи Божественныя заповеди.


\itshape Оставив прежний путь греха, ты с всем усердием и любовью прибегла ко Христу, живя в непроходимых пустынях и в чистоте исполняя Божественные Его заповеди.\normalfont{}


Слава Отцу и Сыну и Святому Духу.


Безначальная, Несозданная Тро́ице, Нераздельная Еди́нице, кающася мя приими, согрешивша спаси, Твое есмь создание, не презри, но пощади и избави огненнаго мя осуждения.


\itshape Безначальная Несозданная Троица, Нераздельная Единица, прими меня кающегося, спаси согрешившего, я "--- Твое создание, не презри, но пощади и избавь меня от осуждения в огонь.\normalfont{}


И ныне и присно и во веки веков. Аминь.


Пречистая Владычице, Богородительнице, Надеждо к Тебе притекающих и пристанище сущих в бури, Милостиваго и Создателя и Сына Твоего умилостиви и мне молитвами Твоими.


\itshape Пречистая Владычица, Богородительница, Надежда прибегающих к Тебе и пристанище для застигнутых бурей, Твоими молитвами приклони на милость и ко мне Милостивого Творца и Сына Твоего.\normalfont{}





\bfseries Песнь 3\normalfont{}


\itshape Ирмо́с:\normalfont{}


Утверди, Господи, на камени заповедей Твоих подвигшееся сердце мое, яко Един Свят еси и Господь.


\itshape Утверди, Господи, на камне Твоих заповедей поколебавшееся сердце мое, ибо Ты один Свят и Господь.\normalfont{}


\itshape Припев:\normalfont{} Помилуй мя, Боже, помилуй мя.


Благословения Симова не наследовала еси, душе окаянная, ни пространное одержание, якоже Иафеф, имела еси на земли оставления.


\itshape Симова благословения не наследовала ты, несчастная душа, и не получила, подобно Иафету, обширного владения на земле "--- отпущения грехов.\normalfont{}


\itshape Припев:\normalfont{} Помилуй мя, Боже, помилуй мя.


От земли Харран изыди от греха, душе моя, гряди в землю, точащую присноживотное нетление, еже Авраам наследствова.


\itshape Удались, душа моя, от земли Харран "--- от греха; иди в землю, источающую вечно живое нетление, которую наследовал Авраам. Быт. 12:1–7\normalfont{}


\itshape Припев:\normalfont{} Помилуй мя, Боже, помилуй мя.


Авраама слышала еси, душе моя, древле оставльша землю отечества и бывша пришельца, сего произволению подражай.


\itshape Ты слышала, душа моя, как в древности Авраам оставил землю отеческую и сделался странником; подражай его решимости. Быт. 12:1–7\normalfont{}


\itshape Припев:\normalfont{} Помилуй мя, Боже, помилуй мя.


У дуба Мамврийскаго учредив патриарх ангелы, наследствова по старости обетования ловитву.


\itshape Угостив Ангелов под дубом Маврийским, патриарх на старости получил, как добычу, обещанное. Быт. 18:1\normalfont{}


\itshape Припев:\normalfont{} Помилуй мя, Боже, помилуй мя.


Исаака, окаянная душе моя, разумевши новую жертву, тайно всесожженную Господеви, подражай его произволению.


\itshape Зная, бедная душа моя, как Исаак принесен таинственно в новую жертву всесожжения Господу, подражай его решимости. Быт. 22:2\normalfont{}


\itshape Припев:\normalfont{} Помилуй мя, Боже, помилуй мя.


Исмаила слышала еси, трезвися, душе моя, изгнана, яко рабынино отрождение, виждь, да не како подобно что постраждеши, ласкосердствующи.


\itshape Ты слышала, душа моя, что Измаил был изгнан, как рожденный рабыней, бодрствуй, смотри, чтобы и тебе не потерпеть бы чего-либо подобного за сладострастие. Быт. 21:10–11\normalfont{}


\itshape Припев:\normalfont{} Преподобная мати Марие, моли Бога о нас.


Содержим есмь бурею и треволнением согрешений, но сама мя, мати, ныне спаси и к пристанищу Божественнаго покаяния возведи.


\itshape Окружен я, матерь, бурей и сильным волнением согрешений, но ты сама ныне спаси меня и приведи к пристанищу Божественного покаяния.\normalfont{}


\itshape Припев:\normalfont{} Преподобная мати Марие, моли Бога о нас.


Рабское моление и ныне, преподобная, принесши ко благоутробней молитвами твоими Богородице, отверзи ми Божественныя входы.


\itshape Усердное моление и ныне, преподобная, принеся к умилостивленной твоими молитвами Богородице, открой и для меня Божественные входы.\normalfont{}


Слава Отцу и Сыну и Святому Духу.


Тро́ице Про́стая, Несозданная, Безначальное Естество, в Тро́ице певаемая Ипостасей, спаси ны, верою покланяющияся державе Твоей.


\itshape Троица Несоставная, Несозданная, Существо Безначальная, в троичности Лиц воспеваемая, спаси нас, с верою поклоняющихся силе Твоей.\normalfont{}


И ныне и присно и во веки веков. Аминь.


От Отца безлетна Сына в лето, Богородительнице, неискусомужно родила еси, странное чудо, пребывши Дева доящи.


\itshape Ты, Богородительница, не испытавши мужа, во времени родила Сына от Отца вне времени и "--- дивное чудо: питая молоком, пребыла Девою.\normalfont{}





\bfseries Песнь 4\normalfont{}


\itshape Ирмо́с:\normalfont{}


Услыша пророк пришествие Твое, Господи, и убояся, яко хощеши от Девы родитися и человеком явитися, и глаголаше: услышах слух Твой и убояхся, слава силе Твоей, Господи.


\itshape Услышал пророк о пришествии Твоем, Господи, и устрашился, что Тебе угодно родиться от Девы и явиться людям, и сказал: услышал я весть о Тебе и устрашился; слава силе Твоей, Господи.\normalfont{}


\itshape Припев:\normalfont{} Помилуй мя, Боже, помилуй мя.


Тело осквернися, дух окаляся, весь острупихся, но яко врач, Христе, обоя покаянием моим уврачуй, омый, очисти, покажи, Спасе мой, паче снега чистейша.


\itshape Тело мое осквернено, дух грязен, весь я покрыт струпами, но Ты, Христе, как врач, уврачуй и то и другое моим покаянием, омой, очисти, яви меня чище снега, Спаситель мой.\normalfont{}


\itshape Припев:\normalfont{} Помилуй мя, Боже, помилуй мя.


Тело Твое и кровь, распинаемый о всех, положил еси, Слове: тело убо, да мя обновиши, кровь, да омыеши мя. Дух же предал еси, да мя приведеши, Христе, Твоему Родителю.


\itshape Твое тело и Кровь, Слово, Ты принес в жертву за всех при распятии; Тело "--- чтобы воссоздать меня, Кровь "--- чтобы омыть меня, и Дух Ты, Христе, предал, чтобы привести меня к Твоему Отцу.\normalfont{}


\itshape Припев:\normalfont{} Помилуй мя, Боже, помилуй мя.


Соделал еси спасение посреде земли, Щедре, да спасемся. Волею на древе распялся еси, Едем затворенный отверзеся, горняя и дольняя тварь, языцы вси, спасени, покланяются Тебе.


\itshape Посреди земли Ты устроил спасение, Милосердный, чтобы мы спаслись; Ты добровольно распялся на древе; Едем затворенный открылся; Тебе поклоняются небесные и земные и все спасенные Тобою народы. Пс. 73:12\normalfont{}


\itshape Припев:\normalfont{} Помилуй мя, Боже, помилуй мя.


Да будет ми купель кровь из ребр Твоих, вкупе и питие, источившее воду оставления, да обоюду очищаюся, помазуяся и пия, яко помазание и питие, Слове, животочная Твоя словеса.


\itshape Да будет мне омовением Кровь из ребр Твоих и вместе питием, источившая оставление грехов, чтобы мне и тем и другим очищаться, Слове, помазуясь и напояясь животворными Твоими словами, как мазью и питием. Ин. 19:34\normalfont{}


\itshape Припев:\normalfont{} Помилуй мя, Боже, помилуй мя.


Чашу Церковь стяжа, ребра Твоя живоносная, из нихже сугубыя нам источи токи оставления и разума во образ древняго и новаго, двоих вкупе заветов, Спасе наш.


\itshape Церковь приобрела себе Чашу в живоносном ребре Твоем, из которого проистек нам двойной поток оставления грехов и разумения, Спаситель наш, в образ обоих Заветов, Ветхого и Нового.\normalfont{}


\itshape Припев:\normalfont{} Помилуй мя, Боже, помилуй мя.


Наг есмь чертога, наг есмь и брака, купно и вечери; светильник угасе, яко безъелейный, чертог заключися мне спящу, вечеря снедеся, аз же по руку и ногу связан, вон низвержен есмь.


\itshape Я лишен брачного чертога, лишен и брака, и вечери; светильник, как без елея, погас; чертог закрылся во время моего сна, вечеря окончена, а я, связанный по рукам и ногам, извержен вон. Мф. 25:1–13; Лк. 12:35–37; Лк. 13:24–27; Лк. 14:7–24\normalfont{}


Слава Отцу и Сыну и Святому Духу.


Нераздельное Существом, Неслитное Лицы богословлю Тя, Троическое Едино Божество, яко Единоцарственное и Сопрестольное, вопию Ти песнь великую, в вышних трегубо песнословимую.


\itshape Нераздельным по существу, неслиянным в Лицах богословски исповедую Тебя, Троичное Единое Божество, Соцарственное и Сопрестольное; возглашаю Тебе великую песнь, в небесных обителях троекратно воспеваемую. Ис. 6:1–3\normalfont{}


И ныне и присно и во веки веков. Аминь.


И раждаеши, и девствуеши, и пребываеши обоюду естеством Дева, Рождейся обновляет законы естества, утроба же раждает нераждающая. Бог идеже хощет, побеждается естества чин: творит бо, елика хощет.


\itshape И рождаешь Ты, и остаешься Девою, в обоих случаях сохраняя по естеству девство. Рожденный Тобою обновляет законы природы, а девственное чрево рождает; когда хочет Бог, то нарушается порядок природы, ибо Он творит, что хочет.\normalfont{}





\bfseries Песнь 5\normalfont{}


\itshape Ирмо́с:\normalfont{}


От нощи утренююща, Человеколюбче, просвети, молюся, и настави и мене на повеления Твоя, и научи мя, Спасе, творити волю Твою.


\itshape От ночи бодрствующего, просвети меня, молю, Человеколюбец, путеводи меня в повелениях Твоих и научи меня, Спаситель, исполнять Твою волю. Пс. 62:2; Пс. 118:35\normalfont{}


\itshape Припев:\normalfont{} Помилуй мя, Боже, помилуй мя.


Яко тяжкий нравом, фараону горькому бых, Владыко, Ианни и Иамври, душею и телом, и погружен умом, но помози ми.


\itshape По упорству я стал как жестокий нравом фараон, Владыко, по душе и телу я "--- Ианний и Иамврий, и по уму погрязший, но помоги мне. Исх. 7:11; 2 Тим. 3:8\normalfont{}


\itshape Припев:\normalfont{} Помилуй мя, Боже, помилуй мя.


Калом смесихся, окаянный, умом, омый мя, Владыко, банею моих слез, молю Тя, плоти моея одежду убелив, яко снег.


\itshape Загрязнил я, несчастный, свой ум, но омой меня, Владыко, в купели слез моих молю Тебя, и убели, как снег, одежду плоти моей.\normalfont{}


\itshape Припев:\normalfont{} Помилуй мя, Боже, помилуй мя.


Аще испытаю моя дела, Спасе, всякаго человека превозшедша грехами себе зрю, яко разумом мудрствуяй, согреших не неведением.


\itshape Когда исследую свои дела, Спаситель, то вижу, что превзошел я грехами всех людей, ибо я грешил с разумным сознанием, а не по неведению.\normalfont{}


\itshape Припев:\normalfont{} Помилуй мя, Боже, помилуй мя.


Пощади, пощади, Господи, создание Твое, согреших, ослаби ми, яко естеством чистый Сам сый Един, и ин разве Тебе никтоже есть кроме скверны.


\itshape Пощади, Господи, пощади, создание Твое: я согрешил, прости мне, ибо только Ты один чист по природе, и никто, кроме Тебя, не чужд нечистоты.\normalfont{}


\itshape Припев:\normalfont{} Помилуй мя, Боже, помилуй мя.


Мене ради Бог сый, вообразился еси в мя, показал еси чудеса, исцелив прокаженныя и разслабленнаго стягнув, кровоточивыя ток уставил еси, Спасе, прикосновением риз.


\itshape Ради меня, будучи Богом, Ты принял мой образ, Спаситель, и, совершая чудеса, исцелял прокаженных, укреплял расслабленных, остановил кровотечение у кровоточивой прикосновением одежды. Мф. 9:20; Мк. 5:25–27; Лк. 8:43–44\normalfont{}


\itshape Припев:\normalfont{} Преподобная мати Марие, моли Бога о нас.


Струи Иорданския прешедши, обрела еси покой безболезненный, плоти сласти избежавши, еяже и нас изми твоими молитвами, преподобная.


\itshape Ты перешла поток Иорданский и приобрела покой безболезненный, оставив плотское удовольствие, от которого избавь и нас твоими молитвами, преподобная.\normalfont{}


Слава Отцу и Сыну и Святому Духу.


Тя, Тро́ице, славим Единаго Бога: Свят, Свят, Свят еси, Отче, Сыне и Душе, Про́стое Существо, Еди́нице присно покланяемая.


\itshape Тебя, Пресвятая Троица, прославляем за Единого Бога: Свят, Свят, Свят Отец, Сын и Дух, Простое Существо, Единица вечно поклоняемая.\normalfont{}


И ныне и присно и во веки веков. Аминь.


Из Тебе облечеся в мое смешение, нетленная, безмужная Мати Дево, Бог, создавый веки, и соедини Себе человеческое естество.


\itshape В Тебе, Нетленная, не познавшая мужа Матерь-Дево, облекся в мой состав сотворивший мир Бог и соединил с Собою человеческую природу.\normalfont{}





\bfseries Песнь 6\normalfont{}


\itshape Ирмо́с:\normalfont{}


Возопих всем сердцем моим к щедрому Богу, и услыша мя от ада преисподняго, и возведе от тли живот мой.


\itshape От всего сердца моего я воззвал к милосердному Богу, и Он услышал меня из ада преисподнего и воззвал жизнь мою от погибели.\normalfont{}


\itshape Припев:\normalfont{} Помилуй мя, Боже, помилуй мя.


Востани и побори, яко Иисус Амалика, плотския страсти, и гаваониты, лестныя помыслы, присно побеждающи.


\itshape Восстань и побеждай плотские страсти, как Иисус Амалика, всегда побеждая и гаваонитян "--- обольстительные помыслы. Исх. 17:8; Нав. 8:21\normalfont{}


\itshape Припев:\normalfont{} Помилуй мя, Боже, помилуй мя.


Преиди, времене текущее естество, яко прежде ковчег, и земли оныя буди во одержании обетования, душе, Бог повелевает.


\itshape Душа, Бог повелевает: перейди, как некогда ковчег Иордан, текущее по своему существу время и сделайся обладательницею обещанной земли. Нав. 3:17\normalfont{}


\itshape Припев:\normalfont{} Помилуй мя, Боже, помилуй мя.


Яко спасл еси Петра, возопивша, спаси, предварив мя, Спасе, от зверя избави, простер Твою руку, и возведи из глубины греховныя.


\itshape Подобно тому как Ты спас Петра, воззвавшего, поспеши, Спаситель, спасти и меня, избавь меня от чудовища, простерши Свою руку, и выведи из глубины греха. Мф. 14:31\normalfont{}


\itshape Припев:\normalfont{} Помилуй мя, Боже, помилуй мя.


Пристанище Тя вем утишное, Владыко, Владыко Христе, но от незаходимых глубин греха и отчаяния мя, предварив, избави.


\itshape Тихое пристанище вижу в Тебе, Владыка, Владыка Христе, поспеши же избавить меня от непроходимых глубин греха и отчаяния.\normalfont{}


Слава Отцу и Сыну и Святому Духу.


Тро́ица есмь Про́ста, Нераздельна, раздельна Личне, и Еди́ница есмь естеством соединена, Отец глаголет, и Сын, и Божественный Дух.


\itshape Я "--- Троица Несоставная, Нераздельная, раздельная в лицах, и Единица, соединенная по существу; свидетельствует Отец, Сын и Божественный Дух.\normalfont{}


И ныне и присно и во веки веков. Аминь.


Утроба Твоя Бога нам роди, воображена по нам: Егоже, яко Создателя всех, моли, Богородице, да молитвами Твоими оправдимся.


\itshape Чрево Твое родило нам Бога, принявшего наш образ; Его, как Создателя всего мира, моли, Богородица, чтобы по молитвам Твоим нам оправдаться.\normalfont{}


Господи, помилуй. \itshape Трижды.\normalfont{}


Слава Отцу и Сыну и Святому Духу. И ныне и присно и во веки веков. Аминь.


Душе моя, душе моя, востани, что спиши? конец приближается, и имаши смутитися: воспряни убо, да пощадит тя Христос Бог, везде сый и вся исполняяй.


\itshape Душа моя, душа моя, восстань, что ты спишь? Конец приближается, и ты смутишься; пробудись же, чтобы пощадил тебя Христос Бог, Вездесущий и все наполняющий.\normalfont{}





\bfseries Песнь 7\normalfont{}


\itshape Ирмо́с:\normalfont{}


Согрешихом, беззаконновахом, неправдовахом пред Тобою, ниже соблюдохом, ниже сотворихом, якоже заповедал еси нам; но не предаждь нас до конца, отцев Боже.


\itshape Мы согрешили, жили беззаконно, неправо поступали пред Тобою, не сохранили, не исполнили, что Ты заповедал нам; но не оставь нас до конца, Боже отцов. Дан. 9:5–6\normalfont{}


\itshape Припев:\normalfont{} Помилуй мя, Боже, помилуй мя.


Манассиева собрала еси согрешения изволением, поставльши яко мерзости страсти и умноживши, душе, негодования, но того покаянию ревнующи тепле, стяжи умиление.


\itshape Ты, душа, добровольно вместила преступления Манассии, поставив вместо идолов страсти и умножив мерзости; но усердно подражай и его покаянию с чувством умиления. 4 Цар. 21:1–2\normalfont{}


\itshape Припев:\normalfont{} Помилуй мя, Боже, помилуй мя.


Ахаавовым поревновала еси сквернам, душе моя, увы мне, была еси плотских скверн пребывалище и сосуд срамлен страстей, но из глубины твоея воздохни и глаголи Богу грехи твоя.


\itshape Ты подражала Ахаву в мерзостях, душа моя; увы, ты сделалась жилищем плотских нечистот и постыдным сосудом страстей; но воздохни из глубины своей и поведай Богу грехи свои. 3 Цар. 16:30\normalfont{}


\itshape Припев:\normalfont{} Помилуй мя, Боже, помилуй мя.


Заключися тебе небо, душе, и глад Божий постиже тя, егда Илии Фесвитянина, якоже Ахаав, не покорися словесем иногда, но Сараффии уподобився, напитай пророчу душу.


\itshape Заключилось небо для тебя, душа, и голод от Бога послан на тебя, как некогда на Ахава за то, что он не послушал слов Илии Фесфитянина; но ты подражай вдове Сарептской, напитай душу пророка. 3 Цар. 17:8–9\normalfont{}


\itshape Припев:\normalfont{} Помилуй мя, Боже, помилуй мя.


Попали Илия иногда дващи пятьдесят Иезавелиных, егда студныя пророки погуби, во обличение Ахаавово, но бегай подражания двою, душе, и укрепляйся.


\itshape Илия попалил некогда дважды по пятьдесят служителей Иезавели, когда истреблял гнусных пророков ее в обличение Ахава; но ты, душа, избегай подражания обоим им и крепись в воздержании. 4 Цар. 1:10–15\normalfont{}


Слава Отцу и Сыну и Святому Духу.


Тро́ице Про́стая, Нераздельная, Единосущная, и Естество Едино, Светове и Свет, и Свята Три, и Едино Свято поется Бог Тро́ица; но воспой, прослави Живот и Животы, душе, всех Бога.


\itshape Троица Простая, Нераздельная, Единосущная, и Одно Божество, Светы и Свет, Три Святы и Одно Лицо Свято, Бог-Троица, воспеваемая в песнопениях; воспой же и ты, душа, прославь Жизнь и Жизни "--- Бога всех.\normalfont{}


И ныне и присно и во веки веков. Аминь.


Поем Тя, благословим Тя, покланяемся Ти, Богородительнице, яко Нераздельныя Тро́ицы породила еси Единаго Христа Бога и Сама отверзла еси нам, сущим на земли, Небесная.


\itshape Воспеваем Тебя, благословляем Тебя, поклоняемся Тебе, Богородительница, ибо Ты родила Одного из Нераздельной Троицы, Христа Бога, и Сама открыла для нас, живущих на земле, небесные обители.\normalfont{}





\bfseries Песнь 8\normalfont{}


\itshape Ирмо́с:\normalfont{}


Егоже воинства Небесная славят, и трепещут херувими и серафими, всяко дыхание и тварь, пойте, благословите и превозносите во вся веки.


\itshape Кого прославляют воинства небесные и пред Кем трепещут Херувимы и Серафимы, Того, все существа и творения, воспевайте, благословляйте и превозносите во все века.\normalfont{}


\itshape Припев:\normalfont{} Помилуй мя, Боже, помилуй мя.


Правосуде Спасе, помилуй и избави мя огня и прещения, еже имам на суде праведно претерпети; ослаби ми прежде конца добродетелию и покаянием.


\itshape Правосудный Спаситель, помилуй и избавь меня от огня и наказания, которому я должен справедливо подвергнуться на суде; прости меня прежде кончины, дав мне добродетель и покаяние.\normalfont{}


\itshape Припев:\normalfont{} Помилуй мя, Боже, помилуй мя.


Яко разбойник, вопию Ти: помяни мя; яко Петр, плачу горце: ослаби ми, Спасе; зову, яко мытарь, слезю, яко блудница; приими мое рыдание, якоже иногда хананеино.


\itshape Как разбойник взываю к Тебе: вспомни меня; как Петр, горько плачу, Спаситель; как мытарь, издаю вопль: будь милостив ко мне; проливаю слезы, как блудница; прими мое рыдание, как некогда от жены Хананейской. Лк. 7:37–38; Лк. 18:13; Лк. 23:42; Лк. 22:62; Мф. 15:22\normalfont{}


\itshape Припев:\normalfont{} Помилуй мя, Боже, помилуй мя.


Гноение, Спасе, исцели смиренныя моея души, Едине Врачу, пластырь мне наложи, и елей, и вино, дела покаяния, умиление со слезами.


\itshape Один Врач "--- Спаситель, исцели гниение моей смиренной души; приложи мне пластырь, елей и вино "--- дела покаяния, умиление со слезами.\normalfont{}


\itshape Припев:\normalfont{} Помилуй мя, Боже, помилуй мя.


Хананею и аз подражая, помилуй мя, вопию, Сыне Давидов; касаюся края ризы, яко кровоточивая, плачу, яко Марфа и Мария над Лазарем.


\itshape Подражая жене Хананейской, и я взываю к Сыну Давидову: помилуй меня; касаюсь одежды Его, как кровоточивая, плачу, как Марфа и Мария над Лазарем. Мф. 9:20; Мф. 15:22; Ин. 11:33\normalfont{}


Слава Отцу и Сыну и Святому Духу.


Безначальне Отче, Сыне Собезначальне, Утешителю Благий, Душе Правый, Слова Божия Родителю, Отца Безначальна Слове, Душе Живый и Зиждяй, Тро́ице Еди́нице, помилуй мя.


\itshape Безначальный Отче, Собезначальный Сын, Утешитель Благий, Дух Правый, Родитель Слова Божия, Безначальное Слово Отца, Дух, Животворящий и Созидающий, Троица Единая, помилуй меня.\normalfont{}


И ныне и присно и во веки веков. Аминь.


Яко от оброщения червленицы, Пречистая, умная багряница Еммануилева внутрь во чреве Твоем плоть исткася. Темже Богородицу воистинну Тя почитаем.


\itshape Мысленная порфира "--- плоть Еммануила соткалась внутри Твоего чрева. Пречистая, как бы из вещества пурпурного; потому мы почитаем Тебя, Истинную Богородицу.\normalfont{}





\bfseries Песнь 9\normalfont{}


\itshape Ирмо́с:\normalfont{}


Безсеменнаго зачатия Рождество несказанное, Матере безмужныя нетленен Плод, Божие бо Рождение обновляет естества. Темже Тя вси роди, яко Богоневестную Матерь, православно величаем.


\itshape Рождество от бессеменного зачатия неизъяснимо, безмужной Матери нетленен Плод, ибо рождение Бога обновляет природу. Поэтому Тебя, как Богоневесту-Матерь мы, все роды, православно величаем.\normalfont{}


\itshape Припев:\normalfont{} Помилуй мя, Боже, помилуй мя.


Недуги исцеляя, нищим благовествоваше Христос Слово, вредныя уврачева, с мытари ядяше, со грешники беседоваше, Иаировы дщере душу предумершую возврати осязанием руки.


\itshape Врачуя болезни, Христос-Слово, благовествовал нищим, исцелял увечных, вкушал с мытарями, беседовал с грешниками и прикосновением руки возвратил вышедшую из тела душу Иаировой дочери. Мф. 4:23; Мф. 9:10–11; Мк. 5:41–42\normalfont{}


\itshape Припев:\normalfont{} Помилуй мя, Боже, помилуй мя.


Мытарь спасашеся, и блудница целомудрствоваше, и фарисей, хваляся, осуждашеся. Ов убо: очисти мя; ова же: помилуй мя; сей же величашеся вопия: Боже, благодарю Тя, и прочия безумныя глаголы.


\itshape Мытарь спасся и блудница сделалась целомудренною, а гордый фарисей подвергся осуждению, ибо первый взывал: «Будь милостив ко мне»; другая: «Помилуй меня»; а последний тщеславно возглашал: «Боже, благодарю Тебя...» и прочие безумные речи. Лк. 7:46–47; Лк. 18:14\normalfont{}


\itshape Припев:\normalfont{} Помилуй мя, Боже, помилуй мя.


Закхей мытарь бе, но обаче спасашеся, и фарисей Симон соблажняшеся, и блудница приимаше оставительная разрешения от Имущаго крепость оставляти грехи, юже, душе, потщися подражати.


\itshape Закхей был мытарь, однако спасся; Симон фарисей соблазнялся, а блудница получила решительное прощение от Имеющего власть отпускать грехи; спеши, душа, и ты подражать ей. Лк. 7:39; Лк. 19:9; Ин. 8:3–11\normalfont{}


\itshape Припев:\normalfont{} Помилуй мя, Боже, помилуй мя.


Блуднице, о окаянная душе моя, не поревновала еси, яже приимши мира алавастр, со слезами мазаше нозе Спасове, отре же власы, древних согрешений рукописание Раздирающаго ея.


\itshape Бедная душа моя, ты не подражала блуднице, которая, взяв сосуд с миром, мазала со слезами и отирала волосами ноги Спасителя, разорвавшего запись прежних ее прегрешений. Лк. 7:37–38\normalfont{}


\itshape Припев:\normalfont{} Помилуй мя, Боже, помилуй мя.


Грады, имже даде Христос благовестие, душе моя, уведала еси, како прокляти быша. Убойся указания, да не будеши якоже оны, ихже содомляном Владыка уподобив, даже до ада осуди.


\itshape Ты знаешь, душа моя, как прокляты города, которым Христос благовестил Евангелие; страшись этого примера, чтобы и тебе не быть, как они, ибо Владыка, уподобив их содомлянам, присудил их к аду. Лк. 10:12–15\normalfont{}


\itshape Припев:\normalfont{} Помилуй мя, Боже, помилуй мя.


Да не горшая, о душе моя, явишися отчаянием, хананеи веру слышавшая, еяже дщи словом Божиим исцелися; Сыне Давидов, спаси и мене, воззови из глубины сердца, якоже она Христу.


\itshape Не окажись, душа моя, по отчаянию хуже хананеянки, слышавшей о вере, по которой Божиим словом исцелена дочь ее; взывай, как она, Христу из глубины сердца: «Сын Давидов, спаси и меня». Мф. 15:22\normalfont{}


Слава Отцу и Сыну и Святому Духу.


Отца прославим, Сына превознесем, Божественному Духу верно поклонимся, Тро́ице Нераздельней, Еди́нице по Существу, яко Свету и Светом, и Животу и Животом, Животворящему и Просвещающему концы.


\itshape Прославим Отца, превознесем Сына, с верою поклонимся Божественному Духу, Нераздельной Троице, Единой по существу, как Свету и Светам, Жизни и Жизням, Животворящему и Просвещающему пределы вселенной.\normalfont{}


И ныне и присно и во веки веков. Аминь.


Град Твой сохраняй, Богородительнице Пречистая, в Тебе бо сей верно царствуяй, в Тебе и утверждается, и Тобою побеждаяй, побеждает всякое искушение, и пленяет ратники, и проходит послушание.


\itshape Сохраняй град Свой, Пречистая Богородительница. Под Твоею защитою он царствует с верою, и от Тебя получает крепость, и при Твоем содействии неотразимо побеждает всякое бедствие, берет в плен врагов и держит их в подчинении.\normalfont{}


\itshape Припев:\normalfont{} Преподобне отче Андрее, моли Бога о нас.


Андрее честный и отче треблаженнейший, пастырю Критский, не престай моляся о воспевающих тя, да избавимся вси гнева, и скорби, и тления, и прегрешений безмерных, чтущии твою память верно.


\itshape Андрей досточтимый, отец преблаженный, пастырь Критский, не переставай молиться за воспевающих тебя, чтобы избавиться от гнева, скорби, погибели и бесчисленных прегрешений нам всем, искренно почитающим память твою.\normalfont{}


\itshape Таже оба лика вкупе поют Ирмо́с:\normalfont{}


Безсеменнаго зачатия Рождество несказанное, Матере безмужныя нетленен Плод, Божие бо Рождение обновляет естества. Темже Тя вси роди, яко Богоневестную Матерь, православно величаем.


\itshape Рождество от бессеменного зачатия неизъяснимо, безмужной Матери нетленен Плод, ибо рождение Бога обновляет природу. Поэтому Тебя, как Богоневесту-Матерь мы, все роды, православно величаем.\normalfont{}




\mychapterending

\mychapter{В четверг первой седмицы Великого Поста}
%http://www.molitvoslov.org/text574.htm 
 






\bfseries Песнь 1\normalfont{}


\itshape Ирмо́с:\normalfont{}


Помощник и Покровитель бысть мне во спасение, Сей мой Бог, и прославлю Его, Бог отца моего, и вознесу Его: славно бо прославися.


\itshape Помощник и Покровитель явился мне ко спасению, Он Бог мой, и прославлю Его, Бога отца моего, и превознесу Его, ибо Он торжественно прославился. Исх. 15:1–2\normalfont{}


\itshape Припев:\normalfont{} Помилуй мя, Боже, помилуй мя.


Агнче Божий, вземляй грехи всех, возьми бремя от мене тяжкое греховное, и, яко благоутробен, даждь ми слезы умиления.


\itshape Агнец Божий, взявший грехи всех, сними с меня тяжкое бремя греховное и, как Милосердный, даруй мне слезы умиления. Ин. 1:29\normalfont{}


\itshape Припев:\normalfont{} Помилуй мя, Боже, помилуй мя.


Тебе припадаю, Иисусе, согреших Ти, очисти мя, возьми бремя от мене тяжкое греховное и, яко благоутробен, даждь ми слезы умиления.


\itshape К Тебе припадаю, Иисусе, согрешил я пред Тобою, умилосердись надо мною, сними с меня тяжкое бремя греховное и, как Милосердный, даруй мне слезы умиления.\normalfont{}


\itshape Припев:\normalfont{} Помилуй мя, Боже, помилуй мя.


Не вниди со мною в суд, нося моя деяния, словеса изыскуя и исправляя стремления. Но в щедротах Твоих презирая моя лютая, спаси мя, Всесильне.


\itshape Не входи со мною в суд, взвешивая мои дела, исследуя слова и обличая стремления, но по Твоим щедротам презирая мои злодеяния, спаси меня, Всесильный.\normalfont{}


\itshape Припев:\normalfont{} Помилуй мя, Боже, помилуй мя.


Покаяния время, прихожду Ти, Создателю моему: возьми бремя от мене тяжкое греховное и, яко благоутробен, даждь ми слезы умиления.


\itshape Время покаяния: к Тебе прихожу, моему Создателю, сними с меня тяжкое бремя греховное и, как Милосердный, даруй мне слезы умиления.\normalfont{}


\itshape Припев:\normalfont{} Помилуй мя, Боже, помилуй мя.


Богатство душевное иждив грехом, пуст есмь добродетелей благочестивых, гладствуя же зову: милости подателю Господи, спаси мя.


\itshape Расточив в грехе духовное богатство, я чужд святых добродетелей, но, испытывая голод, взываю: Источник милости, Господи, спаси меня.\normalfont{}


\itshape Припев:\normalfont{} Преподобная мати Марие, моли Бога о нас.


Приклоньшися Христовым Божественным законом, к сему приступила еси, сладостей неудержимая стремления оставивши, и всякую добродетель всеблагоговейно, яко едину, исправила еси.


\itshape Покорившись перед Божественными заповедями Христа, ты предалась Ему, оставив необузданные стремления к удовольствиям, и все добродетели, как одну, исполнила со всем благоговением.\normalfont{}


Слава Отцу и Сыну и Святому Духу.


Пресущественная Тро́ице, во Еди́нице покланяемая, возьми бремя от мене тяжкое греховное и, яко благоутробна, даждь ми слезы умиления.


\itshape Пресущественная Троица, Которой мы поклоняемся как Единому Существу, сними с меня тяжелое бремя греховное и, как Милосердная, даруй мне слезы умиления.\normalfont{}


И ныне и присно и во веки веков. Аминь.


Богородице, Надежде и Предстательство Тебе поющих, возьми бремя от мене тяжкое греховное и, яко Владычица Чистая, кающася приими мя.


\itshape Богородице, Надежда и Помощь всем воспевающих Тебя, сними с меня тяжелое бремя греховное и, как Владычица Непорочная, прими меня кающегося.\normalfont{}





\bfseries Песнь 2\normalfont{}


\itshape Ирмо́с:\normalfont{}


Видите, видите, яко Аз есмь Бог, манну одождивый и воду из камене источивый древле в пустыни людем Моим, десницею единою и крепостию Моею.


\itshape Видите, видите, что Я "--- Бог, в древности ниспославший манну и источивший воду из камня народу Моему в пустыне "--- одним Своим всемогуществом. Исх. 16:14; Исх. 17:6\normalfont{}


\itshape Припев:\normalfont{} Помилуй мя, Боже, помилуй мя.


Мужа убих, глаголет, в язву мне и юношу в струп, Ламех, рыдая, вопияше; ты же не трепещеши, о душе моя, окалявши плоть и ум осквернивши.


\itshape Мужа убил я, сказал Ламех, в язву себе, и юношу "--- в рану себе, взывал он, рыдая; ты же, душа моя, не трепещешь, осквернив тело и помрачив ум. Быт. 4:23\normalfont{}


\itshape Припев:\normalfont{} Помилуй мя, Боже, помилуй мя.


Столп умудрила еси создати, о душе, и утверждение водрузити твоими похотьми, аще не бы Зиждитель удержал советы твоя и низвергл на землю ухищрения твоя.


\itshape Ты умудрилась, душа, устроить столп и воздвигнуть твердыню своими вожделениями, но Творец обуздал замыслы твои и поверг на землю твои построения. Быт. 11:3–4\normalfont{}


\itshape Припев:\normalfont{} Помилуй мя, Боже, помилуй мя.


О како поревновах Ламеху, первому убийце, душу, яко мужа, ум, яко юношу, яко брата же моего, тело убив, яко Каин убийца, любосластными стремленьми.


\itshape О, как уподобился я древнему убийце Ламеху, убив душу, как мужа, ум "--- как юношу, и подобно убийце Каину "--- тело мое, как брата, сластолюбивыми стремлениями.\normalfont{}


\itshape Припев:\normalfont{} Помилуй мя, Боже, помилуй мя.


Одожди Господь от Господа огнь иногда на беззаконие гневающее, сожег содомляны; ты же огнь вжегла еси геенский, в немже имаши, о душе, сожещися.


\itshape Господь некогда пролил дождем огонь от Господа, попалив неистовое беззаконие содомлян; ты же, душа, разожгла огонь геенский, в котором должна будешь гореть. Быт. 19:24\normalfont{}


\itshape Припев:\normalfont{} Помилуй мя, Боже, помилуй мя.


Уязвихся, уранихся, се стрелы вражия, уязвившия мою душу и тело; се струпи, гноения, омрачения вопиют, раны самовольных моих страстей.


\itshape Изранен я, изъявлен; вот стрелы врага, пронзившие душу мою и тело; вот раны, язвы и струпы вопиют об ударах самопроизвольных моих страстей.\normalfont{}


\itshape Припев:\normalfont{} Преподобная мати Марие, моли Бога о нас.


Простерла еси руце твои к щедрому Богу, Марие, в бездне зол погружаемая; и якоже Петру человеколюбно руку Божественную простре твое обращение всячески Иский.


\itshape Утопая в бездне зла, ты простерла, Мария, руки свои к Милосердному Богу, и Он, всячески ища твоего обращения, человеколюбиво подал тебе, как Петру, Божественную руку. Мф. 14:31\normalfont{}


Слава Отцу и Сыну и Святому Духу.


Безначальная, Несозданная Тро́ице, Нераздельная Еди́нице, кающася мя приими, согрешивша спаси, Твое есмь создание, не презри, но пощади и избави мя огненнаго осуждения.


\itshape Безначальная Несозданная Троица, Нераздельная Единица, прими меня кающегося, спаси согрешившего, я "--- Твое создание, не презри, но пощади и избавь меня от осуждения в огонь.\normalfont{}


И ныне и присно и во веки веков. Аминь.


Пречистая Владычице, Богородительнице, Надеждо к Тебе притекающих и пристанище сущих в бури, Милостиваго и Создателя и Сына Твоего умилостиви и мне молитвами Твоими.


\itshape Пречистая Владычица, Богородительница, Надежда прибегающих к Тебе и пристанище для застигнутых бурей, Твоими молитвами приклони на милость и ко мне Милостивого Творца и Сына Твоего.\normalfont{}





\bfseries Песнь 3\normalfont{}


\itshape Ирмо́с:\normalfont{}


Утверди, Господи, на камени заповедей Твоих подвигшееся сердце мое, яко Един Свят еси и Господь.


\itshape Утверди, Господи, на камне Твоих заповедей поколебавшееся сердце мое, ибо Ты Один Свят и Господь.\normalfont{}


\itshape Припев:\normalfont{} Помилуй мя, Боже, помилуй мя.


Агаре древле, душе, египтяныне уподобилася еси, поработившися произволением и рождши новаго Исмаила, презорство.


\itshape Древней Агари египтянке уподобилась ты, душа, порабощенная своим произволом и родив нового Измаила "--- дерзость. Быт. 16:16\normalfont{}


\itshape Припев:\normalfont{} Помилуй мя, Боже, помилуй мя.


Иаковлю лествицу разумела еси, душе моя, являемую от земли к Небесем: почто не имела еси восхода тверда, благочестия.


\itshape Ты знаешь, душа моя, о лестнице с земли до небес, показанной Иакову; почему же ты не избрала безопасного восхода "--- благочестия? Быт. 28:12\normalfont{}


\itshape Припев:\normalfont{} Помилуй мя, Боже, помилуй мя.


Священника Божия и царя уединена, Христово подобие в мире жития, в человецех подражай.


\itshape Подражай священнику Божию и царю одинокому Мелхиседеку, образу жизни Христа среди людей в мире. Быт. 14:18; Евр. 7:1–3\normalfont{}


\itshape Припев:\normalfont{} Помилуй мя, Боже, помилуй мя.


Обратися, постени, душе окаянная, прежде даже не приимет конец жития торжество, прежде даже дверь не заключит чертога Господь.


\itshape Обратись и воздыхай, несчастная душа, прежде нежели кончится торжество жизни, прежде чем Господь затворит дверь брачного чертога.\normalfont{}


\itshape Припев:\normalfont{} Помилуй мя, Боже, помилуй мя.


Не буди столп сланый, душе, возвратившися вспять, образ да устрашит тя содомский, горе в Сигор спасайся.


\itshape Не сделайся соляным столпом, душа, обратившись назад, да устрашит тебя пример содомлян; спасайся на гору в Сигор. Быт. 19:19–23; Быт. 19:26\normalfont{}


\itshape Припев:\normalfont{} Помилуй мя, Боже, помилуй мя.


Моления, Владыко, Тебе поющих не отвержи, но ущедри, Человеколюбче, и подаждь верою просящим оставление.


\itshape Не отвергни, Владыко, моления воспевающих Тебя, но умилосердись Человеколюбец, и просящим с верою даруй прощение.\normalfont{}


Слава Отцу и Сыну и Святому Духу.


Тро́ица Про́стая, Несозданная, Безначальное Естество, в Тро́ице певаемая Ипостасей, спаси ны, верою покланяющияся державе Твоей.


\itshape Троица Несоставная, Несозданная, Существо Безначальная, в троичности Лиц воспеваемая, спаси нас, с верою поклоняющихся силе Твоей.\normalfont{}


И ныне и присно и во веки веков. Аминь.


От Отца безлетна Сына в лето, Богородительнице, неискусомужно родила еси, странное чудо, пребывши Дева доящи.


\itshape Ты, Богородительница, не испытавши мужа, во времени родила Сына от Отца вне времени и "--- дивное чудо: питая молоком, пребыла Девою.\normalfont{}





\bfseries Песнь 4\normalfont{}


\itshape Ирмо́с:\normalfont{}


Услыша пророк пришествие Твое, Господи, и убояся, яко хощеши от Девы родитися и человеком явитися, и глаголаше: услышах слух Твой и убояхся, слава силе Твоей, Господи.


\itshape Услышал пророк о пришествии Твоем, Господи, и устрашился, что Тебе угодно родиться от Девы и явиться людям, и сказал: услышал я весть о Тебе и устрашился; слава силе Твоей, Господи.\normalfont{}


\itshape Припев:\normalfont{} Помилуй мя, Боже, помилуй мя.


Время живота моего мало и исполнено болезней и лукавства, но в покаянии мя приими и в разум призови, да не буду стяжание ни брашно чуждему, Спасе, Сам мя ущедри.


\itshape Время жизни моей кратко и исполнено огорчений и пороков, но прими меня в покаянии и призови к познанию истины, чтобы не сделаться мне добычею и пищею врага, Спаситель, умилосердись надо мною. Быт. 47:9\normalfont{}


\itshape Припев:\normalfont{} Помилуй мя, Боже, помилуй мя.


Царским достоинством, венцем и багряницею одеян, многоименный человек и праведный, богатством кипя и стады, внезапу богатства, славы царства, обнищав, лишися.


\itshape Человек, облеченный царским достоинством, венцом и багряницею, много имевший и праведный, изобиловавший богатством и стадами, внезапно обнищав, лишился богатства, славы и царства. Иов. 1:1–22\normalfont{}


\itshape Припев:\normalfont{} Помилуй мя, Боже, помилуй мя.


Аще праведен бяше он и непорочен паче всех, и не убеже ловления льстиваго и сети; ты же, грехолюбива сущи, окаянная душе, что сотвориши, аще чесому от недоведомых случится наити тебе?


\itshape Если он, будучи праведным и безукоризненным более всех, не избежал козней и сетей обольстителя диавола, то что сделаешь, ты, грехолюбивая несчастная душа, если что-нибудь неожиданное постигнет тебя?\normalfont{}


\itshape Припев:\normalfont{} Помилуй мя, Боже, помилуй мя.


Высокоглаголив ныне есмь, жесток же и сердцем, вотще и всуе, да не с фарисеем осудиши мя. Паче же мытарево смирение подаждь ми, Едине Щедре, Правосуде, и сему мя сочисли.


\itshape Высокомерен я ныне на словах, дерзок и в сердце, напрасно и тщетно; не осуди меня с фарисеем, но даруй мне смирение мытаря и к нему причисли, Один Милосердный и Правосудный.\normalfont{}


\itshape Припев:\normalfont{} Помилуй мя, Боже, помилуй мя.


Согреших, досадив сосуду плоти моея, вем, Щедре, но в покаянии мя приими и в разум призови, да не буду стяжание ни брашно чуждему, Спасе, Сам мя ущедри.


\itshape Знаю, Милосердный, согрешил я, осквернив сосуд моей плоти, но прими меня в покаянии и призови к познанию истины, чтобы не сделаться мне добычею и пищею врага; Сам, Ты, Спаситель, умилосердись надо мною.\normalfont{}


\itshape Припев:\normalfont{} Помилуй мя, Боже, помилуй мя.


Самоистукан бых страстьми, душу мою вредя, Щедре, но в покаянии мя приими и в разум призови, да не буду стяжание ни брашно чуждему, Спасе, Сам мя ущедри.


\itshape Истуканом я сделал сам себя, исказив душу свою страстями, Милосердный; но прими меня в покаянии и призови к познанию истины, чтобы не сделаться мне добычею и пищею врага; Сам, Ты, Спаситель, умилосердись надо мною.\normalfont{}


\itshape Припев:\normalfont{} Помилуй мя, Боже, помилуй мя.


Не послушах гласа Твоего, преслушах Писание Твое, Законоположника, но в покаянии мя приими и в разум призови, да не буду стяжание ни брашно чуждему, Спасе, Сам мя ущедри.


\itshape Не послушал я голоса Твоего, нарушил Писание Твое, Законодатель; но прими меня в покаянии и призови к познанию истины, чтобы не сделаться мне добычею и пищею врага; Сам, Ты, Спаситель, умилосердись надо мною.\normalfont{}


\itshape Припев:\normalfont{} Преподобная мати Марие, моли Бога о нас.


Великих безместий во глубину низведшися, неодержима была еси; но востекла еси помыслом лучшим к крайней деяньми яве добродетели преславно, ангельское естество, Марие, удививши.


\itshape Увлекшись в глубину великих пороков, ты, Мария, не погрязла в ней, но высшим помыслом через деятельность явно поднялась до совершенной добродетели, дивно изумив ангельскую природу.\normalfont{}


Слава Отцу и Сыну и Святому Духу.


Нераздельное Существом, Неслитное Лицы богословлю Тя, Троическое Едино Божество, яко Единоцарственное и Сопрестольное, вопию Ти песнь великую, в вышних трегубо песнословимую.


\itshape Нераздельным по существу, неслиянным в Лицах богословски исповедую Тебя, Троичное Единое Божество, Соцарственное и Сопрестольное; возглашаю Тебе великую песнь, в небесных обителях троекратно воспеваемую. Ис. 6:1–3\normalfont{}


И ныне и присно и во веки веков. Аминь.


И раждаеши, и девствуеши, и пребываеши обоюду естеством Дева, Рождейся обновляет законы естества, утроба же раждает нераждающая. Бог идеже хощет, побеждается естества чин: творит бо, елика хощет.


\itshape И рождаешь Ты, и остаешься Девою, в обоих случаях сохраняя по естеству девство. Рожденный Тобою обновляет законы природы, а девственное чрево рождает; когда хочет Бог, то нарушается порядок природы, ибо Он творит, что хочет.\normalfont{}





\bfseries Песнь 5\normalfont{}


\itshape Ирмо́с:\normalfont{}


От нощи утренююща, Человеколюбче, просвети, молюся, и настави и мене на повеления Твоя, и научи мя, Спасе, творити волю Твою.


\itshape От ночи бодрствующего, просвети меня, молю, Человеколюбец, путеводи меня в повелениях Твоих и научи меня, Спаситель, исполнять Твою волю. Пс. 62:2; Пс. 118:35\normalfont{}


\itshape Припев:\normalfont{} Помилуй мя, Боже, помилуй мя.


Низу сничащую подражай, о душе, прииди, припади к ногама Иисусовыма, да тя исправит, и да ходиши право стези Господни.


\itshape Подражай, душа, скорченной жене, приди, припади к ногам Иисуса, чтобы Он исправил тебя и ты могла ходить прямо по стезям Господним. Лк. 13:11–13\normalfont{}


\itshape Припев:\normalfont{} Помилуй мя, Боже, помилуй мя.


Аще и кладязь еси глубокий, Владыко, источи ми воду из пречистых Твоих жил, да, яко самаряныня, не ктому, пияй, жажду: жизни бо струи источаеши.


\itshape Если Ты "--- и глубокий колодец, Владыко, то источи мне струи из пречистых ребр Своих, чтобы я, как самарянка, испив, уже не жаждал, ибо Ты источаешь потоки жизни. Ин. 4:11–15\normalfont{}


\itshape Припев:\normalfont{} Помилуй мя, Боже, помилуй мя.


Силоам да будут ми слезы моя, Владыко Господи, да умыю и аз зеницы сердца, и вижду Тя, умна Света превечна.


\itshape Силоамом да будут мне слезы мои, Владыко Господи, чтобы и мне омыть очи сердца и умственно созерцать Тебя, Предвечный Свет. Ин. 9:7\normalfont{}


\itshape Припев:\normalfont{} Преподобная мати Марие, моли Бога о нас.


Несравненным желанием, всебогатая, древу возжелевши поклонитися животному, сподобилася еси желания, сподоби убо и мене улучити вышния славы.


\itshape С чистой любовию возжелав поклониться Древу Жизни, всеблаженная, ты удостоилась желаемого; удостой же и меня достигнуть высшей славы.\normalfont{}


Слава Отцу и Сыну и Святому Духу.


Тя, Тро́ице, славим Единаго Бога: Свят, Свят, Свят еси, Отче, Сыне и Душе, Про́стое Существо, Еди́нице присно покланяемая.


\itshape Тебя, Пресвятая Троица, прославляем за Единого Бога: Свят, Свят, Свят Отец, Сын и Дух, Простое Существо, Единица вечно поклоняемая.\normalfont{}


И ныне и присно и во веки веков. Аминь.


Из Тебе облечеся в мое смешение, нетленная, безмужная Мати Дево, Бог, создавый веки, и соедини Себе человеческое естество.


\itshape В Тебе, Нетленная, не познавшая мужа Матерь-Дево, облекся в мой состав сотворивший мир Бог и соединил с Собою человеческую природу.\normalfont{}





\bfseries Песнь 6\normalfont{}


\itshape Ирмо́с:\normalfont{}


Возопих всем сердцем моим к щедрому Богу, и услыша мя от ада преисподняго, и возведе от тли живот мой.


\itshape От всего сердца моего я воззвал к милосердному Богу, и Он услышал меня из ада преисподнего и воззвал жизнь мою от погибели.\normalfont{}


\itshape Припев:\normalfont{} Помилуй мя, Боже, помилуй мя.


Аз есмь, Спасе, юже погубил еси древле царскую драхму; но вжег светильник, Предтечу Твоего, Слове, взыщи и обрящи Твой образ.


\itshape Я "--- та драхма с царским изображением, которая с древности потеряна у Тебя, Спаситель, но, засветив светильник "--- Предтечу Своего, Слове, поищи и найди Свой образ.\normalfont{}


\itshape Припев:\normalfont{} Помилуй мя, Боже, помилуй мя.


Востани и побори, яко Иисус Амалика, плотския страсти, и гаваониты, лестныя помыслы, присно побеждающи.


\itshape Восстань и низложи плотские страсти, как Иисус Амалика, всегда побеждая и гаваонитян "--- обольстительные помыслы. Исх. 17:8; Нав. 8:21\normalfont{}


\itshape Припев:\normalfont{} Преподобная мати Марие, моли Бога о нас.


Да страстей пламень угасиши, слез капли источала еси присно, Марие, душею распаляема, ихже благодать подаждь и мне, твоему рабу.


\itshape Чтобы угасить пламень страстей, ты, Мария, пылая душой, непрестанно проливала потоки слез, преизобилие которых даруй и мне, рабу твоему.\normalfont{}


\itshape Припев:\normalfont{} Преподобная мати Марие, моли Бога о нас.


Безстрастие Небесное стяжала еси крайним на земли житием, мати. Темже тебе поющим страстей избавитися молитвами твоими молися.


\itshape Возвышеннейшим образом жизни на земле, ты, матерь, приобрела небесное бесстрастие; поэтому ходатайствуй, чтобы воспевающие тебя избавились от страстей по твоим молитвам.\normalfont{}


Слава Отцу и Сыну и Святому Духу.


Тро́ица есмь Про́ста, Нераздельна, раздельна Личне и Еди́ница есмь естеством соединена, Отец глаголет, и Сын, и Божественный Дух.


\itshape Я "--- Троица Несоставная, Нераздельная, раздельная в Лицах, и Единица, соединенная по существу; свидетельствует Отец, Сын и Божественный Дух.\normalfont{}


И ныне и присно и во веки веков. Аминь.


Утроба Твоя Бога нам роди, воображена по нам: Егоже, яко Создателя всех, моли, Богородице, да Твоими молитвами оправдимся.


\itshape Чрево Твое родило нам Бога, принявшего наш образ; Его, как Создателя всего мира, моли, Богородица, чтобы по молитвам Твоим нам оправдаться.\normalfont{}


Господи, помилуй. \itshape Трижды.\normalfont{}


Слава Отцу и Сыну и Святому Духу. И ныне и присно и во веки веков. Аминь.





\bfseries Кондак, глас 6:\normalfont{}


Душе моя, душе моя, востани, что спиши? конец приближается, и имаши смутитися: воспряни убо, да пощадит тя Христос Бог, везде сый и вся исполняяй.


\itshape Душа моя, душа моя, восстань, что ты спишь? Конец приближается, и ты смутишься; пробудись же, чтобы пощадил тебя Христос Бог, Вездесущий и все наполняющий.\normalfont{}





\bfseries Песнь 7\normalfont{}


\itshape Ирмо́с:\normalfont{}


Согрешихом, беззаконновахом, неправдовахом пред Тобою, ниже соблюдохом, ниже сотворихом, якоже заповедал еси нам; но не предаждь нас до конца, отцев Боже.


\itshape Мы согрешили, жили беззаконно, неправо поступали пред Тобою, не сохранили, не исполнили, что Ты заповедал нам; но не оставь нас до конца, Боже отцов. Дан. 9:5–6\normalfont{}


\itshape Припев:\normalfont{} Помилуй мя, Боже, помилуй мя.


Исчезоша дние мои, яко соние востающаго; темже, яко Езекия, слезю на ложи моем, приложитися мне летом живота. Но кий Исаия предстанет тебе, душе, аще не всех Бог?


\itshape Дни мои прошли как сновидение пробуждающегося; поэтому, подобно Езекии, я плачу на ложе моем, чтобы продлились годы жизни моей; но какой Исаия посетит тебя, душа, если не Бог всех? 4 Цар. 20:3; Ис. 38:2–6\normalfont{}


\itshape Припев:\normalfont{} Помилуй мя, Боже, помилуй мя.


Припадаю Ти и приношу Тебе, якоже слезы, глаголы моя: согреших, яко не согреши блудница, и беззаконновах, яко иный никтоже на земли. Но ущедри, Владыко, творение Твое и воззови мя.


\itshape Припадаю к Тебе и приношу Тебе со слезами слова мои: согрешил я, как не согрешила блудница, и жил в беззакониях, как никто другой на земле; но умилосердись, Владыка, над созданием Своим и восстанови меня.\normalfont{}


\itshape Припев:\normalfont{} Помилуй мя, Боже, помилуй мя.


Погребох образ Твой и растлих заповедь Твою, вся помрачися доброта, и страстьми угасися, Спасе, свеща. Но ущедрив, воздаждь ми, якоже поет Давид, радование.


\itshape Затмил я образ Твой и нарушил заповедь Твою; вся красота помрачилась во мне, и светильник погас от страстей; но умилосердись, Спаситель, и возврати мне, как поет Давид, веселие. Пс. 50:14\normalfont{}


\itshape Припев:\normalfont{} Помилуй мя, Боже, помилуй мя.


Обратися, покайся, открый сокровенная, глаголи Богу, вся ведущему: Ты веси моя тайная, Едине Спасе. Но Сам мя помилуй, якоже поет Давид, по милости Твоей.


\itshape Обратись, покайся, открой сокровенное, скажи Богу Всеведущему: Спаситель, Ты Один знаешь мои тайны, но Сам помилуй меня, как поет Давид, по Твоей милости. Пс. 50:3\normalfont{}


\itshape Припев:\normalfont{} Преподобная мати Марие, моли Бога о нас.


Возопивши к Пречистей Богоматери, первее отринула еси неистовство страстей, нужно стужающих, и посрамила еси врага запеншаго. Но даждь ныне помощь от скорби и мне, рабу твоему.


\itshape Воззвавши к Пречистой Богоматери, ты обуздала неистовство страстей, прежде жестоко свирепствовавших, и посрамила врага-обольстителя; даруй же ныне помощь в скорби и мне, рабу твоему. Пс. 59:13\normalfont{}


\itshape Припев:\normalfont{} Преподобная мати Марие, моли Бога о нас.


Егоже возлюбила еси, Егоже возжелела еси, Егоже ради плоть изнурила еси, преподобная, моли ныне Христа о рабех: яко да милостив быв всем нам, мирное состояние дарует почитающим Его.


\itshape Кого ты возлюбила, Кого избрала, для Кого изнуряла плоть, Преподобная, моли ныне Христа о рабах твоих, чтобы Он по Своей милости ко всем даровал мирное состояние почитающим Его.\normalfont{}


Слава Отцу и Сыну и Святому Духу.


Тро́ице Про́стая, Нераздельная, Единосущная и Естество Едино, Светове и Свет, и Свята Три, и Едино Свято поется Бог Тро́ица; но воспой, прослави Живот и Животы, душе, всех Бога.


\itshape Троица Простая, Нераздельная, Единосущная, и Одно Божество, Светы и Свет, Три Святы и Одно Лицо Свято, Бог-Троица, воспеваемая в песнопениях; воспой же и ты, душа, прославь Жизнь и Жизни "--- Бога всех.\normalfont{}


И ныне и присно и во веки веков. Аминь.


Поем Тя, благословим Тя, покланяемся Ти, Богородительнице, яко Неразлучныя Тро́ицы породила еси Единаго Христа Бога и Сама отверзла еси нам, сущим на земли, Небесная.


\itshape Воспеваем Тебя, благословляем Тебя, поклоняемся Тебе, Богородительница, ибо Ты родила Одного из Нераздельной Троицы, Христа Бога, и Сама открыла для нас, живущих на земле, небесные обители.\normalfont{}





\bfseries Песнь 8\normalfont{}


\itshape Ирмо́с:\normalfont{}


Егоже воинства Небесная славят, и трепещут херувими и серафими, всяко дыхание и тварь, пойте, благословите и превозносите во вся веки.


\itshape Кого прославляют воинства небесные и пред Кем трепещут Херувимы и Серафимы, Того, все существа и творения, воспевайте, благословляйте и превозносите во все века.\normalfont{}


\itshape Припев:\normalfont{} Помилуй мя, Боже, помилуй мя.


Слезную, Спасе, сткляницу яко миро истощавая на главу, зову Ти, якоже блудница, милости ищущая, мольбу приношу и оставление прошу прияти.


\itshape Изливая сосуд слез, как миро на голову, Спаситель, взываю к Тебе, как ищущая милости блудница, приношу моление и прошу о получении мне прощения. Мф. 26:6–7; Мк. 14:3; Лк. 7:37–38\normalfont{}


\itshape Припев:\normalfont{} Помилуй мя, Боже, помилуй мя.


Аще и никтоже, якоже аз, согреши Тебе, но обаче приими и мене, благоутробне Спасе, страхом кающася и любовию зовуща: согреших Тебе Единому, помилуй мя, Милостиве.


\itshape Хотя никто не согрешил пред Тобою, как я, но, Милосердный Спаситель, приими меня, кающегося со страхом и с любовию взывающего: я согрешил пред Тобою Одним, помилуй меня, Милосердный!\normalfont{}


\itshape Припев:\normalfont{} Помилуй мя, Боже, помилуй мя.


Пощади, Спасе, Твое создание и взыщи, яко Пастырь, погибшее, предвари заблуждшаго, восхити от волка, сотвори мя овча на пастве Твоих овец.


\itshape Пощади, Спаситель, создание Свое и, как Пастырь, отыщи потерянного, возврати заблудшего, отними у волка и сделай меня агнцем на пастбище Твоих овец. Пс. 118:176\normalfont{}


\itshape Припев:\normalfont{} Помилуй мя, Боже, помилуй мя.


Егда, Судие, сядеши, яко благоутробен, и покажеши страшную славу Твою, Спасе, о каковый страх тогда, пещи горящей, всем боящимся нестерпимаго судища Твоего.


\itshape Когда Ты, Милосердный, воссядешь, как Судия и откроешь грозное величие Свое, Спаситель, о, какой ужас тогда: печь будет гореть, и все трепетать пред неумолимым судом Твоим. Мф. 25:31; Мф. 25:41; Мф. 25:47\normalfont{}


\itshape Припев:\normalfont{} Преподобная мати Марие, моли Бога о нас.


Света незаходимаго Мати тя просветивши, от омрачения страстей разреши. Темже вшедши в духовную благодать, просвети, Марие, тя верно восхваляющия.


\itshape Матерь незаходимаго Света "--- Христа, просветив тебя, освободила от мрака страстей; поэтому, приняв благодать Духа, просвети, Мария, искренно прославляющих тебя.\normalfont{}


\itshape Припев:\normalfont{} Преподобная мати Марие, моли Бога о нас.


Чудо ново видев, ужасашеся божественный в тебе воистинну, мати, Зосима: ангела бо зряше во плоти и ужасом весь исполняшеся, Христа поя во веки.


\itshape Увидев в тебе, матерь, поистине новое чудо, святой Зосима удивился, ибо он увидел Ангела во плоти, и весь преисполнился изумлением, воспевая Христа вовеки.\normalfont{}


Слава Отцу и Сыну и Святому Духу.


Безначальне Отче, Сыне Собезначальне, Утешителю Благий, Душе Правый, Слова Божия Родителю, Отца Безначальна Слове, Душе Живый и Зиждяй, Тро́ице Еди́нице, помилуй мя.


\itshape Безначальный Отче, Собезначальный Сын, Утешитель Благий, Дух Правый, Родитель Слова Божия, Безначальное Слово Отца, Дух, Животворящий и Созидающий, Троица Единая, помилуй меня.\normalfont{}


И ныне и присно и во веки веков. Аминь.


Яко от оброщения червленицы, Пречистая, умная багряница Еммануилева внутрь во чреве Твоем плоть исткася. Темже Богородицу воистинну Тя почитаем.


\itshape Мысленная порфира "--- плоть Еммануила соткалась внутри Твоего чрева. Пречистая, как бы из вещества пурпурного; потому мы почитаем Тебя, Истинную Богородицу.\normalfont{}





\bfseries Песнь 9\normalfont{}


\itshape Ирмо́с:\normalfont{}


Безсеменнаго зачатия Рождество несказанное, Матере безмужныя нетленен Плод, Божие бо Рождение обновляет естества. Темже Тя вси роди, яко Богоневестную Матерь, православно величаем.


\itshape Рождество от бессеменного зачатия неизъяснимо, безмужной Матери нетленен Плод, ибо рождение Бога обновляет природу. Поэтому Тебя, как Богоневесту-Матерь мы, все роды, православно величаем.\normalfont{}


\itshape Припев:\normalfont{} Помилуй мя, Боже, помилуй мя.


Умилосердися, спаси мя, Сыне Давидов, помилуй, беснующияся словом исцеливый, глас же благоутробный, яко разбойнику, мне рцы: аминь, глаголю тебе, со Мною будеши в раи, егда прииду во славе Моей.


\itshape Умилосердись, спаси и помилуй меня, Сын Давидов, словом исцелявший беснующихся, и скажи, как разбойнику, милостивые слова: истинно говорю тебе, со Мною будешь в раю, когда приду Я в славе Моей. Лк. 23:43\normalfont{}


\itshape Припев:\normalfont{} Помилуй мя, Боже, помилуй мя.


Разбойник оглаголоваше Тя, разбойник богословяше Тя: оба бо на кресте свисяста. Но, о Благоутробне, яко верному разбойнику Твоему, познавшему Тя Бога, и мне отверзи дверь славнаго Царствия Твоего.


\itshape Разбойник поносил Тебя, разбойник же и Богом исповедал Тебя, вися оба на кресте; но, Милосердный, как уверовавшему разбойнику, познавшему в Тебе Бога, открой и мне, дверь славного Твоего Царства.\normalfont{}


\itshape Припев:\normalfont{} Помилуй мя, Боже, помилуй мя.


Тварь содрогашеся, распинаема Тя видящи, горы и камения страхом распадахуся, и земля сотрясашеся, и ад обнажашеся, и соомрачашеся свет во дни, зря Тебе, Иисусе, пригвождена ко Кресту.


\itshape Тварь содрогалась, видя Тебя распинаемым, горы и камни от ужаса распадались и колебалась земля, преисподняя пустела, и свет среди дня помрачался, взирая на Тебя, Иисус, плотию ко кресту пригвожденного. Мф. 27:51–52; Мк. 15:38; Лк. 23:45\normalfont{}


\itshape Припев:\normalfont{} Помилуй мя, Боже, помилуй мя.


Достойных покаяния плодов не истяжи от мене, ибо крепость моя во мне оскуде; сердце мне даруй присно сокрушенное, нищету же духовную: да сия Тебе принесу яко приятную жертву, Едине Спасе.


\itshape Достойных плодов покаяния не требуй от меня, Единый Спаситель, ибо сила моя истощилась во мне; даруй мне всегда сокрушенное сердце и духовную нищету, чтобы я принес их Тебе, как благоприятную жертву.\normalfont{}


\itshape Припев:\normalfont{} Помилуй мя, Боже, помилуй мя.


Судие мой и Ведче мой, хотяй паки приити со ангелы, судити миру всему, милостивным Твоим оком тогда видев мя, пощади и ущедри мя, Иисусе, паче всякаго естества человеча согрешивша.


\itshape Судия мой, знающий меня, когда опять придешь Ты с Ангелами, чтобы судить весь мир, тогда, обратив на меня милостивый взор, пощади, Иисусе, и помилуй меня, согрешившего более всего человеческого рода.\normalfont{}


\itshape Припев:\normalfont{} Преподобная мати Марие, моли Бога о нас.


Удивила еси всех странным житием твоим, ангелов чины и человеков соборы, невещественно поживши и естество прешедши: имже, яко невещественныма ногама вшедши, Марие, Иордан прешла еси.


\itshape Ты удивила необычайною своею жизнью всех, как чины ангельские, так и человеческие сонмы, духовно пожив и превзошедши природу; поэтому, Мария, ты, как бесплотная, шествуя стопами, перешла Иордан.\normalfont{}


\itshape Припев:\normalfont{} Преподобная мати Марие, моли Бога о нас.


Умилостиви Создателя о хвалящих тя, преподобная мати, избавитися озлоблений и скорбей, окрест нападающих: да избавившеся от напастей, возвеличим непрестанно прославльшаго тя Господа.


\itshape Склони Творца на милость к восхваляющим тебя, преподобная матерь, чтобы нам избавиться от огорчений и скорбей, отовсюду нападающих на нас, чтобы, избавившись от искушений, мы непрестанно величали прославившего тебя Господа.\normalfont{}


\itshape Припев:\normalfont{} Преподобне отче Андрее, моли Бога о нас.


Андрее честный и отче треблаженнейший, пастырю Критский, не престай моляся о воспевающих тя: да избавимся вси гнева, и скорби, и тления, и прегрешений безмерных, чтущии твою память верно.


\itshape Андрей досточтимый, отец преблаженный, пастырь Критский, не переставай молиться за воспевающих тебя, чтобы избавиться от гнева, скорби, погибели и бесчисленных прегрешений нам всем, искренно почитающим память твою.\normalfont{}


Слава Отцу и Сыну и Святому Духу.


Отца прославим, Сына превознесем, Божественному Духу верно поклонимся, Тро́ице Нераздельней, Еди́нице по Существу, яко Свету и Светом, и Животу и Животом, Животворящему и Просвещающему концы.


\itshape Прославим Отца, превознесем Сына, с верою поклонимся Божественному Духу, Нераздельной Троице, Единой по существу, как Свету и Светам, Жизни и Жизням, Животворящему и Просвещающему пределы вселенной.\normalfont{}


И ныне и присно и во веки веков. Аминь.


Град Твой сохраняй, Богородительнице Пречистая, в Тебе бо сей верно царствуяй, в Тебе и утверждается, и Тобою побеждаяй, побеждает всякое искушение, и пленяет ратники, и проходит послушание.


\itshape Сохраняй град Свой, Пречистая Богородительница. Под Твоею защитою он царствует с верою, и от Тебя получает крепость, и при Твоем содействии неотразимо побеждает всякое бедствие, берет в плен врагов и держит их в подчинении.\normalfont{}


\itshape Таже оба лика вкупе поют Ирмо́с:\normalfont{}


Безсеменнаго зачатия Рождество несказанное, Матере безмужныя нетленен Плод, Божие бо Рождение обновляет естества. Темже Тя вси роди, яко Богоневестную Матерь, православно величаем.


\itshape Рождество от бессеменного зачатия неизъяснимо, безмужной Матери нетленен Плод, ибо рождение Бога обновляет природу. Поэтому Тебя, как Богоневесту-Матерь мы, все роды, православно величаем.\normalfont{}




\mychapterending

\mychapter{В четверг пятой седмицы Великого Поста}
%http://www.molitvoslov.org/text575.htm 
 






\bfseries Песнь 1\normalfont{}


\itshape Ирмо́с:\normalfont{}


Помощник и Покровитель бысть мне во спасение, Сей мой Бог, и прославлю Его, Бог Отца моего, и вознесу Его: славно бо прославися.


\itshape Помощник и Покровитель явился мне ко спасению, Он Бог мой, и прославлю Его, Бога отца моего, и превознесу Его, ибо Он торжественно прославился. Исх. 15:1–2\normalfont{}


\itshape Припев:\normalfont{} Помилуй мя, Боже, помилуй мя.


Откуду начну плакати окаяннаго моего жития деяний? Кое ли положу начало, Христе, нынешнему рыданию? Но, яко благоутробен, даждь ми прегрешений оставление.


\itshape С чего начну я оплакивать деяния злосчастной моей жизни? Какое начало положу, Христе, я нынешнему моему сетованию? Но Ты, как милосердный, даруй мне оставление прегрешений.\normalfont{}


\itshape Припев:\normalfont{} Помилуй мя, Боже, помилуй мя.


Гряди, окаянная душе, с плотию твоею, Зиждителю всех исповеждься, и останися прочее преждняго безсловесия, и принеси Богу в покаянии слезы.


\itshape Прииди, несчастная душа, с плотию своею, исповедайся Создателю всего, воздержись, наконец, от прежнего безрассудства и с раскаянием принеси Богу слезы.\normalfont{}


\itshape Припев:\normalfont{} Помилуй мя, Боже, помилуй мя.


Первозданнаго Адама преступлению поревновав, познах себе обнажена от Бога и присносущнаго Царствия и сладости, грех ради моих.


\itshape Подражая в преступлении первозданному Адаму, я сознаю себя лишенным Бога, вечного Царства и блаженства за мои грехи. Быт. 3:6–7\normalfont{}


\itshape Припев:\normalfont{} Помилуй мя, Боже, помилуй мя.


Увы мне, окаянная душе, что уподобилася еси первей Еве? Видела бо еси зле, и уязвилася еси горце, и коснулася еси древа, и вкусила еси дерзостно безсловесныя снеди.


\itshape Горе мне, моя несчастная душа, для чего ты уподобилась первосозданной Еве? Не с добром ты посмотрела и уязвилась жестоко, прикоснулась к дереву и дерзостно вкусила бессмысленного плода. Быт. 3:6\normalfont{}


\itshape Припев:\normalfont{} Помилуй мя, Боже, помилуй мя.


Вместо Евы чувственныя, мысленная ми бысть Ева, во плоти страстный помысл, показуяй сладкая и вкушаяй присно горькаго напоения.


\itshape Вместо чувственной Евы восстала во мне Ева мысленная "--- плотский страстный помысел, обольщающий приятным, но при вкушении всегда напояющий горечью.\normalfont{}


\itshape Припев:\normalfont{} Помилуй мя, Боже, помилуй мя.


Достойно из Едема изгнан бысть, яко не сохранив едину Твою, Спасе, заповедь Адам; аз же что постражду, отметая всегда животная Твоя словеса?


\itshape Достойно был изгнан из Едема Адам, как не сохранивший одной Твоей заповеди, Спаситель. Что же должен претерпеть я, всегда отвергающий Твои животворные повеления? Быт. 3:23\normalfont{}


\itshape Припев:\normalfont{} Помилуй мя, Боже, помилуй мя.


Каиново прешед убийство, произволением бых убийца совести душевней, оживив плоть и воевав на ню лукавыми моими деяньми.


\itshape Превзойдя Каиново убийство, сознательным произволением, через оживление греховной плоти, я сделался убийцею души, вооружившись против нее злыми моими делами. Быт. 4:8\normalfont{}


\itshape Припев:\normalfont{} Помилуй мя, Боже, помилуй мя.


Авелеве, Иисусе, не уподобихся правде, дара Тебе приятна не принесох когда, ни деяния Божественна, ни жертвы чистыя, ни жития непорочнаго.


\itshape Авелевой праведности, Иисусе, я не подражал, никогда не приносил Тебе приятных даров, ни дел богоугодных, ни жертвы чистой, ни жизни непорочной. Быт. 4:3–4\normalfont{}


\itshape Припев:\normalfont{} Помилуй мя, Боже, помилуй мя.


Яко Каин и мы, душе окаянная, всех Содетелю деяния скверная, и жертву порочную, и непотребное житие принесохом вкупе: темже и осудихомся.


\itshape Как Каин, так и мы, несчастная душа, принесли Создателю всего жертву порочную "--- дела нечестивые и жизнь невоздержанную: поэтому мы и осуждены. Быт. 4:5\normalfont{}


\itshape Припев:\normalfont{} Помилуй мя, Боже, помилуй мя.


Брение Здатель живосоздав, вложил еси мне плоть и кости, и дыхание, и жизнь; но, о Творче мой, Избавителю мой и Судие, кающася приими мя.


\itshape Оживотворивший земной прах, Скудельник, Ты даровал мне плоть и кости, дыхание и жизнь; но, Творец мой, Искупитель мой и Судия, приими меня кающегося! Быт. 2:7\normalfont{}


\itshape Припев:\normalfont{} Помилуй мя, Боже, помилуй мя.


Извещаю Ти, Спасе, грехи, яже содеях, и души и тела моего язвы, яже внутрь убийственнии помыслы разбойнически на мя возложиша.


\itshape Пред Тобою, Спаситель, открываю грехи, сделанные мною, и раны души и тела моего, которые разбойнически нанесли мне внутренние убийственные помыслы. Лк. 10:30\normalfont{}


\itshape Припев:\normalfont{} Помилуй мя, Боже, помилуй мя.


Аще и согреших, Спасе, но вем, яко Человеколюбец еси, наказуеши милостивно и милосердствуеши тепле, слезяща зриши и притекаеши, яко отец, призывая блуднаго.


\itshape Хотя я и согрешил, Спаситель, но знаю, что Ты человеколюбив; наказываешь с состраданием и милуешь с любовью, взираешь на плачущего и спешишь, как Отец, призвать блудного. Лк. 15:20\normalfont{}


\itshape Припев:\normalfont{} Помилуй мя, Боже, помилуй мя.


Повержена мя, Спасе, пред враты Твоими, поне на старость не отрини мене во ад тща, но прежде конца, яко Человеколюбец, даждь ми прегрешений оставление.


\itshape Поверженного пред вратами Твоими, Спаситель, хотя в старости, не низринь меня в ад, как невоздержанного, но прежде кончины, как Человеколюбец, даруй мне оставление прегрешений.\normalfont{}


\itshape Припев:\normalfont{} Помилуй мя, Боже, помилуй мя.


В разбойники впадый аз есмь помышленьми моими, весь от них уязвихся ныне и исполнихся ран, но, Сам ми представ, Христе Спасе, исцели.


\itshape По помыслам моим я человек, попавшийся разбойникам; теперь я весь изранен ими, покрыт язвами, но Ты Сам, Христос Спаситель, приди и исцели меня. Лк. 10:30\normalfont{}


\itshape Припев:\normalfont{} Помилуй мя, Боже, помилуй мя.


Священник, мя предвидев мимо иде, и левит, видев в лютых нага, презре, но из Марии возсиявый Иисусе, Ты, представ, ущедри мя.


\itshape Священник, заметив меня, прошел мимо, и левит, видя меня в беде обнаженного, презрел; но Ты, воссиявший от Марии Иисусе, прииди и умилосердись надо мною. Лк. 10:31–32\normalfont{}


\itshape Припев:\normalfont{} Помилуй мя, Боже, помилуй мя.


Агнче Божий, вземляй грехи всех, возми бремя от мене тяжкое греховное, и яко благоутробен, даждь ми слезы умиления.


\itshape Агнец Божий, взявший грехи всех, сними с меня тяжкое бремя греховное и, как Милосердный, даруй мне слезы умиления. Ин. 1:29\normalfont{}


\itshape Припев:\normalfont{} Помилуй мя, Боже, помилуй мя.


Покаяния время, прихожду Ти, Создателю моему: возми бремя от мене тяжкое греховное и яко благоутробен, даждь ми слезы умиления.


\itshape Время покаяния: к Тебе прихожу, моему Создателю, сними с меня тяжкое бремя греховное и, как Милосердный, даруй мне слезы умиления.\normalfont{}


\itshape Припев:\normalfont{} Помилуй мя, Боже, помилуй мя.


Не возгнушайся мене, Спасе, не отрини от Твоего лица, возьми бремя от мене тяжкое греховное и, яко благоутробен, даждь мне грехопадений оставление.


\itshape Не погнушайся мной, Спаситель, не отвергни от Твоего лица, сними с меня тяжкое бремя греховное и, как благосердный, даруй мне освобождение от грехопадений.\normalfont{}


\itshape Припев:\normalfont{} Помилуй мя, Боже, помилуй мя.


Вольная, Спасе, и невольная прегрешения моя, явленная и сокровенная и ведомая и неведомая, вся простив, яко Бог, очисти и спаси мя.


\itshape Сотворенные по своей воле и невольные прегрешения мои, Спаситель, явные и скрытые, те, о которых знаю и о которых не знаю, все простив, как Бог, изгладь и спаси меня.\normalfont{}


\itshape Припев:\normalfont{} Помилуй мя, Боже, помилуй мя.


От юности, Христе, заповеди Твоя преступих, всестрастно небрегий, унынием преидох житие. Темже зову Ти, Спасе: поне на конец спаси мя.


\itshape С юности, Христе, я пренебрегал Твоими заповедями, всю жизнь провел в страстях, беспечности и нерадении. Поэтому и взываю к Тебе, Спаситель: хотя при кончине спаси меня.\normalfont{}


\itshape Припев:\normalfont{} Помилуй мя, Боже, помилуй мя.


Богатство мое, Спасе, изнурив в блуде, пуст есмь плодов благочестивых, алчен же зову: Отче щедрот, предварив Ты мя ущедри.


\itshape Расточив богатство мое в распутстве, Спаситель, я чужд плодов благочестия, но, чувствуя голод, взываю: Отец Милосердный, поспеши и умилосердись надо мною. Лк. 15:11–14; Лк. 15:17–18\normalfont{}


\itshape Припев:\normalfont{} Помилуй мя, Боже, помилуй мя.


Тебе припадаю, Иисусе, согреших Ти, очисти мя, возми бремя от мене тяжкое греховное и яко благоутробен, даждь ми слезы умиления.


\itshape К Тебе припадаю, Иисусе, согрешил я пред Тобою, умилосердись надо мною, сними с меня тяжкое бремя греховное и, как Милосердный, даруй мне слезы умиления.\normalfont{}


\itshape Припев:\normalfont{} Помилуй мя, Боже, помилуй мя.


Не вниди со мною в суд, нося моя деяния, словеса изыскуя и исправляя стремления. Но в щедротах Твоих презирая моя лютая, спаси мя, Всесильне.


\itshape Не входи со мною в суд, взвешивая мои дела, исследуя слова и обличая стремления, но по Твоим щедротам презирая мои злодеяния, спаси меня, Всесильный.\normalfont{}





\bfseries Иный канон преподобныя матере нашея Марии Египетския, глас 6:\normalfont{}


\itshape Припев:\normalfont{} Преподобная мати Марие, моли Бога о нас.


Ты ми даждь светозарную благодать от Божественнаго свыше Промышления избежати страстей омрачения и пети усердно Твоего, Марие, жития красная исправления.


\itshape Даруй мне, Мария, ниспосланную тебе свыше Божественным Промыслом светозарную благодать "--- избежать мрака страстей и усердно воспеть прекрасные подвиги твоей жизни.\normalfont{}


\itshape Припев:\normalfont{} Преподобная мати Марие, моли Бога о нас.


Приклоньшися Христовым Божественным законом, к Сему приступила еси, сладостей неудержимая стремления оставивши, и всякую добродетель всеблагоговейно, яко едину, исправила еси.


\itshape Покорившись перед Божественными заповедями Христа, ты предалась Ему, оставив необузданные стремления к удовольствиям, и все добродетели, как одну, исполнила со всем благоговением.\normalfont{}


\itshape Припев:\normalfont{} Преподобне отче Андрее, моли Бога о нас.


Молитвами твоими нас, Андрее, избави страстей безчестных и Царствия ныне Христова общники верою и любовию воспевающия тя, славне, покажи, молимся.


\itshape Молитвами твоими, Андрей, избавь нас от позорных страстей и сделай, молимся, ныне участниками Царства Христова воспевающих с верой и любовию тебя, прославленный.\normalfont{}


Слава Отцу и Сыну и Святому Духу.


Пресущественная Троице, во Единице покланяемая, возми бремя от мене тяжкое греховное и, яко благоутробна, даждь ми слезы умиления.


\itshape Пресущественная Троица, Которой мы поклоняемся как Единому Существу, сними с меня тяжелое бремя греховное и, как Милосердная, даруй мне слезы умиления.\normalfont{}


И ныне и присно и во веки веков. Аминь.


Богородице, Надежде и Предстательство Тебе поющих, возми бремя от мене тяжкое греховное и, яко Владычица Чистая, кающася приими мя.


\itshape Богородице, Надежда и Помощь всем воспевающих Тебя, сними с меня тяжелое бремя греховное и, как Владычица Непорочная, прими меня кающегося.\normalfont{}





\bfseries Песнь 2\normalfont{}


\itshape Ирмо́с:\normalfont{}


Вонми, небо, и возглаголю, и воспою Христа, от Девы плотию пришедшаго.


\itshape Внемли, небо, я буду возвещать и воспевать Христа, пришедшего во плоти от Девы.\normalfont{}


\itshape Припев:\normalfont{} Помилуй мя, Боже, помилуй мя.


Вонми, небо, и возглаголю, земле внушай глас, кающийся к Богу и воспевающий Его.


\itshape Внемли, небо, я буду возвещать: земля, услышь голос, кающийся Богу и прославляющий Его.\normalfont{}


\itshape Припев:\normalfont{} Помилуй мя, Боже, помилуй мя.


Вонми ми, Боже, Спасе мой, милостивым Твоим оком, и приими мое теплое исповедание.


\itshape Воззри на меня Боже, Спаситель мой, милостивым Твоим оком и прими мою пламенную исповедь.\normalfont{}


\itshape Припев:\normalfont{} Помилуй мя, Боже, помилуй мя.


Согреших паче всех человек, един согреших Тебе; но ущедри яко Бог, Спасе, творение Твое.


\itshape Согрешил я более всех людей, один я согрешил пред Тобою; но, как Бог, сжалься, Спаситель, над Твоим созданием. 1 Тим. 1:15\normalfont{}


\itshape Припев:\normalfont{} Помилуй мя, Боже, помилуй мя.


Буря мя злых обдержит, благоутробне Господи; но яко Петру, и мне руку простри.


\itshape Буря зла окружает меня, Милосердный Господи, но, как Петру, и мне Ты простри руку. Мф. 14:31\normalfont{}


\itshape Припев:\normalfont{} Помилуй мя, Боже, помилуй мя.


Слезы блудницы, Щедре, и аз предлагаю, очисти мя, Спасе, благоутробием Твоим.


\itshape Как блудница, и я проливаю слезы, Милосердный; смилуйся надо мною, Спаситель, по благоснисхождению Твоему. Лк. 7:37–38\normalfont{}


\itshape Припев:\normalfont{} Помилуй мя, Боже, помилуй мя.


Омрачих душевную красоту страстей сластьми, и всячески весь ум персть сотворих.


\itshape Помрачил я красоту души страстными удовольствиями и весь ум совершенно превратил в прах.\normalfont{}


\itshape Припев:\normalfont{} Помилуй мя, Боже, помилуй мя.


Раздрах ныне одежду мою первую, юже ми изтка Зиждитель изначала, и оттуду лежу наг.


\itshape Разодрал я первую одежду мою, которую вначале соткал мне Создатель, и оттого лежу обнаженным.\normalfont{}


\itshape Припев:\normalfont{} Помилуй мя, Боже, помилуй мя.


Облекохся в раздранную ризу, юже изтка ми змий советом, и стыждуся.


\itshape Облекся я в разодранную одежду, которую соткал мне змий коварством, и стыжусь. Быт. 3:7\normalfont{}


\itshape Припев:\normalfont{} Помилуй мя, Боже, помилуй мя.


Воззрех на садовную красоту, и прельстихся умом; и оттуду лежу наг и срамляюся.


\itshape Взглянул я на красоту дерева и прельстился в уме; оттого лежу обнаженным и стыжусь.\normalfont{}


\itshape Припев:\normalfont{} Помилуй мя, Боже, помилуй мя.


Делаша на хребте моем вси начальницы страстей, продолжающе на мя беззаконие их.


\itshape На хребте моем пахали все вожди страстей, проводя вдоль по мне беззаконие свое. Пс. 128:3\normalfont{}


\itshape Припев:\normalfont{} Помилуй мя, Боже, помилуй мя.


Погубих первозданную доброту и благолепие мое и ныне лежу наг и стыждуся.


\itshape Погубил я первозданную красоту и благообразие мое и теперь лежу обнаженным и стыжусь.\normalfont{}


\itshape Припев:\normalfont{} Помилуй мя, Боже, помилуй мя.


Сшиваше кожныя ризы грех мне, обнаживый мя первыя боготканныя одежды.


\itshape «Кожаные ризы» сшил мне грех, сняв с меня прежнюю Богом сотканную одежду. Быт. 3:11; Быт. 3:21\normalfont{}


\itshape Припев:\normalfont{} Помилуй мя, Боже, помилуй мя.


Обложен есмь одеянием студа, якоже листвием смоковным, во обличение моих самовластных страстей.


\itshape Облекся я одеянием стыда, как листьями смоковницы, во обличение самовольных страстей моих. Быт. 3:7\normalfont{}


\itshape Припев:\normalfont{} Помилуй мя, Боже, помилуй мя.


Одеяхся в срамную ризу и окровавленную студно течением страстнаго и любосластнаго живота.


\itshape Оделся я в одежду, постыдно запятнанную и окровавленную нечистотой страстной и сластолюбивой жизни.\normalfont{}


\itshape Припев:\normalfont{} Помилуй мя, Боже, помилуй мя.


Оскверних плоти моея ризу, и окалях еже по образу, Спасе, и по подобию.


\itshape Осквернил я одежду плоти моей и очернил в себе, Спаситель, то, что было создано по Твоему образу и подобию.. Быт. 3:21\normalfont{}


\itshape Припев:\normalfont{} Помилуй мя, Боже, помилуй мя.


Впадох в страстную пагубу и в вещественную тлю, и оттоле до ныне враг мне досаждает.


\itshape Подвергся я мучению страстей и вещественному тлению, и оттого ныне враг угнетает меня.\normalfont{}


\itshape Припев:\normalfont{} Помилуй мя, Боже, помилуй мя.


Любовещное и любоименное житие невоздержанием, Спасе, предпочет, ныне тяжким бременем обложен есмь.


\itshape Предпочтя нестяжательности жизнь, привязанную к земным вещам и любостяжательную, Спаситель, я теперь нахожусь под тяжким бременем.\normalfont{}


\itshape Припев:\normalfont{} Помилуй мя, Боже, помилуй мя.


Украсих плотский образ скверных помышлений различным обложением, и осуждаюся.


\itshape Украсил я кумир плоти разноцветным одеянием гнусных помыслов и подвергаюсь осуждению.\normalfont{}


\itshape Припев:\normalfont{} Помилуй мя, Боже, помилуй мя.


Внешним прилежно благоукрашением единем попекохся, внутреннюю презрев Богообразную скинию.


\itshape Усердно заботясь об одном внешнем благолепии, я пренебрег внутренней скинией, устроенной по образу Божию. Исх. 25:8–9\normalfont{}


\itshape Припев:\normalfont{} Помилуй мя, Боже, помилуй мя.


Вообразив моих страстей безобразие, любосластными стремленьми, погубих ума красоту.


\itshape Отобразив в себе безобразие моих страстей, сластолюбивыми стремлениями исказил я красоту ума.\normalfont{}


\itshape Припев:\normalfont{} Помилуй мя, Боже, помилуй мя.


Погребох перваго образа доброту, Спасе, страстьми, юже яко иногда драхму взыскав, обрящи.


\itshape Засыпал страстями красоту первобытного образа, Спаситель; ее, как некогда драхму, Ты взыщи и найди. Лк. 15:8\normalfont{}


\itshape Припев:\normalfont{} Помилуй мя, Боже, помилуй мя.


Согреших, якоже блудница вопию Ти: един согреших Тебе, яко миро, приими, Спасе, и моя слезы.


\itshape Согрешил, и, как блудница, взываю к Тебе: один я согрешил пред Тобою, приими, Спаситель, и от меня слезы вместо мира. Лк. 7:37–38\normalfont{}


\itshape Припев:\normalfont{} Помилуй мя, Боже, помилуй мя.


Поползохся, яко Давид, блудно и осквернихся, но омый и мене, Спасе, слезами.


\itshape От невоздержания, как Давид, я пал и осквернился, но омой и меня, Спаситель, слезами. 2 Цар. 11:4\normalfont{}


\itshape Припев:\normalfont{} Помилуй мя, Боже, помилуй мя.


Очисти, якоже мытарь вопию Ти, Спасе, очисти мя: никтоже бо сущих из Адама, якоже аз, согреших Тебе.


\itshape Умилостивись, как мытарь, взываю к Тебе, Спаситель, смилуйся надо мною: ибо как никто из потомков Адамовых я согрешил пред Тобою. Лк. 18:13\normalfont{}


\itshape Припев:\normalfont{} Помилуй мя, Боже, помилуй мя.


Ни слез, ниже покаяния имам, ниже умиления. Сам ми сия, Спасе, яко Бог, даруй.


\itshape Ни слез, ни покаяния, ни умиления нет у меня; Сам Ты, Спаситель, как Бог, даруй мне это.\normalfont{}


\itshape Припев:\normalfont{} Помилуй мя, Боже, помилуй мя.


Дверь Твою не затвори мне тогда, Господи, Господи, но отверзи ми сию, кающемуся Тебе.


\itshape Не затвори предо мною теперь дверь Твою, Господи, Господи, но отвори ее для меня, кающегося Тебе. Мф. 7:21–23; Мф. 25:10–12\normalfont{}


\itshape Припев:\normalfont{} Помилуй мя, Боже, помилуй мя.


Человеколюбче, хотяй всем спастися, Ты воззови мя и приими яко благ кающагося.


\itshape Человеколюбец, желающий всем спасения, Ты призови меня и прими, как Благий, кающегося. 1 Тим. 2:4\normalfont{}


\itshape Припев:\normalfont{} Помилуй мя, Боже, помилуй мя.


Внуши воздыхания души моея и очию моею приими капли, Спасе, и спаси мя.


\itshape Внемли, Спаситель, стенаниям души моей, прими слезы очей моих и спаси меня.\normalfont{}


\itshape Припев:\normalfont{} Пресвятая Богородице, спаси нас.


Пречистая Богородице Дево, Едина Всепетая, моли прилежно во еже спастися нам.


\itshape Пречистая Богородице Дева, Ты Одна, всеми воспеваемая, усердно моли о нашем спасении.\normalfont{}


\itshape Иный Ирмо́с:\normalfont{}


Видите, видите, яко Аз есмь Бог, манну одождивый и воду из камене источивый древле в пустыни людем Моим, десницею единою и крепостию Моею.


\itshape Видите, видите, что Я "--- Бог, в древности ниспославший манну и источивший воду из камня народу Моему в пустыне "--- одним Своим всемогуществом. Исх. 16:14; Исх. 17:6\normalfont{}


\itshape Припев:\normalfont{} Помилуй мя, Боже, помилуй мя.


Видите, видите, яко Аз есмь Бог, внушай, душе моя, Господа вопиюща, и удалися прежняго греха, и бойся яко неумытнаго и яко Судии и Бога.


\itshape Видите, видите, что Я "--- Бог. Внимай, душа моя, взывающему Господу, оставь прежний грех и убойся как праведного Судию и Бога.\normalfont{}


\itshape Припев:\normalfont{} Помилуй мя, Боже, помилуй мя.


Кому уподобилася еси, многогрешная душе? Токмо первому Каину и Ламеху оному, каменовавшая тело злодействы и убившая ум безсловесными стремленьми.


\itshape Кому уподобилась ты, многогрешная душа, как не первому Каину и тому Ламеху, жестоко окаменив тело злодеяниями и убив ум безрассудными стремлениями. Быт. 4:1–26\normalfont{}


\itshape Припев:\normalfont{} Помилуй мя, Боже, помилуй мя.


Вся прежде закона претекши, о душе, Сифу не уподобилася еси, ни Еноса подражала еси, ни Еноха преложением, ни Ноя, но явилася еси убога праведных жизни.


\itshape Имея в виду всех, живших до закона, о душа, не уподобилась ты Сифу, не подражала ни Еносу, ни Еноху через преселение духовное, ни Ною, но оказалась чуждой жизни праведников. Быт. 5:1–32\normalfont{}


\itshape Припев:\normalfont{} Помилуй мя, Боже, помилуй мя.


Едина отверзла еси хляби гнева Бога твоего, душе моя, и потопила еси всю, якоже землю, плоть, и деяния, и житие, и пребыла еси вне спасительнаго ковчега.


\itshape Ты одна, душа моя, открыла бездны гнева Бога своего и потопила, как землю, всю плоть, и дела, и жизнь, и осталась вне спасительного ковчега. Быт. 7:1–24\normalfont{}


\itshape Припев:\normalfont{} Помилуй мя, Боже, помилуй мя.


Мужа убих, глаголет, в язву мне и юношу в струп, Ламех, рыдая вопияше; ты же не трепещеши, о душе моя, окалявши плоть и ум осквернивши.


\itshape Мужа убил я, сказал Ламех, в язву себе, и юношу "--- в рану себе, взывал он, рыдая; ты же, душа моя, не трепещешь, осквернив тело и помрачив ум. Быт. 4:23\normalfont{}


\itshape Припев:\normalfont{} Помилуй мя, Боже, помилуй мя.


О, како поревновах Ламеху, первому убийце, душу яко мужа, ум яко юношу, яко брата же моего, тело убив, яко Каин убийца, любосластными стремленьми.


\itshape О, как уподобился я древнему убийце Ламеху, убив душу, как мужа, ум "--- как юношу, и подобно убийце Каину "--- тело мое, как брата, сластолюбивыми стремлениями. Быт. 4:8\normalfont{}


\itshape Припев:\normalfont{} Помилуй мя, Боже, помилуй мя.


Столп умудрила еси создати, о душе, и утверждение водрузити твоими похотьми, аще не бы Зиждитель удержал советы твоя и низвергл на землю ухищрения твоя.


\itshape Ты умудрилась, душа, устроить столп и воздвигнуть твердыню своими вожделениями, но Творец обуздал замыслы твои и поверг на землю твои построения. Быт. 11:3–4\normalfont{}


\itshape Припев:\normalfont{} Помилуй мя, Боже, помилуй мя.


Уязвихся, уранихся, се стрелы вражия уязвившыя мою душу и тело, се струпи, гноения, омрачения вопиют, раны самовольных моих страстей.


\itshape Изранен я, изъявлен; вот стрелы врага, пронзившие душу мою и тело; вот раны, язвы и струпы вопиют об ударах самопроизвольных моих страстей.\normalfont{}


\itshape Припев:\normalfont{} Помилуй мя, Боже, помилуй мя.


Одожди Господь от Господа огнь иногда на беззаконие гневающее, сожег содомляны; ты же огнь вжегла еси геенский, в немже имаши, о душе, сожещися.


\itshape Господь некогда пролил дождем огонь от Господа, попалив неистовое беззаконие содомлян; ты же, душа, разожгла огонь геенский, в котором должна будешь гореть. Быт. 19:24\normalfont{}


\itshape Припев:\normalfont{} Помилуй мя, Боже, помилуй мя.


Разумейте и видите, яко Аз есмь Бог, испытаяй сердца и умучаяй мысли, обличаяй деяния, и попаляяй грехи, и судяй сиру, и смирену, и нищу.


\itshape Познайте и увидьте, что Я "--- Бог, испытующий сердца и подвергающий наказанию мысли, обличающий деяния и огнем сжигающий грехи, и творящий праведный суд сироте, и уничиженному, и нищему.\normalfont{}


\itshape Припев:\normalfont{} Преподобная мати Марие, моли Бога о нас.


Простерла еси руце твои к щедрому Богу, Марие, в бездне зол погружаемая; и якоже Петру человеколюбно руку Божественную простре, твое обращение всячески Иский.


\itshape Утопая в бездне зла, ты простерла, Мария, руки свои к Милосердному Богу, и Он, всячески ища твоего обращения, человеколюбиво подал тебе, как Петру, Божественную руку. Мф. 14:30–31\normalfont{}


\itshape Припев:\normalfont{} Преподобная мати Марие, моли Бога о нас.


Всем усердием и любовию притекла еси Христу, первый греха путь отвращши, и в пустынях непроходимых питающися, и Того чисте совершающи Божественныя заповеди.


\itshape Оставив прежний путь греха, ты с всем усердием и любовью прибегла ко Христу, живя в непроходимых пустынях и в чистоте исполняя Божественные Его заповеди.\normalfont{}


\itshape Припев:\normalfont{} Преподобне отче Андрее, моли Бога о нас.


Видим, видим человеколюбие, о душе, Бога и Владыки; сего ради прежде конца тому со слезами припадем вопиюще: Андрея молитвами, Спасе, помилуй нас.


\itshape Видим, видим, душа моя, человеколюбие Бога и Владыки; поэтому прежде кончины припадем к Нему со слезами, взывая: "По молитвам Андрея, Спаситель, помилуй нас".\normalfont{}


Слава Отцу и Сыну и Святому Духу.


Безначальная, Несозданная Троице, Нераздельная Единице, кающася мя приими, согрешивша спаси, Твое есмь создание, не презри, но пощади и избави мя огненнаго осуждения.


\itshape Безначальная Несозданная Троица, Нераздельная Единица, прими меня кающегося, спаси согрешившего, я "--- Твое создание, не презри, но пощади и избавь меня от осуждения в огонь.\normalfont{}


И ныне и присно и во веки веков. Аминь.


Пречистая Владычице, Богородительнице, Надеждо к Тебе притекающих и пристанище сущих в бури, Милостиваго и Создателя и Сына Твоего умилостиви и мне молитвами Твоими.


\itshape Пречистая Владычица, Богородительница, Надежда прибегающих к Тебе и пристанище для застигнутых бурей, Твоими молитвами приклони на милость и ко мне Милостивого Творца и Сына Твоего.\normalfont{}





\bfseries Песнь 3\normalfont{}


\itshape Ирмо́с:\normalfont{}


На недвижимом, Христе, камени заповедей Твоих утверди мое помышление.


\itshape На неподвижном камне заповедей Твоих, Христе, утверди мое помышление.\normalfont{}


\itshape Припев:\normalfont{} Помилуй мя, Боже, помилуй мя.


Огнь от Господа иногда Господь одождив, землю содомскую прежде попали.


\itshape Пролив дождем огонь от Господа, Господь попалил некогда землю содомлян. Быт. 19:24\normalfont{}


\itshape Припев:\normalfont{} Помилуй мя, Боже, помилуй мя.


На горе спасайся душе, якоже Лот оный, и в Сигор угонзай.


\itshape Спасайся на горе, душа, как праведный Лот и спеши укрыться в Сигор. Быт. 19:22–23\normalfont{}


\itshape Припев:\normalfont{} Помилуй мя, Боже, помилуй мя.


Бегай запаления, о душе, бегай содомскаго горения, бегай тления Божественнаго пламене.


\itshape Беги, душа, от пламени, беги от горящего Содома, беги от истребления Божественным огнем.\normalfont{}


\itshape Припев:\normalfont{} Помилуй мя, Боже, помилуй мя.


Исповедаюся Тебе, Спасе, согреших, согреших Ти, но ослаби, остави ми, яко благоутробен.


\itshape Исповедуюсь Тебе, Спаситель; согрешил я, согрешил пред Тобою, но отпусти, прости меня, как Милосердный.\normalfont{}


\itshape Припев:\normalfont{} Помилуй мя, Боже, помилуй мя.


Согреших Тебе един аз, согреших паче всех, Христе Спасе, да не презриши мене.


\itshape Согрешил я один пред Тобою, согрешил более всех, Христос Спаситель "--- не презирай меня.\normalfont{}


\itshape Припев:\normalfont{} Помилуй мя, Боже, помилуй мя.


Ты еси Пастырь Добрый, взыщи мене агнца, и заблуждшаго да не презриши мене.


\itshape Ты "--- Пастырь Добрый, отыщи меня "--- агнца, и не презирай меня, заблудившегося. Ин. 10:11–14\normalfont{}


\itshape Припев:\normalfont{} Помилуй мя, Боже, помилуй мя.


Ты еси сладкий Иисусе, Ты еси Создателю мой, в Тебе, Спасе, оправдаюся.


\itshape Ты "--- вожделенный Иисус; Ты "--- Создатель мой, Спаситель, Тобою я оправдаюсь.\normalfont{}


Слава Отцу и Сыну и Святому Духу.


О Троице Единице Боже, спаси нас от прелести, и искушений, и обстояний.


\itshape О, Троица, Единица, Боже, спаси нас от обольщений, от искушений и опасностей.\normalfont{}


И ныне и присно и во веки веков. Аминь.


Радуйся, Богоприятная утробо, радуйся, престоле Господень, радуйся, Мати Жизни нашея.


\itshape Радуйся, чрево, вместившее Бога; радуйся, Престол Господень; радуйся, Матерь Жизни нашей.\normalfont{}


\itshape Иный Ирмо́с:\normalfont{}


Утверди, Господи, на камени заповедей Твоих подвигшееся сердце мое, яко Един Свят еси и Господь.


\itshape Утверди, Господи, на камне Твоих заповедей поколебавшееся сердце мое, ибо Ты один свят и Господь.\normalfont{}


\itshape Припев:\normalfont{} Помилуй мя, Боже, помилуй мя.


Источник живота стяжах Тебе, смерти Низложителя, и вопию Ти от сердца моего прежде конца: согреших, очисти и спаси мя.


\itshape Источник жизни нашел я в Тебе, Разрушитель смерти, и прежде кончины взываю к Тебе от сердца моего: согрешил я, умилостивись, спаси меня\normalfont{}


\itshape Припев:\normalfont{} Помилуй мя, Боже, помилуй мя.


При Нои, Спасе, блудствовавшыя подражах, онех наследовах осуждение в потопе погружения.


\itshape Я подражал, Спаситель, развращенным современникам Ноя и наследовал осуждение их на потопление в потопе.


Быт. 6:1–17\normalfont{}


\itshape Припев:\normalfont{} Помилуй мя, Боже, помилуй мя.


Согреших, Господи, согреших Тебе, очисти мя: несть бо иже кто согреши в человецех, егоже не превзыдох прегрешеньми.


\itshape Согрешил я, Господи, согрешил пред Тобою, смилуйся надо мною, ибо нет грешника между людьми, которого я не превзошел бы прегрешениями.\normalfont{}


\itshape Припев:\normalfont{} Помилуй мя, Боже, помилуй мя.


Хама онаго душе, отцеубийца подражавши, срама не покрыла еси искренняго, вспять зря возвратившися.


\itshape Подражая отцеубийце Хаму, ты, душа, не прикрыла срамоты ближнего с лицом, обращенным назад. Быт. 9:22–23\normalfont{}


\itshape Припев:\normalfont{} Помилуй мя, Боже, помилуй мя.


Благословения Симова не наследовала еси, душе окаянная, ни пространное одержание, якоже Иафеф, имела еси на земли оставления.


\itshape Симова благословения не наследовала ты, несчастная душа, и не получила, подобно Иафету, обширного владения на земле "--- отпущения грехов. Быт. 9:26–27\normalfont{}


\itshape Припев:\normalfont{} Помилуй мя, Боже, помилуй мя.


От земли Харран изыди от греха, душе моя, гряди в землю, точащую присноживотное нетление, еже Авраам наследствова.


\itshape Удались, душа моя, от земли Харран "--- от греха; иди в землю, источающую вечно живое нетление, которую наследовал Авраам. Быт. 12:1–7\normalfont{}


\itshape Припев:\normalfont{} Помилуй мя, Боже, помилуй мя.


Авраама слышала еси, душе моя, древле оставльша землю отечества и бывша пришельца, сего произволению подражай.


\itshape Ты слышала, душа моя, как в древности Авраам оставил землю отеческую и сделался странником; подражай его решимости. Быт. 12:1–7\normalfont{}


\itshape Припев:\normalfont{} Помилуй мя, Боже, помилуй мя.


У дуба Мамврийскаго учредив патриарх Ангелы, наследствова по старости обетования ловитву.


\itshape Угостив Ангелов под дубом Маврийским, патриарх на старости получил, как добычу, обещанное. Быт. 18:1–5\normalfont{}


\itshape Припев:\normalfont{} Помилуй мя, Боже, помилуй мя.


Исаака, окаянная душе моя, разумевши новую жертву, тайно всесожженную Господеви, подражай его произволению.


\itshape Зная, бедная душа моя, как Исаак принесен таинственно в новую жертву всесожжения Господу, подражай его решимости. Быт. 22:2\normalfont{}


\itshape Припев:\normalfont{} Помилуй мя, Боже, помилуй мя.


Исмаила слышала еси, трезвися, душе моя, изгнана, яко рабынино отрождение, виждь, да не како подобно что постраждеши, ласкосердствующи.


\itshape Ты слышала, душа моя, что Измаил был изгнан, как рожденный рабыней, бодрствуй, смотри, чтобы и тебе не потерпеть бы чего-либо подобного за сладострастие. Быт. 21:10–11\normalfont{}


\itshape Припев:\normalfont{} Помилуй мя, Боже, помилуй мя.


Агаре древле, душе, египтяныне уподобилася еси, поработившися произволением и рождши новаго Исмаила, презорство.


\itshape Древней Агари египтянке уподобилась ты, душа, порабощенная своим произволом и родив нового Измаила "--- дерзость. Быт. 16:16\normalfont{}


\itshape Припев:\normalfont{} Помилуй мя, Боже, помилуй мя.


Иаковлю лествицу разумела еси, душе моя, являемую от земли к небесем: почто не имела еси восхода тверда, благочестия.


\itshape Ты знаешь, душа моя, о лестнице с земли до небес, показанной Иакову; почему же ты не избрала безопасного восхода "--- благочестия? Быт. 28:12\normalfont{}


\itshape Припев:\normalfont{} Помилуй мя, Боже, помилуй мя.


Священника Божия и царя уединена, Христово подобие в мире жития, в человецех подражай.


\itshape Подражай священнику Божию и царю одинокому Мелхиседеку, образу жизни Христа среди людей в мире. Быт. 14:18; Евр. 7:1–3\normalfont{}


\itshape Припев:\normalfont{} Помилуй мя, Боже, помилуй мя.


Не буди столп сланый, душе, возвратившися вспять, образ да устрашит тя содомский, горе в Сигор спасайся.


\itshape Не сделайся соляным столпом, душа, обратившись назад, да устрашит тебя пример содомлян; спасайся на гору в Сигор. Быт. 19:19–23; Быт. 19:26\normalfont{}


\itshape Припев:\normalfont{} Помилуй мя, Боже, помилуй мя.


Запаления, якоже Лот, бегай, душе моя, греха бегай Содомы и Гоморры, бегай пламене всякаго безсловеснаго желания.


\itshape Беги, душа моя, от пламени греха; как Лот; беги от Содома и Гоморры; беги от огня всякого безрассудного пожелания. Быт. 19:15–17\normalfont{}


\itshape Припев:\normalfont{} Помилуй мя, Боже, помилуй мя.


Помилуй, Господи, помилуй мя, вопию Ти, егда приидеши со Ангелы Твоими воздати всем по достоянию деяний.


\itshape Помилуй, Господи, взываю к Тебе, помилуй меня, когда придешь с Ангелами Своими воздать всем по достоинству их дел.\normalfont{}


\itshape Припев:\normalfont{} Помилуй мя, Боже, помилуй мя.


Моление, Владыко, Тебе поющих не отвержи, но ущедри, Человеколюбче, и подаждь верою просящим оставление.


\itshape Не отвергни, Владыко, моления воспевающих Тебя, но умилосердись Человеколюбец, и просящим с верою даруй прощение.\normalfont{}


\itshape Припев:\normalfont{} Преподобная мати Марие, моли Бога о нас.


Содержимь есмь бурею и треволнением согрешений, но сама мя, мати, ныне спаси и к пристанищу Божественнаго покаяния возведи.


\itshape Окружен я, матерь, бурей и сильным волнением согрешений, но ты сама ныне спаси меня и приведи к пристанищу Божественного покаяния.\normalfont{}


\itshape Припев:\normalfont{} Преподобная мати Марие, моли Бога о нас.


Рабское моление и ныне, преподобная, принесши ко благоутробней молитвами твоими Богородице, отверзи ми Божественныя входы.


\itshape Усердное моление и ныне, преподобная, принеся к умилостивленной твоими молитвами Богородице, открой и для меня Божественные входы.\normalfont{}


\itshape Припев:\normalfont{} Преподобне отче Андрее, моли Бога о нас.


Твоими молитвами даруй и мне оставление долгов, о Андрее, Критский председателю, покаяния бо ты таинник преизрядный.


\itshape Твоими молитвами, О Андрей, глава (епископ) Крита, даруй и мне прощение долгов, ибо ты лучше других знаешь тайны покаяния.\normalfont{}


Слава Отцу и Сыну и Святому Духу.


Троица Простая, Несозданная, Безначальное Естество, в Троице певаемая Ипостасей, спаси ны, верою покланяющыяся державе Твоей.


\itshape Троица Несоставная, Несозданная, Существо Безначальная, в троичности Лиц воспеваемая, спаси нас, с верою поклоняющихся силе Твоей.\normalfont{}


И ныне и присно и во веки веков. Аминь.


От Отца безлетна Сына в лето, Богородительнице, неискусомужно родила еси, странное чудо, пребывши Дева доящи.


\itshape Ты, Богородительница, не испытавши мужа, во времени родила Сына от Отца вне времени и "--- дивное чудо: питая молоком, пребыла Девою.\normalfont{}


\itshape Катавасия:\normalfont{}


Утверди, Господи, на камени заповедей Твоих подвигшееся сердце мое, яко Един Свят еси и Господь.


\itshape Утверди, Господи, на камне Твоих заповедей поколебавшееся сердце мое, ибо Ты один свят и Господь.\normalfont{}


\itshape Седален, господина Иосифа, глас 8.\normalfont{}


\itshape Подобен:\normalfont{} Воскресл еси от гроба:


Светила богозрачная, Спасовы апостоли, просветите нас во тьме жития, яко да во дни ныне благообразно ходим, светом воздержания нощных страстей отбегающе, и светлыя страсти Христовы узрим, радующеся.


\itshape Рим. 12:13\normalfont{}


Слава Отцу и Сыну и Святому Духу.


\itshape Другий седален, глас 8.\normalfont{}


\itshape Подобен:\normalfont{} Повеленное тайно:


Апостольская двоенадесятице Богоизбранная, мольбу Христу ныне принеси, постное поприще всем прейти, совершающим во умилении молитвы, творящим усердно добродетели, яко да сице предварим видети Христа Бога славное Воскресение, славу и хвалу приносяще.


И ныне и присно и во веки веков. Аминь.


Непостижимаго Бога, Сына и Слово, несказанно паче ума из Тебе рождшееся, моли, Богородице, со апостолы, мир вселенней чистый подати, и согрешений дати нам прежде конца прощение, и Царствия Небеснаго крайния ради благости сподобити рабы Твоя.





\bfseries Песнь 4\normalfont{}





\bfseries Трипеснец, без поклонов, глас 8:\normalfont{}


\itshape Ирмо́с:\normalfont{}


Услышах, Господи, смотрения Твоего таинство, разумех дела Твоя и прославих Твое Божество.


\itshape Припев:\normalfont{} Святии апостоли, молите Бога о нас.


Воздержанием поживше, просвещеннии Христовы апостоли, воздержания время нам ходатайствы Божественными утишают.


\itshape Припев:\normalfont{} Святии апостоли, молите Бога о нас.


Двоенадесятострунный орган песнь воспе спасительную, учеников лик Божественный, лукавая возмущая гласования.


\itshape Припев:\normalfont{} Святии апостоли, молите Бога о нас.


Одождением духовным всю подсолнечную напоисте, сушу отгнавше многобожия, всеблаженнии.


\itshape Припев:\normalfont{} Пресвятая Богородице, спаси нас.


Смирившася спаси мя, высокомудренно пожившаго, рождшая Вознесшаго смиренное естество, Дево Всечистая.


\itshape Иный трипеснец. Ирмос, глас тойже:\normalfont{}


Услышах, Господи, смотрения Твоего таинство, разумех дела Твоя и прославих Твое Божество.


\itshape Припев:\normalfont{} Святии апостоли, молите Бога о нас.


Апостольское всечестное ликостояние, Зиждителя всех молящее, проси помиловати ны, восхваляющия тя.


\itshape Припев:\normalfont{} Святии апостоли, молите Бога о нас.


Яко делателе суще, Христовы апостоли, во всем мире Божественным словом возделавшии, приносите плоды Ему всегда.


\itshape Припев:\normalfont{} Святии апостоли, молите Бога о нас.


Виноград бысте Христов воистинну возлюбленный, вино бо духовное источисте миру, апостоли.


\itshape Припев:\normalfont{} Пресвятая Троице, Боже наш, слава Тебе.


Преначальная, Сообразная, Всесильнейшая Троице Святая, Отче, Слове и Душе Святый, Боже, Свете и Животе, сохрани стадо Твое.


\itshape Припев:\normalfont{} Пресвятая Богородице, спаси нас.


Радуйся, престоле огнезрачный, радуйся, светильниче свещеносный, радуйся, горо освящения, ковчеже Жизни, святых святая сене.


\itshape Великаго канона Ирмо́с:\normalfont{}


Услыша пророк пришествие Твое, Господи, и убояся, яко хощеши от Девы родитися и человеком явитися, и глаголаше: услышах слух Твой и убояхся, слава силе Твоей, Господи.


\itshape Услышал пророк о пришествии Твоем, Господи, и устрашился, что Тебе угодно родиться от Девы и явиться людям, и сказал: услышал я весть о Тебе и устрашился; слава силе Твоей, Господи. Авв. 3:1–3\normalfont{}


\itshape Припев:\normalfont{} Помилуй мя, Боже, помилуй мя.


Дел Твоих да не презриши, создания Твоего да не оставиши, Правосуде. Аще и един согреших яко человек, паче всякаго человека, Человеколюбче; но имаши, яко Господь всех, власть оставляти грехи.


\itshape Не презри творений Твоих, не оставь создания Твоего, Праведный Судия, ибо хотя я, как человек, один согрешил более всякого человека, но Ты, Человеколюбец, как Господь всего мира, имеешь власть отпускать грехи. Мф. 9:6; Мк. 2:10–11\normalfont{}


\itshape Припев:\normalfont{} Помилуй мя, Боже, помилуй мя.


Приближается, душе, конец, приближается, и нерадиши, ни готовишися, время сокращается, востани, близ при дверех Судия есть. Яко соние, яко цвет, время жития течет: что всуе мятемся?


\itshape Конец приближается, душа, приближается, и ты не заботишься, не готовишься; время сокращается "--- восстань: Судия уже близко "--- при дверях; время жизни проходит, как сновиденье, как цвет. Для чего мы напрасно суетимся? Мф. 24:33; Мк. 13:29; Лк. 21:31\normalfont{}


\itshape Припев:\normalfont{} Помилуй мя, Боже, помилуй мя.


Воспряни, о душе моя, деяния твоя яже соделала еси помышляй, и сия пред лице твое принеси, и капли испусти слез твоих; рцы со дерзновением деяния и помышления Христу, и оправдайся.


\itshape Пробудись, душа моя, размысли о делах своих, которые ты сделала, представь их пред своими очами, и пролей капли слез твоих, безбоязненно открой Христу дела и помышления твои и оправдайся.\normalfont{}


\itshape Припев:\normalfont{} Помилуй мя, Боже, помилуй мя.


Не бысть в житии греха, ни деяния, ни злобы, еяже аз, Спасе, не согреших, умом и словом, и произволением, и предложением, и мыслию, и деянием согрешив, яко ин никтоже когда.


\itshape Нет в жизни ни греха, ни деяния, ни зла, в которых я не был бы виновен, Спаситель, умом, и словом, и произволением, согрешив и намерением, и мыслью, и делом так, как никто другой никогда.\normalfont{}


\itshape Припев:\normalfont{} Помилуй мя, Боже, помилуй мя.


Отсюду и осужден бых, отсюду препрен бых аз окаянный от своея совести, еяже ничтоже в мире нужнейше; Судие, Избавителю мой и Ведче, пощади и избави, и спаси мя раба Твоего.


\itshape Потому и обвиняюсь, потому и осуждаюсь я, несчастный, своею совестью, строже которой нет ничего в мире; Судия, Искупитель мой и Испытатель, пощади, избавь и спаси меня, раба Твоего.\normalfont{}


\itshape Припев:\normalfont{} Помилуй мя, Боже, помилуй мя.


Лествица, юже виде древле великий в патриарсех, указание есть, душе моя, деятельнаго восхождения, разумнаго возшествия; аще хощеши убо деянием, и разумом, и зрением пожити, обновися.


\itshape Лестница, которую в древности видел великий из патриархов, служит указанием, душа моя, на восхождение делами, на возвышение разумом; поэтому, если хочешь жить в деятельности и в разумении и созерцании, то обновляйся. Быт. 28:12\normalfont{}


\itshape Припев:\normalfont{} Помилуй мя, Боже, помилуй мя.


Зной дневный претерпе лишения ради патриарх, и мраз нощный понесе, на всяк день снабдения творя, пасый, труждаяйся, работаяй, да две жене сочетает.


\itshape Патриарх по нужде терпел дневной зной и переносил ночной холод, ежедневно сокращая время, пася стада, трудясь и служа, чтобы получить себе две жены. Быт. 31; 7, 40; Быт. 29; 18–27\normalfont{}


\itshape Припев:\normalfont{} Помилуй мя, Боже, помилуй мя.


Жены ми две разумей, деяние же и разум в зрении, Лию убо деяние, яко многочадную, Рахиль же разум, яко многотрудную; ибо кроме трудов, ни деяние, ни зрение, душе, исправится.


\itshape Под двумя женами понимай деятельность и разумение в созерцании: под Лиею, как многочадною, "--- деятельность, а под Рахилью, как полученной через многие труды, "--- разумение, ибо без трудов, душа, ни деятельность, ни созерцание не усовершенствуются.\normalfont{}


\itshape Припев:\normalfont{} Помилуй мя, Боже, помилуй мя.


Бди, о душе моя, изрядствуй якоже древле великий в патриарсех, да стяжеши деяние с разумом, да будеши ум, зряй Бога, и достигнеши незаходящий мрак в видении, и будеши великий купец.


\itshape Бодрствуй, душа моя, будь мужественна, как великий из патриархов, чтобы приобрести себе дело по разуму, чтобы обогатиться умом, видящим Бога, и проникнуть в неприступный мрак в созерцании и получить великое сокровище. Мф. 13:45–46\normalfont{}


\itshape Припев:\normalfont{} Помилуй мя, Боже, помилуй мя.


Дванадесять патриархов великий в патриарсех детотворив, тайно утверди тебе лествицу деятельнаго, душе моя, восхождения: дети, яко основания, степени яко восхождения, премудренно подложив.


\itshape Великий из патриархов, родив двенадцать патриархов, таинственно представил тебе, душа моя, лестницу деятельного восхождения, премудро расположив детей как ступени, а свои шаги, как восхождения вверх.\normalfont{}


\itshape Припев:\normalfont{} Помилуй мя, Боже, помилуй мя.


Исава возненавиденнаго подражала еси, душе, отдала еси прелестнику твоему первыя доброты первенство и отеческия молитвы отпала еси, и дважды поползнулася еси, окаянная, деянием и разумом: темже ныне покайся.


\itshape Подражая ненавиденному Исаву, душа, ты отдала соблазнителю своему первенство первоначальной красоты и лишилась отеческого благословения и, несчастная, пала дважды, деятельностью и разумением, поэтому ныне покайся. Быт. 25:32; Быт. 27:37; Мал. 1:2–3\normalfont{}


\itshape Припев:\normalfont{} Помилуй мя, Боже, помилуй мя.


Едом Исав наречеся, крайняго ради женонеистовнаго смешения: невоздержанием бо присно разжигаемь и сластьми оскверняемь, Едом именовася, еже глаголется разжжение души любогреховныя.


\itshape Исав был назван Едомом за крайнее пристрастие к женолюбию; он непрестанно разжигаясь невоздержанием и оскверняясь любострастием, назван Едомом, что значит "--- «распаление души грехолюбивой». Быт. 25:30\normalfont{}


\itshape Припев:\normalfont{} Помилуй мя, Боже, помилуй мя.


Иова на гноищи слышавши, о душе моя, оправдавшагося, того мужеству не поревновала еси, твердаго не имела еси предложения во всех, яже веси, и имиже искусилася еси, но явилася еси нетерпелива.


\itshape Слышав об Иове, сидевшем на гноище, ты, душа моя, не подражала ему в мужестве, не имела твердой воли во всем, что узнала, что видела, что испытала, но оказалась нетерпеливою. Иов. 1:1–22\normalfont{}


\itshape Припев:\normalfont{} Помилуй мя, Боже, помилуй мя.


Иже первее на престоле, наг ныне на гноище гноен, многий в чадех и славный, безчаден и бездомок напрасно: палату убо гноище и бисерие струпы вменяше.


\itshape Бывший прежде на престоле, теперь "--- на гноище, обнаженный и изъязвленный; имевший многих детей и знаменитый, внезапно стал бездетным и бездомным; гноище считал он своим чертогом и язвы "--- драгоценными камнями. Иов. 2:1–13\normalfont{}


\itshape Припев:\normalfont{} Помилуй мя, Боже, помилуй мя.


Царским достоинством, венцем и багряницею одеян, многоименный человек и праведный, богатством кипя и стады, внезапу богаства, славы царства, обнищав, лишися.


\itshape Человек, облеченный царским достоинством, венцом и багряницею, много имевший и праведный, изобиловавший богатством и стадами, внезапно обнищав, лишился богатства, славы и царства. Иов. 1:1–22\normalfont{}


\itshape Припев:\normalfont{} Помилуй мя, Боже, помилуй мя.


Аще праведен бяше он и непорочен паче всех, и не убеже ловления льстиваго и сети; ты же, грехолюбива сущи, окаянная душе, что сотвориши, аще чесому от недоведомых случится наити тебе?


\itshape Если он, будучи праведным и безукоризненным более всех, не избежал козней и сетей обольстителя диавола, то что сделаешь, ты, грехолюбивая несчастная душа, если что-нибудь неожиданное постигнет тебя?\normalfont{}


\itshape Припев:\normalfont{} Помилуй мя, Боже, помилуй мя.


Тело осквернися, дух окаляся, весь острупихся, но яко врач, Христе, обоя покаянием моим уврачуй, омый, очисти, покажи, Спасе мой, паче снега чистейша.


\itshape Тело мое осквернено, дух грязен, весь я покрыт струпами, но Ты, Христе, как врач, уврачуй и то и другое моим покаянием, омой, очисти, яви меня чище снега, Спаситель мой.\normalfont{}


\itshape Припев:\normalfont{} Помилуй мя, Боже, помилуй мя.


Тело Твое и кровь Распинаемый о всех положил еси, Слове: тело убо, да мя обновиши, кровь, да омыеши мя. Дух же предал еси, да мя приведеши, Христе, Твоему Родителю.


\itshape Твое тело и Кровь, Слово, Ты принес в жертву за всех при распятии; Тело "--- чтобы воссоздать меня, Кровь "--- чтобы омыть меня, и Дух Ты, Христе, предал, чтобы привести меня к Твоему Отцу.\normalfont{}


\itshape Припев:\normalfont{} Помилуй мя, Боже, помилуй мя.


Соделал еси спасение посреде земли, Щедре, да спасемся. Волею на древе распялся еси, Едем затворенный отверзеся, горняя и дольняя тварь, языцы вси, спасени, покланяются Тебе.


\itshape Посреди земли Ты устроил спасение, Милосердный, чтобы мы спаслись; Ты добровольно распялся на древе; Едем затворенный открылся; Тебе поклоняются небесные и земные и все спасенные Тобою народы. Пс. 73:12\normalfont{}


\itshape Припев:\normalfont{} Помилуй мя, Боже, помилуй мя.


Да будет ми купель кровь из ребр Твоих, вкупе и питие, источившее воду оставления, да обоюду очищаюся, помазуяся и пия, яко помазание и питие, Слове, животочная Твоя словеса.


\itshape Да будет мне омовением Кровь из ребр Твоих и вместе питием, источившая оставление грехов, чтобы мне и тем и другим очищаться, Слове, помазуясь и напояясь животворными Твоими словами, как мазью и питием. Ин. 19:33–34\normalfont{}


\itshape Припев:\normalfont{} Помилуй мя, Боже, помилуй мя.


Наг есмь чертога, наг есмь и брака, купно и вечери; светильник угасе, яко безъелейный, чертог заключися мне спящу, вечеря снедеся, аз же по руку и ногу связан, вон низвержен есмь.


\itshape Я лишен брачного чертога, лишен и брака, и вечери; светильник, как без елея, погас; чертог закрылся во время моего сна, вечеря окончена, а я, связанный по рукам и ногам, извержен вон. Мф. 25:1–13; Мф. 22:11–13; Лк. 12:35–37; Лк. 13:24–27; Лк. 14:7–24\normalfont{}


\itshape Припев:\normalfont{} Помилуй мя, Боже, помилуй мя.


Чашу Церковь стяжа, ребра Твоя живоносная, из нихже сугубыя нам источи токи оставления и разума во образ Древняго и Новаго, двоих вкупе Заветов, Спасе наш.


\itshape Церковь приобрела себе Чашу в живоносном ребре Твоем, из которого проистек нам двойной поток оставления грехов и разумения, Спаситель наш, в образ обоих Заветов, Ветхого и Нового.\normalfont{}


\itshape Припев:\normalfont{} Помилуй мя, Боже, помилуй мя.


Время живота моего мало и исполнено болезней и лукавства, но в покаянии мя приими и в разум призови, да не буду стяжание ни брашно чуждему, Спасе, Сам мя ущедри.


\itshape Время жизни моей кратко и исполнено огорчений и пороков, но прими меня в покаянии и призови к познанию истины, чтобы не сделаться мне добычею и пищею врага, Спаситель, умилосердись надо мною. Быт. 47:9\normalfont{}


\itshape Припев:\normalfont{} Помилуй мя, Боже, помилуй мя.


Высокоглаголив ныне есмь, жесток же и сердцем, вотще и всуе, да не с фарисеем осудиши мя. Паче же мытарево смирение подаждь ми, Едине Щедре, Правосуде, и сему мя сочисли.


\itshape Высокомерен я ныне на словах, дерзок и в сердце, напрасно и тщетно; не осуди меня с фарисеем, но даруй мне смирение мытаря и к нему причисли, Один Милосердный и Правосудный.\normalfont{}


\itshape Припев:\normalfont{} Помилуй мя, Боже, помилуй мя.


Согреших, досадив сосуду плоти моея, вем, Щедре, но в покаянии мя приими и в разум призови, да не буду стяжание ни брашно чуждему, Спасе, Сам мя ущедри.


\itshape Знаю, Милосердный, согрешил я, осквернив сосуд моей плоти, но прими меня в покаянии и призови к познанию истины, чтобы не сделаться мне добычею и пищею врага; Сам, Ты, Спаситель, умилосердись надо мною.\normalfont{}


\itshape Припев:\normalfont{} Помилуй мя, Боже, помилуй мя.


Самоистукан бых страстьми, душу мою вредя, Щедре, но в покаянии мя приими и в разум призови, да не буду стяжание ни брашно чуждему, Спасе, Сам мя ущедри.


\itshape Истуканом я сделал сам себя, исказив душу свою страстями, Милосердный; но прими меня в покаянии и призови к познанию истины, чтобы не сделаться мне добычею и пищею врага; Сам, Ты, Спаситель, умилосердись надо мною.\normalfont{}


\itshape Припев:\normalfont{} Помилуй мя, Боже, помилуй мя.


Не послушах гласа Твоего, преслушах Писание Твое, Законоположника, но в покаянии мя приими и в разум призови, да не буду стяжание ни брашно чуждему, Спасе, Сам мя ущедри.


\itshape Не послушал я голоса Твоего, нарушил Писание Твое, Законодатель; но прими меня в покаянии и призови к познанию истины, чтобы не сделаться мне добычею и пищею врага; Сам, Ты, Спаситель, умилосердись надо мною.\normalfont{}


\itshape Припев:\normalfont{} Преподобная мати Марие, моли Бога о нас.


Безплотных жительство в плоти преходящи, благодать, преподобная, к Богу велию воистинну прияла еси, верно о чтущих тя предстательствуй. Темже молим тя, от всяких напастей и нас молитвами твоими избави.


\itshape Припев:\normalfont{} Преподобная мати Марие, моли Бога о нас.


Великих безместий во глубину низведшися, неодержима была еси; но востекла еси помыслом лучшим к крайней деяньми яве добродетели преславно, ангельское естество, Марие, удививши.


\itshape Увлекшись в глубину великих пороков, ты, Мария, не погрязла в ней, но высшим помыслом через деятельность явно поднялась до совершенной добродетели, дивно изумив ангельскую природу.\normalfont{}


\itshape Припев:\normalfont{} Преподобне отче Андрее, моли Бога о нас.


Андрее, отеческая похвало, молитвами твоими не престай, моляся, предстоя Троице Пребожественней, яко да избавимся мучения, любовию предстателя тя Божественнаго, всеблаженне, призывающии, Криту удобрение.


Слава Отцу и Сыну и Святому Духу.


Нераздельное существом, неслитное Лицы богословлю Тя, Троическое Едино Божество, яко Единоцарственное и Сопрестольное, вопию Ти песнь великую, в вышних трегубо песнословимую.


\itshape Нераздельным по существу, неслиянным в Лицах богословски исповедую Тебя, Троичное Единое Божество, Соцарственное и Сопрестольное; возглашаю Тебе великую песнь, в небесных обителях троекратно воспеваемую. Ис. 6:1–3\normalfont{}


И ныне и присно и во веки веков. Аминь.


И раждаеши, и девствуеши, и пребываеши обоюду естеством Дева, Рождейся обновляет законы естества, утроба же раждает нераждающая. Бог идеже хощет, побеждается естества чин: творит бо елика хощет.


\itshape И рождаешь Ты, и остаешься Девою, в обоих случаях сохраняя по естеству девство. Рожденный Тобою обновляет законы природы, а девственное чрево рождает; когда хочет Бог, то нарушается порядок природы, ибо Он творит, что хочет.\normalfont{}





\bfseries Песнь 5\normalfont{}


\itshape Ирмо́с:\normalfont{}


От нощи утренююща, Человеколюбче, просвети, молюся, и настави и мене на повеления Твоя, и научи мя Спасе, творити волю Твою.


\itshape От ночи бодрствующего, просвети меня, молю, Человеколюбец, путеводи меня в повелениях Твоих и научи меня, Спаситель, исполнять Твою волю. Пс. 62:2; Пс. 118:35\normalfont{}


\itshape Припев:\normalfont{} Помилуй мя, Боже, помилуй мя.


В нощи житие мое преидох присно, тьма бо бысть, и глубока мне мгла, нощь греха, но яко дне сына, Спасе, покажи мя.


\itshape Жизнь свою я постоянно проводил в ночи, ибо мраком и глубокою мглою была для меня ночь греха; но покажи меня сыном дня, Спаситель. Еф. 5:8; 1 Фес. 5:5\normalfont{}


\itshape Припев:\normalfont{} Помилуй мя, Боже, помилуй мя.


Рувима подражая окаянный аз, содеях беззаконный и законопреступный совет на Бога Вышняго, осквернив ложе мое, яко отчее он.


\itshape Подобно Рувиму я, несчастный, совершил преступное и беззаконное дело пред Всевышним Богом, осквернив ложе мое, как тот "--- отчее. Быт. 35:22; Быт. 49:3–4\normalfont{}


\itshape Припев:\normalfont{} Помилуй мя, Боже, помилуй мя.


Исповедаюся Тебе Христе Царю: согреших, согреших, яко прежде Иосифа братия продавшии, чистоты плод и целомудрия.


\itshape Исповедаюсь Тебе, Христос-Царь: согрешил я, согрешил, как некогда братья, продавшие Иосифа, "--- плод чистоты и целомудрия. Быт. 37:28\normalfont{}


\itshape Припев:\normalfont{} Помилуй мя, Боже, помилуй мя.


От сродников праведная душа связася, продася в работу сладкий, во образ Господень; ты же вся, душе, продалася еси злыми твоими.


\itshape Сродниками предана была душа праведная; возлюбленный продан в рабство, прообразуя Господа; ты же, душа, сама всю продала себя своим порокам.\normalfont{}


\itshape Припев:\normalfont{} Помилуй мя, Боже, помилуй мя.


Иосифа праведнаго и целомудреннаго ума подражай, окаянная и неискусная душе, и не оскверняйся безсловесными стремленьми, присно беззаконнующи.


\itshape Подражай праведному Иосифу и уму его целомудренному, несчастная и невоздержанная душа, не оскверняйся и не беззаконствуй всегда безрассудными стремлениями.\normalfont{}


\itshape Припев:\normalfont{} Помилуй мя, Боже, помилуй мя.


Аще и в рове поживе иногда Иосиф, Владыко Господи, но во образ погребения и востания Твоего, аз же что Тебе когда сицевое принесу?


\itshape Владыко Господи, Иосиф был некогда во рву, но в прообраз Твоего погребения и воскресения; принесу ли когда-либо что подобное Тебе я?\normalfont{}


\itshape Припев:\normalfont{} Помилуй мя, Боже, помилуй мя.


Моисеов слышала еси ковчежец, душе, водами, волнами носимь речными, яко в чертозе древле бегающий дела, горькаго совета фараонитска.


\itshape Ты слышала, душа, о корзинке с Моисеем, в древности носимом водами в волнах реки, как в чертоге, избегшем горестного последствия замысла фараонова. Исх. 2:3\normalfont{}


\itshape Припев:\normalfont{} Помилуй мя, Боже, помилуй мя.


Аще бабы слышала еси, убивающыя иногда безвозрастное мужеское, душе окаянная, целомудрия деяние, ныне, яко великий Моисей, сси премудрость.


\itshape Если ты слышала, несчастная душа, о повивальных бабках, некогда умерщвлявших новорожденных младенцев мужского пола, то теперь, подобно Моисею, млекопитайся мудростью. Исх. 1:8–22\normalfont{}


\itshape Припев:\normalfont{} Помилуй мя, Боже, помилуй мя.


Яко Моисей великий египтянина, ума уязвивши окаянная, не убила еси, душе; и како вселишися, глаголи, в пустыню страстей покаянием?


\itshape Подобно великому Моисею, поразившему египтянина, ты, не умертвила, несчастная душа, гордого ума; как же, скажи, вселишься ты в пустыню от страстей через покаяние? Исх. 2:11–12\normalfont{}


\itshape Припев:\normalfont{} Помилуй мя, Боже, помилуй мя.


В пустыню вселися великий Моисей; гряди убо, подражай того житие, да и в купине Богоявления, душе, в видении будеши.


\itshape Великий Моисей поселился в пустыне; иди и ты, душа, подражай его жизни, чтобы и тебе увидеть в терновом кусте явление Бога. Исх. 3:1–3\normalfont{}


\itshape Припев:\normalfont{} Помилуй мя, Боже, помилуй мя.


Моисеов жезл воображай, душе, ударяющий море и огустевающий глубину, во образ Креста Божественнаго: имже можеши и ты великая совершити.


\itshape Изобрази, душа, Моисеев жезл, поражающий море и огустевающий глубину, в знамение Божественного Креста, которым и ты можешь совершить великое. Исх. 14:21–22\normalfont{}


\itshape Припев:\normalfont{} Помилуй мя, Боже, помилуй мя.


Аарон приношаше огнь Богу непорочный, нелестный; но Офни и Финеес, яко ты душе, приношаху чуждее Богу, оскверненное житие.


\itshape Аарон приносил Богу огонь чистый, беспримесный, но Офни и Финеес принесли, как ты, душа, отчужденную от Бога нечистую жизнь. 1 Цар. 2:12–13\normalfont{}


\itshape Припев:\normalfont{} Помилуй мя, Боже, помилуй мя.


Яко тяжкий нравом, фараону горькому бых, Владыко, Ианни и Иамври, душею и телом, и погружен умом, но помози ми.


\itshape По упорству я стал как жестокий нравом фараон, Владыко, по душе и телу я "--- Ианний и Иамврий, и по уму погрязший, но помоги мне. Исх. 7:11; 2 Тим. 3:8\normalfont{}


\itshape Припев:\normalfont{} Помилуй мя, Боже, помилуй мя.


Калу примесихся, окаянный, умом, омый мя, Владыко, банею моих слез, молю Тя, плоти моея одежду убелив яко снег.


\itshape Загрязнил я, несчастный, свой ум, но омой меня, Владыко, в купели слез моих молю Тебя, и убели, как снег, одежду плоти моей.\normalfont{}


\itshape Припев:\normalfont{} Помилуй мя, Боже, помилуй мя.


Аще испытаю моя дела, Спасе, всякаго человека превозшедша грехами себе зрю, яко разумом мудрствуяй согреших, не неведением.


\itshape Когда исследую свои дела, Спаситель, то вижу, что превзошел я грехами всех людей, ибо я грешил с разумным сознанием, а не по неведению.\normalfont{}


\itshape Припев:\normalfont{} Помилуй мя, Боже, помилуй мя.


Пощади, пощади, Господи, создание Твое, согреших, ослаби ми, яко естеством чистый Сам сый Един, и ин разве Тебе никтоже есть кроме скверны.


\itshape Пощади, Господи, пощади, создание Твое: я согрешил, прости мне, ибо только Ты один чист по природе, и никто, кроме Тебя, не чужд нечистоты.\normalfont{}


\itshape Припев:\normalfont{} Помилуй мя, Боже, помилуй мя.


Мене ради Бог сый, вообразился еси в мя, показал еси чудеса, исцелив прокаженныя и разслабленнаго стягнув, кровоточивыя ток уставил еси, Спасе, прикосновением риз.


\itshape Ради меня, будучи Богом, Ты принял мой образ, Спаситель, и, совершая чудеса, исцелял прокаженных, укреплял расслабленных, остановил кровотечение у кровоточивой прикосновением одежды. Мф. 9:20; Мк. 5:25–27; Лк. 8:43–44\normalfont{}


\itshape Припев:\normalfont{} Помилуй мя, Боже, помилуй мя.


Кровоточивую исцели прикосновением края ризна Господь, прокаженная очисти, слепыя и хромыя просветив исправи, глухия же и немыя и ничащия низу исцели словом, да ты спасешися, окаянная душе.


\itshape Господь исцелил кровоточивую через прикосновение к одежде Его, очистил прокаженных, дал прозрение слепым, исправил хромых, глухих, немых и уврачевал словом скорченную, чтобы ты спаслась, несчастная душа. Мф. 9:20–22; Мф. 11:4–5; Лк. 13:10–13\normalfont{}


\itshape Припев:\normalfont{} Помилуй мя, Боже, помилуй мя.


Низу сничащую подражай, о душе, прииди, припади к ногама Иисусовыма, да тя исправит, и да ходиши право стези Господни.


\itshape Подражай, душа, скорченной жене, приди, припади к ногам Иисуса, чтобы Он исправил тебя и ты могла ходить прямо по стезям Господним. Лк. 13:11–13\normalfont{}


\itshape Припев:\normalfont{} Помилуй мя, Боже, помилуй мя.


Аще и кладязь еси глубокий, Владыко, источи ми воду из пречистых Твоих жил, да, яко самаряныня, не ктому пияй, жажду жизни бо струи источаеши.


\itshape Если Ты "--- и глубокий колодец, Владыко, то источи мне струи из пречистых ребр Своих, чтобы я, как самарянка, испив, уже не жаждал, ибо Ты источаешь потоки жизни. Ин. 4:11–15\normalfont{}


\itshape Припев:\normalfont{} Помилуй мя, Боже, помилуй мя.


Силоам да будут ми слезы моя, Владыко Господи, да умыю и аз зеницы сердца, и вижду Тя, умна Света превечна.


\itshape Силоамом да будут мне слезы мои, Владыко Господи, чтобы и мне омыть очи сердца и умственно созерцать Тебя, Предвечный Свет. Ин. 9:7\normalfont{}


\itshape Припев:\normalfont{} Преподобная мати Марие, моли Бога о нас.


Несравненным желанием, всебогатая, древу возжелевши поклонитися животному, сподобилася еси желания, сподоби убо и мене улучити вышния славы.


\itshape С чистой любовию возжелав поклониться Древу Жизни, всеблаженная, ты удостоилась желаемого; удостой же и меня достигнуть высшей славы.\normalfont{}


\itshape Припев:\normalfont{} Преподобная мати Марие, моли Бога о нас.


Струи Иорданския прешедши, обрела еси покой безболезненный, плоти сласти избежавши, еяже и нас изми твоими молитвами, преподобная.


\itshape Ты перешла поток Иорданский и приобрела покой безболезненный, оставив плотское удовольствие, от которого избавь и нас твоими молитвами, преподобная.\normalfont{}


\itshape Припев:\normalfont{} Преподобне отче Андрее, моли Бога о нас.


Яко пастырей изряднейша, Андрее премудре, избранна суща тя, любовию велиею и страхом молю, твоими молитвами спасение улучити и жизнь вечную.


\itshape Как превосходнейшего из пастырей, премудрый Андрей, как избранного молю тебя с великой любовью и благоговением, чтобы мне, по молитвам твоим, получить спасение и жизнь вечную.\normalfont{}


Слава Отцу и Сыну и Святому Духу.


Тя, Троице, славим Единаго Бога: Свят, Свят, Свят еси, Отче, Сыне и Душе, Простое Существо, Единице присно покланяемая.


\itshape Тебя, Пресвятая Троица, прославляем за Единого Бога: Свят, Свят, Свят Отец, Сын и Дух, Простое Существо, Единица вечно поклоняемая.\normalfont{}


И ныне и присно и во веки веков. Аминь.


Из Тебе облечеся в мое смешение, нетленная, безмужная Мати Дево, Бог, создавый веки, и соедини Себе человеческое естество.


\itshape В Тебе, Нетленная, не познавшая мужа Матерь-Дево, облекся в мой состав сотворивший мир Бог и соединил с Собою человеческую природу.\normalfont{}





\bfseries Песнь 6\normalfont{}


\itshape Ирмо́с:\normalfont{}


Возопих всем сердцем моим к щедрому Богу, и услыша мя от ада преисподняго, и возведе от тли живот мой.


\itshape От всего сердца моего я воззвал к милосердному Богу, и Он услышал меня из ада преисподнего и воззвал жизнь мою от погибели. Иона 2:3\normalfont{}


\itshape Припев:\normalfont{} Помилуй мя, Боже, помилуй мя.


Слезы, Спасе, очию моею и из глубины воздыхания чисте приношу, вопиющу сердцу: Боже, согреших Ти, очисти мя.


\itshape Искренно приношу Тебе, Спаситель, слезы очей моих и воздыхания из глубины сердца, взывающего: Боже, согрешил я пред Тобою, умилосердись надо мною.\normalfont{}


\itshape Припев:\normalfont{} Помилуй мя, Боже, помилуй мя.


Уклонилася еси, душе, от Господа твоего, якоже Дафан и Авирон, но пощади, воззови из ада преисподняго, да не пропасть земная тебе покрыет.


\itshape Уклонилась ты, душа, от Господа своего, как Дафан и Авирон, но воззови из ада преисподнего: пощади!, чтобы пропасть земная не поглотила тебя Чис. 16:1–3; Чис. 16:28–31\normalfont{}


\itshape Припев:\normalfont{} Помилуй мя, Боже, помилуй мя.


Яко юница, душе, разсвирепевшая, уподобилася еси Ефрему, яко серна от тенет сохрани житие, вперивши деянием ум и зрением.


\itshape Рассвирепев, как телица, ты, душа, уподобилась Ефрему, но как серна спасай от тенет свою жизнь, окрылив ум деятельностью и созерцанием. Иер. 31:18; Ос. 10:11\normalfont{}


\itshape Припев:\normalfont{} Помилуй мя, Боже, помилуй мя.


Рука нас Моисеова да уверит душе, како может Бог прокаженное житие убелити и очистити, и не отчайся сама себе, аще и прокаженна еси.


\itshape Моисеева рука да убедит нас, душа, как Бог может убелить и очистить прокаженную жизнь, и не отчаивайся сама за себя, хотя ты и поражена проказою. Исх. 4:6–7\normalfont{}


\itshape Припев:\normalfont{} Помилуй мя, Боже, помилуй мя.


Волны, Спасе, прегрешений моих, яко в мори Чермнем возвращающеся, покрыша мя внезапу, яко египтяны иногда и тристаты.


\itshape Волны грехов моих, Спаситель, обратившись, как в Чермном море, внезапно покрыли меня, как некогда египтян и их всадников. Исх. 14:26–28; Исх. 15:4–5\normalfont{}


\itshape Припев:\normalfont{} Помилуй мя, Боже, помилуй мя.


Неразумное, душе, произволение имела еси, яко прежде Израиль: Божественныя бо манны предсудила еси безсловесно любосластное страстей объядение.


\itshape Нерассудителен твой выбор, душа, как у древнего Израиля, ибо ты безрассудно предпочла Божественной манне сластолюбивое пресыщение страстями. Чис. 21:5\normalfont{}


\itshape Припев:\normalfont{} Помилуй мя, Боже, помилуй мя.


Кладенцы, душе, предпочла еси хананейских мыслей паче жилы камене, из негоже премудрости река, яко чаша проливает токи богословия.


\itshape Колодцы хананейских помыслов ты, душа, предпочла камню с источником, из которого река премудрости, как чаша, изливает струи богословия. Быт. 21:25; Исх. 17:3; Исх. 17:6\normalfont{}


\itshape Припев:\normalfont{} Помилуй мя, Боже, помилуй мя.


Свиная мяса и котлы и египетскую пищу, паче Небесныя, предсудила еси, душе моя, якоже древле неразумнии людие в пустыни.


\itshape Свиное мясо, котлы и египетскую пищу ты предпочла пище небесной, душа моя, как древний безрассудный народ в пустыне. Исх. 16:3\normalfont{}


\itshape Припев:\normalfont{} Помилуй мя, Боже, помилуй мя.


Яко удари Моисей, раб Твой, жезлом камень, образно животворивая ребра Твоя прообразоваше, из нихже вси питие жизни, Спасе, почерпаем.


\itshape Как Моисей, раб Твой, ударив жезлом о камень, таинственно предызобразил животворное ребро Твое, Спаситель, из которого все мы почерпаем питие жизни.\normalfont{}


\itshape Припев:\normalfont{} Помилуй мя, Боже, помилуй мя.


Испытай, душе, и смотряй, якоже Иисус Навин, обетования землю, какова есть, и вселися в ню благозаконием.


\itshape Исследуй, душа, подобно Иисусу Навину, и обозри обещанную землю, какова она, и поселись в ней путем исполнения закона.\normalfont{}


\itshape Припев:\normalfont{} Помилуй мя, Боже, помилуй мя.


Востани и побори, яко Иисус Амалика, плотския страсти, и гаваониты, лестныя помыслы, присно побеждающи.


\itshape Восстань и побеждай плотские страсти, как Иисус Амалика, всегда побеждая и гаваонитян "--- обольстительные помыслы. Исх. 17:8–9; Исх. 17:13; Нав. 8:21\normalfont{}


\itshape Припев:\normalfont{} Помилуй мя, Боже, помилуй мя.


Прейди, времене текущее естество, яко прежде ковчег, и земли оныя буди во одержании обетования, душе, Бог повелевает.


\itshape Душа, Бог повелевает: перейди, как некогда ковчег Иордан, текущее по своему существу время и сделайся обладательницею обещанной земли. Нав. 3:17\normalfont{}


\itshape Припев:\normalfont{} Помилуй мя, Боже, помилуй мя.


Яко спасл еси Петра, возопивша спаси, предварив мя, Спасе, от зверя избави, простер Твою руку, и возведи из глубины греховныя.


\itshape Подобно тому как Ты спас Петра, воззвавшего, поспеши, Спаситель, спасти и меня, избавь меня от чудовища, простерши Свою руку, и выведи из глубины греха. Мф. 14:28–31\normalfont{}


\itshape Припев:\normalfont{} Помилуй мя, Боже, помилуй мя.


Пристанище Тя вем утишное, Владыко, Владыко Христе, но от незаходимых глубин греха и отчаяния мя, предварив, избави.


\itshape Тихое пристанище вижу в Тебе, Владыка, Владыка Христе, поспеши же избавить меня от непроходимых глубин греха и отчаяния.\normalfont{}


\itshape Припев:\normalfont{} Помилуй мя, Боже, помилуй мя.


Аз есмь, Спасе, юже погубил еси древле царскую драхму; но вжег светильник, Предтечу Твоего, Слове, взыщи и обрящи Твой образ.


\itshape Я "--- та драхма с царским изображением, которая с древности потеряна у Тебя, Спаситель, но, засветив светильник "--- Предтечу Своего, Слове, поищи и найди Свой образ. Лк. 15:8–9\normalfont{}


\itshape Припев:\normalfont{} Преподобная мати Марие, моли Бога о нас.


Да страстей пламень угасиши, слез капли источила еси присно, Марие, душею распаляема, ихже благодать подаждь и мне, твоему рабу.


\itshape Чтобы угасить пламень страстей, ты, Мария, пылая душой, непрестанно проливала потоки слез, преизобилие которых даруй и мне, рабу твоему.\normalfont{}


\itshape Припев:\normalfont{} Преподобная мати Марие, моли Бога о нас.


Безстрастие небесное стяжала еси крайним на земли житием, Мати. Темже тебе поющым страстей избавитися молитвами твоими молися.


\itshape Возвышеннейшим образом жизни на земле, ты, матерь, приобрела небесное бесстрастие; поэтому ходатайствуй, чтобы воспевающие тебя избавились от страстей по твоим молитвам.\normalfont{}


\itshape Припев:\normalfont{} Преподобне отче Андрее, моли Бога о нас.


Критскаго тя пастыря и председателя и вселенныя молитвенника ведый, притекаю, Андрее, и вопию ти: изми мя, отче, из глубины греха.


\itshape Тебя, критского пастыря и главу, и молитвенника за всю вселенную зная, прибегаю к тебе, Андрей, и взываю: «Выведи меня отче, из глубины греха!»\normalfont{}


Слава Отцу и Сыну и Святому Духу.


Троица есмь Проста, Нераздельна, раздельна Личне и Единица есмь естеством соединена, Отец глаголет, и Сын, и Божественный Дух.


\itshape Я "--- Троица Несоставная, Нераздельная, раздельная в Лицах, и Единица, соединенная по существу; свидетельствует Отец, Сын и Божественный Дух.\normalfont{}


И ныне и присно и во веки веков. Аминь.


Утроба Твоя Бога нам роди, воображенна по нам: Егоже, яко Создателя всех, моли, Богородице, да Твоими молитвами оправдимся.


\itshape Чрево Твое родило нам Бога, принявшего наш образ; Его, как Создателя всего мира, моли, Богородица, чтобы по молитвам Твоим нам оправдаться.\normalfont{}


\itshape Катавасия:\normalfont{}


Возопих всем сердцем моим к щедрому Богу, и услыша мя от ада преисподняго, и возведе от тли живот мой.


\itshape От всего сердца моего я воззвал к милосердному Богу, и Он услышал меня из ада преисподнего и воззвал жизнь мою от погибели. Иона 2:3\normalfont{}





\bfseries Кондак, глас 6:\normalfont{}


Душе моя, душе моя, востани, что спиши? Конец приближается, и имаши смутитися; воспряни убо, да пощадит тя Христос Бог, везде сый и вся исполняяй.


\itshape Душа моя, душа моя, восстань, что ты спишь? Конец приближается, и ты смутишься; пробудись же, чтобы пощадил тебя Христос Бог, Вездесущий и все наполняющий.\normalfont{}


\itshape Икос:\normalfont{}


Христово врачевство видя отверсто и от сего Адаму истекающее здравие, пострада, уязвися диавол и, яко бедствуя, рыдаше и своим другом возопи: что сотворю Сыну Мариину, убивает мя Вифлеемлянин, Иже везде сый и вся исполняяй.





\bfseries Блаженны, с поклонами, глас 6:\normalfont{}


Во Царствии Твоем помяни нас, Господи.


Разбойника, Христе, рая жителя сотворил еси, на кресте Тебе возопивша: помяни мя; того покаянию сподоби и мене, недостойнаго.


Блажени нищии духом, яко тех есть Царство Небесное.


Маноя слышавши древле, душе моя, Бога в явлении бывша и из неплодове тогда приемша плод обетования, того благочестие подражай.


\itshape Суд. 13:2–24\normalfont{}


Блажени плачущии, яко тии утешатся.


Сампсоновой поревновавши лености, главу остригла еси, душе, дел твоих, предавши иноплеменником любосластием целомудренную жизнь и блаженную.


\itshape Суд. 16:4–21\normalfont{}


Блажени кротции, яко тии наследят землю.


Прежде челюстию ослею победивый иноплеменники, ныне пленение ласкосердству страстному обретеся; но избегни, душе моя, подражания, деяния и слабости.


\itshape Суд. 15:15\normalfont{}


Блажени алчущии и жаждущии правды, яко тии насытятся.


Варак и Иеффай военачальницы, судии Израилевы предпочтени быша, с нимиже Деворра мужеумная; тех доблестьми, душе, вмужившися, укрепися.


\itshape Суд. 4:4–16; Суд. 11:9–10; Суд. 11:32–33; Суд. 12:4–7\normalfont{}


Блажени милостивии, яко тии помиловани будут.


Иаилино храбрство познала еси, душе моя, Сисара древле прободшую и спасение соделавшую древом острым, слышиши, имже тебе крест образуется.


\itshape Суд. 4:17–22\normalfont{}


Блажени чистии сердцем, яко тии Бога узрят.


Пожри, душе, жертву похвальную, деяние, яко дщерь, принеси от Иеффаевы чистейшую и заколи, яко жертву, страсти плотския Господеви твоему.


\itshape Суд. 11:30–40\normalfont{}


Блажени миротворцы, яко тии сынове Божии нарекутся.


Гедеоново руно помышляй, душе моя, с небесе росу подыми и приникни, якоже пес, и пий воду, от закона текущую, изгнетением письменным.


\itshape Суд. 6:37–38; Суд. 7:4–7\normalfont{}


Блажени изгнани правды ради, яко тех есть Царство Небесное.


Илии священника осуждение, душе моя, восприяла еси, лишением ума приобретши страсти себе, якоже он чада, делати беззаконная.


\itshape 1 Цар. 2:22; 1 Цар. 2:31–34\normalfont{}


Блажени есте, егда поносят вам и изженут и рекут всяк зол глагол на вы лжуще, Мене ради.


В судиях левит небрежением свою жену дванадесятим коленом раздели, душе моя, да скверну обличит от Вениамина беззаконную.


\itshape Суд. 19:27–29\normalfont{}


Радуйтеся и веселитеся, яко мзда ваша многа на Небесех.


Любомудренная Анна молящися, устне убо двизаше ко хвалению, глас же ея не слышашеся, но обаче неплодна сущи, сына молитвы раждает достойна.


\itshape 1 Цар. 1:9–13; 1 Цар. 1:19–20\normalfont{}


Помяни нас, Господи, егда приидеши во Царствии Твоем.


В судиях спричтеся Аннино порождение, великий Самуил, егоже воспитала Армафема в дому Господни; тому поревнуй, душе моя, и суди прежде инех дела твоя.


\itshape 1 Цар. 1:25–28\normalfont{}


Помяни нас, Владыко, егда приидеши во Царствии Твоем.


Давид на царство избран, царски помазася рогом Божественного мира; ты убо, душе моя, аще хощеши вышняго Царствия, миром помажися слезами.


\itshape 1 Цар. 16:1–13\normalfont{}


Помяни нас, Святый, егда приидеши во Царствии Твоем.


Помилуй создание Твое, Милостиве, ущедри руку Твоею творение и пощади вся согрешившия, и мене паче всех, Твоих презревшаго повелений.


Слава Отцу и Сыну и Святому Духу.


Безначальну и рождению же и происхождению Отцу покланяюся рождшему, Сына славлю рожденнаго, пою сопросиявшаго Отцу же и Сыну Духа Святаго.


И ныне и присно и во веки веков. Аминь.


Преестественному Рождеству Твоему покланяемся, по естеству славы Младенца Твоего не разделяюще, Богородительнице: Иже бо Един Лицем, сугубыми исповедуется естествы.





\bfseries Песнь 7\normalfont{}


\itshape Ирмо́с:\normalfont{}


Согрешихом, беззаконновахом, неправдовахом пред Тобою, ниже соблюдохом, ниже сотворихом, якоже заповедал еси нам; но не предаждь нас до конца, отцев Боже.


\itshape Мы согрешили, жили беззаконно, неправо поступали пред Тобою, не сохранили, не исполнили, что Ты заповедал нам; но не оставь нас до конца, Боже отцов. Дан. 9:5–6\normalfont{}


\itshape Припев:\normalfont{} Помилуй мя, Боже, помилуй мя.


Согреших, беззаконновах и отвергох заповедь Твою, яко во гресех произведохся, и приложих язвам струпы себе; но Сам мя помилуй, яко благоутробен, отцев Боже.


\itshape Я согрешил, жил в беззакониях и нарушил заповедь Твою, ибо я рожден в грехах и к язвам своим приложил еще раны, но Сам Ты помилуй меня, как Милосердный Боже отцов.\normalfont{}


\itshape Припев:\normalfont{} Помилуй мя, Боже, помилуй мя.


Тайная сердца моего исповедах Тебе, Судии моему, виждь мое смирение, виждь и скорбь мою, и вонми суду моему ныне, и Сам мя помилуй, яко благоутробен, отцев Боже.


\itshape Тайны сердца моего я открыл пред Тобою, Судьей моим; воззри на смирение мое, воззри и на скорбь мою, обрати внимание на мое ныне осуждение и Сам помилуй меня, как Милосердный, Боже отцов. Пс. 37:19; Пс. 24:18; Пс. 34:23\normalfont{}


\itshape Припев:\normalfont{} Помилуй мя, Боже, помилуй мя.


Саул иногда, яко погуби отца своего, душе, ослята, внезапу царство обрете к прослутию; но блюди, не забывай себе, скотския похоти твоя произволивши паче Царства Христова.


\itshape Саул, некогда потеряв ослиц своего отца, неожиданно с известием о них получил царство; душа, не забывайся, предпочитая свои скотские стремления Христову Царству. 1 Цар. 9:1–27; 1 Цар. 10:1\normalfont{}


\itshape Припев:\normalfont{} Помилуй мя, Боже, помилуй мя.


Давид иногда Богоотец, аще и согреши сугубо, душе моя, стрелою убо устрелен быв прелюбодейства, копием же пленен быв убийства томлением; но ты сама тяжчайшими делы недугуеши, самохотными стремленьми.


\itshape Если богоотец Давид некогда и вдвойне согрешил, будучи уязвлен стрелою прелюбодеяния, сражен был копьем мщения за убийства; но ты, душа моя, сама страдаешь более тяжко, нежели этими делами, произвольными стремлениями. 2 Цар. 11:2–6; 2 Цар. 11:14–15\normalfont{}


\itshape Припев:\normalfont{} Помилуй мя, Боже, помилуй мя.


Совокупи убо Давид иногда беззаконию беззаконие, убийству же любодейство растворив, покаяние сугубое показа абие; но сама ты, лукавнейшая душе, соделала еси, не покаявшися Богу.


\itshape Давид некогда присовокупил беззаконие к беззаконию, ибо с убийством соединил прелюбодеяние, но скоро принес и усиленное покаяние, а ты, коварнейшая душа, совершив бОльшие грехи, не раскаялась пред Богом.\normalfont{}


\itshape Припев:\normalfont{} Помилуй мя, Боже, помилуй мя.


Давид иногда вообрази, списав яко на иконе песнь, еюже деяние обличает, еже содея, зовый: помилуй мя, Тебе бо Единому согреших всех Богу, Сам очисти мя.


\itshape Давид некогда, изображая как бы на картине, начертал песнь, которой обличает совершенный им проступок, взывая: помилуй мя, ибо согрешил я пред Тобою, Одним, Богом всех; Сам очисти меня. Пс. 50:3–6\normalfont{}


\itshape Припев:\normalfont{} Помилуй мя, Боже, помилуй мя.


Кивот яко ношашеся на колеснице, Зан оный, егда превращшуся тельцу, точию коснуся, Божиим искусися гневом; но того дерзновения убежавши, душе, почитай Божественная честне.


\itshape Когда ковчег везли на колеснице, то Оза, когда вол свернул в сторону, лишь только прикоснулся, испытал на себе гнев Божий, но, душа, избегая его дерзости, благоговейно почитай Божественное. 2 Цар. 6:6–7\normalfont{}


\itshape Припев:\normalfont{} Помилуй мя, Боже, помилуй мя.


Слышала еси Авессалома, како на естество воста, познала еси того скверная деяния, имиже оскверни ложе Давида отца; но ты подражала еси того страстная и любосластная стремления.


\itshape Ты слышала об Авессаломе, как он восстал на самую природу, знаешь гнусные его деяния, которыми он обесчестил ложе отца "--- Давида; но ты сама подражала его страстным и сластолюбивым порывам. 2 Цар. 15:1–37; 2 Цар. 16:20–22\normalfont{}


\itshape Припев:\normalfont{} Помилуй мя, Боже, помилуй мя.


Покорила еси неработное твое достоинство телу твоему, иного бо Ахитофела обретше врага, душе, снизшла еси сего советом; но сия разсыпа Сам Христос, да ты всяко спасешися.


\itshape Свободное свое достоинство ты, душа, подчинила своему телу, ибо, нашедши другого Ахитофела-врага, ты склонилась на его советы, но их рассеял Сам Христос, чтобы ты спасена была. 2 Цар. 16:20–21\normalfont{}


\itshape Припев:\normalfont{} Помилуй мя, Боже, помилуй мя.


Соломон чудный и благодати премудрости исполненный, сей лукавое иногда пред Богом сотворив, отступи от Него; емуже ты проклятым твоим житием, душе, уподобилася еси.


\itshape Чудный Соломон, будучи преисполнен дара премудрости, некогда, сотворив злое пред Богом, отступил от Него; ему ты уподобилась, душа, своей жизнью, достойной проклятия. 3 Цар. 3:12; 3 Цар. 11:4–6\normalfont{}


\itshape Припев:\normalfont{} Помилуй мя, Боже, помилуй мя.


Сластьми влеком страстей своих, оскверняшеся, увы мне, рачитель премудрости, рачитель блудных жен, и странен от Бога; егоже ты подражала еси умом, о душе, сладострастьми скверными.


\itshape Увлекшись сластолюбивыми страстями, осквернился, увы, ревнитель премудрости, возлюбив нечестивых женщин и отчуждившись от Бога; ему, душа, ты сама подражала в уме постыдным сладострастием. 3 Цар. 11:6–8\normalfont{}


\itshape Припев:\normalfont{} Помилуй мя, Боже, помилуй мя.


Ровоаму поревновала еси, не послушавшему совета отча, купно же и злейшему рабу Иеровоаму, прежнему отступнику, душе, но бегай подражания и зови Богу: согреших, ущедри мя.


\itshape Ты поревновала, душа, Ровоаму, не послушавшему совета отеческого, и вместе злейшему рабу Иеровоаму, древнему мятежнику; избегай подражание им и взывай к Богу: согрешила я, умилосердись надо мною. 3 Цар. 12:13–14; 3 Цар. 20\normalfont{}


\itshape Припев:\normalfont{} Помилуй мя, Боже, помилуй мя.


Ахаавовым поревновала еси сквернам, душе моя, увы мне, была еси плотских скверн пребывалище и сосуд срамлен страстей, но из глубины твоея воздохни и глаголи Богу грехи твоя.


\itshape Ты подражала Ахаву в мерзостях, душа моя; увы, ты сделалась жилищем плотских нечистот и постыдным сосудом страстей; но воздохни из глубины своей и поведай Богу грехи свои. 3 Цар. 16:29–31\normalfont{}


\itshape Припев:\normalfont{} Помилуй мя, Боже, помилуй мя.


Попали Илиа иногда дващи пятьдесят Иезавелиных, егда студныя пророки погуби, во обличение Ахаавово, но бегай подражания двою, душе, и укрепляйся.


\itshape Илия попалил некогда дважды по пятьдесят служителей Иезавели, когда истреблял гнусных пророков ее в обличение Ахава; но ты, душа, избегай подражания обоим им и крепись в воздержании. 3 Цар. 18:40; 4 Цар. 1:9–15\normalfont{}


\itshape Припев:\normalfont{} Помилуй мя, Боже, помилуй мя.


Заключися тебе небо, душе, и глад Божий постиже тя, егда Илии Фесвитянина, якоже Ахаав, не покорися словесем иногда, но Сараффии уподобився, напитай пророчу душу.


\itshape Заключилось небо для тебя, душа, и голод от Бога послан на тебя, как некогда на Ахава за то, что он не послушал слов Илии Фесфитянина; но ты подражай вдове Сарептской, напитай душу пророка. 3 Цар. 17:1; 3 Цар. 17:8–9\normalfont{}


\itshape Припев:\normalfont{} Помилуй мя, Боже, помилуй мя.


Манассиева собрала еси согрешения изволением, поставльши яко мерзости страсти и умноживши, душе, негодования, но того покаянию ревнующи тепле, стяжи умиление.


\itshape Ты, душа, добровольно вместила преступления Манассии, поставив вместо идолов страсти и умножив мерзости; но усердно подражай и его покаянию с чувством умиления 4 Цар. 21:1–2\normalfont{}


\itshape Припев:\normalfont{} Помилуй мя, Боже, помилуй мя.


Припадаю Ти и приношу Тебе, якоже слезы, глаголы моя: согреших, яко не согреши блудница, и беззаконновах, яко иный никтоже на земли. Но ущедри, Владыко, творение Твое и воззови мя.


\itshape Припадаю к Тебе и приношу Тебе со слезами слова мои: согрешил я, как не согрешила блудница, и жил в беззакониях, как никто другой на земле; но умилосердись, Владыка, над созданием Своим и восстанови меня.\normalfont{}


\itshape Припев:\normalfont{} Помилуй мя, Боже, помилуй мя.


Погребох образ Твой и растлих заповедь Твою, вся помрачися доброта, и страстьми угасися, Спасе, свеща. Но ущедрив, воздаждь ми, якоже поет Давид, радование.


\itshape Затмил я образ Твой и нарушил заповедь Твою; вся красота помрачилась во мне, и светильник погас от страстей; но умилосердись, Спаситель, и возврати мне, как поет Давид, веселие. Пс. 50:14\normalfont{}


\itshape Припев:\normalfont{} Помилуй мя, Боже, помилуй мя.


Обратися, покайся, открый сокровенная, глаголи Богу, вся ведущему: Ты веси моя тайная, Едине Спасе. Но Сам мя помилуй, якоже поет Давид, по милости Твоей.


\itshape Обратись, покайся, открой сокровенное, скажи Богу Всеведущему: Спаситель, Ты Один знаешь мои тайны, но Сам помилуй меня, как поет Давид, по Твоей милости. Пс. 50:3\normalfont{}


\itshape Припев:\normalfont{} Помилуй мя, Боже, помилуй мя.


Исчезоша дние мои, яко соние востающаго; темже, яко Езекиа, слезю на ложи моем, приложитися мне летом живота. Но кий Исаия предстанет тебе, душе, аще не всех Бог?


\itshape Дни мои прошли как сновидение пробуждающегося; поэтому, подобно Езекии, я плачу на ложе моем, чтобы продлились годы жизни моей; но какой Исаия посетит тебя, душа, если не Бог всех? 4 Цар. 20:1–6; Ис. 38:1–6\normalfont{}


\itshape Припев:\normalfont{} Преподобная мати Марие, моли Бога о нас.


Возопивши к Пречистей Богоматери, первее отринула еси неистовство страстей, нужно стужающих, и посрамила еси врага запеншаго. Но даждь ныне помощь от скорби и мне, рабу твоему.


\itshape Воззвавши к Пречистой Богоматери, ты обуздала неистовство страстей, прежде жестоко свирепствовавших, и посрамила врага-обольстителя; даруй же ныне помощь в скорби и мне, рабу твоему. Пс. 59:13\normalfont{}


\itshape Припев:\normalfont{} Преподобная мати Марие, моли Бога о нас.


Егоже возлюбила еси, Егоже возжелела еси, Егоже ради плоть изнурила еси, преподобная, моли ныне Христа о рабех: яко да милостив быв всем нам, мирное состояние дарует почитающим Его.


\itshape Кого ты возлюбила, Кого избрала, для Кого изнуряла плоть, Преподобная, моли ныне Христа о рабах твоих, чтобы Он по Своей милости ко всем даровал мирное состояние почитающим Его.\normalfont{}


\itshape Припев:\normalfont{} Преподобне отче Андрее, моли Бога о нас.


На камени мя веры молитвами твоими утверди, отче, страхом мя Божественным ограждая, и покаяние, Андрее, подаждь ми, молюся ти, и избави мя от сети врагов, ищущих мя.


Слава Отцу и Сыну и Святому Духу.


Троице Простая, Нераздельная, Единосущная и Естество Едино, Светове и Свет, и Свята Три, и Едино Свято поется Бог Троица; но воспой, прослави Живот и Животы, душе, всех Бога.


\itshape Троица Простая, Нераздельная, Единосущная, и Одно Божество, Светы и Свет, Три Святы и Одно Лицо Свято, Бог-Троица, воспеваемая в песнопениях; воспой же и ты, душа, прославь Жизнь и Жизни "--- Бога всех.\normalfont{}


И ныне и присно и во веки веков. Аминь.


Поем Тя, благословим Тя, покланяемся Ти, Богородительнице, яко Неразлучныя Троицы породила еси Единаго Христа Бога и Сама отверзла еси нам, сущим на земли, Небесная.


\itshape Воспеваем Тебя, благословляем Тебя, поклоняемся Тебе, Богородительница, ибо Ты родила Одного из Нераздельной Троицы, Христа Бога, и Сама открыла для нас, живущих на земле, небесные обители.\normalfont{}





\bfseries Песнь 8\normalfont{}





\bfseries Трипеснец, глас 8:\normalfont{}


\itshape Ирмо́с:\normalfont{}


Безначальнаго Царя славы, Егоже трепещут Небесныя силы, пойте, священницы, людие, превозносите во вся веки.


\itshape Припев:\normalfont{} Святии апостоли, молите Бога о нас.


Яко углие невещественнаго огня, попалите вещественныя страсти моя, возжизающе ныне во мне желание Божественныя любве, апостоли.


\itshape Припев:\normalfont{} Святии апостоли, молите Бога о нас.


Трубы благогласныя Слова почтим, имиже падоша стены неутверждены вражия и богоразумия утвердишася забрала.


\itshape Припев:\normalfont{} Святии апостоли, молите Бога о нас.


Кумиры страстныя души моея сокрушите, иже храмы и столпы сокрушисте врага, апостоли Господни, храмове освященнии.


\itshape Припев:\normalfont{} Пресвятая Богородице, спаси нас.


Вместила еси Невместимаго естеством, носила еси Носящаго вся, доила еси, Чистая, питающаго тварь Христа Жизнодавца.


\itshape Иный трипеснец. Ирмо́с:\normalfont{}


Безначальнаго Царя славы, Егоже трепещут Небесныя силы, пойте, священницы, людие, превозносите во вся веки.


\itshape Припев:\normalfont{} Святии апостоли, молите Бога о нас.


Духа началохитростием создавше всю Церковь, апостоли Христовы, в ней благословите Христа во веки.


\itshape Припев:\normalfont{} Святии апостоли, молите Бога о нас.


Вострубивше трубою учений, низвергоша апостоли всю лесть идольскую, Христа превозносяща во вся веки.


\itshape Припев:\normalfont{} Святии апостоли, молите Бога о нас.


Апостоли, доброе преселение, назирателие мира и Небеснии жителие, вас присно восхваляющия избавите от бед.


\itshape Припев:\normalfont{} Пресвятая Троице, Боже наш, слава Тебе.


Трисолнечное Всесветлое Богоначалие, Единославное и Единопрестольное Естеству, Отче Вседетелю, Сыне и Божественный Душе, пою Тя во веки.


\itshape Припев:\normalfont{} Пресвятая Богородице, спаси нас.


Яко честный и превышший престол, воспоим Божию Матерь непрестанно, людие, Едину по рождестве Матерь и Деву.


\itshape Великаго канона Ирмо́с:\normalfont{}


Егоже воинства небесная славят, и трепещут Херувими и Серафими, всяко дыхание и тварь, пойте, благословите и превозносите во вся веки.


\itshape Кого прославляют воинства небесные и пред Кем трепещут Херувимы и Серафимы, Того, все существа и творения, воспевайте, благословляйте и превозносите во все века.\normalfont{}


\itshape Припев:\normalfont{} Помилуй мя, Боже, помилуй мя.


Согрешивша, Спасе, помилуй, воздвигни мой ум ко обращению, приими мя кающагося, ущедри вопиюща: согреших Ти, спаси, беззаконновах, помилуй мя.


\itshape Помилуй меня, грешника, Спаситель, пробуди мой ум к обращению, приими кающегося, умилосердись над взывающим: я согрешил пред Тобою, спаси; я жил в беззакониях, помилуй меня.\normalfont{}


\itshape Припев:\normalfont{} Помилуй мя, Боже, помилуй мя.


Колесничник Илиа колесницею добродетелей вшед, яко на небеса, ношашеся превыше иногда от земных; сего убо, душе моя, восход помышляй.


\itshape Везомый на колеснице Илия, взойдя на колесницу добродетелей, некогда вознесся как бы на небеса, превыше всего земного; помышляй, душа моя, об его восходе. 4 Цар. 2:11–13\normalfont{}


\itshape Припев:\normalfont{} Помилуй мя, Боже, помилуй мя.


Иорданова струя первее милотию Илииною Елиссеем ста сюду и сюду; ты же, о душе моя, сея не причастилася еси благодати за невоздержание.


\itshape Елисея милотию Илии некогда разделил поток Иордана на ту и другую сторону; но ты, душа моя, не получила этой благодати за невоздержание. 4 Цар. 2:14\normalfont{}


\itshape Припев:\normalfont{} Помилуй мя, Боже, помилуй мя.


Елиссей иногда прием милоть Илиину, прият сугубую благодать от Бога; ты же, о душе моя, сея не причастилася еси благодати за невоздержание.


\itshape Некогда Елисей, приняв милоть (плащ) Илии, получил сугубую благодать от Господа; но ты, душа моя, не получила этой благодати за невоздержание. 4 Цар. 2:9; 4 Цар. 12–13\normalfont{}


\itshape Припев:\normalfont{} Помилуй мя, Боже, помилуй мя.


Соманитида иногда праведнаго учреди о душе, нравом благим; ты же не ввела еси в дом ни странна, ни путника. Темже чертога изринешися вон, рыдающи.


\itshape Соманитянка некогда угостила праведника с добрым усердием; а ты, душа, не приняла в свой дом ни странника, ни пришельца; за то будешь извержена вон из брачного чертога с рыданием. 4 Цар. 4:8\normalfont{}


\itshape Припев:\normalfont{} Помилуй мя, Боже, помилуй мя.


Гиезиев подражала еси окаянная разум скверный всегда, душе, егоже сребролюбие отложи поне на старость; бегай геенскаго огня, отступивши злых твоих.


\itshape Ты, несчастная душа, непрестанно подражала нечистому нраву Гиезия; хотя в старости отвергни его сребролюбие и, оставив свои злодеяния, избегни огня геенского. 4 Цар. 5:20–27\normalfont{}


Ты Озии, душе, поревновавши, сего прокажение в себе стяжала еси сугубо: безместная бо мыслиши, беззаконная же дееши; остави, яже имаши, и притецы к покаянию.


\itshape Соревновав Озии, душа, ты получила себе вдвойне его проказу, ибо помышляешь недолжное и делаешь беззаконное; оставь, что у тебя есть и приступи к покаянию. 4 Цар. 15:5; 2 Пар. 26:19\normalfont{}


\itshape Припев:\normalfont{} Помилуй мя, Боже, помилуй мя.


Ниневитяны, душе, слышала еси кающыяся Богу, вреищем и пепелом, сих не подражала еси, но явилася еси злейшая всех, прежде закона и по законе прегрешивших.


\itshape Ты слышала, душа, о ниневитянах, в рубище и пепле каявшихся Богу; им ты не подражала, но оказалась упорнейшею всех, согрешивших до закона и после закона. Иона 3:5; Иона 3:8\normalfont{}


\itshape Припев:\normalfont{} Помилуй мя, Боже, помилуй мя.


В рове блата слышала еси Иеремию, душе, града Сионя рыданьми вопиюща и слез ищуща: подражай сего плачевное житие и спасешися.


\itshape Ты слышала, душа, как Иеремия, в нечистом рве с рыданиями взывал к городу Сиону и искал слез; подражай плачевной его жизни и спасешься. Иер. 38:6\normalfont{}


\itshape Припев:\normalfont{} Помилуй мя, Боже, помилуй мя.


Иона в Фарсис побеже, проразумев обращение ниневитянов, разуме бо, яко пророк, Божие благоутробие: темже ревноваше пророчеству не солгатися.


\itshape Иона побежал в Фарсис, предвидя обращение ниневитян, ибо он, как пророк, знал милосердие Божие и вместе ревновал, чтобы пророчество не оказалось ложным. Иона 1:3\normalfont{}


\itshape Припев:\normalfont{} Помилуй мя, Боже, помилуй мя.


Даниила в рове слышала еси, како загради уста, о душе, зверей; уведела еси, како отроцы, иже о Азарии, погасиша верою пещи пламень горящий.


\itshape Ты слышала„ душа, как Даниил во рве заградил уста зверей; ты узнала, как юноши, бывшие с Азариею, верою угасили разожженный пламень печи. Дан. 14:31; Дан 3:24\normalfont{}


\itshape Припев:\normalfont{} Помилуй мя, Боже, помилуй мя.


Ветхаго Завета вся приведох ти, душе, к подобию; подражай праведных боголюбивая деяния, избегни же паки лукавых грехов.


\itshape Из Ветхого Завета всех я привел тебе в пример, душа; подражай богоугодным деяниям праведных, и избегай грехов людей порочных.\normalfont{}


\itshape Припев:\normalfont{} Помилуй мя, Боже, помилуй мя.


Правосуде Спасе, помилуй и избави мя огня и прещения, еже имам на суде праведно претерпети; ослаби ми прежде конца добродетелию и покаянием.


\itshape Правосудный Спаситель, помилуй и избавь меня от огня и наказания, которому я должен справедливо подвергнуться на суде; прости меня прежде кончины, дав мне добродетель и покаяние.\normalfont{}


\itshape Припев:\normalfont{} Помилуй мя, Боже, помилуй мя.


Яко разбойник вопию Ти: помяни мя; яко Петр, плачу горце: ослаби ми, Спасе; зову, яко мытарь, слезю, яко блудница; приими мое рыдание, якоже иногда хананеино.


\itshape Как разбойник взываю к Тебе: вспомни меня; как Петр, горько плачу, Спаситель; как мытарь, издаю вопль: будь милостив ко мне; проливаю слезы, как блудница; прими мое рыдание, как некогда от жены Хананейской. Лк. 7:37–38; Лк. 18:13; Лк. 23:42; Лк. 22:61–62; Мф. 15:22\normalfont{}


\itshape Припев:\normalfont{} Помилуй мя, Боже, помилуй мя.


Гноение, Спасе, исцели смиренныя моея души, Едине Врачу, пластырь мне наложи, и елей, и вино, дела покаяния, умиление со слезами.


\itshape Один Врач "--- Спаситель, исцели гниение моей смиренной души; приложи мне пластырь, елей и вино "--- дела покаяния, умиление со слезами.\normalfont{}


\itshape Припев:\normalfont{} Помилуй мя, Боже, помилуй мя.


Хананею и аз подражая, помилуй мя, вопию, Сыне Давидов; касаюся края ризы, яко кровоточивая, плачу, яко Марфа и Мария над Лазарем.


\itshape Подражая жене Хананейской, и я взываю к Сыну Давидову: помилуй меня; касаюсь одежды Его, как кровоточивая, плачу, как Марфа и Мария над Лазарем. Мф. 9:20; Мф. 15:22; Ин. 11:33\normalfont{}


\itshape Припев:\normalfont{} Помилуй мя, Боже, помилуй мя.


Слезную, Спасе, сткляницу яко миро истощавая на главу, зову Ти, якоже блудница, милости ищущая, мольбу приношу и оставление прошу прияти.


\itshape Изливая сосуд слез, как миро на голову, Спаситель, взываю к Тебе, как ищущая милости блудница, приношу моление и прошу о получении мне прощения. Мф. 26:6–7; Мк. 14:3; Лк. 7:37–38\normalfont{}


\itshape Припев:\normalfont{} Помилуй мя, Боже, помилуй мя.


Аще и никтоже, якоже аз, согреши Тебе, но обаче приими и мене, благоутробне Спасе, страхом кающася и любовию зовуща: согреших Тебе Единому, помилуй мя, Милостиве.


\itshape Хотя никто не согрешил пред Тобою, как я, но, Милосердный Спаситель, приими меня, кающегося со страхом и с любовию взывающего: я согрешил пред Тобою Одним, помилуй меня, Милосердный!\normalfont{}


\itshape Припев:\normalfont{} Помилуй мя, Боже, помилуй мя.


Пощади, Спасе, Твое создание и взыщи, яко Пастырь, погибшее, предвари заблуждшаго, восхити от волка, сотвори мя овча на пастве Твоих овец.


\itshape Пощади, Спаситель, создание Свое и, как Пастырь, отыщи потерянного, возврати заблудшего, отними у волка и сделай меня агнцем на пастбище Твоих овец. Пс. 118:176\normalfont{}


\itshape Припев:\normalfont{} Помилуй мя, Боже, помилуй мя.


Егда Судие сядеши, яко благоутробен, и покажеши страшную славу Твою, Спасе, о каковый страх тогда пещи горящей, всем боящимся нестерпимаго судища Твоего.


\itshape Когда Ты, Милосердный, воссядешь, как Судия и откроешь грозное величие Свое, Спаситель, о, какой ужас тогда: печь будет гореть, и все трепетать пред неумолимым судом Твоим. Мф. 25:31; Мф. 41:47\normalfont{}


\itshape Припев:\normalfont{} Преподобная мати Марие, моли Бога о нас.


Света незаходимаго Мати, тя просветивши, от омрачения страстей разреши. Темже вшедши в духовную благодать, просвети, Марие, тя верно восхваляющыя.


\itshape Матерь незаходимаго Света "--- Христа, просветив тебя, освободила от мрака страстей; поэтому, приняв благодать Духа, просвети, Мария, искренно прославляющих тебя.\normalfont{}


\itshape Припев:\normalfont{} Преподобная мати Марие, моли Бога о нас.


Чудо ново видев, ужасашеся божественный в тебе воистинну, мати, Зосима: ангела бо зряше во плоти и ужасом весь исполняшеся, Христа поя во веки.


\itshape Увидев в тебе, матерь, поистине новое чудо, святой Зосима удивился, ибо он увидел Ангела во плоти, и весь преисполнился изумлением, воспевая Христа вовеки .\normalfont{}


\itshape Припев:\normalfont{} Преподобне отче Андрее, моли Бога о нас.


Яко дерзновение имый ко Господу, Андрее Критский, честная похвало, молю, молися разрешение от уз беззакония ныне обрести мне молитвами твоими, яко покаяния учитель и преподобных слава.


Благословим Отца и Сына и Святаго Духа Господа.


Безначальне Отче, Сыне Собезначальне, Утешителю Благий, Душе Правый, Слова Божия Родителю, Отца Безначальна Слове, Душе Живый и Зиждяй, Троице Единице, помилуй мя.


\itshape Безначальный Отче, Собезначальный Сын, Утешитель Благий, Дух Правый, Родитель Слова Божия, Безначальное Слово Отца, Дух, Животворящий и Созидающий, Троица Единая, помилуй меня.\normalfont{}


И ныне и присно и во веки веков. Аминь.


Яко от оброщения червленицы, Пречистая, умная багряница Еммануилева внутрь во чреве Твоем плоть исткася. Темже Богородицу воистинну Тя почитаем.


\itshape Мысленная порфира "--- плоть Еммануила соткалась внутри Твоего чрева. Пречистая, как бы из вещества пурпурного; потому мы почитаем Тебя, Истинную Богородицу.\normalfont{}


Хвалим, благословим, покланяемся Господеви, поюще и превозносяще во вся веки.


\itshape Катавасия:\normalfont{}


Егоже воинства небесная славят, и трепещут Херувими и Серафими, всяко дыхание и тварь, пойте, благословите и превозносите во вся веки.


\itshape Кого прославляют воинства небесные и пред Кем трепещут Херувимы и Серафимы, Того, все существа и творения, воспевайте, благословляйте и превозносите во все века.\normalfont{}





\bfseries Песнь 9\normalfont{}





\bfseries Трипеснец, глас 8:\normalfont{}


\itshape Ирмо́с:\normalfont{}


Воистинну Богородицу Тя исповедуем, спасеннии Тобою, Дево чистая, с безплотными лики Тя величающе.


\itshape Припев:\normalfont{} Святии апостоли, молите Бога о нас.


Источницы спасительныя воды явльшеся апостоли, истаявшую душу мою греховною жаждою оросите.


\itshape Припев:\normalfont{} Святии апостоли, молите Бога о нас.


Плавающаго в пучине погибели и в погружении уже бывша Твоею десницею, якоже Петра, Господи, спаси мя.


\itshape Припев:\normalfont{} Святии апостоли, молите Бога о нас.


Яко соли, вкусных суще учений, гнильство ума моего изсушите и неведения тьму отжените.


\itshape Припев:\normalfont{} Пресвятая Богородице, спаси нас.


Радость яко родившая, плач мне подаждь, имже Божественное утешение, Владычице, в будущем дни обрести возмогу.


\itshape Иный трипеснец. Ирмо́с:\normalfont{}


Тя, Небесе и земли Ходатаицу вси роди ублажаем: плотски бо вселися в Тя исполнение, Дево, Божества.


\itshape Припев:\normalfont{} Святии апостоли, молите Бога о нас.


Тя, благославное апостольское собрание, песньми величаем: вселенней бо светила светлая явистеся, прелесть отгоняще.


\itshape Припев:\normalfont{} Святии апостоли, молите Бога о нас.


Благовестною мрежею вашею словесныя рыбы уловивше, сия приносите всегда снедь Христу, апостоли блаженнии.


\itshape Припев:\normalfont{} Святии апостоли, молите Бога о нас.


К Богу вашим прошением помяните нас, апостоли, от всякаго избавитися искушения, молимся, любовию воспевающия вас.


\itshape Припев:\normalfont{} Пресвятая Троице, Боже наш, слава Тебе.


Тя, Триипостасную Единицу, Отче, Сыне со Духом, Единаго Бога Единосущна пою, Троицу Единосильную Безначальную.


\itshape Припев:\normalfont{} Пресвятая Богородице, спаси нас.


Тя, Детородительницу и Деву, вси роди ублажаем, яко Тобою избавльшеся от клятвы: радость бо нам родила еси, Господа.


\itshape Великаго канона Ирмо́с:\normalfont{}


Безсеменнаго зачатия рождество несказанное, Матере безмужныя нетленен Плод, Божие бо Рождение обновляет естества. Темже Тя вси роди, яко Богоневестную Матерь, православно величаем.


\itshape Рождество от бессеменного зачатия неизъяснимо, безмужной Матери нетленен Плод, ибо рождение Бога обновляет природу. Поэтому Тебя, как Богоневесту-Матерь мы, все роды, православно величаем.\normalfont{}


\itshape Припев:\normalfont{} Помилуй мя, Боже, помилуй мя.


Ум острупися, тело оболезнися, недугует дух, слово изнеможе, житие умертвися, конец при дверех. Темже, моя окаянная душе, что сотвориши, егда приидет Судия испытати твоя?


\itshape Ум изранился, тело расслабилось, дух болезнует, слово потеряло силу, жизнь замерла, конец при дверях. Что же сделаешь ты, несчастная душа, когда придет Судия исследовать дела твои?\normalfont{}


\itshape Припев:\normalfont{} Помилуй мя, Боже, помилуй мя.


Моисеово приведох ти, душе, миробытие, и от того все Заветное Писание, поведающее тебе праведныя и неправедныя; от нихже вторыя, о душе, подражала еси, а не первыя, в Бога согрешивши.


\itshape Я воспроизвел пред тобою, душа, сказание Моисея о бытии мира и затем все Заветное Писание, повествующее о праведных и неправедных; из них ты, душа, подражала последним, а не первым, согрешая пред Богом.\normalfont{}


\itshape Припев:\normalfont{} Помилуй мя, Боже, помилуй мя.


Закон изнеможе, празднует Евангелие, Писание же все в тебе небрежено бысть, пророцы изнемогоша и все праведное слово; струпи твои, о душе, умножишася, не сущу врачу, исцеляющему тя.


\itshape Ослабел закон, не воздействует Евангелие, пренебрежено все Писание тобою, пророки и всякое слово праведника потеряли силу; язвы твои, душа, умножились, без Врача, исцеляющего тебя.\normalfont{}


\itshape Припев:\normalfont{} Помилуй мя, Боже, помилуй мя.


Новаго привожду ти Писания указания, вводящая тя, душе, ко умилению; праведным убо поревнуй, грешных же отвращайся и умилостиви Христа молитвами же и пощеньми, и чистотою, и говением.


\itshape Из Новозаветного Писания привожу тебе примеры, душа, возбуждающие в тебе умиление; так подражай праведным и отвращайся примера грешных и умилостивляй Христа молитвою, постом, чистотою и непорочностью.\normalfont{}


\itshape Припев:\normalfont{} Помилуй мя, Боже, помилуй мя.


Христос вочеловечися, призвав к покаянию разбойники и блудницы; душе, покайся, дверь отверзеся Царствия уже, и предвосхищают е фарисее и мытари и прелюбодеи кающиися


\itshape Христос, сделавшись человеком, призвал к покаянию разбойников и блудниц; покайся, душа, дверь Царства уже открылась, и прежде тебя входят в нее кающиеся фарисеи, мытари и прелюбодеи. Мф. 11:12; Мф. 21:31; Лк. 16:16\normalfont{}


\itshape Припев:\normalfont{} Помилуй мя, Боже, помилуй мя.


Христос вочеловечися, плоти приобщився ми и вся елика суть естества хотением исполни, греха кроме, подобие тебе, о душе, и образ предпоказуя Своего снисхождения.


\itshape Христос, сделался человеком, приобщившись ко мне плотию, и добровольно испытал все, что свойственно природе, за исключением греха, показывая тебе, душа, пример и образец Своего снисхождения.\normalfont{}


\itshape Припев:\normalfont{} Помилуй мя, Боже, помилуй мя.


Христос волхвы спасе, пастыри созва, младенец множества показа мученики, старцы прослави и старыя вдовицы, ихже не поревновала еси, душе, ни деянием, ни житию, но горе тебе внегда будеши судитися.


\itshape Христос спас волхвов, призвал к Себе пастухов, множество младенцев сделал мучениками, прославил старца и престарелую вдовицу; их деяниям и жизни ты не подражала, душа, но горе тебе, когда будешь судима! Мф. 2:1–2; Мф. 2:16; Лк. 2:4–8; Лк. 2:25–26; Лк. 2:36–38\normalfont{}


\itshape Припев:\normalfont{} Помилуй мя, Боже, помилуй мя.


Постився Господь дний четыредесять в пустыни, последи взалка, показуя человеческое; душе, да не разленишися, аще тебе приложится враг, молитвою же и постом от ног твоих да отразится.


\itshape Господь, постившись сорок дней в пустыне, наконец взалкал, обнаруживая в Себе человеческую природу. Не унывай, душа, если враг устремится на тебя, но да отразится он от ног твоих молитвами и постом. Исх. 34:28; Мф. 4:2; Лк. 4:2; Мк. 1:13\normalfont{}


\itshape Припев:\normalfont{} Помилуй мя, Боже, помилуй мя.


Христос искушашеся, диавол искушаше, показуя камение, да хлеби будут, на гору возведе видети вся царствия мира во мгновении; убойся, о душе, ловления, трезвися, молися на всякий час Богу.


\itshape Христос был искушаем; диавол искушал, показывая камни, чтобы они обратились в хлебы; возвел Его на гору, чтобы видеть все царства мира в одно мгновение; бойся, душа, этого обольщения, бодрствуй и ежечасно молись Богу. Мф. 4:1–10; Мк. 1; 12–13; Лк. 4:1–12\normalfont{}


\itshape Припев:\normalfont{} Помилуй мя, Боже, помилуй мя.


Горлица пустыннолюбная, глас вопиющаго возгласи, Христов светильник, проповедуяй покаяние, Ирод беззаконнова со Иродиадою. Зри, душе моя, да не увязнеши в беззаконныя сети, но облобызай покаяние.


\itshape Пустыннолюбивая горлица, голос вопиющего, Христов светильник взывал, проповедуя покаяние, а Ирод беззаконствовал с Иродиадою; смотри, душа моя, чтобы не впасть тебе в сети беззаконных, но возлюби покаяние. Песн. 2:12; Ис. 40:3; Мф. 3:1–8; Мф. 14:3–4; Мк. 6:17; Лк. 3:19 -20\normalfont{}


\itshape Припев:\normalfont{} Помилуй мя, Боже, помилуй мя.


В пустыню вселися благодати Предтеча, и Иудея вся и Самария, слышавше, течаху и исповедаху грехи своя, крещающеся усердно: ихже ты не подражала еси, душе.


\itshape Предтеча благодати обитал в пустыне и все иудеи и самаряне стекались слушать его и исповедовали грехи свои, с усердием принимая крещение. Но ты, душа, не подражала им. Мф. 3:1–6; Мк. 1:3–6\normalfont{}


\itshape Припев:\normalfont{} Помилуй мя, Боже, помилуй мя.


Брак убо честный и ложе нескверно, обоя бо Христос прежде благослови, плотию ядый, и в Кане же на браце воду в вино совершая, и показуя первое чудо, да ты изменишися, о душе.


\itshape Брак честен и ложе непорочно, ибо Христос благословил их некогда, в Кане на браке вкушая пищу плотию и претворяя воду в вино, совершая первое чудо, чтобы ты, душа, изменилась. Евр. 13:4; Ин. 2:1–11\normalfont{}


\itshape Припев:\normalfont{} Помилуй мя, Боже, помилуй мя.


Разслабленнаго стягну Христос, одр вземша, и юношу умерша воздвиже, вдовиче рождение, и сотнича отрока, и самаряныне явися, в дусе службу тебе, душе, предживописа.


\itshape Христос укрепил расслабленного, взявшего постель свою; воскресил умершего юного сына вдовы, исцелил слугу сотника и, открыв Себя самарянке, предначертал тебе, душа, служение Богу духом. Мф. 9:6; Мф. 8:13; Лк. 7:12–15; Ин. 4:7–26\normalfont{}


\itshape Припев:\normalfont{} Помилуй мя, Боже, помилуй мя.


Кровоточивую исцели прикосновением края ризна Господь, прокаженныя очисти, слепыя и хромыя просветив исправи, глухия же, и немыя, и ничащия низу исцели словом: да ты спасешися, окаянная душе.


\itshape Господь исцелил кровоточивую через прикосновение к одежде Его, очистил прокаженных, дал прозрение слепым, исправил хромых, глухих, немых и уврачевал словом скорченную, чтобы ты спаслась, несчастная душа. Мф. 9:20–22; Мф. 11:4–5; Лк. 13:10–13\normalfont{}


\itshape Припев:\normalfont{} Помилуй мя, Боже, помилуй мя.


Недуги исцеляя, нищим благовествоваше Христос Слово, вредныя уврачева, с мытари ядяше, со грешники беседоваше, Иаировы дщере душу предумершую возврати осязанием руки.


\itshape Врачуя болезни, Христос-Слово, благовествовал нищим, исцелял увечных, вкушал с мытарями, беседовал с грешниками и прикосновением руки возвратил вышедшую из тела душу Иаировой дочери. Мф. 4:23; Мф. 9:10–11; Мк. 5:41–42\normalfont{}


\itshape Припев:\normalfont{} Помилуй мя, Боже, помилуй мя.


Мытарь спасашеся, и блудница целомудрствоваше, и фарисей, хваляся, осуждашеся. Ов убо, очисти мя; ова же, помилуй мя; сей же величашеся вопия: Боже, благодарю Тя, и прочыя безумныя глаголы.


\itshape Мытарь спасся и блудница сделалась целомудренною, а гордый фарисей подвергся осуждению, ибо первый взывал: «Будь милостив ко мне»; другая: «Помилуй меня»; а последний тщеславно возглашал: «Боже, благодарю Тебя...» и прочие безумные речи. Лк. 7:37–38; Лк. 7:46–47; Лк. 18:11–14\normalfont{}


\itshape Припев:\normalfont{} Помилуй мя, Боже, помилуй мя.


Закхей мытарь бе, но обаче спасашеся, и фарисей Симон соблажняшеся, и блудница приимаше оставительная разрешения от Имущаго крепость оставляти грехи, юже, душе, потщися подражати.


\itshape Закхей был мытарь, однако спасся; Симон фарисей соблазнялся, а блудница получила прощение от Имеющего власть отпускать грехи; спеши, душа, и ты подражать ей. Лк. 7:39; Лк. 19:9; Ин. 8:3–11\normalfont{}


\itshape Припев:\normalfont{} Помилуй мя, Боже, помилуй мя.


Блуднице, о окаянная душе моя, не поревновала еси, яже приимши мира алавастр, со слезами мазаше нозе Спасове, отре же власы, древних согрешений рукописание Раздирающаго ея.


\itshape Бедная душа моя, ты не подражала блуднице, которая, взяв сосуд с миром, мазала со слезами и отирала волосами ноги Спасителя, разорвавшего запись прежних ее прегрешений. Лк. 7:37–38\normalfont{}


\itshape Припев:\normalfont{} Помилуй мя, Боже, помилуй мя.


Грады, имже даде Христос благовестие, душе моя, уведала еси, како прокляти быша. Убойся указания, да не будеши якоже оны, ихже содомляном Владыка уподобив, даже до ада осуди.


\itshape Ты знаешь, душа моя, как прокляты города, которым Христос благовестил Евангелие; страшись этого примера, чтобы и тебе не быть, как они, ибо Владыка, уподобив их содомлянам, присудил их к аду. Лк. 10:12–15\normalfont{}


\itshape Припев:\normalfont{} Помилуй мя, Боже, помилуй мя.


Да не горшая, о душе моя, явишися отчаянием, хананеи веру слышавшая, еяже дщи словом Божиим исцелися; Сыне Давидов, спаси и мене, воззови из глубины сердца, якоже она Христу.


\itshape Не окажись, душа моя, по отчаянию хуже хананеянки, слышавшей о вере, по которой Божиим словом исцелена дочь ее; взывай, как она, Христу из глубины сердца: «Сын Давидов, спаси и меня». Мф. 15:22\normalfont{}


\itshape Припев:\normalfont{} Помилуй мя, Боже, помилуй мя.


Умилосердися, спаси мя, Сыне Давидов, помилуй, беснующыяся словом исцеливый, глас же благоутробный, яко разбойнику, мне рцы: аминь, глаголю тебе, со Мною будеши в раи, егда прииду во славе Моей.


\itshape Умилосердись, спаси и помилуй меня, Сын Давидов, словом исцелявший беснующихся, и скажи, как разбойнику, милостивые слова: истинно говорю тебе, со Мною будешь в раю, когда приду Я в славе Моей. Лк. 23:42–43\normalfont{}


\itshape Припев:\normalfont{} Помилуй мя, Боже, помилуй мя.


Разбойник оглаголоваше Тя, разбойник богословяше Тя, оба бо на кресте свисяста. Но, о Благоутробне, яко верному разбойнику Твоему, познавшему Тя Бога, и мне отверзи дверь славнаго Царствия Твоего.


\itshape Разбойник поносил Тебя, разбойник же и Богом исповедал Тебя, вися оба на кресте; но, Милосердный, как уверовавшему разбойнику, познавшему в Тебе Бога, открой и мне, дверь славного Твоего Царства.\normalfont{}


\itshape Припев:\normalfont{} Помилуй мя, Боже, помилуй мя.


Тварь содрогашеся, распинаема Тя видящи, горы и камения страхом распадахуся, и земля сотрясашеся, и ад обнажашеся, и соомрачашеся свет во дни, зря Тебе, Иисусе, пригвождена ко Кресту.


\itshape Тварь содрогалась, видя Тебя распинаемым, горы и камни от ужаса распадались и колебалась земля, преисподняя пустела, и свет среди дня помрачался, взирая на Тебя, Иисус, плотию ко кресту пригвожденного. Мф. 27:51–52; Мк. 15:38; Лк. 23:45\normalfont{}


\itshape Припев:\normalfont{} Помилуй мя, Боже, помилуй мя.


Достойных покаяния плодов не истяжи от мене, ибо крепость моя во мне оскуде; сердце мне даруй присно сокрушенное, нищету же духовную: да сия Тебе принесу, яко приятную жертву, Едине Спасе.


\itshape Достойных плодов покаяния не требуй от меня, Единый Спаситель, ибо сила моя истощилась во мне; даруй мне всегда сокрушенное сердце и духовную нищету, чтобы я принес их Тебе, как благоприятную жертву.\normalfont{}


\itshape Припев:\normalfont{} Помилуй мя, Боже, помилуй мя.


Судие мой и Ведче мой, хотяй паки приити со ангелы, судити миру всему, милостивным Твоим оком тогда видев мя, пощади и ущедри мя, Иисусе, паче всякаго естества человеча согрешивша.


\itshape Судия мой, знающий меня, когда опять придешь Ты с Ангелами, чтобы судить весь мир, тогда, обратив на меня милостивый взор, пощади, Иисусе, и помилуй меня, согрешившего более всего человеческого рода.\normalfont{}


\itshape Припев:\normalfont{} Преподобная мати Марие, моли Бога о нас.


Удивила еси всех странным житием твоим, ангелов чины и человеков соборы, невещественно поживши и естество прешедши: имже, яко невещественныма ногама вшедши, Марие, Иордан прешла еси.


\itshape Ты удивила необычайною своею жизнью всех, как чины ангельские, так и человеческие сонмы, духовно пожив и превзошедши природу; поэтому, Мария, ты, как бесплотная, шествуя стопами, перешла Иордан.\normalfont{}


\itshape Припев:\normalfont{} Преподобная мати Марие, моли Бога о нас.


Умилостиви Создателя о хвалящих тя, преподобная мати, избавитися озлоблений и скорбей, окрест нападающих, да избавившеся от напастей, возвеличим непрестанно прославльшаго тя Господа.


\itshape Склони Творца на милость к восхваляющим тебя, преподобная матерь, чтобы нам избавиться от огорчений и скорбей, отовсюду нападающих на нас, чтобы, избавившись от искушений, мы непрестанно величали прославившего тебя Господа.\normalfont{}


\itshape Припев:\normalfont{} Преподобне отче Андрее, моли Бога о нас.


Андрее честный и отче треблаженнейший, пастырю Критский, не престай моляся о воспевающих тя: да избавимся вси гнева и скорби, и тления, и прегрешений безмерных, чтущии твою память верно.


\itshape Андрей досточтимый, отец преблаженный, пастырь Критский, не переставай молиться за воспевающих тебя, чтобы избавиться от гнева, скорби, погибели и бесчисленных прегрешений нам всем, искренно почитающим память твою.\normalfont{}


Слава Отцу и Сыну и Святому Духу.


Троице Единосущная, Единице Триипостасная, Тя воспеваем, Отца славяще, Сына величающе и Духу покланяющеся, Единому Естеству воистинну Богу, Жизни же и живущему Царству безконечному.


И ныне и присно и во веки веков. Аминь.


Град Твой сохраняй, Богородительнице Пречистая, в Тебе бо сей верно царствуяй, в Тебе и утверждается, и Тобою побеждаяй, побеждает всякое искушение, и пленяет ратники, и проходит послушание.


\itshape Сохраняй град Свой, Пречистая Богородительница. Под Твоею защитою он царствует с верою, и от Тебя получает крепость, и при Твоем содействии неотразимо побеждает всякое бедствие, берет в плен врагов и держит их в подчинении.\normalfont{}


\itshape Таже оба лика вкупе поют Ирмо́с:\normalfont{}


Безсеменнаго зачатия рождество несказанное, Матере безмужныя нетленен Плод, Божие бо Рождение обновляет естества. Темже Тя вси роди, яко Богоневестную Матерь, православно величаем.


\itshape Рождество от бессеменного зачатия неизъяснимо, безмужной Матери нетленен Плод, ибо рождение Бога обновляет природу. Поэтому Тебя, как Богоневесту-Матерь мы, все роды, православно величаем.\normalfont{}




\mychapterending