

\mypart{МОЛИТВЫ ЗА БОЛЯЩИХ}\label{_content_obolyachih}
%http://www.molitvoslov.org/content/obolyachih

 

\mychapter{Канон за болящего, глас 3-й}\begin{mymulticols}
%http://www.molitvoslov.org/text577.htm 
 


\mysubtitle{Песнь 1}


\irmos{Пресекаемое море жезлом древле, Израиль пройде яко по пустыни, и крестообразно яве предуготовляет стези. Сего ради поим во хвалении чудному Богу нашему, яко прославися.}

\pripev{Милостиве Господи, услыши молитву раб Твоих, молящихся Тебе.}

В день печали, нашедшия на ны, к Тебе, Христе Спасе, припадающе, Твоея милости просим. Облегчи болезнь раба Твоего, изреки нам яко и сотнику: иди, се здрав есть отрок твой.

\pripev{Милостиве Господи, услыши молитву раб Твоих, молящихся Тебе.}

Мольбы и моления, с воздыханием к Тебе вопием, Сыне Божий, помилуй нас. Воздвигни со одра лежащаго, яко разслабленнаго словом: возьми одр твой, глаголя, отпущаются ти греси твои.

\slava

Твоего, Христе, образа подобию поклоняющеся, верою целуем, и болящему здравия просим, подражающе кровоточивей, яже подольца риз Твоих коснуся, и исцеление недуга прият.

\inyne

Пречистая Госпоже Богородице, всем известная Помощница, не презри нас к Тебе припадающих, моли яко блага Твоего Сына и Бога нашего, дати здравие болящему, да Тя с нами прославляет.


\mysubtitle{Песнь 3}


\irmos{Иже от не сущих вся приведый, словом созидаемая, совершаемая духом, Вседержителю Вышний, в любви Твоей утверди мене.}

\pripev{Милостиве Господи, услыши молитву раб Твоих, молящихся Тебе.}

Иже от тяжких болезней на земли повержен, к Тебе, Христе, с нами вопиет, подаждь здравие телеси его, яко же Иезекии плакавшуся к Тебе.

\pripev{Милостиве Господи, услыши молитву раб Твоих, молящихся Тебе.}

Призри, Господи, на наше смирение, и не помяни беззаконий наших, но веры ради болящаго, яко прокаженнаго словом исцели ему болезнь, да Твое, Христе Боже, славится имя.

\slava

Церковь, юже еси освятил, на ту, Христе, не даждь поношения, но воздвигни невидимо в недузе на одре лежащаго, в нейже молим Ти ся: да не рекут невернии, где есть Бог их.

\inyne

К Твоему пречистому, Богородице, образу руце воздеюще вопием, услыши раб Твоих молитву и спаси в болезни лежащаго, да от болезни востав, воздаст обеты, яже в печали глаголаша уста его.


\mysubtitle{Седален, глас 8-й:}


На одре греховнем лежаща мя, и страстьми уязвена, и якоже воздвигл еси Петрову тещу и спасе разслабленнаго носима со одром, тако и ныне посети, Милостиве, болящаго, понесый недуги рода нашего. Тебе Единаго вемы, терпелива и милосерда, милостива Врача душам и телесем, Христа Бога нашего, наводяща недуги и паки исцеляюща, подавающа прощение кающимся о согрешениих, Единаго Милосердаго и Милостиваго.

\slava

Аз грешный плачуся на одре своем лежа, прощение ми подаждь, Христе Боже, и от болезни сея воздвигни мя, и яже есмь от юности грехи сотворил, избыти ми сих молитвами Богородицы даруй.

\inyne

Умилосердися и спаси мя, воздвигни мя от одра болезненнаго, мощь бо моя во мне изнеможе и весь нечаянием одержим есмь, Мати Божия Пречистая, исцели недугующа люте, Ты бо еси Помощница Христианом.


\mysubtitle{Песнь 4}


\irmos{Положил еси к нам твердую любовь, Господи: Единороднаго бо Твоего Сына за ны на смерть дал еси, темже Ти зовем благодаряще: слава силе Твоей, Господи.}

\pripev{Милостиве Господи, услыши молитву раб Твоих, молящихся Тебе.}

Уже отчаянна лютым недугом и к смерти приближившася, возврати, Христе, на живот и даждь плачущимся утеху, да вси прославляем Твоя святая чудеса.

\pripev{Милостиве Господи, услыши молитву раб Твоих, молящихся Тебе.}

К Тебе, Творче, каемся в своих гресех, яко не хощеши смерти грешничи, оживи, уздрави болящаго, да востав послужит Ти, исповедуя с нами Твою благость.

\slava

Слезы Манассиины, Ниневитян покаяние, Давидово исповедание приим, вскоре спасл еси, и наши молитвы ныне приими, даждь здравие болящему, о нем же Тя молим.

\inyne

Подаждь нам Твою милость, Госпоже, всегда на Тя надеющимся, испроси здравие болящему, врачебнии Твои руце с Предтечею, Богородице, ко Господу Богу простирающи.


\mysubtitle{Песнь 5}


\irmos{На земли Невидимый явился еси, и человеком волею сожил еси Непостижимый, и к Тебе утренююще, воспеваем Тя, Человеколюбче.}

\pripev{Милостиве Господи, услыши молитву раб Твоих, молящихся Тебе.}

Дщерь Иаирову уже умершу, яко Бог оживил еси, и ныне возведи, Христе Боже, от врат смертных болящаго, Ты бо еси Путь и Живот всем.

\pripev{Милостиве Господи, услыши молитву раб Твоих, молящихся Тебе.}

Сына вдовича оживив Спасе, и тоя слезы преложив на радость, спаси от недуга тлеющаго раба Твоего, да и наша скорбь и болезнь на радость приидет.

\slava

Огненну болезнь тещи Петрове прикосновением Твоим исцелив, и ныне возстави болящаго раба Твоего, да востав яко Иона, послужит Ти.

\inyne

Скорбнии, смиреннии, грешнии, не имущии дерзновения к Тебе, Пречистая Богомати, вопием, умоли Сына Твоего Христа дати болящему здравие телеси.


\mysubtitle{Песнь 6}


\irmos{Бездна последняя грехов обыде мя, и исчезает дух мой: но прострый, Владыко, высокую Твою мышцу, яко Петра мя, Управителю, спаси.}

\pripev{Милостиве Господи, услыши молитву раб Твоих, молящихся Тебе.}

Бездну милосердия и милости имея, Христе Боже, услыши моления раб Твоих. Ты бо еси Петром Тавифу воскресил, Ты и ныне в болезни лежащаго воздвигни, послушав церковных молитвенник.

\pripev{Милостиве Господи, услыши молитву раб Твоих, молящихся Тебе.}

Врачу душам и телесем нашим, понесый недуги всего мира, Христе, и Енея Петром исцеливый, Ты и ныне болящаго исцели святых апостол молитвами.

\slava

Преложи, Христе, на радость рыдание о недужнем скорбящем, да Твою получивше милость, внидут в дом Твой со обетными дарми, славяще Тя в Троице Единаго Бога.

\inyne

Приидите, о друзи, поклонимся молящеся о болящем Божией Матери. Та бо имеет власть недужныя исцеляти, со безмездникома вкупе, духовным невидимо помазующе маслом.


\mysubtitle{Кондак, глас 3-й:}


Душу мою, Господи, во гресех всяческих, безместными деяньми разслаблену воздвигни божественным Твоим человеколюбием, яко же и разслабленнаго воздвигл еси древле, да зову Ти спасаем: щедре Христе, даждь ми изцеление.


\mysubtitle{Икос:}


Иже руки Своей горстию содержай концы, Иисусе Боже, Иже Отцу собезначальный и Духу Святому совладычествуя, яко плотию явился еси недуги исцеляя и страсти очистил еси, слепыя просветил еси, и разслабленнаго словом Божественным совоставил еси, сего право ходяща сотворив и одр повелел еси на раму взяти. Темже вси с ним воспеваем и поем: щедре Христе, даждь ми иcцеление.


\mysubtitle{Песнь 7}


\irmos{Прежде образу златому, персидскому чтилищу отроцы не поклонишася, трие поюще посреде пещи: отцев Боже, благословен еси.}

\pripev{Милостиве Господи, услыши молитву раб Твоих, молящихся Тебе.}

О, Кресте Пресвятый Христов, честное Древо Животное. Тобою смерть погибе и мертвии воскресоша, и ныне болящаго изцели и оживи, якоже при Елене умершую девицу.

\pripev{Милостиве Господи, услыши молитву раб Твоих, молящихся Тебе.}

Долгую и лютую болезнь Иовлю на гноищи и в червех, того молящася, словом изцелил еси, Господи. И ныне же в церкви о болящем молим Тя: яко благ, изцели невидимо молитвами святых Твоих.

\slava

Вси вемы, яко умрети нам есть, Тебе Богу тако изволившу, но на мало время, Милостиве, здравия просим болящему, премени от смерти на живот, даждь скорбящим утеху.

\inyne

Пособствуй и помогай нашему сиротству, Богородице, Ты бо веси время и час, когда умолити Сына Твоего и Бога нашего, дати болящему здравие и прощение от всех грех.


\mysubtitle{Песнь 8}


\irmos{Служити Живому Богу, в Вавилоне отроцы претерпевше, о мусикийских органех нерадиша, и посреде пламене стояще, боголепную песнь воспеваху, глаголюще: благословите вся дела Господня Господа.}

\pripev{Милостиве Господи, услыши молитву раб Твоих, молящихся Тебе.}

Милость покажи, Владыко, в болезни раба Твоего, и скоро изцели, милостиве Христе Боже, и аще смерти не предаждь судней, да Ти покаяние воздаст. Сам бо рекл еси: не хощу смерти грешничи.

\pripev{Милостиве Господи, услыши молитву раб Твоих, молящихся Тебе.}

Господи милостиве, Твоя преславная чудеса да и до нас ныне достигнут: бесы прожени, недуги погуби, раны изцели, болезни уврачуй, и потворы и чародеяния и всякия язи избави ны.

\slava

Запретивый, Христе, морским ветром, и страх ученик на радость преложивый, и ныне запрети тяжким болезнем, труждающим раба Твоего, да вси возвеселимся хваляща Тя во веки.

\inyne

Избави, Богородице, от обышедших ны печалей, различных недуг, отравы же и чародейства, и бесовскаго мечтания, и от навета злых человек и от напрасныя смерти, молим Ти ся.


\mysubtitle{Песнь 9}


\irmos{На Синайстей горе виде Тя в купине Моисей, неопально огнь Божества заченшую во чреве: Даниил же Тя виде гору несекомую, жезл прозябший, Исайа взываше, от корене Давидова.}

\pripev{Милостиве Господи, услыши молитву раб Твоих, молящихся Тебе.}

Источниче жизни, подателю, Христе, милости, не отврати лица Твоего от нас. Облегчи болезнь утруженому болезнию, и воздвигни яко Фаддеом Авгаря, да присно Тебе славит со Отцем и Святым Духом.

\pripev{Милостиве Господи, услыши молитву раб Твоих, молящихся Тебе.}

Евангельскому верующе гласу, Твоего ищем обета, Христе: просите бо, рече, и дастся вам. Тем и ныне предстояще молим Тя, возстави со одра здрава лютым повержена недугом, да Тя с нами вкупе величает.

\slava

Мучимый недугом, внутрь невидимыми ранами, Тебе, Христе, с нами вопиет, не нам, Господи, не нас ради, вси бо мы исполнены грехов, но Матерними Твоими и Предтечевыми молитвами даждь исцеление болящему, да Тя вси возвеличим.

\inyne

Божия Мати Пречистая, со всеми святыми призываем Тя, со ангелы и архангелы, с пророки и патриархи, со апостолы, с преподобными и праведными молися Христу Богу нашему дати здравие болящему, да Тя вси величаем.


\mysubtitle{Молитва:}


Боже сильный, милостию строяй вся на спасение роду человеческому, посети раба Твоего сего \myemph{ (имя), }нарицающа имя Христа Твоего, исцели его от всякого недуга плотскаго: и отпусти грех и греховныя соблазны, и всяку напасть, и всяко нашествие неприязнено далече сотвори от раба Твоего. И воздвигни от одра греховнаго, и устрой его во святую Твою Церковь здрава душею и телом, и делы добрыми славящаго со всеми людьми имя Христа Твоего, яко Тебе славу возсылаем, со Безначальным Ти Сыном и со Святым Духом, ныне и присно и во веки веков. Аминь.

\end{mymulticols}

\mychapterending


\mychapter{Молитва до и после чтения Евангелия}\begin{mymulticols}
%http://www.molitvoslov.org/text553.htm 
 


\myemph{ Каждый день нужно читать по одной главе Евангелия, а перед и после главы "--- эту молитву:}




Спаси, Господи, и помилуй раба Твоего \myemph{( имя)} словами Божественнаго Евангелия, чтомыми о спасении раба Твоего. 

Попали, Господи, терние всех его согрешений, и да вселится в него благодать Твоя, опаляющая, очищающая, освящающая всего человека во имя Отца и Сына и Святаго Духа. Аминь.




\end{mymulticols}

\mychapterending


\mychapter{Молитва болящей}\begin{mymulticols}
%http://www.molitvoslov.org/text552.htm 
 


Господи, видишь Ты мою болезнь. Ты знаешь, как я грешна и немощна; помоги мне терпеть и благодарить Твою Благость. Господи, сотвори, чтобы болезнь эта была в очищение многих моих грехов. Владыко Господи, я в руках Твоих, помилуй меня по воле Твоей и, если мне полезно, исцели меня вскоре. Достойное по делам моим приемлю; помяни мя, Господи, во Царствии Твоем! Слава Богу за все!




\end{mymulticols}

\mychapterending


\mychapter{Молитва благодарственная, святого Иоанна Кронштадского, читаемая после исцеления от болезни}\begin{mymulticols}
%http://www.molitvoslov.org/text556.htm 
 


Слава Тебе, Господи, Иисусе Христе, Сыне Единородный Безначальнаго Отца, едине изцеляяй всяк недуг и всяку язю в людех, яко помиловал мя еси грешнаго и избавил еси мя от болезни моей, не попустив ей развиться и умертвить меня по грехам моим. Даруй мне отныне, Владыко, силу твердо творить волю Твою во спасение души моея окаянныя и во славу Твою со Безначальным Твоим Отцем и Единосущным Твоим Духом, ныне и присно и во веки веков. Аминь.




\end{mymulticols}

\mychapterending


\mychapter{Молитва в болезни}\begin{mymulticols}
%http://www.molitvoslov.org/text551.htm 
 


Господи Боже, Владыко жизни моей, Ты по благости Твоей сказал: не хочу смерти грешника, но чтоб он обратился и жив был. Я знаю, что эта болезнь, которою я страдаю, есть наказание Твое за мои грехи и беззакония; знаю, что по делам моим я заслужил тягчайшее наказание, но, Человеколюбче, поступай со мною не по злобе моей, а по беспредельному милосердию Твоему. Не пожелай смерти моей, но дай мне силы, чтобы я терпеливо сносил болезнь, как заслуженное мною испытание, и по исцелении от нее обратился всем сердцем, всею душою и всеми моими чувствами к Тебе, Господу Богу, Создателю моему, и жив был для исполнения святых Твоих заповедей, для спокойствия моих родных и для моего благополучия. Аминь. 




\end{mymulticols}

\mychapterending


\mychapter{Молитва во время эпидемии}\begin{mymulticols}
%http://www.molitvoslov.org/text555.htm 
 




Господи Боже наш! Услышь с высоты святаго Твоего Престола нас, грешных и недостойных рабов Твоих, благость Твою грехами своими прогневавших и милосердие Твое удаливших, и не взыскивай с рабов Твоих, но отврати страшный гнев Твой, справедливо нас постигший, прекрати пагубное наказание, удали страшный меч Твой, невидимо и безвременно нас поражающий, и пощади несчастных и слабых рабов Твоих, и не обрекай на смерть души наши, в покаянии прибегающие с истомленным сердцем и со слезами к Тебе, Богу Милосердому, мольбам нашим внимающему и перемену подающему. Ибо Тебе (одному только) принадлежит милость и спасение, Боже наш, и Тебе славословие приносим, Отцу и Сыну и Святому Духу, ныне и присно и во веки веков. Аминь.




\end{mymulticols}

\mychapterending


\mychapter{Молитва на всякую немощь}\begin{mymulticols}
%http://www.molitvoslov.org/text550.htm 
 


Владыко Вседержителю, Врачу душ и телес, смиряяй и возносяй, наказуяй и паки исцеляяй, брата нашего \myemph{( имя)} немощствующа посети милостию Твоею, простри мышцу Твою, исполнену исцеления и врачбы, и исцели его, возставляй от одра и немощи, запрети духу немощи, остави от него всяку язву, всяку болезнь, всяку рану, всяку огневицу и трясавицу. И аще есть в нем согрешение или беззаконие, ослаби, остави, прости, Твоего ради человеколюбия.




\end{mymulticols}

\mychapterending


\mychapter{Молитва за немощного и неспящего}\begin{mymulticols}
%http://www.molitvoslov.org/text554.htm 
 


Боже Великий, Хвальный и Непостижимый, и Неисповедимый, создавый человека рукою Твоею, персть взем от земли и образом Твоим почтивый его, явися на рабе Твоем \myemph{ (имя)}  и даждь ему сон успокоения, сон телесный, здравия и спасения живота, и крепость душевную и телесную. Сам убо, Человеколюбче Царю, предстани и ныне наитием Святаго Твоего Духа, и посети раба Твоего \myemph{ (имя)}, даруй ему здравие, крепость и благомощие Твоею благостию: яко от Тебе есть всяко даяние благо и всяк дар совершен. Ты бо еси Врач душ наших, и Тебе славу, и благодарение, и поклонение возсылаем со Безначальным Твоим Отцем и с Пресвятым и Благим и Животворящим Твоим Духом, ныне и присно и во веки веков. Аминь. 


\myemph{ О том же молятся святым семи отрокам и Ангелу Хранителю болящего.}




\end{mymulticols}

\mychapterending


\mychapter{Молитва о том, чтобы с любовью ухаживать за болящими}\begin{mymulticols}
%http://www.molitvoslov.org/text549.htm 
 


Господи, Иисусе Христе, Сыне Бога Живаго, Агнче Божий, вземляй грехи мира, Пастырю добрый, положивый душу Твою за овцы Твоя, Небесный Врачу душ и телес наших, исцеляяй всякий недуг и всякую язву в людех Твоих! Тебе припадаю, помози мне, недостойной рабе Твоей. Призри, Многомилостиве, на делание и служение мое, даждь ми быти верною в мале; послужити болящим, Тебе ради, носити немощи немощных, и не себе, но Тебе Единому угождати во вся дни живота моего. Ты бо рекл еси, о, Сладчайший Иисусе: «Понеже сотвористе единому от сих братий Моих меньших, Мне сотвористе». Ей, Господи, суди мне, грешной, по сему глаголу Твоему, яко да сподоблюся творити благую Твою волю во отраду и утешение искушаемых, недугующих раб Твоих, ихже искупил еси честною Твоею Кровию. Ниспосли ми благодать Твою, попаляющую во мне страстей терние, призвавый мя, грешную, на дело служения о Имени Твоем; без Тебе не можем творити ничесоже: посети убо нощию и искуси сердце мое, внегда предстояти ми у возглавия болящих и низверженных; уязви душу мою Твоею любовию, вся терпящею и николиже отпадающею. Тогда возмогу, Тобою укрепляема, подвигом добрым подвизатися и веру соблюсти, даже до последнего моего издыхания. Ты бо еси Источник исцелений душевных же и телесных, Христе Боже наш, и Тебе, яко Спасителю человеков и Жениху душ, грядущему в полунощи, славу и благодарение и поклонение возсылаем, ныне и присно и во веки веков. Аминь. 




\end{mymulticols}

\mychapterending


\mychapter{Молитва Пресвятой Богородице за болящего}
%http://www.molitvoslov.org/content/Molitva-Presvyatoi-Bogoroditse-za-bolyashchego 
 


Пресвятая Богородица, всесильным заступлением Твоим помоги мне умолить Сына Твоего, Бога моего, об исцелении раба Божия \myemph{( имя)}


\mychapterending


\mychapter{Молитва всем святым и ангелам за болящего}
%http://www.molitvoslov.org/content/Molitva-vsem-svyatym-i-angelam-za-bolyashchego 
 


Все святые и ангелы Господни, молите Бога о больном рабе Его \myemph{( имя)}. Аминь.


\mychapterending

