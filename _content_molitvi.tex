\mypart{МОЛИТВЫ}\label{_content_molitvi}
%http://www.molitvoslov.org/content/molitvi

\mychapter{Молитва Господня. Отче наш}\begin{mymulticols}

%http://www.molitvoslov.org/node/37

\myfigure{Spasnatrone}

\'{О}тче наш, \'{И}же ес\'{и} на небес\'{е}х! Да свят\'{и}тся \'{и}мя Тво\'{е}, да при\'{и}дет Ц\'{а}рствие Тво\'{е}, да б\'{у}дет в\'{о}ля Тво\'{я}, \'{я}ко на небес\'{и} и на земл\'{и}. Хлеб наш нас\'{у}щный даждь нам днесь; и ост\'{а}ви нам д\'{о}лги н\'{а}ша, \'{я}коже и мы оставл\'{я}ем должник\'{о}м н\'{а}шим; и не введ\'{и} нас во искуш\'{е}ние, но изб\'{а}ви нас от лук\'{а}ваго.

Отче наш, сущий на небесах! да святится имя Твое; да приидет Царствие Твое; да будет воля Твоя и на земле, как на небе; хлеб наш насущный дай нам на сей день; и прости нам долги наши, как и мы прощаем должникам нашим; и не введи нас в искушение, но избавь нас от лукавого. Ибо Твое есть Царство и сила и слава во веки. Аминь. (Матф. 6:9--13)


\end{mymulticols}

\mychapterending


\mychapter{Иисусова молитва}
%http://www.molitvoslov.org/text596.htm

\myfigure[0.5]{456}

{\centering Г\'{о}споди Иис\'{у}се Христ\'{е}, С\'{ы}не Б\'{о}жий, пом\'{и}луй мя, гр\'{е}шнаго.\par}

\mychapterending

\mychapter{Молитва Честн\'{о}му Кресту (Да воскреснет Бог\ldots)}\begin{mymulticols}
%/content/molitva-Chestnomu-Krestu

\myfigure{cross.jpg}

\mysubtitle{Знаменуй себя крестом и говори молитву Честн\'{о}му Кресту}

Да воскр\'{е}снет Бог, и расточ\'{а}тся враз\'{и} Ег\'{о}, и да беж\'{а}т от лиц\'{а} Ег\'{о} ненав\'{и}дящии Ег\'{о}. \'{Я}ко исчез\'{а}ет дым, да исч\'{е}знут; \'{я}ко т\'{а}ет воск от лиц\'{а} огн\'{я}, т\'{а}ко да пог\'{и}бнут б\'{е}си от лиц\'{а} л\'{ю}бящих Б\'{о}га и зн\'{а}менующихся кр\'{е}стным зн\'{а}мением, и в вес\'{е}лии глаг\'{о}лющих: р\'{а}дуйся, Пречестн\'{ы}й и Животвор\'{я}щий Кр\'{е}сте Госп\'{о}день, прогон\'{я}яй б\'{е}сы с\'{и}лою на теб\'{е} проп\'{я}таго Г\'{о}спода н\'{а}шего Иис\'{у}са Христ\'{а}, во ад сш\'{е}дшаго и попр\'{а}вшего с\'{и}лу ди\'{а}волю, и даров\'{а}вшаго нам теб\'{е} Крест Свой Честн\'{ы}й на прогн\'{а}ние вс\'{я}каго супост\'{а}та. О, Пречестн\'{ы}й и Животвор\'{я}щий Кр\'{е}сте Госп\'{о}день! Помог\'{а}й ми со Свят\'{о}ю Госпож\'{е}ю Д\'{е}вою Богор\'{о}дицею и со вс\'{е}ми свят\'{ы}ми во в\'{е}ки. Ам\'{и}нь.

\myemph{Ил\'{и} кратко:}

Оград\'{и} мя, Г\'{о}споди, с\'{и}лою Честн\'{а}го и Животвор\'{я}щаго Твоег\'{о} Крест\'{а}, и сохр\'{а}ни мя от вс\'{я}каго зла.

\end{mymulticols}

\mychapterending

\mychapter{Благодарение за всякое благодеяние Божие}\begin{mymulticols}
%http://www.molitvoslov.org/text889.htm

\myfigure{777}

\mysubtitle{Тропарь, глас 4-й}

Благода'рни су'ще недосто'йнии раби' Твои', Го'споди, о Твои'х вели'ких благодея'ниих на нас бы'вших, сла'вяще Тя хва'лим, благослови'м, благодари'м, пое'м и велича'ем Твое' благоутро'бие, и ра'бски любо'вию вопие'м Ти: Благоде'телю Спа'се наш, сла'ва Тебе'.

\mysubtitle{Кондак, глас 3-й}

Твои'х благодея'ний и даро'в ту'не, я'ко раби' непотре'бнии, сподо'бльшеся, Влады'ко, к Тебе' усе'рдно притека'юще, благодаре'ние по си'ле прино'сим, и Тебе' я'ко Благоде'теля и Творца' сла'вяще, вопие'м: сла'ва Тебе', Бо'же Всеще'дрый.

\slavainynen

\Bogorodichen{Богоро'дице, христиа'ном Помо'щнице, Твое' предста'тельство стяжа'вше раби' Твои', благода'рно Тебе' вопие'м: ра'дуйся, Пречи'стая Богоро'дице Де'во, и от всех нас бед Твои'ми моли'твами всегда' изба'ви, Еди'на вско'ре предста'тельствующая.}

\end{mymulticols}

\mychapterending


\mychapter{Молитвы утренние}\begin{mymulticols}
%http://www.molitvoslov.org/text893.htm

\myfigure{795}

\footnote{Выделенные от основного текста \myemph{примечания} и названия молитв не читаются во время молитвы.} \myemph{Востав от сна, прежде всякого другого дела, стань благоговейно, представляя себя пред Всевидящим Богом, и, совершая крестное знамение, произнеси:}

Во \'{и}мя Отц\'{а}, и С\'{ы}на, и Свят\'{а}го Д\'{у}ха, Ам\'{и}нь.

\medskip \myemph{Затем немного подожди, пока все чувства твои не придут в тишину и мысли твои не оставят все земное, и тогда произноси следующие молитвы, без поспешности и со вниманием сердечным:}

\mysubtitle{Молитва мытаря \myemph{(Евангелие от Луки, глава 18, стих 13)}}

Б\'{о}же, м\'{и}лостив б\'{у}ди мне гр\'{е}шному. \myemph{(Поклон)}

\mysubtitle{Молитва предначинательная}

Г\'{о}споди Иис\'{у}се Христ\'{е}, С\'{ы}не Б\'{о}жий, молитв р\'{а}ди Преч\'{и}стыя Твое\'{я} М\'{а}тере и всех свят\'{ы}х, пом\'{и}луй нас. Ам\'{и}нь.

Сл\'{а}ва Теб\'{е}, Б\'{о}же наш, сл\'{а}ва Теб\'{е}.

\mysubtitle{Молитва Святому Духу}

Цар\'{ю} Неб\'{е}сный, Ут\'{е}шителю, Д\'{у}ше \'{и}стины, \'{И}же везд\'{е} сый и вся исполн\'{я}яй, Сокр\'{о}вище благ\'{и}х и ж\'{и}зни Под\'{а}телю, приид\'{и} и всел\'{и}ся в ны, и оч\'{и}сти ны от вс\'{я}кия скв\'{е}рны, и спас\'{и}, Бл\'{а}же, д\'{у}ши н\'{а}ша\footnote{От Пасхи до Вознесения вместо этой молитвы читается тропарь: «Христ\'{о}с воскр\'{е}се из м\'{е}ртвых, см\'{е}ртию смерть попр\'{а}в, и с\'{у}щим во гроб\'{е}х жив\'{о}т даров\'{а}в». \myemph{(Трижды)} От Вознесения до Троицы начинаем молитвы со «Свят\'{ы}й Б\'{о}же\ldots», опуская все предшествующие.
Это замечание относится и к молитвам на сон грядущим.}.

\mysubtitle{Трисвятое}

Свят\'{ы}й Б\'{о}же, Свят\'{ы}й Кр\'{е}пкий, Свят\'{ы}й Безсм\'{е}ртный, пом\'{и}луй нас. \myemph{(Читается трижды, с крестным знамением и поясным поклоном.)} 

Сл\'{а}ва Отц\'{у} и С\'{ы}ну и Свят\'{о}му Д\'{у}ху, и н\'{ы}не и пр\'{и}сно и во в\'{е}ки век\'{о}в. Ам\'{и}нь\footnote{В молитвословах и богослужебных книгах часто пишут кратко «Сл\'{а}ва», «И н\'{ы}не», но читать надо полностью: «Сл\'{а}ва Отц\'{у} и С\'{ы}ну и Свят\'{о}му Д\'{у}ху», «И н\'{ы}не и пр\'{и}сно и во в\'{е}ки век\'{о}в. Ам\'{и}нь»}.

\mysubtitle{Молитва ко Пресвятой Тр\'{о}ице}

Пресвят\'{а}я Тр\'{о}ице, пом\'{и}луй нас; Г\'{о}споди, оч\'{и}сти грех\'{и} н\'{а}ша; Влад\'{ы}ко, прост\'{и} беззак\'{о}ния н\'{а}ша; Свят\'{ы}й, посет\'{и} и исцел\'{и} н\'{е}мощи н\'{а}ша, \'{и}мене Тво\'{е}го р\'{а}ди.

Г\'{о}споди, пом\'{и}луй. \myemph{(Трижды)}.

Сл\'{а}ва Отц\'{у} и С\'{ы}ну и Свят\'{о}му Д\'{у}ху, и н\'{ы}не и пр\'{и}сно и во в\'{е}ки век\'{о}в. Ам\'{и}нь.

\mysubtitle{Молитва Господня}

\'{О}тче наш, \'{И}же ес\'{и} на небес\'{е}х! Да свят\'{и}тся \'{и}мя Тво\'{е}, да при\'{и}дет Ц\'{а}рствие Тво\'{е}, да б\'{у}дет в\'{о}ля Тво\'{я}, \'{я}ко на небес\'{и} и на земл\'{и}. Хлеб наш нас\'{у}щный д\'{а}ждь нам днесь; и ост\'{а}ви нам д\'{о}лги н\'{а}ша, \'{я}коже и мы оставл\'{я}ем должник\'{о}м н\'{а}шим; и не введ\'{и} нас во искуш\'{е}ние, но изб\'{а}ви нас от лук\'{а}ваго.

\mysubtitle{Тропари Троичные}

Вост\'{а}вше от сна, прип\'{а}даем Ти, Бл\'{а}же, и \'{а}нгельскую песнь вопи\'{е}м Ти, С\'{и}льне: Свят, Свят, Свят ес\'{и}, Б\'{о}же, Богор\'{о}дицею пом\'{и}луй нас.

\slavan

От одр\'{а} и сна воздв\'{и}гл мя ес\'{и}, Г\'{о}споди, ум мой просвет\'{и} и с\'{е}рдце, и устн\'{е} мо\'{и} отв\'{е}рзи, во \'{е}же п\'{е}ти Тя, Свят\'{а}я Тр\'{о}ице: Свят, Свят, Свят ес\'{и}, Б\'{о}же, Богор\'{о}дицею пом\'{и}луй нас.

\inynen

Внез\'{а}пно Суди\'{я} при\'{и}дет, и коег\'{о}ждо де\'{я}ния обнаж\'{а}тся, но стр\'{а}хом зовем\footnote{В церковнославянском языке нет звука ё, а поэтому надо читать «зов\'{е}м», а не «зовём», «тво\'{е}», а не «твоё», «мо\'{е}», а не «моё» и т.~д.} в пол\'{у}нощи: Свят, Свят, Свят ес\'{и}, Б\'{о}же, Богор\'{о}дицею пом\'{и}луй нас.

Г\'{о}споди, пом\'{и}луй. \myemph{(12 раз)}

\mysubtitle{Молитва ко Пресвятой Тр\'{о}ице}

От сна вост\'{а}в, благодар\'{ю} Тя, Свят\'{а}я Тр\'{о}ице, \'{я}ко мн\'{о}гия р\'{а}ди Твое\'{я} бл\'{а}гости и долготерп\'{е}ния не прогн\'{е}вался ес\'{и} на мя, лен\'{и}ваго и гр\'{е}шнаго, ниж\'{е} погуб\'{и}л мя ес\'{и} со беззак\'{о}ньми мо\'{и}ми; но человекол\'{ю}бствовал ес\'{и} об\'{ы}чно и в неч\'{а}янии леж\'{а}щаго воздв\'{и}гл мя ес\'{и}, во \'{е}же \'{у}треневати и славосл\'{о}вити держ\'{а}ву Тво\'{ю}. И н\'{ы}не просвет\'{и} мо\'{и} \'{о}чи м\'{ы}сленныя, отв\'{е}рзи мо\'{я} уст\'{а} поуч\'{а}тися словес\'{е}м Тво\'{и}м, и разум\'{е}ти з\'{а}поведи Тво\'{я}, и твор\'{и}ти в\'{о}лю Тво\'{ю}, и п\'{е}ти Тя во исповед\'{а}нии серд\'{е}чнем, и воспев\'{а}ти всесвят\'{о}е \'{и}мя Тво\'{е}, Отц\'{а} и С\'{ы}на и Свят\'{а}го Д\'{у}ха, н\'{ы}не и пр\'{и}сно и во в\'{е}ки век\'{о}в. Ам\'{и}нь.

Приид\'{и}те, поклон\'{и}мся Цар\'{е}ви н\'{а}шему Б\'{о}гу. \myemph{(Поклон)}

Приид\'{и}те, поклон\'{и}мся и припад\'{е}м Христ\'{у}, Цар\'{е}ви н\'{а}шему Б\'{о}гу. \myemph{(Поклон)}

Приид\'{и}те, поклон\'{и}мся и припад\'{е}м Самом\'{у} Христ\'{у}, Цар\'{е}ви и Б\'{о}гу н\'{а}шему. \myemph{(Поклон)}

\mysubtitle{Псалом 50}

\PsalmFifty

\mysubtitle{Символ веры}

\SymbolOfFaith

\mysubtitle{Молитва первая, святого Макария Великого}

Б\'{о}же, оч\'{и}сти мя гр\'{е}шнаго, \'{я}ко никол\'{и}же сотвор\'{и}х благ\'{о}е пред Тоб\'{о}ю; но изб\'{а}ви мя от лук\'{а}ваго, и да б\'{у}дет во мне в\'{о}ля Тво\'{я}, да неосужд\'{е}нно отв\'{е}рзу уст\'{а} мо\'{я} недост\'{о}йная и восхвал\'{ю} \'{и}мя Тво\'{е} свят\'{о}е, Отц\'{а} и С\'{ы}на и Свят\'{а}го Д\'{у}ха, н\'{ы}не и пр\'{и}сно и во в\'{е}ки век\'{о}в. Ам\'{и}нь.

\mysubtitle{ Молитва вторая, того же святого}

От сна вост\'{а}в, пол\'{у}нощную песнь принош\'{у} Ти, Сп\'{а}се, и прип\'{а}дая вопи\'{ю} Ти: не д\'{а}ждь ми усн\'{у}ти во грех\'{о}вней см\'{е}рти, но ущ\'{е}дри мя, распн\'{ы}йся в\'{о}лею, и леж\'{а}щаго мя в л\'{е}ности ускор\'{и}в возст\'{а}ви, и спас\'{и} мя в предсто\'{я}нии и мол\'{и}тве, и по сне нощн\'{е}м возси\'{я}й ми день безгр\'{е}шен, Христ\'{е} Б\'{о}же, и спас\'{и} мя.

\mysubtitle{Молитва третья, того же святого}

К Теб\'{е}, Влад\'{ы}ко Человекол\'{ю}бче, от сна вост\'{а}в прибег\'{а}ю, и на дел\'{а} Тво\'{я} подвиз\'{а}юся милос\'{е}рдием Тво\'{и}м, и мол\'{ю}ся Теб\'{е}: помоз\'{и} ми на вс\'{я}кое вр\'{е}мя, во вс\'{я}кой в\'{е}щи, и изб\'{а}ви мя от вс\'{я}кия мирск\'{и}я зл\'{ы}я в\'{е}щи и ди\'{а}вольскаго поспеш\'{е}ния, и спас\'{и} мя, и введ\'{и} в Ц\'{а}рство Тво\'{е} в\'{е}чное. Ты бо ес\'{и} мой Сотвор\'{и}тель и вс\'{я}кому бл\'{а}гу Пром\'{ы}сленник и Под\'{а}тель, о Теб\'{е} же все упов\'{а}ние мо\'{е}, и Теб\'{е} сл\'{а}ву возсыл\'{а}ю, н\'{ы}не и пр\'{и}сно и во в\'{е}ки век\'{о}в. Ам\'{и}нь.

\mysubtitle{Молитва четвертая, того же святого}

Г\'{о}споди, \'{И}же мн\'{о}гою Тво\'{е}ю бл\'{а}гостию и вел\'{и}кими щедр\'{о}тами Тво\'{и}ми дал ес\'{и} мне, раб\'{у} Твоем\'{у}, мимош\'{е}дшее вр\'{е}мя н\'{о}щи се\'{я} без нап\'{а}сти прейт\'{и} от вс\'{я}каго зла прот\'{и}вна; Ты Сам, Влад\'{ы}ко, вс\'{я}ческих Тв\'{о}рче, спод\'{о}би мя \'{и}стинным Тво\'{и}м св\'{е}том и просвещ\'{е}нным с\'{е}рдцем твор\'{и}ти в\'{о}лю Тво\'{ю}, н\'{ы}не и пр\'{и}сно и во в\'{е}ки век\'{о}в. Ам\'{и}нь.

\mysubtitle{Молитва пятая, святого Василия Великого}

Г\'{о}споди Вседерж\'{и}телю, Б\'{о}же сил и вс\'{я}кия пл\'{о}ти, в в\'{ы}шних жив\'{ы}й и на смир\'{е}нныя призир\'{а}яй, с\'{е}рдца же и утр\'{о}бы испыт\'{у}яй и сокров\'{е}нная челов\'{е}ков \'{я}ве предв\'{е}дый, Безнач\'{а}льный и Приснос\'{у}щный Св\'{е}те, у Нег\'{о} же несть премен\'{е}ние, ил\'{и} прелож\'{е}ния осен\'{е}ние; Сам, Безсм\'{е}ртный Цар\'{ю}, приим\'{и} мол\'{е}ния н\'{а}ша, \'{я}же в насто\'{я}щее вр\'{е}мя, на мн\'{о}жество Тво\'{и}х щедр\'{о}т дерз\'{а}юще, от скв\'{е}рных к Теб\'{е} уст\'{е}н твор\'{и}м, и ост\'{а}ви нам прегреш\'{е}ния н\'{а}ша, \'{я}же д\'{е}лом, и сл\'{о}вом, и м\'{ы}слию, в\'{е}дением, ил\'{и} нев\'{е}дением согреш\'{е}нная н\'{а}ми; и оч\'{и}сти ны от вс\'{я}кия скв\'{е}рны пл\'{о}ти и д\'{у}ха. И д\'{а}руй нам б\'{о}дренным с\'{е}рдцем и тр\'{е}звенною м\'{ы}слию всю насто\'{я}щаго жити\'{я} нощь прейт\'{и}, ожид\'{а}ющим приш\'{е}ствия св\'{е}тлаго и явл\'{е}ннаго дне Единор\'{о}днаго Тво\'{е}го С\'{ы}на, Г\'{о}спода и Б\'{о}га и Сп\'{а}са н\'{а}шего Иис\'{у}са Христ\'{а}, в \'{о}ньже со сл\'{а}вою Суди\'{я} всех при\'{и}дет, ком\'{у}ждо отд\'{а}ти по д\'{е}лом ег\'{о}; да не п\'{а}дше и облен\'{и}вшеся, но б\'{о}дрствующе и воздв\'{и}жени в д\'{е}лание обр\'{я}щемся гот\'{о}ви, в р\'{а}дость и Бож\'{е}ственный черт\'{о}г сл\'{а}вы Ег\'{о} совн\'{и}дем, ид\'{е}же пр\'{а}зднующих глас непрест\'{а}нный, и неизреч\'{е}нная сл\'{а}дость зр\'{я}щих Тво\'{е}го лиц\'{а} добр\'{о}ту неизреч\'{е}нную. Ты бо ес\'{и} \'{и}стинный Свет, просвещ\'{а}яй и освящ\'{а}яй вс\'{я}ческая, и Тя по\'{е}т вся тварь во в\'{е}ки век\'{о}в. Ам\'{и}нь.

\mysubtitle{Молитва шестая, того же святого}

Тя благослов\'{и}м, в\'{ы}шний Б\'{о}же и Г\'{о}споди м\'{и}лости, твор\'{я}щаго пр\'{и}сно с н\'{а}ми вел\'{и}кая же и неизсл\'{е}дованная, сл\'{а}вная же и уж\'{а}сная, \'{и}хже несть числ\'{а}, под\'{а}вшаго нам сон во упоко\'{е}ние н\'{е}мощи н\'{а}шея, и ослабл\'{е}ние труд\'{о}в многотр\'{у}дныя пл\'{о}ти. Благодар\'{и}м Тя, якo не погуб\'{и}л ес\'{и} нас со беззак\'{о}ньми н\'{а}шими, но человекол\'{ю}бствовал ес\'{и} об\'{ы}чно, и в неч\'{а}янии леж\'{а}щия ны воздв\'{и}гл ес\'{и}, во \'{е}же славосл\'{о}вити держ\'{а}ву Тво\'{ю}. Т\'{е}мже м\'{о}лим безм\'{е}рную Тво\'{ю} бл\'{а}гость, просвет\'{и} н\'{а}ша м\'{ы}сли, очес\'{а}, и ум наш от т\'{я}жкаго сна л\'{е}ности возст\'{а}ви: отв\'{е}рзи н\'{а}ша уст\'{а}, и исп\'{о}лни я Тво\'{е}го хвал\'{е}ния, \'{я}ко да возм\'{о}жем непокол\'{е}блемо п\'{е}ти же и испов\'{е}датися Теб\'{е}, во всех, и от всех сл\'{а}вимому Б\'{о}гу, Безнач\'{а}льному Отц\'{у}, со Единор\'{о}дным Тво\'{и}м С\'{ы}ном, и Всесвят\'{ы}м и Благ\'{и}м и Животвор\'{я}щим Тво\'{и}м Д\'{у}хом, н\'{ы}не и пр\'{и}сно и во в\'{е}ки век\'{о}в. Ам\'{и}нь.

\mysubtitle{Молитва седьмая, ко Пресвятой Богор\'{о}дице}

Воспев\'{а}ю благод\'{а}ть Тво\'{ю}, Влад\'{ы}чице, мол\'{ю} Тя, ум мой облагодат\'{и}. Ступ\'{а}ти пр\'{а}во мя наст\'{а}ви, пут\'{е}м Христ\'{о}вых з\'{а}поведей. Бд\'{е}ти к п\'{е}сни укреп\'{и}, ун\'{ы}ния сон отгон\'{я}ющи. Св\'{я}зана плен\'{и}цами грехопад\'{е}ний, мольб\'{а}ми Тво\'{и}ми разреш\'{и}, Богонев\'{е}сто. В нощ\'{и} мя и во дни сохран\'{я}й, бор\'{ю}щих враг избавл\'{я}ющи мя. Жизнод\'{а}теля Б\'{о}га р\'{о}ждшая, умерщвл\'{е}на мя страстьм\'{и} ожив\'{и}. \'{Я}же Свет невеч\'{е}рний р\'{о}ждшая, д\'{у}шу мо\'{ю} осл\'{е}пшую просвет\'{и}. О д\'{и}вная Влад\'{ы}чня пал\'{а}то, дом Д\'{у}ха Бож\'{е}ственна мен\'{е} сотвор\'{и}. Врач\'{а} р\'{о}ждшая, уврач\'{у}й душ\'{и} мое\'{я} многол\'{е}тныя стр\'{а}сти. Волн\'{у}ющася жит\'{е}йскою б\'{у}рею, ко стез\'{и} мя пока\'{я}ния напр\'{а}ви. Изб\'{а}ви мя огн\'{я} в\'{е}чнующаго, и ч\'{е}рвия же зл\'{а}го, и т\'{а}ртара. Да мя не яв\'{и}ши бес\'{о}м р\'{а}дование, \'{и}же мн\'{о}гим грех\'{о}м пов\'{и}нника. Н\'{о}ва сотвор\'{и} мя, обетш\'{а}вшаго неч\'{у}вственными, Пренепор\'{о}чная, согреш\'{е}нии. Стр\'{а}нна м\'{у}ки вс\'{я}кия покаж\'{и} мя, и всех Влад\'{ы}ку умол\'{и}. Неб\'{е}сная ми улуч\'{и}ти вес\'{е}лия, со вс\'{е}ми свят\'{ы}ми, спод\'{о}би. Пресвят\'{а}я Д\'{е}во, усл\'{ы}ши глас непотр\'{е}бнаго раб\'{а} Твоег\'{о}. Стру\'{ю} дав\'{а}й мне слез\'{а}м, Преч\'{и}стая, душ\'{и} мое\'{я} скв\'{е}рну очищ\'{а}ющи. Стен\'{а}ния от с\'{е}рдца принош\'{у} Ти непрест\'{а}нно, ус\'{е}рдствуй, Влад\'{ы}чице. Мол\'{е}бную сл\'{у}жбу мо\'{ю} приим\'{и}, и Б\'{о}гу благоутр\'{о}бному принес\'{и}. Прев\'{ы}шшая \'{А}нгел, мирск\'{а}го мя прев\'{ы}шша сл\'{и}тия сотвор\'{и}. Светон\'{о}сная С\'{е}не неб\'{е}сная, дух\'{о}вную благод\'{а}ть во мне напр\'{а}ви. Р\'{у}це возд\'{е}ю и устн\'{е} к похвал\'{е}нию, оскверн\'{е}ны скв\'{е}рною, Всенепор\'{о}чная. Душетл\'{е}нных мя п\'{а}костей изб\'{а}ви, Христ\'{а} прил\'{е}жно умол\'{я}ющи; Ем\'{у}же честь и поклон\'{е}ние подоб\'{а}ет, н\'{ы}не и пр\'{и}сно и во в\'{е}ки век\'{о}в. Ам\'{и}нь.

\mysubtitle{Молитва восьмая, ко Г\'{о}споду н\'{а}шему Иис\'{у}су Христ\'{у}}

Многом\'{и}лостиве и Всем\'{и}лостиве Б\'{о}же мой, Г\'{о}споди Иис\'{у}се Христ\'{е}, мн\'{о}гия р\'{а}ди любв\'{е} сшел и воплот\'{и}лся ес\'{и}, \'{я}ко да спас\'{е}ши всех. И п\'{а}ки, Сп\'{а}се, спас\'{и} мя по благод\'{а}ти, мол\'{ю} Тя; \'{а}ще бо от дел спас\'{е}ши мя, несть се благод\'{а}ть, и дар, но долг п\'{а}че. Ей, мн\'{о}гий в щедр\'{о}тах и неизреч\'{е}нный в м\'{и}лости! В\'{е}руяй бо в Мя, рекл ес\'{и}, о Христ\'{е} мой, жив б\'{у}дет и не \'{у}зрит см\'{е}рти во в\'{е}ки. \'{А}ще \'{у}бо в\'{е}ра, \'{я}же в Тя, спас\'{а}ет отч\'{а}янныя, се в\'{е}рую, спас\'{и} мя, \'{я}ко Бог мой ес\'{и} Ты и Созд\'{а}тель. В\'{е}ра же вм\'{е}сто дел да вмен\'{и}тся мне, Б\'{о}же мой, не обр\'{я}щеши бо дел отн\'{ю}д оправд\'{а}ющих мя. Но та в\'{е}ра мо\'{я} да довл\'{е}ет вм\'{е}сто всех, та да отвещ\'{а}ет, та да оправд\'{и}т мя, та да пок\'{а}жет мя прич\'{а}стника сл\'{а}вы Твое\'{я} в\'{е}чныя. Да не \'{у}бо пох\'{и}тит мя сатан\'{а}, и похв\'{а}лится, Сл\'{о}ве, \'{е}же отт\'{о}ргнути мя от Твое\'{я} рук\'{и} и огр\'{а}ды; но ил\'{и} хощ\'{у}, спас\'{и} мя, ил\'{и} не хощ\'{у}, Христ\'{е} Сп\'{а}се мой, предвар\'{и} ск\'{о}ро, ск\'{о}ро погиб\'{о}х: Ты бо ес\'{и} Бог мой от чр\'{е}ва м\'{а}тере мое\'{я}. Спод\'{о}би мя, Г\'{о}споди, н\'{ы}не возлюб\'{и}ти Тя, \'{я}коже возлюб\'{и}х иногд\'{а} той с\'{а}мый грех; и п\'{а}ки пораб\'{о}тати Теб\'{е} без л\'{е}ности т\'{о}щно, \'{я}коже пораб\'{о}тах пр\'{е}жде сатан\'{е} льст\'{и}вому. Наип\'{а}че же пораб\'{о}таю Теб\'{е}, Г\'{о}споду и Б\'{о}гу моем\'{у} Иис\'{у}су Христ\'{у}, во вся дни живот\'{а} моег\'{о}, н\'{ы}не и пр\'{и}сно и во в\'{е}ки век\'{о}в. Ам\'{и}нь.

\mysubtitle{Молитва девятая, к Ангелу хранителю}

Свят\'{ы}й \'{А}нгеле, предсто\'{я}й ока\'{я}нной мо\'{е}й душ\'{и} и стр\'{а}стной мо\'{е}й ж\'{и}зни, не ост\'{а}ви мен\'{е} гр\'{е}шнаго, ниж\'{е} отступ\'{и} от мен\'{е} за невоздерж\'{а}ние мо\'{е}. Не д\'{а}ждь м\'{е}ста лук\'{а}вому д\'{е}мону облад\'{а}ти мн\'{о}ю, нас\'{и}льством см\'{е}ртнаго сег\'{о} телес\'{е}; укреп\'{и} б\'{е}дствующую и худ\'{у}ю мо\'{ю} р\'{у}ку и наст\'{а}ви мя на путь спас\'{е}ния. Ей, свят\'{ы}й \'{А}нгеле Б\'{о}жий, хран\'{и}телю и покров\'{и}телю ока\'{я}нныя мое\'{я} душ\'{и} и т\'{е}ла, вся мне прост\'{и}, ел\'{и}кими тя оскорб\'{и}х во вся дни живот\'{а} моег\'{о}, и \'{а}ще что согреш\'{и}х в преш\'{е}дшую нощь си\'{ю}, покр\'{ы}й мя в насто\'{я}щий день, и сохран\'{и} мя от вс\'{я}каго искуш\'{е}ния прот\'{и}внаго, да ни в к\'{о}ем грес\'{е} прогн\'{е}ваю Б\'{о}га, и мол\'{и}ся за мя ко Г\'{о}споду, да утверд\'{и}т мя в стр\'{а}се Сво\'{е}м, и дост\'{о}йна пок\'{а}жет мя раб\'{а} Свое\'{я} бл\'{а}гости. Ам\'{и}нь.

\mysubtitle{Молитва десятая, ко Пресвятой Богор\'{о}дице}

Пресвят\'{а}я Влад\'{ы}чице мо\'{я} Богор\'{о}дице, свят\'{ы}ми Тво\'{и}ми и всес\'{и}льными мольб\'{а}ми отжен\'{и} от мен\'{е}, смир\'{е}ннаго и ока\'{я}ннаго раб\'{а} Твоег\'{о}, ун\'{ы}ние, забв\'{е}ние, нераз\'{у}мие, нерад\'{е}ние, и вся скв\'{е}рная, лук\'{а}вая и х\'{у}льная помышл\'{е}ния от ока\'{я}ннаго моег\'{о} с\'{е}рдца и от помрач\'{е}ннаго ум\'{а} моег\'{о}; и погас\'{и} пл\'{а}мень страст\'{е}й мо\'{и}х, \'{я}ко нищ есмь и ока\'{я}нен. И изб\'{а}ви мя от мн\'{о}гих и л\'{ю}тых воспомин\'{а}ний и предпри\'{я}тий, и от всех действ злых свобод\'{и} мя. \'{Я}ко благослов\'{е}на ес\'{и} от всех род\'{о}в, и сл\'{а}вится пречестн\'{о}е \'{и}мя Тво\'{е} во в\'{е}ки век\'{о}в. Ам\'{и}нь.

\mysubtitle{Молитвенное призывание святого, имя которого носишь}

Мол\'{и} Б\'{о}га о мне, свят\'{ы}й уг\'{о}дниче Б\'{о}жий \myemph{(имя)}, \'{я}ко аз ус\'{е}рдно к теб\'{е} прибег\'{а}ю, ск\'{о}рому пом\'{о}щнику и мол\'{и}твеннику о душ\'{е} мо\'{е}й.

\mysubtitle{Песнь Пресвятой Богородице}

Богор\'{о}дице Д\'{е}во, р\'{а}дуйся, Благод\'{а}тная Мар\'{и}е, Госп\'{о}дь с Тоб\'{о}ю; благослов\'{е}на Ты в жен\'{а}х и благослов\'{е}н плод чр\'{е}ва Твоег\'{о}, \'{я}ко Сп\'{а}са родил\'{а} ес\'{и} душ н\'{а}ших.

\mysubtitle{Тропарь Кресту и молитва за отечество}

Спас\'{и}, Г\'{о}споди, л\'{ю}ди Тво\'{я}, и благослов\'{и} досто\'{я}ние Тво\'{е}, поб\'{е}ды правосл\'{а}вным христи\'{а}ном на сопрот\'{и}вныя д\'{а}руя, и Тво\'{е} сохран\'{я}я Крест\'{о}м Тво\'{и}м ж\'{и}тельство.

\mysubtitle{Молитва о живых}

Спас\'{и}, Г\'{о}споди, и пом\'{и}луй отц\'{а} моег\'{о} дух\'{о}внаго \myemph{(имя)}, род\'{и}телей мо\'{и}х \myemph{(имена)}, ср\'{о}дников \myemph{(имена)}, нач\'{а}льников, наст\'{а}вников, благод\'{е}телей \myemph{(имена)} и всех правосл\'{а}вных христи\'{а}н.

\mysubtitle{Молитва о усопших}

Упок\'{о}й, Г\'{о}споди, д\'{у}ши ус\'{о}пших раб Тво\'{и}х: род\'{и}телей мо\'{и}х, ср\'{о}дников, благод\'{е}телей \myemph{(имена)}, и всех правосл\'{а}вных христи\'{а}н, и прост\'{и} им вся согреш\'{е}ния в\'{о}льная и нев\'{о}льная, и д\'{а}руй им Ц\'{а}рствие Неб\'{е}сное.

\myemph{Если можешь, читай вместо кратких молитв о живых и усопших этот помянник:}

\mysubtitle{О живых}

Помян\'{и}, Г\'{о}споди Иис\'{у}се Христ\'{е}, Б\'{о}же наш, м\'{и}лости и щедр\'{о}ты Тво\'{я} от в\'{е}ка с\'{у}щия, \'{и}хже р\'{а}ди и вочелов\'{е}чился ес\'{и}, и расп\'{я}тие и смерть, спас\'{е}ния р\'{а}ди пр\'{а}во в Тя в\'{е}рующих, претерп\'{е}ти изв\'{о}лил ес\'{и}; и воскр\'{е}с из м\'{е}ртвых, возн\'{е}слся ес\'{и} на небес\'{а} и сед\'{и}ши одесн\'{у}ю Б\'{о}га Отц\'{а}, и призир\'{а}еши на смир\'{е}нныя мольб\'{ы} всем с\'{е}рдцем призыв\'{а}ющих Тя: приклон\'{и} \'{у}хо Тво\'{е}, и усл\'{ы}ши смир\'{е}нное мол\'{е}ние мен\'{е}, непотр\'{е}бнаго раб\'{а} Твоег\'{о}, в вон\'{ю} благоух\'{а}ния дух\'{о}внаго, Теб\'{е} за вся л\'{ю}ди Тво\'{я} принос\'{я}щаго. И в п\'{е}рвых помян\'{и} Ц\'{е}рковь Тво\'{ю} Свят\'{у}ю, Соб\'{о}рную и Ап\'{о}стольскую, \'{ю}же снабд\'{е}л ес\'{и} честн\'{о}ю Тво\'{е}ю Кр\'{о}вию, и утверд\'{и}, и укреп\'{и}, и разшир\'{и}, умн\'{о}жи, умир\'{и}, и непребор\'{и}му \'{а}довыми врат\'{ы} во в\'{е}ки сохран\'{и}; раздир\'{а}ния Церкв\'{е}й утиш\'{и}, шат\'{а}ния яз\'{ы}ческая угас\'{и}, и ерес\'{е}й вост\'{а}ния ск\'{о}ро разор\'{и} и искорен\'{и}, и в ничт\'{о}же с\'{и}лою Свят\'{а}го Твоег\'{о} Д\'{у}ха обрат\'{и}. \myemph{(Поклон)}

Спас\'{и}, Г\'{о}споди, и пом\'{и}луй богохран\'{и}мую стран\'{у} н\'{а}шу, вл\'{а}сти и в\'{о}инство е\'{я}, оград\'{и} м\'{и}ром держ\'{а}ву их, и покор\'{и} под н\'{о}зе правосл\'{а}вных вс\'{я}каго враг\'{а} и супост\'{а}та, и глаг\'{о}ли м\'{и}рная и благ\'{а}я в сердц\'{а}х их о Ц\'{е}ркви Тво\'{е}й Свят\'{е}й, и о всех л\'{ю}дех Тво\'{и}х: да т\'{и}хое и безм\'{о}лвное жити\'{е} пожив\'{е}м во правов\'{е}рии, и во вс\'{я}ком благоч\'{е}стии и чистот\'{е}. \myemph{(Поклон)}

Спас\'{и}, Г\'{о}споди, и пом\'{и}луй Вел\'{и}каго Господ\'{и}на и Отц\'{а} н\'{а}шего Свят\'{е}йшего Патри\'{а}рха Кир\'{и}лла, преосвящ\'{е}нныя митропол\'{и}ты, архиеп\'{и}скопы и еп\'{и}скопы правосл\'{а}вныя, иер\'{е}и же и ди\'{а}коны, и весь пр\'{и}чет церк\'{о}вный, \'{я}же пост\'{а}вил ес\'{и} паст\'{и} слов\'{е}сное Тво\'{е} ст\'{а}до, и мол\'{и}твами их пом\'{и}луй и спас\'{и} мя гр\'{е}шнаго. \myemph{(Поклон)}

Спас\'{и}, Г\'{о}споди, и пом\'{и}луй отц\'{а} моег\'{о} дух\'{о}внаго \myemph{(имя его)}, и свят\'{ы}ми ег\'{о} мол\'{и}твами прост\'{и} мо\'{я} согреш\'{е}ния. \myemph{(Поклон)}

Спас\'{и}, Г\'{о}споди, и пом\'{и}луй род\'{и}тели мо\'{я} \myemph{(имена их)}, бр\'{а}тию и с\'{е}стры, и ср\'{о}дники мо\'{я} по пл\'{о}ти, и вся бл\'{и}жния р\'{о}да моег\'{о}, и др\'{у}ги, и д\'{а}руй им м\'{и}рная Тво\'{я} и прем\'{и}рная благ\'{а}я. \myemph{(Поклон)}

Спас\'{и}, Г\'{о}споди, и пом\'{и}луй по мн\'{о}жеству щедр\'{о}т Тво\'{и}х вся священно\'{и}ноки, \'{и}ноки же и \'{и}нокини, и вся в д\'{е}встве же и благогов\'{е}нии и п\'{о}стничестве жив\'{у}щия в монастыр\'{е}х, в пуст\'{ы}нях, в пещ\'{е}рах, гор\'{а}х, столп\'{е}х, затв\'{о}рех, разс\'{е}линах к\'{а}менных, остров\'{е}х же морск\'{и}х, и на вс\'{я}ком м\'{е}сте влад\'{ы}чествия Твоег\'{о} правов\'{е}рно жив\'{у}щия, и благоч\'{е}стно служ\'{а}щия Ти, и мол\'{я}щияся Теб\'{е}: облегч\'{и} им тягот\'{у}, и ут\'{е}ши их скорбь, и к п\'{о}двигу о Теб\'{е} с\'{и}лу и кр\'{е}пость им под\'{а}ждь, и мол\'{и}твами их д\'{а}руй ми оставл\'{е}ние грех\'{о}в. \myemph{(Поклон)}

Спас\'{и}, Г\'{о}споди, и пом\'{и}луй ст\'{а}рцы и \'{ю}ныя, н\'{и}щия и сирот\'{ы} и вдов\'{и}цы, и с\'{у}щия в бол\'{е}зни и в печ\'{а}лех, бед\'{а}х же и ск\'{о}рбех, обсто\'{я}ниих и плен\'{е}ниих, темн\'{и}цах же и заточ\'{е}ниих, изр\'{я}днее же в гон\'{е}ниих, Теб\'{е} р\'{а}ди и в\'{е}ры правосл\'{а}вныя, от яз\'{ы}к безб\'{о}жных, от отст\'{у}пник и от еретик\'{о}в, с\'{у}щия раб\'{ы} Тво\'{я}, и помян\'{и} я, посет\'{и}, укреп\'{и}, ут\'{е}ши, и вск\'{о}ре с\'{и}лою Тво\'{е}ю осл\'{а}бу, своб\'{о}ду и изб\'{а}ву им под\'{а}ждь. \myemph{(Поклон)} 

Спас\'{и}, Г\'{о}споди, и пом\'{и}луй благотвор\'{я}щия нам, м\'{и}лующия и пит\'{а}ющия нас, д\'{а}вшия нам м\'{и}лостыни, и запов\'{е}давшия нам недост\'{о}йным мол\'{и}тися о них, и упокоев\'{а}ющия нас, и сотвор\'{и} м\'{и}лость Тво\'{ю} с н\'{и}ми, д\'{а}руя им вся, \'{я}же ко спас\'{е}нию прош\'{е}ния, и в\'{е}чных благ воспри\'{я}тие. \myemph{(Поклон)}

Спас\'{и}, Г\'{о}споди, и пом\'{и}луй п\'{о}сланныя в сл\'{у}жбу, путеш\'{е}ствующия, отц\'{ы} и бр\'{а}тию н\'{а}шу, и вся правосл\'{а}вныя христи\'{а}ны. \myemph{(Поклон)}

Спас\'{и}, Г\'{о}споди, и пом\'{и}луй \'{и}хже аз без\'{у}мием мо\'{и}м соблазн\'{и}х, и от пут\'{и} спас\'{и}тельнаго отврат\'{и}х, к д\'{е}лом злым и непод\'{о}бным привед\'{о}х; Бож\'{е}ственным Тво\'{и}м Пр\'{о}мыслом к пут\'{и} спас\'{е}ния п\'{а}ки возврат\'{и}. \myemph{(Поклон)} 

Спас\'{и}, Г\'{о}споди, и пом\'{и}луй ненав\'{и}дящия и об\'{и}дящия мя, и твор\'{я}щия ми нап\'{а}сти, и не ост\'{а}ви их пог\'{и}бнути мен\'{е} р\'{а}ди, гр\'{е}шнаго. \myemph{(Поклон)}

Отступ\'{и}вшия от правосл\'{а}вныя в\'{е}ры и пог\'{и}бельными ересьм\'{и} ослепл\'{е}нныя, св\'{е}том Твоег\'{о} позн\'{а}ния просвет\'{и} и Свят\'{е}й Тво\'{е}й Ап\'{о}стольстей Соб\'{о}рней Ц\'{е}ркви причт\'{и}. \myemph{(Поклон)} 

\mysubtitle{О усопших}

Помян\'{и}, Г\'{о}споди, от жити\'{я} сег\'{о} отш\'{е}дшия правов\'{е}рныя цар\'{и} и цар\'{и}цы, благов\'{е}рныя кн\'{я}зи и княг\'{и}ни, свят\'{е}йшия патри\'{а}рхи, преосвящ\'{е}нныя митропол\'{и}ты, архиеп\'{и}скопы и еп\'{и}скопы правосл\'{а}вныя, во иер\'{е}йстем же и в пр\'{и}чте церк\'{о}внем, и мон\'{а}шестем ч\'{и}не Теб\'{е} послуж\'{и}вшия, и в в\'{е}чных Тво\'{и}х сел\'{е}ниих со свят\'{ы}ми упок\'{о}й. \myemph{(Поклон.)}

Помян\'{и}, Г\'{о}споди, д\'{у}ши ус\'{о}пших раб\'{о}в Тво\'{и}х, род\'{и}телей мо\'{и}х \myemph{(имена их)}, и всех ср\'{о}дников по пл\'{о}ти; и прост\'{и} их вся согреш\'{е}ния в\'{о}льная и нев\'{о}льная, д\'{а}руя им Ц\'{а}рствие и прич\'{а}стие в\'{е}чных Тво\'{и}х благ\'{и}х и Твое\'{я} безкон\'{е}чныя и блаж\'{е}нныя ж\'{и}зни наслажд\'{е}ние. \myemph{(Поклон)} 

Помян\'{и}, Г\'{о}споди, и вся в над\'{е}жди воскрес\'{е}ния и ж\'{и}зни в\'{е}чныя ус\'{о}пшия, отц\'{ы} и бр\'{а}тию н\'{а}шу, и с\'{е}стры, и зде леж\'{а}щия и повс\'{ю}ду, правосл\'{а}вныя христи\'{а}ны, и со свят\'{ы}ми Тво\'{и}ми, ид\'{е}же присещ\'{а}ет свет лиц\'{а} Твоег\'{о}, всел\'{и}, и нас пом\'{и}луй, \'{я}ко Благ и Человекол\'{ю}бец. Ам\'{и}нь. \myemph{(Поклон)} 

Под\'{а}ждь, Г\'{о}споди, оставл\'{е}ние грех\'{о}в всем пр\'{е}жде отш\'{е}дшим в в\'{е}ре и над\'{е}жди воскрес\'{е}ния, отц\'{е}м, бр\'{а}тиям и с\'{е}страм н\'{а}шим и сотвор\'{и} им в\'{е}чную п\'{а}мять. \myemph{(Трижды)}

\mysubtitle{Окончание молитв}

Дост\'{о}йно \'{е}сть \'{я}ко во\'{и}стину блаж\'{и}ти Тя Богор\'{о}дицу, Присноблаж\'{е}нную и Пренепор\'{о}чную и М\'{а}терь Б\'{о}га н\'{а}шего. Честн\'{е}йшую Херув\'{и}м и сл\'{а}внейшую без сравн\'{е}ния Сераф\'{и}м, без истл\'{е}ния Б\'{о}га Сл\'{о}ва р\'{о}ждшую, с\'{у}щую Богор\'{о}дицу Тя велич\'{а}ем\footnote{От Пасхи до Вознесения вместо этой молитвы читается припев и ирмос 9-й песни пасхального канона: «\'{А}нгел вопи\'{я}ше Благод\'{а}тней: Ч\'{и}стая Д\'{е}во, р\'{а}дуйся! И п\'{а}ки рек\'{у}: р\'{а}дуйся! Твой Сын воскр\'{е}се тридн\'{е}вен от гр\'{о}ба и м\'{е}ртвыя воздв\'{и}гнувый; л\'{ю}дие, весел\'{и}теся! Свет\'{и}ся, свет\'{и}ся, н\'{о}вый Иерусал\'{и}ме, сл\'{а}ва бо Госп\'{о}дня на теб\'{е} возси\'{я}. Лик\'{у}й н\'{ы}не и весел\'{и}ся, Си\'{о}не. Ты же, Ч\'{и}стая, крас\'{у}йся, Богор\'{о}дице, о вост\'{а}нии Рождеств\'{а} Твоег\'{о}». Это замечание относится и к вечерним молитвам.} .

\slavainynen

Г\'{о}споди, пом\'{и}луй. \myemph{(Трижды)}

Г\'{о}споди, Иис\'{у}се Христ\'{е}, С\'{ы}не Б\'{о}жий, молитв р\'{а}ди Преч\'{и}стыя Твое\'{я} М\'{а}тере, препод\'{о}бных и богон\'{о}сных от\'{е}ц н\'{а}ших и всех свят\'{ы}х пом\'{и}луй нас. Ам\'{и}нь.

\end{mymulticols}

\mychapterending


\mychapter{Молитвы на сон грядущим}\begin{mymulticols}
%http://www.molitvoslov.org/text2.htm

\myfigure{1_1}

Во \'{и}мя Отц\'{а}, и С\'{ы}на, и Свят\'{а}го Д\'{у}ха. Ам\'{и}нь.

Г\'{о}споди Иис\'{у}се Христ\'{е}, С\'{ы}не Б\'{о}жий, молитв р\'{а}ди Преч\'{и}стыя Твое\'{я} М\'{а}тере, препод\'{о}бных и богон\'{о}сных от\'{е}ц н\'{а}ших и всех свят\'{ы}х, пом\'{и}луй нас. Ам\'{и}нь.

Сл\'{а}ва Теб\'{е}, Б\'{о}же наш, сл\'{а}ва Теб\'{е}.

\TsariuNebesnyj

\TrisviatoePoOtcheNash

\mysubtitle{Тропари}

\TroparPomilujNas

Г\'{о}споди, пом\'{и}луй. \myemph{(12 раз)}

\mysubtitle{Молитва 1-я, святого Макария Великого, к Богу Отцу}

Б\'{о}же в\'{е}чный и Цар\'{ю} вс\'{я}каго созд\'{а}ния, спод\'{о}бивый мя д\'{а}же в час сей досп\'{е}ти, прост\'{и} ми грех\'{и}, \'{я}же сотвор\'{и}х в сей день д\'{е}лом, сл\'{о}вом и помышл\'{е}нием, и оч\'{и}сти, Г\'{о}споди, смир\'{е}нную мо\'{ю} д\'{у}шу от вс\'{я}кия скв\'{е}рны пл\'{о}ти и д\'{у}ха. И д\'{а}ждь ми, Г\'{о}споди, в нощ\'{и} сей сон прейт\'{и} в м\'{и}ре, да вост\'{а}в от смир\'{е}ннаго ми л\'{о}жа, благоугожд\'{у} пресвят\'{о}му \'{и}мени Твоем\'{у}, во вс\'{я} дни живот\'{а} моег\'{о}, и попер\'{у} бор\'{ю}щия мя враг\'{и} плотск\'{и}я и безпл\'{о}тныя. И изб\'{а}ви мя, Г\'{о}споди, от помышл\'{е}ний с\'{у}етных, оскверн\'{я}ющих мя, и п\'{о}хотей лук\'{а}вых. \'{Я}ко Тво\'{е} \'{е}сть ц\'{а}рство, и с\'{и}ла и сл\'{а}ва, Отц\'{а} и С\'{ы}на и Свят\'{а}го Д\'{у}ха, н\'{ы}не и пр\'{и}сно и во в\'{е}ки век\'{о}в. Ам\'{и}нь.

\mysubtitle{Молитва 2-я, святого Антиоха, ко Господу нашему Иисусу Христу}

Вседерж\'{и}телю, Сл\'{о}во \'{О}тчее, Сам соверш\'{е}н сый, Иис\'{у}се Христ\'{е}, мн\'{о}гаго р\'{а}ди милос\'{е}рдия Твоег\'{о} никогд\'{а}же отлуч\'{а}йся мен\'{е}, раб\'{а} Твоег\'{о}, но всегд\'{а} во мне почив\'{а}й. Иис\'{у}се, д\'{о}брый П\'{а}стырю Тво\'{и}х ов\'{е}ц, не пред\'{а}ждь мен\'{е} крамол\'{е} зми\'{и}не, и жел\'{а}нию сатанин\'{у} не ост\'{а}ви мен\'{е}, \'{я}ко с\'{е}мя тли во мне есть. Ты \'{у}бо, Г\'{о}споди Б\'{о}же поклан\'{я}емый, Цар\'{ю} Свят\'{ы}й, Иис\'{у}се Христ\'{е}, сп\'{я}ща мя сохр\'{а}ни немерц\'{а}ющим св\'{е}том, Д\'{у}хом Тво\'{и}м Свят\'{ы}м, \'{И}мже освят\'{и}л ес\'{и} Тво\'{я} ученик\'{и}. Даждь, Г\'{о}споди, и мне, недост\'{о}йному раб\'{у} Твоем\'{у}, спас\'{е}ние Тво\'{е} на л\'{о}жи мо\'{е}м: просвет\'{и} ум мой св\'{е}том р\'{а}зума свят\'{а}го Ев\'{а}нгелия Твоег\'{о}, д\'{у}шу люб\'{о}вию Крест\'{а} Твоег\'{о}, с\'{е}рдце чистот\'{о}ю словес\'{е} Твоег\'{о}, т\'{е}ло мо\'{е} Тво\'{е}ю стр\'{а}стию безстр\'{а}стною, мысль мо\'{ю} Тво\'{и}м смир\'{е}нием сохр\'{а}ни, и воздв\'{и}гни мя во вр\'{е}мя под\'{о}бно на Тво\'{е} славосл\'{о}вие. \'{Я}ко препросл\'{а}влен ес\'{и} со Безнач\'{а}льным Тво\'{и}м Отц\'{е}м и с Пресвят\'{ы}м Д\'{у}хом во в\'{е}ки. Ам\'{и}нь.

\mysubtitle{Молитва 3-я, ко Пресвятому Духу}

Г\'{о}споди, Цар\'{ю} Неб\'{е}сный, Ут\'{е}шителю, Д\'{у}ше \'{и}стины, умилос\'{е}рдися и пом\'{и}луй мя гр\'{е}шнаго раб\'{а} Твоег\'{о}, и отпуст\'{и} ми недост\'{о}йному, и прост\'{и} вс\'{я}, ел\'{и}ка Ти согреш\'{и}х днесь \'{я}ко челов\'{е}к, п\'{а}че же и не \'{я}ко челов\'{е}к, но и гор\'{е}е скот\'{а}, в\'{о}льныя м\'{о}я грех\'{и} и нев\'{о}льныя, в\'{е}домыя и нев\'{е}домыя: \'{я}же от \'{ю}ности и на\'{у}ки злы, и \'{я}же суть от н\'{а}гльства и ун\'{ы}ния. \'{А}ще \'{и}менем Тво\'{и}м кл\'{я}хся, ил\'{и} пох\'{у}лих е в помышл\'{е}нии мо\'{е}м; ил\'{и} ког\'{о} укор\'{и}х; ил\'{и} оклевет\'{а}х ког\'{о} гн\'{е}вом мо\'{и}м, ил\'{и} опеч\'{а}лих, ил\'{и} о чем прогн\'{е}вахся; ил\'{и} солг\'{а}х, ил\'{и} безг\'{о}дно спах, ил\'{и} нищ при\'{и}де ко мне, и презр\'{е}х ег\'{о}; ил\'{и} бр\'{а}та моег\'{о} опеч\'{а}лих, ил\'{и} св\'{а}дих, ил\'{и} ког\'{о} осуд\'{и}х; ил\'{и} развелич\'{а}хся, ил\'{и} разгорд\'{е}хся, ил\'{и} разгн\'{е}вахся; ил\'{и} сто\'{я}щу ми на мол\'{и}тве, ум мой о лук\'{а}вствии м\'{и}ра сег\'{о} подв\'{и}жеся, ил\'{и} развращ\'{е}ние пом\'{ы}слих; ил\'{и} объяд\'{о}хся, ил\'{и} оп\'{и}хся, ил\'{и} без ум\'{а} сме\'{я}хся; ил\'{и} лук\'{а}вое пом\'{ы}слих, ил\'{и} добр\'{о}ту чужд\'{у}ю в\'{и}дев, и т\'{о}ю уязвлен бых с\'{е}рдцем; ил\'{и} непод\'{о}бная глаг\'{о}лах, ил\'{и} грех\'{у} бр\'{а}та моег\'{о} посме\'{я}хся, м\'{о}я же суть безч\'{и}сленная согреш\'{е}ния; ил\'{и} о мол\'{и}тве не рад\'{и}х, ил\'{и} \'{и}но чт\'{о} сод\'{е}ях лук\'{а}вое, не п\'{о}мню, та бо вс\'{я} и б\'{о}льша сих сод\'{е}ях. Пом\'{и}луй мя, Тв\'{о}рче мой Влад\'{ы}ко, ун\'{ы}лаго и недост\'{о}йнаго раб\'{а} Твоег\'{о}, и ост\'{а}ви ми, и отпуст\'{и}, и прост\'{и} мя, \'{я}ко Благ и Человекол\'{ю}бец, да с м\'{и}ром л\'{я}гу, усн\'{у} и поч\'{и}ю, бл\'{у}дный, гр\'{е}шный и ока\'{я}нный аз, и поклон\'{ю}ся, и воспо\'{ю}, и просл\'{а}влю пречестн\'{о}е \'{и}мя Тво\'{е}, со Отц\'{е}м, и Единор\'{о}дным Ег\'{о} С\'{ы}ном, н\'{ы}не и пр\'{и}сно и во в\'{е}ки. Ам\'{и}нь.

\mysubtitle{Молитва 4-я, святого Макария Великого}

Чт\'{о} Ти принес\'{у}, ил\'{и} чт\'{о} Ти возд\'{а}м, великодаров\'{и}тый Безсм\'{е}ртный Цар\'{ю}, щ\'{е}дре и Человекол\'{ю}бче Г\'{о}споди, \'{я}ко лен\'{я}щася мен\'{е} на Тво\'{е} угожд\'{е}ние, и ничт\'{о}же бл\'{а}го сотв\'{о}рша, прив\'{е}л ес\'{и} на кон\'{е}ц мимош\'{е}дшаго дне сег\'{о}, обращ\'{е}ние и спас\'{е}ние душ\'{и} мо\'{е}й стр\'{о}я? М\'{и}лостив ми б\'{у}ди гр\'{е}шному и обнаж\'{е}нному вс\'{я}каго д\'{е}ла бл\'{а}га, возст\'{а}ви п\'{а}дшую мо\'{ю} д\'{у}шу, оскверн\'{и}вшуюся в безм\'{е}рных согреш\'{е}ниих, и отым\'{и} от мен\'{е} весь п\'{о}мысл лук\'{а}вый в\'{и}димаго сег\'{о} жити\'{я}. Прост\'{и} м\'{о}я согреш\'{е}ния, ед\'{и}не Безгр\'{е}шне, \'{я}же Ти согреш\'{и}х в сей день, в\'{е}дением и нев\'{е}дением, сл\'{о}вом, и д\'{е}лом, и помышл\'{е}нием, и вс\'{е}ми мо\'{и}ми ч\'{у}вствы. Ты Сам, покрыв\'{а}я, сохр\'{а}ни мя от вс\'{я}каго сопрот\'{и}внаго обсто\'{я}ния Бож\'{е}ственною Тво\'{е}ю вл\'{а}стию, и неизреч\'{е}нным человекол\'{ю}бием, и с\'{и}лою. Оч\'{и}сти, Б\'{о}же, оч\'{и}сти мн\'{о}жество грех\'{о}в мо\'{и}х. Благовол\'{и}, Г\'{о}споди, изб\'{а}вити мя от с\'{е}ти лук\'{а}ваго, и спас\'{и} стр\'{а}стную мо\'{ю} д\'{у}шу, и осен\'{и} мя св\'{е}том лиц\'{а} Твоег\'{о}, егд\'{а} при\'{и}деши во сл\'{а}ве, и неосужд\'{е}нна н\'{ы}не сном усн\'{у}ти сотвор\'{и}, и без мечт\'{а}ния, и несмущ\'{е}н п\'{о}мысл раб\'{а} Твоег\'{о} соблюд\'{и}, и всю сатанин\'{у} д\'{е}тель отжен\'{и} от мен\'{е}, и просвет\'{и} ми раз\'{у}мныя \'{о}чи серд\'{е}чныя, да не усн\'{у} в смерть. И посл\'{и} ми \'{А}нгела м\'{и}рна, хран\'{и}теля и наст\'{а}вника д\'{у}ши и т\'{е}лу моем\'{у}, да изб\'{а}вит мя от враг мо\'{и}х; да вост\'{а}в со одр\'{а} моег\'{о}, принес\'{у} Ти благод\'{а}рственныя мольб\'{ы}. Ей, Г\'{о}споди, усл\'{ы}ши мя гр\'{е}шнаго и уб\'{о}гаго раб\'{а} Твоег\'{о}, извол\'{е}нием и с\'{о}вестию; д\'{а}руй ми вост\'{а}вшу словес\'{е}м Тво\'{и}м поуч\'{и}тися, и ун\'{ы}ние бес\'{о}вское дал\'{е}че от мен\'{е} отгн\'{а}но б\'{ы}ти сотвор\'{и} Тво\'{и}ми \'{А}нгелы; да благословл\'{ю} \'{и}мя Тво\'{е} свят\'{о}е, и просл\'{а}влю, и сл\'{а}влю Преч\'{и}стую Богор\'{о}дицу Мар\'{и}ю, \'{Ю}же дал ес\'{и} нам гр\'{е}шным заступл\'{е}ние, и приим\'{и} Си\'{ю} мол\'{я}щуюся за ны; вем бо, \'{я}ко подраж\'{а}ет Тво\'{е} человекол\'{ю}бие, и мол\'{я}щися не преста\'{е}т. То\'{я} заступл\'{е}нием, и Честн\'{а}го Крест\'{а} зн\'{а}мением, и всех свят\'{ы}х Тво\'{и}х р\'{а}ди, уб\'{о}гую д\'{у}шу мо\'{ю} соблюд\'{и}, Иис\'{у}се Христ\'{е} Б\'{о}же наш, \'{я}ко Свят ес\'{и}, и препросл\'{а}влен во в\'{е}ки. Ам\'{и}нь.

\mysubtitle{Молитва 5-я}

Г\'{о}споди Б\'{о}же наш, \'{е}же согреш\'{и}х во дни сем сл\'{о}вом, д\'{е}лом и помышл\'{е}нием, \'{я}ко Благ и Человекол\'{ю}бец прост\'{и} ми. М\'{и}рен сон и безмят\'{е}жен д\'{а}руй ми. \'{А}нгела Твоег\'{о} хран\'{и}теля посл\'{и}, покрыв\'{а}юща и соблюд\'{а}юща мя от вс\'{я}каго зла, \'{я}ко Ты ес\'{и} хран\'{и}тель душ\'{а}м и телес\'{е}м н\'{а}шим, и Теб\'{е} сл\'{а}ву возсыл\'{а}ем, Отц\'{у} и С\'{ы}ну и Свят\'{о}му Д\'{у}ху, н\'{ы}не и пр\'{и}сно и во в\'{е}ки век\'{о}в. Ам\'{и}нь.

\mysubtitle{Молитва 6-я}

Г\'{о}споди Б\'{о}же наш, в Нег\'{о}же в\'{е}ровахом, и Ег\'{о}же \'{и}мя п\'{а}че вс\'{я}каго \'{и}мен\'{е} призыв\'{а}ем, д\'{а}ждь нам, ко сну отход\'{я}щим, осл\'{а}бу душ\'{и} и т\'{е}лу, и соблюд\'{и} нас от вс\'{я}каго мечт\'{а}ния, и т\'{е}мныя сл\'{а}сти кром\'{е}; уст\'{а}ви стремл\'{е}ние страст\'{е}й, угас\'{и} разжж\'{е}ния вост\'{а}ния тел\'{е}снаго. Даждь нам целом\'{у}дренне пож\'{и}ти д\'{е}лы и словес\'{ы}; да доброд\'{е}тельное ж\'{и}тельство воспри\'{е}млюще, обетов\'{а}нных не отпад\'{е}м благ\'{и}х Тво\'{и}х, \'{я}ко благослов\'{е}н ес\'{и} во в\'{е}ки. Ам\'{и}нь.

\mysubtitle{Молитва 7-я, святого Иоанна Златоуста (24 молитвы, по числу часов дня и ночи)}

Г\'{о}споди, не лиш\'{и} мен\'{е} неб\'{е}сных Тво\'{и}х благ.
Г\'{о}споди, изб\'{а}ви мя в\'{е}чных мук.
Г\'{о}споди, ум\'{о}м ли ил\'{и} помышл\'{е}нием, сл\'{о}вом ил\'{и} д\'{е}лом согреш\'{и}х, прост\'{и} мя.
Г\'{о}споди, изб\'{а}ви мя вс\'{я}каго нев\'{е}дения и забв\'{е}ния, и малод\'{у}шия, и окамен\'{е}ннаго неч\'{у}вствия.
Г\'{о}споди, изб\'{а}ви мя от вс\'{я}каго искуш\'{е}ния.
Г\'{о}споди, просвет\'{и} мо\'{е} с\'{е}рдце, \'{е}же помрач\'{и} лук\'{а}вое похот\'{е}ние.
Г\'{о}споди, аз \'{я}ко челов\'{е}к согреш\'{и}х, Ты же \'{я}ко Бог щедр, пом\'{и}луй мя, в\'{и}дя н\'{е}мощь д\'{у}ши мое\'{я}.
Г\'{о}споди, посл\'{и} благод\'{а}ть Тво\'{ю} в п\'{о}мощь мне, да просл\'{а}влю \'{и}мя Тво\'{е} свят\'{о}е.
Г\'{о}споди Иис\'{у}се Христ\'{е}, напиш\'{и} мя раб\'{а} Твоег\'{о} в кн\'{и}зе жив\'{о}тней и д\'{а}руй ми кон\'{е}ц благ\'{и}й.
Г\'{о}споди, Б\'{о}же мой, \'{а}ще и ничт\'{о}же бл\'{а}го сотвор\'{и}х пред Тоб\'{о}ю, но д\'{а}ждь ми по благод\'{а}ти Тво\'{е}й полож\'{и}ти нач\'{а}ло благ\'{о}е.
Г\'{о}споди, окроп\'{и} в с\'{е}рдце мо\'{е}м р\'{о}су благод\'{а}ти Твое\'{я}.
Г\'{о}споди небес\'{е} и земл\'{и}, помян\'{и} мя гр\'{е}шнаго раб\'{а} Твоег\'{о}, ст\'{у}днаго и неч\'{и}стаго, во Ц\'{а}рствии Тво\'{е}м. Ам\'{и}нь.

Г\'{о}споди, в пока\'{я}нии приим\'{и} мя.
Г\'{о}споди, не ост\'{а}ви мен\'{е}.
Г\'{о}споди, не введ\'{и} мен\'{е} в нап\'{а}сть.
Г\'{о}споди, д\'{а}ждь ми мысль бл\'{а}гу.
Г\'{о}споди, д\'{а}ждь ми сл\'{е}зы и п\'{а}мять см\'{е}ртную, и умил\'{е}ние.
Г\'{о}споди, д\'{а}ждь ми п\'{о}мысл испов\'{е}дания грех\'{о}в мо\'{и}х.
Г\'{о}споди, д\'{а}ждь ми смир\'{е}ние, целом\'{у}дрие и послуш\'{а}ние.
Г\'{о}споди, д\'{а}ждь ми терп\'{е}ние, великод\'{у}шие и кр\'{о}тость.
Г\'{о}споди, всел\'{и} в мя к\'{о}рень благ\'{и}х, страх Твой в с\'{е}рдце мо\'{е}.
Г\'{о}споди, спод\'{о}би мя люб\'{и}ти Тя от все\'{я} душ\'{и} мое\'{я} и помышл\'{е}ния и твор\'{и}ти во всем в\'{о}лю Тво\'{ю}.
Г\'{о}споди, покр\'{ы}й мя от челов\'{е}к н\'{е}которых, и бес\'{о}в, и страст\'{е}й, и от вс\'{я}кия ин\'{ы}я непод\'{о}бныя в\'{е}щи.
Г\'{о}споди, в\'{е}си, \'{я}ко твор\'{и}ши, \'{я}коже Ты в\'{о}лиши, да б\'{у}дет в\'{о}ля Тво\'{я} и во мне гр\'{е}шнем, \'{я}ко благослов\'{е}н ес\'{и} во в\'{е}ки. Ам\'{и}нь.

\mysubtitle{Молитва 8-я, ко Господу нашему Иисусу Христу}

Г\'{о}споди Иис\'{у}се Христ\'{е}, С\'{ы}не Б\'{о}жий, р\'{а}ди честн\'{е}йшия М\'{а}тере Твое\'{я}, и безпл\'{о}тных Тво\'{и}х \'{А}нгел, Прор\'{о}ка же и Предт\'{е}чи и Крест\'{и}теля Твоег\'{о}, богоглаг\'{о}ливых же ап\'{о}стол, св\'{е}тлых и добропоб\'{е}дных м\'{у}ченик, препод\'{о}бных и богон\'{о}сных от\'{е}ц, и всех свят\'{ы}х мол\'{и}твами, изб\'{а}ви мя насто\'{я}щаго обсто\'{я}ния бес\'{о}вскаго. Ей, Г\'{о}споди мой и Тв\'{о}рче, не хот\'{я}й см\'{е}рти гр\'{е}шнаго, но \'{я}коже обрат\'{и}тися и ж\'{и}ву б\'{ы}ти ем\'{у}, д\'{а}ждь и мне обращ\'{е}ние ока\'{я}нному и недост\'{о}йному; изм\'{и} мя от уст п\'{а}губнаго зм\'{и}я, зи\'{я}ющаго пожр\'{е}ти мя и свест\'{и} во ад ж\'{и}ва. Ей, Г\'{о}споди мой, утеш\'{е}ние мо\'{е}, \'{И}же мен\'{е} р\'{а}ди ока\'{я}ннаго в тл\'{е}нную плоть оболк\'{и}йся, ист\'{о}ргни мя от ока\'{я}нства, и утеш\'{е}ние под\'{а}ждь душ\'{и} мо\'{е}й ока\'{я}нней. Всад\'{и} в с\'{е}рдце мо\'{е} твор\'{и}ти Тво\'{я} повел\'{е}ния, и ост\'{а}вити лук\'{а}вая де\'{я}ния, и получ\'{и}ти блаж\'{е}нства Тво\'{я}: на Тя бо, Г\'{о}споди, упов\'{а}х, спас\'{и} мя.

\mysubtitle{Молитва 9-я, ко Пресвятой Богородице, Петра Студийского}

К Теб\'{е} Преч\'{и}стей Б\'{о}жией М\'{а}тери аз ока\'{я}нный прип\'{а}дая мол\'{ю}ся: в\'{е}си, Цар\'{и}це, \'{я}ко безпрест\'{а}ни согреш\'{а}ю и прогневл\'{я}ю С\'{ы}на Твоег\'{о} и Б\'{о}га моег\'{о}, и мн\'{о}гажды \'{а}ще к\'{а}юся, лож пред Б\'{о}гом обрет\'{а}юся, и к\'{а}юся треп\'{е}ща: неуж\'{е}ли Госп\'{о}дь пораз\'{и}т мя, и по час\'{е} п\'{а}ки т\'{а}яжде твор\'{ю}; в\'{е}дущи си\'{я}, Влад\'{ы}чице м\'{о}я Госпож\'{е} Богор\'{о}дице, мол\'{ю}, да пом\'{и}луеши, да укреп\'{и}ши, и благ\'{а}я твор\'{и}ти да под\'{а}си ми. В\'{е}си бо, Влад\'{ы}чице м\'{о}я Богор\'{о}дице, \'{я}ко отн\'{ю}д \'{и}мам в н\'{е}нависти зл\'{а}я м\'{о}я д\'{е}ла, и вс\'{е}ю м\'{ы}слию любл\'{ю} зак\'{о}н Б\'{о}га моег\'{о}; но не вем, Госпож\'{е} Преч\'{и}стая, отк\'{у}ду \'{я}же ненав\'{и}жду, та и любл\'{ю}, а благ\'{а}я преступ\'{а}ю. Не попущ\'{а}й, Преч\'{и}стая, в\'{о}ли мо\'{е}й соверш\'{а}тися, не уг\'{о}дна бо есть, но да б\'{у}дет в\'{о}ля С\'{ы}на Твоег\'{о} и Б\'{о}га моег\'{о}: да мя спас\'{е}т, и вразум\'{и}т, и под\'{а}ст благод\'{а}ть Свят\'{а}го Д\'{у}ха, да бых аз отс\'{е}ле прест\'{а}л сквернод\'{е}йства, и пр\'{о}чее пож\'{и}л бых в повел\'{е}нии С\'{ы}на Твоег\'{о}, Ем\'{у}же подоб\'{а}ет вс\'{я}кая сл\'{а}ва, честь и держ\'{а}ва, со Безнач\'{а}льным Ег\'{о} Отц\'{е}м, и Пресвят\'{ы}м и Благ\'{и}м и Животвор\'{я}щим Ег\'{о} Д\'{у}хом, н\'{ы}не и пр\'{и}сно, и во в\'{е}ки век\'{о}в. Aминь.

\mysubtitle{Молитва 10-я, ко Пресвятой Богородице}

Благ\'{а}го Цар\'{я} благ\'{а}я М\'{а}ти, Преч\'{и}стая и Благослов\'{е}нная Богор\'{о}дице Мар\'{и}е, м\'{и}лость С\'{ы}на Твоег\'{о} и Б\'{о}га н\'{а}шего изл\'{е}й на стр\'{а}стную мо\'{ю} д\'{у}шу и Тво\'{и}ми мол\'{и}твами наст\'{а}ви мя на де\'{я}ния благ\'{а}я, да пр\'{о}чее вр\'{е}мя живот\'{а} моег\'{о} без пор\'{о}ка прейд\'{у} и Тоб\'{о}ю рай да обр\'{я}щу, Богор\'{о}дице Д\'{е}во, ед\'{и}на Ч\'{и}стая и Благослов\'{е}нная.

\mysubtitle{Молитва 11-я, ко святому Ангелу хранителю}

\'{А}нгеле Христ\'{о}в, хран\'{и}телю мой свят\'{ы}й и покров\'{и}телю душ\'{и} и т\'{е}ла моег\'{о}, вс\'{я} ми прост\'{и}, ел\'{и}ка согреш\'{и}х во дн\'{е}шний день, и от вс\'{я}каго лук\'{а}вствия прот\'{и}внаго ми враг\'{а} изб\'{а}ви мя, да ни в к\'{о}емже грес\'{е} прогн\'{е}ваю Б\'{о}га моег\'{о}; но мол\'{и} за мя гр\'{е}шнаго и недост\'{о}йнаго раб\'{а}, \'{я}ко да дост\'{о}йна мя пок\'{а}жеши бл\'{а}гости и м\'{и}лости Всесвят\'{ы}я Тр\'{о}ицы и М\'{а}тере Г\'{о}спода моег\'{о} Иис\'{у}са Христ\'{а} и всех свят\'{ы}х. Ам\'{и}нь.

\mysubtitle{Кондак Богородице}

Взбр\'{а}нной Воев\'{о}де побед\'{и}тельная, \'{я}ко изб\'{а}вльшеся от злых, благод\'{а}рственная воспис\'{у}ем Ти раб\'{и} Тво\'{и}, Богор\'{о}дице, но \'{я}ко им\'{у}щая держ\'{а}ву непобед\'{и}мую, от вс\'{я}ких нас бед свобод\'{и}, да зов\'{е}м Ти; р\'{а}дуйся, Нев\'{е}сто Ненев\'{е}стная.

Пресл\'{а}вная Приснод\'{е}во, М\'{а}ти Христ\'{а} Б\'{о}га, принес\'{и} н\'{а}шу мол\'{и}тву С\'{ы}ну Твоем\'{у} и Б\'{о}гу н\'{а}шему, да спас\'{е}т Тоб\'{о}ю д\'{у}ши н\'{а}ша.

Все упов\'{а}ние мо\'{е} на Тя возлаг\'{а}ю, М\'{а}ти Б\'{о}жия, сохр\'{а}ни мя под кр\'{о}вом Тво\'{и}м.

Богор\'{о}дице Д\'{е}во, не пр\'{е}зри мен\'{е}, гр\'{е}шнаго, тр\'{е}бующа Твое\'{я} п\'{о}мощи и Твоег\'{о} заступл\'{е}ния, на Тя бо упов\'{а} душ\'{а} м\'{о}я, и пом\'{и}луй мя.

\mysubtitle{Молитва святого Иоанникия}

Упов\'{а}ние мо\'{е} От\'{е}ц, приб\'{е}жище мо\'{е} Сын, покр\'{о}в мой Дух Свят\'{ы}й: Тр\'{о}ице Свят\'{а}я, сл\'{а}ва Теб\'{е}.

\Chestneyshuyu

Сл\'{а}ва Отц\'{у} и С\'{ы}ну и Свят\'{о}му Д\'{у}ху, и н\'{ы}не и пр\'{и}сно и во в\'{е}ки век\'{о}в. Ам\'{и}нь.

Г\'{о}споди, пом\'{и}луй. \myemph{(Трижды)}

Г\'{о}споди Иис\'{у}се Христ\'{е}, С\'{ы}не Б\'{о}жий, молитв р\'{а}ди Преч\'{и}стыя Твое\'{я} М\'{а}тере, препод\'{о}бных и богон\'{о}сных от\'{е}ц н\'{а}ших и всех свят\'{ы}х, пом\'{и}луй нас. Ам\'{и}нь.

\mysubtitle{Молитва святого Иоанна Дамаскина}

Влад\'{ы}ко Человекол\'{ю}бче, неуж\'{е}ли мне одр сей гроб б\'{у}дет, ил\'{и} ещ\'{е} ока\'{я}нную мо\'{ю} д\'{у}шу просвет\'{и}ши днем? Се ми гроб предлеж\'{и}т, се ми смерть предсто\'{и}т. Суд\'{а} Твоег\'{о}, Г\'{о}споди, бо\'{ю}ся и м\'{у}ки безкон\'{е}чныя, зл\'{о}е же твор\'{я} не преста\'{ю}: Теб\'{е} Г\'{о}спода Б\'{о}га моег\'{о} всегд\'{а} прогневл\'{я}ю, и Преч\'{и}стую Тво\'{ю} М\'{а}терь, и вс\'{я} Неб\'{е}сныя с\'{и}лы, и свят\'{а}го \'{А}нгела хран\'{и}теля моег\'{о}. Вем \'{у}бо, Г\'{о}споди, \'{я}ко недост\'{о}ин есмь человекол\'{ю}бия Твоег\'{о}, но дост\'{о}ин есмь вс\'{я}каго осужд\'{е}ния и м\'{у}ки. Но, Г\'{о}споди, ил\'{и} хощ\'{у}, ил\'{и} не хощ\'{у}, спас\'{и} мя. \'{А}ще бо пр\'{а}ведника спас\'{е}ши, ничт\'{о}же в\'{е}лие; и \'{а}ще ч\'{и}стаго пом\'{и}луеши, ничт\'{о}же д\'{и}вно: дост\'{о}йни бо суть м\'{и}лости Твое\'{я}. Но на мне гр\'{е}шнем удив\'{и} м\'{и}лость Тво\'{ю}: о сем яв\'{и} человекол\'{ю}бие Тво\'{е}, да не одол\'{е}ет м\'{о}я зл\'{о}ба Тво\'{е}й неизглаг\'{о}ланней бл\'{а}гости и милос\'{е}рдию: и \'{я}коже х\'{о}щеши, устр\'{о}й о мне вещь.

Просвет\'{и} \'{о}чи мо\'{и}, Христ\'{е} Б\'{о}же, да не когд\'{а} усн\'{у} в смерть, да не когд\'{а} реч\'{е}т враг мой: укреп\'{и}хся на нег\'{о}.

\slavan

Заст\'{у}пник душ\'{и} мое\'{я} б\'{у}ди, Б\'{о}же, \'{я}ко посред\'{е} хожд\'{у} сет\'{е}й мн\'{о}гих; изб\'{а}ви мя от них и спас\'{и} мя, Бл\'{а}же, \'{я}ко Человекол\'{ю}бец.

\inynen

Пресл\'{а}вную Б\'{о}жию М\'{а}терь, и свят\'{ы}х \'{А}нгел Свят\'{е}йшую, нем\'{о}лчно воспо\'{и}м с\'{е}рдцем и уст\'{ы}, Богор\'{о}дицу си\'{ю} испов\'{е}дающе, \'{я}ко во\'{и}стинну р\'{о}ждшую нам Б\'{о}га воплощ\'{е}нна, и мол\'{я}щуюся непрест\'{а}нно о душ\'{а}х н\'{а}ших.

\mysubtitle{Знаменуй себя крестом и говори молитву Честн\'{о}му Кресту:}

Да воскр\'{е}снет Бог, и расточ\'{а}тся враз\'{и} Ег\'{о}, и да беж\'{а}т от лиц\'{а} Ег\'{о} ненав\'{и}дящии Ег\'{о}. \'{Я}ко исчез\'{а}ет дым, да исч\'{е}знут; \'{я}ко т\'{а}ет воск от лиц\'{а} огн\'{я}, т\'{а}ко да пог\'{и}бнут б\'{е}си от лиц\'{а} л\'{ю}бящих Б\'{о}га и зн\'{а}менующихся кр\'{е}стным зн\'{а}мением, и в вес\'{е}лии глаг\'{о}лющих: р\'{а}дуйся, Пречестн\'{ы}й и Животвор\'{я}щий Кр\'{е}сте Госп\'{о}день, прогон\'{я}яй б\'{е}сы с\'{и}лою на теб\'{е} проп\'{я}таго Г\'{о}спода н\'{а}шего Иис\'{у}са Христ\'{а}, во ад сш\'{е}дшаго и попр\'{а}вшего с\'{и}лу ди\'{а}волю, и даров\'{а}вшаго нам теб\'{е} Крест Свой Честн\'{ы}й на прогн\'{а}ние вс\'{я}каго супост\'{а}та. О, Пречестн\'{ы}й и Животвор\'{я}щий Кр\'{е}сте Госп\'{о}день! Помог\'{а}й ми со Свят\'{о}ю Госпож\'{е}ю Д\'{е}вою Богор\'{о}дицею и со вс\'{е}ми свят\'{ы}ми во в\'{е}ки. Ам\'{и}нь.

\myemph{Или кратко:}

Оград\'{и} мя, Г\'{о}споди, с\'{и}лою Честн\'{а}го и Животвор\'{я}щаго Твоег\'{о} Крест\'{а}, и сохр\'{а}ни мя от вс\'{я}каго зла.

\mysubtitle{Молитва}

Осл\'{а}би, ост\'{а}ви, прост\'{и}, Б\'{о}же, прегреш\'{е}ния н\'{а}ша, в\'{о}льная и нев\'{о}льная, \'{я}же в сл\'{о}ве и в д\'{е}ле, \'{я}же в в\'{е}дении и не в в\'{е}дении, \'{я}же во дни и в нощ\'{и}, \'{я}же во ум\'{е} и в помышл\'{е}нии: вс\'{я} нам прост\'{и}, \'{я}ко Благ и Человекол\'{ю}бец.

\mysubtitle{Молитва}

Ненав\'{и}дящих и об\'{и}дящих нас прост\'{и}, Г\'{о}споди Человекол\'{ю}бче. Благотвор\'{я}щим благосотвор\'{и}. Бр\'{а}тиям и ср\'{о}дником н\'{а}шим д\'{а}руй \'{я}же ко спас\'{е}нию прош\'{е}ния и жизнь в\'{е}чную. В н\'{е}мощех с\'{у}щия посет\'{и} и исцел\'{е}ние д\'{а}руй. \'{И}же на м\'{о}ри упр\'{а}ви. Путеш\'{е}ствующим спутеш\'{е}ствуй. Правосл\'{а}вным христи\'{а}ном споб\'{о}рствуй. Служ\'{а}щим и м\'{и}лующим нас грех\'{о}в оставл\'{е}ние д\'{а}руй. Запов\'{е}давших нам недост\'{о}йным мол\'{и}тися о них пом\'{и}луй по вел\'{и}цей Тво\'{е}й м\'{и}лости. Помян\'{и}, Г\'{о}споди, пр\'{е}жде ус\'{о}пших от\'{е}ц и бр\'{а}тий н\'{а}ших и упок\'{о}й их, ид\'{е}же присещ\'{а}ет свет лиц\'{а} Твоег\'{о}. Помян\'{и}, Г\'{о}споди, бр\'{а}тий н\'{а}ших плен\'{е}нных и изб\'{а}ви я от вс\'{я}каго обсто\'{я}ния. Помян\'{и}, Г\'{о}споди, плодонос\'{я}щих и доброд\'{е}лающих во свят\'{ы}х Тво\'{и}х ц\'{е}рквах, и д\'{а}ждь им \'{я}же ко спас\'{е}нию прош\'{е}ния и жизнь в\'{е}чную. Помян\'{и}, Г\'{о}споди, и нас, смир\'{е}нных и гр\'{е}шных и недост\'{о}йных раб Тво\'{и}х, и просвет\'{и} наш ум св\'{е}том р\'{а}зума Твоег\'{о}, и наст\'{а}ви нас на стез\'{ю} з\'{а}поведей Тво\'{и}х, мол\'{и}твами Преч\'{и}стыя Влад\'{ы}чицы н\'{а}шея Богор\'{о}дицы и Приснод\'{е}вы Мар\'{и}и и всех Тво\'{и}х свят\'{ы}х: \'{я}ко благослов\'{е}н ес\'{и} во в\'{е}ки век\'{о}в. Ам\'{и}нь.

\mysubtitle{Исповедание грехов повседневное}

Испов\'{е}даю Теб\'{е} Г\'{о}споду Б\'{о}гу моем\'{у} и Творц\'{у}, во Свят\'{е}й Тр\'{о}ице Ед\'{и}ному, сл\'{а}вимому и поклан\'{я}емому, Отц\'{у} и С\'{ы}ну и Свят\'{о}му Д\'{у}ху, вс\'{я} м\'{о}я грех\'{и}, \'{я}же сод\'{е}ях во вс\'{я} дни живот\'{а} моег\'{о}, и на вс\'{я}кий час, и в насто\'{я}щее вр\'{е}мя, и в преш\'{е}дшия дни и н\'{о}щи, д\'{е}лом, сл\'{о}вом, помышл\'{е}нием, объяд\'{е}нием, пи\'{я}нством, тайнояд\'{е}нием, праздносл\'{о}вием, ун\'{ы}нием, л\'{е}ностию, прекосл\'{о}вием, непослуш\'{а}нием, оклевет\'{а}нием, осужд\'{е}нием, небреж\'{е}нием, самол\'{ю}бием, многостяж\'{а}нием, хищ\'{е}нием, неправдоглаг\'{о}ланием, скверноприб\'{ы}тчеством, мшело\'{и}мством, ревнов\'{а}нием, з\'{а}вистию, гн\'{е}вом, памятозл\'{о}бием, н\'{е}навистию, лихо\'{и}мством и вс\'{е}ми мо\'{и}ми ч\'{у}вствы: зр\'{е}нием, сл\'{у}хом, обон\'{я}нием, вк\'{у}сом, осяз\'{а}нием и пр\'{о}чими мо\'{и}ми грех\'{и}, душ\'{е}вными вк\'{у}пе и тел\'{е}сными, \'{и}миже Теб\'{е} Б\'{о}га моег\'{о} и Творц\'{а} прогн\'{е}вах, и бл\'{и}жняго моег\'{о} онепр\'{а}вдовах: о сих жал\'{е}я, в\'{и}нна себ\'{е} Теб\'{е} Б\'{о}гу моем\'{у} представл\'{я}ю, и им\'{е}ю в\'{о}лю к\'{а}ятися: т\'{о}чию, Г\'{о}споди Б\'{о}же мой, помоз\'{и} ми, со слез\'{а}ми смир\'{е}нно мол\'{ю} Тя: преш\'{е}дшая же согреш\'{е}ния м\'{о}я милос\'{е}рдием Тво\'{и}м прост\'{и} ми, и разреш\'{и} от всех сих, \'{я}же изглаг\'{о}лах пред Тоб\'{о}ю, \'{я}ко Благ и Человекол\'{ю}бец.

\mysubtitle{ \myemph{Когда отходишь ко сну, произноси:}}

В р\'{у}це Тво\'{и}, Г\'{о}споди Иис\'{у}се Христ\'{е}, Б\'{о}же мой, преда\'{ю} дух мой: Ты же мя благослов\'{и}, Ты мя пом\'{и}луй и жив\'{о}т в\'{е}чный д\'{а}руй ми. Ам\'{и}нь.

\end{mymulticols}

\mychapterending


\mychapter{Канон покаянный ко Господу нашему Иисусу Христу}\begin{mymulticols}
%http://www.molitvoslov.org/text3.htm 

\myfigure{748}

\mysubtitle{Глас 6-й, Песнь 1}

\irmos{Яко по суху пешешествовав Израиль, по бездне стопами, гонителя фараона видя потопляема, Богу победную песнь поим, вопияше.}

\pripev{Помилуй мя, Боже, помилуй мя.}

Ныне приступих аз грешный и обремененный к Тебе, Владыце и Богу моему; не смею же взирати на небо, токмо молюся, глаголя: даждь ми, Господи, ум, да плачуся дел моих горько.

\pripev{Помилуй мя, Боже, помилуй мя.}

О, горе мне грешному! Паче всех человек окаянен есмь, покаяния несть во мне; даждь ми, Господи, слезы, да плачуся дел моих горько.

\slava

Безумне, окаянне человече, в лености время губиши; помысли житие твое, и обратися ко Господу Богу, и плачися о делех твоих горько.

\inyne

Мати Божия Пречистая, воззри на мя грешного, и от сети диаволи избави мя, и на путь покаяния настави мя, да плачуся дел моих горько.

\mysubtitle{Песнь 3}

\irmos{Несть свят, якоже Ты, Господи Боже мой, вознесый рог верных Твоих, Блаже, и утвердивый нас на камени исповедания Твоего.}

\pripev{Помилуй мя, Боже, помилуй мя.}

Внегда поставлени будут престоли на судищи страшнем, тогда всех человек дела обличатся; горе тамо будет грешным, в муку отсылаемым; и то ведущи, душе моя, покайся от злых дел твоих.

\pripev{Помилуй мя, Боже, помилуй мя.}

Праведницы возрадуются, а грешнии восплачутся, тогда никтоже возможет помощи нам, но дела наша осудят нас, темже прежде конца покайся от злых дел твоих.

\slava

Увы мне великогрешному, иже делы и мысльми осквернився, ни капли слез имею от жестосердия; ныне возникни от земли, душе моя, и покайся от злых дел твоих.

\inyne

Се, взывает, Госпоже, Сын Твой, и поучает нас на доброе, аз же грешный добра всегда бегаю; но Ты, Милостивая, помилуй мя, да покаюся от злых моих дел.

\mysubtitle{Седален, глас 6-й}

Помышляю день страшный и плачуся деяний моих лукавых: како отвещаю Безсмертному Царю, или коим дерзновением воззрю на Судию, блудный аз? Благоутробный Отче, Сыне Единородный и Душе Святый, помилуй мя.

Слава Отцу и Сыну и Святому Духу. И ныне и присно и во веки веков. Аминь.

\Bogorodichen{Связан многими ныне пленицами грехов и содержим лютыми страстьми и бедами, к Тебе прибегаю, моему спасению, и вопию: помози ми, Дево, Мати Божия.}

\mysubtitle{Песнь 4}

\irmos{Христос моя сила, Бог и Господь, честная Церковь боголепно поет, взывающи от смысла чиста, о Господе празднующи.}

\pripev{Помилуй мя, Боже, помилуй мя.}

Широк путь зде и угодный сласти творити, но горько будет в последний день, егда душа от тела разлучатися будет: блюдися от сих, человече, Царствия ради Божия.

\pripev{Помилуй мя, Боже, помилуй мя.}

Почто убогаго обидиши, мзду наемничу удержуеши, брата твоего не любиши, блуд и гордость гониши? Остави убо сия, душе моя, и покайся Царствия ради Божия.

\slava

О, безумный человече, доколе углебаеши, яко пчела, собирающи богатство твое? Вскоре бо погибнет, яко прах и пепел: но более взыщи Царствия Божия.

\inyne

Госпоже Богородице, помилуй мя грешного, и в добродетели укрепи, и соблюди мя, да наглая смерть не похитит мя неготоваго, и доведи мя, Дево, Царствия Божия.

\mysubtitle{Песнь 5}

\irmos{Божиим светом Твоим, Блаже, утренюющих Ти души любовию озари, молюся, Тя ведети, Слове Божий, истиннаго Бога, от мрака греховнаго взывающа.}

\pripev{Помилуй мя, Боже, помилуй мя.}

Воспомяни, окаянный человече, како лжам, клеветам, разбою, немощем, лютым зверем, грехов ради порабощен еси; душе моя грешная, того ли восхотела еси?

\pripev{Помилуй мя, Боже, помилуй мя.}

Трепещут ми уди, всеми бо сотворих вину: очима взираяй, ушима слышай, языком злая глаголяй, всего себе геенне предаяй; душе моя грешная, сего ли восхотела еси?

\slava

Блудника и разбойника кающася приял еси, Спасе, аз же един леностию греховною отягчихся и злым делом поработихся, душе моя грешная, сего ли восхотела еси?

\inyne

Дивная и скорая помощнице всем человеком, Мати Божия, помози мне недостойному, душа бо моя грешная того восхоте.

\mysubtitle{Песнь 6}

\irmos{Житейское море, воздвизаемое зря напастей бурею, к тихому пристанищу Твоему притек, вопию Ти: возведи от тли живот мой, Многомилостиве.}

\pripev{Помилуй мя, Боже, помилуй мя.}

Житие на земли блудно пожих и душу во тьму предах, ныне убо молю Тя, Милостивый Владыко: свободи мя от работы сея вражия, и даждь ми разум творити волю Твою.

\pripev{Помилуй мя, Боже, помилуй мя.}

Кто творит таковая, якоже аз? Якоже бо свиния лежит в калу, тако и аз греху служу. Но Ты, Господи, исторгни мя от гнуса сего и даждь ми сердце творити заповеди Твоя.

\slava

Воспряни, окаянный человече, к Богу, воспомянув своя согрешения, припадая ко Творцу, слезя и стеня; Той же, яко милосерд, даст ти ум знати волю Свою.

\inyne

Богородице Дево, от видимаго и невидимаго зла сохрани мя, Пречистая, и приими молитвы моя, и донеси я Сыну Твоему, да даст ми ум творити волю Его.

\mysubtitle{Кондак}

Душе моя, почто грехами богатееши, почто волю диаволю твориши, в чесом надежду полагаеши? Престани от сих и обратися к Богу с плачем, зовущи: милосерде Господи, помилуй мя грешнаго.

\mysubtitle{Икос}

Помысли, душе моя, горький час смерти и страшный суд Творца твоего и Бога: Ангели бо грознии поймут тя, душе, и в вечный огнь введут: убо прежде смерти покайся, вопиющи: Господи, помилуй мя грешнаго.

\mysubtitle{Песнь 7}

\irmos{Росодательну убо пещь содела Ангел преподобным отроком, халдеи же опаляющее веление Божие, мучителя увеща вопити: благословен еси, Боже отец наших.}

\pripev{Помилуй мя, Боже, помилуй мя.}

Не надейся, душе моя, на тленное богатство и на неправедное собрание, вся бо сия не веси кому оставиши, но возопий: помилуй мя, Христе Боже, недостойнаго.

\pripev{Помилуй мя, Боже, помилуй мя.}

Не уповай, душе моя, на телесное здравие и на скоромимоходящую красоту, видиши бо, яко сильнии и младии умирают; но возопий: помилуй мя, Христе Боже, недостойнаго.

\slava

Воспомяни, душе моя, вечное житие, Царство Небесное, уготованное святым, и тьму кромешную и гнев Божий злым, и возопий: помилуй мя, Христе Боже, недостойнаго.

\inyne

Припади, душе моя, к Божией Матери и помолися Той, есть бо скорая помощница кающимся, умолит Сына Христа Бога, и помилует мя недостойнаго.

\mysubtitle{Песнь 8}

\irmos{Из пламене преподобным росу источил еси и праведнаго жертву водою попалил еси: вся бо твориши, Христе, токмо еже хотети. Тя превозносим во вся веки.}

\pripev{Помилуй мя, Боже, помилуй мя.}

Како не имам плакатися, егда помышляю смерть, видех бо во гробе лежаща брата моего, безславна и безобразна? Что убо чаю, и на что надеюся? Токмо даждь ми, Господи, прежде конца покаяние. \myemph{(Дважды)}

\slava

Верую, яко приидеши судити живых и мертвых, и вси во своем чину станут, старии и младии, владыки и князи, девы и священницы; где обрящуся аз? Сего ради вопию: даждь ми, Господи, прежде конца покаяние.

\inyne

Пречистая Богородице, приими недостойную молитву мою и сохрани мя от наглыя смерти, и даруй ми прежде конца покаяние.

\mysubtitle{Песнь 9}

\irmos{Бога человеком невозможно видети, на Негоже не смеют чини Ангельстии взирати; Тобою же, Всечистая, явися человеком Слово Воплощенно, Егоже величающе, с небесными вои Тя ублажаем.}

\pripev{Помилуй мя, Боже, помилуй мя.}

Ныне к вам прибегаю, Ангели, Архангели и вся небесныя силы, у Престола Божия стоящии, молитеся ко Творцу своему, да избавит душу мою от муки вечныя.

\pripev{Помилуй мя, Боже, помилуй мя.}

Ныне плачуся к вам, святии патриарси, царие и пророцы, апостоли и святителие и вси избраннии Христовы: помозите ми на суде, да спасет душу мою от силы вражия.

\slava

Ныне к вам воздежу руце, святии мученицы, пустынницы, девственницы, праведницы и вси святии, молящиися ко Господу за весь мир, да помилует мя в час смерти моея.

\inyne

Мати Божия, помози ми, на Тя сильне надеющемуся, умоли Сына Своего, да поставит мя недостойнаго одесную Себе, егда сядет судяй живых и мертвых, аминь.

\mysubtitle{Молитва}

Господи Иисусе Христе, Сыне Божий, помилуй мя грешнаго.

Владыко Христе Боже, Иже страстьми Своими страсти моя исцеливый и язвами Своими язвы моя уврачевавый, даруй мне, много Тебе прегрешившему, слезы умиления; сраствори моему телу от обоняния Животворящаго Тела Твоего, и наслади душу мою Твоею Честною Кровию от горести, еюже мя сопротивник напои; возвыси мой ум к Тебе, долу поникший, и возведи от пропасти погибели: яко не имам покаяния, не имам умиления, не имам слезы утешительныя, возводящия чада ко своему наследию. Омрачихся умом в житейских страстех, не могу воззрети к Тебе в болезни, не могу согретися слезами, яже к Тебе любве. Но, Владыко Господи Иисусе Христе, сокровище благих, даруй мне покаяние всецелое и сердце люботрудное во взыскание Твое, даруй мне благодать Твою и обнови во мне зраки Твоего образа. Оставих Тя, не остави мене; изыди на взыскание мое, возведи к пажити Твоей и сопричти мя овцам избраннаго Твоего стада, воспитай мя с ними от злака Божественных Твоих Таинств, молитвами Пречистыя Твоея Матере и всех святых Твоих. Аминь.

\end{mymulticols}

\mychapterending


\mychapter{Канон молебный ко Пресвятой Богородице}\begin{mymulticols}
%http://www.molitvoslov.org/text4.htm 

\myfigure{3}

\mysubtitle{Поемый во всякой скорби душевной и обстоянии.}

\mysubtitle{Tворение Феостирикта монаха.}

\mysubtitle{Тропaрь Богородице, глас 4-й}

К Богородице прилежно ныне притецем, грешнии и смиреннии, и припадем, в покаянии зовуще из глубины души: Владычице, помози, на ны милосердовавши, потщися, погибaем от множества прегрешений, не отврати Твоя рабы тщи, Тя бо и едину надежду имамы. \myemph{(Дважды)}

Слава Отцу и Сыну и Святому Духу. И ныне и присно и во веки веков. Аминь.

Не умолчим никогда, Богородице, силы Твоя глаголати, недостойнии: aще бо Ты не бы предстояла молящи, кто бы нас избaвил от толиких бед, кто же бы сохранил до ныне свободны? Не отступим, Владычице, от Тебе: Твоя бо рабы спасaеши присно от всяких лютых.

\mysubtitle{Псалом 50}

\PsalmFifty

\mysubtitle{Канон ко Пресвятой Богородице, глас 8-й}

\mysubtitle{Песнь 1}

\irmos{Воду прошед яко сушу, и египетскаго зла избежaв, изрaильтянин вопияше: избaвителю и Богу нашему поим.}

\pripev{Пресвят\'{а}я Богор\'{о}дице, спас\'{и} нас.}

Многими содержимь напaстьми, к Тебе прибегаю, спасения иский: о, Мaти Слова и Дево, от тяжких и лютых мя спаси.

\pripev{Пресвят\'{а}я Богор\'{о}дице, спас\'{и} нас.}

Страстей мя смущaют прилози, многаго уныния исполнити мою душу; умири, Отроковице, тишиною Сына и Бога Твоего, Всенепорочная.

\slava

Спaса рождшую Тя и Бога, молю, Дево, избaвитися ми лютых: к Тебе бо ныне прибегaя, простирaю и душу и помышление.

\inyne

Недугующа телом и душею, посещения Божественнаго и промышления от Тебе сподоби, едина Богомaти, яко благая, Благaго же Родительница.

\mysubtitle{Песнь 3}

\irmos{Небеснаго круга Верхотворче, Господи, и Церкве Зиждителю, Ты мене утверди в любви Твоей, желaний крaю, верных утверждение, едине Человеколюбче.}

\pripev{Пресвят\'{а}я Богор\'{о}дице, спас\'{и} нас.}

Предстaтельство и покров жизни моея полагaю Тя, Богородительнице Дево: Ты мя окорми ко пристaнищу Твоему, благих виновна; верных утверждение, едина Всепетая.

\pripev{Пресвят\'{а}я Богор\'{о}дице, спас\'{и} нас.}

Молю, Дево, душевное смущение и печали моея бурю разорити: Ты бо, Богоневестная, начальника тишины Христа родилa еси, едина Пречистая.

\slava

Благодетеля рождши добрых виновнаго, благодеяния богатство всем источи, вся бо можеши, яко сильнаго в крепости Христа рождши, Богоблаженная.

\inyne

Лютыми недуги и болезненными страстьми истязaему, Дево, Ты ми помози: исцелений бо неоскудное Тя знаю сокровище, Пренепорочная, неиждивaемое.

Спаси от бед рабы Твоя, Богородице, яко вси по Бозе к Тебе прибегaем, яко нерушимей стене и предстaтельству.

Призри благосердием, всепетая Богородице, на мое лютое телесе озлобление, и исцели души моея болезнь.

\mysubtitle{Тропарь, глас 2-й}

Моление теплое и стенa необоримая, милости источниче, мирови прибежище, прилежно вопием Ти: Богородице Владычице, предвари, и от бед избaви нас, едина вскоре предстaтельствующая.

\mysubtitle{Песнь 4}

\irmos{Услышах, Господи, смотрения Твоего тaинство, разумех дела Твоя и прослaвих Твое Божество.}

\pripev{Пресвят\'{а}я Богор\'{о}дице, спас\'{и} нас.}

Страстей моих смущение, кормчию рождшая Господа, и бурю утиши моих прегрешений, Богоневестная.

\pripev{Пресвят\'{а}я Богор\'{о}дице, спас\'{и} нас.}

Милосердия Твоего бездну призывaющу подaждь ми, яже Благосердаго рождшая и Спaса всех поющих Тя.

\pripev{Пресвят\'{а}я Богор\'{о}дице, спас\'{и} нас.}

Наслаждaющеся, Пречистая, Твоих даровaний, благодaрственное воспевaем пение, ведуще Тя Богомaтерь.

\slava

На одре болезни моея и немощи низлежaщу ми, яко Благолюбива, помози, Богородице, едина Приснодево.

\inyne

Надежду и утверждение и спасения стену недвижиму имуще Тя, Всепетая, неудобства всякаго избавляемся.

\mysubtitle{Песнь 5}

\irmos{Просвети нас повелении Твоими, Господи, и мышцею Твоею высокою Твой мир подaждь нам, Человеколюбче.}

\pripev{Пресвят\'{а}я Богор\'{о}дице, спас\'{и} нас.}

Исполни, Чистая, веселия сердце мое, Твою нетленную дающи радость, веселия рождшая виновнаго.

\pripev{Пресвят\'{а}я Богор\'{о}дице, спас\'{и} нас.}

Избaви нас от бед, Богородице чистая, вечное рождши избавление, и мир, всяк ум преимущий.

\slava

Разреши мглу прегрешений моих, Богоневесто, просвещением Твоея светлости, Свет рождшая Божественный и превечный.

\inyne

Исцели, Чистая, души моея неможение, посещения Твоего сподобльшая, и здрaвие молитвами Твоими подaждь ми.

\mysubtitle{Песнь 6}

\irmos{Молитву пролию ко Господу, и Тому возвещу печали моя, яко зол душа моя исполнися, и живот мой аду приближися, и молюся яко Иона: от тли, Боже, возведи мя.}

\pripev{Пресвят\'{а}я Богор\'{о}дице, спас\'{и} нас.}

Смерти и тли яко спасл есть, Сам Ся издaв смерти, тлением и смертию мое естество, ято бывшее, Дево, моли Господа и Сына Твоего, врагов злодействия мя избaвити.

\pripev{Пресвят\'{а}я Богор\'{о}дице, спас\'{и} нас.}

Предстaтельницу Тя живота вем и хранительницу тверду, Дево, и напaстей решaщу молвы, и налоги бесов отгоняющу; и молюся всегда, от тли страстей моих избaвити мя.

\slava

Яко стену прибежища стяжaхом, и душ всесовершенное спасение, и прострaнство в скорбех, Отроковице, и просвещением Твоим присно рaдуемся: о, Владычице, и ныне нас от страстей и бед спаси.

\inyne

На одре ныне немощствуяй лежу, и несть исцеления плоти моей: но, Бога и Спaса миру и Избaвителя недугов рождшая, Тебе молюся, Благой: от тли недуг возстaви мя.

\mysubtitle{Кондaк, глас 6-й}

Предстaтельство христиан непостыдное, ходaтайство ко Творцу непреложное, не презри грешных молений глaсы, но предвари, яко Благaя, на помощь нас, верно зовущих Ти; ускори на молитву, и потщися на умоление, предстaтельствующи присно, Богородице, чтущих Тя.

\mysubtitle{Другой кондaк, глас тот же}

Не имамы иныя помощи, не имамы иныя надежды, разве Тебе, Пречистая Дево. Ты нам помози, на Тебе надеемся, и Тобою хвaлимся, Твои бо есмы рабы, да не постыдимся.

\mysubtitle{Стихира, глас тот же}

Не ввери мя человеческому предстaтельству, Пресвятая Владычице, но приими моление раба Твоего: скорбь бо обдержит мя, терпети не могу демонскаго стреляния, покрова не имам, ниже где прибегну, окаянный, всегда побеждaемь, и утешения не имам, разве Тебе, Владычице мира, уповaние и предстaтельство верных, не презри моление мое, полезно сотвори.

\mysubtitle{Песнь 7}

\irmos{От Иудеи дошедше отроцы, в Вавилоне иногдa, верою Троическою плaмень пещный попрaша, поюще: отцев Боже, благословен еси.}

\pripev{Пресвят\'{а}я Богор\'{о}дице, спас\'{и} нас.}

Наше спасение якоже восхотел еси, Спaсе, устроити, во утробу Девыя вселился еси, Юже миру предстaтельницу показал еси: отец наших Боже, благословен еси.

\pripev{Пресвят\'{а}я Богор\'{о}дице, спас\'{и} нас.}

Волителя милости, Егоже родилa еси, Мaти чистая, умоли избaвитися от прегрешений и душевных скверн верою зовущим: отец наших Боже, благословен еси.

\slava

Сокровище спасения и Источник нетления, Тя рождшую, и столп утверждения, и дверь покаяния, зовущим показал еси: отец наших Боже, благословен еси.

\inyne

Телесныя слабости и душевныя недуги, Богородительнице, любовию приступaющих к крову Твоему, Дево, исцелити сподоби, Спaса Христа нам рождшая.

\mysubtitle{Песнь 8}

\irmos{Царя Небеснаго, Егоже поют вои aнгельстии, хвалите и превозносите во вся веки.}

\pripev{Пресвят\'{а}я Богор\'{о}дице, спас\'{и} нас.}

Помощи яже от Тебе требующия не презри, Дево, поющия и превозносящия Тя во веки.

\pripev{Пресвят\'{а}я Богор\'{о}дице, спас\'{и} нас.}

Неможение души моея исцеляеши и телесныя болезни, Дево, да Тя прослaвлю, Чистая, во веки.

\slava

Исцелений богатство изливaеши верно поющим Тя, Дево, и превозносящим неизреченное Твое рождество.

\inyne

Напaстей Ты прилоги отгоняеши и страстей находы, Дево: темже Тя поем во вся веки.

\mysubtitle{Песнь 9}

\irmos{Воистинну Богородицу Тя исповедуем, спасеннии Тобою, Дево чистая, с безплотными лики Тя величaюще.}

\pripev{Пресвят\'{а}я Богор\'{о}дице, спас\'{и} нас.}

Тока слез моих не отвратися, Яже от всякаго лица всяку слезу отъемшаго, Дево, Христа рождшая.

\pripev{Пресвят\'{а}я Богор\'{о}дице, спас\'{и} нас.}

Радости мое сердце исполни, Дево, Яже радости приемшая исполнение, греховную печаль потребляющи.

\pripev{Пресвят\'{а}я Богор\'{о}дице, спас\'{и} нас.}

Пристaнище и предстaтельство к Тебе прибегaющих буди, Дево, и стена нерушимая, прибежище же и покров и веселие.

\slava

Света Твоего зарями просвети, Дево, мрак неведения отгоняющи, благоверно Богородицу Тя исповедающих.

\inyne

На месте озлобления немощи смирившагося, Дево, исцели, из нездрaвия во здрaвие претворяющи.

\mysubtitle{Стихиры, глас 2-й}

Высшую небес и чистшую светлостей солнечных, избaвльшую нас от клятвы, Владычицу мира песньми почтим.

От многих моих грехов немощствует тело, немощствует и душа моя; к Тебе прибегaю, Благодaтней, надеждо ненадежных, Ты ми помози.

Владычице и Мaти Избaвителя, приими моление недостойных раб Твоих, да ходaтайствуеши к Рождшемуся от Тебе; о, Владычице мира, буди Ходaтаица!

Поем прилежно Тебе песнь ныне, всепетой Богородице, рaдостно: со Предтечею и всеми святыми моли, Богородице, еже ущедрити ны.

Вся aнгелов воинства, Предтече Господень, апостолов двоенадесятице, святии вси с Богородицею, сотворите молитву, во еже спастися нам.

\mysubtitle{Молитвы ко Пресвятой Богородице}

Пресвятая Богородице, спаси мя.

Царице моя преблагaя, надеждо моя Богородице, приятелище сирых и странных предстaтельнице, скорбящих рaдосте, обидимых покровительнице! Зриши мою беду, зриши мою скорбь, помози ми яко немощну, окорми мя яко стрaнна. Обиду мою веси, разреши ту, яко волиши: яко не имам иныя помощи разве Тебе, ни иныя предстaтельницы, ни благия утешительницы, токмо Тебе, о Богомaти, яко да сохраниши мя и покрыеши во веки веков. Аминь.

К кому возопию, Владычице? К кому прибегну в горести моей, aще не к Тебе, Царице Небесная? Кто плач мой и воздыхaние мое приимет, aще не Ты, Пренепорочная, надеждо христиан и прибежище нам, грешным? Кто пaче Тебе в напaстех защитит? Услыши убо стенaние мое, и приклони ухо Твое ко мне, Владычице Мaти Бога моего, и не презри мене, требующаго Твоея помощи, и не отрини мене, грешнаго. Вразуми и научи мя, Царице Небесная; не отступи от мене, раба Твоего, Владычице, за роптaние мое, но буди мне Мaти и заступница. Вручaю себе милостивому покрову Твоему: приведи мя, грешнаго, к тихой и безмятежной жизни, да плaчуся о гресех моих. К кому бо прибегну повинный аз, aще не к Тебе, уповaнию и прибежищу грешных, надеждою на неизреченную милость Твою и щедроты Твоя окриляемь? О, Владычице Царице Небесная! Ты мне уповaние и прибежище, покров и заступление и помощь. Царице моя преблагaя и скорая заступнице! Покрый Твоим ходaтайством моя прегрешения, защити мене от враг видимых и невидимых; умягчи сердца злых человек, возстающих на мя. О, Мaти Господа моего Творцa! Ты еси корень девства и неувядaемый цвет чистоты. О, Богородительнице! Ты подaждь ми помощь немощствующему плотскими страстьми и болезнующему сердцем, едино бо Твое и с Тобою Твоего Сына и Бога нашего имам заступление; и Твоим пречудным заступлением да избaвлюся от всякия беды и напaсти, о пренепорочная и преслaвная Божия Мaти Марие. Темже со уповaнием глаголю и вопию: радуйся, благодaтная, радуйся, обрaдованная; радуйся, преблагословенная, Господь с Тобою.

\end{mymulticols}

\mychapterending


\mychapter{Канон Ангелу Хранителю}\begin{mymulticols}
%http://www.molitvoslov.org/text5.htm

\myfigure{8}

\mysubtitle{Тропарь, глас 6-й}

Ангеле Божий, хранителю мой святый, живот мой соблюди во страсе Христа Бога, ум мой утверди во истиннем пути, и к любви горней уязви душу мою, да тобою направляемь, получу от Христа Бога велию милость.

Слава Отцу и Сыну и Святому Духу. И ныне и присно и во веки веков. Аминь.

\mysubtitle{Богородичен}

Святая Владычице, Христа Бога нашего Мати, яко всех Творца недоуменно рождшая, моли благость Его всегда, со хранителем моим ангелом, спасти душу мою, страстьми одержимую, и оставление грехов даровати ми.

\mysubtitle{Канон, глас 8-й}

\mysubtitle{Песнь 1}

\irmos{Поим Господеви, проведшему люди Своя сквозе Чермное море, яко един славно прославися.}

\pripev[Иисусу:]{Господи Иисусе Христе Боже мой, помилуй мя.}

Песнь воспети и восхвалити, Спасе, Твоего раба достойно сподоби, безплотному Aнгелу, наставнику и хранителю моему.

\pripev{Святый Aнгеле Божий, хранителю мой, моли Бога о мне.}

Един аз в неразумии и в лености ныне лежу, наставниче мой и хранителю, не остави мене, погибающа.

\slava

Ум мой твоею молитвою направи, творити ми Божия повеления, да получу от Бога отдание грехов, и ненавидети ми злых настави мя, молюся ти.

\inyne

Молися, Девице, о мне, рабе Твоем, ко Благодателю, со хранителем моим Aнгелом, и настави мя творити заповеди Сына Твоего и Творца моего.

\mysubtitle{Песнь 3}

\irmos{Ты еси утверждение притекающих к Тебе, Господи, Ты еси свет омраченных, и поет Тя дух мой.}

\pripev{Святый Aнгеле Божий, хранителю мой, моли Бога о мне.}

Все помышление мое и душу мою к тебе возложих, хранителю мой; ты от всякия мя напасти вражия избави.

\pripev{Святый Aнгеле Божий, хранителю мой, моли Бога о мне.}

Враг попирает мя, и озлобляет, и поучает всегда творити своя хотения; но ты, наставниче мой, не остави мене погибающа.

\slava

Пети песнь со благодарением и усердием Творцу и Богу даждь ми, и тебе, благому Aнгелу хранителю моему: избавителю мой, изми мя от враг озлобляющих мя.

\inyne

Исцели, Пречистая, моя многонедужныя струпы, яже в души, прожени враги, иже присно борются со мною.

\mysubtitle{Седален, глас 2-й}

От любве душевныя вопию ти, хранителю моея души, всесвятый мой Aнгеле: покрый мя и соблюди от лукаваго ловления всегда, и к жизни настави небесней, вразумляя и просвещая и укрепляя мя.

Слава Отцу и Сыну и Святому Духу. И ныне и присно и во веки веков. Аминь.

\mysubtitle{Богородичен:}

Богородице безневестная Пречистая, Яже без семени рождши всех Владыку, Того со Aнгелом хранителем моим моли, избавити ми ся всякаго недоумения, и дати умиление и свет души моей и согрешением очищение, Яже едина вскоре заступающи.

\mysubtitle{Песнь 4}

\irmos{Услышах, Господи, смотрения Твоего таинство, разумех дела Твоя, и прославих Твое Божество.}

\pripev{Святый Aнгеле Божий, хранителю мой, моли Бога о мне.}

Моли Человеколюбца Бога ты, хранителю мой, и не остави мене, но присно в мире житие мое соблюди и подаждь ми спасение необоримое.

\pripev{Святый Aнгеле Божий, хранителю мой, моли Бога о мне.}

Яко заступника и хранителя животу моему прием тя от Бога, Aнгеле, молю тя, святый, от всяких мя бед свободи.

\slava

Мою скверность твоею святынею очисти, хранителю мой, и от части шуия да отлучен буду молитвами твоими и причастник славы явлюся.

\inyne

Недоумение предлежит ми от обышедших мя зол, Пречистая, но избави мя от них скоро: к Тебе бо единей прибегох.

\mysubtitle{Песнь 5}

\irmos{Утренююще вопием Ти: Господи, спаси ны; Ты бо еси Бог наш, разве Тебе иного не вемы.}

\pripev{Святый Aнгеле Божий, хранителю мой, моли Бога о мне.}

Яко имея дерзновение к Богу, хранителю мой святый, Сего умоли от оскорбляющих мя зол избавити.

\pripev{Святый Aнгеле Божий, хранителю мой, моли Бога о мне.}

Свете светлый, светло просвети душу мою, наставниче мой и хранителю, от Бога данный ми Aнгеле.

\slava

Спяща мя зле тяготою греховною, яко бдяща сохрани, Aнгеле Божий, и возстави мя на славословие молением твоим.

\inyne

Марие, Госпоже Богородице безневестная, надеждо верных, вражия возношения низложи, поющия же Тя возвесели.

\newpage\mysubtitle{Песнь 6}

\irmos{Ризу ми подаждь светлу, одеяйся светом яко ризою, многомилостиве Христе Боже наш.}

\pripev{Святый Aнгеле Божий, хранителю мой, моли Бога о мне.}

Всяких мя напастей свободи, и от печалей спаси, молюся ти, святый Aнгеле, данный ми от Бога, хранителю мой добрый.

\pripev{Святый Aнгеле Божий, хранителю мой, моли Бога о мне.}

Освети ум мой, блаже, и просвети мя, молюся ти, святый Aнгеле, и мыслити ми полезная всегда настави мя.

\slava

Устави сердце мое от настоящаго мятежа, и бдети укрепи мя во благих, хранителю мой, и настави мя чудно к тишине животней.

\inyne

Слово Божие в Тя вселися, Богородице, и человеком Тя показа небесную лествицу; Тобою бо к нам Вышний сошел есть.

\mysubtitle{Кондак, глас 4-й}

Явися мне милосерд, святый Aнгеле Господень, хранителю мой, и не отлучайся от мене, сквернаго, но просвети мя светом неприкосновенным и сотвори мя достойна Царствия Небеснаго.

\mysubtitle{Икос}

Уничиженную душу мою многими соблазны, ты, святый предстателю, неизреченныя славы небесныя сподоби, и певец с лики безплотных сил Божиих, помилуй мя и сохрани, и помыслы добрыми душу мою просвети, да твоею славою, Aнгеле мой, обогащуся, и низложи зломыслящия мне враги, и сотвори мя достойна Царствия Небеснаго.

\mysubtitle{Песнь 7}

\irmos{От Иудеи дошедше отроцы, в Вавилоне иногда, верою Троическою пламень пещный попраша, поюще: отцев Боже, благословен еси.}

\pripev{Святый Aнгеле Божий, хранителю мой, моли Бога о мне.}

Милостив буди ми, и умоли Бога, Господень Aнгеле, имею бо тя заступника во всем животе моем, наставника же и хранителя, от Бога дарованнаго ми во веки.

\pripev{Святый Aнгеле Божий, хранителю мой, моли Бога о мне.}

Не остави в путь шествующия души моея окаянныя убити разбойником, святый Aнгеле, яже ти от Бога предана бысть непорочне; но настави ю на путь покаяния.

\slava

Всю посрамлену душу мою привожду от лукавых ми помысл и дел: но предвари, наставниче мой, и исцеление ми подаждь благих помысл, уклоняти ми ся всегда на правыя стези.

\inyne

Премудрости исполни всех и крепости Божественныя, Ипостасная Премудросте Вышняго, Богородицы ради, верою вопиющих: отец наших Боже, благословен еси.

\mysubtitle{Песнь 8}

\irmos{Царя Небеснаго, Егоже поют вои ангельстии, хвалите и превозносите во вся веки.}

\pripev{Святый Aнгеле Божий, хранителю мой, моли Бога о мне.}

От Бога посланный, утверди живот мой, раба твоего, преблагий Aнгеле, и не остави мене во веки.

\pripev{Святый Aнгеле Божий, хранителю мой, моли Бога о мне.}

Ангела тя суща блага, души моея наставника и хранителя, преблаженне, воспеваю во веки.

\slava

Буди ми покров и забрало в день испытания всех человек, воньже огнем искушаются дела благая же и злая.

\inyne

Буди ми помощница и тишина, Богородице Приснодево, рабу Твоему, и не остави мене лишена быти Твоего владычества.

\mysubtitle{Песнь 9}

\irmos{Воистинну Богородицу Тя исповедуем, спасеннии Тобою, Дево чистая, с безплотными лики Тя величающе.}

\pripev[Иисусу:]{Господи Иисусе Христе Боже мой, помилуй мя.}

Помилуй мя, едине Спасе мой, яко милостив еси и милосерд, и праведных ликов сотвори мя причастника.

\pripev{Святый Aнгеле Божий, хранителю мой, моли Бога о мне.}

Мыслити ми присно и творити, Господень Aнгеле, благая и полезная даруй, яко сильна яви в немощи и непорочна.

\slava

Яко имея дерзновение к Царю Небесному, Того моли, с прочими безплотными, помиловати мя, окаяннаго.

\inyne

Много дерзновение имущи, Дево, к Воплощшемуся из Тебе, преложи мя от уз и разрешение ми подаждь и спасение, молитвами Твоими.

\mysubtitle{Молитва к Aнгелу Xранителю}

Святый Aнгеле Божий, хранителю мой, моли Бога о мне.

Ангеле Христов святый, к тебе припадая молюся, хранителю мой святый, приданный мне на соблюдение души и телу моему грешному от святаго крещения, аз же своею леностию и своим злым обычаем прогневах твою пречистую светлость и отгнах тя от себе всеми студными делы: лжами, клеветами, завистию, осуждением, презорством, непокорством, братоненавидением, и злопомнением, сребролюбием, прелюбодеянием, яростию, скупостию, объядением без сытости и опивством, многоглаголанием, злыми помыслы и лукавыми, гордым обычаем и блудным возбешением, имый самохотение на всякое плотское вожделение. О, злое мое произволение, егоже и скоти безсловеснии не творят! Да како возможеши воззрети на мя, или приступити ко мне, аки псу смердящему? Которыма очима, ангеле Христов, воззриши на мя, оплетшася зле во гнусных делех? Да како уже возмогу отпущения просити горьким и злым моим и лукавым деянием, в няже впадаю по вся дни и нощи и на всяк час? Но молюся ти припадая, хранителю мой святый, умилосердися на мя грешнаго и недостойнаго раба твоего \myemph{(имя)}, буди ми помощник и заступник на злаго моего сопротивника, святыми твоими молитвами, и Царствия Божия причастника мя сотвори со всеми святыми, всегда, и ныне и присно и во веки веков. Аминь.

\end{mymulticols}

\mychapterending


\mychapter{Последование ко Святому Причащению}\begin{mymulticols}
%http://www.molitvoslov.org/text207.htm

\myfigure{132}

\MolitvamiSviatyhOtecNashih

\TsariuNebesnyj

\TrisviatoePoOtcheNash

Г\'{о}споди, пом\'{и}луй.\myemph{(12 раз)}

\slavainynen

\priiditepoklonimsia

\mysubtitle{Псалом 22}

Госп\'{о}дь пас\'{е}т мя, и ничт\'{о}же мя лиш\'{и}т. На м\'{е}сте зл\'{а}чне, т\'{а}мо всел\'{и} мя, на вод\'{е} пок\'{о}йне воспит\'{а} мя. Д\'{у}шу мо\'{ю} обрат\'{и}, наст\'{а}ви мя на стез\'{и} пр\'{а}вды, \'{и}мене р\'{а}ди Своег\'{о}. \'{А}ще бо и пойд\'{у} посред\'{е} с\'{е}ни см\'{е}ртныя, не убо\'{ю}ся зла, \'{я}ко Ты со мн\'{о}ю ес\'{и}, жезл Твой и п\'{а}лица Тво\'{я}, та мя ут\'{е}шиста. Угот\'{о}вал ес\'{и} пр\'{е}до мн\'{о}ю трап\'{е}зу сопрот\'{и}в стуж\'{а}ющим мне, ум\'{а}стил ес\'{и} ел\'{е}ом глав\'{у} мо\'{ю}, и ч\'{а}ша Тво\'{я} упояв\'{а}ющи мя, \'{я}ко держ\'{а}вна. И м\'{и}лость Тво\'{я} пожен\'{е}т мя вся дни живот\'{а} моег\'{о}, и \'{е}же всел\'{и}ти ми ся в дом Госп\'{о}день, в долгот\'{у} дний.

\mysubtitle{Псалом 23}

Госп\'{о}дня земл\'{я}, и исполн\'{е}ние е\'{я}, всел\'{е}нная, и вс\'{и} жив\'{у}щии на ней. Той на мор\'{я}х основ\'{а}л ю есть, и на рек\'{а}х угот\'{о}вал ю есть. Кто вз\'{ы}дет на г\'{о}ру Госп\'{о}дню? Ил\'{и} кто ст\'{а}нет на м\'{е}сте свят\'{е}м Ег\'{о}? Непов\'{и}нен рук\'{а}ма и чист с\'{е}рдцем, \'{и}же не при\'{я}т вс\'{у}е д\'{у}шу сво\'{ю}, и не кл\'{я}тся л\'{е}стию \'{и}скреннему своем\'{у}. Сей при\'{и}мет благослов\'{е}ние от Г\'{о}спода, и м\'{и}лостыню от Б\'{о}га, Сп\'{а}са своег\'{о}. Сей род \'{и}щущих Г\'{о}спода, \'{и}щущих лиц\'{е} Б\'{о}га И\'{а}ковля. Возм\'{и}те врат\'{а}, кн\'{я}зи в\'{а}ша, и возм\'{и}теся врат\'{а} в\'{е}чная; и вн\'{и}дет Царь Сл\'{а}вы. Кто есть сей Царь Сл\'{а}вы? Госп\'{о}дь кр\'{е}пок и с\'{и}лен, Госп\'{о}дь с\'{и}лен в бр\'{а}ни. Возм\'{и}те врат\'{а}, кн\'{я}зи в\'{а}ша, и возм\'{и}теся врат\'{а} в\'{е}чная; и вн\'{и}дет Царь Сл\'{а}вы. Кто есть сей Царь Сл\'{а}вы? Госп\'{о}дь сил, Той есть Царь Сл\'{а}вы.

\mysubtitle{Псалом 115}

В\'{е}ровах, т\'{е}мже возглаг\'{о}лах, аз же смир\'{и}хся зел\'{о}. Аз же рех во изступл\'{е}нии мо\'{е}м: всяк челов\'{е}к ложь. Что возд\'{а}м Г\'{о}сподеви о всех, яже воздад\'{е} ми? Ч\'{а}шу спас\'{е}ния приим\'{у}, и \'{и}мя Госп\'{о}дне призов\'{у}, мол\'{и}твы мо\'{я} Г\'{о}сподеви возд\'{а}м пред вс\'{е}ми людьм\'{и} Ег\'{о}. Честн\'{а} пред Г\'{о}сподем смерть препод\'{о}бных Ег\'{о}. О, Г\'{о}споди, аз раб Твой, аз раб Твой и сын раб\'{ы}ни Твое\'{я}; растерз\'{а}л ес\'{и} \'{у}зы мо\'{я}. Теб\'{е} пожр\'{у} ж\'{е}ртву хвал\'{ы}, и во \'{и}мя Госп\'{о}дне призов\'{у}. Мол\'{и}твы мо\'{я} Г\'{о}сподеви возд\'{а}м пред вс\'{е}ми людьм\'{и} Ег\'{о}, во дв\'{о}рех д\'{о}му Госп\'{о}дня, посред\'{е} теб\'{е}, Иерусал\'{и}ме.

\slavainynen

Аллил\'{у}ия. \myemph{(Трижды с тремя поклонами)}

\mysubtitle{Тропари, глас 8-й}

Беззак\'{о}ния мо\'{я} пр\'{е}зри, Г\'{о}споди, от Д\'{е}вы рожд\'{е}йся, и с\'{е}рдце мо\'{е} оч\'{и}сти, храм то твор\'{я} преч\'{и}стому Твоем\'{у} Т\'{е}лу и Кр\'{о}ви, ниж\'{е} отр\'{и}ни мен\'{е} от Твоег\'{о} лиц\'{а}, без числ\'{а} им\'{е}яй в\'{е}лию м\'{и}лость.

\slava

Во прич\'{а}стие свят\'{ы}нь Тво\'{и}х к\'{а}ко дерзн\'{у} [вн\'{и}ду], недост\'{о}йный? \'{А}ще бо дерзн\'{у} к Теб\'{е} приступ\'{и}ти с дост\'{о}йными, хит\'{о}н мя облич\'{а}ет, \'{я}ко несть веч\'{е}рний, и осужд\'{е}ние исход\'{а}тайствую многогр\'{е}шной душ\'{и} мо\'{е}й. Оч\'{и}сти, Г\'{о}споди, скв\'{е}рну душ\'{и} мое\'{я}, и спас\'{и} мя, \'{я}ко Человекол\'{ю}бец.

\inyne

Мн\'{о}гая мн\'{о}жества мо\'{и}х, Богор\'{о}дице, прегреш\'{е}ний, к Теб\'{е} прибег\'{о}х, Ч\'{и}стая, спас\'{е}ния тр\'{е}буя: посет\'{и} немощств\'{у}ющую мо\'{ю} д\'{у}шу, и мол\'{и} С\'{ы}на Твоег\'{о} и Б\'{о}га н\'{а}шего, д\'{а}ти ми оставл\'{е}ние, \'{я}же сод\'{е}ях л\'{ю}тых, Ед\'{и}на благослов\'{е}нная.

\mysubtitle{[Во Свят\'{у}ю же Четыредес\'{я}тницу:}

Егд\'{а} сл\'{а}внии учениц\'{ы} на умов\'{е}нии в\'{е}чери просвещ\'{а}хуся, тогд\'{а} И\'{у}да злочест\'{и}вый среброл\'{ю}бием нед\'{у}говав омрач\'{а}шеся, и беззак\'{о}нным суди\'{я}м Теб\'{е} пр\'{а}веднаго Суди\'{ю} преда\'{е}т. Виждь, им\'{е}ний рач\'{и}телю, сих р\'{а}ди удавл\'{е}ние употреб\'{и}вша: беж\'{и} нес\'{ы}тыя душ\'{и}, Уч\'{и}телю таков\'{а}я дерзн\'{у}вшия. \'{И}же о всех благ\'{и}й Г\'{о}споди, сл\'{а}ва Теб\'{е}.]

\mysubtitle{Псалом 50}

\PsalmFifty

\mysubtitle{Канон, глас 2-й}

\mysubtitle{Песнь 1}

\irmos{Гряд\'{и}те л\'{ю}дие, по\'{и}м песнь Христ\'{у} Б\'{о}гу, разд\'{е}льшему м\'{о}ре, и наст\'{а}вльшему л\'{ю}ди, \'{я}же извед\'{е} из раб\'{о}ты ег\'{и}петския, \'{я}ко просл\'{а}вися.}

\pripev{С\'{е}рдце ч\'{и}сто соз\'{и}жди во мне, Б\'{о}же, и дух прав обнов\'{и} во утр\'{о}бе мо\'{е}й.}

Хлеб живот\'{а} в\'{е}чнующаго да б\'{у}дет ми Т\'{е}ло Тво\'{е} Свят\'{о}е, благоутр\'{о}бне Г\'{о}споди, и Честн\'{а}я Кровь, и нед\'{у}г многообр\'{а}зных исцел\'{е}ние.

\pripev{Не отв\'{е}ржи мен\'{е} от лиц\'{а} Твоег\'{о}, и Д\'{у}ха Твоег\'{о} Свят\'{а}го не отым\'{и} от мен\'{е}.}

Оскверн\'{е}н д\'{е}лы безм\'{е}стными ока\'{я}нный, Твоег\'{о} Преч\'{и}стаго Т\'{е}ла и Бож\'{е}ственныя Кр\'{о}ве недост\'{о}ин есмь, Христ\'{е}, причащ\'{е}ния, ег\'{о}же мя спод\'{о}би.

\pripev{Пресвят\'{а}я Богор\'{о}дице, спас\'{и} нас.}

\Bogorodichen{Земл\'{е} благ\'{а}я, благослов\'{е}нная Богонев\'{е}сто, клас проз\'{я}бшая неор\'{а}нный и спас\'{и}тельный м\'{и}ру, спод\'{о}би мя сей яд\'{у}ща спаст\'{и}ся.}

\mysubtitle{Песнь 3}

\irmos{На камени мя веры утвердив, разширил еси уста моя на враги моя. Возвесели бо ся дух мой, внегда пети: несть свят, якоже Бог наш, и несть праведен паче Тебе, Господи.}

\pripev{С\'{е}рдце ч\'{и}сто соз\'{и}жди во мне, Б\'{о}же, и дух прав обнов\'{и} во утр\'{о}бе мо\'{е}й.}

Сл\'{е}зныя ми под\'{а}ждь, Христ\'{е}, к\'{а}пли, скв\'{е}рну с\'{е}рдца моег\'{о} очищ\'{а}ющия: \'{я}ко да благ\'{о}ю с\'{о}вестию очищ\'{е}н, в\'{е}рою прихожд\'{у} и стр\'{а}хом, Влад\'{ы}ко, ко причащ\'{е}нию Бож\'{е}ственных Дар\'{о}в Тво\'{и}х.

\pripev{Не отв\'{е}ржи мен\'{е} от лиц\'{а} Твоег\'{о}, и Д\'{у}ха Твоег\'{о} Свят\'{а}го не отым\'{и} от мен\'{е}.}

Во оставл\'{е}ние да б\'{у}дет ми прегреш\'{е}ний Преч\'{и}стое Т\'{е}ло Тво\'{е}, и Бож\'{е}ственная Кровь, Д\'{у}ха же Свят\'{а}го общ\'{е}ние, и в жизнь в\'{е}чную, Человекол\'{ю}бче, и страст\'{е}й и скорб\'{е}й отчужд\'{е}ние.

\pripev{Пресвят\'{а}я Богор\'{о}дице, спас\'{и} нас.}

\Bogorodichen{Хл\'{е}ба жив\'{о}тнаго Tрап\'{е}за Пресвят\'{а}я, св\'{ы}ше м\'{и}лости р\'{а}ди сш\'{е}дшаго, и м\'{и}рови н\'{о}вый жив\'{о}т да\'{ю}щаго, и мен\'{е} н\'{ы}не спод\'{о}би недост\'{о}йнаго, со стр\'{а}хом вкус\'{и}ти сег\'{о}, и ж\'{и}ву б\'{ы}ти.}

\mysubtitle{Песнь 4}

\irmos{Приш\'{е}л ес\'{и} от Д\'{е}вы, не ход\'{а}тай, ни \'{А}нгел, но Сам, Г\'{о}споди, воплощься, и спасл ес\'{и} всег\'{о} мя челов\'{е}ка. Тем зов\'{у} Ти: сл\'{а}ва с\'{и}ле Тво\'{е}й, Г\'{о}споди.}

\pripev{С\'{е}рдце ч\'{и}сто соз\'{и}жди во мне, Б\'{о}же, и дух прав обнов\'{и} во утр\'{о}бе мо\'{е}й.}

Восхот\'{е}л ес\'{и}, нас р\'{а}ди вопл\'{о}щся, Многом\'{и}лостиве, з\'{а}клан б\'{ы}ти \'{я}ко овч\'{а}, грех р\'{а}ди челов\'{е}ческих: т\'{е}мже мол\'{ю} Тя, и мо\'{я} оч\'{и}сти согреш\'{е}ния.

\pripev{Не отв\'{е}ржи мен\'{е} от лиц\'{а} Твоег\'{о}, и Д\'{у}ха Твоег\'{о} Свят\'{а}го не отым\'{и} от мен\'{е}.}

Исцел\'{и} душ\'{и} мое\'{я} \'{я}звы, Г\'{о}споди, и всег\'{о} освят\'{и}: и спод\'{о}би, Влад\'{ы}ко, \'{я}ко да причащ\'{у}ся т\'{а}йныя Твое\'{я} Бож\'{е}ственныя в\'{е}чери, ока\'{я}нный.

\pripev{Пресвят\'{а}я Богор\'{о}дице, спас\'{и} нас.}

\Bogorodichen{Ум\'{и}лостиви и мне С\'{у}щаго от утр\'{о}бы Твое\'{я}, Влад\'{ы}чице, и соблюд\'{и} мя нескв\'{е}рна раб\'{а} Тво\'{е}го и непор\'{о}чна, \'{я}ко да при\'{е}м \'{у}мнаго б\'{и}сера, освящ\'{у}ся.}

\mysubtitle{Песнь 5}

\irmos{Св\'{е}та Под\'{а}телю и век\'{о}в Тв\'{о}рче, Г\'{о}споди, во св\'{е}те Тво\'{и}х повел\'{е}ний наст\'{а}ви нас; р\'{а}зве бо Теб\'{е} ин\'{о}го б\'{о}га не зн\'{а}ем.}

\pripev{С\'{е}рдце ч\'{и}сто соз\'{и}жди во мне, Б\'{о}же, и дух прав обнов\'{и} во утр\'{о}бе мо\'{е}й.}

\'{Я}коже предр\'{е}кл ес\'{и}, Христ\'{е}, да б\'{у}дет \'{у}бо худ\'{о}му раб\'{у} Твоем\'{у}, и во мне преб\'{у}ди, \'{я}коже обещ\'{а}лся ес\'{и}: се бо Т\'{е}ло Тво\'{е} ям Бож\'{е}ственное, и пи\'{ю} Кровь Тво\'{ю}.

\pripev{Не отв\'{е}ржи мен\'{е} от лиц\'{а} Твоег\'{о}, и Д\'{у}ха Твоег\'{о} Свят\'{а}го не отым\'{и} от мен\'{е}.}

Сл\'{о}ве Б\'{о}жий и Б\'{о}же, угль Т\'{е}ла Тво\'{е}го да б\'{у}дет мне помрач\'{е}нному в просвещ\'{е}ние, и очищ\'{е}ние оскверн\'{е}нной душ\'{и} мо\'{е}й Кровь Тво\'{я}.

\pripev{Пресвят\'{а}я Богор\'{о}дице, спас\'{и} нас.}

\Bogorodichen{Мар\'{и}е, М\'{а}ти Б\'{о}жия, благоух\'{а}ния честн\'{о}е сел\'{е}ние, Тво\'{и}ми мол\'{и}твами сос\'{у}д мя избр\'{а}нный сод\'{е}лай, \'{я}ко да освящ\'{е}ний причащ\'{у}ся С\'{ы}на Тво\'{е}го.}

\mysubtitle{Песнь 6}

\irmos{В б\'{е}здне грех\'{о}вней вал\'{я}яся, неизсл\'{е}дную милос\'{е}рдия Тво\'{е}го призыв\'{а}ю б\'{е}здну: от тли, Б\'{о}же, мя возвед\'{и}.}

\pripev{С\'{е}рдце ч\'{и}сто соз\'{и}жди во мне, Б\'{о}же, и дух прав обнов\'{и} во утр\'{о}бе мо\'{е}й.}

Ум, д\'{у}шу и с\'{е}рдце освят\'{и}, Сп\'{а}се, и т\'{е}ло мо\'{е}, и спод\'{о}би неосужд\'{е}нно, Влад\'{ы}ко, к стр\'{а}шным Т\'{а}йнам приступ\'{и}ти.

\pripev{Не отв\'{е}ржи мен\'{е} от лиц\'{а} Твоег\'{о}, и Д\'{у}ха Твоег\'{о} Свят\'{а}го не отым\'{и} от мен\'{е}.}

Да бых устран\'{и}лся от страст\'{е}й, и Твое\'{я} благод\'{а}ти им\'{е}л бы прилож\'{е}ние, живот\'{а} же утвержд\'{е}ние, причащ\'{е}нием Свят\'{ы}х, Христ\'{е}, Т\'{а}ин Тво\'{и}х.

\pripev{Пресвят\'{а}я Богор\'{о}дице, спас\'{и} нас.}

\Bogorodichen{Б\'{о}жие, Б\'{о}же, Сл\'{о}во Свят\'{о}е, всег\'{о} мя освят\'{и}, н\'{ы}не приход\'{я}щаго к Бож\'{е}ственным Тво\'{и}м Т\'{а}йнам, Свят\'{ы}я М\'{а}тере Твое\'{я} мольб\'{а}ми.}

\mysubtitle{Кондак, глас 2-й}

Хлеб, Христ\'{е}, вз\'{я}ти не пр\'{е}зри мя, Т\'{е}ло Тво\'{е}, и Бож\'{е}ственную Тво\'{ю} н\'{ы}не Кровь, преч\'{и}стых, Влад\'{ы}ко, и стр\'{а}шных Тво\'{и}х Т\'{а}ин причаст\'{и}тися ока\'{я}ннаго, да не б\'{у}дет ми в суд, да б\'{у}дет же ми в жив\'{о}т в\'{е}чный и безсм\'{е}ртный.

\mysubtitle{Песнь 7}

\irmos{Т\'{е}лу злат\'{о}му прем\'{у}дрыя д\'{е}ти не послуж\'{и}ша, и в пл\'{а}мень с\'{а}ми поид\'{о}ша, и б\'{о}ги их обруг\'{а}ша, сред\'{и} пл\'{а}мен\'{е} возоп\'{и}ша, и орос\'{и} я \'{А}нгел: усл\'{ы}шася уж\'{е} уст в\'{а}ших мол\'{и}тва.}

\pripev{С\'{е}рдце ч\'{и}сто соз\'{и}жди во мне, Б\'{о}же, и дух прав обнов\'{и} во утр\'{о}бе мо\'{е}й.}

Ист\'{о}чник благ\'{и}х, причащ\'{е}ние, Христ\'{е}, безсм\'{е}ртных Тво\'{и}х н\'{ы}не Т\'{а}инств да б\'{у}дет ми свет, и жив\'{о}т, и безстр\'{а}стие, и к преспе\'{я}нию же и умнож\'{е}нию доброд\'{е}тели Бож\'{е}ственнейшия ход\'{а}тайственно, ед\'{и}не Бл\'{а}же, \'{я}ко да сл\'{а}влю Тя.

\pripev{Не отв\'{е}ржи мен\'{е} от лиц\'{а} Твоег\'{о}, и Д\'{у}ха Твоег\'{о} Свят\'{а}го не отым\'{и} от мен\'{е}.}

Да изб\'{а}влюся от страст\'{е}й, и враг\'{о}в, и н\'{у}жды, и вс\'{я}кия ск\'{о}рби, тр\'{е}петом и люб\'{о}вию со благогов\'{е}нием, Человекол\'{ю}бче, приступ\'{а}яй н\'{ы}не к Тво\'{и}м безсм\'{е}ртным и Бож\'{е}ственным Т\'{а}йнам, и п\'{е}ти Теб\'{е} спод\'{о}би: благослов\'{е}н ес\'{и}, Г\'{о}споди, Б\'{о}же от\'{е}ц н\'{а}ших.

\pripev{Пресвят\'{а}я Богор\'{о}дице, спас\'{и} нас.}

\Bogorodichen{Сп\'{а}са Христ\'{а} р\'{о}ждшая п\'{а}че ум\'{а}, Богоблагод\'{а}тная, мол\'{ю} Тя н\'{ы}не, раб Твой, Ч\'{и}стую неч\'{и}стый: хот\'{я}щаго мя н\'{ы}не к преч\'{и}стым Т\'{а}йнам приступ\'{и}ти, оч\'{и}сти всег\'{о} от скв\'{е}рны пл\'{о}ти и д\'{у}ха.}

\mysubtitle{Песнь 8}

\irmos{В пещь \'{о}гненную ко отрок\'{о}м евр\'{е}йским снизш\'{е}дшаго, и пл\'{а}мень в р\'{о}су прел\'{о}жшаго Б\'{о}га, п\'{о}йте дел\'{а} \'{я}ко Г\'{о}спода, и превознос\'{и}те во вся в\'{е}ки.}

\pripev{С\'{е}рдце ч\'{и}сто соз\'{и}жди во мне, Б\'{о}же, и дух прав обнов\'{и} во утр\'{о}бе мо\'{е}й.}

Неб\'{е}сных, и стр\'{а}шных, и свят\'{ы}х Тво\'{и}х, Христ\'{е}, н\'{ы}не Т\'{а}ин, и Бож\'{е}ственныя Твое\'{я} и т\'{а}йныя в\'{е}чери \'{о}бщника б\'{ы}ти и мен\'{е} спод\'{о}би отч\'{а}яннаго, Б\'{о}же, Сп\'{а}се мой.

\pripev{Не отв\'{е}ржи мен\'{е} от лиц\'{а} Твоег\'{о}, и Д\'{у}ха Твоег\'{о} Свят\'{а}го не отым\'{и} от мен\'{е}.}

Под Тво\'{е} приб\'{е}г благоутр\'{о}бие, Бл\'{а}же, со стр\'{а}хом зов\'{у} Ти: во мне преб\'{у}ди, Сп\'{а}се, и аз, \'{я}коже рекл ес\'{и}, в Теб\'{е}; се бо дерз\'{а}я на м\'{и}лость Тво\'{ю}, ям Т\'{е}ло Тво\'{е}, и пи\'{ю} Кровь Тво\'{ю}.

\pripev{Пресвят\'{а}я Тр\'{о}ице, Б\'{о}же наш, сл\'{а}ва Теб\'{е}.}

\myemph{Тр\'{о}ичен:} Треп\'{е}щу, при\'{е}мля огнь, да не опал\'{ю}ся \'{я}ко воск и \'{я}ко трав\'{а}; \'{o}ле стр\'{а}шнаго т\'{а}инства! \'{o}ле благоутр\'{о}бия Б\'{о}жия! К\'{а}ко Бож\'{е}ственнаго Т\'{е}ла и Кр\'{о}ве бр\'{е}ние причащ\'{а}юся, и нетл\'{е}нен сотвор\'{я}юся?

\mysubtitle{Песнь 9}

\irmos{Безнач\'{а}льна Род\'{и}теля Сын, Бог и Госп\'{о}дь, вопл\'{о}щся от Д\'{е}вы нам яв\'{и}ся, омрач\'{е}нная просвет\'{и}ти, собр\'{а}ти расточ\'{е}нная: тем всеп\'{е}тую Богор\'{о}дицу велич\'{а}ем.}

\pripev{С\'{е}рдце ч\'{и}сто соз\'{и}жди во мне, Б\'{о}же, и дух прав обнов\'{и} во утр\'{о}бе мо\'{е}й.}

Христ\'{о}с \'{е}сть, вкус\'{и}те и в\'{и}дите: Госп\'{о}дь нас р\'{а}ди, по нам бо др\'{е}вле б\'{ы}вый, ед\'{и}ною Себ\'{е} прин\'{е}с, \'{я}ко принош\'{е}ние Отц\'{у} Своем\'{у}, пр\'{и}сно закал\'{а}ется, освящ\'{а}яй причащ\'{а}ющияся.

\pripev{Не отв\'{е}ржи мен\'{е} от лиц\'{а} Твоег\'{о}, и Д\'{у}ха Твоег\'{о} Свят\'{а}го не отым\'{и} от мен\'{е}.}

Душ\'{е}ю и т\'{е}лом да освящ\'{у}ся, Влад\'{ы}ко, да просвещ\'{у}ся, да спас\'{у}ся, да б\'{у}ду дом Твой причащ\'{е}нием свящ\'{е}нных Т\'{а}ин, жив\'{у}щаго Тя им\'{е}я в себ\'{е} со Отц\'{е}м и Д\'{у}хом, Благод\'{е}телю Многом\'{и}лостиве.

\pripev{Возд\'{а}ждь ми р\'{а}дость спас\'{е}ния Тво\'{е}го и Д\'{у}хом Владычним утверд\'{и} мя.}

\'{Я}коже огнь да б\'{у}дет ми, и \'{я}ко свет, Т\'{е}ло Тво\'{е} и Кровь, Сп\'{а}се мой, пречестн\'{а}я, опал\'{я}я грех\'{о}вное веществ\'{о}, сжиг\'{а}я же страст\'{е}й т\'{е}рние, и всег\'{о} мя просвещ\'{а}я, поклан\'{я}тися Божеств\'{у} Твоем\'{у}.

\pripev{Пресвят\'{а}я Богор\'{о}дице, спас\'{и} нас.}

\Bogorodichen{Бог воплот\'{и}ся от ч\'{и}стых кров\'{е}й Тво\'{и}х; т\'{е}мже вс\'{я}кий род по\'{е}т Тя, Влад\'{ы}чице, \'{у}мная же мн\'{о}жества сл\'{а}вят, \'{я}ко Тоб\'{о}ю \'{я}ве узр\'{е}ша вс\'{е}ми Влад\'{ы}чествующаго, осуществов\'{а}вшагося челов\'{е}чеством.}

\mysubtitle{Далее}

\Chestneyshuyu

\TrisviatoePoOtcheNash

\myemph{Если неделя, тропарь воскресный по гласу. Если же нет, настоящие тропари, глас 6-й:}

Пом\'{и}луй нас, Г\'{о}споди, пом\'{и}луй нас; вс\'{я}каго бо отв\'{е}та недоум\'{е}юще, си\'{ю} Ти мол\'{и}тву, \'{я}ко Влад\'{ы}це, гр\'{е}шнии прин\'{о}сим: пом\'{и}луй нас.

\slava

Г\'{о}споди, пом\'{и}луй нас, на Тя бо упов\'{а}хом; не прогн\'{е}вайся на ны зел\'{о}, ниж\'{е} помян\'{и} беззак\'{о}ний н\'{а}ших, но пр\'{и}зри и н\'{ы}не \'{я}ко благоутр\'{о}бен, и изб\'{а}ви ны от враг н\'{а}ших. Ты бо ес\'{и} Бог наш, и мы л\'{ю}дие Тво\'{и}, вс\'{и} дел\'{а} рук\'{у} Тво\'{е}ю, и \'{и}мя Тво\'{е} призыв\'{а}ем.

\inyne

Милос\'{е}рдия дв\'{е}ри отв\'{е}рзи нам, благослов\'{е}нная Богор\'{о}дице, над\'{е}ющиися на Тя да не пог\'{и}бнем, но да изб\'{а}вимся Тоб\'{о}ю от бед: Ты бо ес\'{и} спас\'{е}ние р\'{о}да христи\'{а}нскаго.

Г\'{о}споди, пом\'{и}луй. \myemph{(40 раз) И поклоны, сколько хочешь.}

\myemph{И стихи:}

\begin{verse}
Хот\'{я} \'{я}сти, челов\'{е}че, Т\'{е}ло Влад\'{ы}чне,

Стр\'{а}хом приступ\'{и}, да не опал\'{и}шися: огнь бо \'{е}сть.

Бож\'{е}ственную же пи\'{я} Кровь ко общ\'{е}нию,

П\'{е}рвее примир\'{и}ся тя опеч\'{а}лившим.

Т\'{а}же дерз\'{а}я, т\'{а}инственное бр\'{а}шно яждь.

Пр\'{е}жде прич\'{а}стия стр\'{а}шныя ж\'{е}ртвы,

Животвор\'{я}щаго Т\'{е}ла Влад\'{ы}чня,

Сим помол\'{и}ся \'{о}бразом со тр\'{е}петом:
\end{verse}

\mysubtitle{Молитва 1-я, Василия Великого}

Влад\'{ы}ко Г\'{о}споди Иис\'{у}се Христ\'{е}, Б\'{о}же наш, Ист\'{о}чниче ж\'{и}зни и безсм\'{е}ртия, все\'{я} тв\'{а}ри в\'{и}димыя и нев\'{и}димыя Сод\'{е}телю, безнач\'{а}льнаго Отц\'{а} соприснос\'{у}щный С\'{ы}не и собезнач\'{а}льный, премн\'{о}гия р\'{а}ди бл\'{а}гости в посл\'{е}дния дни в плоть оболк\'{и}йся, и распн\'{ы}йся, и погреб\'{ы}йся за ны неблагод\'{а}рныя и злонр\'{а}вныя, и Тво\'{е}ю Кр\'{о}вию обнов\'{и}вый растл\'{е}вшее грех\'{о}м естеств\'{о} н\'{а}ше, Сам, Безсм\'{е}ртный Цар\'{ю}, приим\'{и} и мо\'{е} гр\'{е}шнаго пока\'{я}ние, и приклон\'{и} \'{у}хо Тво\'{е} мне, и усл\'{ы}ши глаг\'{о}лы мо\'{я}. Согреш\'{и}х бо, Г\'{о}споди, согреш\'{и}х на н\'{е}бо и пред Тоб\'{о}ю, и несмь дост\'{о}ин воззр\'{е}ти на высот\'{у} сл\'{а}вы Твое\'{я}: прогн\'{е}вах бо Тво\'{ю} бл\'{а}гость, Тво\'{я} з\'{а}поведи преступ\'{и}в, и не посл\'{у}шав Тво\'{и}х повел\'{е}ний. Но Ты, Г\'{о}споди, незл\'{о}бив сый, долготерпел\'{и}в же и многом\'{и}лостив, не пр\'{е}дал ес\'{и} мя пог\'{и}бнути со беззак\'{о}ньми мо\'{и}ми, моег\'{о} вс\'{я}чески ожид\'{а}я обращ\'{е}ния. Ты бо рекл ес\'{и}, Человекол\'{ю}бче, прор\'{о}ком Тво\'{и}м: \'{я}ко хот\'{е}нием не хощ\'{у} см\'{е}рти гр\'{е}шника, но \'{е}же обрат\'{и}тся и ж\'{и}ву б\'{ы}ти ем\'{у}. Не х\'{о}щеши бо, Влад\'{ы}ко, созд\'{а}ния Тво\'{е}ю рук\'{у} погуб\'{и}ти, ниж\'{е} благовол\'{и}ши о пог\'{и}бели челов\'{е}честей, но х\'{о}щеши всем спаст\'{и}ся, и в р\'{а}зум \'{и}стины приит\'{и}. Т\'{е}мже и аз, \'{а}ще и недост\'{о}ин есмь небес\'{е} и земл\'{и}, и се\'{я} привр\'{е}менныя ж\'{и}зни, всег\'{о} себ\'{е} повин\'{у}в грех\'{у}, и сласт\'{е}м пораб\'{о}тив, и Твой оскверн\'{и}в \'{о}браз; но твор\'{е}ние и созд\'{а}ние Тво\'{е} быв, не отчаяв\'{а}ю своег\'{о} спас\'{е}ния ока\'{я}нный, на Тво\'{е} же безм\'{е}рное благоутр\'{о}бие дерз\'{а}я, прихожд\'{у}. Приим\'{и} \'{у}бо и мен\'{е}, Человекол\'{ю}бче Г\'{о}споди, \'{я}коже блудн\'{и}цу, \'{я}ко разб\'{о}йника, \'{я}ко мытар\'{я} и \'{я}ко бл\'{у}днаго, и возм\'{и} мо\'{е} т\'{я}жкое бр\'{е}мя грех\'{о}в, грех вз\'{е}мляй м\'{и}ра, и н\'{е}мощи челов\'{е}ческия исцел\'{я}яй, тружд\'{а}ющияся и обремен\'{е}нныя к Себ\'{е} призыв\'{а}яй и упокоев\'{а}яй, не приш\'{е}дый призв\'{а}ти пр\'{а}ведныя, но гр\'{е}шныя на пока\'{я}ние. И оч\'{и}сти мя от вс\'{я}кия скв\'{е}рны пл\'{о}ти и д\'{у}ха, и науч\'{и} мя соверш\'{а}ти свят\'{ы}ню во стр\'{а}се Тво\'{е}м: \'{я}ко да ч\'{и}стым св\'{е}дением с\'{о}вести мое\'{я}, свят\'{ы}нь Тво\'{и}х часть при\'{е}мля, соедин\'{ю}ся свят\'{о}му Т\'{е}лу Твоем\'{у} и Кр\'{о}ви, и им\'{е}ю Теб\'{е} во мне жив\'{у}ща и пребыв\'{а}юща, со Отц\'{е}м, и Свят\'{ы}м Тво\'{и}м Д\'{у}хом. Ей, Г\'{о}споди Иис\'{у}се Христ\'{е}, Б\'{о}же мой, и да не в суд ми б\'{у}дет прич\'{а}стие преч\'{и}стых и животвор\'{я}щих Т\'{а}ин Тво\'{и}х, ниж\'{е} да н\'{е}мощен б\'{у}ду душ\'{е}ю же и т\'{е}лом, от \'{е}же нед\'{о}стойне тем причащ\'{а}тися, но даждь ми, д\'{а}же до кон\'{е}чнаго моег\'{о} издых\'{а}ния, неосужд\'{е}нно восприим\'{а}ти часть свят\'{ы}нь Тво\'{и}х, в Д\'{у}ха Свят\'{а}го общ\'{е}ние, в нап\'{у}тие живот\'{а} в\'{е}чнаго, и во благопри\'{я}тен отв\'{е}т на Стр\'{а}шнем суд\'{и}щи Тво\'{е}м: \'{я}ко да и аз со вс\'{е}ми избр\'{а}нными Тво\'{и}ми \'{о}бщник б\'{у}ду нетл\'{е}нных Тво\'{и}х благ, \'{я}же угот\'{о}вал ес\'{и} л\'{ю}бящим Тя, Г\'{о}споди, в н\'{и}хже препросл\'{а}влен ес\'{и} во в\'{е}ки. Ам\'{и}нь.

\mysubtitle{Молитва 2-я, святого Иоанна Златоустого}

Г\'{о}споди Б\'{о}же мой, вем, \'{я}ко несмь дост\'{о}ин, ниж\'{е} дов\'{о}лен, да под кров вн\'{и}деши хр\'{а}ма душ\'{и} мое\'{я}, зан\'{е}же весь пуст и п\'{а}лся \'{е}сть, и не \'{и}маши во мне м\'{е}ста дост\'{о}йна \'{е}же глав\'{у} подклон\'{и}ти: но \'{я}коже с высот\'{ы} нас р\'{а}ди смир\'{и}л ес\'{и} Себ\'{е}, смир\'{и}ся и н\'{ы}не смир\'{е}нию моем\'{у}; и \'{я}коже воспри\'{я}л ес\'{и} в верт\'{е}пе и в \'{я}слех безслов\'{е}сных возлещ\'{и}, с\'{и}це восприим\'{и} и в \'{я}слех безслов\'{е}сныя мое\'{я} душ\'{и}, и во оскверн\'{е}нное мо\'{е} т\'{е}ло вн\'{и}ти. И \'{я}коже не неудост\'{о}ил ес\'{и} вн\'{и}ти, и свечер\'{я}ти со гр\'{е}шники в д\'{о}му С\'{и}мона прокаж\'{е}ннаго, т\'{а}ко изв\'{о}ли вн\'{и}ти и в дом смир\'{е}нныя мое\'{я} душ\'{и}, прокаж\'{е}нныя и гр\'{е}шныя; и \'{я}коже не отр\'{и}нул ес\'{и} под\'{о}бную мне блудн\'{и}цу и гр\'{е}шную, приш\'{е}дшую и прикосн\'{у}вшуюся Теб\'{е}, с\'{и}це умилос\'{е}рдися и о мне гр\'{е}шнем, приход\'{я}щем и прикас\'{а}ющем Ти ся; и \'{я}коже не возгнуш\'{а}лся ес\'{и} скв\'{е}рных е\'{я} уст и неч\'{и}стых, цел\'{у}ющих Тя, ниж\'{е} мо\'{и}х возгнуш\'{а}йся скв\'{е}рнших \'{о}ныя уст и неч\'{и}стших, ниж\'{е} м\'{е}рзких мо\'{и}х и неч\'{и}стых уст\'{е}н, и скв\'{е}рнаго и неч\'{и}стейшаго моег\'{о} яз\'{ы}ка. Но да б\'{у}дет ми угль пресвят\'{а}го Тво\'{е}го Т\'{е}ла, и честн\'{ы}я Твое\'{я} Кр\'{о}ве, во освящ\'{е}ние и просвещ\'{е}ние и здр\'{а}вие смир\'{е}нней мо\'{е}й душ\'{и} и т\'{е}лу, во облегч\'{е}ние т\'{я}жестей мн\'{о}гих мо\'{и}х согреш\'{е}ний, в соблюд\'{е}ние от вс\'{я}каго ди\'{а}вольскаго д\'{е}йства, во отгн\'{а}ние и возбран\'{е}ние зл\'{а}го моег\'{о} и лук\'{а}ваго об\'{ы}чая, во умерщвл\'{е}ние страст\'{е}й, в снабд\'{е}ние з\'{а}поведей Тво\'{и}х, в прилож\'{е}ние Бож\'{е}ственныя Твое\'{я} благод\'{а}ти, и Тво\'{е}го Ц\'{а}рствия присво\'{е}ние. Не бо \'{я}ко презир\'{а}яй прихожд\'{у} к Теб\'{е}, Христ\'{е} Б\'{о}же, но \'{я}ко дерз\'{а}я на неизреч\'{е}нную Тво\'{ю} бл\'{а}гость, и да не на мн\'{о}зе удал\'{я}яйся общ\'{е}ния Тво\'{е}го, от м\'{ы}сленнаго в\'{о}лка звероуловл\'{е}н б\'{у}ду. Т\'{е}мже мол\'{ю}ся Теб\'{е}: \'{я}ко ед\'{и}н сый Свят, Влад\'{ы}ко, освят\'{и} мо\'{ю} д\'{у}шу и т\'{е}ло, ум и с\'{е}рдце, чревес\'{а} и утр\'{о}бы, и всег\'{о} мя обнов\'{и}, и вкорен\'{и} страх Твой во удес\'{е}х мо\'{и}х, и освящ\'{е}ние Тво\'{е} неотъ\'{е}млемо от мен\'{е} сотвор\'{и}; и б\'{у}ди ми пом\'{о}щник и заст\'{у}пник, окормл\'{я}я в м\'{и}ре жив\'{о}т мой, сподобл\'{я}я мя и одесн\'{у}ю Теб\'{е} предсто\'{я}ния со свят\'{ы}ми Тво\'{и}ми, мол\'{и}твами и мол\'{е}ньми Преч\'{и}стыя Твое\'{я} М\'{а}тере, невещ\'{е}ственных Тво\'{и}х служ\'{и}телей и преч\'{и}стых сил, и всех свят\'{ы}х, от в\'{е}ка Теб\'{е} благоугод\'{и}вших. Ам\'{и}нь.

\mysubtitle{Молитва 3-я, Симеона Метафраста}

Ед\'{и}не ч\'{и}стый и нетл\'{е}нный Г\'{о}споди, за неизреч\'{е}нную м\'{и}лость человекол\'{ю}бия н\'{а}ше все воспри\'{е}мый смеш\'{е}ние, от ч\'{и}стых и д\'{е}вственных кров\'{е}й п\'{а}че естеств\'{а} р\'{о}ждшия Тя, Д\'{у}ха Бож\'{е}ственнаго наш\'{е}ствием, и благовол\'{е}нием Отц\'{а} приснос\'{у}щнаго, Христ\'{е} Иис\'{у}се, прем\'{у}дросте Б\'{о}жия, и м\'{и}ре, и с\'{и}ло; Тво\'{и}м воспри\'{я}тием животвор\'{я}щая и спас\'{и}тельная страд\'{а}ния воспри\'{е}мый, крест, гв\'{о}здия, копи\'{е}, смерть, умертв\'{и} мо\'{я} душетл\'{е}нныя стр\'{а}сти тел\'{е}сныя. Погреб\'{е}нием Тво\'{и}м \'{а}дова плен\'{и}вый ц\'{а}рствия, погреб\'{и} мо\'{я} благ\'{и}ми п\'{о}мыслы лук\'{а}вая сов\'{е}тования, и лук\'{а}вствия д\'{у}хи разор\'{и}. Тридн\'{е}вным Тво\'{и}м и живон\'{о}сным воскрес\'{е}нием п\'{а}дшаго пр\'{а}отца возст\'{а}вивый, возст\'{а}ви мя грех\'{о}м поп\'{о}лзшагося, \'{о}бразы мне пока\'{я}ния предлаг\'{а}я. Пресл\'{а}вным Тво\'{и}м вознес\'{е}нием плотск\'{о}е обож\'{и}вый воспри\'{я}тие, и си\'{е} десн\'{ы}м Отц\'{а} сед\'{е}нием почт\'{ы}й, спод\'{о}би мя прич\'{а}стием свят\'{ы}х Тво\'{и}х Т\'{а}ин десн\'{у}ю часть спас\'{а}емых получ\'{и}ти. Сн\'{и}тием Ут\'{е}шителя Тво\'{е}го Д\'{у}ха сос\'{у}ды ч\'{е}стны свящ\'{е}нныя Тво\'{я} ученик\'{и} сод\'{е}лавый, при\'{я}телище и мен\'{е} покаж\'{и} Тог\'{о} приш\'{е}ствия. Хот\'{я}й п\'{а}ки прийти суд\'{и}ти всел\'{е}нней пр\'{а}вдою, благовол\'{и} и мне уср\'{е}сти Тя на \'{о}блацех, Суди\'{ю} и Созд\'{а}теля моег\'{о}, со вс\'{е}ми свят\'{ы}ми Тво\'{и}ми: да безкон\'{е}чно славосл\'{о}влю и воспев\'{а}ю Тя, со безнач\'{а}льным Тво\'{и}м Отц\'{е}м, и Пресвят\'{ы}м и Благ\'{и}м и Животвор\'{я}щим Тво\'{и}м Д\'{у}хом, н\'{ы}не и пр\'{и}сно, и во в\'{е}ки век\'{о}в. Ам\'{и}нь.

\mysubtitle{Молитва 4-я, его же}
\'{Я}ко на Стр\'{а}шнем Тво\'{е}м и нелицепри\'{е}мнем предсто\'{я}й Суд\'{и}лищи, Христ\'{е} Б\'{о}же, и осужд\'{е}ния подъ\'{е}мля, и сл\'{о}во твор\'{я} о сод\'{е}янных мн\'{о}ю злых; с\'{и}це днесь, пр\'{е}жде д\'{а}же не приит\'{и} дн\'{е}ви осужд\'{е}ния моег\'{о}, у свят\'{а}го Твоег\'{о} Ж\'{е}ртвенника предсто\'{я} пред Тоб\'{о}ю и пред стр\'{а}шными и свят\'{ы}ми \'{А}нгелы Тво\'{и}ми, преклон\'{е}н от свое\'{я} с\'{о}вести, принош\'{у} лук\'{а}вая мо\'{я} и беззак\'{о}нная де\'{я}ния, явл\'{я}яй си\'{я} и облич\'{а}яй. Виждь, Г\'{о}споди, смир\'{е}ние мо\'{е}, и ост\'{а}ви вся грех\'{и} мо\'{я}; виждь, \'{я}ко умн\'{о}жишася п\'{а}че влас глав\'{ы} мое\'{я} беззак\'{о}ния мо\'{я}. К\'{о}е \'{у}бо не сод\'{е}ях зло? Кий грех не сотвор\'{и}х? К\'{о}е зло не вообраз\'{и}х в душ\'{и} мо\'{е}й? Уж\'{е} бо и д\'{е}лы сод\'{е}ях: блуд, прелюбод\'{е}йство, г\'{о}рдость, кич\'{е}ние, укор\'{е}ние, хул\'{у}, праздносл\'{о}вие, смех непод\'{о}бный, пи\'{я}нство, гортаноб\'{е}сие, объяд\'{е}ние, н\'{е}нависть, з\'{а}висть, среброл\'{ю}бие, любостяж\'{а}ние, лихо\'{и}мство, самол\'{ю}бие, славол\'{ю}бие, хищ\'{е}ние, непр\'{а}вду, злоприобр\'{е}тение, р\'{е}вность, оклевет\'{а}ние, беззак\'{о}ние; вс\'{я}кое мо\'{е} ч\'{у}вство и вс\'{я}кий уд оскверн\'{и}х, растл\'{и}х, непотр\'{е}бен сотвор\'{и}х, д\'{е}лателище быв вс\'{я}чески ди\'{а}воле. И в\'{е}м, Г\'{о}споди, \'{я}ко беззак\'{о}ния мо\'{я} превзыд\'{о}ша глав\'{у} мою; но безм\'{е}рно есть мн\'{о}жество щедр\'{о}т Тво\'{и}х, и м\'{и}лость неизреч\'{е}нна незл\'{о}бивыя Твое\'{я} бл\'{а}гости, и несть грех побежд\'{а}ющ человекол\'{ю}бие Тво\'{е}. Т\'{е}мже, преч\'{у}дный Цар\'{ю}, незл\'{о}биве Г\'{о}споди, удив\'{и} и на мне, гр\'{е}шнем, м\'{и}лости Тво\'{я}, покаж\'{и} бл\'{а}гости Твое\'{я} с\'{и}лу и яв\'{и} кр\'{е}пость благоутр\'{о}бнаго милос\'{е}рдия Твоег\'{о}, и обращ\'{а}ющася приим\'{и} мя гр\'{е}шнаго. Приим\'{и} мя, \'{я}коже при\'{я}л ес\'{и} бл\'{у}днаго, разб\'{о}йника, блудн\'{и}цу. Приим\'{и} мя, пребезм\'{е}рне и сл\'{о}вом, и д\'{е}лом, и п\'{о}хотию безм\'{е}стною, и помышл\'{е}нием безслов\'{е}сным согреш\'{и}вша Теб\'{е}. И \'{я}коже во единонадес\'{я}тый час приш\'{е}дших при\'{я}л ес\'{и}, ничт\'{о}же дост\'{о}йно сод\'{е}лавших, т\'{а}ко приим\'{и} и мен\'{е}, гр\'{е}шнаго: мн\'{о}го бо согреш\'{и}х и оскверн\'{и}хся, и опеч\'{а}лих Д\'{у}ха Твоег\'{о} Свят\'{а}го, и огорч\'{и}х человекол\'{ю}бную утр\'{о}бу Тво\'{ю} и д\'{е}лом, и сл\'{о}вом, и помышл\'{е}нием, в нощ\'{и} и во дни, явл\'{е}нне же и неявл\'{е}нне, в\'{о}лею же и нев\'{о}лею. И вем, \'{я}ко предст\'{а}виши грех\'{и} мо\'{я} пр\'{е}до мн\'{о}ю таков\'{ы}, \'{я}ковы же мн\'{о}ю сод\'{е}яшася, и ист\'{я}жеши сл\'{о}во со мн\'{о}ю о \'{и}хже р\'{а}зумом непрощ\'{е}нно согреш\'{и}х. Но Г\'{о}споди, Г\'{о}споди, да не пр\'{а}ведным суд\'{о}м Тво\'{и}м, ниж\'{е} \'{я}ростию Тво\'{е}ю облич\'{и}ши мя, ниж\'{е} гн\'{е}вом Тво\'{и}м нак\'{а}жеши мя; пом\'{и}луй мя, Г\'{о}споди, \'{я}ко не т\'{о}кмо н\'{е}мощен \'{е}смь, но и Тво\'{е} есмь созд\'{а}ние. Ты \'{у}бо, Г\'{о}споди, утверд\'{и}л ес\'{и} на мне страх Твой, аз же лук\'{а}вое пред Тоб\'{о}ю сотвор\'{и}х. Теб\'{е} \'{у}бо ед\'{и}ному согреш\'{и}х, но мол\'{ю} Тя, не вн\'{и}ди в суд с раб\'{о}м Тво\'{и}м. Аще бо беззак\'{о}ния н\'{а}зриши, Г\'{о}споди, Г\'{о}споди, кто посто\'{и}т? Аз бо есмь пуч\'{и}на грех\'{а}, и несмь дост\'{о}ин, ниж\'{е} дов\'{о}лен воззр\'{е}ти и в\'{и}дети высот\'{у} неб\'{е}сную, от мн\'{о}жества грех\'{о}в мо\'{и}х, \'{и}хже несть числ\'{а}: вс\'{я}кое бо злоде\'{я}ние и ков\'{а}рство, и ухищр\'{е}ние сатанин\'{о}, и растл\'{е}ния, злопомн\'{е}ния, сов\'{е}тования ко грех\'{у} и ин\'{ы}е тьмоч\'{и}сленныя стр\'{а}сти не оскуд\'{е}ша от мен\'{е}. К\'{и}ими бо не растл\'{и}хся грех\'{и}? К\'{и}ими не содерж\'{а}хся зл\'{ы}ми? Всяк грех сод\'{е}ях, вс\'{я}кую нечистот\'{у} влож\'{и}х в д\'{у}шу мо\'{ю}, непотр\'{е}бен бых Теб\'{е}, Б\'{о}гу моем\'{у}, и челов\'{е}ком. Кто возст\'{а}вит мя, в сицев\'{а}я зл\'{а}я и тол\'{и}ка п\'{а}дшаго согреш\'{е}ния? Г\'{о}споди Б\'{о}же мой, на Тя упов\'{а}х; \'{а}ще есть ми спас\'{е}ния упов\'{а}ние, \'{а}ще побежд\'{а}ет человекол\'{ю}бие Тво\'{е} мн\'{о}жества беззак\'{о}ний мо\'{и}х, б\'{у}ди ми спас\'{и}тель, и по щедр\'{о}там Тво\'{и}м и м\'{и}лостем Тво\'{и}м, осл\'{а}би, ост\'{а}ви, прост\'{и} ми вся, ел\'{и}ка Ти согреш\'{и}х, яко мн\'{о}гих зол исп\'{о}лнися душ\'{а} мо\'{я}, и несть во мне спас\'{е}ния над\'{е}жды. Пом\'{и}луй мя, Б\'{о}же, по вел\'{и}цей м\'{и}лости Тво\'{е}й и не возд\'{а}ждь ми по дел\'{о}м мо\'{и}м, и не осуд\'{и} мя по де\'{я}ниям мо\'{и}м, но обрат\'{и}, заступ\'{и}, изб\'{а}ви д\'{у}шу мою от совозраст\'{а}ющих ей зол и л\'{ю}тых воспри\'{я}тий. Спас\'{и} мя р\'{а}ди м\'{и}лости Твое\'{я}, да ид\'{е}же умн\'{о}жится грех, преизоб\'{и}лует благод\'{а}ть Тво\'{я}; и восхвал\'{ю} и просл\'{а}влю Тя всегд\'{а}, вся дни живот\'{а} моег\'{о}. Ты бо ес\'{и} Бог к\'{а}ющихся и Спас согреш\'{а}ющих; и Теб\'{е} сл\'{а}ву возсыл\'{а}ем со Безнач\'{а}льным Тво\'{и}м Отц\'{е}м и Пресвят\'{ы}м и Благ\'{и}м, и Животвор\'{я}щим Тво\'{и}м Д\'{у}хом н\'{ы}не и пр\'{и}сно, и во в\'{е}ки век\'{о}в. Ам\'{и}нь.

\mysubtitle{Молитва 5-я, святого Иоанна Дамаскина}

Влад\'{ы}ко Г\'{о}споди Иис\'{у}се Христ\'{е}, Б\'{о}же наш, ед\'{и}не им\'{е}яй власть челов\'{е}ком оставл\'{я}ти грех\'{и}, \'{я}ко благ и Человекол\'{ю}бец пр\'{е}зри мо\'{я} вся в в\'{е}дении и не в в\'{е}дении прегреш\'{е}ния, и спод\'{о}би мя неосужд\'{е}нно причаст\'{и}тися Бож\'{е}ственных, и пресл\'{а}вных, и преч\'{и}стых, и животвор\'{я}щих Тво\'{и}х Т\'{а}ин, не в т\'{я}жесть, ни в м\'{у}ку, ни в прилож\'{е}ние грех\'{о}в, но во очищ\'{е}ние, и освящ\'{е}ние, и обруч\'{е}ние б\'{у}дущаго Живота и ц\'{а}рствия, в ст\'{е}ну и п\'{о}мощь, и в возраж\'{е}ние сопрот\'{и}вных, во истребл\'{е}ние мн\'{о}гих мо\'{и}х согреш\'{е}ний. Ты бо ес\'{и} Бог м\'{и}лости, и щедр\'{о}т, и человекол\'{ю}бия, и Теб\'{е} сл\'{а}ву возсыл\'{а}ем, со Отц\'{е}м, и Свят\'{ы}м Д\'{у}хом, н\'{ы}не и пр\'{и}сно, и во в\'{е}ки век\'{о}в. Ам\'{и}нь.

\mysubtitle{Молитва 6-я, святого Василия Великого}

Вем, Г\'{о}споди, \'{я}ко нед\'{о}стойне причащ\'{а}юся преч\'{и}стаго Тво\'{е}го Т\'{е}ла и честн\'{ы}я Твое\'{я} Кр\'{о}ве, и пов\'{и}нен есмь, и суд себ\'{е} ям и пи\'{ю}, не разсужд\'{а}я Т\'{е}ла и Кр\'{о}ве Теб\'{е} Христ\'{а} и Б\'{о}га моег\'{о}, но на щедр\'{о}ты Тво\'{я} дерз\'{а}я прихожд\'{у} к Теб\'{е} р\'{е}кшему: яд\'{ы}й Мо\'{ю} плоть, и пи\'{я}й Мо\'{ю} кровь, во Мне пребыв\'{а}ет, и Аз в нем. Умилос\'{е}рдися \'{у}бо, Г\'{о}споди, и не облич\'{и} мя гр\'{е}шнаго, но сотвор\'{и} со мн\'{о}ю по м\'{и}лости Тво\'{е}й; и да б\'{у}дут ми св\'{я}тая си\'{я} во исцел\'{е}ние, и очищ\'{е}ние, и просвещ\'{е}ние, и сохран\'{е}ние, и спас\'{е}ние, и во освящ\'{е}ние душ\'{и} и т\'{е}ла; во отгн\'{а}ние вс\'{я}каго мечт\'{а}ния, и лук\'{а}ваго де\'{я}ния, и д\'{е}йства ди\'{а}вольскаго, м\'{ы}сленнe во удес\'{е}х мо\'{и}х д\'{е}йствуемаго, в дерзнов\'{е}ние и люб\'{о}вь, \'{я}же к Теб\'{е}; во исправл\'{е}ние жити\'{я} и утвержд\'{е}ние, в возращ\'{е}ние доброд\'{е}тели и соверш\'{е}нства; во исполн\'{е}ние з\'{а}поведей, в Д\'{у}ха Свят\'{а}го общ\'{е}ние, в нап\'{у}тие живот\'{а} в\'{е}чнаго, во отв\'{е}т благопри\'{я}тен на Стр\'{а}шнем суд\'{и}щи Тво\'{е}м: не в суд ил\'{и} во осужд\'{е}ние.

\mysubtitle{Молитва 7-я,святого Симеона Нового Богослова}

От скв\'{е}рных уст\'{е}н, от м\'{е}рзкаго с\'{е}рдца, от неч\'{и}стаго яз\'{ы}ка, от душ\'{и} оскверн\'{е}ны, приим\'{и} мол\'{е}ние, Христ\'{е} мой, и не пр\'{е}зри мо\'{и}х ни слов\'{е}с, ниж\'{е} образ\'{о}в, ниж\'{е} безст\'{у}дия. Даждь ми дерзнов\'{е}нно глаг\'{о}лати, \'{я}же хощ\'{у}, Христ\'{е} мой, п\'{а}че же и науч\'{и} мя, что ми подоб\'{а}ет твор\'{и}ти и глаг\'{о}лати. Согреш\'{и}х п\'{а}че блудн\'{и}цы, \'{я}же ув\'{е}де, где обит\'{а}еши, м\'{и}ро куп\'{и}вши, при\'{и}де д\'{е}рзостне пом\'{а}зати Тво\'{и} н\'{о}зе, Б\'{о}га моег\'{о}, Влад\'{ы}ки и Христ\'{а} моег\'{о}. \'{Я}коже \'{о}ну не отр\'{и}нул ес\'{и} приш\'{е}дшую от с\'{е}рдца, ниж\'{е} мен\'{е} возгнуш\'{а}йся, Сл\'{о}ве: Тво\'{и} же ми под\'{а}ждь н\'{о}зе, и держ\'{а}ти и целов\'{а}ти, и стру\'{я}ми сл\'{е}зными, \'{я}ко многоц\'{е}нным м\'{и}ром, си\'{я} д\'{е}рзостно пом\'{а}зати. Ом\'{ы}й мя слез\'{а}ми мо\'{и}ми, оч\'{и}сти мя \'{и}ми, Сл\'{о}ве. Ост\'{а}ви и прегреш\'{е}ния мо\'{я}, и прощ\'{е}ние ми под\'{а}ждь. В\'{е}си зол мн\'{о}жество, в\'{е}си и стр\'{у}пы мо\'{я}, и \'{я}звы зр\'{и}ши мо\'{я}, но и в\'{е}ру в\'{е}си, и произвол\'{е}ние зр\'{и}ши, и воздых\'{а}ние сл\'{ы}шиши. Не та\'{и}тся Теб\'{е}, Б\'{о}же мой, Тв\'{о}рче мой, Изб\'{а}вителю мой, ниж\'{е} к\'{а}пля сл\'{е}зная, ниж\'{е} к\'{а}пли часть н\'{е}кая. Несод\'{е}ланное мо\'{е} в\'{и}десте \'{о}чи Тво\'{и}, в кн\'{и}зе же Тво\'{е}й и ещ\'{е} несод\'{е}янная нап\'{и}сана Теб\'{е} суть. Виждь смир\'{е}ние мо\'{е}, виждь труд мой ел\'{и}к, и грех\'{и} вся ост\'{а}ви ми, Б\'{о}же вс\'{я}ческих: да ч\'{и}стым с\'{е}рдцем, притр\'{е}петною м\'{ы}слию, и душ\'{е}ю сокруш\'{е}нною, нескв\'{е}рных Тво\'{и}х причащ\'{у}ся и пресвят\'{ы}х Т\'{а}ин, \'{и}миже оживл\'{я}ется и обож\'{а}ется всяк яд\'{ы}й же и пи\'{я}й ч\'{и}стым с\'{е}рдцем; Ты бо рекл ес\'{и}, Влад\'{ы}ко мой: всяк яд\'{ы}й Мо\'{ю} Плоть, и пи\'{я}й Мо\'{ю} Кровь, во Мне \'{у}бо сей пребыв\'{а}ет, в н\'{е}мже и Аз есмь. Истинно сл\'{о}во вс\'{я}ко Влад\'{ы}ки и Б\'{о}га моег\'{о}: бож\'{е}ственных бо причащ\'{а}яйся и боготвор\'{я}щих благод\'{а}тей, не \'{у}бо есмь ед\'{и}н, но с Тоб\'{о}ю, Христ\'{е} мой, Св\'{е}том трис\'{о}лнечным, просвещ\'{а}ющим мир. Да \'{у}бо не ед\'{и}н преб\'{у}ду кром\'{е} Теб\'{е} Живод\'{а}вца, дых\'{а}ния моег\'{о}, живот\'{а} моег\'{о}, р\'{а}дования моег\'{о}, спас\'{е}ния м\'{и}ру. Сег\'{о} р\'{а}ди к Теб\'{е} приступ\'{и}х, \'{я}коже зр\'{и}ши, со слез\'{а}ми, и душ\'{е}ю сокруш\'{е}нною избавл\'{е}ния мо\'{и}х прегреш\'{е}ний прош\'{у} при\'{я}ти ми, и Тво\'{и}х живод\'{а}тельных и непор\'{о}чных Т\'{а}инств причаст\'{и}тися неосужд\'{е}нно, да преб\'{у}деши, \'{я}коже рекл ес\'{и}, со мн\'{о}ю треока\'{я}нным: да не кром\'{е} обр\'{е}т мя Твое\'{я} благод\'{а}ти, прел\'{е}стник восх\'{и}тит мя льст\'{и}вне, и прельст\'{и}в отвед\'{е}т боготвор\'{я}щих Тво\'{и}х слов\'{е}с. Сег\'{о} р\'{а}ди к Теб\'{е} прип\'{а}даю, и т\'{е}пле вопи\'{ю} Ти: \'{я}коже бл\'{у}днаго при\'{я}л ес\'{и}, и блудн\'{и}цу приш\'{е}дшую, т\'{а}ко приим\'{и} мя бл\'{у}днаго и скв\'{е}рнаго, Щ\'{е}дре. Душ\'{е}ю сокруш\'{е}нною, н\'{ы}не бо к Теб\'{е} приход\'{я}, вем, Сп\'{а}се, \'{я}ко ин\'{ы}й, \'{я}коже аз, не прегреш\'{и} Теб\'{е}, ниж\'{е} сод\'{е}я де\'{я}ния, \'{я}же аз сод\'{е}ях. Но си\'{е} п\'{а}ки вем, \'{я}ко не вел\'{и}чество прегреш\'{е}ний, ни грех\'{о}в мн\'{о}жество превосх\'{о}дит Б\'{о}га моег\'{о} мн\'{о}гое долготерп\'{е}ние, и человекол\'{ю}бие кр\'{а}йнее; но м\'{и}лостию состр\'{а}стия т\'{е}пле к\'{а}ющияся, и ч\'{и}стиши, и св\'{е}тлиши, и св\'{е}та твор\'{и}ши прич\'{а}стники, \'{о}бщники Божеств\'{а} Тво\'{е}го сод\'{е}ловаяй незав\'{и}стно, и стр\'{а}нное и Ангелом, и челов\'{е}ческим м\'{ы}слем, бес\'{е}дуеши им мн\'{о}гажды, \'{я}коже друг\'{о}м Тво\'{и}м \'{и}стинным. Си\'{я} д\'{е}рзостна твор\'{я}т мя, си\'{я} впер\'{я}ют мя, Христ\'{е} мой. И дерз\'{а}я Тво\'{и}м бог\'{а}тым к нам благоде\'{я}нием, р\'{а}дуяся вк\'{у}пе и треп\'{е}ща, огн\'{е}ви причащ\'{а}юся трав\'{а} сый, и стр\'{а}нно ч\'{у}до, орош\'{а}ем неоп\'{а}льно, \'{я}коже \'{у}бо купин\'{а} др\'{е}вле неоп\'{а}льне гор\'{я}щи. Н\'{ы}не благод\'{а}рною м\'{ы}слию, благод\'{а}рным же с\'{е}рдцем, благод\'{а}рными удес\'{ы} мо\'{и}ми, душ\'{и} и т\'{е}ла моег\'{о}, поклан\'{я}юся и велич\'{а}ю, и славосл\'{о}влю Тя, Б\'{о}же мой, \'{я}ко благослов\'{е}нна с\'{у}ща, н\'{ы}не же и во в\'{е}ки.

\mysubtitle{Молитва 8-я, святого Иоанна Златоустого}

Б\'{о}же, осл\'{а}би, ост\'{а}ви, прост\'{и} ми согреш\'{е}ния мо\'{я}, ел\'{и}ка Ти согреш\'{и}х, \'{а}ще сл\'{о}вом, \'{а}ще д\'{е}лом, \'{а}ще помышл\'{е}нием, в\'{о}лею ил\'{и} нев\'{о}лею, р\'{а}зумом ил\'{и} нераз\'{у}мием, вся ми прост\'{и} \'{я}ко благ и Человекол\'{ю}бец, и мол\'{и}твами Преч\'{и}стыя Твое\'{я} М\'{а}тере, \'{у}мных Тво\'{и}х служ\'{и}телей и свят\'{ы}х сил, и всех свят\'{ы}х от в\'{е}ка Теб\'{е} благоугод\'{и}вших, неосужд\'{е}нно благовол\'{и} при\'{я}ти ми свят\'{о}е и преч\'{и}стое Тво\'{е} Т\'{е}ло и честн\'{у}ю Кровь, во исцел\'{е}ние душ\'{и} же и т\'{е}ла, и во очищ\'{е}ние лук\'{а}вых мо\'{и}х помышл\'{е}ний. \'{Я}ко Тво\'{е} \'{е}сть ц\'{а}рство и с\'{и}ла и сл\'{а}ва, со Отц\'{е}м и Свят\'{ы}м Д\'{у}хом, н\'{ы}не и пр\'{и}сно, и во в\'{е}ки век\'{о}в. Ам\'{и}нь.

\mysubtitle{Его же, 9-я}

Несмь дов\'{о}лен, Влад\'{ы}ко Г\'{о}споди, да вн\'{и}деши под кров душ\'{и} мое\'{я}; но пон\'{е}же х\'{о}щеши Ты, \'{я}ко Человекол\'{ю}бец, ж\'{и}ти во мне, дерз\'{а}я приступ\'{а}ю; повелев\'{а}еши, да отв\'{е}рзу дв\'{е}ри, \'{я}же Ты ед\'{и}н созд\'{а}л ес\'{и}, и вн\'{и}деши со человекол\'{ю}бием \'{я}коже ес\'{и}, вн\'{и}деши и просвещ\'{а}еши помрач\'{е}нный мой п\'{о}мысл. В\'{е}рую, \'{я}ко си\'{е} сотвор\'{и}ши: не бо блудн\'{и}цу, со слез\'{а}ми приш\'{е}дшую к Теб\'{е}, отгн\'{а}л ес\'{и}; ниж\'{е} мытар\'{я} отв\'{е}ргл ес\'{и} пок\'{а}явшася; ниж\'{е} разб\'{о}йника, позн\'{а}вша Ц\'{а}рство Тво\'{е}, отгн\'{а}л ес\'{и}; ниж\'{е} гон\'{и}теля пок\'{а}явшася ост\'{а}вил ес\'{и}, \'{е}же бе: но от пока\'{я}ния Теб\'{е} приш\'{е}дшия вся, в л\'{и}це Тво\'{и}х друг\'{о}в вчин\'{и}л ес\'{и}, Ед\'{и}н сый благослов\'{е}нный всегд\'{а}, н\'{ы}не и в безкон\'{е}чныя в\'{е}ки. Ам\'{и}нь.

\mysubtitle{Его же, 10-я}

Г\'{о}споди Иис\'{у}се Христ\'{е} Б\'{о}же мой, осл\'{а}би, ост\'{а}ви, оч\'{и}сти и прост\'{и} ми гр\'{е}шному, и непотр\'{е}бному, и недост\'{о}йному раб\'{у} Твоем\'{у}, прегреш\'{е}ния, и согреш\'{е}ния, и грехопад\'{е}ния мо\'{я}, ел\'{и}ка Ти от \'{ю}ности мое\'{я}, д\'{а}же до насто\'{я}щего дне и час\'{а} согреш\'{и}х: \'{а}ще в р\'{а}зуме и в нераз\'{у}мии, \'{а}ще в словес\'{е}х ил\'{и} д\'{е}лех, ил\'{и} помышл\'{е}ниих и м\'{ы}слех, и начин\'{а}ниих, и всех мо\'{и}х ч\'{у}вствах. И мол\'{и}твами безс\'{е}менно р\'{о}ждшия Тя Преч\'{и}стыя и Приснод\'{е}вы Мар\'{и}и, М\'{а}тере Твое\'{я}, ед\'{и}ныя непост\'{ы}дныя над\'{е}жды и предст\'{а}тельства и спас\'{е}ния моег\'{о}, спод\'{о}би мя неосужд\'{е}нно причаст\'{и}тися преч\'{и}стых, безсм\'{е}ртных, животвор\'{я}щих и стр\'{а}шных Тво\'{и}х Т\'{а}инств, во оставл\'{е}ние грех\'{о}в и в жизнь в\'{е}чную: во освящ\'{е}ние, и просвещ\'{е}ние, кр\'{е}пость, исцел\'{е}ние, и здр\'{а}вие душ\'{и} же и т\'{е}ла, и в потребл\'{е}ние и всесоверш\'{е}нное погубл\'{е}ние лук\'{а}вых мо\'{и}х помысл\'{о}в, и помышл\'{е}ний, и предпри\'{я}тий, и нощн\'{ы}х мечт\'{а}ний, т\'{е}мных и лук\'{а}вых дух\'{о}в; \'{я}ко Тво\'{е} \'{е}сть ц\'{а}рство, и с\'{и}ла, и сл\'{а}ва, и честь, и поклон\'{е}ние, со Отц\'{е}м и Свят\'{ы}м Тво\'{и}м Д\'{у}хом, н\'{ы}не и пр\'{и}сно, и во в\'{е}ки век\'{о}в. Ам\'{и}нь.

\mysubtitle{Молитва 11-я, святого Иоанна Дамаскина}

Пред дв\'{е}рьми хр\'{а}ма Тво\'{е}го предсто\'{ю} и л\'{ю}тых помышл\'{е}ний не отступ\'{а}ю; но Ты, Христ\'{е} Б\'{о}же, мытар\'{я} оправд\'{и}вый, и ханан\'{е}ю пом\'{и}ловавый, и разб\'{о}йнику ра\'{я} дв\'{е}ри отв\'{е}рзый, отв\'{е}рзи ми утр\'{о}бы человекол\'{ю}бия Тво\'{е}го и приим\'{и} мя приход\'{я}ща и прикас\'{а}ющася Теб\'{е}, \'{я}ко блудн\'{и}цу, и кровоточ\'{и}вую: \'{о}ва \'{у}бо кр\'{а}я р\'{и}зы Твое\'{я} косн\'{у}вшися, уд\'{о}бь исцел\'{е}ние при\'{я}т, \'{о}ва же преч\'{и}стеи Тво\'{и} н\'{о}зе удерж\'{а}вши, разреш\'{е}ние грех\'{о}в понес\'{е}. Аз же, ока\'{я}нный, все Тво\'{е} Т\'{е}ло дерз\'{а}я воспри\'{я}ти, да не опал\'{е}н б\'{у}ду; но приим\'{и} мя, \'{я}коже \'{о}ныя, и просвет\'{и} мо\'{я} душ\'{е}вныя ч\'{у}вства, попал\'{я}я мо\'{я} грех\'{о}вныя вин\'{ы}, мол\'{и}твами безс\'{е}менно \'{Р}ождшия Тя, и Неб\'{е}сных сил; \'{я}ко благослов\'{е}н ес\'{и} во в\'{е}ки век\'{о}в. Ам\'{и}нь.

\mysubtitle{Молитва святого Иоанна Златоустого}

В\'{е}рую, Г\'{о}споди, и испов\'{е}дую, \'{я}ко Ты ес\'{и} во\'{и}стинну Христ\'{о}с, Сын Б\'{о}га жив\'{а}го, приш\'{е}дый в мир гр\'{е}шныя спаст\'{и}, от н\'{и}хже п\'{е}рвый есмь аз. Ещ\'{е} в\'{е}рую, \'{я}ко си\'{е} \'{е}сть с\'{а}мое преч\'{и}стое Т\'{е}ло Тво\'{е}, и си\'{я} с\'{а}мая \'{е}сть честн\'{а}я Кровь Тво\'{я}. Мол\'{ю}ся \'{у}бо Теб\'{е}: пом\'{и}луй мя, и прост\'{и} ми прегреш\'{е}ния мо\'{я}, в\'{о}льная и нев\'{о}льная, \'{я}же сл\'{о}вом, \'{я}же д\'{е}лом, \'{я}же в\'{е}дением и нев\'{е}дением, и спод\'{о}би мя неосужд\'{е}нно причаст\'{и}тися преч\'{и}стых Тво\'{и}х Т\'{а}инств, во оставл\'{е}ние грех\'{о}в, и в жизнь в\'{е}чную. Ам\'{и}нь.

\mysubtitle{Приходя же причаститься, произноси мысленно эти стихи Метафраста:}

\begin{verse}
Се приступ\'{а}ю к Бож\'{е}ственному Причащ\'{е}нию.

Сод\'{е}телю, да не опал\'{и}ши мя приобщ\'{е}нием:

Огнь бо ес\'{и}, недост\'{о}йныя попал\'{я}яй.

Но \'{у}бо оч\'{и}сти мя от вс\'{я}кия скв\'{е}рны.
\end{verse}

\mysubtitle{Затем:}

В\'{е}чери Твое\'{я} т\'{а}йныя днесь, С\'{ы}не Б\'{о}жий, прич\'{а}стника мя приим\'{и}; не бо враг\'{о}м Тво\'{и}м т\'{а}йну пов\'{е}м, ни лобз\'{а}ния Ти дам, \'{я}ко И\'{у}да, но \'{я}ко разб\'{о}йник испов\'{е}даю Тя: помян\'{и} мя, Г\'{о}споди, во Ц\'{а}рствии Тво\'{е}м.

\mysubtitle{И стихи:}

\begin{verse}
Боготвор\'{я}щую Кровь ужасн\'{и}ся, челов\'{е}че, зря:

Огнь бо \'{е}сть, недост\'{о}йныя попал\'{я}яй.

Бож\'{е}ственное Т\'{е}ло и обож\'{а}ет мя и пит\'{а}ет:

Обож\'{а}ет дух, ум же пит\'{а}ет стр\'{а}нно.
\end{verse}

\mysubtitle{Потом тропари:}

Услад\'{и}л мя ес\'{и} люб\'{о}вию, Христ\'{е}, и измен\'{и}л мя ес\'{и} Бож\'{е}ственным Тво\'{и}м рач\'{е}нием; но попал\'{и} огн\'{е}м невещ\'{е}ственным грех\'{и} мо\'{я}, и нас\'{ы}титися \'{е}же в Теб\'{е} наслажд\'{е}ния спод\'{о}би: да лик\'{у}я возвелич\'{а}ю, Бл\'{а}же, два приш\'{е}ствия Тво\'{я}.

Во св\'{е}тлостех Свят\'{ы}х Тво\'{и}х к\'{а}ко вн\'{и}ду недост\'{о}йный? \'{А}ще бо дерзн\'{у} совн\'{и}ти в черт\'{о}г, од\'{е}жда мя облич\'{а}ет, \'{я}ко несть бр\'{а}чна, и св\'{я}зан изв\'{е}ржен б\'{у}ду от \'{А}нгелов. Оч\'{и}сти, Г\'{о}споди, скв\'{е}рну душ\'{и} мое\'{я}, и спас\'{и} мя, \'{я}ко Человекол\'{ю}бец.

\mysubtitle{Также молитву:}

Влад\'{ы}ко Человекол\'{ю}бче, Г\'{о}споди Иис\'{у}се Христ\'{е} Б\'{о}же мой, да не в суд ми б\'{у}дут Свят\'{а}я си\'{я}, за \'{е}же недост\'{о}йну ми б\'{ы}ти: но во очищ\'{е}ние и освящ\'{е}ние душ\'{и} же и т\'{е}ла, и во обруч\'{е}ние б\'{у}дущия ж\'{и}зни и ц\'{а}рствия. Мне же, \'{е}же прилепл\'{я}тися Б\'{о}гу, бл\'{а}го \'{е}сть, полаг\'{а}ти во Г\'{о}споде упов\'{а}ние спас\'{е}ния моег\'{о}.

\mysubtitle{И еще:}

В\'{е}чери Твое\'{я} т\'{а}йныя днесь, С\'{ы}не Б\'{о}жий, прич\'{а}стника мя приим\'{и}; не бо враг\'{о}м Тво\'{и}м т\'{а}йну пов\'{е}м, ни лобз\'{а}ния Ти дам, \'{я}ко И\'{у}да, но \'{я}ко разб\'{о}йник испов\'{е}даю Тя: помян\'{и} мя, Г\'{о}споди, во Ц\'{а}рствии Тво\'{е}м.

\end{mymulticols}

\myparsep[0.25]

\medskip Желающий причаститься должен достойно приготовится к этому святому таинству. Приготовление это (в церковной практике оно называется говением) продолжается несколько дней и касается как телесной, так и духовной жизни человека. Телу предписывается воздержание, т.~е. телесная чистота (воздержание от супружеских отношений) и ограничение в пищи (пост). В дни поста исключается пища животного происхождения "--- мясо, молоко, яйца и, про строгом посте, рыба. Хлеб, овощи, фрукты употребляются в умеренном количестве. Ум не должен рассеиваться по мелочам житейским и развлекаться.

В дни говения надлежит посещать богослужения в храме, если позволят обстоятельства, и более прилежно выполнять домашнее молитвенное правило: кто читает обычно не все утренние и вечерние молитвы, пусть читает все полностью, кто не читает каноны, пусть в эти дни читает хотя бы по одному канону. Накануне причащения надо быть на вечернем богослужении и прочитать дома, кроме обычных молитв на сон грядущим, канон покаянный, канон Богородице и Ангелу хранителю. Каноны читают или один за другим полностью, или соединяя таким образом: читается ирмос первой песни покаянного канона («Яко по суху петешествовав Израиль, по бездне стопами, гонителя фараона видя потопляема, Богу победную песнь поим, вопияше») и тропари, затем тропари первой песни канона Богородице («Многими содержимь напaстьми, к Тебе прибегаю, спасения иский: о, Мaти Слова и Дево, от тяжких и лютых мя спаси»), опуская ирмос «Воду прошед…», и тропари канона Ангелу хранителю, тоже без ирмоса («Поим Господеви, проведшему люди Своя сквозе Чермное море, яко един славно прославися»). Так же читают и следующие песни. Тропари перед каноном Богородице и Ангелу хранителю, а также стихиры после канона Богородице в таком случае опускаются.

Читается также канон ко причащению и, кто пожелает,"--- акафист Иисусу Сладчайшему. После полуночи уже не едят и не пьют, ибо принято приступать к Таинству Причащения натощак. Утром прочитываются утренние молитвы и все последование ко Святому Причащению, кроме канона, прочитанного накануне.

Перед причащением необходима исповедь "--- вечером ли, или утром, перед литургией.

\mychapterending

\mychapter{Акафист Иисусу Сладчайшему}\begin{mymulticols}
%http://www.molitvoslov.org/text11.htm

\myfigure{B-2757_01-IV}

\mysubtitle{Кондак 1}

Возбранный Воеводо и Господи, ада победителю, яко избавлься от вечныя смерти, похвальная восписую Ти, создание и раб Твой; но, яко имеяй милосердие неизреченное, от всяких мя бед свободи, зовуща: Иисусе, Сыне Божий, помилуй мя.

\mysubtitle{Икос 1}


Ангелов
Творче и Господи cил, отверзи ми недоуменный ум и язык на похвалу пречистаго Твоего имене, якоже глухому и гугнивому древле слух и язык отверзл еси, и, глаголаше зовый таковая: Иисусе пречудный, aнгелов удивление; Иисусе пресильный, прародителей избавление. Иисусе пресладкий, патриархов величание; Иисусе преславный, царей укрепление. Иисусе прелюбимый, пророков исполнение; Иисусе предивный, мучеников крепосте. Иисусе претихий, монахов радосте; Иисусе премилостивый,
пресвитеров сладосте. Иисусе премилосердый, постников воздержание; Иисусе пресладостный, преподобных радование. Иисусе пречестный, девственных целомудрие; Иисусе предвечный, грешников спасение. Иисусе, Сыне Божий, помилуй мя.

\mysubtitle{Кондак 2}

Видя
вдовицу зельне плачущу, Господи, якоже бо тогда умилосердився, сына ея на погребение несома воскресил еси; сице и о мне умилосердися, Человеколюбче, и грехми умерщвленную мою душу воскреси, зовущую: Аллилуиа.

\mysubtitle{Икос 2}

Разум
неуразуменный разумети Филипп ища, Господи, покажи нам Отца, глаголаше; Ты же к нему: толикое время сый со Мною, не познал ли еси, яко Отец во Мне, и Аз во Отце есмь? Темже, Неизследованне, со страхом зову Ти: Иисусе, Боже предвечный; Иисусе, Царю пресильный. Иисусе, Владыко долготерпеливый; Иисусе, Спасе премилостивый. Иисусе, хранителю мой преблагий; Иисусе, очисти грехи моя. Иисусе, отыми беззакония моя; Иисусе, отпусти неправды моя. Иисусе, надеждо моя, не остави мене; Иисусе, помощниче мой, не отрини мене. Иисусе, Создателю мой, не забуди
мене; Иисусе, Пастырю мой, не погуби мене. Иисусе, Сыне Божий, помилуй мя.

\mysubtitle{Кондак 3}

Силою
свыше апостолы облекий, Иисусе, во Иерусалиме седящия, облецы и мене, обнаженнаго от всякаго благотворения, теплотою Духа Святаго Твоего и даждь ми с любвью пети Тебе: Аллилуиа.

\mysubtitle{Икос 3}

Имеяй
богатство милосердия, мытари и грешники, и неверныя призвал еси, Иисусе; не презри и мене ныне, подобнаго им, но, яко многоценное миро, приими песнь сию: Иисусе, сило непобедимая; Иисусе, милосте
безконечная. Иисусе, красото пресветлая; Иисусе, любы неизреченная. Иисусе, Сыне Бога Живаго; Иисусе, помилуй мя грешнаго. Иисусе, услыши мя в беззакониих зачатаго; Иисусе, очисти мя во гресех рожденнаго. Иисусе, научи мя непотребнаго; Иисусе, освети мя темнаго. Иисусе, очисти мя сквернаго; Иисусе, возведи мя блуднаго. Иисусе, Сыне Божий, помилуй мя.

\mysubtitle{Кондак 4}

Бурю
внутрь имеяй помышлений сумнительных, Петр утопаше; узрев же во плоти Тя суща, Иисусе, и по водам ходяща, позна Тя Бога истиннаго и, руку спасения получив, рече: Аллилуиа.

\mysubtitle{Икос 4}

Слыша
слепый мимоходяща Тя, Господи, путем вопияше: Иисусе, Сыне Давидов, помилуй мя! И, призвав, отверзл еси очи его. Просвети убо милостию Твоею очи мысленныя сердца и мене, вопиюща Ти и глаголюща: Иисусе, вышних Создателю; Иисусе, нижних Искупителю. Иисусе, преисподних потребителю; Иисусе, всея твари украсителю. Иисусе, души моея утешителю; Иисусе, ума моего просветителю. Иисусе, сердца моего веселие; Иисусе, тела моего здравие. Иисусе, Спасе мой, спаси мя; Иисусе, свете мой, просвети мя. Иисусе, муки всякия избави мя; Иисусе, спаси мя, недостойнаго. Иисусе, Сыне Божий, помилуй мя.

\mysubtitle{Кондак 5}

Боготочною
Кровию якоже искупил еси нас древле от законныя клятвы, Иисусе, сице изми нас от сети, еюже змий запят ны страстьми плотскими, и блудным наваждением, и злым унынием, вопиющия Ти: Аллилуиа.

\mysubtitle{Икос 5}

Видевше
отроцы еврейстии во образе человечестем Создавшаго рукою человека, и Владыку разумевше Его, потщашася ветвьми угодити Ему, осанна вопиюще. Мы же песнь приносим Ти, глаголюще: Иисусе, Боже истинный; Иисусе, Сыне Давидов. Иисусе, Царю преславный; Иисусе, Агнче непорочный. Иисусе, Пастырю предивный; Иисусе, хранителю во младости моей. Иисусе, кормителю во юности моей; Иисусе, похвало в старости моей. Иисусе, надежде в смерти моей; Иисусе, животе по смерти моей. Иисусе, утешение мое на суде Твоем; Иисусе, желание мое, не посрами мене тогда. Иисусе, Сыне Божий, помилуй мя.

\mysubtitle{Кондак 6}

Проповедник
богоносных вещание и глаголы исполняя, Иисусе, на земли явлься и с человеки Невместимый пожил еси, и болезни наша подъял еси, отнюдуже ранами Твоими мы исцелевше, пети навыкохом: Аллилуиа.

\mysubtitle{Икос 6}

Возсия
вселенней просвещение истины Твоея, и отгнася лесть бесовская: идоли бо, Спасе наш, не терпяще Твоея крепости, падоша. Мы же, спасение получивше, вопием Ти: Иисусе, истино, лесть отгонящая; Иисусе, свете, превышший всех светлостей. Иисусе, Царю, премогаяй всех крепости; Иисусе, Боже, пребываяй в милости. Иисусе, Хлебе Животный, насыти мя алчущаго; Иисусе, источниче разума, напой мя жаждущаго. Иисусе, одеждо веселия, одей мя тленнаго; Иисусе, покрове радости, покрый мя недостойнаго. Иисусе, подателю просящим, даждь мне плач за грехи моя; Иисусе, обретение ищущим, обрящи душу мою. Иисусе, отверзителю толкущим, отверзи сердце мое окаянное; Иисусе, Искупителю грешных, очисти беззакония моя. Иисусе, Сыне Божий, помилуй мя.

\mysubtitle{Кондак 7}

Хотя
сокровенную тайну от века открыти, яко овча на заколение веден был еси, Иисусе, и яко агнец прямо стригущаго его безгласен, и яко Бог из мертвых воскресл еси, и со славою на небеса вознеслся еси, и нас совоздвигл еси, зовущих: Аллилуиа.

\mysubtitle{Икос 7}

Дивную
показа тварь, явлейся Творец нам: без семене от Девы воплотися, из гроба, печати не рушив, воскресе, и ко апостолом, дверем затворенным, с плотию вниде. Темже чудящеся, воспоим: Иисусе, Слове необыменный; Иисусе, Слове несоглядаемый. Иисусе, сило непостижимая; Иисусе,
мудросте недомыслимая. Иисусе, Божество неописанное; Иисусе, господство неисчетное. Иисусе, царство непобедимое; Иисусе, владычество безконечное. Иисусе, крепосте высочайшая; Иисусе, власте вечная. Иисусе, Творче мой, ущедри мя; Иисусе, Спасе мой, спаси мя. Иисусе, Сыне Божий, помилуй мя.

\mysubtitle{Кондак 8}

Странно
Бога вочеловечшася видяще, устранимся суетнаго мира и ум на Божественная возложим. Сего бо ради Бог на землю сниде, да нас на небеса возведет, вопиющих Ему: Аллилуиа.

\mysubtitle{Икос 8}

Весь
бе в нижних, и вышних никакоже отступи Неисчетный, егда волею нас ради пострада, и смертию Своею нашу смерть умертви, и воскресением живот дарова поющим: Иисусе, сладосте сердечная; Иисусе, крепосте телесная. Иисусе, светлосте душевная; Иисусе, быстрото умная. Иисусе, радосте совестная; Иисусе, надеждо известная. Иисусе, памяте предвечная; Иисусе, похвало высокая. Иисусе, славо моя превознесенная; Иисусе, желание мое, не отрини мене. Иисусе, Пастырю мой, взыщи мене; Иисусе, Спасе мой, спаси мене. Иисусе, Сыне Божий, помилуй мя.

\mysubtitle{Кондак 9}

Все
естество ангельское безпрестани славит пресвятое имя Твое, Иисусе, на небеси: Свят, Свят, Свят, вопиюще; мы же, грешнии на земли бренными устнами вопием: Аллилуиа.

\mysubtitle{Икос 9}

Ветия
многовещанны, якоже рыбы безгласныя видим о Тебе, Иисусе, Спасе наш: недоумеют бо глаголати, како Бог непреложний и человек совершенный пребываеши? Мы же таинству дивящеся, вопием верно: Иисусе, Боже предвечный; Иисусе, Царю царствующих. Иисусе, Владыко владеющих; Иисусе, Судие живых и мертвых. Иисусе, надеждо ненадежных; Иисусе, утешение плачущих. Иисусе, славо нищих; Иисусе, не осуди мя по делом моим. Иисусе, очисти мя по милости Твоей; Иисусе, отжени от мене уныние. Иисусе, просвети моя мысли сердечныя; Иисусе, даждь ми память смертную. Иисусе, Сыне Божий, помилуй мя.

\mysubtitle{Кондак 10}

Спасти
хотя мир, Восточе востоком, к темному западу "--- естеству нашему пришед, смирился еси до смерти; темже превознесеся имя Твое паче всякаго имене, и от всех колен небесных и земных слышиши: Аллилуиа.

\mysubtitle{Икос 10}

Царю
Превечный, Утешителю, Христе истинный, очисти ны от всякия скверны, якоже очистил еси десять прокаженных, и исцели ны, якоже исцелил еси сребролюбивую душу Закхеа мытаря, да вопием Ти, во умилении зовуще: Иисусе, сокровище нетленное; Иисусе, богатство неистощимое. Иисусе, пище крепкая; Иисусе, питие неисчерпаемое. Иисусе, нищих одеяние; Иисусе, вдов заступление. Иисусе, сирых защитниче; Иисусе, труждающихся помоще. Иисусе, странных наставниче; Иисусе, плавающих кормчий. Иисусе, бурных отишие; Иисусе Боже, воздвигни мя падшаго. Иисусе, Сыне Божий, помилуй мя.

\mysubtitle{Кондак 11}

Пение
всеумиленное приношу Ти недостойный, вопию Ти яко хананеа: Иисусе, помилуй мя; не дщерь бо, но плоть имам страстьми люте бесящуюся и яростию палимую, и исцеление даждь вопиющу Ти: Аллилуиа.

\mysubtitle{Икос 11}

Светоподательна
светильника сущим во тьме неразумия, прежде гоняй Тя Павел, богоразумнаго гласа силу внуши и душевную быстроту уясни; сице и мене темныя зеницы душевныя просвети, зовуща: Иисусе, Царю мой прекрепкий; Иисусе, Боже мой пресильный. Иисусе, Господи мой пребезсмертный; Иисусе, Создателю мой преславный. Иисусе, Наставниче мой предобрый; Иисусе, Пастырю мой прещедрый. Иисусе, Владыко мой премилостивый; Иисусе, Спасе мой премилосердый. Иисусе, просвети моя чувствия, потемненныя страстьми; Иисусе, исцели мое тело, острупленное грехми. Иисусе, очисти мой ум от помыслов суетных; Иисусе, сохрани сердце мое от похотей лукавых. Иисусе, Сыне Божий, помилуй мя.

\mysubtitle{Кондак 12}

Благодать
подаждь ми, всех долгов решителю, Иисусе, и приими мя кающася, якоже приял еси Петра, отвергшагося Тебе, и призови мя унывающаго, якоже древле Павла, гоняща Тя, и услыши мя, вопиюща Ти: Аллилуиа.

\mysubtitle{Икос 12}

Поюще
Твое вочеловечение, восхваляем Тя вси, и веруем со Фомою, яко Господь и Бог еси, седяй со Отцем и хотяй судити живым и мертвым. Тогда убо сподоби мя деснаго стояния, вопиющаго: Иисусе, Царю предвечный, помилуй мя; Иисусе, цвете благовонный, облагоухай мя. Иисусе, теплото любимая, огрей мя; Иисусе, храме предвечный, покрый мя. Иисусе, одеждо светлая, украси мя; Иисусе, бисере честный, осияй мя. Иисусе, каменю драгий, просвети мя; Иисусе, солнце правды, освети мя. Иисусе, свете святый, облистай мя; Иисусе, болезни душевныя и телесныя избави мя. Иисусе, из руки сопротивныя изми мя; Иисусе, огня неугасимаго и прочих вечных мук свободи мя. Иисусе, Сыне Божий, помилуй мя.

\mysubtitle{Кондак 13}

О,
пресладкий и всещедрый Иисусе! Приими ныне малое моление сие наше, якоже приял еси вдовицы две лепте, и сохрани достояние Твое от враг видимых и невидимых, от нашествия иноплеменних, от недуга и глада, от всякия скорби и смертоносныя раны, и грядущия изми муки всех, вопиющих Ти: Аллилуиа, aллилуиа, aллилуиа.

\mysubtitle{(Kондак читается трижды)}

\mysubtitle{Икос 1}

Ангелов
Творче и Господи cил, отверзи ми недоуменный ум и язык на похвалу пречистаго Твоего имене, якоже глухому и гугнивому древле слух и язык отверзл еси, и, глаголаше зовый таковая: Иисусе пречудный, aнгелов удивление; Иисусе пресильный, прародителей избавление. Иисусе пресладкий, патриархов величание; Иисусе преславный, царей укрепление. Иисусе прелюбимый, пророков исполнение; Иисусе предивный, мучеников крепосте. Иисусе претихий, монахов радосте; Иисусе премилостивый, пресвитеров сладосте. Иисусе премилосердый, постников воздержание; Иисусе пресладостный, преподобных радование. Иисусе пречестный, девственных целомудрие; Иисусе предвечный, грешников спасение. Иисусе, Сыне Божий, помилуй мя.

\mysubtitle{Кондак 1}

Возбранный
Воеводо и Господи, ада победителю, яко избавлься от вечныя смерти, похвальная восписую Ти, создание и раб Твой; но, яко имеяй милосердие неизреченное, от всяких мя бед свободи, зовуща: Иисусе, Сыне Божий, помилуй мя.

\mysubtitle{Молитва}

Владыко
Господи Иисусе Христе Боже мой, иже неизреченнаго ради Твоего человеколюбия на конец веков во плоть оболкийся от Приснодевы Марии, славлю о мне Твое спасительное промышление, раб Твой, Владыко;
песнословлю Тя, яко Тебе ради Отца познах; благословлю Тя, Егоже ради и Дух Святый в мир прииде; покланяюся Твоей по плоти Пречистой Матери, таковей страшней тайне послужившей; восхваляю Твоя Ангельская ликостояния, яко воспеватели и служители Твоего величествия; ублажаю Предтечу Иоанна, Тебе крестившаго, Господи; почитаю и провозвестившия Тя пророки, прославляю апостолы Твоя святыя; торжествую же и мученики, священники же Твоя славлю; покланяюся преподобным Твоим, и вся Твоя праведники пестунствую. Таковаго и толикаго многаго и неизреченнаго лика Божественнаго в молитву привожду Тебе, всещедрому Богу, раб Твой, и сего ради прошу моим согрешением прощения, еже даруй ми всех Твоих ради святых, изряднее же святых Твоих щедрот, яко благословен еси во веки. Аминь.


\end{mymulticols}

\mychapterending


\mychapter{Акафист Пресвятой Богородице}\begin{mymulticols}
%http://www.molitvoslov.org/text13.htm

\myfigure{5_0}

\mysubtitle{Кондак 1}

Взбранной Воеводе победительная, яко избавльшеся от злых, благодарственная восписуем Ти раби Твои, Богородице; но яко имущая державу непобедимую, от всяких нас бед свободи, да зовем Ти: радуйся, Невесто Неневестная.

\mysubtitle{Икос 1}

Ангел предстатель с небесе послан бысть рещи Богородице: радуйся, и со безплотным гласом воплощаема Тя зря, Господи, ужасашеся и стояше, зовый к Ней таковая: Радуйся, Еюже рaдocть возсияет; радуйся, Еюже клятва изчезнет. Радуйся, падшаго Адама воззвание; радуйся, слез Евиных избавление. Радуйся, высото неудобовосходимая человеческими помыслы; радуйся, глубино неудобозримая и ангельскима очима. Радуйся, яко еси Царево седалище; радуйся, яко носиши Носящаго вся. Радуйся, Звездо, являющая Солнце; радуйся, утробо Божественнаго воплощения. Радуйся, Еюже обновляется тварь; радуйся, Еюже покланяемся Творцу. Радуйся, Невесто Неневестная.

\mysubtitle{Кондак 2}

Видящи Святая Себе в чистоте, глаголет Гавриилу дерзостно: преславное твоего гласа неудобоприятельно души Моей является: безсеменнаго бо зачатия рождество како глаголеши, зовый: Аллилуиа.

\mysubtitle{Икос 2}

Разум недоразумеваемый разумети Дева ищущи, возопи к служащему: из боку чисту, Сыну како есть родитися мощно, рцы Ми? К Нейже он рече со страхом, обаче зовый сице: Радуйся, совета неизреченнаго Таиннице; радуйся, молчания просящих веро. Радуйся, чудес Христовых начало; радуйся, велений Его главизно. Радуйся, лествице небесная, Еюже сниде Бог; радуйся, мосте, преводяй сущих от земли на небо. Радуйся, Ангелов многословущее чудо; радуйся, бесов многоплачевное поражение. Радуйся, Свет неизреченно родившая; радуйся, еже како, ни единаго же научившая. Радуйся, премудрых превосходящая разум; радуйся, верных озаряющая смыслы. Радуйся, Невесто Неневестная.

\mysubtitle{Кондак 3}

Сила Вышняго осени тогда к зачатию Браконеискусную, и благоплодная Тоя ложесна, яко село показа сладкое, всем хотящим жати спасение, всегда пети сице: Аллилуиа.

\mysubtitle{Икос 3}

Имущи Богоприятную Дева утробу, востече ко Елисавети: младенец же оноя абие познав Сея целование, радовашеся, и играньми яко песньми вопияше к Богородице: Радуйся, отрасли неувядаемыя розго; радуйся, Плода безсмертнаго стяжание. Радуйся, Делателя делающая Человеколюбца; радуйся, Садителя жизни нашея рождшая. Радуйся, ниво, растящая гобзование щедрот; радуйся, трапезо, носящая обилие очищения. Радуйся, яко рай пищный процветаеши; радуйся, яко пристанище душам готовиши. Радуйся, приятное молитвы кадило; радуйся, всего мира очищение. Радуйся, Божие к смертным благоволение; радуйся, смертных к Богу дерзновение. Радуйся, Невесто Неневестная.

\mysubtitle{Кондак 4}

Бурю внутрь имея помышлений сумнительных, целомудренный Иосиф смятеся, к Тебе зря небрачней, и бракоокрадованную помышляя, Непорочная; уведев же Твое зачатие от Духа Свята, рече: Аллилуиа.

\mysubtitle{Икос 4}

Слышаша пастырие Ангелов поющих плотское Христово пришествие, и текше яко к Пастырю видят Сего яко агнца непорочна, во чреве Мариине упасшася, Юже поюще реша: Радуйся, Агнца и Пастыря Мати; радуйся, дворе словесных овец. Радуйся, невидимых врагов мучение; радуйся, райских дверей отверзение. Радуйся, яко небесная срадуются земным; радуйся, яко земная сликовствуют небесным. Радуйся, апостолов немолчная уста; радуйся, страстотерпцев непобедимая дерзосте. Радуйся, твердое веры утверждение; радуйся, светлое благодати познание. Радуйся, Еюже обнажися ад; радуйся, Еюже облекохомся славою. Радуйся, Невесто Неневестная.

\mysubtitle{Кондак 5}

Боготечную звезду узревше волсви, тоя последоваша зари, и яко светильник держаще ю, тою испытаху крепкаго Царя, и достигше Непостижимаго, возрадовашася, Ему вопиюще: Аллилуиа.

\mysubtitle{Икос 5}

Видеша отроцы халдейстии на руку Девичу Создавшаго руками человеки, и Владыку разумевающе Его, аще и рабий прият зрак, потщашася дарми послужити Ему, и возопити Благословенней: Радуйся, Звезды незаходимыя Мати; радуйся, заре таинственнаго дне. Радуйся, прелести пещь угасившая; радуйся, Троицы таинники просвещающая. Радуйся, мучителя безчеловечнаго изметающая от начальства; радуйся, Господа Человеколюбца показавшая Христа. Радуйся, варварскаго избавляющая служения; радуйся, тимения изымающая дел. Радуйся, огня поклонение угасившая; радуйся, пламене страстей изменяющая. Радуйся, верных наставнице целомудрия; радуйся, всех родов веселие. Радуйся, Невесто Неневестная.

\mysubtitle{Кондак 6}

Проповедницы богоноснии, бывше волсви, возвратишася в Вавилон, скончавше Твое пророчество, и проповедавше Тя Христа всем, оставиша Ирода яко буесловна, не ведуща пети: Аллилуиа.

\mysubtitle{Икос 6}

Возсиявый во Египте просвещение истины, отгнал еси лжи тьму: идоли бо его, Спасе, не терпяще Твоея крепости, падоша, сих же избавльшиися вопияху к Богородице: Радуйся, исправление человеков; радуйся, низпадение бесов. Радуйся, прелести державу поправшая; радуйся, идольскую лесть обличившая. Радуйся, море, потопившее фараона мысленнаго; радуйся, каменю, напоивший жаждущия жизни. Радуйся, огненный столпе, наставляяй сущия во тьме; радуйся, покрове миру, ширший облака. Радуйся, пище, манны приемнице; радуйся, сладости святыя служительнице. Радуйся, земле обетования; радуйся, из неяже течет мед и млеко. Радуйся, Невесто Неневестная.

\mysubtitle{Кондак 7}

Хотящу Симеону от нынешняго века преставитися прелестнаго, вдался еси яко младенец тому, но познался еси ему и Бог совершенный. Темже удивися Твоей неизреченней премудрости, зовый: Аллилуиа.

\mysubtitle{Икос 7}

Новую показа тварь, явлься Зиждитель нам от Него бывшим, из безсеменныя прозяб утробы, и сохранив Ю, якоже бе, нетленну, да чудо видяще, воспоем Ю, вопиюще: Радуйся, цвете нетления; радуйся, венче воздержания. Радуйся, воскресения образ облистающая; радуйся, ангельское житие являющая. Радуйся, древо светлоплодовитое, от негоже питаются вернии; радуйся, древо благосеннолиственное, имже покрываются мнози. Радуйся, во чреве носящая Избавителя плененным; радуйся, рождшая Наставника заблудшим. Радуйся, Судии праведнаго умоление; радуйся, многих согрешений прощение. Радуйся, одеждо нагих дерзновения; радуйся, любы, всякое желание побеждающая. Радуйся, Невесто Неневестная.

\mysubtitle{Кондак 8}

Странное рождество видевше, устранимся мира, ум на небеса преложше: сего бо ради высокий Бог на земли явися смиренный человек, хотяй привлещи к высоте Тому вопиющия: Аллилуиа.

\mysubtitle{Икос 8}

Весь бе в нижних и вышних никакоже отступи неописанное Слово: снизхождение бо Божественное, не прехождение же местное бысть, и рождество от Девы Богоприятныя, слышащия сия: Радуйся, Бога невместимаго вместилище; радуйся, честнаго таинства двери. Радуйся, неверных сумнительное слышание; радуйся, верных известная похвало. Радуйся, колеснице пресвятая Сущаго на Херувимех; радуйся, селение преславное Сущаго на Серафимех. Радуйся, противная в тожде собравшая; радуйся, девство и рождество сочетавшая. Радуйся, Еюже разрешися преступление; радуйся, Еюже отверзеся рай. Радуйся, ключу Царствия Христова; радуйся, надеждо благ вечных. Радуйся, Невесто Неневестная.

\mysubtitle{Кондак 9}

Всякое естество ангельское удивися великому Твоего вочеловечения делу; неприступнаго бо яко Бога, зряще всем приступнаго Человека, нам убо спребывающа, слышаща же от всех: Аллилуиа.

\mysubtitle{Икос 9}

Ветия многовещанныя, яко рыбы безгласныя видим о Тебе, Богородице, недоумевают бо глаголати, еже како и Дева пребываеши, и родити возмогла еси. Мы же, таинству дивящеся, верно вопием: Радуйся, премудрости Божия приятелище, радуйся, промышления Его сокровище. Радуйся, любомудрыя немудрыя являющая; радуйся, хитрословесныя безсловесныя обличающая. Радуйся, яко обуяша лютии взыскателе; радуйся, яко увядоша баснотворцы. Радуйся, афинейская плетения растерзающая; радуйся, рыбарския мрежи исполняющая. Радуйся, из глубины неведения извлачающая; радуйся, многи в разуме просвещающая. Радуйся, кораблю хотящих спастися; радуйся, пристанище житейских плаваний. Радуйся, Невесто Неневестная.

\mysubtitle{Кондак 10}

Спасти хотя мир, Иже всех Украситель, к сему самообетован прииде, и Пастырь сый, яко Бог, нас ради явися по нам человек: подобным бо подобное призвав, яко Бог слышит: Аллилуиа.

\mysubtitle{Икос 10}

Стена еси девам, Богородице Дево, и всем к Тебе прибегающим: ибо небесе и земли Творец устрои Тя, Пречистая, вселься во утробе Твоей, и вся приглашати Тебе научи: Радуйся, столпе девства; радуйся, дверь спасения. Радуйся, начальнице мысленнаго наздания; радуйся, подательнице Божественныя благости. Радуйся, Ты бо обновила еси зачатыя студно; радуйся, Ты бо наказала еси окраденныя умом. Радуйся, тлителя смыслов упраждняющая; радуйся, Сеятеля чистоты рождшая. Радуйся, чертоже безсеменнаго уневещения; радуйся, верных Господеви сочетавшая. Радуйся, добрая младопитательнице девам; радуйся, невестокрасительнице душ святых. Радуйся, Невесто Неневестная.

\mysubtitle{Кондак 11}

Пение всякое побеждается, спростретися тщащееся ко множеству многих щедрот Твоих: равночисленныя бо песка песни аще приносим Ти, Царю Святый, ничтоже совершаем достойно, яже дал еси нам, Тебе вопиющим: Аллилуиа.

\mysubtitle{Икос 11}

Светоприемную свещу, сущим во тьме явльшуюся, зрим Святую Деву, невещественный бо вжигающи огнь, наставляет к разуму Божественному вся, зарею ум просвещающая, званием же почитаемая, сими: Радуйся, луче умнаго Солнца; радуйся, светило незаходимаго Света. Радуйся, молние, души просвещающая; радуйся, яко гром враги устрашающая. Радуйся, яко многосветлое возсияваеши просвещение; радуйся, яко многотекущую источаеши реку. Радуйся, купели живописующая образ; радуйся, греховную отъемлющая скверну. Радуйся, бане, омывающая совесть; радуйся, чаше, черплющая рaдocть. Радуйся, обоняние Христова благоухания; радуйся, животе тайнаго веселия. Радуйся, Невесто Неневестная.

\mysubtitle{Кондак 12}

Благодать дати восхотев, долгов древних, всех долгов Решитель человеком, прииде Собою ко отшедшим Того благодати, и раздрав рукописание, слышит от всех сице: Аллилуиа.

\mysubtitle{Икос 12}

Поюще Твое Рождество, хвалим Тя вси, яко одушевленный храм, Богородице: во Твоей бо вселився утробе содержай вся рукою Господь, освяти, прослави и научи вопити Тебе всех: Радуйся, селение Бога и Слова; радуйся, святая святых большая. Радуйся, ковчеже, позлащенный Духом; радуйся, сокровище живота неистощимое. Радуйся, честный венче людей благочестивых; радуйся, честная похвало иереев благоговейных. Радуйся, церкве непоколебимый столпе; радуйся, Царствия нерушимая стено. Радуйся, Еюже воздвижутся победы; радуйся, Еюже низпадают врази. Радуйся, тела моего врачевание; радуйся, души моея спасение. Радуйся, Невесто Неневестная.

\mysubtitle{Кондак 13}

О, Всепетая Мати, рождшая всех святых Святейшее Слово! Нынешнее приемши приношение, от всякия избави напасти всех, и будущия изми муки, о Тебе вопиющих: Аллилуиа, aллилуиа, aллилуиа.

\myemph{(Kондак читается трижды)}

\mysubtitle{Икос 1}

Ангел предстатель с небесе послан бысть рещи Богородице: радуйся, и со безплотным гласом воплощаема Тя зря, Господи, ужасашеся и стояше, зовый к Ней таковая: Радуйся, Еюже рaдocть возсияет; радуйся, Еюже клятва изчезнет. Радуйся, падшаго Адама воззвание; радуйся, слез Евиных избавление. Радуйся, высото неудобовосходимая человеческими помыслы; радуйся, глубино неудобозримая и ангельскима очима. Радуйся, яко еси Царево седалище; радуйся, яко носиши Носящаго вся. Радуйся, Звездо, являющая Солнце; радуйся, утробо Божественнаго воплощения. Радуйся, Еюже обновляется тварь; радуйся, Еюже покланяемся Творцу. Радуйся, Невесто Неневестная.

\mysubtitle{Кондак 1}

Взбранной Воеводе победительная, яко избавльшеся от злых, благодарственная восписуем Ти раби Твои, Богородице; но яко имущая державу непобедимую, от всяких нас бед свободи, да зовем Ти: радуйся, Невесто Неневестная.

\mysubtitle{Молитвы}

О, Пресвятая Госпоже Владычице Богородице, вышши еси всех Ангел и Архангел, и всея твари честнейши, помощнице еси обидимых, ненадеющихся надеяние, убогих заступнице, печальных утешение, алчущих кормительнице, нагих одеяние, больных исцеление, грешных спасение, христиан всех поможение и заступление. О, Всемилостивая Госпоже, Дево Богородице Владычице, милостию Твоею спаси и помилуй святейшия патриархи православныя, преосвященныя митрополиты, архиепископы и епископы и весь священнический и иноческий чин, и вся православныя христианы ризою Твоею честною защити; и умоли, Госпоже, из Тебе без семене воплотившагося Христа Бога нашего, да препояшет нас силою Своею свыше, на невидимыя и видимыя враги наша. О, Всемилостивая Госпоже Владычице Богородице! Воздвигни нас из глубины греховныя и избави нас от глада, губительства, от труса и потопа, от огня и меча, от нахождения иноплеменных и междоусобныя брани, и от напрасныя смерти, и от нападения вражия, и от тлетворных ветр, и от смертоносныя язвы, и от всякаго зла. Подаждь, Госпоже, мир и здравие рабом Твоим, всем православным христианом, и просвети им ум, и очи сердечнии, еже ко спасению; и сподоби ны, грешныя рабы Твоя, Царствия Сына Твоего, Христа Бога нашего; яко держава Его благословена и препрославлена, со Безначальным Его Отцем, и с Пресвятым, и Благим, и Животворящим Его Духом, ныне и присно, и во веки веков. Аминь.

О, Пресвятая Дево Мати Господа, Царице Небесе и земли! Вонми многоболезненному воздыханию души нашея, призри с высоты святыя Твоея на нас, с верою и любовию покланяющихся пречистому образу Твоему. Се бо грехми погружаемии и скорбьми обуреваемии, взирая на Твой образ, яко живей Ти сущей с нами, приносим смиренныя моления наша. Не имамы бо ни иныя помощи, ни инаго предстательства, ни утешения, токмо Тебе, о, Мати всех скорбящих и обремененных. Помози нам немощным, утоли скорбь нашу, настави на путь правый нас, заблуждающих, уврачуй и спаси безнадежных, даруй нам прочее время живота нашего в мире и тишине проводити, подаждь христианскую кончину, и на страшнем суде Сына Твоего явися нам милосердая Заступница, да всегда поем, величаем и славим Тя, яко благую Заступницу рода христианскаго, со всеми угодившими Богу. Аминь.


\end{mymulticols}

\mychapterending


\mychapter{Благодарственные молитвы по Святом Причащении}\begin{mymulticols}
%http://www.molitvoslov.org/text206.htm

Сл\'{а}ва Теб\'{е}, Б\'{о}же. Сл\'{а}ва Теб\'{е}, Б\'{о}же. Сл\'{а}ва Теб\'{е}, Б\'{о}же.

\mysubtitle{Благодарственная мол\'{и}тва, 1-я}

Благодар\'{ю} Тя, Г\'{о}споди, Б\'{о}же мой, \'{я}ко не отр\'{и}нул мя ес\'{и} гр\'{е}шнаго, но \'{о}бщника мя б\'{ы}ти свят\'{ы}нь Тво\'{и}х спод\'{о}бил ес\'{и}. Благодар\'{ю} Тя, \'{я}ко мен\'{е} недост\'{о}йнаго причаст\'{и}тися Преч\'{и}стых Тво\'{и}х и Неб\'{е}сных Дар\'{о}в спод\'{о}бил ес\'{и}. Но Влад\'{ы}ко Человекол\'{ю}бче, нас р\'{а}ди ум\'{е}рый же и воскрес\'{ы}й, и даров\'{а}вый нам стр\'{а}шная си\'{я} и животвор\'{я}щая Т\'{а}инства во благоде\'{я}ние и освящ\'{е}ние душ и тел\'{е}с н\'{а}ших, д\'{а}ждь б\'{ы}ти сим и мне во исцел\'{е}ние душ\'{и} же и т\'{е}ла, во отгн\'{а}ние вс\'{я}каго сопрот\'{и}внаго, в просвещ\'{е}ние \'{о}чию с\'{е}рдца моег\'{о}, в мир душ\'{е}вных мо\'{и}х сил, в в\'{е}ру непост\'{ы}дну, в люб\'{о}вь нелицем\'{е}рну, во исполн\'{е}ние прем\'{у}дрости, в соблюд\'{е}ние з\'{а}поведей Тво\'{и}х, в прилож\'{е}ние Бож\'{е}ственныя Твое\'{я} благод\'{а}ти и Твоег\'{о} Ц\'{а}рствия присво\'{е}ние; да во свят\'{ы}ни Тво\'{е}й т\'{е}ми сохран\'{я}емь, Тво\'{ю} благод\'{а}ть помин\'{а}ю всегд\'{а}, и не ктом\'{у} себ\'{е} жив\'{у}, но Теб\'{е}, н\'{а}шему Влад\'{ы}це и Благод\'{е}телю; и т\'{а}ко сег\'{о} жити\'{я} изш\'{е}д о над\'{е}жди живот\'{а} в\'{е}чнаго, в приснос\'{у}щный дост\'{и}гну пок\'{о}й, ид\'{е}же пр\'{а}зднующих глас непрест\'{а}нный, и безкон\'{е}чная сл\'{а}дость, зр\'{я}щих Твоег\'{о} лиц\'{а} добр\'{о}ту неизреч\'{е}нную. Ты бо ес\'{и} \'{и}стинное жел\'{а}ние, и неизреч\'{е}нное вес\'{е}лие л\'{ю}бящих Тя, Христ\'{е} Б\'{о}же наш, и Тя по\'{е}т вс\'{я} тварь во в\'{е}ки. Ам\'{и}нь.

\mysubtitle{Молитва 2-я, святого Василия Великого}

Влад\'{ы}ко Христ\'{е} Б\'{о}же, Цар\'{ю} век\'{о}в, и Сод\'{е}телю всех, благодар\'{ю} Тя о всех, \'{я}же ми п\'{о}дал благ\'{и}х, и о причащ\'{е}нии преч\'{и}стых и животвор\'{я}щих Тво\'{и}х Т\'{а}инств. Мол\'{ю} \'{у}бо Тя, Бл\'{а}же и Человекол\'{ю}бче: сохр\'{а}ни мя под кр\'{о}вом Тво\'{и}м, и в с\'{е}ни крил\'{у} Тво\'{е}ю; и д\'{а}руй ми ч\'{и}стою с\'{о}вестию, д\'{а}же до посл\'{е}дняго моег\'{о} издых\'{а}ния, дост\'{о}йно причащ\'{а}тися свят\'{ы}нь Тво\'{и}х, во оставл\'{е}ние грех\'{о}в, и в жизнь в\'{е}чную. Ты бо ес\'{и} хлеб жив\'{о}тный, ист\'{о}чник свят\'{ы}ни, Под\'{а}тель благ\'{и}х, и Теб\'{е} сл\'{а}ву возсыл\'{а}ем, со Отц\'{е}м и Свят\'{ы}м Д\'{у}хом, н\'{ы}не и пр\'{и}сно, и во в\'{е}ки век\'{о}в. Ам\'{и}нь.

\mysubtitle{Молитва 3-я, Симеона Метафраста}

Д\'{а}вый п\'{и}щу мне плоть Тво\'{ю} в\'{о}лею, огнь сый и опал\'{я}яй недост\'{о}йныя, да не опал\'{и}ши мен\'{е}, Сод\'{е}телю мой; п\'{а}че же пройд\'{и} во \'{у}ды м\'{о}я, во вс\'{я} сост\'{а}вы, во утр\'{о}бу, в с\'{е}рдце. Попал\'{и} т\'{е}рние всех мо\'{и}х прегреш\'{е}ний. Д\'{у}шу оч\'{и}сти, освят\'{и} помышл\'{е}ния. Сост\'{а}вы утверд\'{и} с костьм\'{и} вк\'{у}пе. Чувств просвет\'{и} прост\'{у}ю пятер\'{и}цу. Всег\'{о} мя спригвозд\'{и} стр\'{а}ху Твоем\'{у}. Пр\'{и}сно покр\'{ы}й, соблюд\'{и} же, и сохр\'{а}ни мя от вс\'{я}каго д\'{е}ла и сл\'{о}ва душетл\'{е}ннаго. Оч\'{и}сти и ом\'{ы}й, и украс\'{и} мя; удобр\'{и}, вразум\'{и}, и просвет\'{и} мя. Покаж\'{и} мя Тво\'{е} сел\'{е}ние ед\'{и}наго Д\'{у}ха, и не ктом\'{у} сел\'{е}ние грех\'{а}. Да \'{я}ко Твоег\'{о} д\'{о}му, вх\'{о}дом причащ\'{е}ния, \'{я}ко огн\'{я} мен\'{е} беж\'{и}т всяк злод\'{е}й, вс\'{я}ка страсть. Мол\'{и}твенники Теб\'{е} принош\'{у} вс\'{я} свят\'{ы}я, чинонач\'{а}лия же безпл\'{о}тных, Предт\'{е}чу Твоег\'{о}, прем\'{у}дрыя Апостолы, к сим же Тво\'{ю} нескв\'{е}рную ч\'{и}стую М\'{а}терь, \'{и}хже мольб\'{ы} Благоутр\'{о}бне приим\'{и}, Христ\'{е} мой, и с\'{ы}ном св\'{е}та сод\'{е}лай Твоег\'{о} служ\'{и}теля. Ты бо ес\'{и} освящ\'{е}ние и Ед\'{и}ный н\'{а}ших, Бл\'{а}же, душ и св\'{е}тлость; и Теб\'{е} лепопод\'{о}бно, \'{я}ко Б\'{о}гу и Влад\'{ы}це, сл\'{а}ву вс\'{и} возсыл\'{а}ем на всяк день.

\mysubtitle{Молитва 4-я}

Т\'{е}ло Тво\'{е} Свят\'{о}е, Г\'{о}споди, Иис\'{у}се Христ\'{е}, Б\'{о}же наш, да б\'{у}дет ми в жив\'{о}т в\'{е}чный, и Кровь Тво\'{я} Честн\'{а}я во оставл\'{е}ние грех\'{о}в: б\'{у}ди же ми благодар\'{е}ние си\'{е} в р\'{а}дость, здр\'{а}вие и вес\'{е}лие; в стр\'{а}шное же и втор\'{о}е приш\'{е}ствие Тво\'{е} спод\'{о}би мя гр\'{е}шнаго ст\'{а}ти одесн\'{у}ю сл\'{а}вы Твое\'{я}, мол\'{и}твами Преч\'{и}стыя Твое\'{я} М\'{а}тере, и всех свят\'{ы}х.

\mysubtitle{Молитва 5-я, ко Пресвятой Богородице}

Пресвят\'{а}я Влад\'{ы}чице Богор\'{о}дице, св\'{е}те помрач\'{е}нныя мое\'{я} душ\'{и}, над\'{е}ждо, покр\'{о}ве, приб\'{е}жище, утеш\'{е}ние, р\'{а}дование мо\'{е}, благодар\'{ю} Тя, \'{я}ко спод\'{о}била мя ес\'{и} недост\'{о}йнаго, прич\'{а}стника б\'{ы}ти Преч\'{и}стаго Т\'{е}ла и Честн\'{ы}я Кр\'{о}ве С\'{ы}на Твоег\'{о}. Но р\'{о}ждшая \'{и}стинный Свет, просвет\'{и} м\'{о}я \'{у}мныя \'{о}чи с\'{е}рдца; \'{Я}же Ист\'{о}чник безсм\'{е}ртия р\'{о}ждшая, оживотвор\'{и} мя умерщвл\'{е}ннаго грех\'{о}м; \'{Я}же м\'{и}лостиваго Б\'{о}га любоблагоутр\'{о}бная М\'{а}ти, пом\'{и}луй мя, и д\'{а}ждь ми умил\'{е}ние и сокруш\'{е}ние в с\'{е}рдце мо\'{е}м, и смир\'{е}ние в м\'{ы}слех мо\'{и}х, и воззв\'{а}ние в плен\'{е}ниих помышл\'{е}ний мо\'{и}х; и спод\'{о}би мя до посл\'{е}дняго издых\'{а}ния, неосужд\'{е}нно приим\'{а}ти преч\'{и}стых Т\'{а}ин освящ\'{е}ние, во исцел\'{е}ние д\'{у}ши же и т\'{е}ла. И под\'{а}ждь ми сл\'{е}зы пока\'{я}ния и испов\'{е}дания, во \'{е}же п\'{е}ти и сл\'{а}вити Тя во вс\'{я} дни живот\'{а} моег\'{о}, \'{я}ко благослов\'{е}нна и препрoславленна ес\'{и} во в\'{е}ки. Ам\'{и}нь.

Н\'{ы}не отпущ\'{а}еши раб\'{а} Твоег\'{о}, Влад\'{ы}ко, по глаг\'{о}лу Твоем\'{у}, с м\'{и}ром: \'{я}ко в\'{и}деста \'{о}чи мо\'{и} спас\'{е}ние Тво\'{е}, \'{е}же ес\'{и} угот\'{о}вал пред лиц\'{е}м всех люд\'{е}й, свет во откров\'{е}ние яз\'{ы}ков и сл\'{а}ву люд\'{е}й Тво\'{и}х, Изр\'{а}иля.

\TrisviatoePoOtcheNash

\mysubtitle{Тропарь св. Иоанну Златоустому, глас 8-й:}

Уст тво\'{и}х, \'{я}коже св\'{е}тлость огн\'{я}, возси\'{я}вши благод\'{а}ть, всел\'{е}нную просвет\'{и}: не среброл\'{ю}бия м\'{и}рови сокр\'{о}вища сниск\'{а}, высот\'{у} нам смиренном\'{у}дрия показ\'{а}, но тво\'{и}ми словес\'{ы} наказ\'{у}я, \'{о}тче Ио\'{а}нне Злато\'{у}сте, мол\'{и} Сл\'{о}ва Христ\'{а} Б\'{о}га спаст\'{и}ся душ\'{а}м н\'{а}шим.

\mysubtitle{Кондак, глас 6-й:}

\slava

От неб\'{е}с при\'{я}л ес\'{и} Бож\'{е}ственную благод\'{а}ть, и тво\'{и}ми устн\'{а}ма вс\'{я} уч\'{и}ши поклан\'{я}тися в Тр\'{о}ице единому Б\'{о}гу, Ио\'{а}нне Злато\'{у}сте, всеблаж\'{е}нне препод\'{о}бне, дост\'{о}йно хв\'{а}лим тя: ес\'{и} бо наст\'{а}вник, \'{я}ко бож\'{е}ственная явл\'{я}я.

\inyne

\Bogorodichen{\Predstatelstvo}

\myemph{Если совершалась литургия святого Василия Великого, читай}

\mysubtitle{Тропарь Василию Великому, глас 1-й:}

Во всю з\'{е}млю из\'{ы}де вещ\'{а}ние тво\'{е}, \'{я}ко при\'{е}мшую сл\'{о}во тво\'{е}, \'{и}мже богол\'{е}пно науч\'{и}л ес\'{и}, естеств\'{о} с\'{у}щих уясн\'{и}л ес\'{и}, челов\'{е}ческия об\'{ы}чаи украс\'{и}л ес\'{и}, ц\'{а}рское свящ\'{е}ние, \'{о}тче препод\'{о}бне, мол\'{и} Христ\'{а} Б\'{о}га, спаст\'{и}ся душ\'{а}м н\'{а}шим.

\vspace{-\baselineskip}\mysubtitle{Кондак, глас 4-й:}

\slava

Яв\'{и}лся ес\'{и} основ\'{а}ние непоколеб\'{и}мо\'{е} Ц\'{е}ркве, пода\'{я} всем некрад\'{о}мое госп\'{о}дство челов\'{е}ком, запечатл\'{е}я тво\'{и}ми вел\'{е}ньми, небоявл\'{е}нне Вас\'{и}лие препод\'{о}бне.

\inyne

\Bogorodichen{\Predstatelstvo}

\myemph{Если совершалась Литургия Преждеосвященных Даров, читай}

\mysubtitle{Тропарь святому Григорию Двоеслову, глас 4-й:}

\'{И}же от Б\'{о}га св\'{ы}ше бож\'{е}ственную благод\'{а}ть воспри\'{е}м, сл\'{а}вне Григ\'{о}рие, и Тог\'{о} с\'{и}лою укрепл\'{я}емь, ев\'{а}нгельски ш\'{е}ствовати изв\'{о}лил ес\'{и}, отон\'{у}дуже у Христ\'{а} возм\'{е}здие труд\'{о}в при\'{я}л ес\'{и} всеблаж\'{е}нне: Ег\'{о}же мол\'{и}, да спас\'{е}т д\'{у}ши н\'{а}ша.

\vspace{-\baselineskip}\mysubtitle{Кондак, глас 3-й:}

\slava

\ifpdf
    \begin{small}
\else
\fi

Подобонач\'{а}льник показ\'{а}лся ес\'{и} Нач\'{а}льника п\'{а}стырем Христ\'{а}, \'{и}ноков чред\'{ы}, \'{о}тче Григ\'{о}рие, ко огр\'{а}де неб\'{е}сней наставл\'{я}я, и отт\'{у}ду науч\'{и}л ес\'{и} ст\'{а}до Христ\'{о}во з\'{а}поведем Ег\'{о}: н\'{ы}не же с н\'{и}ми р\'{а}дуешися, и лик\'{у}еши в неб\'{е}сных кр\'{о}вех.

\ifpdf
    \end{small}
\else
\fi

\inyne

\Bogorodichen{\Predstatelstvo}

Г\'{о}споди, пом\'{и}луй. \myemph{(12 раз)}

Сл\'{а}ва Отц\'{у} и С\'{ы}ну и Свят\'{о}му Д\'{у}ху, и н\'{ы}не и пр\'{и}сно и во в\'{е}ки век\'{о}в. Ам\'{и}нь.

Честн\'{е}йшую Херув\'{и}м и сл\'{а}внейшую без сравн\'{е}ния Сераф\'{и}м, без истл\'{е}ния Б\'{о}га Сл\'{о}ва р\'{о}ждшую, с\'{у}щую Богор\'{о}дицу Тя велич\'{а}ем.

\myemph{После Причащения да пребывает каждый в чистоте, воздержании и немногословии, чтобы достойно сохранить в себе принятого Христа.}

\end{mymulticols}

\mychapterending

\mychapter{Символ веры}\begin{mymulticols}
%http://www.molitvoslov.org/content/Simvol-very 

\SymbolOfFaith

\end{mymulticols}

\mychapterending

