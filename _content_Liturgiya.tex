

\mypart{БОЖЕСТВЕННАЯ ЛИТУРГИЯ}\label{_content_Liturgiya}
%http://www.molitvoslov.org/content/Liturgiya

\mychapter{О Литургии}

В разделе \textbf{БОЖЕСТВЕННАЯ ЛИТУРГИЯ} приведены песнопения Божественных литургий святителей Иоанна Златоуста и Василия Великого. Избранные песнопения литургии Преждеосвященных Даров помещены в главе «Песнопения из служб Триоди постной».

Литургия святителя Иоанна Златоуста совершается в Православной Церкви в течении всего года, кроме Великого Поста, когда она совершается по субботам, в Благовещение Пресвятой Богородицы и в Неделю ваий.

Литургия святителя Василия Великого совершается 10 раз в году: в воскресные дни Великого поста, в Великие четверг и субботу, в день памяти святителя Василия Великого (1 января), в Рождественский и Крещенский сочельники (если же эти дни выпадают на субботу или воскресенье, то совершается в самые праздники Рождества Христова и Крещения Господня).

Литургия состоит из трех частей: проскомидии, литургии оглашенных и литургии верных.

\mychapter{Проском\'{и}дия}
%http://www.molitvoslov.org/content/Proskomidiya 

До чтения третьего и шестого часов, или во время чтения их, в алтаре совершаются священнодействия \textbf{проском\'{и}дии}, через которые из принесенных хлеба и вина приготовляется вещество для Св. Евхаристии и при этом совершается предварительное поминовение членов Церкви Христовой "--- небесной и земной.

От древнего обычая приносить в храм хлеб и вино для таинства Св. Евхаристии первая часть литургии и называется проском\'{и}дией "--- приношением.

\mychapterending

\mychapter{Литургия оглашенных}
%http://www.molitvoslov.org/text906.htm 

Вторая часть литургии называется \textbf{литургией оглашенных}. Такое название эта часть службы получила от содержания в ее составе молитвословий, песнопений, священнодействий и поучений, имеющих учительный, огласительный характер. В древней Церкви во время ее совершения могли присутствовать, вместе с верными, и оглашенные, т.~е. лица, готовящиеся ко Св. Крещению, а также кающиеся, отлученные от Св. Причастия.

\begin{mymulticols}

\myemph{Диакон:} Благослови, владыко.

\myemph{Иерей:} Благословено Царство Отца и Сына и Святаго Духа, ныне и присно и во веки веков. 

\myemph{Хор:} Аминь.

\mysubtitle{Великая ектения}

\myemph{Диакон:} Миром Господу помолимся. 

\myemph{Хор:} Господи, помилуй.

\myemph{Диакон:} О свышнем мире и спасении душ наших, Господу помолимся. 

\myemph{Хор:} Господи, помилуй.

\myemph{Диакон:} О мире всего мира, благостоянии Святых Божиих Церквей и соединении всех, Господу помолимся.

\myemph{Хор:} Господи, помилуй.

\myemph{Диакон:} О святем храме сем и с верою, благоговением и страхом Божиим входящих в онь, Господу помолимся. 

\myemph{Хор:} Господи, помилуй.

\myemph{Диакон:} О Великом Господине и Отце нашем Святейшем Патриархе Кирилле, и о Господине нашем Преосвященнейшем митрополите ( \myemph{или:} архиепископе, \myemph{или:} епископе) \myemph{(имярек)}, честнем пресвитерстве, во Христе диаконстве, о всем причте и людех, Господу помолимся. 

\myemph{Хор:} Господи, помилуй.

\myemph{Диакон:} О Богохранимей стране нашей, властех и воинстве ея, Господу помолимся.

\myemph{Хор:} Господи, помилуй.

\myemph{Диакон:} О граде сем ( \myemph{или:} o веси сей, \myemph{если в монастыре, то:} о святей обители сей), всяком граде, стране и верою живущих в них, Господу помолимся. 

\myemph{Хор:} Господи, помилуй.

\myemph{Диакон:} О благорастворении воздухов, о изобилии плодов земных и временех мирных, Господу помолимся.

\myemph{Хор:} Господи, помилуй.

\myemph{Диакон:} О плавающих, путешествующих, недугующих, страждущих, плененных и о спасении их, Господу помолимся. 

\myemph{Хор:} Господи, помилуй.

\myemph{Диакон:} О избавитися нам от всякия скорби, гнева и нужды, Господу помолимся. 

\myemph{Хор:} Господи, помилуй.

\myemph{Диакон:} Заступи, спаси, помилуй и сохрани нас, Боже, Твоею благодатию. 

\myemph{Хор:} Господи, помилуй.

\myemph{Диакон:} Пресвятую, Пречистую, Преблагословенную, Славную Владычицу нашу Богородицу и Приснодеву Марию, со всеми святыми помянувше, сами себе, и друг друга, и весь живот наш Христу Богу предадим. 

\myemph{Хор:} Тебе, Господи. 

\myemph{Иерей:} Яко подобает Тебе всякая слава, честь и поклонение, Отцу и Сыну и Святому Духу, ныне и присно и во веки веков. 

\myemph{Хор:} Аминь.

\mysubtitle{Антифоны}

\myemph{Антифоны на литургии бывают трех родов: праздничные, изобразительные и вседневные (будничные). Какие из них поются, определяется на каждый день церковным Уставом. Праздничные антифоны поются в праздники Господни, за исключением Сретения (праздничные антифоны в Неделю ваий, на Пасху, на Вознесение и в День Святой Троицы приведены в главе «Песнопения из служб Триоди цветной»).}

\myemph{Вседневные антифоны положено петь в будни. Чаще всего в воскресные и праздничные дни, поются антифоны изобразительные (псалмы 102, 145 и Блаженны "--- Мф. 5, 3-12)}

\mysubtitle{Первый антифон} 

\myemph{Хор 1-й:} Благослови, душе моя, Господа. Благословен еси, Господи. Благослови, душе моя, Господа, и вся внутренняя моя Имя святое Его. 

\myemph{Хор 2-й:} Благослови, душе моя, Господа, и не забывай всех воздаяний Его. 

\myemph{Хор 1-й:} Очищающаго вся беззакония твоя, исцеляющаго вся недуги твоя. 

\myemph{Хор 2-й:} Избавляющаго от истления живот твой, венчающаго тя милостию и щедротами. 

\myemph{Хор 1-й:} Исполняющаго во благих желание твое: обновится, яко орля, юность твоя. 

\myemph{Хор 2-й:} Творяй милостыни Господь, и судьбу всем обидимым. 

\myemph{Хор 1-й:} Сказа пути Своя Моисеови, сыновом Израилевым хотения Своя. 

\myemph{Хор 2-й:} Щедр и милостив Господь, долготерпелив и многомилостив.

\myemph{Хор 1-й:} Не до конца прогневается, ниже в век враждует. 

\myemph{Хор 2-й:} Не по беззаконием нашим сотворил есть нам, ниже по грехом нашим воздал есть нам. 

\myemph{Хор 1-й:} Яко по высоте небесней от земли, утвердил есть Господь милость Свою на боящихся Его. 

\myemph{Хор 2-й:} Елико отстоят востоцы от запад, удалил есть от нас беззакония наша. 

\myemph{Хор 1-й:} Якоже щедрит отец сыны, ущедри Господь боящихся Его. 

\myemph{Хор 2-й:} Яко Той позна создание наше, помяну, яко персть есмы. 

\myemph{Хор 1-й:} Человек, яко трава дние его, яко цвет сельный, тако оцветет. 

\myemph{Хор 2-й:} Яко дух пройде в нем, и не будет, и не познает ктому места своего. 

\myemph{Хор 1-й:} Милость же Господня от века и до века на боящихся Его. 

\myemph{Хор 2-й:} И правда Его на сынех сынов, хранящих завет Его, и помнящих заповеди Его творити я. 

\myemph{Хор 1-й:} Господь на небеси уготова Престол Свой, и Царство Его всеми обладает. 

\myemph{Хор 2-й:} Благословите Господа, ангели Его, сильнии крепостию, творящии слово Его, услышати глас словес Его. 

\myemph{Хор 1-й:} Благословите Господа, вся силы Его, слуги Его, творящии волю Его. 

\myemph{Хор 2-й:} Благословите Господа, вся дела Его, на всяком месте владычества Его. 

\myemph{Хор 1-й:} Слава Отцу и Сыну и Святому Духу. 

\myemph{Хор 2-й:} И ныне и присно и во веки веков. Аминь. 

\myemph{Хор 1-й:} Благослови, душе моя, Господа, и вся внутренняя моя, имя святое Его. Благословен еси, Господи.

\mysubtitle{Ектения малая} 

\myemph{Диакон:} Паки и паки миром Господу помолимся. 

\myemph{Хор:} Господи, помилуй. 

\myemph{Диакон:} Заступи, спаси, помилуй и сохрани нас, Боже, Твоею благодатию. 

\myemph{Хор:} Господи, помилуй. 

\myemph{Диакон:} Пресвятую, Пречистую, Преблагословенную, Славную Владычицу нашу Богородицу и Приснодеву Марию, со всеми святыми помянувше, сами себе, и друг друга, и весь живот наш Христу Богу предадим. 

\myemph{Хор:} Тебе, Господи. 

\myemph{Иерей:} Яко Твоя держава, и Твое есть Царство, и сила, и слава, Отца и Сына и Святаго Духа, ныне и присно и во веки веков. 

\myemph{Хор:} Аминь. 

\mysubtitle{Второй антифон} 

\myemph{Во время второго антифона зажигается пономарская свеча. Алтарник берёт свечу во время «Единородный Сыне…» и становится с ней на горнем месте.}

\myemph{Хор:} Слава Отцу, и Сыну, и Святому Духу. 

\myemph{Хор 1-й:} Хвали, душе моя, Господа. Восхвалю Господа в животе моем, пою Богу моему, дондеже есмь. 

\myemph{Хор 2-й:} Не надейтеся на князи, на сыны человеческия, в них же несть спасения. 

\myemph{Хор 1-й:} Изыдет дух его, и возвратится в землю свою: в той день погибнут вся помышления его. 

\myemph{Хор 2-й:} Блажен, емуже Бог Иаковль помощник его, упование его на Господа Бога своего. 

\myemph{Хор 1-й:} Сотворшаго небо и землю, море и вся, яже в них. 

\myemph{Хор 2-й:} Хранящаго истину в век, творящаго суд обидимым, дающаго пищу алчущим. 

\myemph{Хор 1-й:} Господь решит окованныя, Господь умудряет слепцы. 

\myemph{Хор 2-й:} Господь возводит низверженныя, Господь любит праведники. 

\myemph{Хор 1-й:} Господь хранит пришельцы, сира и вдову приимет, и путь грешных погубит. 

\myemph{Хор 2-й:} Воцарится Господь во век, Бог твой, Сионе, в род и род. И ныне и присно и во веки веков. Аминь. 

\myemph{Хор:} И ныне и присно и во веки веков. Аминь. 

\mysubtitle{Песнь Господу Иисусу Христу} 

\myemph{Хор:} Единородный Сыне и Слове Божий, Безсмертен Сый, и изволивый спасения нашего ради воплотитися от Святыя Богородицы и Приснодевы Марии, непреложно вочеловечивыйся; распныйся же, Христе Боже, смертию смерть поправый, един Сый Святыя Троицы, спрославляемый Отцу и Святому Духу, спаси нас. 

\mysubtitle{Ектения малая} 

\myemph{Диакон:} Паки и паки миром Господу помолимся. 

\myemph{Хор:} Господи, помилуй. 

\myemph{Диакон:} Заступи, спаси, помилуй и сохрани нас, Боже, Твоею благодатию. 

\myemph{Хор:} Господи, помилуй. 

\myemph{Диакон:} Пресвятую, Пречистую, Преблагословенную, Славную Владычицу нашу Богородицу и Приснодеву Марию, со всеми святыми помянувше, сами себе, и друг друга, и весь живот наш Христу Богу предадим.

\myemph{Хор:} Тебе, Господи. 

\myemph{Иерей:} Яко Твоя держава, и Твое есть Царство, и сила, и слава, Отца и Сына и Святаго Духа, ныне и присно и во веки веков. 

\myemph{Хор:} Аминь.

\mysubtitle{Третий антифон; Блаженны} 

\myemph{«Блаженны» положено петь с тропарями, назначенными в этот день церковным Уставом: особыми тропарями на «Блаженных», или тропарями из песней утреннего канон празднику или святому.}

\myemph{Хор 1-й:} Во Царствии Твоем помяни нас, Господи, егда приидеши во Царствии Твоем. 

\myemph{Хор 2-й, стих 12-й:} Блажени нищий духом, яко тех есть Царство Небесное. 

\myemph{Хор 1-й:} Блажени плачущии, яко тии утешатся.

\myemph{Хор 2-й, стих 10-й:} Блажени кротции, яко тии наследят землю. 

\myemph{Хор 1-й:} Блажени алчущии и жаждущии правды, яко тии насытятся.

\myemph{Хор 2-й, стих 8-й:} Блажени милостивии, яко тии помиловани будут. 

\myemph{Хор 1-й:} Блажени чистии сердцем, яко тии Бога узрят.

\myemph{Хор 2-й, стих 6-й:} Блажени миротворцы, яко тии сынове Божии нарекутся. 

\myemph{Хор 1-й:} Блажени изгнани правды ради, яко тех есть Царство Небесное.

\myemph{Хор 2-й, стих 4-й:} Блажени есте, егда поносят вам, и изженут, и рекут всяк зол глагол на вы, лжуще Мене ради. 

\myemph{Хор 1-й:} Радуйтеся и веселитеся, яко мзда ваша многа на небесех. 

\myemph{Хор:} Слава Отцу и Сыну и Святому Духу. 

\myemph{Хор:} И ныне и присно и во веки веков. Аминь. 

\mysubtitle{Антифоны вседневные (будничные)} 

\myparagraph{Антифон 1-й}

\myemph{Хор 1-й:} Благо есть исповедатися Господеви. Молитвами Богородицы, Спасе, спаси нас. 

\myemph{Хор 2-й:} Благо есть исповедатися Господеви, и пети имени Твоему, Вышний. Молитвами Богородицы, Спасе, спаси нас. 

\myemph{Хор 1-й:} Возвещати заутра милость Твою, и истину Твою на всяку нощь. Молитвами Богородицы, Спасе, спаси нас. 

\myemph{Хор 2-й:} Яко прав Господь Бог наш, и несть неправды в Нем. Молитвами Богородицы, Спасе, спаси нас.

\myemph{Хор 1-й:} Слава Отцу и Сыну и Святому Духу: Молитвами Богородицы, Спасе, спаси нас.

\myemph{Хор 2-й:} И ныне и присно и во веки веков. Аминь. Молитвами Богородицы, Спасе, спаси нас.

\myparagraph{Антифон 2-й}

\myemph{Хор 1-й:} Господь воцарися, в лепоту облечеся. Молитвами святых Твоих, Спасе, спаси нас. 

\myemph{Хор 2-й:} Господь воцарися, в лепоту облечеся, облечеся Господь в силу, и препоясася. Молитвами святых Твоих, Спасе, спаси нас.

\myemph{Хор 1-й:} Ибо утверди вселенную, яже не подвижится. Молитвами святых Твоих, Спасе, спаси нас.

\myemph{Хор 2-й:} Свидения Твоя уверишася зело: дому Твоему подобает святыня, Господи, в долготу дний. Молитвами святых Твоих, Спасе, спаси нас.

\myemph{Хор:} Слава Отцу и Сыну и Святому Духу. 

\myemph{Хор:} И ныне и присно и во веки веков. Аминь. 

\myparagraph{ Песнь Господу Иисусу Христу}

\myemph{Хор:} Единородный Сыне и Слове Божий, Безсмертен Сый, и изволивый спасения нашего ради воплотитися от Святыя Богородицы и Приснодевы Марии, непреложно вочеловечивыйся; распныйся же, Христе Боже, смертию смерть поправый, един Сый Святыя Троицы, спрославляемый Отцу и Святому Духу, спаси нас. 

\myparagraph{Антифон 3-й}

\myemph{Хор 1-й:} Приидите возрадуемся Господеви, воскликнем Богу Спасителю нашему. Спаси ны, Сыне Божий, во святых дивен сый, поющия Ти: аллилуиа. 

\myemph{Хор 2-й:} Предварим лице Его во исповедании, и во псалмех воскликнем Ему: Спаси ны, Сыне Божий, во святых дивен сый, поющия Ти: аллилуиа. 

\myemph{Хор 1-й:} Яко Бог Велий Господь, и Царь Велий по всей земли. Спаси ны, Сыне Божий, во святых дивен сый, поющия Ти: аллилуиа. 

\myemph{Хор 2-й:} Яко в руце Его вси концы земли, и высоты гор Того суть. Спаси ны, Сыне Божий, во святых дивен сый, поющия Ти: аллилуиа. 

\myemph{Хор 1-й:} Яко Того есть море, и Той сотвори е, и сушу руце Его создаете. Спаси ны, Сыне Божий, во святых дивен сый, поющия Ти: аллилуиа. 

\mysubtitle{Вход с Евангелием} 

\myemph{Вход с Евангелием. Дьякон заходит в алтарь, открывает Царские врата, вместе со священником крестится и целует престол и берёт евангелие, алтарник в этот момент крестится с ними синхронно, кланяется, горнему месту, священнику и в момент перехода священника от престола к горнему месту идёт к северным вратам. Когда священник с диаконом тоже направятся к вратам, открывает дверь и по амвону проходит до царских врат, затем сворачивает к аналою и становится перед ним спиной к народу, когда священник зайдёт в алтарь, алтарник заходит через южные врата. В алтаре понамарь проходит до горнего места, крестится, кланяется горнему месту, священнику и проходит, чтобы поставить свечу на место.}

\myemph{Диакон:} Премудрость, пр\'{о}сти.

\myemph{Хор:} Приидите, поклонимся и припадем ко Христу. Спаси Сыне Божий, воскресый из мертвых, поющия Ти: аллилуиа.

\myemph{[B таком виде это песнопение поется во все обычные воскресенья, на Пасху и все дни пасхальной седмицы. Вместо слов «воскресый из мертвых» в будни поется «во святых Дивен сын», а на праздники "--- по смыслу праздника, как указано в Церковном уставе: на Рождество "--- «рождейся от Девы»; на Крещение "--- «во Иордане крестивыйся»; на Вознесение "--- «вознесыйся ко славе» в Пятидесятницу и в День Святого Духа "--- «Спаси ны. Утешителю Благий»; на Преображение "--- «преобразивыйся на горе»; на Воздвижение "--- «плотию распныйся»; в Неделю ваий "--- «возседый на жребя». В праздники Богородицы "--- «молитвами Богородицы». Праздничные входные стихи поются и в дни попразднства, до отдания.]}

\mysubtitle{Тропари и кондаки «по входе»} 

\myemph{Хор поет тропари и кондаки «по входе», назначенные в этот день церковным Уставом (воскресные тропари и кондаки приведены в главе «Песнопения из служб воскресных», дневные "--- в главе «Песнопения из служб будничных», общие ликам святых "--- в главе «Песнопения из служб общих ликам святых», праздничные "--- в главе «Песнопения из служб праздничных»).}

\myemph{Иерей:} Яко Свят еси, Боже наш, и Тебе славу возсылаем, Отцу и Сыну и Святому Духу, ныне и присно.

\myemph{Диакон:} И во веки веков.

\myemph{Хор:} Аминь.

\mysubtitle{Трисвятое} 

\myemph{[В праздники Рождества Христова, Богоявления, в Лазареву и Великую субботы, во все дни пасхальной седмицы и в период Пятидесятницы вместо Трисвятого поется: «Елицы во Христа крестистеся, во Христа облекостеся. Аллилуя». В праздник Воздвижения Креста Господня и в Неделю крестопоклонную поется: «Кресту Твоему поклоняемся Владыко, и Святое Воскресение Твое славим»]}

\myemph{Хор:} Святый Боже, Святый Крепкий, Святый Безсмертный, помилуй нас. \myemph{(Трижды)}

\myemph{Хор:} Слава Отцу и Сыну и Святому Духу, и ныне и присно и во веки веков. Аминь.

\myemph{Хор:} Святый Безсмертный, помилуй нас.

\myemph{Хор:} Святый Боже, Святый Крепкий, Святый Бессмертный, помилуй нас.

\mysubtitle{Прокимен} 

\myemph{Дьякону подаётся кадило}

\myemph{Диакон:} Вонмем.

\myemph{Иерей:} Мир всем.

\myemph{Чтец Апостола:} И духови твоему. Прокимен. Псалом Давидов, глас..

\myemph{[В Богородичные праздники: «Прокимен, песнь Богородицы: Величит душа Моя Господа и возрадовася дух Мой о Бозе Спасе Моем».]}

\myemph{Произносится один или два прокимна, назначенные в этот день на литургии церковным Уставом (воскресные прокимны со своими стихами приведены в главе «Песнопения из служб воскресных восьми гласов», дневные (будничные) "--- в главе «Песнопения из служб будничных», из служб Триодей постной и цветной "--- в главах «Песнопения из служб Триоди постной» и «Песнопения из служб Триоди цветной».}

\myemph{Чтец произносит Прокимен, называя глас его, хор поет прокимен, чтец произносит стих, хор повторяет прокимен, чтец произносит первую половину прокимна, хор поет вторую половину его. Когда Устав назначает два прокимна, первый поется дважды, т.~е. чтец: прокимен, хор: прокимен, чтец: стих, хор: прокимен, затем чтец произносит второй прокимен, и хор поет его один раз.}

\mysubtitle{Прокимны и аллилуиарии воскресные на литургии}

\myemph{Глас 1-й:} Буди, Господи, милость Твоя на нас, якоже уповахом на Тя.

\myemph{Стих:} Радуйтеся, праведнии, о Господе, правым подобает похвала.

\myemph{Аллилуиа:} Бог даяй отмщение мне и покоривый люди под мя.

\myemph{Стих:} Величай спасения царева и творяй милость Христу Своему Давиду и семени его до века.

\myemph{Глас 2-й:} Крепость моя и пение мое Господь. и бысть мне во спасение.

\myemph{Стих:} Наказуя наказа мя Господь, смерти же не предаде мя.

\myemph{Аллилуиа:} Услышит тя Господь в день печали, защитит тя имя Бога Иаковля.

\myemph{Стих:} Господи, спаси царя и услыши ны, в оньже аще день призовем Тя.

\myemph{Глас 3-й:} Пойте Богу нашему, пойте пойте Цареви нашему, пойте.

\myemph{Стих:} Вси языцы, восплещите руками, воскликните Богу гласом радования.

\myemph{Аллилуиа:} На Тя, Господи, уповах, да не постыжуся во век.

\myemph{Стих:} Буди ми в Бога Защитителя и в дом прибежища, еже спасти мя.

\myemph{Глас 4-й:} Яко возвеличишася дела Твоя, Господи, вся премудростию сотворил еси.

\myemph{Стих:} Благослови, душе моя, Господа, Господи Боже мой, возвеличился еси зело.

\myemph{Аллилуиа:} Наляцы и успевай и царствуй, истины ради и кротости, и правды.

\myemph{Стих:} Возлюбил еси правду и возненавидел еси беззконие.

\myemph{Глас 5-й:} Ты, Господи, сохраниши ны и соблюдеши ны от рода сего и во век.

\myemph{Стих:} Спаси мя, Господи, яко оскуде преподобный.

\myemph{Аллилуиа:} Милости Твоя, Господи, во век воспою, в род и род возвещу истину Твою усты моими.

\myemph{Стих:} Зане рекл еси: в век милость созиждется, на небесех уготовится истина Твоя.

\myemph{Глас 6-п:} Спаси, Господи, люди Твоя и благослови достояние Твое.

\myemph{Стих:} К Тебе, Господи, воззову, Боже мой, да не премолчиши от мене.

\myemph{Аллилуиа:} Живый в помощи Вышняго, в крове Бога Небеснаго водворится.

\myemph{Стих:} Речет Господеви: Заступник мой еси и Прибежище мое, Бог мой, и уповаю на Него.

\myemph{Глас 7-й:} Господь крепость людем Своим даст Господь благословит люди Своя миром.

\myemph{Стих:} Принесите Господеви сынове Божии, принесите Господеви сыны овни,

\myemph{Аллилуиа:} Благо есть исповедатися Господеви и пети Имени Твоему, Вышний.

\myemph{Стих:} Возвещати заутра милость Твою, и истину Твою на всяку нощь.

\myemph{Глас 8-й:} Помолитеся и воздадите Господеви Богу нашему.

\myemph{Стих:} Ведом во Иудеи Бог, во Израили велие Имя Его.

\myemph{Аллилуиа:} Приидите, возрадуемся Господеви, воскликнем Богу Спасителю нашему.

\myemph{Стих:} Предварим лице Его во исповедании, и во псалмех воскликнем Ему.

\mysubtitle{Прокимны и аллилуиарии дневные (будничные)}

\myemph{В понедельник, гл. 4-й:} Творяй ангелы Своя духи, и слуги Своя пламень огненный.

\myemph{Стих:} Благослови, душе моя. Господа, Господи Боже мой, возвеличился еси зело.

\myemph{Аллилуиа, гл. 5-й:} Хвалите Господа, вси ангели Его, хвалите Его, вся силы Его.

\myemph{Стих:} Яко Той рече, и быша; Той повеле, и создашася.

\myemph{Во вторник, гл. 7-й:} Возвеселится праведник о Господе и уповает на Него.

\myemph{Стих:} Услыши, Боже, глас мой, внегда молитися ми к Тебе.

\myemph{Аллилуиа, гл. 4-й:} Праведник яко финикс процветет, яко кедр, иже в Ливане, умножится

\myemph{Стих:} Насаждени в дому Господни, во дворех Бога нашего процветут.

\myemph{В среду, гл. 3-й:} Величит душа Моя Господа, и возрадовася дух Мой о Бозе Спасе Моем

\myemph{Стих:} Яко призре на смирение рабы Своея, се бо отныне ублажат Мя вси роди.

\myemph{Аллилуиа, гл. 8-й:} Слыши, Дщи, и виждь, и приклони ухо Твое.

\myemph{Стих:} Лицу Твоему помолятся богатии людстии.

\myemph{В четверг, гл. 8-й:} Во всю землю изыде вещание их, и в концы вселенныя глаголы их.

\myemph{Стих:} Небеса поведают славу Божию, творение же руку Его возвещает твердь.

\myemph{Аллилуиа, гл. 1-й;} Исповедят небеса чудеса, Господи, ибо истину Твою в Церкви святых

\myemph{Стих:} Бог прославляем в совете святых.

\myemph{В пятницу, гл. 7-й:} Возносите Господа Бога нашего, и покланяйтеся подножию ногу Его, яко свято есть.

\myemph{Стих:} Господь воцарися, да гневаются людие.

\myemph{Аллилуиа, гл. 1-й:} Помяни сонм Твой, егоже стяжал еси исперва.

\myemph{Стих:} Бог же Царь наш прежде века, содела спасение посреди земли. 

\myemph{В субботу, гл. 8-й:} Веселитеся о Господе, и радуйтеся, праведнии.

\myemph{Стих:} Блажени, ихже оставишася беззакония и ихже прикрышася греси.

\myemph{Заупокойный, гл. 6-й:} Души их во благих водворятся.

\myemph{Аллилуиа, гл. 4-й:} Воззваша праведнии, и Господь услыша их, и от всех скорбей их избави их.

\myemph{Стих:} Многи скорби праведным, и от всех их избавит я Господь.

\myemph{Стих:} Блажени, яже избрал и приял еси, Господи, и память их в род и род.

\myemph{Диакон:} Премудрость.

\myemph{Чтец:} Деяний святых апостол чтение. \myemph{Или:} Соборнаго послания Петрова \myemph{[или:} Иоаннова\myemph{, причем не принято говорить, какое это послание "--- первое, или второе, или третье}] чтение. \myemph{Или:} К римляном [К коринфяном; К галатом; К Тимофею\myemph{и т.п.}] послания святаго апостола Павла чтение.

\myemph{Диакон:} Вонмем.

\mysubtitle{Чтение Апостола} 

\myemph{Во время чтения апостола, ставится аналой на амвоне для Евангелия. Когда чтение закончится, иерей говорит чтецу:} Мир ти.

\myemph{Чтец:} И духови твоему.

\mysubtitle{Аллилуиа} 

\myemph{Диакон:} Премудрость.

\myemph{Чтец:} Аллилуиа, глас… \myemph{Если прислуживает один алтарник, то выносится пономарская свеча и ставится перед аналоем (с Евангелием), если два алтарника то во время пения аллилуя вдвоём подходят на горнее место со свечами, синхронно крестятся, кланяются горнему месту, священнику, друг другу, и выходят на амвон северными и южными вратами, до чтения евангелия стоят лицом к иконостасу, не кланяясь и не крестясь, в начале чтения поворачиваются лицом к евангелию, по окончанию кланяются иконам и заходят теми же вратами в алтарь, также крестятся и кланяются на горнем месте и проходят, чтобы поставить свечи на место. Не забудьте убрать аналой.}

\myemph{Хор поет «Аллилуиа» "--- трижды на указанный глас, чтец произносит первый стих аллилуиария, хор: «Аллилуиа», чтец произносит второй стих аллилуиария, хор поет в третий раз «Аллилуиа». В богослужебных книгах перед первым стихом аллилуария пишется «Аллилуиа, глас…», а перед вторым "--- «Стих» (воскресные аллилуиарии приведены в главе «Песнопения из служб воскресных восьми гласов», дневные (будничные) "--- в главе «Песнопения из служб будничных», аллилуиарии из служб Триодей постной и цветной "--- в главах «Песнопения из служб Триоди постной» и «Песнопения из служб Триоди цветной».)} 

\myemph{Диакон:} Благослови, владыко, благовестителя святаго апостола и евангелиста \myemph{(имя евангелиста)}.

\myemph{Иерей, благословляя его, произносит:} Бог, молитвами святаго, славнаго, всехвальнаго апостола и евангелиста \myemph{(имярек)}, да даст тебе глагол, благовествующему силою многою, во исполнение Евангелиа Возлюбленнаго Сына Своего, Господа нашего Иисуса Христа.

\myemph{Диакон:} Аминь.

\myemph{Иерей:} Премудрость, пр\'{о}сти, услышим святаго Евангелиа. Мир всем.

\myemph{Хор:} И духови твоему.

\myemph{Диакон:} От \myemph{(имя)} святаго Евангелиа чтение.

\myemph{Хор}: Слава Тебе, Господи, слава Тебе.

\myemph{Иерей:} Вонмем.

\mysubtitle{Чтение Евангелия}

\myemph{Читается Евангелие. Церковный Устав назначает определенные евангельские чтения на каждый день (евангельские чтения Пресвятой Богородице о общие ликам святых приведены в главе «Песнопения из служб общих ликам святых»).}

\myemph{По окончании чтения хор:} Слава Тебе, Господи, слава Тебе.

\myemph{Выносятся записки о здравии и о упокоении.}

\mysubtitle{Ектения сугубая}

\myemph{Диакон:} Рцем вси от всея души, и от всего помышления нашего рцем.

\myemph{Хор:} Господи, помилуй.

\myemph{Диакон:} Господи Вседержителю, Боже отец наших, молим Ти ся, услыши и помилуй.

\myemph{Хор:} Господи, помилуй.

\myemph{Диакон:} Помилуй нас, Боже, по велицей милости Твоей, молим Ти ся, услыши и помилуй.

\myemph{Хор:} Господи, помилуй. \myemph{(Трижды)} 

\myemph{Диакон:} Еще молимся о Великом Господине и Отце нашем Святейшем Патриархе \myemph{(имярек)}, и о Господине нашем Преосвященнейшем митрополите \myemph{(или:} архиепископе, \myemph{или:} епископе \myemph{) (имярек)}, и всей во Христе братии нашей. 

\myemph{Хор:} Господи, помилуй. \myemph{(Трижды)} 

\myemph{Диакон:} Еще молимся о Богохранимей стране нашей, властех и воинстве ея, да тихое и безмолвное житие поживем во всяком благочестии и чистоте. 

\myemph{Хор:} Господи, помилуй. \myemph{(Трижды)} 

\myemph{Диакон:} Еще молимся о братиях наших, священицех, священномонасех и всем во Христе братстве нашем.

\myemph{Хор:} Господи, помилуй. \myemph{(Трижды)} 

\myemph{Диакон:} Еще молимся о блаженных и приснопамятных создателех святаго храма сего \myemph{(если в монастыре:} святыя обители сея), и о всех преждепочивших отцех и братиях, зде лежащих и повсюду, православных. 

\myemph{Хор:} Господи, помилуй. \myemph{(Трижды)} 

\myemph{Диакон:} Еще молимся о милости, жизни, мире, здравии, спасении, посещении, прощении и оставлении грехов рабов Божиих, братии святаго храма сего ( \myemph{если в монастыре:} святыя обители сея). 

\myemph{Хор:} Господи, помилуй. \myemph{(Трижды)} 

\myemph{Диакон:} Еще молимся о плодоносящих и добродеющих во святем и всечестнем храме сем, труждающихся, поющих и предстоящих людех, ожидающих от Тебе великия и богатыя милости.

\myemph{Хор:} Господи, помилуй. \myemph{(Трижды)} 

\myemph{Иерей:} Яко Милостив и Человеколюбец Бог еси, и Тебе славу возсылаем, Отцу и Сыну и Святому Духу, ныне и присно и во веки веков.

\myemph{Хор}: Аминь.

\mysubtitle{Ектения заупокойная} 

\myemph{[В некоторые дни церковного года (кроме двунадесятых и храмовых праздников) за сугубой ектенией читается следующая ектения об усопших, при открытых царских вратах и с кадильницей:]}

\myemph{Диакон:} Помилуй нас, Боже, по велицей милости Твоей, молим Ти ся, услыши и помилуй.

\myemph{Хор:} Господи помилуй. \myemph{(трижды)}. 

\myemph{Диакон:} Еще молимся о упокоении душ усопших рабов Божиих \myemph{(имена)} и о еже проститися им всякому прегрешению, вольному же и невольному.

\myemph{Хор:} Господи помилуй. \myemph{(трижды)}. 

\myemph{Диакон:} Яко да Господь Бог учинит души их, идеже праведнии упокояются.

\myemph{Хор:} Господи помилуй. \myemph{(трижды)}. 

\myemph{Диакон:} Милости Божия, Царства Небеснаго и оставления грехов их у Христа, Безсмертнаго Царя и Бога нашего, просим.

\myemph{Хор:} Подай, Господи.

\myemph{Диакон:} Господу помолимся.

\myemph{Хор:} Господи, помилуй.

\myemph{Иерей:} Яко Ты еси воскресение, и живот, и покой усопших раб Твоих \myemph{(имена рек)}, Христе Боже наш, и Тебе славу возсылаем, со Безначальным Твоим Отцем и Пресвятым и Благим и Животворящим Твоим Духом, ныне и присно и во веки веков.

\myemph{Хор}: Аминь. \myemph{Царские врата закрываются.}

\mysubtitle{Ектения об оглашенных} 

\myemph{Диакон:} Помолитеся, оглашеннии, Господеви.

\myemph{Хор:} Господи, помилуй.

\myemph{Диакон:} Вернии, о оглашенных помолимся, да Господь помилует их.

\myemph{Хор:} Господи, помилуй.

\myemph{Диакон:} Огласит их словом истины.

\myemph{Хор:} Господи, помилуй.

\myemph{Диакон:} Открыет им Евангелие правды.

\myemph{Хор:} Господи, помилуй.

\myemph{Диакон:} Соединит их Святей Своей, Соборней и Апостольстей Церкви.

\myemph{Хор:} Господи, помилуй.

\myemph{Диакон:} Спаси, помилуй, заступи и сохрани их, Боже, Твоею благодатию.

\myemph{Хор:} Господи, помилуй.

\myemph{Диакон:} Оглашеннии, главы ваша Господеви приклоните.

\myemph{Хор:} Тебе, Господи.

\myemph{Иерей:} Да и тии с нами славят пречестное и великолепое Имя Твое, Отца и Сына и Святаго Духа, ныне и присно и во веки веков.

\myemph{Хор:} Аминь.

\myemph{Диакон:} Елицы оглашеннии, изыдите, оглашеннии, изыдите; елицы оглашеннии, изыдите. Да никто от оглашенных, елицы вернии, паки и паки миром Господу помолимся.

\myemph{Хор:} Господи, помилуй.

\myemph{Возглашением диакона: «Оглашеннии, изыдите…» заканчивается вторая часть литургии.}

\end{mymulticols}

\mychapterending

\mychapter{Литургия верных}
%http://www.molitvoslov.org/node/456 

\textbf{Литургия верных} "--- третья, самая важная часть литургии, на которой Св. Дары, приготовленные на проскомидии, силою и действием Святого Духа пресуществляются в Тело и Кровь Христовы и возносятся в спасительную для людей жертву Богу Отцу, а затем преподаются верующим для причащения. Эта часть литургии получила название оттого, что присутствовать при ее совершении и приступать к причащению Св. Тайн могут только верные, то есть лица, принявшие православную веру через Св. Крещение и оставшиеся верными обетам, данным при Св. Крещении.

На литургии верных воспоминаются страдания Господа Иисуса Христа, Его смерть, погребение, Воскресение, Вознесение на небо, седение одесную Бога Отца и второе славное пришествие на землю.

В состав этой части литургии входят важнейшие священнодействия:

\begin{enumerate}

\item Перенесение Честных Даров с жертвенника на престол, приготовление верующих молитвенному участию при совершении Бескровной Жертвы.

\item Самое совершение Таинства, с молитвенным воспоминанием членов Церкви Небесной и земной. 

\item Приготовление к причащению и причащение священнослужителей и мирян.

\item Благодарение за причащение и благословение на исход из храма (отпуст).

\end{enumerate}

\begin{mymulticols}

\mysubtitle{Ектении}

\myemph{Диакон от лица верных произносит две ектении:}

\myemph{Диакон:} Заступи, спаси, помилуй и сохрани нас, Боже, Твоею благодатию.

\myemph{Хор:} Господи, помилуй.

\myemph{Диакон:} Премудрость.

\myemph{Иерей:} Яко подобает Тебе всякая слава, честь и поклонение, Отцу и Сыну и Святому Духу, ныне и присно и во веки веков.

\myemph{Хор:} Аминь.

\myemph{Готовится кадило, зажигаются понамарская свеча.}

\myemph{Диакон:} Паки и паки миром Господу помолимся.

\myemph{Хор:} Господи, помилуй.

\myemph{Диакон:} О свышнем мире и спасении душ наших Господу помолимся. 

\myemph{Хор:} Господи, помилуй.

\myemph{Диакон:} О мире всего мира, благостоянии святых Божиих Церквей и соединении всех Господу помолимся. 

\myemph{Хор:} Господи, помилуй.

\myemph{Диакон:} О святем храме сем и с верою, благоговением и страхом Божиим входящих в онь Господу помолимся. 

\myemph{Хор:} Господи, помилуй.

\myemph{Диакон:} О избавитися нам от всякия скорби, гнева и нужды Господу помолимся. 

\myemph{Хор:} Господи, помилуй.

\myemph{Диакон:} Заступи, спаси, помилуй и сохрани нас, Боже, Твоею благодатию. 

\myemph{Диакон:} Премудрость. 

\myemph{Иерей:} Яко да под державою Твоею всегда храними, Тебе славу возсылаем, Отцу и Сыну и Святому Духу, ныне и присно и во веки веков. 

\myemph{(отверзаются царские врата.)}

\myemph{Подаётся кадило. Алтарник становится на горнем месте так, чтобы не мешать каждению. Когда каждение закончится и дьякон зайдёт в алтарь, синхронно крестится и кланяется со священнослужителями, на третий раз кланяется как обычно (горнее место, священник) и проходит к северным вратам. По сигналу священника открывает дверь и выходит как обычно к аналою. Стоит перед аналоем до тех пор пока не закроются Царские врата. По обычаю заходит в Алтарь}. 

\myemph{Хор:} Аминь, \myemph{и поет Херувимскую песнь} 

\mysubtitle{Херувимская песнь} 

Иже Херувимы тайно образующе и Животворящей Троице Трисвятую песнь припевающе, всякое ныне житейское отложим попечение… 

\mysubtitle{ Великий вход} 

\myemph{Диакон и иерей, взяв Святые Дары, выходят из алтаря на солею.}

\myemph{Диакон:} Великаго Господина и Отца нашего \myemph{(имярек),} Святейшаго Патриарха Московскаго и всея Руси, и Господина нашего Преосвященнейшаго \myemph{(имя епархиального архиерея),} да помянет Господь Бог во Царствии Своем, всегда, ныне и присно и во веки веков.

\myemph{Иерей:} Вас и всех православных христиан да помянет Господь Бог во Царствии Своем, всегда, ныне и присно и во веки веков.

\myemph{Хор:} Аминь. Яко да Царя всех подымем, ангельскими невидимо дароносима чинми. Аллилуиа, аллилуиа, аллилуиа.

\myemph{[ Вместо Херувимской на литургии в Великий четверг поется} «Вечери Твоея Тайныя…», \myemph{а в Великую субботу "---} «Да молчит всякая плоть…» \myemph{(эти песнопения приведены в главе «Песнопения из служб Триоди постной»).]}

\myemph{Режутся просфоры для причастников}. 

\mysubtitle{Просительная ектения} 

\myemph{Диакон:} Исполним молитву нашу Господеви. 

\myemph{Хор:} Господи, помилуй.

\myemph{Диакон:} О предложенных честных Дарех Господу помолимся. 

\myemph{Хор:} Господи, помилуй.

\myemph{Диакон:} О святем храме сем и с верою, благоговением и страхом Божиим входящих в онь Господу помолимся. 

\myemph{Хор:} Господи, помилуй.

\myemph{Диакон:} О избавитися нам от всякия скорби, гнева и нужды Господу помолимся. 

\myemph{Хор:} Господи, помилуй.

\myemph{Диакон:} Заступи, спаси, помилуй и сохрани нас, Боже, Твоею благодатию. 

\myemph{Хор:} Господи, помилуй.

\myemph{Диакон:} Дне всего совершенна, свята, мирна и безгрешна у Господа просим. 

\myemph{Хор:} Подай, Господи. 

\myemph{Диакон:} Ангела мирна, верна наставника, хранителя душ и телес наших у Господа просим. 

\myemph{Хор:} Подай, Господи. 

\myemph{Диакон:} Прощения и оставления грехов и прегрешений наших у Господа просим. 

\myemph{Хор:} Подай, Господи. 

\myemph{Диакон:} Добрых и полезных душам нашим и мира мирови у Господа просим. 

\myemph{Хор:} Подай, Господи. 

\myemph{Диакон:} Прочее время живота нашего в мире и покаянии скончати, у Господа просим. 

\myemph{Хор:} Подай, Господи. 

\myemph{Диакон:} Христианския кончины живота нашего, безболезнены, непостыдны, мирны, и добраго ответа на Страшном Судищи Христове просим. 

\myemph{Хор:} Подай, Господи. 

\myemph{Диакон:} Пресвятую, Пречистую, Преблагословенную, Славную Владычицу нашу Богородицу и Приснодеву Марию, со всеми святыми помянувше, сами себе и друг друга, и весь живот наш Христу Богу предадим.

\myemph{Хор:} Тебе, Господи.

\myemph{Иерей:} Щедротами Единороднаго Сына Твоего, с Нимже благословен еси, со Пресвятым и Благим и Животворящим Твоим Духом, ныне и присно, и во реки веков.

\myemph{Хор:} Аминь.

\myemph{Иерей:} Мир всем.

\myemph{Хор:} И духови твоему.

\myemph{Диакон:} Возлюбим друг друга, да единомыслием исповемы.

\myemph{Хор:} Отца и Сына и Святаго Духа, Троицу Единосущую и Нераздельную.

\myemph{Диакон:} Двери, двери, премудростию вонмем. \myemph{(Открывается завеса царских врат.)}

\myemph{Ставится кипятиться чайник.} 

\mysubtitle{ Символ веры} 

\myemph{Хор (или все молящиеся):}

\begin{enumerate}

\item Верую во Единаго Бога Отца Вседержителя, Творца небу и земли, видимым же всем и невидимым. 

\item И во Единаго Господа Иисуса Христа, Сына Божия, Единороднаго, Иже от Отца рожденнаго прежде всех век. Света от Света, Бога истинна от Бога истинна, рожденна, несотворенна, единосущна Отцу, Имже вся быша.

\item Нас ради, человек, и нашего ради спасения сшедшаго с Небес, и воплотившагося от Духа Свята и Марии Девы, и вочеловечшася.

\item Распятаго же за ны при Понтийстем Пилате, и страдавша, и погребенна.

\item И воскресшаго в третий день по Писанием.

\item И восшедшаго на Небеса, и седяща одесную Отца.

\item И паки грядущаго со славою судити живым и мертвым, Его же Царствию не будет конца.

\item И в Духа Святаго, Господа Животворящаго, Иже от Отца исходящаго, Иже со Отцем и Сыном споклоняема и сславима, глаголавшаго пророки.

\item Во едину Святую Соборную и Апостольскую Церковь.

\item Исповедую едино Крещение во оставление грехов.

\item Чаю воскресения мертвых,

\item и жизни будущаго века. Аминь. 

\end{enumerate}

\mysubtitle{ Евхаристический канон.} 

\myemph{Диакон:} Станем добре, станем со страхом, вонмем, Святое Возношение в мире приносити.

\myemph{Хор:} Милость мира, Жертву хваления. 

\myemph{Иерей:} Благодать Господа нашего Иисуса Христа, и любы Бога и Отца, и причастие Святаго Духа, буди со всеми вами.

\myemph{Хор:} И со духом твоим. 

\myemph{Иерей:} Горе имеим сердца. 

\myemph{Хор:} Имамы ко Господу. 

\myemph{Иерей:} Благодарим Господа. 

\myemph{Хор:} Достойно и праведно есть покланятися Отцу и Сыну и Святому Духу, Троице Единосущной и Нераздельней.

\myemph{Иерей:} Победную песнь поюще, вопиюще, взывающе и глаголюще:

\myemph{Хор:} Свят, Свят, Свят Господь Саваоф, исполнь Небо и земля славы Твоея; осанна в вышних, благословен Грядый во Имя Господне, осанна в вышних.

\myemph{Иерей:} Приимите, ядите, Сие есть Тело Мое, еже за вы ломимое во оставление грехов.

\myemph{Хор:} Аминь.

\myemph{Иерей:} Пийте от нея вси, сия есть Кровь Моя Новаго Завета, яже за вы и за многи изливаемая во оставление грехов.

\myemph{Хор:} Аминь. 

\myemph{(На литургии св. Василия Великого последние возгласа иерея начинаются словами: «Даде святым Своим учеником и апостолом, рек:».)}

\myemph{Иерей:} Твоя от Твоих Тебе приносяще о всех и за вся. \myemph{Готовится кадило}.

\myemph{Хор:} Тебе поем, Тебе благословим, Тебе благодарим, Господи, и молим Ти ся. Боже наш. \myemph{Подаётся кадило во время «Тебе поем…», после слов священника в алтаре «Приложив Духом Твоим Святым. Аминь. Аминь. Аминь.»}

\myemph{Иерей:} Изрядно о Пресвятей, Пречистей, Преблагословенней, Славней Владычице нашей Богородице и Приснодеве Марии.

\myemph{Хор:} Достойно есть, яко воистинну блажити Тя, Богородицу, Присноблаженную и Пренепорочную и Матерь Бога нашего. Честнейшую Херувим и Славнейшую без сравнения Серафим, без истления Бога Слова рождшую, сущую Богородицу Тя величаем. 

\myemph{[В двунадесятые праздники и их попразднства вместо «Достойно…» поется припев и ирмос 9-й песни праздничного канона, так называемый «задостойник». В Великий четверг поется ирмос 9-й песни «Странствия Владычня…», в Великую субботу "--- «Не рыдай Мене, Мати…», в Неделю ваий "--- «Бог Господь…» (эти песнопения приведены в главах «Песнопения из служб Триоди постной» и «Песнопения из служб Триоди цветной»).}

\myemph{Если же литургия св. Василия Великого, вместо «Достойно…» поем:} 

О Тебе радуется, Благодатная, всякая тварь, ангельский собор и человеческий род, освященный храме и раю словесный, девственная похвало, из Неяже Бог воплотися и Младенец бысть, прежде век сый Бог наш; ложесна бо Твоя престол сотвори и чрево Твое пространнее небес содела. О Тебе радуется, Благодатная, всякая тварь, слава Тебе. \myemph{]}

\myemph{Иерей:} В первых помяни Господи, Великаго Господина и Отца нашего \myemph{(имярек),} Святейшаго Патриарха Московскаго и всея Руси, и Господина нашего Преосвященнейшаго \myemph{(имя епархиального епископа)}, ихже даруй святым Твоим Церквам в мире, целых, честных, здравых, долгоденствующих, право правящих слово Твоей истины. 

\myemph{Хор:} И всех и вся.

\myemph{Иерей:} И даждь нам единеми усты и единем сердцем славити и воспевати Пречестное и Великолепное Имя Твое, Отца и Сына и Святаго Духа, ныне и присно и во веки веков. 

\myemph{Хор:} Аминь.

\myemph{Иерей:} И да будут милости Великаго Бога и Спаса нашего Иисуса Христа со всеми вами.

\myemph{Хор:} И со духом твоим.

\myemph{Готовится чаша для теплоты и плат для причастия}. 

\mysubtitle{ Ектения просительная} 

\myemph{Диакон:} Вся святыя помянувше, паки и паки миром Господу помолимся. 

\myemph{Хор:} Господи, помилуй.

\myemph{Диакон:} О принесенных и освященных Честных Дарех, Господу помолимся. 

\myemph{Хор:} Господи, помилуй.

\myemph{Диакон:} Яко да Человеколюбец Бог наш, приемь я во святый, и пренебесный, и мысленный Свой Жертвенник, в воню благоухания духовнаго, возниспослет нам Божественную благодать и дар Святаго Духа, помолимся. 

\myemph{Хор:} Господи, помилуй.

\myemph{Диакон:} О избавится нам от всякия скорби, гнева и нужды, Господу помолимся. 

\myemph{Хор:} Господи, помилуй.

\myemph{Диакон:} Заступи, спаси, помилуй и сохрани нас, Твоею благодатию. 

\myemph{Хор:} Господи, помилуй.

\myemph{Диакон:} Дне всего совершенна, свята, мирна и безгрешна, у Господа просим. 

\myemph{Хор:} Подай, Господи.

\myemph{Диакон:} Ангела мирна, верна наставника, хранителя душ и телес наших, у Господа просим. 

\myemph{Хор:} Подай, Господи.

\myemph{Диакон:} Прощения и оставления грехов и прегрешений наших, у Господа просим. 

\myemph{Хор:} Подай, Господи.

\myemph{Диакон:} Добрых и полезных душам нашим и мира мирови, у Господа просим. 

\myemph{Хор:} Подай, Господи.

\myemph{Диакон:} Прочее время живота нашего в мире и покаянии скончати, у Господа просим. 

\myemph{Хор:} Подай, Господи.

\myemph{Диакон:} Христианския кончины живота нашего, безболезнены, непостыдны, мирны, и добраго ответа на Страшнем Судищи Христове, просим. 

\myemph{Хор:} Подай, Господи.

\myemph{Диакон:} Соединение веры и причастие Святаго Духа испросивше, сами себе, и друг друга, и весь живот наш Христу Богу предадим. 

\myemph{Хор:} Тебе, Господи. 

\myemph{Иерей:} И сподоби нас, Владыко, со дерзновением, неосужденно смети призывати Тебе, Небеснаго Бога Отца, и глаголати: 

\mysubtitle{ Отче наш} 

\myemph{Хор (или все молящиеся):} Отче наш, Иже еси на Небесех! Да святится Имя Твое, да приидет Царствие Твое, да будет воля Твоя, яко на небеси и на земли. Хлеб наш насущный даждь нам днесь, и остави нам долги наша, якоже и мы оставляем должником нашим; и не введи нас во искушение, но избави нас от лукаваго. 

\myemph{Иерей:} Яко Твое есть Царство, и сила, и слава. Отца и Сына и Святаго Духа, ныне и присно и во веки веков. 

\myemph{Хор:} Аминь. 

\myemph{Иерей:} Мир всем. 

\myemph{Хор:} И духови твоему.

\myemph{Диакон:} Главы ваша Господеви приклоните.

\myemph{Хор:} Тебе, Господи.

\myemph{Иерей:} Благодатию, и щедротами, и человеколюбием Единороднаго Сына Твоего, с Нимже благословен еси, со Пресвятым и Благим и Животворящим Твоим Духом, ныне и присно и во веки веков. 

\myemph{Хор:} Аминь.

\myemph{(Закрываются царские врата и завеса)}

\myemph{Диакон:} Вонмем. 

\myemph{Подносится теплота.}

\myemph{Иерей:} Святая святым. 

\myemph{Хор:} Един Свят, един Господь Иисус Христос, во славу Бога Отца. Аминь. 

\mysubtitle{ Причащение священнослужителей} 

\myemph{В алтаре причащаются священнослужители.} 

\myemph{Хор поет назначенный в этот день церковным Уставом причастен "--- стих, оканчивающихся троекратным «Аллилуиа». Причастнов может быть назначено два, однако «Аллилуия» поется только после второго.}

\mysubtitle{ Причастны} 

\myemph{Во время причастных выносится понамарская свеча и ставится перед Царскими вратами. Затем выносятся запивка и просфоры для причастников.}

\myemph{В воскресенье:} Хвалите Господа с небес, хвалите Его в вышних. Аллилуиа, аллилуиа, аллилуиа. 

\myemph{В понедельник:} Творяй ангелы Своя духи, и слуги Своя пламень огненный. 

\myemph{Во вторник:} В память вечную будет праведник, от слуха зла не убоится. 

\myemph{В среду:} Чашу спасения прииму и Имя Господне призову. 

\myemph{В четверг:} Во всю землю изыде вещание их, и в концы вселенныя глаголы их. 

\myemph{В пятницу:} Спасение соделал еси посреде земли, Боже. 

\myemph{В субботу:} Радуйтеся, праведнии, о Господе, правым подобает похвала. 

\myemph{Заупокойный:} Блажени, яже избрал и приял еси, Господи, и память их в род и род. 

\myemph{В праздники Богородицы:} Чашу спасения прииму и Имя Господне призову. 

\myemph{В праздники апостолов:} Во всю землю изыде вещание их, и в концы вселенныя глаголы их. 

\myemph{В дни памяти святых:} В память вечную будет праведник, от слуха зла не убоится. 

\myemph{Забирается свеча}.

\myemph{Открываются царские врата. Диакон, вынося Святую Чашу, возглашает:} Со страхом Божиим и верою приступите! 

\myemph{(Передает Чашу иерею.)} 

\myemph{Хор:} Благословен Грядый во Имя Господне, Бог Господь и явися нам.

\myemph{[В пасхальную седмицу вместо этого поется «Христос воскресе…».]}

\myemph{Иерей (и с ним все, желающие причаститься):} Верую, Господи, и исповедую, яко Ты еси воистину Христос, Сын Бога живаго, пришедый в мир грешныя спасти, от них же первый есмь аз. Еще верую, яко Cие есть самое Пречистое Тело Твое, и сия есть самая Честная Кровь Твоя. Молюся убо Тебе: помилуй мя, и прости ми прегрешения моя, вольная и невольная, яже словом, яже делом, яже ведением и неведением, и сподоби мя неосужденно причаститися Пречистых Твоих таинств, во оставление грехов, и в Жизнь Вечную. Аминь.

Вечери Твоея Тайныя днесь, Сыне Божий, причастника мя приими; не бо врагом Твом тайну повем, ми лобзания Ти дам, яко Ииуда, на яко разбойник исповедаю Тя: помяни мя, Господи, во Царствии Твом.

Да не в суд или во осуждение будет мне причащение Святых Таин, Господи, но во исцеление души и тела. Аминь.

\mysubtitle{Причащение мирян}

\myemph{Причащая мирян, иерей говорит:} Причащается раб Божий \myemph{(имя)} Честнаго и Святаго Тела и Крове Господа и Бога и Спаса нашего Иисуса Христа, во оставление грехов своих и в Жизнь Вечную. 

\myemph{Хор (во время причащения)}: Тело Христово приимите, Источника безсмертнаго вкусите.

\myemph{[В Великий четверг вместо этого поется «Вечери Твоея тайныя…» (это песнопение приведено в главе «Песнопения из служб Триоди постной»); в пасхальную седмицу "--- «Христос воскресе…».]}

\myemph{Иерей:} Спаси, Боже, люди Твоя и благослови достояние Твое.

\myemph{Подносится кадило в алтаре.} 

\myemph{Хор:} Видехом Свет истинный, прияхом Духа Небеснаго, обретохом веру истинную, Нераздельней Троице покланяемся: Та бо нас спасла есть.

\myemph{[Вместо «Видехом свет истинный…» от Пасхи до отдания поется «Христос воскресе…»; от Вознесения до отдания "--- тропарь Вознесения (эти песнопения приведены в главе «Песнопения из служб Триоди цветной»); в Троицкую родительскую субботу "--- «Глубиною мудрости…» (этот тропарь приведен в главе «Песнопения из служб Триоди цветной», в службе мясопустной родительской субботы).]}

\myemph{Иерей:} Всегда, ныне и присно и во веки веков.

\myemph{Хор:} Аминь. Да исполнятся уста наша хваления Твоего, Господи, яко да поем славу Твою, яко сподобил еси нас причаститися Святым Твоим, Божественным, Безсмертным и Животворящим Тайнам; соблюди нас во Твоей святыни, весь день поучатися правде Твоей. Аллилуиа, аллилуиа, аллилуиа. 

\myemph{[В Великий четверг вместо «Да исполнятся…» поется «Вечери Твоея тайныя…» (это песнопение приведено в главе «Песнопения из служб Триоди постной»); в пасхальную седмицу "--- «Христос воскресе…».]}

\mysubtitle{Ектения} 

\myemph{Диакон:} Пр\'{о}сти приимше Божественных, Святых, Пречистых, Безсмертных, Небесных и Животворящих, Страшных Христовых Тайн, достойно благодарим Господа. 

\myemph{Хор:} Господи, помилуй. 

\myemph{Диакон:} Заступи, спаси, помилуй и сохрани нас, Боже, Твоею благодатию. 

\myemph{Хор:} Господи, помилуй. 

\myemph{Диакон:} День весь совершен, свят, мирен и безгрешен испросивше, сами себе, и друг друга, и весь живот наш Христу Богу предадим. 

\myemph{Хор:} Тебе, Господи. 

\myemph{Иерей:} Яко Ты еси Освящение наше, и Тебе славу возсылаем, Отцу и Сыну и Святому Духу, ныне и присно и во веки веков,

\myemph{Хор:} Аминь.

\myemph{Иерей:} С миром изыдем.

\myemph{Хор:} О имени Господни. 

\myemph{Диакон:} Господу помолимся.

\myemph{Хор:} Господи, помилуй. 

\mysubtitle{Молитва заамвонная} 

\myemph{Иерей (стоя пред амвоном):} Благословляяй благословящия Тя, Господи, и освящаяй на Тя уповающия, спаси люди Твоя и благослови достояние Твое, исполнение Церкве Твоея сохрани, освяти любящия благолепие дому Твоему. Ты тех возпрослави Божественною Твоею силою и не остави нас, уповающих на Тя. Мир мирови Твоему даруй, Церквам Твоим, священником, воинству и всем людем Твоим. Яко всякое даяние благо и всяк дар совершен свыше есть, сходяй от Тебе, Отца Светов. И Тебе славу, и благодарение, и поклонение возсылаем, Отцу и Сыну и Святому Духу, ныне и присно, и во веки веков. 

\myemph{Хор:} Аминь. Буди Имя Господне благословено отныне и до века. \myemph{(Tрижды)}

\myemph{[На пасхальной седмице вместо этого поется «Христос воскресе…».]}

\mysubtitle{ Псалом 33} 

\myemph{Хор:} Благословлю Господа на всякое время, выну хвала Его во устех моих. О Господе похвалится душа моя. Да услышат кротции, и возвеселятся. Возвеличите Господа со мною, и вознесем Имя Его вкупе. Взысках Господа, и услыша мя, и от всех скорбей моих избави мя. Приступите к Нему и просветитеся, и лица ваша не постыдятся. Сей нищий воззва, и Господь услыша и, и от всех скорбей его спасе и. Ополчится ангел Господень окрест боящихся Его, и избавит их. Вкусите, и видите, яко благ Господь; блажен муж, иже уповает Нань. Бойтеся Господа вси святии Его, яко несть лишения боящимся Его. Богатии обнищаша и взалкаша: взыскающии же Господа не лишатся всякаго блага. Приидийте, чада, послушайте мене, страху Господню научу вас. Кто есть человек хотяй живот, любяй дни видети благи? Удержи язык твой от зла, и устне твои, еже не глаголати льсти. Уклонися от зла, и сотвори благо, взыщи мира, и пожени и. Очи Господни на праведныя и уши Его в молитву их. Лице же Господне на творящия злая, еже потребити от земли память их. Воззваша праведнии, и Господь услыша их, и от всех скорбей их избави их. Близ Господь сокрушенных сердцем, и смиренныя духом спасет. Многи скорби праведным, и от всех их избавит я Господь. Хранит Господь вся кости их, ни едина от них сокрушится. Смерть грешников люта, и ненавидящии праведнаго прегрешат. Избавит Господь души раб Своих, и не прегрешат вси уповающий на Него.

\myemph{[На пасхальной седмице вместо этого поется «Христос воскресе…».]}

\myemph{Иерей:} Благословение Господне на вас. Того благодатию и человеколюбием, всегда, ныне и присно и во веки веков.

\myemph{Хор:} Аминь.

\myemph{Иерей:} Слава Тебе, Христе Боже, Упование наше, слава Тебе.

\myemph{[На Пасху, в пасхальную седмицу и в отдание Пасхи вместо «Слава Тебе, Христе Боже…» священнослужители поют «Христос воскресе из мертвых, смертию смерть поправ», а хор заканчивает: «и сущим во гробех живот даровав».} 

\myemph{От Недели о Фоме до отдания Пасхи священник произносит: «Слава Тебе, Христе Боже, Упование наше, Слава Тебе», а хор поет «Христос воскресе…» (Трижды).]}

\myemph{Хор:} Слава Отцу и Сыну и Святому Духу. И ныне и присно и во веки веков. Аминь. 

\myemph{Хор:} Господи, помилуй \myemph{(Трижды).}

\myemph{Хор:} Благослови. 

\mysubtitle{ Отпуст} 

\myemph{Иерей произносит отпуст. В воскресенье:} Воскресый из мертвых, Христос, истинный Бог наш, молитвами Пречистыя Своея Матере, святых славных и всехвальных Апостол, иже во святых отца нашего Иоанна, архиепископа Константина града, Златоустаго ( \myemph{или:} св. Василия Великаго, архиепископа Кесарии Каппадокийския), и святаго \myemph{(храма и святого, которого память в этот день),} святых и праведных Богоотец Иоакима и Анны и всех святых, помилует и спасет нас, яко Благ и Человеколюбец.

\mysubtitle{ Многолетие} 

\myemph{Хор:} Великаго Господина и Отца нашего \myemph{(имярек)}, Святейшаго Патриарха Московскаго и всея Руси, и Господина нашего Преосвященнейшаго \myemph{(имя)} митрополита ( \myemph{или:} архиепископа, \myemph{или:} епископа) \myemph{(епархиальный титул его),} богохранимую державу нашу Российскую, настоятеля, братию и прихожан святаго храма сего и вся православныя христианы, Господи, сохрани их на многая лета.

\myemph{По обычаю, перед отпустом священник берет крест с престола и после отпуста, осенив крестом народ и сам поцеловав крест, дает его для целования молящимся, а чтец читает благодарственные молитвы; затем священник опять осеняет крестом народ и возвращается в алтарь, причем царские врата и завеса закрываются.}

\myemph{Алтарники убираются в алтаре, чистят кадило и готовятся к вечернему Богослужению.}

\end{mymulticols}

\mychapterending
